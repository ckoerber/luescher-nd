\documentclass[11pt]{article}
\usepackage{geometry}                % See geometry.pdf to learn the layout options. There are lots.
\geometry{a4paper}                   % ... or a4paper or a5paper or ... 
%\geometry{landscape}                % Activate for for rotated page geometry
%\usepackage[parfill]{parskip}    % Activate to begin paragraphs with an empty line rather than an indent
\usepackage{graphicx}
\usepackage{amssymb}
\usepackage{amsmath}
\usepackage{lipsum}
\usepackage{authblk}
%\usepackage{amsaddr}
\usepackage{epstopdf}
\usepackage{booktabs}
\usepackage{xcolor}
\usepackage{fancyhdr}
\usepackage[yyyymmdd,hhmmss]{datetime}
\pagestyle{fancy}
\rfoot{Compiled on \today\ at \currenttime}
\cfoot{}
\lfoot{Page \thepage}
\RequirePackage[colorinlistoftodos,prependcaption,textsize=tiny]{todonotes} % look for '\todo'
\definecolor{darkred}{rgb}{0.4,0.0,0.0}
\definecolor{darkgreen}{rgb}{0.0,0.4,0.0}
\definecolor{darkblue}{rgb}{0.0,0.0,0.4}
\usepackage[bookmarks,linktocpage,colorlinks,
    linkcolor = darkred,
    urlcolor  = darkblue,
    citecolor = darkgreen]{hyperref}

%\DeclareGraphicsRule{.tif}{png}{.png}{`convert #1 `dirname #1`/`basename #1 .tif`.png}

\renewcommand{\headrulewidth}{0pt}
\fancyhead[L]{
%\includegraphics[width=4cm]{/Users/tomluu/Research/talks/fzjTemplate/uniBonn_logo.jpg}
}
\fancyhead[R]{
\includegraphics[width=4cm]{figs/fzj_logo.jpg}
}
\pagestyle{plain}

\title{Tower of operators}
\author[1,2]{Thomas Luu}
\affil[1]{Institute for Advanced Simulation 4\\
Forschungszentrum J\"ulich, Germany}
\affil[2]{Rheinische Friedrich-Williams-Universit\"at Bonn, Germany}

%\email{t.luu@fz-juelich.de}
\date{}                                           % Activate to display a given date or no date


\begin{document}
\maketitle
\begin{center}
email: \href{mailto:t.luu@fz-juelich.de}{t.luu@fz-juelich.de}
\end{center}
\abstract{
Here I try to determine the scattering amplitude for an interaction of the type
\begin{displaymath}
V(p',p)=C_0+\frac{C_2}{2}\left(p'^2+p^2\right)
\end{displaymath}
and I assume a hard-cutoff regulator.
}

\thispagestyle{fancy}

\clearpage{}
%\tableofcontents
%\newpage

\section{Interaction}
I assume an interaction of the form
\begin{equation}\label{eqn:potential}
V(p',p)=\left(C_0+C_2\frac{\left(p'^2+p^2\right)}{2}\right)f_\Lambda(p')f_\Lambda(p)
\end{equation}
with an implicit hard-cutoff regulator at some scale $\Lambda$,
\begin{equation}
f_\Lambda(p)=\Theta\left(|p|-\Lambda\right)\ .
\end{equation}
 Note that this interaction is a sum of separable terms, and there is NO delta function (we are in momentum space).  The coefficients $C_n$ are bare parameters and must be ``renormalized"
 
 \section{T-matrix}
 The T-matrix is defined as
 \begin{equation}
 T(p',p;E)\equiv V(p',p)+\int \frac{d^3q}{(2\pi)^3} V(p',q) G_0(q;E) T(q,p;E)\ ,
 \end{equation}
 where
 \begin{equation}
 G_0(q;E)=\frac{1}{E-q^2/m+i\epsilon}\ .
 \end{equation}
 Another relation, which is actually more useful when deriving the T-matrix, is the expression for the fully interacting propagator, which I write in operator form
 \begin{equation}\label{eqn:full G}
 \hat G(E)\equiv \frac{1}{E-\hat H_0-\hat V+i\epsilon}=\hat G_0(E)+\hat G_0(E)\hat T(E)\hat G_0(E)\ .
 \end{equation}
 
 \subsection{Summing the T-matrix}
I can sum the T-matrix to all orders in $V(p',p)$ since $V(p',p)$ is a sum of separable terms.   I will present my derivations (based off eq.~\eqref{eqn:full G}) for the full T-matrix in later sections, but only express the result here
\begin{multline}\label{eqn:T-matrix 1}
T(p',p;E)=\\
\frac{\mathbb{C}_0+\mathbb{C}_2\frac{\left(p'^2+p^2\right)}{2}+\mathbb{X}_4\  p'^2p^2}
{1-C_0I_0(E)-C_2I_2(E)-\frac{1}{4}C^2_2\left[I_0(E)I_4(E)-I_2(E)^2\right]}f_\Lambda(p')f_\Lambda(p)\ ,
\end{multline}
where
\begin{align}
\mathbb{C}_0&=C_0+\frac{1}{4}C^2_2I_4(E)\\
\mathbb{C}_2&=C_2-\frac{1}{2}C^2_2I_2(E)\\
\mathbb{X}_4&=\frac{1}{4}C^2_2I_0(E)\ ,
\end{align}
and\footnote{I assume we are in the infinite-volume, continuum limit, and therefore perform the integrals in a spherical basis.}
\begin{equation}
I_n(E)\equiv \frac{1}{2\pi^2}\int_0^\Lambda dq\frac{q^{2+n}}{E-q^2/m+i\epsilon}\ .
\end{equation}
Note that the solution to the T-matrix has an \emph{induced} higher-order term $\mathbb{X}_4p'^2p^2$.  

We can also relate all terms $I_{n>0}(E)$ to $I_0(E)$,
\begin{align}
I_2(E)&=-\frac{m}{6\pi^2}\Lambda^3+[mE]\  I_0(E)\\
I_4(E)&=-\frac{m}{10\pi^2}\Lambda^5-[mE]\ \frac{m}{6\pi^2}\Lambda^3+[mE]^2\ I_0(E)\ .
\end{align}
And finally, we have that
\begin{equation}
I_0(E)=-\frac{m}{2\pi^2}\Lambda -i \frac{m}{4\pi}\sqrt{mE}+\frac{m}{2\pi^2\Lambda}mE+\mathcal{O}(\Lambda^{-3})\ .
\end{equation}
With these relations, the \emph{denominator} in eq.~\eqref{eqn:T-matrix 1} becomes
\begin{multline}\label{eqn:denominator}
1-C_0 I_0(E)-C_2 \left([mE]\ I_0(E) -\frac{\Lambda ^3 m}{6 \pi ^2}\right)\\
+C_2^2 \left(I_0(E) \left(\frac{\Lambda ^5 m}{40 \pi
   ^2}-\frac{\Lambda ^3 m\ [mE]}{24 \pi ^2}\right)+\frac{\Lambda ^6 m^2}{144 \pi
   ^4}\right)
   \end{multline}

\section{Scattering amplitude near threshold}
Near threshold I set everything on-shell, $p'=p=\sqrt{mE}$.  In this limit, the \emph{numerator} of eq.~\eqref{eqn:T-matrix 1} becomes
\begin{equation}\label{eqn:numerator}
C_0+C_2 p^2+C_2^2 \left(\frac{\Lambda ^3 m p^2}{24 \pi ^2}-\frac{\Lambda ^5 m}{40 \pi
   ^2}\right)\ .
   \end{equation}
% and one has that
%\begin{equation}
%T(p'=p; E=p^2/m)\equiv T(p)=\frac{4\pi/m}{p\cot \delta -ip}
%\end{equation}
%or equivalently
%\begin{equation}
%p\cot \delta -ip=\frac{4\pi}{m\ T(p)}\ .
%\end{equation}
%Plugging everything in, I get
%\begin{multline}\label{eqn:matching 1}
%p\cot \delta -ip=\\
%\frac{4\pi}{m}\frac{1-(C_0 +C_2 p^2)I_0(E)+C_2 \frac{\Lambda ^3 m}{6 \pi ^2}
%+C_2^2 \left(I_0(E) \left(\frac{\Lambda ^5 m}{40 \pi
%   ^2}-\frac{\Lambda ^3 mp^2}{24 \pi ^2}\right)+\frac{\Lambda ^6 m^2}{144 \pi
%   ^4}\right)}{C_0+C_2 p^2+C_2^2 \left(\frac{\Lambda ^3 m p^2}{24 \pi ^2}-\frac{\Lambda ^5 m}{40 \pi
%   ^2}\right)}
%   \end{multline}
%   
Now I define the renormalized coefficients,
\begin{align}
C_0(\Lambda)&=C_0-C_2^2\frac{\Lambda^5 m}{40 \pi^2}\\
C_2(\Lambda)&=C_2\left(1+C_2\frac{\Lambda^3 m}{24\pi^2}\right)\ ,
\end{align}
which means eq.~\eqref{eqn:numerator} becomes
\begin{equation}
\sum_{n=0,1}C_{2n}(\Lambda)p^{2n}\ .
\end{equation}
The denominator, eq.~\eqref{eqn:denominator}, becomes
\begin{multline}
1-I_0(E)\sum_{n=0,1}C_{2n}(\Lambda)p^{2n}+\frac{\Lambda ^3 m}{6 \pi ^2}C_2 \left(1+
C_2 \frac{\Lambda ^3 m}{24 \pi
   ^2}\right)\\
   =1-I_0(E)\sum_{n=0,1}C_{2n}(\Lambda)p^{2n}+C_2(\Lambda) \frac{\Lambda ^3 m}{6 \pi ^2}\ .
\end{multline}
So the full on-shell T-matrix near threshold is
\begin{equation}
T(E=p^2/m)=-\frac{4\pi/m}{p\cot \delta -ip}=\frac{\sum_{n=0,1}C_{2n}(\Lambda)p^{2n}}{1-I_0(E)\sum_{n=0,1}C_{2n}(\Lambda)p^{2n}+C_2(\Lambda) \frac{\Lambda ^3 m}{6 \pi ^2}}\ .
\end{equation}
This expression is \emph{almost} like eq.(1) of Beane et al. (hep-lat0312004) ``Two-Nucleons on a Lattice",
\begin{displaymath}
\mathcal{A}=\frac{\sum C_{2 n}(\mu) p^{2 n}}{1-I_{0} \sum C_{2 n}(\mu) p^{2 n}}\ ,
\end{displaymath}
except for the extra \emph{induced} term $C_2(\Lambda) \frac{\Lambda ^3 m}{6 \pi ^2}$ in the denominator.  

\section{So do we have a ``Tower of Interactions"?}
I \emph{suspect} what happens (this is by no means a proof) is that as we include more terms in our potential, eq.~\eqref{eqn:potential}, the induced term gets pushed to higher and higher order.  One must be consistent in adding more terms into the potential, however (i.e. derivative expansion w/ Galilean invariance, etc. . .--essentially pionless EFT).  

For example, if we include a $C_4$ term into our potential, then I suspect we would get for the scattering amplitude
 \begin{equation}
 \frac{\sum_{n=0,1,2}C_{2n}(\Lambda)p^{2n}}{1-I_0(E)\sum_{n=0,1,2}C_{2n}(\Lambda)p^{2n}+C_4(\Lambda) \mathcal{C}\Lambda^5}\ ,
 \end{equation}
 where $\mathcal{C}$ is some coefficient.  If this is true (and I'm not going to even attempt to prove this), then only in the limit of an ``infinite tower of interactions" do we recover Beane et al.'s expression,
\begin{equation}
\frac{\sum C_{2 n}(\Lambda) p^{2 n}}{1-I_{0}(E) \sum C_{2 n}(\Lambda) p^{2 n}}\ ,
\end{equation}

\section{Summing the T-matrix}
Later. . .

%Then eq.~\eqref{eqn:matching 1} becomes
%\begin{align}\label{eqn:matching 2}
%p\cot \delta -ip&=
%\frac{4\pi}{m}\frac{1-I_0(E)\sum_{n=0,1}C_{2n}(\Lambda)p^{2n}+C_2 \frac{\Lambda ^3 m}{6 \pi ^2}\left(1+
%C_2 \frac{\Lambda ^3 m}{24 \pi
%   ^2}\right)}{\sum_{n=0,1}C_{2n}(\Lambda)p^{2n}}\\\
%   &=\frac{4\pi}{m}\left(\frac{1}{\sum_{n=0,1}C_{2n}(\Lambda)p^{2n}}-I_0(E)+\frac{C_2 \frac{\Lambda ^3 m}{6 \pi ^2}\left(1+
%C_2 \frac{\Lambda ^3 m}{24 \pi
%   ^2}\right)}{\sum_{n=0,1}C_{2n}(\Lambda)p^{2n}}\right)
%\end{align}
%\subsection{}

%\newpage
%\appendix

%\clearpage
%\bibliography{references}

\end{document}  
