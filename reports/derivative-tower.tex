% !TEX encoding = UTF-8 Unicode
% !TEX spellcheck = en-US
\documentclass[
    aps,
    prl,
    showkeys,
    nofootinbib,
    %twocolumn,
    floatfix
]{revtex4}

\usepackage[utf8]{inputenc} % UTF-8
\usepackage{braket}

\usepackage{amsmath}
\usepackage{amssymb}
\usepackage{bbm}

\usepackage{graphicx}
\usepackage{color}

\renewcommand{\vec}[1]{\boldsymbol{#1}}


\newcommand{\fm}{\,\mathrm{fm}}
\newcommand{\ifm}{\,\mathrm{fm}^{-1}}
\newcommand{\MeV}{\,\mathrm{MeV}}

\begin{document}

\title{Idea behind Lüscher's formalism}

\author{C.~Körber}

\date{\today}

\begin{abstract}%
	I try to derive the on shell $T$-matrix expression KSW used in their paper
	$$
		T =-\frac{\sum C_{2 n} p^{2 n}}{1+i(M p / 4 \pi) \sum C_{2 n} p^{2 n}} \, .
	$$
\end{abstract}

\maketitle

\section{Physical scattering information}

Scattering of two particles in quantum mechanics are described by the $T$-matrix 
\begin{align}
	T(\vec p', \vec p, E)
	&=
	V(\vec p', \vec p) + \lim\limits_{\epsilon \to 0}\int \frac{d \vec k^3}{(2\pi)^3} V(\vec p', \vec k) G(\vec k, E + i \epsilon) T(\vec k, \vec p, E) \, ,
	&
	G(\vec k, E+ i \epsilon) = \frac{1}{E + i \epsilon - \frac{k^2}{2\mu}}
	\, ,
\end{align}
Where $\mu$ is the reduced mass of the two particle system (e.g., $\mu = 2 m$ for particles of equal mass $m$).

The \textbf{on-shell} $T$-matrix is related to the scattering phase shifts by
\begin{equation}
	T_l(p, p , E_p) = \frac{2}{\mu} \mathcal F \frac{1}{\cot (\delta_l(p)) - i} \, ,
\end{equation}
where $l$ denotes the partial wave, $E_p = p^2 / (2 \mu)$ and $\mathcal F$ is a factor which depends on the spatial dimension of the two particle system.
Note that the on-shell $T$-matrix is evaluated for all momenta and energies equal to $p$ and the partial wave decomposition removed the remaining vectorial dependence
\begin{equation}
	\left\langle \vec{p}^{\prime}| \hat T(E) | \vec{p}\right\rangle=\frac{2}{\pi} \sum_{l, m} T_{l}\left(p^{\prime}, p, E\right) Y_{l m}^{*}\left(\Omega_{k^{\prime}}\right) Y_{l m}\left(\Omega_{k}\right)
\end{equation}
For scalar potentials in three spatial dimensions, this becomes
\begin{equation}\label{eq-phase-shifts}
	T(p) \equiv T(p, p, E_p) = \frac{2}{\mu} \frac{1}{p\cot (\delta(p)) - ip} \, .
\end{equation}

\section{Special cases}

For certain potentials, the $T$-matrix has a simple analytic expression.

\subsection{Contact interaction}
\begin{equation}
	V(\vec p', \vec p) = c(\Lambda) \, .
\end{equation}

Because the potential is independent of $p$ and $p'$, the T-matrix becomes

\begin{align}
	T(p', p, E) 
	= c(\Lambda) + c(\Lambda) \lim\limits_{\epsilon \to 0}\int\limits_{|\vec k| < \Lambda} \frac{d \vec k^3}{(2\pi)^3} G( k, E + i \epsilon) T( k,  p, E) 
	= c(\Lambda) + c(\Lambda) \Gamma(E , p)
	\, .
\end{align}
with
\begin{equation}
	\Gamma(E, p)
	=
	\lim\limits_{\epsilon \to 0}\int\limits_{|\vec k| < \Lambda} \frac{d \vec k^3}{(2\pi)^3} G( k, E + i \epsilon) T( k,  p, E)
	=
	\left[c(\Lambda) + c(\Lambda) \Gamma(E, \epsilon , p)\right]\lim\limits_{\epsilon \to 0}
	\int\limits_{|\vec k| < \Lambda} \frac{d \vec k^3}{(2\pi)^3} G( k, E + i \epsilon)
\end{equation}



In this case the $T$-matrix becomes
\begin{align}\label{eq:quantization-contact-physical}
	T(p', p, E) &= \frac{c(\Lambda)}{1 - c(\Lambda) I_0(E, \Lambda)} \, , &
	I_0(E) = \lim\limits_{\epsilon \to 0} \int\limits_{|\vec k| < \Lambda} \frac{d \vec k^3}{(2\pi)^3} G(\vec k, E + i \epsilon)
	\, .
\end{align}

So the expected form is as in the KSW case.

\subsection{Contact Tower}
I know specifically look at the on-shell $T$-matrix because I do not know how $V$ is defined in detail
\begin{equation}
	V(p, p) = \sum_n c_n (\Lambda) p^{2n} \, .
\end{equation}

In this case, the loop integral becomes

\begin{align}
	\lim\limits_{\epsilon \to 0}\int\limits_{|\vec k| < \Lambda}  \frac{d \vec k^3}{(2\pi)^3} V(\vec k, \vec k) G(\vec k, E + i \epsilon)
	&= \sum_n c_n (\Lambda) I_n(E) \, 
	&
	I_n(E) &= \lim\limits_{\epsilon \to 0}\int\limits_{|\vec k| < \Lambda}  \frac{d \vec k^3}{(2\pi)^3} k^{2 n} G(\vec k, E + i \epsilon)
\end{align}
I am not sure why I look at $V(k, k)$ here...

Mathematica tells me that this is for $\gamma = 2 \mu |E|$
\begin{align}
	I_n(E)
	&=
	-\frac{i\mu}{2\pi}\gamma^{2n + 1} 
		- \frac{2}{2n+1} \frac{\mu}{2 \pi^2}\Lambda^{2n+1}\left[1 - \frac{2n+1}{2n-1} \frac{\gamma^2}{\Lambda^2} + \mathcal O \left( \frac{\gamma^4}{\Lambda^4}\right)\right]
\end{align}

which agrees quite nicely with the KSW result (which used a MS scheme) [$M = 2 \mu$]
\begin{equation}
	I_{n}^{M S}=(M E)^{n}\left(\frac{M}{4 \pi}\right) \sqrt{-M E-i \epsilon}=-i\left(\frac{M}{4 \pi}\right) p^{2 n+1}
\end{equation}

Thus we have 
\begin{equation}
	\sum_n c_n (\Lambda) I_n(E)
	=
	-I_0(E) V(\gamma, \gamma)
	- \sum_n c_n(\Lambda) F_n(\Lambda) \left[ 1 + \mathcal O \left( \frac{\gamma^2}{\Lambda^2}\right) \right] \, .
\end{equation}


\end{document}