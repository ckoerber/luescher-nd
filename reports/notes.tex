% !TEX encoding = UTF-8 Unicode
% !TEX spellcheck = en-US
\documentclass[aps, prl, showkeys, nofootinbib, twocolumn, floatfix]{revtex4-1}

\usepackage[utf8]{inputenc} % UTF-8
\usepackage{braket}

\usepackage{amsmath}
\usepackage{amssymb}



\begin{document}

\title{Discretization}

\author{C.~Körber}

\date{\today}

\begin{abstract}%
	How does the contact interaction depend on discretization effects?
\end{abstract}

\maketitle

\section{No discretization}
\subsection{1-d}
\subsubsection{Positive energies}

The momentum space Schrödinger equation
\begin{equation}
	\frac{p^2}{2 \mu} \psi(p) + \int \frac{d p}{2\pi} \braket{p | \hat V | p'} \psi(p') = E \psi(p')
\end{equation}
for contact interactions this becomes
\begin{equation}
	\frac{p^2}{2 \mu} \psi(p) +  c I_0 = E \psi(p') \, , \qquad I_0 = \int \frac{d p'}{2\pi} \psi(p') \, .
\end{equation}
Thus
\begin{equation}
	\psi(p) = \frac{c I_0}{E - \frac{p^2}{2\mu}}
\end{equation}
For scattering solutions we need $i\epsilon$ prescription
\begin{align}
	I_0 &= \int \frac{d p}{2\pi}  \frac{c I_0}{E - \frac{p^2}{2\mu} + i \epsilon} \\
	&= I_0 \times \frac{\mu c }{\pi} \int\limits_{ - \infty}^{\infty} \frac{d p}{\gamma^2 - {p^2} + i \tilde\epsilon} \, , \quad \gamma^2 = 2 \mu E > 0\,.
\end{align}
The remaining integral can be solved by the residual theorem: the zeros of the denominator are
\begin{align}
	\gamma^2 - {p^2} + i \tilde\epsilon &= (\gamma + i \tilde\epsilon - p)(\gamma + i \tilde\epsilon + p) \, .
\end{align}
Note that the zeros are independent on the sign of $\epsilon$ (one root in upper and one root in lower plane).
Completing the integral in the upper plane selects the residual $p_0 = \gamma + i \tilde \epsilon$ and thus
\begin{equation}
	\int\limits_{ - \infty}^{\infty} \frac{d p}{\gamma^2 - {p^2} + i \tilde\epsilon}
	=
	 \frac{2 \pi i}{2 \gamma} \, .
\end{equation}
and therefore
\begin{equation}
	1 = \frac{\mu c}{\pi} \times \frac{\pi i}{\gamma} \, ,
\end{equation}
or
\begin{equation}
	\gamma = i \mu c \, \Rightarrow  E = - \frac{\mu c^2}{2}
\end{equation}
In other words, you do not get positive energies for $c \in \mathbb{R}$.

It seems like, you only get positive energies once you are in a box...

\clearpage
\section{Understanding the unitary limit}


From my understanding, the unitary limit appears when the total cross section is independent of microscopic parameters.
The cross section is given by
\begin{equation}
	\frac{d \sigma}{d \Omega} = | f(\Omega, p) | ^ 2 \, ,
\end{equation}
where
\begin{align}
	f(\Omega, p) &= \sum_{l=0}^{\infty} f_l(\Omega, p) \, , \\ 
	f_l(\Omega, p) &\sim g_l(\Omega) \times \left[\exp( 2 i \delta_l (p)) - 1\right]
\end{align}
This is generally true for all $d = 1, 2, 3$ cases modulo not important factors.

In our system, for a simple contact interaction of strength $c_0$, we only find contributions to $l=0$.
Regarding your eq. (7), one finds that
\begin{equation}
	\cot \delta_0 (p) = \begin{cases}
		a_0 p &, \, d = 1 \\
		2 / \pi \log(a_0 p) &, \, d = 2 \\
		- 1/(a_0 p) &, \, d = 3
	\end{cases}
\end{equation}

In other words, we are at the unitary limit if $\cot \delta_0$ is independent of $a_0$.
This can only happen if $a_0 \rightarrow \infty$ or $a_0 \rightarrow \pm 0$ because $a_0$ must be independent of $p$.

\textbf{Does this make sense?}

To discriminate between infinite volume and lattice, I will label the momentum points which are extracted from lattice (as in continuum but finite volume) as $\gamma_i$.
Here, $\gamma_i^2 = 2 \mu E_i$, $\mu$ being the reduced two-particle mass and $E_i$ being the lattice energy spectrum.

Using your equations (35-37) and combining this with your eq. (7), one finds
\begin{equation}\label{def:a0}
	a_0 = \begin{cases}
		\frac{L}{2\pi^2}S_1(x_i) & \, , d =1 \\
		\frac{L}{2\pi} \exp \left\{ S_2(x_i) / (2\pi) \right\} & \, , d =2 \\
		\frac{ \pi L}{ S_3(x_i) } & \, , d =3
	\end{cases}
	\qquad \forall_i \, ,
\end{equation}
with $S_j(x_i)$ being the Lüscher Zeta function in $j$ dimensions and $ x_i = \mu E_i L ^2 / (2 \pi^2) $.
In other words, if your computation does the right thing, you fix your contact interaction strength $c_0$, which will give you a set of energy levels $E_i(c_0)$.
For all of the eigenstates $E_i$, all points $S_j(x_i)$ lay on a line because, because $a_0$ is independent of the energy level and eq.~\eqref{def:a0} tells us that there is a one-to-one relation between $S_j(x_i)$ and $a_0$.

So in which cases do we find that $a_0 \to 0, \pm \infty$?

\begin{itemize}
	\item For $d = 1$ find $c_0$ such that $S_1(x_i(c_0)) \to 0, \pm \infty$ $\forall_i$.
		Regarding your plots,
		\begin{itemize}
			\item $S_1(x_i(c_0)) = 0$ is not possible $\forall_i$ (ground state only approaches zero).
			\item $S_1(x_i(c_0)) \to - \infty$ is not possible $\forall_i$ (ground state $>0$).
			\item $S_1(x_i(c_0)) \to + \infty$ is possible $\forall_i$ if $c_0 \to 0^-$.
		\end{itemize}
	\item For $d = 2$ find $c_0$ such that $S_2(x_i(c_0)) \to + \infty$ $\forall_i$.
		Regarding your plots,
		\begin{itemize}
			\item $S_2(x_i(c_0)) \to + \infty$ is possible for $c_0 \to 0^-$.
		\end{itemize}
	\item For $d = 3$ find $c_0$ such that $S_3(x_i(c_0)) \to 0, \pm \infty$ $\forall_i$.
		Regarding your plots,
		\begin{itemize}
			\item $S_3(x_i(c_0)) = 0$ is possible $\forall_i$.
			\item $S_3(x_i(c_0)) \to \pm \infty$ is possible for $c_0 \to 0^\mp$.
		\end{itemize}
\end{itemize}

Obviously, the cases for $c_0 \to 0^\pm$ make sense --- if there is no interaction, than there is only the mass scale.
However this solution is not of interest.
This leaves $d=3$ as the only interesting case.


\end{document}