% !TEX encoding = UTF-8 Unicode
% !TEX spellcheck = en-US
\documentclass[aps, prl, showkeys, nofootinbib, twocolumn, floatfix]{revtex4-1}

\usepackage[utf8]{inputenc} % UTF-8
\usepackage{braket}

\usepackage{amsmath}
\usepackage{amssymb}



\begin{document}

\title{Discretization}

\author{C.~Körber}

\date{\today}

\begin{abstract}%
	How does the contact interaction depend on discretization effects?
\end{abstract}

\maketitle

\section{No discretization}
\subsection{1-d}
\subsubsection{Positive energies}

The momentum space Schrödinger equation
\begin{equation}
	\frac{p^2}{2 \mu} \psi(p) + \int \frac{d p}{2\pi} \braket{p | \hat V | p'} \psi(p') = E \psi(p')
\end{equation}
for contact interactions this becomes
\begin{equation}
	\frac{p^2}{2 \mu} \psi(p) +  c I_0 = E \psi(p') \, , \qquad I_0 = \int \frac{d p'}{2\pi} \psi(p') \, .
\end{equation}
Thus
\begin{equation}
	\psi(p) = \frac{c I_0}{E - \frac{p^2}{2\mu}}
\end{equation}
For scattering solutions we need $i\epsilon$ prescription
\begin{align}
	I_0 &= \int \frac{d p}{2\pi}  \frac{c I_0}{E - \frac{p^2}{2\mu} + i \epsilon} \\
	&= I_0 \times \frac{\mu c }{\pi} \int\limits_{ - \infty}^{\infty} \frac{d p}{\gamma^2 - {p^2} + i \tilde\epsilon} \, , \quad \gamma^2 = 2 \mu E > 0\,.
\end{align}
The remaining integral can be solved by the residual theorem: the zeros of the denominator are
\begin{align}
	\gamma^2 - {p^2} + i \tilde\epsilon &= (\gamma + i \tilde\epsilon - p)(\gamma + i \tilde\epsilon + p) \, .
\end{align}
Note that the zeros are independent on the sign of $\epsilon$ (one root in upper and one root in lower plane).
Completing the integral in the upper plane selects the residual $p_0 = \gamma + i \tilde \epsilon$ and thus
\begin{equation}
	\int\limits_{ - \infty}^{\infty} \frac{d p}{\gamma^2 - {p^2} + i \tilde\epsilon}
	=
	 \frac{2 \pi i}{2 \gamma} \, .
\end{equation}
and therefore
\begin{equation}
	1 = \frac{\mu c}{\pi} \times \frac{\pi i}{\gamma} \, ,
\end{equation}
or
\begin{equation}
	\gamma = i \mu c \, \Rightarrow  E = - \frac{\mu c^2}{2}
\end{equation}
In other words, you do not get positive energies for $c \in \mathbb{R}$.

It seems like, you only get positive energies once you are in a box...

\end{document}