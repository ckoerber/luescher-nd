\documentclass[aps,superscriptaddress,tightenlines,nofootinbib,floatfix,longbibliography,notitlepage]{revtex4-1}
\usepackage[left=18mm,right=19mm,top=23mm,bottom=16mm]{geometry}
\usepackage{amsmath,amssymb}
\usepackage{bm,bbm}
\usepackage{comment}
\usepackage{graphicx}
\usepackage{color}
\usepackage[dvipsnames]{xcolor}
\usepackage{slashed}
\usepackage[multidot]{grffile} %because multiple dots in filenames confuse latex
\usepackage[
    colorlinks=true,
    allcolors=blue
]{hyperref}
\usepackage{pgfplots}
%%%%
%%%%    Document preparation
%%%%

\begin{document}

\section{Why "truncated" dispersion Lüscher works}

Let's start with infinite volume continuum with a cutoff.
We know that for a contact interaction, the on-shell $T$-matrix is given by
\begin{equation}
	V(p', p) = c(\Lambda) \qquad \Rightarrow \qquad
	T(p, \Lambda) = \frac{c(\Lambda)}{1 - c(\Lambda) I(p, \Lambda)} \, .
\end{equation}
In three-d, we also find that
\begin{equation}
	I(p, \Lambda)
	=
	\frac{4 \pi}{(2\pi)^3} \lim_{\epsilon \to 0}\int_{0}^{\Lambda} d q q^2 \frac{2 \mu}{p^2 - q^2 + i \epsilon}
	=
	-\frac{\mu}{2\pi^2}   \left(2 \Lambda +p \log \left(1-\frac{2 p}{\Lambda +p}\right)\right) + \frac{i \mu p}{2 \pi}
	\, .
\end{equation}
and thus we know that
\begin{equation}
	p \cot \delta(p) 
	= \frac{2 \pi}{\mu}\frac{1}{T(p, \Lambda)} + i p
	= \frac{2 \pi}{\mu} \frac{1}{c(\Lambda)} + \frac{2 \Lambda}{\pi} + \frac{2 p}{\pi} \log \left( \frac{\Lambda - p}{\Lambda + p}\right)
\end{equation}
So if you choose
\begin{equation}
	\frac{2 \pi}{\mu} \frac{1}{c(\Lambda)} + \frac{2 \Lambda}{\pi} = - \frac{1}{a_0} \, ,
\end{equation}
the limit of $\Lambda \to \infty$ is well defined and reproduces the scattering length $a_0$.

In other words, if you fix your contact interaction properly, you get
\begin{equation}
	- \frac{1}{a_0}
	=
	\lim\limits_{\Lambda \to \infty} \mathcal R \left( \frac{2 p}{\mu} \frac{1}{T(p, \Lambda)} \right)
	=
	\lim\limits_{\Lambda \to \infty} \mathcal R \left(\frac{2 p}{\mu}   \frac{1 - c(\Lambda) I(p, \Lambda)}{c(\Lambda)} \right)
	\, .
\end{equation}
Note that this expression is constant in $p$.

Now we play a few funny tricks.
First, let me identify $\Lambda$ with $\frac{\pi}{\epsilon}$.
This gives
\begin{equation}
	- \frac{1}{a_0}
	=
	\frac{2 \pi}{\mu}
	\lim\limits_{\epsilon \to 0} \mathcal R \left(\frac{1 - c\left(\frac{\pi}{\epsilon}\right) I\left(p, \frac{\pi}{\epsilon}\right)}{c\left(\frac{\pi}{\epsilon}\right)} \right)
	\, .
\end{equation}
Next, I introduce this parameter $n_s$ with deforms the $q^2$ inside $I(p, \epsilon/\pi)$ in such a way\footnote{Also I change the integration boundaries to be cubic but that's another story we found to not make a difference in the $\Lambda \to \infty$ limit.} that
\begin{equation}
	q^2_{n_s} = \frac{1}{\epsilon^2}\sum_{i=1}^3\sum_{n=0}^{n_s}  \gamma_n^{(n_s)} \cos(\epsilon q_i n \epsilon) \, ,
\end{equation}
with coefficients $\gamma_n^{(n_s)}$ such that $\lim\limits_{n_s \to \infty} q^2_{n_s} = q^2$ and $q^2_{n_s} = q^2(1 + \mathcal O(q\epsilon)^{2 n_s})$.
Note that because $p_\text{max} = \pi / \epsilon$, we actually have to do the $n_s \to \infty$ limit first.
Though probably one can find an argument why they commute.
Also, because I am particularly funny today, I also say that $c$ depends on this $n_s$ parameter as well and is smooth in the limit of $n_s \to \infty$.

So we have
\begin{equation}
	- \frac{1}{a_0}
	=
	\frac{2 \pi}{\mu}
	\lim\limits_{\epsilon \to 0} \lim\limits_{n_s \to \infty}\mathcal R \left(\frac{1 - c^{(n_s)}\left(\frac{\pi}{\epsilon}\right) I^{(n_s)}\left(p, \frac{\pi}{\epsilon}\right)}{c^{(n_s)}\left(\frac{\pi}{\epsilon}\right)} \right)
	\, .
\end{equation}
Furthermore, one can show that solving the Schrödinger equation on the lattice in finite volume is equal to
\begin{equation}
	0 = 1 - c^{(n_s)}\left(\frac{\pi}{\epsilon}\right) I^{(FV, n_s)}\left(\sqrt{2\mu E_i^{(FV, n_s)}}, \frac{\pi}{\epsilon}\right)
	\, ,
\end{equation}
where $E_i$ is an eigen energy solution to the Schrödinger equation which was solved for a given $c^{(n_s)}\left(\frac{\pi}{\epsilon}\right)$.
Note the the energies know about finite volume and is independent of the previous $p$!
The finite volume "integral" $I^{(FV, n_s)}$, now sums over discrete $q_i$ which range from $[0, N)$ where $N = L / \epsilon$ with a momentum lattice spacing of $2\pi/L$.

If one adds this delicate zero, we find that
\begin{align}
	- \frac{1}{a_0}
	&=
	\frac{2 \pi}{\mu}
	\lim\limits_{\epsilon \to 0} \lim\limits_{n_s \to \infty}\mathcal R \left(
		\frac{1 - c^{(n_s)}\left(\frac{\pi}{\epsilon}\right) I^{(n_s)}\left(p, \frac{\pi}{\epsilon}\right)
		-
			\left[1 - c^{(n_s)}\left(\frac{\pi}{\epsilon}\right) I^{(FV, n_s)}\left(\sqrt{2\mu E_i^{(FV, n_s)}}, \frac{\pi}{\epsilon}\right)\right]
		}
		{c^{(n_s)}\left(\frac{\pi}{\epsilon}\right)} 
	\right)
	\\
	&= 
	\frac{2 \pi}{\mu}
	\lim\limits_{\epsilon \to 0} \lim\limits_{n_s \to \infty}\mathcal R \left(
		I^{(FV, n_s)}\left(\sqrt{2\mu E_i^{(FV, n_s)}}, \frac{\pi}{\epsilon}\right)
		-
		I^{(n_s)}\left(p, \frac{\pi}{\epsilon}\right)
	\right)
\end{align}
Here comes the thing Evan and I have been discussing today.

If everything is convergent for particular values of $p$, then we also find that
\begin{align}
	- \frac{1}{a_0}
	&= 
	\frac{2 \pi}{\mu}
	\lim\limits_{\epsilon \to 0} \lim\limits_{n_s \to \infty}\mathcal R \left(
		I^{(FV, n_s)}\left(\sqrt{2\mu E_i^{(FV, n_s)}}, \frac{\pi}{\epsilon}\right)
		-
		I^{(n_s)}\left(p=0, \frac{\pi}{\epsilon}\right)
	\right)
	\, ,
\end{align}
Which I will call (truncated) dispersion Lüscher zeta function from now on (modulo factors)
\begin{equation}
	\tilde S^{(FV, n_s)}(E_i^{(FV, n_s)}, \epsilon)
	\equiv
	\mathcal R \left(
		I^{(FV, n_s)}\left(\sqrt{2\mu E_i^{(FV, n_s)}}, \frac{\pi}{\epsilon}\right)
		-
		I^{(n_s)}\left(p=0, \frac{\pi}{\epsilon}\right)
	\right)
	\, .
\end{equation}

Because $I^{(n_s)}\left(p=0, \frac{\pi}{\epsilon}\right)$ is independent off $E_i^{(FV, n_s)}$ and since $I^{(FV, n_s)}\left(\sqrt{2\mu E_i^{(FV, n_s)}}, \frac{\pi}{\epsilon}\right)$ is a constant ($= - 1 / c(\pi / \epsilon)$) if evaluated at eigen energies of the Hamiltonian with contact interaction $c(\pi / \epsilon)$, we know that 
\begin{equation}
	\tilde S^{(FV, n_s)}(E_i^{(FV, n_s)}, \epsilon)
	=
	- \frac{1}{c^{(n_s)}(\frac{\pi}{\epsilon})} - C^{(n_s)}(\epsilon)
	\,, \qquad
	C^{(n_s)}(\epsilon)
	= 
	\mathcal R \left[(I^{(n_s)}\left(p=0, \frac{\pi}{\epsilon}\right)\right]
	\, .
\end{equation}
So in other words, if I choose
\begin{equation}
	\label{eq:Counterterm-vs-contact}
	- \frac{1}{c^{(n_s)}(\frac{\pi}{\epsilon})} - C^{(n_s)}(\epsilon) = - \frac{\mu}{2 \pi} \frac{1}{a_0} \, ,
\end{equation}
then everything works out.

More particularly, if I fit them in such a way, the dispersion Lüscher zeta function is constant in $\epsilon$ and in $n_s$ (for eigen energies) and thus I actually do not have to run the limits at all: I can simply convert my energy level in discrete finite space to the infinite volume continuum results.

Note that:
\begin{itemize}
	\item eq.~\ref{eq:Counterterm-vs-contact} is a prediction -- we can go back and check wether this relation is True!
	\item @Evan: Everything I said here translates to $p \neq 0$ (as long as not evaluated at poles).
		Thus you can see that non-truncated dispersion zeta (the above zeta evaluated at $p_i^{(FV, n_s)} = \sqrt{2 \mu E_i^{(FV, n_s)}}$) will cause an additional $f^{(n_s)}(p, \epsilon)$ dependence in the LHS of eq.~\ref{eq:Counterterm-vs-contact}.
		But since $c^{(n_s)}$ is $p$ independent, this will cause finite $p_\text{fit}^{(FV, n_s)}$ residue:
		$$
			- \frac{1}{c^{(n_s)}(\frac{\pi}{\epsilon})} - C^{(n_s)}(\epsilon)  - f^{(n_s)}(p_\text{fit}^{(FV, n_s)}, \epsilon)= - \frac{\mu}{2 \pi} \frac{1}{a_0} \, .
		$$
		Thus, evaluating for different energies will cause a shift in the dispersion zeta of kind
		$$
			\hat S^{(FV, n_s)}(E_i^{(FV, n_s)}, \epsilon)
			=
			\tilde S^{(FV, n_s)}(E_i^{(FV, n_s)}, \epsilon)
			+
			\Delta^{(FV, n_s)}(\epsilon, \text{fit}, i)
		$$
		with
		$$
			\Delta^{(FV, n_s)}(\epsilon, \text{fit}, i)
			=
			f^{(n_s)}(p_\text{fit}^{(FV, n_s)}, \epsilon)
			-
			f^{(n_s)}(p_i^{(FV, n_s)}, \epsilon)
		$$
		I don't see that this generally cancels out.
\end{itemize}

\end{document}
