In two dimensions, assuming a contact interaction, it is convenient to write the effective range expansion \eqref{ere} in terms of a \emph{reduced} scattering length,
\begin{equation}\label{eq:2d contact phase shift}
    \cot \left(\delta_{2}(p)\right)
    =
    \frac{2}{\pi} \ln \left(p \tilde a_{2}\right),
%
    \hspace{0.5in}
    \text{where}
    \hspace{0.5in}
%
    \tilde a_{2}
    =
    R_{20}\exp\left\{-\frac{\pi}{2 a_{20}}\right\}
\end{equation}
with no additional shape parameters.
Using the infinite-volume relation \eqref{IV pole} with $p=0$, $D=2$, and a finite cutoff $\Lambda$ gives
\begin{equation}\label{eq:C2}
    C(\Lambda)
    =
    -\frac{\pi}{\mu \log \left(\tilde a_{2} \Lambda\right)}.
\end{equation}
Using this in the finite-volume relation \eqref{FV pole} and quantization condition \eqref{general luscher} yields
\begin{equation}\label{eq:first 2d}
    \frac{2}{\pi} \log \left(p\tilde a_{2}\right)
    =
    \lim_{\Lambda\to\infty}
    \left(
        -\frac{2}{\mu L^{2}} \sum_{\vec{q}}^{\Lambda} \frac{1}{E-\frac{\vec{q}^{2}}{2\mu}}
        -\frac{2}{\pi} \log (\Lambda / p)
    \right).
\end{equation}
The logarithmic dependence on $p$ makes this relation difficult for analysis, particularly for small $p$.
Furthermore, for $\tilde a_{2}$ sufficiently small (but positive)\footnote{
    Our definition of the scattering length in 2-d requires $\tilde a_{2}\ge 0$~\cite{Pupyshev:2014}.
} a bound state can occur, with imaginary momentum $p\to i\gamma$.
Then both sides become complex, further complicating the analysis.
Also, the momentum-\emph{independent} logarithmic counterterm needed to regulate the infinite sum is not manifest in the above expression.
To make the counterterm manifest, and to address the issue of small $p$ states and bound states, we subtract $\frac{2}{\pi}\log\left(\frac{pL}{2\pi}\right)$ on both sides,
\begin{align}
    \frac{2}{\pi} \log \left(\frac{2\pi \tilde a_{2}}{L}\right)
    &=
    \lim_{\Lambda\to\infty}
    \left(
        -\frac{2}{\mu L^{2}} \sum_{\vec{q}}^{\Lambda} \frac{1}{E-\frac{\vec{q}^{2}}{2\mu}}
        -\frac{2}{\pi} \log \left(\frac{\Lambda L}{2\pi}\right)
    \right)
    \nonumber\\
    &=
    \lim_{\Lambda\to\infty}
    \left(
        \frac{1}{\pi^2} \sum_{\vec{q}}^{\Lambda} \frac{1}{\left(\frac{\vec{q}L}{2\pi}\right)^2-x}-\frac{2}{\pi} \log \left(\frac{\Lambda L}{2\pi}\right)
    \right)
    \label{eq:second 2d}
\end{align}
where $x=2\mu EL^2/4\pi^2$ as always.
Setting $N=\Lambda L/\pi$ and $\left(\frac{\vec{q}L}{2\pi}\right)^2=\vec{n}^2$ gives
\begin{equation}
    \frac{2}{\pi} \log \left(\frac{2\pi \tilde a_{2}}{L}\right)
    =
    \frac{1}{\pi^2}
    \lim_{N\to\infty}
    \left(
        \sum_{|\vec{n}|\le \frac{N}{2}} \frac{1}{\vec{n}^2-x}
        -2\pi \log \left(\frac{N}{2}\right)
    \right)
    \equiv
    \frac{1}{\pi^2}S^{\spherical}_2\left(x\right)\ ,\label{eq:2d luscher}
\end{equation}
which defines the two-dimensional zeta function $S^{\spherical}_2$.
This matches the general result \eqref{spherical S} as long as we allow the limit
\begin{equation}
    \lim_{D\to2}\counterterm_D^\spherical \left(\frac{N}{2}\right)^{D-2}=2\pi \log \left(\frac{N}{2}\right).
\end{equation}

This two-dimensional \Luscher function \eqref{2d luscher} encompasses both bound and scattering states for the contact interaction.
Note the logarithmic dependence of the scattering length $\tilde a_{2}$ which requires an accompanying scale to render the argument of the logarithm dimensionless---we choose the infrared scale $L$, the linear size of the finite volume.
Finally, we note that for general finite-range \emph{S-wave} interactions, the L\"uscher formula in 2-D is
\begin{equation}\label{eq:full 2d luescher}
\cot(\delta_{2}(p))-\frac{2}{\pi}\log\left(\frac{pL}{2\pi}\right) = \frac{1}{\pi^2}S^\bigcirc_2\left(\left(\frac{pL}{2\pi}\right)^2\right)\ .
\end{equation}
This form was originally derived in \Ref{Beane:2010ny}, and is also consistent with \Ref{Zhu:2019dho} once the subtraction of the logarithm and the difference in definition of our zeta functions are taken into account. For higher partial waves we refer the reader to Ref.~\cite{Fiebig:1994qi}.
