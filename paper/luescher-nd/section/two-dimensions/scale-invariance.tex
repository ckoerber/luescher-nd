\subsection{The scale invariant limit in 2 dimensions}

In the spirit of Nussinov and Nussinov~\cite{nussinov2004bcs}, Nishida and Son~\cite{Nishida:2006eu} analyzed the two-dimensional unitary limit (or near it) by investigating the system in $2+\epsilon$ dimensions, using $\epsilon$ as a perturbative parameter (not to be confused with the lattice spacing).
They found the trivial solution in the scale-invariant limit; namely, the two-body system becomes non-interacting as the scattering length $\tilde a_{20} \goesto\infty$.
Our results also show this behavior, since, for fixed $L$, the intersection of a flat scattering amplitude given by $\tilde a_{20}$ goes to the non-interacting finite-volume energies as the scattering length grows large with either sign, indicating a trivial theory, as seen in \Figref{luescher2d}.

In exactly two dimensions, however, the logarithmic dependence of the scattering length requires an accompanying scale so as to make the argument of the logarithm dimensionless, a consequence of the fact that in 2 dimensions, the interaction parameters of the Schr\"odinger equation are dimensionless.
\emph{Dimensionful} observables occur via dimensional transmutation \cite{} which requires some fiducial external scale, which in any finite-volume calculation is naturally given by the size of the volume.
Thus the scale-invariant limit becomes sensitive to the order in which one takes either $\tilde a_{20}\goesto\infty$ and $L\goesto\infty$ limits.
Taking either limit first, indpendently of the other, will give the trivial scale-invariant limit found in \cite{Nishida:2006eu}.
However, there is a particular limit in which the scale-invariant system gives non-trivial solutions.
To see this, note that when $\tilde a_{20}=\frac{L}{2\pi}$ the the finite-volume energies must zero the zeta function in the quanization condition~\eqref{2d luscher}, as in three dimensions when $a_{30}\to\infty$ and one dimension when $a_{10}\to 0$.

In those dimensions, such a situation is akin to the unitary limit, once the infinite volume limit has been taken.
This limit can be done independently of taking $a_{30}\to\infty$ (3-D) or $a_{10}\to0$ (1-D).
However, in two dimensions, because of the logarithmic dependence, demanding that the zeta function remains zeroed axis requires the infinite volume limit $L\to\infty$ taken together with $\tilde a_{20}$.
In particular, for the \emph{non-trivial} scale-invariant limit, one must take both $\tilde a_{20}\goesto \infty$ and $L\to\infty$ limits simultaneously \emph{such that} $L/\tilde a_{20}=2\pi$.
Such a procedure keeps the dispersion zeta function $S^{\dispersion}_2$ zeroed at each step of the extrapolation, resulting in a non-trivial, scale-invariant, two-body system in two dimensions.
