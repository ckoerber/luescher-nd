%!TEX root =  ../master.tex
\subsection{Two dimensions}
\todo{@TL: Stuff}

Using eq.~\eqref{spherical FD} with eq.~\eqref{IV pole} in two dimensions gives
\begin{equation}\label{eq:Q1}
\cot \delta_{0}(p)-i=\frac{4}{m} \left(\frac{-1}{\sum_nC_{2n}(\Lambda)p^{2n}}+I_{0}(p, \Lambda)\right)\ ,
\end{equation}
and the corresponding ERE in two dimensions is
\begin{equation}\label{eq:Q2}
\cot\delta_0(p)=\frac{2}{\pi}\log(p a_0)+\frac{1}{2} r p^2 + \ldots\ ,
\end{equation}
where $a_0$ is the s-wave scattering length and $r$ is some shape parameter\footnote{Note that $r$ has units of area.}.  The ellipsis also represents higher order shape parameters and contains terms that are even powers of $p$, i.e.  terms that are analytic in the energy.   The finite volume integral $I_0$, assuming a hard cutoff $\Lambda$, is
\begin{equation}\label{eq:I0 2d}
I_0(p,\Lambda)=\frac{1}{(2 \pi)^{2}} \int^{\Lambda} \mathrm{d}^2q\ \left[\mathcal{P}\left(\frac{1}{\frac{p^2}{m}-\frac{\bm{q}^{2}}{m}}\right)-i \pi m\delta(\bm q^2-p^2)\right]\ .
\end{equation}
Drawing from our discussion of L\"uscher's formula in three dimensions, we immediately evaluate the two-dimensional $I_0$ in a cartesian basis.  
%We first do the calculation in polar coordinates.  In this case eq.~\eqref{eqn:I0} becomes
%\begin{align}
%I_0(p,\Lambda)&=\frac{m}{(2 \pi)} \int^{\Lambda}_0 \mathrm{d}q \ q\ \left[\mathcal{P}\left(\frac{1}{p^2-q^{2}}\right)-i \pi \delta(q^2-p^2)\right]\\
%&=\frac{m}{(2 \pi)} \int^{1}_0 \mathrm{d}\tilde q\  \tilde q\ \left[\mathcal{P}\left(\frac{1}{\tilde p^2-\tilde q^{2}}\right)-i \frac{\pi}{2|\tilde q|} \delta( \tilde q-\tilde p)\right]\\
%&=\frac{m}{(2 \pi)} \int^{1}_0 \mathrm{d}\tilde q\ \mathcal{P}\left(\frac{\tilde q}{\tilde p^2-\tilde q^{2}}\right)-i \frac{m}{4}  \,\label{eqn:Q3}
%\end{align}
%where $\tilde q=q/\Lambda$ and $\tilde p=p/\Lambda$.  \texttt{Mathematica} tells me that this integral can be done,
%\begin{align}
%\frac{m}{(2 \pi)} \int^{1}_0 \mathrm{d}\tilde q\ \mathcal{P}\left(\frac{\tilde q}{\tilde p^2-\tilde q^{2}}\right)&=
%\frac{m}{(2 \pi)} \log\left(\frac{\tilde p}{\sqrt{1-\tilde p^2}}\right)\\
%&=\frac{m}{2 \pi} \log\left(\frac{p}{\Lambda}\right)+\frac{m}{4 \pi}\tilde p^2+\frac{m}{8 \pi}\tilde p^4+\mathcal{O}(\tilde p^6)\ ,
%\end{align}
%where in the second line I expanded in small $\tilde p$. Combining this result with eqs.~\eqref{eqn:Q1} and~\eqref{eqn:Q2} gives
%\begin{align}
%\frac{2}{\pi}\log(p a_0)&=\frac{4}{m} \left(\frac{-1}{C_{0}(\Lambda)}+\frac{m}{2 \pi} \log\left(\frac{p}{\Lambda}\right)+\mathcal{O}(\tilde p^2)\right)\nonumber\\
%&=\frac{4}{m} \frac{-1}{C_{0}(\Lambda)}+\frac{2}{\pi}\log\left(\frac{p}{\Lambda}\right)+\mathcal{O}(\tilde p^2)\ ,
%\end{align}
%which implies that
%\begin{equation}
% \frac{-1}{C_{0}(\Lambda)}=\frac{m}{2\pi}\log\left(a_0\Lambda\right)\ .
% \end{equation}
%
%  
% \subsection{Result in cartesian coordinates}
We first move to dimensionless variables $\tilde p=p/\Lambda$ and $\tilde q= q/\Lambda$, giving
 \begin{equation}
 \frac{m}{(2 \pi)^2} \int^{1} \mathrm{d}^2\tilde q\ \mathcal{P}\left(\frac{1}{\tilde p^2-\tilde q_x^{2}-\tilde q_y^{2}}\right)-i \frac{m}{4} \ .
 \end{equation}
We then split the integral into a region within a unit sphere $I_0^s$ and the remaining corners of the unit cube $I_0^c$,
 \begin{align}
 \frac{m}{(2 \pi)^2} \int^{1} \mathrm{d}^2\tilde q\ \mathcal{P}\left(\frac{1}{\tilde p^2-\tilde q_x^{2}-\tilde q_y^{2}}\right)=&I_0^s+I_0^c\\
 =&\frac{m}{(2 \pi)} \int^{1}_0 \mathrm{d}\tilde q\ \mathcal{P}\left(\frac{\tilde q}{\tilde p^2-\tilde q^{2}}\right)+\left.\frac{8m}{(2\pi)^2}\int_0^{\pi/4}d\phi\int_1^{R(\phi)}dr\frac{r}{\tilde p^2-r^2}\right|_{\tilde p=0}\ ,
 \end{align}
% The factor of 4 comes from using the symmetry of the kernal under $\tilde q_i\mapsto-\tilde q_i$, and therefore reducing the integration region
% \begin{displaymath}
% \int_{-1}^1d\tilde q_i\mapsto2\int_0^1d\tilde q_i\ .
% \end{displaymath}
% In two dimensions $I_0$ is logarithmically divergent.  The issue is the constant offset when doing the cartesian integration.  This offset comes from the corners of the integration of the Brillouin zone.  In 2-d, this offset is a finite number, and so we can just try to directly calculate the integral in these corners.  A little monkeying around shows that this offset is given by
%\begin{equation}
%\text{offset} = \left.8\int_0^{\pi/4}d\phi\int_1^{R(\phi)}dr\frac{r}{\tilde p^2-r^2}\right|_{\tilde p=0}\ .
%\end{equation}
where $R(\phi)=1/\cos(\phi)$.  {\color{red}Need to make a figure to show the integration regions!}  In two dimensions $I_0$ is logarithmically divergent, as can be seen when evaluating $I_0^s$,
\begin{equation}\label{eq:I0s}
I_0^s=\frac{m}{2 \pi} \log\left(\frac{p}{\Lambda}\right)+\frac{m}{4 \pi}\tilde p^2+\frac{m}{8 \pi}\tilde p^4+\mathcal{O}(\tilde p^6)\ .
\end{equation}
The remaining integral, $I_0^c$, is finite.  Performing the integral over $r$ gives
\begin{equation}
I_0^c=-\frac{4m}{(2\pi)^2}\int_0^{\pi/4}d\phi \log\left(\frac{\tilde p^2-\cos(\phi)^{-2}}{\tilde p^2-1}\right)\ .
\end{equation}
Though we cannot directly evaluate the remaining integral over $\phi$, there is nothing pathologically wrong with it (i.e. no divergences).  We therefore do a series expansion of the kernal and integrate the expansion term by term, giving
\begin{equation}\label{eq:I0c}
I_0^c=\frac{m}{(2\pi)^2}\left[4 \left(G-\frac{\pi}{2}\log(2)\right) - \frac{1}{2} \tilde p^2 (-2 + \pi) - 
 \frac{1}{16}\tilde p^4 (-8 + 5 \pi)\right] +\mathcal{O}(\tilde p^6)\ .
\end{equation}
Combining eqs.~\eqref{I0s} and~\eqref{I0c}, and ignoring all terms that vanish in the $\Lambda\to\infty$ (i.e. $\tilde p\to 0$) limit gives the desired result,
\begin{equation}
I_0=-i \frac{m}{4}+\frac{m}{2 \pi} \log\left(\frac{p}{\Lambda}\right)+\frac{m}{\pi^2} \left(G-\frac{\pi}{2}\log(2)\right)\ .
\end{equation}

\subsubsection{Determing $C_0(\Lambda)$} 
NEEDS WORK!!!

Now that we have a stable counterterm, we can combine everything into the quantization condition (making sure I account for factors of $m$ and $\pi^2$, etc.) and get
\begin{align}
\frac{2}{\pi}\log(p a_0)&=\frac{4}{m} \left(\frac{-1}{C_{0}(\Lambda)}+\frac{m}{2 \pi} \log\left(\frac{p}{\Lambda}\right)+\frac{m}{\pi^2}\left(G-\frac{\pi}{2}\log(2)\right)+\mathcal{O}(\tilde p^2)\right)\nonumber\\
&=\frac{4}{m} \frac{-1}{C_{0}(\Lambda)}+\frac{2}{\pi}\log\left(\frac{p}{\Lambda}\right)+\frac{4}{\pi^2}\left(G-\frac{\pi}{2}\log(2)\right)+\mathcal{O}(\tilde p^2)\ .
\end{align}
For this equality to hold, we must have
\begin{equation}\label{eqn:ta da}
\boxed{
 \frac{-1}{C_{0}(\Lambda)}=\frac{m}{2\pi}\log\left(a_0\Lambda\right)-\frac{m}{\pi^2}\left(G-\frac{\pi}{2}\log(2)\right)
 }\ .
 \end{equation}
We display the coefficient to 15 decimal places,
\begin{displaymath}
G-\frac{\pi}{2}\log(2)=-0.1728274509745820501957409\ldots
\end{displaymath}

\subsubsection{Cartesian $S_2$}
We have
\begin{equation}
 \frac{-1}{C_{0}(\Lambda)}+\frac{1}{L^2}\sum_{q_i\in\ (-\Lambda,\Lambda]}\frac { 1 } { E - \frac{\bm{q}^2}{m} }=0\ .
 \end{equation}
 Using eq.~\eqref{eqn:ta da} gives
\begin{multline}
\frac{m}{2\pi}\log\left(a_0\Lambda\right)-\frac{m}{\pi^2}\left(G-\frac{\pi }{2}\log(2)\right)=-\frac{1}{L^2}\sum_{q_i\in\ (-\Lambda,\Lambda]} \frac { 1 } { E - \frac{\bm{q}^2}{m} }\\
\implies
\frac{2}{\pi}\log (pa_0)+\frac{2}{\pi}\log(\Lambda/p)=-\frac{4}{mL^2}\sum_{q_i\in\ (-\Lambda,\Lambda]}  \frac { 1 } { E - \frac{\bm{q}^2}{m} }+\frac{4}{\pi^2}\left(G-\frac{\pi}{2}\log(2)\right)\\
\implies
\frac{2}{\pi}\log (pa_0)=-\frac{4}{mL^2}\sum_{q_i\in\ (-\Lambda,\Lambda]}  \frac { 1 } { E - \frac{\bm{q}^2}{m} }-\frac{2}{\pi}\log(\Lambda/p)+\frac{4}{\pi^2}\left(G-\frac{\pi}{2}\log(2)\right)
\end{multline}
If we use the phase shift relation for 2-d, $\cot\delta(p)=\frac{2}{\pi}\log (pa_0)$ and replace $E=p^2/m$, then we have
\begin{align}
\cot\delta(p)&=-\frac{4}{mL^2}\sum_{q_i\in\ (-\Lambda,\Lambda]}  \frac { 1 } { \frac{p^2}{m} - \frac{\bm{q}^2}{m} }-\frac{2}{\pi}\log\left(\frac{\Lambda}{p}\right)+\frac{4}{\pi^2}\left(G-\frac{\pi }{2}\log(2)\right)\\
&=-\frac{4}{L^2}\sum_{q_i\in\ (-\Lambda,\Lambda]}  \frac { 1 } {p^2 - \bm{q}^2 }-\frac{2}{\pi}\log\left(\frac{\Lambda}{p}\right)+\frac{4}{\pi^2}\left(G-\frac{\pi }{2}\log(2)\right)\\
&=4\sum_{q_i\in\ (-\Lambda,\Lambda]}  \frac { 1 } {(\bm{q}L)^2-(pL)^2 }-\frac{2}{\pi}\log\left(\frac{\Lambda L}{pL}\right)+\frac{4}{\pi^2}\left(G-\frac{\pi }{2}\log(2)\right)\\
&=\frac{1}{\pi^2}\sum_{q_i\in\ (-\Lambda,\Lambda]}  \frac { 1 } {\left(\frac{\bm{q}L}{2\pi}\right)^2-\left(\frac{pL}{2\pi}\right)^2 }-\frac{2}{\pi}\log\left(\frac{\Lambda L}{pL}\right)+\frac{4}{\pi^2}\left(G-\frac{\pi }{2}\log(2)\right).
\end{align}
We now define the \emph{cartesian} $S_2$\label{eqn:S2}\footnote{For comparison, the \emph{spherical} (polar) $S_2$ function is
\begin{equation}\label{eqn:S2 polar}
S_2(x)\equiv\sum_{|\bm n|\le\Lambda'}\frac { 1 } { \bm{n}^2 -x}-2\pi\log\left(\Lambda'\right)\ .
\end{equation}}
\begin{align}
S_2(x)\equiv&\sum_{n_i\in\ (-\Lambda',\Lambda']}\frac { 1 } { \bm{n}^2 -x}-2\pi\log\left(\Lambda'\right)+4\left(G-\frac{\pi }{2}\log(2)\right)\\
=&\sum_{n_i\in\ (-\Lambda',\Lambda']}\frac { 1 } { \bm{n}^2 -x}-2\pi\log\left(\mathcal{L}_\square\Lambda'\right)\ ,
\end{align}
where $\Lambda'=\frac{\Lambda L}{2\pi} \in\ \mathbb{N}$ and
\begin{equation}
\mathcal{L}_\square=\exp\left(\log(2)-G\frac{2}{\pi}\right)=1.116306393581637659468497\ldots
\end{equation}
So we finally have
\begin{equation}\label{eqn:S2}
\cot\delta(p)=\frac{1}{\pi^2}S_2\left(\left(\frac{p L}{2\pi}\right)^2\right)+\frac{2}{\pi}\log\left(\frac{pL}{2\pi}\right)\ .
\end{equation}
Because of the logarithmic dependence in $p$ of the phase shift relation in 2-d, it is instead more convenient to move the logarithmic term on the RHS of the above equation to the LHS, which cancels the logarithmic dependence on $p$,
\begin{equation}
\frac{2}{\pi}\log\left(\frac{2\pi a_0 }{L}\right)=\frac{1}{\pi^2}S_2\left(\left(\frac{p L}{2\pi}\right)^2\right)\ .
\end{equation}

\subsubsection{Zero-range contact interaction}

