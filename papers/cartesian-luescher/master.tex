\documentclass[aps,superscriptaddress,tightenlines,nofootinbib,floatfix,longbibliography,notitlepage]{revtex4-1}
\usepackage[left=18mm,right=19mm,top=23mm,bottom=16mm]{geometry}
\usepackage{amsmath,amssymb}
\usepackage{bm}
\usepackage{comment}
\usepackage{graphicx}
\usepackage[dvipsnames]{xcolor}
\usepackage{slashed}
\usepackage[multidot]{grffile} %because multiple dots in filenames confuse latex
\usepackage[
    colorlinks=true,
    allcolors=blue
]{hyperref}

\newcommand\todo[1]{{\bf\color{red}TODO: #1}}

%%%
%%%     Incorporate Repository Information
%%%

\providecommand{\repositoryInformationSetup}{} % Fallback definition if not compiled with `make DRAFT=1`.
\repositoryInformationSetup

%%%
%%%     Include latex-base macros
%%%

\input{macros} % input rather than include so we don't create macros.aux

%%%%
%%%%    Document preparation
%%%%

\begin{document}

\title{Don't Cut Corners when Deriving L\"{u}scher's Formula}
\newcommand{\ikp}{
    Institut f\"{u}r Kernphysik,
    Forschungszentrum J\"{u}lich, 54245 J\"{u}lich Germany
}

\newcommand{\ias}{
    Institute for Advanced Simulation,
    Forschungszentrum J\"{u}lich, 54245 J\"{u}lich Germany
}

\newcommand{\bonn}{
    Helmholtz-Institut f\"{u}r Strahlen- und Kernphysik,
    Rheinische Friedrich-Williams-Universit\"{a}t Bonn, 53012 Bonn Germany
}

\newcommand{\jsc}{
    J\"{u}lich Supercomputing Center,
    Forschungszentrum J\"{u}lich, 54245 J\"{u}lich Germany
}

\newcommand{\berkeley}{
    Department of Physics,
    University of California, Berkeley, CA 94720, USA
}


\author{Christopher K\"orber}   \affiliation{\ias} \affiliation{\ikp} \affiliation{\bonn} \affiliation{\berkeley}
\author{Evan Berkowitz}         \affiliation{\ias} \affiliation{\ikp}
\author{Thomas Luu}             \affiliation{\ias} \affiliation{\ikp} \affiliation{\bonn}





\date{\today}

\begin{abstract}
L\"{u}scher's generalized zeta functions, encountered when translating finite-volume spectra into infinite-volume phase shifts, are sums typically over a momentum whose magnitude is cut off and regularized using a spherical integral.  By exactly solving two-body quantum mechanical problems we show that the correct procedure is to formulate these functions so that they match the procedure executed when taking the continuum limit, so that in a cubic volume the entirety of the Brillouin zone is accounted for, changing the counter term.
\todo{Also we are smart in other ways.  Dispersion etc.}
\end{abstract}

\maketitle

\section{Introduction}\label{sec:intro}

Many physically interesting systems comprise strongly-interacting fermions.
In three spatial dimensions the scattering of fermions with a short-range interaction can be completely characterized by a scattering length, and when that length diverges the details of the potential are washed out and no dimensionful scales remain.
Such \emph{unitary fermions} exhibit interactions as strong as can be without forming bound states, and provide an interesting guide for understanding other strong interactions because of their universal behavior.
For example, the nuclear interaction in the deuteron channel has an extremely long scattering length, and trapped ultracold atoms can be tuned to unitarity by applying external magnetic fields and leveraging Feshbach resonances.

By tuning a quantum-mechanical two-body contact interaction, one should be able to completely control the scattering length and, absent other interactions, have that scattering length completely describe the scattering.
With such an interaction in hand, a variety of interesting many-body problems are unlocked.
Since all other dimensionful quantities are gone, all observables must be determined by naive dimensional analysis in the the density, times some non-perturbative numerical factor, such as the Bertsch parameter\cite{PhysRevC.60.054311} in the case of the energy density.

In fact, a contact interaction can be shown to always produce momentum-independent scattering amplitudes (in three dimensions, for example, a momentum-independent $p \cot \delta$), and it ought to be possible to produce any amplitude, unless otherwise restricted by the Wigner bound\cite{Wigner:1955zz,Phillips:1996ae,Hammer:2010fw}.

Such scale-free results must result from peculiar potentials.
In three dimensions, for example, a delta function potential requires regulation, and to get scale-free dynamics its dimensionful strength must be sent to zero with the removal of the regulator in just such a way as to keep the phase shift at $\pi/2$.
In one dimension the strength of the contact interaction is also dimensionful and a delta function potential needs no regulation, but nevertheless is regulated when space is discretized; in two dimensions the strength of the delta function potential is dimensionless, which entails a more complicated story we discuss in \Secref{2D}.

Numerical computations are often performed in discretized boxes with periodic boundary conditions.
\Luscher's finite-volume formalism\cite{Hamber198399,luscher:1986I,luscher:1986II,wiese1989,Luscher1991,Luscher1991237} is the method by which one can extract infinite-volume real-time scattering data from the finite-volume Euclidean spectrum of a theory, taking advantage of the interplay between the physical scattering and the finite-volume boundary conditions in determining the spectrum.  Recently there has been an investigation of \Luscher's formalism for continuous scattering within a crystal lattice~\cite{Valiente:2015oya}.

The usual understanding of \Luscher's formalism is that one should find the continuum zero-temperature finite-volume energy levels, holding the physical volume fixed, and put that cold, continuum spectrum through \Luscher's formula to extract continuum scattering data. 

In practice, few results of lattice QCD calculations are zero-temperature- or, more seriously, continuum-extrapolated, but are nevertheless put through \Luscher's formula to get an estimate of the continuum scattering data, assuming thermal and discretization effects to be much smaller than the statistical uncertainties\todo{cite cite cite}.
In particular, no continuum-limit study of any baryonic channel exists, even at unphysically heavy pion masses.

While alternatives, including the potential method (see, for example \Refs{Ishii:2006ec,Nemura:2008sp,Aoki:2009ji,Murano:2011nz,Aoki:2012bb,Kurth:2013tua,Sugiura:2017vwo,Yamazaki:2019vid,Aoki:2017yru,Yamazaki:2018qut,Iritani:2017rlk,Iritani:2018zbt,Gongyo:2018gou,Akahoshi:2019klc,Namekawa:2019xiy}), the mapping onto harmonic oscillators \Ref{McElvain:2019ltw} and the imposition of spherical walls\cite{Borasoy:2007vy,Borasoy:2007vi,Lee:2008fa,Epelbaum:2008vj,Epelbaum:2010xt,Lu:2015riz,Elhatisari:2015iga,Elhatisari:2016owd,Elhatisari:2016hby,Klein:2018lqz,Li:2019ldq,Bovermann:2019jbt,Lahde:2019npb}, can be used to translate finite-volume physics to infinite-volume observables, here we focus on the \Luscher finite-volume formalism.
Moreover, to our knowledge, no numerical work leveraging these methods is in the continuum, either.  
Here, we construct example Hamiltonians explicitly and diagonalize them exactly, albeit numerically.
This allows us to circumvent all of the issues of statistical uncertainty that accompanies Monte Carlo data, and lets us completely isolate the features of the formalism itself, removing, for example, any finite-temperature effects that should in principle be extrapolated away in any finite-temperature method like Lattice QCD.

We find that it is in practice difficult to reliably extrapolate the spectrum to the continuum limit in a way that reproduces the exact known result, but that taking the continuum limit of the lattice-artifact-contaminated phase shifts sometimes can produce a more reliable result.

Our main innovation, however, is to explain how to incorporate lattice artefacts into \Luscher's formula, for systems described by a contact interaction, accounting both for the Brillouin zone of the lattice and the lattice-induced dispersion relation.

While not universal, this lattice improvement can be quite useful for a contact interaction.
In pursuit of a lattice formulation of unitary fermions, the authors of \Ref{Endres:2011er} followed the tuning procedure of \Ref{Lee:2007ae}, parametrizing the contact interaction as a sum of a tower of Galilean-invariant operators, tuning their coefficients so as to drive the lowest interacting energy levels to the zeros of the \Luscher finite-volume zeta function.
However, in \Ref{Endres:2012cw} they found that even with a highly-improved construction the states ultimately deviated from a $\pi/2$ phase shift (see, for example, Figure 3).
In \Ref{He:2019ipt} the lattice implementation was smeared to reduce errors due to discretization, however a direct comparison of other methods with theirs was not possible for us since we were not able to identify the discretization parameters for the presented phase shifts (Fig. 7).

We introduce a new continuum-limit prescription for achieving unitarity in lattice simulations by tuning just the simplest contact operator, but taking the discretization effects into account by incorporating the lattice dispersion relation into the finite-volume zeta function, both in the tuning step and in the analysis step.
By re-tuning the interaction at each lattice spacing we can very easily and smoothly take the continuum limit after applying the lattice-aware finite-volume formula.
We demonstrate that this allows us to maintain a constant phase shift deep into the spectrum, covering as many \Aoneg states as exist in the lattice of interest.

This paper is organized as follow.  In \Secref{scattering} we give a brief summary of two particle scattering in $D$ dimensions.
In \Secref{hamiltonian} we give specifics about the latticized contact-interaction Hamiltonians we study numerically.
In \Secref{luescher} we provide a traditional continuum derivation of \Luscher's formula and in \Secref{dispersion} explain how to adapt it to include finite spacing effects by truncating the usual sum to just the momentum modes in the lattice and incorporating the dispersion relation into the appropriate propagators, yielding a lattice-improved generalized \Luscher zeta function.

Then, we leverage our dispersion zeta function, studying concrete examples.
In \Secref{3D} we study the three-dimensional case.
First we compare a continuum-extrapolated energy spectrum fed through the continuum zeta function and the continuum extrapolation of the finite-spacing spectra fed through the continuum zeta.
In \Secref{3D} we tune and analyze the same problem using our lattice-aware dispersion zeta function, and show that the resulting scattering $p\cot\delta$ remains constant deep into the spectrum; when we tune to unitarity the results stay at the expected value as accurately as the initial tuning is made modulo propagated numerical uncertainties.
We then study the one dimensional case in \Secref{1D}, where the absence of a counterterm makes things particularly simple.
In \Secref{2D} we repeat the story for the more intricate two-dimensional case, where dimensional transmutation and logarithmic singularities require special attention and care.
We find that our lattice-aware \Luscher function handles this case with no difficulty.
Further, in all dimensions considered here we provide correction terms that come about when using energies calculated in a discrete space but fed through continuum \Luscher formula, which when applied to three dimensions corrects for the deviation found in \Ref{Endres:2012cw}.
Our corrections are valid only for the case of a contact interaction.
Finally, we recapitulate our findings in \Secref{conclusion} and discuss future directions.

\section{Discretized Hamiltonian}\label{sec:hamiltonian}

A two-body system interacting via a contact interaction is the focus of our study.
By definition, only S-wave information are not vanishing for a contact interaction and the corresponding ERE has no effective range (and higher terms).
Thus, to obtain unitarity, one only has to tune the interaction to the S-wave scattering length.
The hamiltonian, with contact strength $C$, describing the system is given by
\begin{equation}
    \label{eq:particle hamiltonian}
    \hat H = \frac{\hat p_1^2}{2 m_1} + \frac{\hat p_2^2}{2 m_2} + C \delta(\hat x_1 - \hat x_2)
    \, .
\end{equation}
The subscripts indicate the particle of the position and momentum operators.
This hamiltonian becomes, moving to center-of-mass and relative coordinates,
\begin{equation}
    \label{eq:hamiltonian}
    \hat H = \frac{\hat P^2}{2 M} + \frac{\hat p^2}{2 \mu} + C \delta(\hat x)
\end{equation}
where capital letters represent center-of-mass variables, lower case implies relative coordinates.
The problem is reducded to an effective one-body problem once we specialize to the rest frame, setting $P=0$.

We consider a finite cubic volume (FV) of linear size $L$ with periodic boundary conditions and lattice spacing $\epsilon$ so that $N=L/\epsilon$ is an even integer that counts the number of sites in one spatial direction.

The contact interaction is implemented on the lattice as an entirely local operator, vanishing everywhere except at the origin where it is of strength $C$ (e.g., the contact interaction is not smeared).

In contrast, to analyze the effects of discretizations, we study a variety of kinetic operators which we distinguish by the $\nstep$ label, which indicates how many nearest neighbors in each direction go into the finite-difference Laplacian.
For example, $\nstep=1$ denotes the symmetric nearest-neighbor finite-difference Laplacian.
We consider further stencils which extend on-axis steps so that the finite difference Laplacian is a $(1+2\nstep D)$-point stencil in $D$ dimensions,
\begin{equation}
    \left\langle \vec{r}' \middle| H \middle| \vec{r} \right\rangle
    \rightarrow
    H_{\vec{r}',\vec{r}}^{(L,\epsilon,\nstep)}
    =
    - \frac{1}{2 \mu \epsilon^2}
        \sum_{d=1}^{D} \sum_{s=-\nstep}^{+\nstep}
            \gamma^{(\nstep)}_{|s|} \delta_{\vec{r}',\vec{r}+\epsilon s \vec{e}_d}
    + \frac{1}{\epsilon^D}C(\epsilon) \delta_{\vec{r}',\vec{r}}\delta_{\vec{r},\vec{0}}
\end{equation}
where the spatial indices are understood modulo the periodic boundary conditions of the lattice.
In momentum space, this Hamiltonian may be written as
\begin{align}
    \label{eq:p space hamiltonian}
    \left\langle \vec{p}' \middle| H \middle| \vec{p} \right\rangle
    \rightarrow
    H_{\vec{p}',\vec{p}}^{(L,\epsilon,\nstep)}
    &=
    \delta_{\vec{p}',\vec{p}} \frac{1}{2\mu} \sum_{d=1}^{D} \omega^{(\nstep)}(p_d,\epsilon)
    +\frac{1}{L^D}C(\epsilon)
    \\
    \label{eq:gamma definition}
    \omega^{(\nstep)}(p_d,\epsilon)
    &= \frac{1}{\epsilon^2} \sum_{s=0}^{\nstep} \gamma_{s}^{(\nstep)} \cos(s p_d \epsilon)
\end{align}
where $\vec{p} = 2\pi \vec{n}/L$ for a $D$-plet of integers $\vec{n} \in (-N/2, +N/2]^D$, and the coefficients $\gamma_{s}^{(\nstep)}$ are determined by requiring the dispersion relation be as quadratic as possible,
\begin{equation}
    \label{eq:gamma determination}
    \omega^{(\nstep)}(p_d,\epsilon) \overset{!}{=} p_d^2 \left[ 1 + \order{(\epsilon p_d)^{2\nstep}}\right].
\end{equation}
The resulting dispersion relations are presented in \Figref{dispersion relation} for a variety of $\nstep$s and
in \Appref{coefficients} we collect the required $\gamma$ coefficients.
In addition, we use a nonlocal operator, denoted by $\nstep=\infty$ which, in momentum space can be implemented to multiplying by $p^2$ directly,
\begin{equation}
    \omega^{\infty}(p_d,\epsilon) = p_d^2,
\end{equation}
including at the edge of the Brillouin zone, the Laplacian implementation of the ungauged SLAC derivative.
Including the edge does not introduce a discontinuity at the boundary (though it does introduce a cusp).

Once constructed, a projection operator, is added to this Hamiltonian
\begin{equation}
    H(\alpha) = H + \alpha (\one - P_{\Aoneg}) \, ,
\end{equation}
where $P_{\Aoneg}$ is a projector to the \Aoneg irrep (needed for extracting S-wave information in the infinite volume).
Because $P_{\Aoneg}$ commutes with $H$, $H$ and $H(\alpha)$ have the same spectrum within the $\Aoneg$ irrep.
If $\alpha$ is much larger than the expected energies of the Hamiltonian, the \Aoneg states remain low-lying and all other states are shifted to much higher energies.
Then, exactly diagonalizing $H(\alpha)$ instead of $H$ provides an easier extraction of \Aoneg eigenenergies.

Throughout we focus on a three-dimensional system, though in \Appref{two-d} we study a two-dimensional system, where logarithmic divergences warrant special attention.

\begin{figure}
    \subsubsection{Dispersion L\"{u}scher removes induced artifacts}

In this section, we again attempt to tune our contact interaction to unitarity by matching the first zero of the \Luscher zeta function.
However, the difference is that at each lattice spacing we tune to that spacing's respective $S^{\dispersion N}_D$, leveraging the dispersion relation for that derivative.
Then, when we extract finite-volume and finite-spacing energy levels, we put them through the dispersion equation \todo{eqref} using the same $S$ function.\footnote{We actually make the replacement $\F_D^\epsilon\goesto\F_D$ to avoid numerically evaluating $F_D^\epsilon$.  With a controlled continuum limit the error from this change obviously vanishes.}
The numerical results of said procedure are shown in \Figref{unimproved dispersion}.
Note that the results for $p\cot\delta$ are now flat across the spectrum, matching the known result for a contact interaction.
Moreover, comparing the scale to that in, for example, \Figref{unimproved spherical}, there the deviations were of order~1, while here the results remain within $10^{-8}$ of zero, with the value entirely reflecting how well the contact interaction was tuned.

\begin{figure}[htb]
    \scalebox{0.9}{\input{figure/ere-contact-fitted_a-inv_+0.0_zeta_dispersion_projector_a1g_n-eigs_200.pgf}}
    \caption{The same as \Figref{unimproved spherical}, but tuned and subsequently analyzed using the appropriate latticized \Luscher function, matching the cutoff on the sum to the lattice scale and accounting for the dispersion relation.}
    \label{fig:unimproved dispersion}
\end{figure}

\begin{figure}[hbt]
    \scalebox{0.9}{\input{figure/ere-contact-fitted_a-inv_-5.0_zeta_dispersion_projector_a1g_n-eigs_200.pgf}}
    \caption{The same as \Figref{unimproved spherical}, but tuned and subsequently analyzed using the appropriate latticized \Luscher function, matching the cutoff on the sum to the lattice scale and accounting for the dispersion relation for finite scattering lenght.}
    \label{fig:unimproved dispersion finite a}
\end{figure}

In \Figref{dispersion running of strength} we show how the strength of the contact interaction runs with the lattice scale.  According to \todo{something we know} it should be \todo{some formula that depends on cutoff}.

\begin{figure}
    \input{figure/contact-scaling-contact-fitted_a-inv_+0.0_zeta_dispersion_projector_a1g_n-eigs_200.pgf}
    \caption{
        Scaling of the contact interaction strength $C(\epsilon)$ fitted using the dispersion method at unitarity.
        Data points are fitted values, solid lines are analytical scaling predictions following $C(\epsilon) = \frac{2}{\mu} \frac{1}{\mathcal L^{\dispersion n_\mathrm{step}}} \epsilon $ and the dashed line corresponds to the spherical predictions.
        Bar diagrams below present the absolute error between prediction and extracted value.
    }
    \label{fig:dispersion running of strength}
\end{figure}

\clearpage

    \caption{We show the continuum dispersion relation of energy as a function of momentum for different one-dimensional $\nstep$ derivatives.  For a finite number of lattice points $N$, the allowed momenta are evenly-spaced in steps of $2\pi/N$.
    As additional steps are incorporated into the finite difference, the dispersion relation more and more faithfully reproduces the desired $p^2$~behavior of $\nstep=\infty$.
    }
    \label{fig:dispersion relation}
\end{figure}

\section{The Usual, Spherical \Luscher's Formula}\label{sec:spherical}


Here we present a $D$-dimensional derivation of \Luscher's formula that roughly follows \Ref{Beane:2003da}, although the technology and sophistication of the finite-volume formalism has grown substantially \todo{cite cite cite}.  Assuming an interaction given by an tower of derivative contact operators
\begin{equation}
    V(p) = +\sum_n C_{2n}(\Lambda) p^{2n}
\end{equation}
where the interaction strengths depend on the regulator and carry spatial-dimension-dependent units.
The scattering amplitude is given by the bubble sum depicted in \Figref{bubbleSum}.

\begin{figure}[ht!]
\center
\includegraphics[width=.8\columnwidth]{figure/bubbleSum.pdf}
\caption{Bubble sum. Each line represents a propagator, each vertex represents $-i \sum_n C_{2n}(\Lambda) p^{2n}$, and the bubble is given by $I_0$ (see also \Figref{I0}).\label{fig:bubbleSum}}
\end{figure}

This bubble sum is a geometric series and gives\cite{Kaplan:1998we,Beane:2003da}
\begin{equation}\label{eq:scattering amplitude}
\amplitude = \frac{-\sum_n C_{2n}(\Lambda) p^{2n}}{1-I_0(p,\Lambda) \sum_n C_{2n}(\Lambda) p^{2n}},
\end{equation}
where $p$ is the relative momentum,  and $I_0(p,\Lambda)$ is a $D$-dependent function that arises from integrating the loop shown in \Figref{I0},
\begin{align}
    I_0(p)
    &=-i\int^{\Lambda/2}
        \frac { \mathrm {d}q_0}{2\pi}\ \frac{\mathrm { d } ^ { D } \vec{ q } } { (2\pi)^ { D } }
        \left( \frac { i } { \frac{E}{2} + q _ { 0 } - \frac{\vec{q}^2}{2m} + i \epsilon } \right)
        \left( \frac { i } { \frac{E}{2} - q _ { 0 } - \frac{\vec{q}^2}{2m} + i \epsilon } \right)
    \nonumber\\
    &=\frac{\Omega_D}{(2\pi)^D}\int^{\Lambda/2}  \mathrm { d } q \ q^{D-1}\left[\mathcal{P} \left( \frac { 1 } { E - \frac{\vec{q}^2}{m} } \right)
-i\frac{\pi m}{2q}\delta(q-\sqrt{mE})\right]
    \\
    &=\frac{\Omega_D}{(2\pi)^2}\frac{m}{L^{D-2}}\int^{\Lambda L/4\pi}  \mathrm { d } n \ n^{D-1}\left[\mathcal{P} \left( \frac { 1 } { \left(\frac{pL}{2\pi}\right)^2 - n^2 } \right)
-i\frac{\pi^2}{L n}\delta\left(\frac{2\pi}{L}n -p\right)\right]
    \label{eq:I0}
\end{align}
where $\mathcal{P}$ refers to Principle (Cauchy) Value, we have used the on-shell condition $mE=p^2$, and the geometric factor
\begin{equation}
\Omega_D=\frac{2\pi^{D/2}}{\Gamma(D/2)}=
    \begin{cases}
        4\pi    &   (D=3)\\
        2\pi    &   (D=2)\\
        2       &   (D=1)
    \end{cases}\ ,
\end{equation}

\begin{figure}[h!]
\center
\includegraphics[width=.35\columnwidth]{figure/I0.eps}
\caption{Loop diagram contributing to $I_0$.\label{fig:I0} \todo{Make it a tower of interactions!  Also this figure is way too big cf the font.}}
\end{figure}

In the $s$-wave, the momentum-dependent scattering amplitude is related to the phase shift $\delta_0(p)$ by
\begin{equation}\label{eq:cot delta}
    \amplitude = \frac{4}{m}\F_d\frac{1}{\cot \delta_0(p)-i}\ ,
\end{equation}
where
\begin{equation}
    \F_D
    =
    \begin{cases}
        \pi/p   & (D=3)\\
        1       & (D=2)\\
        p/2     & (D=1)
\end{cases}
\end{equation}
is a dimension-dependent kinematic factor.
This fixes the coefficients $C(\Lambda)$ as a function of the scattering data,
\begin{equation}
    \frac{1}{\sum_n C_{2n}(\Lambda) p^{2n}}
    =
    I_0(p) + \frac{m}{4 \F_D}\left(\cot \delta_0(p) - i\right)
\end{equation}

In a finite volume, the energy eigenstates cause the amplitude to diverge, so that
\begin{equation}
    \frac{1}{\sum_n C_{2n}(\Lambda) p^{2n}} - I_{0,\FV}(p,L) = 0
\end{equation}
and the infinite-volume integral $I_0$ has been replaced by the matching finite-volume sum,
\begin{align}
I_{0,\FV}(p,L)
    &=-i\int \frac { \mathrm {d}q_0}{2\pi} \frac{1}{L^D}\sum_{\vec{q}}^{q < \Lambda/2} \left( \frac { i } { \frac{E}{2} + q _ { 0 } - \frac{\vec{q}^2}{2m} + i \epsilon } \right) \left( \frac { i } { \frac{E}{2} - q _ { 0 } - \frac{\vec{q}^2}{2m} + i \epsilon } \right)
    \\
    &=\frac{1}{L^D}\sum_{\vec{q}}^{q < \Lambda/2} \frac { 1 } { E - \frac{\vec{q}^2}{m} }
    =\frac{m}{(2\pi)^2 L^{D-2}} \sum_{\vec{n}}^{n < \frac{\Lambda L}{4\pi}} \frac{1}{x-n^2}
    &
    x &= \left( \frac{pL}{2\pi}\right)^2
\end{align}
where we have used the on-shell condition $mE=p^2$.

Combining the infinite-volume result with the finite-volume result, one finds
\begin{equation}
    \frac{1}{4\F_D}\left(\cot \delta_0(p) - i\right) = \frac{1}{(2\pi)^2 L^{D-2}}\left[ \left(\sum_n- \int_n\right) \frac{1}{x-n^2} + \frac{-i \pi^2\Omega_D}{L} \int \mathrm{d}n\ n^{D-2} \delta\left(\frac{2\pi}{L}n - p\right) \right]
\end{equation}
where both the sum and integral are cut off by a restriction on the magnitude of $n$, $n^2 < (\Lambda L / 4\pi)^2$, and the integral implicitly carries a factor of $\Omega_D n^{D-1}$.
In a seemingly miraculous (but required) cancellation, the imaginary part on the left hand side exactly cancels the last term in the sum on the right, and we are left with
\begin{equation}
    p \cot \delta_0(p) = \frac{\F_D\ p}{\pi^2 L^{D-2}} \left(\sum_n-\int_n\right) \frac{1}{x-n^2}
\end{equation}
where $x=(pL/2\pi)^2$.
Because we cut off the sum and the integral in exactly the same way, in dimensions where $I_0$ diverges with $\Lambda$, the divergence cancels against the divergence in the sum.
Let $N=\Lambda L/2\pi$.
Then, defining, with a finite cutoff $N$,
\begin{equation}\label{eq:spherical cutoff S}
    S^{\spherical N}_D(x) = \left(\sum_n- \int_n\right) \frac{1}{x-n^2}
\end{equation}
where the $\spherical$ superscript reminds us that we cut off our sum and integral in a spherical way, based on the magnitude of $n<N$, we recover the usual \Luscher zeta functions by taking
\begin{equation}
    S^\spherical_D(x)
    =
    \lim_{N\goesto\infty} S^{\spherical N}_D(x)
    =
    \lim_{N\rightarrow\infty}\left( \sum_n^{n < N/2} \frac{1}{x-n^2} - \counterterm_D^\spherical \left(\frac{N}{2}\right)^{D-2}\right)
\end{equation}
where dimension-dependent counterterm $\counterterm_D^\spherical$ comes from the integral; we evaluate said counterterms in \Appref{counterterm/spherical}.
\todo{I MUST HAVE LOST A SIGN SOMEWHERE, IT SHOULD BE $n^2-x$.}
Finally,
\begin{equation}\label{eq:spherical quantization}
    p \cot \delta_0(p) = \frac{\F_D\ p}{\pi^2 L^{D-2}} S^\spherical_D(x).
\end{equation}
This is the usual \Luscher finite-volume quantization condition.

\todo{STILL REMAINING:}
\begin{itemize}
    \item ERE / matching
    \item Describe tuning
    \item Show results in 2, 3D
    \item Something is wrong :(?  More like :D because we are smart!
\end{itemize}

\section{The Cartesian \Luscher Finite-Volume Counterterm}\label{sec:counterterm/cartesian}

The regulation of the sum in \eqref{cartesian S} is provided by
\begin{equation}
    \label{eq:cartesian S counterterm}
    \int_{-N/2}^{N/2} \mathrm{d}^Dn \frac{1}{n^2-x}
    =
    \left(\frac{N}{2}\right)^{D-2} \int_{-1}^{+1} \mathrm{d}^D\nu \frac{1}{\nu^2 - \tilde{x}}
\end{equation}
where we rescaled and $\tilde{x}=x/(N/2)$, and we can numerically evaluate the right-hand side with ease to achieve a complete improvement of the $S_D^{\cartesian N}$ function.

We can determine the counterterm in the three-dimensional Cartesian volume,evaluating the large-$N$ limit of the integral
\begin{equation}\label{eq:cartesian integral}
    \int_{-N/2}^{+N/2} \mathrm{d}n_x\ \mathrm{d}n_y\ \mathrm{d}n_z \frac{1}{n^2}
\end{equation}
where $n^2 = n_x^2+n_y^2+n_z^2$ and we henceforth suppress the $(x=0)$ argument.
We can evaluate it as follows.  Consider the integral
\begin{equation}
	J^{\cartesian N}_0(x, m) = \int_{-N/2}^{+N/2} dn_x\ dn_y\ dn_z\ \frac{e^{-m n^2}}{n^2}.
\end{equation}
Then, we know $\lim_{m\goesto\infty} J^{\cartesian N}_0(m) = 0$ and $J^{\cartesian N}_0(0)$ is the integral on the right-hand-side of \eqref{cartesian integral}.
We can take advantage of the fundamental theorem of calculus,
\begin{align}
	\int_{-N/2}^{+N/2} \mathrm{d}^3n \frac{1}{n^2}
    &=
    J^{\cartesian N}_0(0) - J^{\cartesian N}_0(\infty)
		&&= 	\int_{\infty}^{0} dm\ \partial_m J^{\cartesian N}_0(m)
		% \\
		&&=	\int_{\infty}^{0} dm\ \int_{-N/2}^{+N/2} \mathrm{d}^3n -e^{-m n^2}
		\nonumber\\
		&&&=	\int_0^\infty dm\ \left(\int_{-N/2}^{+N/2} dn\ e^{-m n^2}\right)^3
		% \\
		&&=	\int_0^\infty dm\ \left(\sqrt{\frac{\pi}{m}} \erf(\sqrt{m N^2/4})\right)^3
\end{align}
where we exchanged the order of partial differentiation by $m$ and momenta integration, recognize the resulting integral as three identical copies of the same integral (as long as the volume is a cube), and evaluate it, yielding the error function.
Changing variables to isolate the dependence on $N$, $n^2 = m N^2/4$, we find
\begin{align}
    \int_{-N/2}^{+N/2} \mathrm{d}^3n \frac{1}{n^2}
    &=
    \pi^{3/2}\ \frac{N}{2} \int_{0}^\infty 2n\ \mathrm{d}n\ \left(\frac{\erf(n)}{n}\right)^3
\end{align}
and the integral can be easily evaluated numerically, yielding $2.75634$ so that
\begin{equation}
    \label{eq:cartesian counterterm}
    \counterterm^\cartesian_3 \left(\frac{N}{2}\right) = \lim_{N\goesto\infty}\int_{-N/2}^{+N/2} \mathrm{d}^3n \frac{1}{n^2} = 15.34824844488746404710\ \left(\frac{N}{2}\right)
\end{equation}
and more digits are readily available; this constant appears in \eqref{cartesian S}.
This can be compared to the spherical counterterm, which we can read off from \eqref{improved spherical S}, where the counter term can be seen to be $4\pi \approx 12.6 $; that the Cartesian result is larger reflects the fact that more of the momentum space is included in the integration domain for a fixed $N$, as discussed in \Secref{cartesian}.

This method of evaluation can be repeated for higher spatial dimensions.
For $D\geq3$ spatial dimensions one finds
\begin{equation}
    \pi^{D/2} \left(\frac{N}{2}\right)^{D-2} \int_0^\infty 2\mu\ \mathrm{d}\mu\ \left(\frac{\erf(\mu)}{\mu}\right)^D
\end{equation}
For four dimensions one finds $17.14741624920737 (N/2)^2$, $24.49922817921121 (N/2)^3$ for five dimensions, for six dimensions $38.50096808074375 (N/2)^4$, and so on.  However, in these higher dimensions we must also subtract subleading divergences, which requires the determination of counterterms we do not here compute.

The logarithmic divergence in two dimensions must be handled especially carefully.
\todo{TOM'S MAGIC INTEGRAL AND CATALAN'S CONSTANT}.

In fact, we can execute a similar construction in three (and higher) dimensions.  In three dimensions, \todo{TOM'S G+POLYLOG MAGIC}.

\subsection{The Effective Range Expansion}\label{sec:ere}

We can use the effective range expansion to express the scattering data as the sum of some known terms that do not vanish at low momentum and a series in $p^2$.  The expansion differs in different dimensions,
\begin{align}
    p\ (\cot \delta_0(p) - i)
    &=
    -\frac{1}{a} - ip + \frac{1}{2} r_0 p^2 \sum_j s_j (r_0^2 p^2)^j
    &
    (D&=3)
    \nonumber\\
    \cot \delta_0(p) - i
    &=
    \frac{2}{\pi} \log(a p) -i + \frac{1}{2} r_0^2 p^2 \sum_j s_j (r_0^2 p^2)^j
    &
    (D&=2)
    \\\nonumber
    \frac{\cot\delta_0(p)-i}{p}
    &=
    -\frac{i}{p} + a + \frac{1}{2} r_0^3 p^2 \sum_j s_j( r_0^2 p^2)^j
    &
    (D&=1)
\end{align}
where $a$ is the scattering length, $r_0$ the length scale called the effective range, and the $s_j$ are dimensionless shape parameters.
We can solve for the phase shift and encapsulate the low-energy pieces of the right-hand side into
\begin{equation}
    \lowEnergyA(p) =
    \begin{cases}
        -\frac{1}{ap}           &   (D=3)\\
        \frac{2}{\pi} \log ap   &   (D=2)\\
        ap                      &   (D=1)
    \end{cases}\ ,
\end{equation}
so that
\begin{equation}
    \cot \delta_0(p) - i = -i + \lowEnergyA(p) + \frac{1}{2} (r_0 p)^{4-D} \sum_j s_j \left(r_0^2p^2\right)^j
\end{equation}

\subsubsection{Dispersion L\"{u}scher removes induced artifacts}

In this section, we again attempt to tune our contact interaction to unitarity by matching the first zero of the \Luscher zeta function.
However, the difference is that at each lattice spacing we tune to that spacing's respective $S^{\dispersion N}_D$, leveraging the dispersion relation for that derivative.
Then, when we extract finite-volume and finite-spacing energy levels, we put them through the dispersion equation \todo{eqref} using the same $S$ function.\footnote{We actually make the replacement $\F_D^\epsilon\goesto\F_D$ to avoid numerically evaluating $F_D^\epsilon$.  With a controlled continuum limit the error from this change obviously vanishes.}
The numerical results of said procedure are shown in \Figref{unimproved dispersion}.
Note that the results for $p\cot\delta$ are now flat across the spectrum, matching the known result for a contact interaction.
Moreover, comparing the scale to that in, for example, \Figref{unimproved spherical}, there the deviations were of order~1, while here the results remain within $10^{-8}$ of zero, with the value entirely reflecting how well the contact interaction was tuned.

\begin{figure}[htb]
    \scalebox{0.9}{\input{figure/ere-contact-fitted_a-inv_+0.0_zeta_dispersion_projector_a1g_n-eigs_200.pgf}}
    \caption{The same as \Figref{unimproved spherical}, but tuned and subsequently analyzed using the appropriate latticized \Luscher function, matching the cutoff on the sum to the lattice scale and accounting for the dispersion relation.}
    \label{fig:unimproved dispersion}
\end{figure}

\begin{figure}[hbt]
    \scalebox{0.9}{\input{figure/ere-contact-fitted_a-inv_-5.0_zeta_dispersion_projector_a1g_n-eigs_200.pgf}}
    \caption{The same as \Figref{unimproved spherical}, but tuned and subsequently analyzed using the appropriate latticized \Luscher function, matching the cutoff on the sum to the lattice scale and accounting for the dispersion relation for finite scattering lenght.}
    \label{fig:unimproved dispersion finite a}
\end{figure}

In \Figref{dispersion running of strength} we show how the strength of the contact interaction runs with the lattice scale.  According to \todo{something we know} it should be \todo{some formula that depends on cutoff}.

\begin{figure}
    \input{figure/contact-scaling-contact-fitted_a-inv_+0.0_zeta_dispersion_projector_a1g_n-eigs_200.pgf}
    \caption{
        Scaling of the contact interaction strength $C(\epsilon)$ fitted using the dispersion method at unitarity.
        Data points are fitted values, solid lines are analytical scaling predictions following $C(\epsilon) = \frac{2}{\mu} \frac{1}{\mathcal L^{\dispersion n_\mathrm{step}}} \epsilon $ and the dashed line corresponds to the spherical predictions.
        Bar diagrams below present the absolute error between prediction and extracted value.
    }
    \label{fig:dispersion running of strength}
\end{figure}

\clearpage

\section{Conclusion}\label{sec:conclusion}

We are smart.

LQCD people should care about cartesian \Luscher.  Does it explain HALQCD against the world?

Calculations of the Bertsch parameter should take advantage of the tuning that can be done with the dispersion function.


\section*{Acknowledgements}\label{sec:acknowledgements}

The authors thank
Tom Cohen,
Ben H\"{o}rz,
Colin Morningstar,
Andr\'{e} Walker-Loud,
Ken McElvain
and
Jan-Lukas Wynen
for stimulating discussion, feedback, and technical help during the course of this work.
C.K. gratefully acknowledges funding through the Alexander von Humboldt Foundation through a Feodor Lynen Research Fellowship.
This work was done in part through financial support from the Deutsche Forschungsgemeinschaft (Sino-German CRC 110).
\todo{Thanks for computing time at \$FACILITY. NERSC?}

\appendix
\section{Misc}

We can put the derivation of the counter terms etc. in appendices.

Do we want to put tables?  Make numerical data directly available?  We should be good, reproducible scientists.

\section{Dispersion Relation Coefficients}\label{sec:coefficients}

In \eqref{gamma definition} and \eqref{gamma determination} we give the definition and how to determine the $\gamma_s^{(\nstep)}$ coefficients that give us finite difference formulas.  Here we collect and list them for convenience.

\begin{table}[ht]
    \caption{Values for $\gamma_s^{(\nstep)}$ for a variety of different $\nstep$s that give the optimal approximation $\omega^{(\nstep)}(p,\epsilon) = (\epsilon p)^2\left[1+ \order{(\epsilon p)^{2 \nstep}}\right]$.}
    \label{tab:dispersion coefficients}
    \begin{tabular}{cccccc}
        $\gamma_s^{(\nstep)}$   &   $s=0$   &   $s=1$   &   $s=2$   &   $s=3$       &   $s=4$   \\
        $\nstep=1$              &   $2$     &   $-2$    &           &               &           \\
        $\nstep=2$              &   $5/2$   &   $-8/3$  &   $1/6$   &               &           \\
        $\nstep=3$              &   $49/18$ &   $-3$    &   $3/10$  &   $-1/45$     &           \\
        $\nstep=4$              &   $205/72$&   $-16/5$ &   $2/5$   &   $-16/315$   &   $1/280$
    \end{tabular}
\end{table}

\section{The Usual, Spherical \Luscher's Formula}\label{sec:spherical}


Here we present a $D$-dimensional derivation of \Luscher's formula that roughly follows \Ref{Beane:2003da}, although the technology and sophistication of the finite-volume formalism has grown substantially \todo{cite cite cite}.  Assuming an interaction given by an tower of derivative contact operators
\begin{equation}
    V(p) = +\sum_n C_{2n}(\Lambda) p^{2n}
\end{equation}
where the interaction strengths depend on the regulator and carry spatial-dimension-dependent units.
The scattering amplitude is given by the bubble sum depicted in \Figref{bubbleSum}.

\begin{figure}[ht!]
\center
\includegraphics[width=.8\columnwidth]{figure/bubbleSum.pdf}
\caption{Bubble sum. Each line represents a propagator, each vertex represents $-i \sum_n C_{2n}(\Lambda) p^{2n}$, and the bubble is given by $I_0$ (see also \Figref{I0}).\label{fig:bubbleSum}}
\end{figure}

This bubble sum is a geometric series and gives\cite{Kaplan:1998we,Beane:2003da}
\begin{equation}\label{eq:scattering amplitude}
\amplitude = \frac{-\sum_n C_{2n}(\Lambda) p^{2n}}{1-I_0(p,\Lambda) \sum_n C_{2n}(\Lambda) p^{2n}},
\end{equation}
where $p$ is the relative momentum,  and $I_0(p,\Lambda)$ is a $D$-dependent function that arises from integrating the loop shown in \Figref{I0},
\begin{align}
    I_0(p)
    &=-i\int^{\Lambda/2}
        \frac { \mathrm {d}q_0}{2\pi}\ \frac{\mathrm { d } ^ { D } \vec{ q } } { (2\pi)^ { D } }
        \left( \frac { i } { \frac{E}{2} + q _ { 0 } - \frac{\vec{q}^2}{2m} + i \epsilon } \right)
        \left( \frac { i } { \frac{E}{2} - q _ { 0 } - \frac{\vec{q}^2}{2m} + i \epsilon } \right)
    \nonumber\\
    &=\frac{\Omega_D}{(2\pi)^D}\int^{\Lambda/2}  \mathrm { d } q \ q^{D-1}\left[\mathcal{P} \left( \frac { 1 } { E - \frac{\vec{q}^2}{m} } \right)
-i\frac{\pi m}{2q}\delta(q-\sqrt{mE})\right]
    \\
    &=\frac{\Omega_D}{(2\pi)^2}\frac{m}{L^{D-2}}\int^{\Lambda L/4\pi}  \mathrm { d } n \ n^{D-1}\left[\mathcal{P} \left( \frac { 1 } { \left(\frac{pL}{2\pi}\right)^2 - n^2 } \right)
-i\frac{\pi^2}{L n}\delta\left(\frac{2\pi}{L}n -p\right)\right]
    \label{eq:I0}
\end{align}
where $\mathcal{P}$ refers to Principle (Cauchy) Value, we have used the on-shell condition $mE=p^2$, and the geometric factor
\begin{equation}
\Omega_D=\frac{2\pi^{D/2}}{\Gamma(D/2)}=
    \begin{cases}
        4\pi    &   (D=3)\\
        2\pi    &   (D=2)\\
        2       &   (D=1)
    \end{cases}\ ,
\end{equation}

\begin{figure}[h!]
\center
\includegraphics[width=.35\columnwidth]{figure/I0.eps}
\caption{Loop diagram contributing to $I_0$.\label{fig:I0} \todo{Make it a tower of interactions!  Also this figure is way too big cf the font.}}
\end{figure}

In the $s$-wave, the momentum-dependent scattering amplitude is related to the phase shift $\delta_0(p)$ by
\begin{equation}\label{eq:cot delta}
    \amplitude = \frac{4}{m}\F_d\frac{1}{\cot \delta_0(p)-i}\ ,
\end{equation}
where
\begin{equation}
    \F_D
    =
    \begin{cases}
        \pi/p   & (D=3)\\
        1       & (D=2)\\
        p/2     & (D=1)
\end{cases}
\end{equation}
is a dimension-dependent kinematic factor.
This fixes the coefficients $C(\Lambda)$ as a function of the scattering data,
\begin{equation}
    \frac{1}{\sum_n C_{2n}(\Lambda) p^{2n}}
    =
    I_0(p) + \frac{m}{4 \F_D}\left(\cot \delta_0(p) - i\right)
\end{equation}

In a finite volume, the energy eigenstates cause the amplitude to diverge, so that
\begin{equation}
    \frac{1}{\sum_n C_{2n}(\Lambda) p^{2n}} - I_{0,\FV}(p,L) = 0
\end{equation}
and the infinite-volume integral $I_0$ has been replaced by the matching finite-volume sum,
\begin{align}
I_{0,\FV}(p,L)
    &=-i\int \frac { \mathrm {d}q_0}{2\pi} \frac{1}{L^D}\sum_{\vec{q}}^{q < \Lambda/2} \left( \frac { i } { \frac{E}{2} + q _ { 0 } - \frac{\vec{q}^2}{2m} + i \epsilon } \right) \left( \frac { i } { \frac{E}{2} - q _ { 0 } - \frac{\vec{q}^2}{2m} + i \epsilon } \right)
    \\
    &=\frac{1}{L^D}\sum_{\vec{q}}^{q < \Lambda/2} \frac { 1 } { E - \frac{\vec{q}^2}{m} }
    =\frac{m}{(2\pi)^2 L^{D-2}} \sum_{\vec{n}}^{n < \frac{\Lambda L}{4\pi}} \frac{1}{x-n^2}
    &
    x &= \left( \frac{pL}{2\pi}\right)^2
\end{align}
where we have used the on-shell condition $mE=p^2$.

Combining the infinite-volume result with the finite-volume result, one finds
\begin{equation}
    \frac{1}{4\F_D}\left(\cot \delta_0(p) - i\right) = \frac{1}{(2\pi)^2 L^{D-2}}\left[ \left(\sum_n- \int_n\right) \frac{1}{x-n^2} + \frac{-i \pi^2\Omega_D}{L} \int \mathrm{d}n\ n^{D-2} \delta\left(\frac{2\pi}{L}n - p\right) \right]
\end{equation}
where both the sum and integral are cut off by a restriction on the magnitude of $n$, $n^2 < (\Lambda L / 4\pi)^2$, and the integral implicitly carries a factor of $\Omega_D n^{D-1}$.
In a seemingly miraculous (but required) cancellation, the imaginary part on the left hand side exactly cancels the last term in the sum on the right, and we are left with
\begin{equation}
    p \cot \delta_0(p) = \frac{\F_D\ p}{\pi^2 L^{D-2}} \left(\sum_n-\int_n\right) \frac{1}{x-n^2}
\end{equation}
where $x=(pL/2\pi)^2$.
Because we cut off the sum and the integral in exactly the same way, in dimensions where $I_0$ diverges with $\Lambda$, the divergence cancels against the divergence in the sum.
Let $N=\Lambda L/2\pi$.
Then, defining, with a finite cutoff $N$,
\begin{equation}\label{eq:spherical cutoff S}
    S^{\spherical N}_D(x) = \left(\sum_n- \int_n\right) \frac{1}{x-n^2}
\end{equation}
where the $\spherical$ superscript reminds us that we cut off our sum and integral in a spherical way, based on the magnitude of $n<N$, we recover the usual \Luscher zeta functions by taking
\begin{equation}
    S^\spherical_D(x)
    =
    \lim_{N\goesto\infty} S^{\spherical N}_D(x)
    =
    \lim_{N\rightarrow\infty}\left( \sum_n^{n < N/2} \frac{1}{x-n^2} - \counterterm_D^\spherical \left(\frac{N}{2}\right)^{D-2}\right)
\end{equation}
where dimension-dependent counterterm $\counterterm_D^\spherical$ comes from the integral; we evaluate said counterterms in \Appref{counterterm/spherical}.
\todo{I MUST HAVE LOST A SIGN SOMEWHERE, IT SHOULD BE $n^2-x$.}
Finally,
\begin{equation}\label{eq:spherical quantization}
    p \cot \delta_0(p) = \frac{\F_D\ p}{\pi^2 L^{D-2}} S^\spherical_D(x).
\end{equation}
This is the usual \Luscher finite-volume quantization condition.

\todo{STILL REMAINING:}
\begin{itemize}
    \item ERE / matching
    \item Describe tuning
    \item Show results in 2, 3D
    \item Something is wrong :(?  More like :D because we are smart!
\end{itemize}

\section{The Cartesian \Luscher Finite-Volume Counterterm}\label{sec:counterterm/cartesian}

The regulation of the sum in \eqref{cartesian S} is provided by
\begin{equation}
    \label{eq:cartesian S counterterm}
    \int_{-N/2}^{N/2} \mathrm{d}^Dn \frac{1}{n^2-x}
    =
    \left(\frac{N}{2}\right)^{D-2} \int_{-1}^{+1} \mathrm{d}^D\nu \frac{1}{\nu^2 - \tilde{x}}
\end{equation}
where we rescaled and $\tilde{x}=x/(N/2)$, and we can numerically evaluate the right-hand side with ease to achieve a complete improvement of the $S_D^{\cartesian N}$ function.

We can determine the counterterm in the three-dimensional Cartesian volume,evaluating the large-$N$ limit of the integral
\begin{equation}\label{eq:cartesian integral}
    \int_{-N/2}^{+N/2} \mathrm{d}n_x\ \mathrm{d}n_y\ \mathrm{d}n_z \frac{1}{n^2}
\end{equation}
where $n^2 = n_x^2+n_y^2+n_z^2$ and we henceforth suppress the $(x=0)$ argument.
We can evaluate it as follows.  Consider the integral
\begin{equation}
	J^{\cartesian N}_0(x, m) = \int_{-N/2}^{+N/2} dn_x\ dn_y\ dn_z\ \frac{e^{-m n^2}}{n^2}.
\end{equation}
Then, we know $\lim_{m\goesto\infty} J^{\cartesian N}_0(m) = 0$ and $J^{\cartesian N}_0(0)$ is the integral on the right-hand-side of \eqref{cartesian integral}.
We can take advantage of the fundamental theorem of calculus,
\begin{align}
	\int_{-N/2}^{+N/2} \mathrm{d}^3n \frac{1}{n^2}
    &=
    J^{\cartesian N}_0(0) - J^{\cartesian N}_0(\infty)
		&&= 	\int_{\infty}^{0} dm\ \partial_m J^{\cartesian N}_0(m)
		% \\
		&&=	\int_{\infty}^{0} dm\ \int_{-N/2}^{+N/2} \mathrm{d}^3n -e^{-m n^2}
		\nonumber\\
		&&&=	\int_0^\infty dm\ \left(\int_{-N/2}^{+N/2} dn\ e^{-m n^2}\right)^3
		% \\
		&&=	\int_0^\infty dm\ \left(\sqrt{\frac{\pi}{m}} \erf(\sqrt{m N^2/4})\right)^3
\end{align}
where we exchanged the order of partial differentiation by $m$ and momenta integration, recognize the resulting integral as three identical copies of the same integral (as long as the volume is a cube), and evaluate it, yielding the error function.
Changing variables to isolate the dependence on $N$, $n^2 = m N^2/4$, we find
\begin{align}
    \int_{-N/2}^{+N/2} \mathrm{d}^3n \frac{1}{n^2}
    &=
    \pi^{3/2}\ \frac{N}{2} \int_{0}^\infty 2n\ \mathrm{d}n\ \left(\frac{\erf(n)}{n}\right)^3
\end{align}
and the integral can be easily evaluated numerically, yielding $2.75634$ so that
\begin{equation}
    \label{eq:cartesian counterterm}
    \counterterm^\cartesian_3 \left(\frac{N}{2}\right) = \lim_{N\goesto\infty}\int_{-N/2}^{+N/2} \mathrm{d}^3n \frac{1}{n^2} = 15.34824844488746404710\ \left(\frac{N}{2}\right)
\end{equation}
and more digits are readily available; this constant appears in \eqref{cartesian S}.
This can be compared to the spherical counterterm, which we can read off from \eqref{improved spherical S}, where the counter term can be seen to be $4\pi \approx 12.6 $; that the Cartesian result is larger reflects the fact that more of the momentum space is included in the integration domain for a fixed $N$, as discussed in \Secref{cartesian}.

This method of evaluation can be repeated for higher spatial dimensions.
For $D\geq3$ spatial dimensions one finds
\begin{equation}
    \pi^{D/2} \left(\frac{N}{2}\right)^{D-2} \int_0^\infty 2\mu\ \mathrm{d}\mu\ \left(\frac{\erf(\mu)}{\mu}\right)^D
\end{equation}
For four dimensions one finds $17.14741624920737 (N/2)^2$, $24.49922817921121 (N/2)^3$ for five dimensions, for six dimensions $38.50096808074375 (N/2)^4$, and so on.  However, in these higher dimensions we must also subtract subleading divergences, which requires the determination of counterterms we do not here compute.

The logarithmic divergence in two dimensions must be handled especially carefully.
\todo{TOM'S MAGIC INTEGRAL AND CATALAN'S CONSTANT}.

In fact, we can execute a similar construction in three (and higher) dimensions.  In three dimensions, \todo{TOM'S G+POLYLOG MAGIC}.

\subsubsection{Dispersion L\"{u}scher removes induced artifacts}

In this section, we again attempt to tune our contact interaction to unitarity by matching the first zero of the \Luscher zeta function.
However, the difference is that at each lattice spacing we tune to that spacing's respective $S^{\dispersion N}_D$, leveraging the dispersion relation for that derivative.
Then, when we extract finite-volume and finite-spacing energy levels, we put them through the dispersion equation \todo{eqref} using the same $S$ function.\footnote{We actually make the replacement $\F_D^\epsilon\goesto\F_D$ to avoid numerically evaluating $F_D^\epsilon$.  With a controlled continuum limit the error from this change obviously vanishes.}
The numerical results of said procedure are shown in \Figref{unimproved dispersion}.
Note that the results for $p\cot\delta$ are now flat across the spectrum, matching the known result for a contact interaction.
Moreover, comparing the scale to that in, for example, \Figref{unimproved spherical}, there the deviations were of order~1, while here the results remain within $10^{-8}$ of zero, with the value entirely reflecting how well the contact interaction was tuned.

\begin{figure}[htb]
    \scalebox{0.9}{\input{figure/ere-contact-fitted_a-inv_+0.0_zeta_dispersion_projector_a1g_n-eigs_200.pgf}}
    \caption{The same as \Figref{unimproved spherical}, but tuned and subsequently analyzed using the appropriate latticized \Luscher function, matching the cutoff on the sum to the lattice scale and accounting for the dispersion relation.}
    \label{fig:unimproved dispersion}
\end{figure}

\begin{figure}[hbt]
    \scalebox{0.9}{\input{figure/ere-contact-fitted_a-inv_-5.0_zeta_dispersion_projector_a1g_n-eigs_200.pgf}}
    \caption{The same as \Figref{unimproved spherical}, but tuned and subsequently analyzed using the appropriate latticized \Luscher function, matching the cutoff on the sum to the lattice scale and accounting for the dispersion relation for finite scattering lenght.}
    \label{fig:unimproved dispersion finite a}
\end{figure}

In \Figref{dispersion running of strength} we show how the strength of the contact interaction runs with the lattice scale.  According to \todo{something we know} it should be \todo{some formula that depends on cutoff}.

\begin{figure}
    \input{figure/contact-scaling-contact-fitted_a-inv_+0.0_zeta_dispersion_projector_a1g_n-eigs_200.pgf}
    \caption{
        Scaling of the contact interaction strength $C(\epsilon)$ fitted using the dispersion method at unitarity.
        Data points are fitted values, solid lines are analytical scaling predictions following $C(\epsilon) = \frac{2}{\mu} \frac{1}{\mathcal L^{\dispersion n_\mathrm{step}}} \epsilon $ and the dashed line corresponds to the spherical predictions.
        Bar diagrams below present the absolute error between prediction and extracted value.
    }
    \label{fig:dispersion running of strength}
\end{figure}

\clearpage

\section{Comparison in Two Dimensions}\label{sec:two-d}

Here we do the whole thing for 2D.


\bibliography{master}

\end{document}
