\section{The Cartesian \Luscher Finite-Volume Counterterm}\label{sec:counterterm/cartesian}

The regulation of the sum in \eqref{cartesian S} is provided by
\begin{equation}
    \label{eq:cartesian integral}
    \int_{-N/2}^{N/2} \mathrm{d}^Dn\; \PV\; \frac{1}{n^2-x}
    =
    \left(\frac{N}{2}\right)^{D-2} \int_{-1}^{+1} \mathrm{d}^D\nu\; \PV\; \frac{1}{\nu^2 - \xtilde}
\end{equation}
where we rescaled and $\xtilde=x/(N/2)^2$, and we can numerically evaluate the right-hand side with ease to achieve a complete improvement of the $S_D^{\cartesian N}$ function.

We can determine the counterterm in the three-dimensional Cartesian volume, evaluating the large-$N$ limit of the integral on the right-hand side of \eqref{cartesian integral} as follows.  Consider the integral
\begin{equation}
	J^{\cartesian N}_0(\xtilde, m) = \int_{-1}^{+1} \mathrm{d}^D\nu\; \frac{e^{-m (\nu^2-\xtilde)}}{\nu^2-\xtilde}.
\end{equation}
Then, we know $\lim_{m\goesto\infty} J^{\cartesian N}_0(\xtilde\leq0, m) = 0$ and $J^{\cartesian N}_0(\xtilde, 0)$ is the integral in \eqref{cartesian integral} we wish to evaluate.
We can take advantage of the fundamental theorem of calculus,
\begin{align}
	\int_{-1}^{+1} \mathrm{d}^D\nu\; \frac{1}{\nu^2-\xtilde}
    &=
    J^{\cartesian N}_0(\xtilde,0) - J^{\cartesian N}_0(\xtilde,\infty)
		&&= 	\int_{\infty}^{0} \mathrm{d}m\; \partial_m J^{\cartesian N}_0(\xtilde,m)
		% \\
		&&=	\int_{\infty}^{0} \mathrm{d}m\ \int_{-1}^{+1} \mathrm{d}^D\nu\; -e^{-m (\nu^2-\xtilde)}
		\nonumber\\
		&&&=	\int_0^\infty \mathrm{d}m\; e^{m \xtilde^2}\left(\int_{-1}^{+1} \mathrm{d}\nu\; e^{-m \nu^2}\right)^D
		% \\
		&&=	\int_0^\infty \mathrm{d}m\; e^{m\xtilde}\left(\sqrt{\frac{\pi}{m}} \erf(\sqrt{m})\right)^D
        \nonumber\\
    &&&=
    \pi^{D/2} \int_{0}^{\infty} 2\mu\; \mathrm{d}\mu\; e^{\xtilde \mu^2} \left(\frac{\erf(\mu)}{\mu}\right)^D
    &&(\xtilde<0)
\end{align}
where we exchange the order of partial differentiation by $m$ and momenta integration, recognize the resulting integral as three identical copies of the same integral (as long as the volume is a cube), evaluate it, yielding the error function, and finally change variables $m\goesto\mu^2$, easing the numerical integration.
When $\xtilde\leq0$ our procedure is legitimate, and for $\xtilde=0$ in three dimensions one finds $2.75634$ so that
\begin{equation}
    \label{eq:cartesian counterterm}
    \counterterm^\cartesian_3 \left(\frac{N}{2}\right) = \lim_{N\goesto\infty}\int_{-N/2}^{+N/2} \mathrm{d}^3n \frac{1}{n^2} = 15.34824844488746404710\ \left(\frac{N}{2}\right)
\end{equation}
and more digits are readily available; this constant appears in \eqref{cartesian S}.
This can be compared to the spherical counterterm, which we can read off from \eqref{improved spherical S}, where the counter term can be seen to be $4\pi \approx 12.6 $; that the Cartesian result is larger reflects the fact that more of the momentum space is included in the integration domain for a fixed $N$, as discussed in \Secref{cartesian}.

For $D\geq3$ spatial dimensions one can immediately use the same method to find the leading divergences.
For four dimensions one finds $17.14741624920737 (N/2)^2$, $24.49922817921121 (N/2)^3$ for five dimensions, for six dimensions $38.50096808074375 (N/2)^4$, and so on.  However, in these higher dimensions we must also subtract subleading divergences, which requires the determination of additional counterterms we do not here compute.

When $\xtilde > D$ essentially the same game may be played to evaluate the integral, though one must then consider the opposite sign $m$.  Ultimately one finds, letting $\erfi$ be the imaginary error function,
\begin{align}
    \int_{-1}^{+1} \mathrm{d}^D\nu\; \PV \frac{1}{\nu^2-\xtilde} &= -\pi^{D/2} \int_0^{\infty} 2\mu\; \mathrm{d}\mu\; e^{-\mu^2 x} \left(\frac{\erfi \mu}{\mu}\right)^D
    &
    (\xtilde&>D)
\end{align}
though when $N\goesto\infty$ $\xtilde$ vanishes for fixed $x$ and the value here is not physically valuable.  \todo{And at intermediate $\xtilde$, evaluate along some other curve?}

The logarithmic divergence in two dimensions must be handled especially carefully.
\todo{TOM'S MAGIC INTEGRAL AND CATALAN'S CONSTANT}.

In fact, we can execute a similar construction in three (and higher) dimensions.  In three dimensions, \todo{TOM'S G+POLYLOG MAGIC}.
