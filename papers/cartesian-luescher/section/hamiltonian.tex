\section{Discretized Hamiltonian}\label{sec:hamiltonian}

We study the simplest interacting two-body problem, two particles interacting via a contact interaction,
\begin{equation}
    H = \frac{p_1^2}{2 m_1} + \frac{p_2^2}{2 m_2} + C \delta(x_1-x_2)
\end{equation}
where the subscripts indicate the position of the position and momenta.  Moving to center-of-mass and relative coordinates, this becomes
\begin{equation}
    H = \frac{P^2}{2 M} + \frac{p^2}{2 m} + C \delta(x)
\end{equation}
where capital letters represent center-of-mass variables, lower case implies relative coordinates, and $m$ is the reduced mass.  We henceforth specialize to the rest frame, setting $P=0$, which reduces the problem to an effective one-body problem.

We consider a finite cubic volume of linear size $L$ with periodic boundary conditions and lattice spacing $\epsilon$ so that $L/\epsilon$ is an integer that counts the number of sites in one spatial direction.

On the lattice we implement the contact interaction as entirely local, vanishing everywhere except at the origin where it is of strength $C$.  We study a variety of kinetic operators, in contrast.  We denote the symmetric nearest-neighbor finite difference $\nstep=1$.  We consider additional stencils which extend on-axis additional steps so that the finite difference is a $(1+2\nstep D)$-point stencil in $D$ dimensions.  In addition, we use a nonlocal operator, denoted $\nstep=\infty$ which, in momentum space amounts to multiplying by $p^2$, the Laplacian implementation of the ungauged SLAC derivative.

Once constructed, we diagonalize this Hamiltonian exactly to extract eigenenergies.
