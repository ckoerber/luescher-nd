\section{Discretized Hamiltonian}\label{sec:hamiltonian}

We study the simplest interacting two-body problem, two particles interacting via a contact interaction,
\begin{equation}
    H = \frac{p_1^2}{2 m_1} + \frac{p_2^2}{2 m_2} + C \delta(x_1-x_2)
\end{equation}
where the subscripts indicate the position of the position and momenta.  Moving to center-of-mass and relative coordinates, this becomes
\begin{equation}
    \label{eq:hamiltonian}
    H = \frac{P^2}{2 M} + \frac{p^2}{2 \mu} + C \delta(x)
\end{equation}
where capital letters represent center-of-mass variables, lower case implies relative coordinates, and $\mu$ is the reduced mass.  We henceforth specialize to the rest frame, setting $P=0$, which reduces the problem to an effective one-body problem.

We consider a finite cubic volume of linear size $L$ with periodic boundary conditions and lattice spacing $\epsilon$ so that $N=L/\epsilon$ is an even integer that counts the number of sites in one spatial direction.

On the lattice we implement the contact interaction as entirely local, vanishing everywhere except at the origin where it is of strength $C$.  We study a variety of kinetic operators, in contrast.  We denote the symmetric nearest-neighbor finite difference $\nstep=1$.  We consider additional stencils which extend on-axis additional steps so that the finite difference is a $(1+2\nstep D)$-point stencil in $D$ dimensions,
\begin{equation}
    \left\langle \vec{r}' \middle| H \middle| \vec{r} \right\rangle
    \rightarrow
    H_{\vec{r}',\vec{r}}^{(L,\epsilon,\nstep)}
    = - \frac{1}{2 \mu \epsilon^2} \sum_{d=1}^{D} \sum_{s=-\nstep}^{+\nstep} c^{(\nstep)}_{|s|} \delta_{\vec{r}',\vec{r}+\epsilon s \vec{e}_d} + C(\epsilon) \delta_{\vec{r}',\vec{r}}\delta_{\vec{r},\vec{0}}
\end{equation}
where the indices are understood to be modded by the periodic boundary conditions of the lattice.
In momentum space, this Hamiltonian may be written
\begin{align}
    \label{eq:p space hamiltonian}
    \left\langle \vec{p}' \middle| H \middle| \vec{p} \right\rangle
    \rightarrow
    H_{\vec{p}',\vec{p}}^{(L,\epsilon,\nstep)}
    &= \delta_{\vec{p}',\vec{p}} \frac{1}{2\mu} \sum_{d=1}^{D} \omega^{(\nstep)}(p_d,\epsilon) + C(\epsilon)
    \\
    \label{eq:gamma definition}
    \omega^{(\nstep)}(p_d,\epsilon)
    &= \frac{1}{\epsilon^2} \sum_{s=0}^{\nstep} \gamma_{s}^{(\nstep)} \cos(s p_d \epsilon)
\end{align}
where $\vec{p} = 2\pi \vec{n}/L$ for a $D$-plet of integers $\vec{n} \in (-N/2, +N/2]^D$, and the coefficients $\gamma_{s}^{(\nstep)}$ are determined by requiring the dispersion relation be as quadratic as possible,
\begin{equation}
    \label{eq:gamma determination}
    \omega^{(\nstep)}(p_d,\epsilon) \overset{!}{=} p_d^2 \left[ 1 + \order{(\epsilon p_d)^{2\nstep}}\right].
\end{equation}
In \Appref{coefficients} we collect the required $\gamma$ coefficients for a variety of $\nstep$s, and we show the resulting dispersion relations in \Figref{dispersion relation}.
In addition, we use a nonlocal operator, denoted by $\nstep=\infty$ which, in momentum space can be implemented to multiplying by $p^2$ directly,
\begin{equation}
    \omega^{\infty}(p_d,\epsilon) = p_d^2,
\end{equation}
including at the edge of the Brillouin zone, the Laplacian implementation of the ungauged SLAC derivative.
Including the edge does not introduce a discontinuity at the boundary (though it does introduce a cusp).

Once constructed, we diagonalize this Hamiltonian exactly to extract eigenenergies.
Throughout we focus on a three-dimensional system, though in \Appref{two-d} we study a two-dimensional system, where logarithmic divergences warrant special attention.

\begin{figure}
    \includegraphics[width=0.5\textwidth]{figure/dispersion.pdf}
    \caption{We show the continuum dispersion relation of energy as a function of momentum for different one-dimensional $\nstep$ derivatives.  For a finite number of lattice points $N$, the allowed momenta are evenly-spaced in steps of $2\pi/N$.
    As additional steps are incorporated into the finite difference, the dispersion relation more and more faithfully reproduces the desired $p^2$~behavior of $\nstep=\infty$.
    }
    \label{fig:dispersion relation}
\end{figure}
