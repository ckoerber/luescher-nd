\section{Matching, and the Effective Range Expansion}\label{sec:ere}

We can use the effective range expansion to express the scattering data as the sum of some known terms that do not vanish at low momentum and a series in $p^2$.  The expansion differs in different dimensions,
\begin{align}
    p\ (\cot \delta_0(p) - i)
    &=
    -\frac{1}{a} - ip + \frac{1}{2} r_0 p^2 \sum_j s_j (r_0^2 p^2)^j
    &
    (D&=3)
    \nonumber\\
    \cot \delta_0(p) - i
    &=
    \frac{2}{\pi} \log(a p) -i + \frac{1}{2} r_0^2 p^2 \sum_j s_j (r_0^2 p^2)^j
    &
    (D&=2)
    \\\nonumber
    \frac{\cot\delta_0(p)-i}{p}
    &=
    -\frac{i}{p} + a + \frac{1}{2} r_0^3 p^2 \sum_j s_j( r_0^2 p^2)^j
    &
    (D&=1)
\end{align}
where $a$ is the scattering length, $r_0$ the length scale called the effective range, and the $s_j$ are dimensionless shape parameters.
We can solve for the phase shift and encapsulate the low-energy pieces of the right-hand side into
\begin{equation}
    \lowEnergyA(p) =
    \begin{cases}
        -\frac{1}{ap}           &   (D=3)\\
        \frac{2}{\pi} \log ap   &   (D=2)\\
        ap                      &   (D=1)
    \end{cases}\ ,
\end{equation}
so that
\begin{equation}\label{eq:ERE}
    \cot \delta_0(p) - i = -i + \lowEnergyA(p) + \frac{1}{2} (r_0 p)^{4-D} \sum_j s_j \left(r_0^2p^2\right)^j.
\end{equation}
We can take this expression and plug it into \eqref{IV pole}, matching the dependence on $p$.
Using the spherical $I_0(p)$, one finds
\begin{align}
    \nonumber
    D&=3:
    &
    0&=\frac{4 \pi}{m C_0}-\frac{1}{a} + \frac{2N}{L}
    &
    0&=- \frac{4 \pi C_2}{m C_0^2} + \frac{1}{2}r_0 s_0 - \frac{2 L}{N \pi^2}
    &
    0&=\frac{4\pi(C_2^2-C_0 C_4)}{m C_0^3} + \frac{1}{2} r_0^3 s_1 - \frac{2L^3}{3\pi^4 N^3}
    &\ldots&
    \\
    D&=2:
    &
    0&=\frac{4}{m C_0} + \frac{2}{\pi} \log \frac{a N \pi}{L}
    &
    0&=- \frac{4 C_2}{m C_0^2} + \frac{1}{2}r_0^2 s_0  - \frac{L^2}{N^2 \pi^3}
    &
    0&=\frac{4(C_2^2-C_0 C_4)}{m C_0^3} + \frac{1}{2} r_0^4 s_1 - \frac{2 L^4}{4\pi^5 N^4}
    &\ldots&
    \\
    D&=1:
    &
    0&=\frac{2}{m C_0}+a-\frac{2L}{\pi^2N}
    &
    0&=- \frac{2 C_2}{m C_0^2} + \frac{1}{2}r_0^3 s_0 - \frac{2 L^3}{3N^3 \pi^4}
    &
    0&=\frac{2(C_2^2-C_0 C_4)}{m C_0^3} + \frac{1}{2} r_0^5 s_1 - \frac{2 L^5}{5\pi^6 N^5 }
    &\ldots&
    \nonumber
\end{align}
where $N/2$ is the cutoff on the integral, the first term in each relation comes from expanding $1/\sum_n C_{2n} p^{2n}$, the second from the effective range expansion \eqref{ERE}, and the third from expanding $I_0$ as a function of $p$, holding a fixed cutoff.  These equations can be solved order-by-order for the coefficients in the potential, or, if we know the coefficients, we can read these equations as determining the scattering data ($r_0$ and $s_j$) and how to correct for a finite cutoff.
\todo{Using $L=N\epsilon$ and $\epsilon=\pi/\Lambda$ I have checked that setting $C_{>0}=0$ reproduces 3-2-1-blastoff's (16), (17), (18) for D=3, (25) for D=2, and, when the cutoff is infinite, (28) for D=1.}

Of special note is the matching condition for $C_0$ when $D=2$, where the momentum dependence at small $p$ is logarithmic, and we cannot take an independent continuum and finite-volume limit.
\todo{What are we supposed again?  I used to understand this.  There's some dimensional transmutation magic going on here.}

We can also use this understanding to construct an improved \Luscher's formula.
Note that when passing from the cut off \eqref{spherical cutoff S} to \eqref{spherical S} we replaced the integral with its limiting value, the divergence-cancelling counterterm.
Rather than jump straight to the leading dependence on the cutoff, we can subtract subleading terms as well.
One finds, truncating our improvement scheme at order $J$,
\begin{align}
    S_3^\spherical(x)  = \lim_{N\goesto\infty} S^{\spherical N}_3(x)
    &=
    \lim_{N\goesto\infty}\left[\sum_{\vec{n}}^{n<N/2} \frac{1}{x-n^2}
        + 4\pi \sum_{j=0}^{J} \frac{x^j}{2j-1} \left(\frac{N}{2}\right)^{-(2j-1)}
    \right]
    &
    (D&=3)
    \nonumber\\
    S_2^\spherical(x)  = \lim_{N\goesto\infty} S^{\spherical N}_2(x)
    &=
    \lim_{N\goesto\infty}\left[\sum_{\vec{n}}^{n<N/2} \frac{1}{x-n^2}
        + \pi \log \frac{x}{(N/2)^2}
        + \pi \sum_{j=1}^{J} \frac{x^j}{j} \left(\frac{N}{2}\right)^{-2j}
    \right]
    &
    (D&=2)
    \\\nonumber
    S_1^\spherical(x)  = \lim_{N\goesto\infty} S^{\spherical N}_1(x)
    &=
    \lim_{N\goesto\infty}\left[\sum_{\vec{n}}^{n<N/2} \frac{1}{x-n^2}
        + 2 \sum_{j=0}^{J} \frac{x^j}{2j+1}\left(\frac{N}{2}\right)^{-(2j+1)}
        \right]
    &
    (D&=1)
\end{align}
Note that the subleading terms are all powers of $x/(N/2)^2$.
Since $x=(pL/2\pi)$, this implies that if one calculates the $S$ function with a finite (albeit very large) $N$, one will see a momentum-dependent deviation from the true $\cot\delta$.
This shows up, for example, as a truncation-induced effective range and higher-order shape parameters.

\begin{center}
    \boxed{\textrm{{\color{red}REALLY TOUGH CHALLENGE: TELL THE ABOVE STORY WITH A CARTESIAN CUTOFF?}}}
\end{center}

\todo{I'm no longer convinced that (assuming you take the cutoff to infinity) spherical and cartesian should be different.  I think they should come out the same.}
