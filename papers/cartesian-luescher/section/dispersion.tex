\section{The Dispersion Function}\label{sec:dispersion}

In the usual procedure leveraging \Luscher's finite volume formalism, one must take the fixed-volume continuum-limit spectrum and feed it through the $N\goesto\infty$ limit of the $S$ function.
In practice, however, most lattice QCD calculations cannot take a continuum limit in a fixed volume---many calculations indeed do not attempt a continuum limit because it is costly and the discretization artifacts are expected to be overwhelmed by the statistical uncertainty.
The hope is to avoid the requirement of first taking a continuum limit of the spectrum.

One physical motivation is to consider formulating scattering on an infinite-volume grid.
Scattering is perfectly well-defined there, and indeed has many physical examples in the form of quasiparticle scattering in crystals.
The central idea is that if we exactly match the limit on the sums and integrals with the volume in which we calculate, we can extrapolate to the continuum limit (if physically required) \emph{after} converting the spectrum to the infinite-volume scattering data.

What role does the $N\goesto\infty$ limit play in formulating $S$?
It removes a momentum cutoff from the formulation of the zeta functions.
However, a momentum cutoff is exactly what we have when we have a finite-spacing lattice.
Consider a box with linear size $L$ and $N$ lattice sites in each direction so that the lattice spacing is $\epsilon=L/N$.
Then, one may intuitively reach for the zeta function $S_D^{\cartesian N}(x)$.  That is,
\begin{equation}\label{eq:cutoff cartesian S}
    \cot \delta_0^N(p) = \frac{\F_D}{\pi^2 L^{D-2}} S_D^{\cartesian N}(x)
\end{equation}
where $\delta_0^N$ indicates that the phase shift is calculated on a lattice with $N$ sites.

Note that on an infinite square lattice, the continuum symmetry group SO(3) is broken by the discretization.
So, we should classify the infinite-volume scattering states not according to irreps of SO(3) but of the remaining subgroup.
The lattice spacing and dispersion relation can introduce anisotropies in propagation speed, so that nice spherical waves distort.
While working out the associated Clebsch-Gordan subduction coefficients is a worthwhile exercise, we here focus on a contact interaction, which is an $SO(3)$ singlet and an $A_1^+$ interaction at finite spacing, and in the continuum limit the full continuum rotational SO(3) should be restored.
We therefore will proceed to talk as though we have direct access to the $s$-wave, even at finite spacing.

In \Figref{results cutoff cartesian S} we show the results of tuning to reproduce the first zero of $S_D^{\cartesian N}$ with a variety of {\nstep}s.
Note that the quality of the results is \nstep-dependent, with $\nstep=\infty$ producing exact agreement.

\begin{figure}
    \includegraphics{example-image-a}
    \caption{Here we show the finite-volume finite-spacing spectrum converted to scattering data through \eqref{cutoff cartesian S} for different {\nstep}s.
    \todo{It should look good, but not great, except for $\nstep=\infty$.}}
    \label{fig:results cutoff cartesian S}
\end{figure}

The difference comes from the fact that when $\nstep=\infty$ we have the exact dispersion relation that corresponds to the on-shell condition that we used to eliminate the energy in favor of $x=(pL/2\pi)$.
Therefore, we return to the derivation of \Luscher's finite-volume formalism and, recognizing that we're interested in incorporating these lattice artifacts from the start, replace the continuum dispersion relation with the lattice dispersion relation in the propagators.
That is,
\begin{align}
    I_0^{\dispersion}(E)
    % &=-i\int_{-\Lambda}^{+\Lambda}
    %     \frac{ \mathrm {d}q_0}{2\pi}\
    %     \frac{\mathrm{d}^{D} \vec{q} } { (2\pi)^ { D } }
    %     \left( \frac { i } { \frac{E}{2} + q_{ 0 } - \frac{\sum_d \omega^{(\nstep)}(q_d,\epsilon)}{2m_1} + i \epsilon } \right)
    %     \left( \frac { i } { \frac{E}{2} + q_{ 0 } - \frac{\sum_d \omega^{(\nstep)}(q_d,\epsilon)}{2m_2} + i \epsilon } \right)
    % \nonumber\\
    &=\frac{1}{(2\pi)^D}
    \int_{-\Lambda}^{+\Lambda}
        \mathrm{d}^D \vec{q}
        \left[
            \mathcal{P} \left( \frac { 1 } { E - \frac{\sum_d \omega^{(\nstep)}(q_d,\epsilon)}{2\mu} } \right)
            -i \todo{RESIDUE}
        \right]
    \\
    &=\frac{2\mu}{L^{D-2}}
    \int_{-N/2}^{+N/2}
        \mathrm{d}^D\vec{n}
        \left[
            \mathcal{P} \left( \frac { 1 } { 2\mu E L^2 - N^2 \sum_{ds} \gamma_s^{(\nstep)} \cos \frac{2\pi s n_d}{N}} \right)
            -i \todo{RESIDUE}
        \right]
    \label{eq:dispersion I0}
\end{align}
where we adopte a $\dispersion$ superscript to indicate the quantity is defined accounting for the lattice dispersion relation.
Following the same procedure as before, we need to solve
\begin{equation}
    \frac{\mu}{2\F_D^N}(\cot \delta_0^N(E) - i) = I_0^{\dispersion N}(E) - I_{0,\FV}^{\dispersion N}(p)
\end{equation}
\todo{Not sure the $-i$ is exactly right; $\F_D^N$ will be determined by the kinematics of the pole.  I \emph{hope} it doesn't actually know about $N$ but only $\epsilon$, because otherwise it's hard to understand how $\F$ is an infinite-volume quantity.  Same with $\delta_0^N$.}

Here already you can see we will be ultimately be able to define a function of $2\mu E L^2$ where $N$ but no other parameter appears anywhere.
\todo{Usually we go on-shell now, trading $E$ for $p$, but I think that's a mistake here.}
\todo{Does it make sense to define $S$-wave on an infinite lattice?  I think this is related to going the "other" way---we actually are getting $A_1^+$ data, but since $A_1^+$ is magically simple we can understand it as $S$-wave.}

\todo{Arrive at the dispersion Luscher equation.}

\begin{itemize}
    \item Show errors of putting not-continuum spectrum through continuum S.  Dispersion S even accounts for lattice spacing!
    \item If we tune to dispersion S rather than continuum S at each lattice spacing the continuum limit is much nicer, especially for low modes.
    \item 2D, 3D.
    \item Show $C(\Lambda)$.
\end{itemize}

\subsection{Numerical Results}

In this section, we again attempt to tune our contact interaction to unitarity by matching the first zero of the \Luscher zeta function.
However, the difference is that at each lattice spacing we tune to that spacing's respective $S^{\dispersion N}_D$, leveraging the dispersion relation for that derivative.
Then, when we extract finite-volume and finite-spacing energy levels, we put them through the dispersion equation \todo{eqref} using the same $S$ function.
The numerical results of said procedure are shown in \Figref{unimproved dispersion}.
Note that the results for $p\cot\delta$ are now flat across the spectrum, matching the known result for a contact interaction.
Moreover, comparing the scale to that in, for example, \Figref{unimproved spherical}, there the deviations were of order~1, while here the results remain within $10^{-8}$ of zero, with the value entirely reflecting how well the contact interaction was tuned.

\begin{figure}
    \includegraphics[width=\textwidth]{figure/db-contact-fv-d-fitted-parity-lg.pdf}
    \caption{The same as \Figref{unimporved spherical}, but tuned and subsequently analyzed using the appropriate latticized \Luscher function, matching the cutoff on the sum to the lattice scale and accounting for the dispersion relation.}
    \label{fig:unimproved dispersion}
\end{figure}

\begin{figure}[th]
    \includegraphics[width=\textwidth]{figure/db-contact-fv-d-fitted-parity-a-inv-lg.pdf}
    \caption{The same as \Figref{unimporved spherical}, but tuned and subsequently analyzed using the appropriate latticized \Luscher function, matching the cutoff on the sum to the lattice scale and accounting for the dispersion relation for finite scattering lenght.}
    \label{fig:unimproved dispersion finite a}
\end{figure}

In \Figref{8} we show how the strength of the contact interaction runs with the lattice scale.  According to \todo{something we know} it should be \todo{some formula that depends on cutoff}.

\begin{figure}
    \includegraphics[width=\textwidth,height=2.5in]{example-image-b}
    \caption{$C(\epsilon)$ for dispersion method.  \todo{Here put the result of \issue{14}.}}
    \label{fig:running of strength}
\end{figure}
