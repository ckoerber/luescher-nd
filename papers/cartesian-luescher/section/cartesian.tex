\section{The Cartesian Function}\label{sec:cartesian}

In the previous section, we kept emphasizing the manner in which we cut off the sum and integral---we performed a spherical truncation, meaning we cut off based on the magnitude of $\vec{n}$, $n^2 < (\Lambda L/2\pi)^2$.

In a lattice calculation, however, the cutoff is imposed by the lattice itself.
Holding the physical volume fixed, we take the continuum limit (thereby sending the momentum cutoff to infinity).
However, the cutoff is not sent to infinity in a `spherical' way---it is sent to infinity in a Cartesian way: the Brillouin zone in the calculation is a box.
We therefore consider the generalized zeta functions, but require the sums and integrals to diverge in the same way as our numerical calculation.
In other words, we incorporate the corners of the Brillouin zone into the definition of the zeta functions, for two and more spatial dimensions (for one spatial dimension the sphere is a line segment, indistinguishable from a Cartesian volume).

So, by analogy with \eqref{spherical cutoff S} we define
\begin{equation}
        S^{\cartesian N}_D(x)
        = \left(
            \sum_{\vec{n} \in (N/2, N/2]^D}
            -
            \int_{-N/2}^{N/2} \mathrm{d}n_x\ \mathrm{d}n_y\ \mathrm{d}n_z
        \right) \frac{1}{x-n^2}
\end{equation}
where the $\cartesian$ superscript indicates that the sum is cut off independently for each component.\footnote{Note that in the previous section we cut the integrals and sums off at a radius of $\Lambda/2$ or $N/2$ or, more suggestively, at a diameter of $\Lambda$ or $N$ in anticipation of comparing with a box whose edge length is $N$.}
We similarly define the Cartesian zeta function,
\begin{equation}
    S^{\cartesian}_D(x)
    =
    \lim_{N\goesto\infty} S^{\cartesian N}_D(x)
    =
    \lim_{N\rightarrow\infty}\left( \sum_{\vec{n} \in (N/2, N/2]^D} \frac{1}{x-n^2} - \counterterm_D^\cartesian \left(\frac{N}{2}\right)^{D-2}\right)
\end{equation}
and repair the quantization condition to read
\begin{equation}\label{eq:cartesian quantization}
    p \cot \delta_0(p) = \frac{\F_D\ p}{\pi^2 L^{D-2}} S^\cartesian_D(x).
\end{equation}
The counterterms $\counterterm_D^\cartesian$ are evaluated in \Appref{counterterm/cartesian}.

On possible concern when executing this change is: what happened to the delicate cancellation between the $-i$ that accompanies $\cot\delta_0(p)$ in \eqref{cot delta} with the residue from the energy integration in \eqref{I0}.
Note that the residue is accompanied by a delta function that, even if we do a Cartesian integral, still restricts us to a spherical surface where the delta function has support.
Therefore, the integrated residue is independent of the choice of integration domain, so long as the domain contains the entire sphere of support.

\begin{itemize}
    \item Show cartesian S in 2D, 3D
    \item Do the contact interaction with a fixed C, and show we get perfect perfection if we take the continuum spectrum and shove it through cartesian S but not spherical S.
\end{itemize}

\begin{equation}
    \counterterm_3^\cartesian =  0.7775513496363339
\end{equation}
