\section{The Cartesian Function}\label{sec:cartesian}

In the previous section, we kept emphasizing the manner in which we cut off the sum and integral---we performed a spherical truncation, meaning we cut off based on the magnitude of $\vec{n}$, $n^2 < (\Lambda L/2\pi)^2$.

In a lattice calculation, however, the cutoff is imposed by the lattice itself.
Holding the physical volume fixed, we take the continuum limit (thereby sending the cutoff to infinity).
However, the cutoff is not sent to infinity in a `spherical' way---it is sent to infinity in a Cartesian way: the Brillouin zone in the calculation is a box.
We therefore consider the generalized zeta functions, but require the sums and integrals to diverge in the same way as our calculation.
In other words, we incorporate the corners of the Brillouin zone into the definition of the zeta functions.

So, by analogy with \eqref{spherical cutoff S} we define
\begin{equation}
        S^{\cartesian\Lambda}_D(x) = \left(\sum_n- \int_n\right) \frac{1}{x-n^2}
\end{equation}
where the $\cartesian$ superscript indicates that the sum is cut off independently for each component.  In other words,
\begin{equation}
    \text{(something clear)}.
\end{equation}
We similarly define the Cartesian zeta function,
\begin{equation}
    S^{\cartesian}_D(x)
    =
    \lim_{\Lambda\goesto\infty} S^{\cartesian\Lambda}_D(x)
    =
    \lim_{\Lambda\rightarrow\infty}\left( \sum_n^{\text{notation}} \frac{1}{x-n^2} - \counterterm_D^\cartesian \Lambda^{D-2}\right)
\end{equation}
and repair the quantization condition to read
\begin{equation}\label{eq:cartesian quantization}
    p \cot \delta_0(p) = \frac{\F_D\ p}{\pi^2 L^{D-2}} S^\cartesian_D(x).
\end{equation}
The counterterms $\counterterm_D^\cartesian$ are evaluated in \Appref{counterterm/cartesian}.

\begin{itemize}
    \item Show cartesian S in 2D, 3D
    \item Do the contact interaction with a fixed C, and show we get perfect perfection if we take the continuum spectrum and shove it through cartesian S but not spherical S.
\end{itemize}

\begin{equation}
    \counterterm_3^\cartesian =  0.7775513496363339
\end{equation}
