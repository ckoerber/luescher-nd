\usepackage{xspace}
\usepackage{bbm}

%%%%
%%%%    Project Specifics
%%%%
\newcommand{\Luscher}{L\"{u}scher\xspace}
% \newcommand{\nstep}{\ensuremath{{n_\text{step}}}\xspace}
\newcommand{\nstep}{\ensuremath{{n_s}}\xspace}
\newcommand{\spherical}{\ensuremath{\bigcirc}\xspace}
\newcommand{\cartesian}{\ensuremath{\square}\xspace}
\newcommand{\counterterm}{\ensuremath{\mathcal{L}}}
\newcommand{\dispersion}{\ensuremath{\boxplus}\xspace}
% \newcommand{\dispersion}{\ensuremath{\sharp}\xspace}
\newcommand{\PV}{\ensuremath{\mathcal{P}}}
\newcommand{\normalization}{\ensuremath{\mathcal{N}}\xspace}

\newcommand{\lowEnergyA}{\ensuremath{\mathcal{A}}}
\newcommand{\F}{\ensuremath{\mathcal{F}}}
\newcommand{\FV}{\ensuremath{\textrm{FV}}}
\newcommand{\xtilde}{\ensuremath{\tilde{x}}\xspace}
\newcommand{\Aoneg}{\ensuremath{A_{1g}}\xspace}
\renewcommand{\l}{\ensuremath{\ell}\xspace}
\newcommand{\Laplacian}{\ensuremath{\mathop{}\!\mathbin\bigtriangleup}}
\newcommand{\BZ}{\text{B.Z.}}

%%%%
%%%%    Referring to Parts of the Repo
%%%%
\newcommand{\repoURL}{https://github.com/ckoerber/luescher-nd}
\newcommand{\issue}[1]{\href{\repoURL/issues/#1}{Issue #1}}
\newcommand{\pullrequest}[1]{\href{\repoURL/pulls/#1}{Pull Request #1}}

%%%%
%%%%    Referring to Parts of the Document
%%%%
\newcommand{\secref}[1]{Sec.~\ref{sec:#1}}
\newcommand{\Secref}[1]{Section~\ref{sec:#1}}
\newcommand{\appref}[1]{App.~\ref{sec:#1}}
\newcommand{\Appref}[1]{Appendix~\ref{sec:#1}}
\newcommand{\tabref}[1]{Tab.~\ref{tab:#1}\xspace}
\newcommand{\Tabref}[1]{Table~\ref{tab:#1}\xspace}
\newcommand{\figref}[1]{Fig.~\ref{fig:#1}\xspace}
\newcommand{\Figref}[1]{Figure~\ref{fig:#1}\xspace}
\renewcommand{\eqref}[1]{(\ref{eq:#1})\xspace}
\newcommand{\Eqref}[1]{Equation~\ref{eq:#1}\xspace}
\newcommand{\Ref}[1]{Ref.~\cite{#1}}
\newcommand{\Reference}[1]{Reference~\cite{#1}}
\newcommand{\Refs}[1]{Refs.~\cite{#1}}
\newcommand{\References}[1]{References~\cite{#1}}
%%%%
%%%%    Referring to Other Documents
%%%%

\renewcommand{\doi}[1]{\href{http://doi.org/#1}{[#1]}}
\newcommand{\arxiv}[1]{\href{http://www.arxiv.org/abs/#1}{arXiv:#1}}

%%%%
%%%%    Mathematical Symbols
%%%%

\newcommand{\goesto}{\ensuremath{\rightarrow}}
\newcommand{\infinity}{\infty}
\newcommand{\Integers}{\mathbb{Z}\xspace}
\newcommand{\integers}{\Integers}
\newcommand{\one}{\ensuremath{\mathbbm{1}}}
\newcommand{\order}[1]{\ensuremath{\mathcal{O}\left(#1\right)}\xspace}
\newcommand{\Rationals}{\mathbb{Q}\xspace}
\newcommand{\Reals}{\mathbb{R}\xspace}
\newcommand{\union}{\ensuremath{\cup}}
\DeclareMathOperator{\erf}{erf}
\DeclareMathOperator{\erfi}{erfi}
\renewcommand{\mod}[1]{\ensuremath{\ \left(\text{mod }#1\right)}}

%%%%
%%%%    Hyperbolic Trig
%%%%

% Most are already available if you \usepackage{amsmath}.
% However, those below are missing

\DeclareMathOperator{\sech}{sech}
\DeclareMathOperator{\csch}{csch}
\DeclareMathOperator{\arccosh}{arccosh}
\DeclareMathOperator{\arcsinh}{arcsinh}
\DeclareMathOperator{\arctanh}{arctanh}
\DeclareMathOperator{\arcsech}{arcsech}
\DeclareMathOperator{\arccsch}{arccsch}
\DeclareMathOperator{\arccoth}{arccoth}

% Additionally, there are some missing trig functions:

\DeclareMathOperator{\arcsec}{arcsec}
\DeclareMathOperator{\arccot}{arccot}
\DeclareMathOperator{\arccsc}{arccsc}

%%%%
%%%%    Fractions
%%%%

\newcommand{\oneover}[1]{\ensuremath{\frac{1}{#1}}}                             %   1/[argument]
\newcommand{\inverse}{\ensuremath{^{-1}}}                                       %   argument^-1
\newcommand{\half}{\ensuremath{\frac{1}{2}} }                                   %   1/2
\newcommand{\quarter}{\ensuremath{\frac{1}{4}} }                                %   1/4


%%%%
%%%%    Mathematical Delimiters
%%%%
\newcommand{\abs}[1]{\ensuremath{\left| #1 \right|}\xspace}
\newcommand{\magnitude}{\abs}
\newcommand{\average}[1]{\ensuremath{\left\langle #1 \right\rangle}\xspace}

% Quantum mechanics bra-ket notation:
\newcommand{\ket}[1]{\ensuremath{\left|\;#1\;\right\rangle}}
\newcommand{\bra}[1]{\ensuremath{\left\langle\;#1\;\right|}}
\newcommand{\bracket}[2]{\ensuremath{\left\langle\;#1\;\middle|\;#2\;\right\rangle}}
\let\braket\bracket
\newcommand{\braMket}[3]{\ensuremath{\left\langle\;#1\;\middle|\;#2\;\middle|\;#3\;\right\rangle}}


%%%%
%%%%    Matrices
%%%%

\newcommand{\identity}{\ensuremath{\mathds{1}}}
\newcommand{\diag}[1]{\ensuremath{\text{diag}\left(#1\right)}}
\newcommand{\tr}[1]{\ensuremath{\text{tr}\left[#1\right]}}
\newcommand{\transpose}{\ensuremath{{}^{\top}}}

%%%%
%%%%    Physical Quantities and Constants
%%%%




%%%%
%%%%    Software
%%%%

\newcommand{\bash}{\texttt{bash}\xspace}
\newcommand{\git}{\texttt{git}\xspace}
\newcommand{\make}{\texttt{make}\xspace}
\newcommand{\mpi}{\texttt{MPI}\xspace}
\newcommand{\python}{\texttt{python}\xspace}

% Put an xspace after \LaTeX:
\let\builtinLaTeX\LaTeX
\def\LaTeX{\builtinLaTeX\xspace}
