\documentclass[aps,superscriptaddress,tightenlines,nofootinbib,floatfix,longbibliography,notitlepage]{revtex4-1}
\usepackage[left=18mm,right=19mm,top=23mm,bottom=16mm]{geometry}
\usepackage{amsmath,amssymb}
\usepackage{bm,bbm}
\usepackage{comment}
\usepackage{graphicx}
\usepackage{color}
\usepackage[dvipsnames]{xcolor}
\usepackage{slashed}
\usepackage[multidot]{grffile} %because multiple dots in filenames confuse latex
\usepackage[
    colorlinks=true,
    allcolors=blue
]{hyperref}
\usepackage{pgfplots}

\newcommand\todo[1]{{\bf\color{red}TODO: #1}}

%%%
%%%     Incorporate Repository Information
%%%

\providecommand{\repositoryInformationSetup}{} % Fallback definition if not compiled with `make DRAFT=1`.
\repositoryInformationSetup

%%%
%%%     Include latex-base macros
%%%

\input{macros} % input rather than include so we don't create macros.aux

%%%%
%%%%    Document preparation
%%%%

\begin{document}

\title{L\"{u}scher's Formalism in 1-, 2- and 3-dimensions\\--the unitary limit on the lattice and it's caveats}
\newcommand{\ikp}{
    Institut f\"{u}r Kernphysik,
    Forschungszentrum J\"{u}lich, 54245 J\"{u}lich Germany
}

\newcommand{\ias}{
    Institute for Advanced Simulation,
    Forschungszentrum J\"{u}lich, 54245 J\"{u}lich Germany
}

\newcommand{\bonn}{
    Helmholtz-Institut f\"{u}r Strahlen- und Kernphysik,
    Rheinische Friedrich-Williams-Universit\"{a}t Bonn, 53012 Bonn Germany
}

\newcommand{\jsc}{
    J\"{u}lich Supercomputing Center,
    Forschungszentrum J\"{u}lich, 54245 J\"{u}lich Germany
}

\newcommand{\berkeley}{
    Department of Physics,
    University of California, Berkeley, CA 94720, USA
}


\author{Christopher K\"orber}   \affiliation{\ias} \affiliation{\ikp} \affiliation{\bonn} \affiliation{\berkeley}
\author{Evan Berkowitz}         \affiliation{\ias} \affiliation{\ikp}
\author{Thomas Luu}             \affiliation{\ias} \affiliation{\ikp} \affiliation{\bonn}





\date{\today}

\begin{abstract}
Studies of strongly interacting many-fermion systems at unitarity, defined invariance under scaling, require tuning the two-fermion subsystem such that phase shifts vanish.
While Finite Volume (FV) lattice descriptions allow access to systems with large numbers of fermions, FV lattice observables need to be transported to the Infinite Volume (IV) and continuum.
Possible complications with the extrapolation procedure make it difficult to study the unitary limit in three-dimensions, while for a two-dimensional system it is currently unknown how to describe the unitary limit.
In particular, L\"{u}scher's formalism directly converts FV continuum energy levels to IV continuum phase shifts at the two-body level.
Applying L\"{u}scher's formalism can be difficult because it is not always practically and sometimes even not possible to perform a continuum limit at fixed volumes.
In this paper we concentrate on the interplay of lattice discretization and FV effects.
We describe how to tune one-, two- and three-dimensional two-fermion systems to unitarity while highlighting possible difficulties.
We find that tuning to the two-dimensional unitary limit requires a simultaneous continuum and FV extrapolation.
In the three-dimensional case discretization artifacts induce finite range interactions which can be systematically removed.
These findings may help to systematically improve lattice computations of strongly interacting many-fermion observables at unitarity in two- and three-dimensional systems.
\end{abstract}

\maketitle


\section{Introduction}\label{sec:intro}

Many physically interesting systems comprise strongly-interacting fermions.
In three spatial dimensions the scattering of fermions with a short-range interaction can be completely characterized by a scattering length, and when that length diverges the details of the potential are washed out and no dimensionful scales remain.
Such \emph{unitary fermions} exhibit interactions as strong as can be without forming bound states, and provide an interesting guide for understanding other strong interactions because of their universal behavior.
For example, the nuclear interaction in the deuteron channel has an extremely long scattering length, and trapped ultracold atoms can be tuned to unitarity by applying external magnetic fields and leveraging Feshbach resonances.

By tuning a quantum-mechanical two-body contact interaction, one should be able to completely control the scattering length and, absent other interactions, have that scattering length completely describe the scattering.
With such an interaction in hand, a variety of interesting many-body problems are unlocked.
Since all other dimensionful quantities are gone, all observables must be determined by naive dimensional analysis in the the density, times some non-perturbative numerical factor, such as the Bertsch parameter\cite{PhysRevC.60.054311} in the case of the energy density.

In fact, a contact interaction can be shown to always produce momentum-independent scattering amplitudes (in three dimensions, for example, a momentum-independent $p \cot \delta$), and it ought to be possible to produce any amplitude, unless otherwise restricted by the Wigner bound\cite{Wigner:1955zz,Phillips:1996ae,Hammer:2010fw}.

Such scale-free results must result from peculiar potentials.
In three dimensions, for example, a delta function potential requires regulation, and to get scale-free dynamics its dimensionful strength must be sent to zero with the removal of the regulator in just such a way as to keep the phase shift at $\pi/2$.
In one dimension the strength of the contact interaction is also dimensionful and a delta function potential needs no regulation, but nevertheless is regulated when space is discretized; in two dimensions the strength of the delta function potential is dimensionless, which entails a more complicated story we discuss in \Secref{2D}.

Numerical computations are often performed in discretized boxes with periodic boundary conditions.
\Luscher's finite-volume formalism\cite{Hamber198399,luscher:1986I,luscher:1986II,wiese1989,Luscher1991,Luscher1991237} is the method by which one can extract infinite-volume real-time scattering data from the finite-volume Euclidean spectrum of a theory, taking advantage of the interplay between the physical scattering and the finite-volume boundary conditions in determining the spectrum.  Recently there has been an investigation of \Luscher's formalism for continuous scattering within a crystal lattice~\cite{Valiente:2015oya}.

The usual understanding of \Luscher's formalism is that one should find the continuum zero-temperature finite-volume energy levels, holding the physical volume fixed, and put that cold, continuum spectrum through \Luscher's formula to extract continuum scattering data. 

In practice, few results of lattice QCD calculations are zero-temperature- or, more seriously, continuum-extrapolated, but are nevertheless put through \Luscher's formula to get an estimate of the continuum scattering data, assuming thermal and discretization effects to be much smaller than the statistical uncertainties\todo{cite cite cite}.
In particular, no continuum-limit study of any baryonic channel exists, even at unphysically heavy pion masses.

While alternatives, including the potential method (see, for example \Refs{Ishii:2006ec,Nemura:2008sp,Aoki:2009ji,Murano:2011nz,Aoki:2012bb,Kurth:2013tua,Sugiura:2017vwo,Yamazaki:2019vid,Aoki:2017yru,Yamazaki:2018qut,Iritani:2017rlk,Iritani:2018zbt,Gongyo:2018gou,Akahoshi:2019klc,Namekawa:2019xiy}), the mapping onto harmonic oscillators \Ref{McElvain:2019ltw} and the imposition of spherical walls\cite{Borasoy:2007vy,Borasoy:2007vi,Lee:2008fa,Epelbaum:2008vj,Epelbaum:2010xt,Lu:2015riz,Elhatisari:2015iga,Elhatisari:2016owd,Elhatisari:2016hby,Klein:2018lqz,Li:2019ldq,Bovermann:2019jbt,Lahde:2019npb}, can be used to translate finite-volume physics to infinite-volume observables, here we focus on the \Luscher finite-volume formalism.
Moreover, to our knowledge, no numerical work leveraging these methods is in the continuum, either.  
Here, we construct example Hamiltonians explicitly and diagonalize them exactly, albeit numerically.
This allows us to circumvent all of the issues of statistical uncertainty that accompanies Monte Carlo data, and lets us completely isolate the features of the formalism itself, removing, for example, any finite-temperature effects that should in principle be extrapolated away in any finite-temperature method like Lattice QCD.

We find that it is in practice difficult to reliably extrapolate the spectrum to the continuum limit in a way that reproduces the exact known result, but that taking the continuum limit of the lattice-artifact-contaminated phase shifts sometimes can produce a more reliable result.

Our main innovation, however, is to explain how to incorporate lattice artefacts into \Luscher's formula, for systems described by a contact interaction, accounting both for the Brillouin zone of the lattice and the lattice-induced dispersion relation.

While not universal, this lattice improvement can be quite useful for a contact interaction.
In pursuit of a lattice formulation of unitary fermions, the authors of \Ref{Endres:2011er} followed the tuning procedure of \Ref{Lee:2007ae}, parametrizing the contact interaction as a sum of a tower of Galilean-invariant operators, tuning their coefficients so as to drive the lowest interacting energy levels to the zeros of the \Luscher finite-volume zeta function.
However, in \Ref{Endres:2012cw} they found that even with a highly-improved construction the states ultimately deviated from a $\pi/2$ phase shift (see, for example, Figure 3).
In \Ref{He:2019ipt} the lattice implementation was smeared to reduce errors due to discretization, however a direct comparison of other methods with theirs was not possible for us since we were not able to identify the discretization parameters for the presented phase shifts (Fig. 7).

We introduce a new continuum-limit prescription for achieving unitarity in lattice simulations by tuning just the simplest contact operator, but taking the discretization effects into account by incorporating the lattice dispersion relation into the finite-volume zeta function, both in the tuning step and in the analysis step.
By re-tuning the interaction at each lattice spacing we can very easily and smoothly take the continuum limit after applying the lattice-aware finite-volume formula.
We demonstrate that this allows us to maintain a constant phase shift deep into the spectrum, covering as many \Aoneg states as exist in the lattice of interest.

This paper is organized as follow.  In \Secref{scattering} we give a brief summary of two particle scattering in $D$ dimensions.
In \Secref{hamiltonian} we give specifics about the latticized contact-interaction Hamiltonians we study numerically.
In \Secref{luescher} we provide a traditional continuum derivation of \Luscher's formula and in \Secref{dispersion} explain how to adapt it to include finite spacing effects by truncating the usual sum to just the momentum modes in the lattice and incorporating the dispersion relation into the appropriate propagators, yielding a lattice-improved generalized \Luscher zeta function.

Then, we leverage our dispersion zeta function, studying concrete examples.
In \Secref{3D} we study the three-dimensional case.
First we compare a continuum-extrapolated energy spectrum fed through the continuum zeta function and the continuum extrapolation of the finite-spacing spectra fed through the continuum zeta.
In \Secref{3D} we tune and analyze the same problem using our lattice-aware dispersion zeta function, and show that the resulting scattering $p\cot\delta$ remains constant deep into the spectrum; when we tune to unitarity the results stay at the expected value as accurately as the initial tuning is made modulo propagated numerical uncertainties.
We then study the one dimensional case in \Secref{1D}, where the absence of a counterterm makes things particularly simple.
In \Secref{2D} we repeat the story for the more intricate two-dimensional case, where dimensional transmutation and logarithmic singularities require special attention and care.
We find that our lattice-aware \Luscher function handles this case with no difficulty.
Further, in all dimensions considered here we provide correction terms that come about when using energies calculated in a discrete space but fed through continuum \Luscher formula, which when applied to three dimensions corrects for the deviation found in \Ref{Endres:2012cw}.
Our corrections are valid only for the case of a contact interaction.
Finally, we recapitulate our findings in \Secref{conclusion} and discuss future directions.


\section{Two-particle scattering}

\subsection{The effective range expansion}
Scattering information of a two-particle system are captured by the scattering-matrix which is related to the $T$-matrix.
The off-shell $T$-matrix can be obtained for a two-particle interaction $\hat V$ by solving the Lippmann-Schwinger equation
\begin{align}
	T_D(\vec p', \vec p, E)
	&=
	V(\vec p', \vec p) + \lim\limits_{\epsilon \to 0}\int \frac{d \vec k^D}{(2\pi)^D} V(\vec p', \vec k) G(\vec k, E + i \epsilon) T(\vec k, \vec p, E) \, ,
	&
	G(\vec k, E+ i \epsilon) = \frac{1}{E + i \epsilon - \frac{k^2}{2\mu}}
	\, .
\end{align}
After projecting the on-shell $T$-matrix onto partial waves,
the $T$-matrix is related to scattering phase shifts by
\begin{align}\label{eq:on-shell-T}
	\frac{1}{T_{lD}(p)}
    \equiv
    \frac{1}{T_{lD}(p, p, E_p)}
    = \frac{\mu}{2}
    \frac{1}{\mathcal F_{l D}(p)} \left[\cot (\delta_{l D}(p)) - i\right] \, ,
\end{align}
where $E_p = p^2 / (2 \mu)$ and $\mathcal F_{l D}(p)$ is a dimension-dependent kinematic function of the on-shell momentum.

Unitarity is obtained if the scattering is scale invariant, e.g., the phase shifts are independent of dimensional quantites
\begin{align}
    \cot (\delta_{l D}(p)) &= 0
    \,, &
    \frac{\partial}{\partial p}\cot (\delta_{l D}(p)) &= 0
    \, &
    \, \forall p\,.
\end{align}

Low energy theories usually consider the expansion of eq.~(\ref{eq:on-shell-T}) in scattering momenta $p$, called the effective range expansion (ERE), which takes the form \cite{Hammer:2010fw}
\begin{align}
    \cot \left(\delta_{l D}(p)\right)
    &=
    \delta_D \frac{2}{\pi}  \ln \left(p R_{l D}\right)
    -
    \frac{1}{a_{l D}} p^{2 - 2 l - D} +\frac{1}{2} r_{l D} p^{4 - 2 l - D} + O\left(p^{6 - 2 l - D}\right)
    \, , &
    \delta_D &= \begin{cases}
        0 & D \;\text{odd} \\ 1 & D \;\text{even}
    \end{cases}
    \, .
\end{align}
The parameter $R_{l D}$ is an arbitrary length scale.
The following terms containing $a_{l D}$ (for $l=0$ and $d=3$ called the scattering length), $r_{l D}$, (or $l=0$ and $d=3$ called effective range) and terms coming with higher powers of the scattering momenta correspond to properties of the two-particle interaction.

\todo{Address cutoff}
For a contact interaction, all contributions to the ERE for partial waves different than S-wave ($l=0$) vanish
\begin{align}\label{eq:quantization-contact-physical}
	V(\vec p', \vec p) &= c(\Lambda)
	\, \Rightarrow &
	T_D(p) = T_{0D}(p) &= \frac{c(\Lambda)}{1 - c(\Lambda) I_D(E_p, \Lambda)}
	\, , &
	I_D(E_p, \Lambda) = \lim\limits_{\epsilon \to 0} \int\limits_{|\vec k| < \Lambda} \frac{d \vec k^D}{(2\pi)^D} G(\vec k, E_p + i \epsilon)
	\, .
\end{align}
In particular, for a contact interaction, one finds
\begin{align}
    \cot \left(\delta_{l D}(p)\right)
    =
    \delta_{l,0}
    \cot \left(\delta_{0 D}(p)\right)
    & =
    i +
    \frac{2}{\mu}
    \frac{1}{\mathcal{F}_{0D}}
    \left[
        \frac{1}{c(\Lambda)}
        -
        \lim\limits_{\epsilon\to0}{}_2F_1\left(1, \frac{D}{2}, \frac{D+3}{2}, \frac{\Lambda^2}{(p + i \epsilon)^2} \right)
    \right]
    \\
    &=
    \begin{cases}
        ... & (D=1)\\
        ... & (D=2)\\
        ... & (D=3)
    \end{cases}
\end{align}
\todo{needs to be finalized}

\section{Discretized Hamiltonian}\label{sec:hamiltonian}

A two-body system interacting via a contact interaction is the focus of our study.
By definition, only S-wave information are not vanishing for a contact interaction and the corresponding ERE has no effective range (and higher terms).
Thus, to obtain unitarity, one only has to tune the interaction to the S-wave scattering length.
The hamiltonian, with contact strength $C$, describing the system is given by
\begin{equation}
    \label{eq:particle hamiltonian}
    \hat H = \frac{\hat p_1^2}{2 m_1} + \frac{\hat p_2^2}{2 m_2} + C \delta(\hat x_1 - \hat x_2)
    \, .
\end{equation}
The subscripts indicate the particle of the position and momentum operators.
This hamiltonian becomes, moving to center-of-mass and relative coordinates,
\begin{equation}
    \label{eq:hamiltonian}
    \hat H = \frac{\hat P^2}{2 M} + \frac{\hat p^2}{2 \mu} + C \delta(\hat x)
\end{equation}
where capital letters represent center-of-mass variables, lower case implies relative coordinates.
The problem is reducded to an effective one-body problem once we specialize to the rest frame, setting $P=0$.

We consider a finite cubic volume (FV) of linear size $L$ with periodic boundary conditions and lattice spacing $\epsilon$ so that $N=L/\epsilon$ is an even integer that counts the number of sites in one spatial direction.

The contact interaction is implemented on the lattice as an entirely local operator, vanishing everywhere except at the origin where it is of strength $C$ (e.g., the contact interaction is not smeared).

In contrast, to analyze the effects of discretizations, we study a variety of kinetic operators which we distinguish by the $\nstep$ label, which indicates how many nearest neighbors in each direction go into the finite-difference Laplacian.
For example, $\nstep=1$ denotes the symmetric nearest-neighbor finite-difference Laplacian.
We consider further stencils which extend on-axis steps so that the finite difference Laplacian is a $(1+2\nstep D)$-point stencil in $D$ dimensions,
\begin{equation}
    \left\langle \vec{r}' \middle| H \middle| \vec{r} \right\rangle
    \rightarrow
    H_{\vec{r}',\vec{r}}^{(L,\epsilon,\nstep)}
    =
    - \frac{1}{2 \mu \epsilon^2}
        \sum_{d=1}^{D} \sum_{s=-\nstep}^{+\nstep}
            \gamma^{(\nstep)}_{|s|} \delta_{\vec{r}',\vec{r}+\epsilon s \vec{e}_d}
    + \frac{1}{\epsilon^D}C(\epsilon) \delta_{\vec{r}',\vec{r}}\delta_{\vec{r},\vec{0}}
\end{equation}
where the spatial indices are understood modulo the periodic boundary conditions of the lattice.
In momentum space, this Hamiltonian may be written as
\begin{align}
    \label{eq:p space hamiltonian}
    \left\langle \vec{p}' \middle| H \middle| \vec{p} \right\rangle
    \rightarrow
    H_{\vec{p}',\vec{p}}^{(L,\epsilon,\nstep)}
    &=
    \delta_{\vec{p}',\vec{p}} \frac{1}{2\mu} \sum_{d=1}^{D} \omega^{(\nstep)}(p_d,\epsilon)
    +\frac{1}{L^D}C(\epsilon)
    \\
    \label{eq:gamma definition}
    \omega^{(\nstep)}(p_d,\epsilon)
    &= \frac{1}{\epsilon^2} \sum_{s=0}^{\nstep} \gamma_{s}^{(\nstep)} \cos(s p_d \epsilon)
\end{align}
where $\vec{p} = 2\pi \vec{n}/L$ for a $D$-plet of integers $\vec{n} \in (-N/2, +N/2]^D$, and the coefficients $\gamma_{s}^{(\nstep)}$ are determined by requiring the dispersion relation be as quadratic as possible,
\begin{equation}
    \label{eq:gamma determination}
    \omega^{(\nstep)}(p_d,\epsilon) \overset{!}{=} p_d^2 \left[ 1 + \order{(\epsilon p_d)^{2\nstep}}\right].
\end{equation}
The resulting dispersion relations are presented in \Figref{dispersion relation} for a variety of $\nstep$s and
in \Appref{coefficients} we collect the required $\gamma$ coefficients.
In addition, we use a nonlocal operator, denoted by $\nstep=\infty$ which, in momentum space can be implemented to multiplying by $p^2$ directly,
\begin{equation}
    \omega^{\infty}(p_d,\epsilon) = p_d^2,
\end{equation}
including at the edge of the Brillouin zone, the Laplacian implementation of the ungauged SLAC derivative.
Including the edge does not introduce a discontinuity at the boundary (though it does introduce a cusp).

Once constructed, a projection operator, is added to this Hamiltonian
\begin{equation}
    H(\alpha) = H + \alpha (\one - P_{\Aoneg}) \, ,
\end{equation}
where $P_{\Aoneg}$ is a projector to the \Aoneg irrep (needed for extracting S-wave information in the infinite volume).
Because $P_{\Aoneg}$ commutes with $H$, $H$ and $H(\alpha)$ have the same spectrum within the $\Aoneg$ irrep.
If $\alpha$ is much larger than the expected energies of the Hamiltonian, the \Aoneg states remain low-lying and all other states are shifted to much higher energies.
Then, exactly diagonalizing $H(\alpha)$ instead of $H$ provides an easier extraction of \Aoneg eigenenergies.

Throughout we focus on a three-dimensional system, though in \Appref{two-d} we study a two-dimensional system, where logarithmic divergences warrant special attention.

\begin{figure}
    \subsubsection{Dispersion L\"{u}scher removes induced artifacts}

In this section, we again attempt to tune our contact interaction to unitarity by matching the first zero of the \Luscher zeta function.
However, the difference is that at each lattice spacing we tune to that spacing's respective $S^{\dispersion N}_D$, leveraging the dispersion relation for that derivative.
Then, when we extract finite-volume and finite-spacing energy levels, we put them through the dispersion equation \todo{eqref} using the same $S$ function.\footnote{We actually make the replacement $\F_D^\epsilon\goesto\F_D$ to avoid numerically evaluating $F_D^\epsilon$.  With a controlled continuum limit the error from this change obviously vanishes.}
The numerical results of said procedure are shown in \Figref{unimproved dispersion}.
Note that the results for $p\cot\delta$ are now flat across the spectrum, matching the known result for a contact interaction.
Moreover, comparing the scale to that in, for example, \Figref{unimproved spherical}, there the deviations were of order~1, while here the results remain within $10^{-8}$ of zero, with the value entirely reflecting how well the contact interaction was tuned.

\begin{figure}[htb]
    \scalebox{0.9}{\input{figure/ere-contact-fitted_a-inv_+0.0_zeta_dispersion_projector_a1g_n-eigs_200.pgf}}
    \caption{The same as \Figref{unimproved spherical}, but tuned and subsequently analyzed using the appropriate latticized \Luscher function, matching the cutoff on the sum to the lattice scale and accounting for the dispersion relation.}
    \label{fig:unimproved dispersion}
\end{figure}

\begin{figure}[hbt]
    \scalebox{0.9}{\input{figure/ere-contact-fitted_a-inv_-5.0_zeta_dispersion_projector_a1g_n-eigs_200.pgf}}
    \caption{The same as \Figref{unimproved spherical}, but tuned and subsequently analyzed using the appropriate latticized \Luscher function, matching the cutoff on the sum to the lattice scale and accounting for the dispersion relation for finite scattering lenght.}
    \label{fig:unimproved dispersion finite a}
\end{figure}

In \Figref{dispersion running of strength} we show how the strength of the contact interaction runs with the lattice scale.  According to \todo{something we know} it should be \todo{some formula that depends on cutoff}.

\begin{figure}
    \input{figure/contact-scaling-contact-fitted_a-inv_+0.0_zeta_dispersion_projector_a1g_n-eigs_200.pgf}
    \caption{
        Scaling of the contact interaction strength $C(\epsilon)$ fitted using the dispersion method at unitarity.
        Data points are fitted values, solid lines are analytical scaling predictions following $C(\epsilon) = \frac{2}{\mu} \frac{1}{\mathcal L^{\dispersion n_\mathrm{step}}} \epsilon $ and the dashed line corresponds to the spherical predictions.
        Bar diagrams below present the absolute error between prediction and extracted value.
    }
    \label{fig:dispersion running of strength}
\end{figure}

\clearpage

    \caption{We show the continuum dispersion relation of energy as a function of momentum for different one-dimensional $\nstep$ derivatives.  For a finite number of lattice points $N$, the allowed momenta are evenly-spaced in steps of $2\pi/N$.
    As additional steps are incorporated into the finite difference, the dispersion relation more and more faithfully reproduces the desired $p^2$~behavior of $\nstep=\infty$.
    }
    \label{fig:dispersion relation}
\end{figure}

\input{section/two-particle-scattering/spherical-3d}

\section{Results}
\section{Three Dimensions}\label{sec:3D}


Now we can attempt a numerical characterization of two fermions at unitarity (and finite for finite scattering).
We implement the Hamiltonian in \eqref{p space hamiltonian} in a three-dimensional cubic box of linear size $L$ with lattice spacing $\epsilon$.
The interaction strength $C(\Lambda)$ is tuned such that the ground state energy $E_0$ matches the first zero of the spherical zeta function $S^{\spherical}_3$ (see also \figref{tuning}).
In our computations, the low-lying energy levels for fixed physical volume and fixed lattice spacing are extracted using numeric exact diagonalization after the strength is tuned to machine precision.
The zeta function is evaluated using software provided by \Refs{Morningstar:2017spu,Morningstar:2hib}.

This tuning procedure ensures that the FV effects are incorporated in the energy levels and thus the contact interaction strength is independent of the volume length $L$.
However, the interaction strength still depends on the implementation of the Kinetic operator and the lattice spacing, $C(\Lambda)\to C^{(n_s)}(\epsilon)$.
Therefore the strength has to be retuned for each lattice implementation.
This lattice dependence has the consequence that each lattice observable, e.g., the energy spectrum (besides the input ground state), has to extrapolated to the continuum limit first to extract \textit{pure} FV energy levels.
These FV energy levels can be used in Lüscher's formalism to extract infinite volume scattering data.

In practice, it is not always possible to compute a continuum limit before inserting the FV spectrum into Lüscher's formalism.
We therefore attempt both approaches, performing a continuum limit before (as intended) and after applying Lüscher (as happens in practice) to the obtained spectrum.



\begin{figure}
\center
\includegraphics[width=.7\textwidth]{figure/tuning.pdf}
\caption{\label{fig:tuning}
    The Finite Volume energy spectrum is extracted by computing the intersections of the Lüscher Zeta function (modulo factors) with the phase shifts (see \eqref{spherical quantization}).
    The first intersection $x_0$, which corresponds to the binding energy $E_0 < 0$, is used to tune the interaction $c^{(n_s)}(\epsilon)$.
    This treatment allows to determine $c^{(n_s)}(\epsilon)$ independent of volume sizes.
}
\end{figure}


\subsection{Continuum extrapolation before infinite volume limit}

After tuning the contact interaction to the first zero of the spherical zeta function, we compute the spectrum of the hamiltonian.
Next, we extrapolate the obtained energy eigenvalues to the continuum $\epsilon \to 0$ using a polynomial fit
\begin{equation}
    E^{(n_s)}_i(\epsilon) = E_i^{(n_s)} + \sum\limits_{n=1}^{n_\mathrm{max}} e_{i,n}^{(n_s)} \epsilon^n \, .
\end{equation}
Because the contact interaction is expected to scale linear with the momentum cutoff and thus linear in $\frac{1}{\epsilon}$ (see \eqref{three-d-counterterm}), one cannot generally expect the fit coefficients $e_{i,n}^{(n_s)}$ to be zero for odd $n$ or $n < n_s$ as the dispersion suggests.
Nevertheless, we would expect the small $n$ coefficient for larger $n_s$ to be relatively smaller then small $n$ coefficients for smaller $n_s$: $e_{i,n}^{(n_{s_1})} < e_{i,n}^{(n_{s_2})}$ on average for $n_{s_1} > n_{s_2}$.

We individually fit each discretization implementation to extract the continuum energies $E_i^{(n_s)}$ using the software provided by \Ref{peter_lepage_2016_60221}.
Because our numerical uncertainties have an estimated relative error at the order $~10^{-13}$, we must in principle fit the energy for relatively high values of $n_\mathrm{max}$ which would require having many data points over different scales of $\epsilon$.
For this reason we add further epsilon values
\begin{equation}
	\left\{
		\epsilon \, [\mathrm{fm}] =
		\frac{1}{4}, \frac{1}{5}, \frac{1}{10},
		\frac{1}{15}, \frac{1}{20}, \frac{1}{25},
		\frac{1}{ 30}, \frac{1}{ 35}, \frac{1}{ 40},
		\frac{1}{ 41}, \frac{1}{ 42}, \frac{1}{ 43},
		\frac{1}{ 44}, \frac{1}{45}, \frac{1}{ 46},
		\frac{1}{ 47}, \frac{1}{ 48}, \frac{1}{ 49}, \frac{1}{ 50}
	\right\}
	\, .
\end{equation}
However, we still obtain $\chi^2_{\mathrm{d.o.f}} \gg 1$ up to the point where it is computationally not feasible to add new data points for even smaller lattice spacings as the dimension of the hamiltonian scales with $(L/\epsilon)^3$.

For this reason, we have decided to fit multiple fit models over the span of $n_\mathrm{max} = \{2, 3, 4, 5\}$ and compare their results to estimate a systematic extrapolation uncertainty (unweighted average and standard deviation of results over models).
We repeat this procedure for each discretization and compare different continuum energies to decide wether the fits are consistent.
These values are compared to the spectrum predicted by L\"uscher's formalism.

We present the model average over best fits of the spectrum in \figref{continuum-extrapolation-spectrum}.
Also, we provide access to the raw data and fitting scripts online at \cite{luescher-nd_201}\todo{update repo link once done. See \href{https://github.com/AASJournals/Tutorials/blob/master/Repositories/CitingRepositories.md}{how to cite repos}}.
We observe that the model average for polynomials of degree 2 up to 5 is consistent over different discretization and agrees with the expected continuum results.
We noted that including higher polynomials with $n_\mathrm{max} > 6$ resulted in overfitting of higher energy levels visible in oscillating fit functions which were generally were more favorable in model selection criteria\footnote{
    A potential cure for overfitting of higher polynomials would have been the marginalization of higher contributions which would cast the contributions of higher neglected epsilon terms into the uncertainty of the data.
    We eventually settled for an unweighted model average over smaller $n_\mathrm{max}$ because the continuum extrapolated spectrum was more consistent over different $n_s$.
}.
As expected, the continuum limit becomes more uncertain for excited states.
Furthermore, the $n_s = \infty$ implementation provides the most precise results.
Surprisingly it seems like the $n_s = 1$ implementation seems to be more precise on average than some improved implementations -- even though non-extrapolated values are further apart from the continuum as in the other cases.
This effect is related to the continuum convergence pattern.
While the $n_s = 1$ (and $n_s = \infty$) energy values seem to converge against the continuum result from below (and respectively from above) for all excited states, the improved derivative eigenvalues change their convergence pattern.
The slope of the extrapolation function changes it sign from $E_2 \to E_4$ for $n_s = 2$ and from $E_6 \to E_8$ for  $n_s = 4$.
This suggests that the importance of fit model coefficients $e_{i,n}^{(n_s)}$ changes and thus makes it more difficult to perform the continuum limit.

In the next step, we use the continuum extrapolated spectrum to convert it to phase shifts using the spherical zeta function.
We present the phase shifts in \figref{continuum-extrapolation-ere}.
Independent of discretization scheme, we observe that the continuum extrapolated results agree with the constant input phase shifts.
Because the zeta function is relatively steep, uncertainties in the continuum limit get drastically enhanced when converting to phase shifts (on average more than an order of magnitude).
We observe that for for $x > 5$ all discretization besides the $n_s = \infty$ discretization come with significant uncertainties.


\begin{figure}[htb]
\scalebox{1.0}{\input{figure/continuum-extrapolation.pgf}}
\caption{
    \label{fig:continuum-extrapolation-spectrum}
    Continuum extrapolation of the discrete finite volume spectrum with $L=1\,[\mathrm{fm}]$.
    Each column represents a different implementation of the Kinetic operator, rows correspond to eigenvalues of the hamiltonian sorted by size.
    For visualization purposes we present each second eigenvalue starting at $E_2$ because $E_0$ was used to tune the interaction and is thus constant by definition.
    Black dots are the eigenvalues at different lattice spacings, the green band is the model averaged fit function for best parameters and the blue band parallel to the x-axis is the continuum extrapolated energy.
    The uncertainty is dominated by the fluctuations over models, the propagated numerical uncertainty is negligible in comparison.
    The dashed line corresponds to the expected result obtained by computing the intersection of the zeta function $S^{\spherical}_3$ with the phase shifts.
    The boundary of each frame corresponds to the poles of the zeta function.
    Different energy extrapolations in the continuum agree with Lüscher's prediction within uncertainty.
    For finite discretization implementations ($n_s < \infty$), the uncertainty drastically increases with the number of excited states ($\sim 3$ orders of magnitude from $E_2^{(n_s)}$ to $E_{20}^{(n_s)}$).
}
\end{figure}


\begin{figure}[hbt]
    \scalebox{1.0}{\input{figure/continuum-extrapolation-ere.pgf}}
    \caption{
        \label{fig:continuum-extrapolation-ere}
        Phase shifts computed by inserting the continuum extrapolated spectrum for different discretization implementations $n_s$ and finite volumes $L$ in the zeta function $S^{\spherical}_3$.
        Data points indicate locations of the eigenvalue.
        For visualization reasons we express the propagated error associated with continuum extrapolation as an uncertainty band.
        The black dashed line represents physical phase shifts.
        Bands stop at different $x$ values because we stop presenting results after uncertainties become too large (but are still consistent with the physical phase shifts).
        }
\end{figure}


We emphasize that these findings are not unique to the unitary case, we obtain similar results for a non-zero scattering length.
We present data for the non-unitarity scenario, e.g., $a_{30} = - 5 \, [\mathrm{fm}]$ in our repository \cite{luescher-nd_201}.

\subsection{Infinite volume limit before continuum extrapolation}
Next we want to discuss what effects finite discretization artifacts have when applying L\"uscher's formalism to a spectrum for finite lattice spacings.
We insert the energy levels presented in \figref{continuum-extrapolation-spectrum} before taking the continuum limit and present results in figure \figref{unimproved spherical}.

\begin{figure}[th]
    \scalebox{0.9}{\input{figure/ere-contact-fitted_a-inv_+0.0_zeta_spherical_projector_a1g_n-eigs_200.pgf}}
    \caption{
        \label{fig:unimproved spherical}
        We present energy eigenvalues presented in \figref{continuum-extrapolation-spectrum} directly inserted $S^{\spherical}_3$--without a continuum limit.
        In the top row we show results for $L=1.0$~fm, in bottom we show $L=2.0$~fm, while in different columns we show different discretization schemes.
        Even though results for $n_s = 2$ seem to be close to the continuum limit result, they start to drastically oscillate for higher energies.
        While more improved discretization schemes seem to oscillate less, they do not lay on top of the continuum result where the difference is related to the lattice spacing.
    }
\end{figure}

We note that the phase shifts for $x > 10$ start to oscillate wildly.
This is the case because energy values are close to the poles of the zeta function, e.g., close to the frame boundaries in \figref{continuum-extrapolation-spectrum}.

Furthermore it seems like the small $x$ results for $n_s = 2$ seem to be closer to the expected flat result than other discretization schemes.
This behavior can be explained by \figref{continuum-extrapolation-spectrum}.
While other discretization schemes for $x < 8$ monotonically converge against the continuum limit, $n_s = 2$ data points converge non-monotonically and are therefore closer to the continuum by accident.
In this sense it is possible to select a discretization scheme which in principle converges slower against the continuum, but has an accidental good agreement with the continuum even though it is discrete.

For small energies, better discretization schemes or small lattice spacings, we observe that the phase shifts do not oscillate and monotonically increase in $x$ with no or small curvature.
This non-flat $x$-dependence seems to depend less on the employed discretization scheme but certainly on the value of the lattice spacing.
This suggests that artifacts of the imperfect kinetic operator are negligible compared to cutoff effects of the lattice spacing itself.
The non-zero lattice spacing induces effective-range-like effects.
As we will show in the next section, this effect arises from using the continuum $S^{\spherical}$ rather than the lattice-aware $S^{\dispersion}$.

We visualize the continuum limit of phase shift points in \figref{iv-continuum}.
Similar to the case where we first extrapolated the spectrum to the continuum and computed phase shifts afterwards, the best discretization allows to also extrapolate higher excited states to zero--visible by the linear dependence of the phase shifts on epsilon.
We note that similar to the case where we first extrapolated the spectrum to the continuum, the extrapolation of the phase shifts seems to work best for the same discretization schemes in the similar energy range.
For example, while we find a linear log-log scaling region in \figref{iv-continuum} for $n_s = 2$ and $x<6$, uncertainties of the $n_s =2$ extrapolation also start to increase in \figref{continuum-extrapolation-spectrum} after $x>6$.
However the $n_s = 4$ implementation seems to be stable longer in \figref{iv-continuum} which is related to the $x>9$ state having a relatively larger continuum extrapolation uncertainty while also being close to the continuum value.


\begin{figure}[th]
    \scalebox{1.0}{\input{figure/ere-continuum-extrapolation.pgf}}
    \caption{
        \label{fig:iv-continuum}
        Continuum limit of different phase shift points computed by inserting finite lattice spacing eigenstates in $S^{\spherical}_3$ (see \figref{unimproved spherical}).
        Each column represents a different kinetic operator and each row tracks a different eigenvalue of the discrete finite-volume hamiltonian.
        Note that both axis have a log scale and thus on these scales, a linear trend for the phase shifts suggests that they extrapolate to zero.
    }
\end{figure}

\subsubsection{Dispersion L\"{u}scher removes induced artifacts}

In this section, we again attempt to tune our contact interaction to unitarity by matching the first zero of the \Luscher zeta function.
However, the difference is that at each lattice spacing we tune to that spacing's respective $S^{\dispersion N}_D$, leveraging the dispersion relation for that derivative.
Then, when we extract finite-volume and finite-spacing energy levels, we put them through the dispersion equation \todo{eqref} using the same $S$ function.\footnote{We actually make the replacement $\F_D^\epsilon\goesto\F_D$ to avoid numerically evaluating $F_D^\epsilon$.  With a controlled continuum limit the error from this change obviously vanishes.}
The numerical results of said procedure are shown in \Figref{unimproved dispersion}.
Note that the results for $p\cot\delta$ are now flat across the spectrum, matching the known result for a contact interaction.
Moreover, comparing the scale to that in, for example, \Figref{unimproved spherical}, there the deviations were of order~1, while here the results remain within $10^{-8}$ of zero, with the value entirely reflecting how well the contact interaction was tuned.

\begin{figure}[htb]
    \scalebox{0.9}{\input{figure/ere-contact-fitted_a-inv_+0.0_zeta_dispersion_projector_a1g_n-eigs_200.pgf}}
    \caption{The same as \Figref{unimproved spherical}, but tuned and subsequently analyzed using the appropriate latticized \Luscher function, matching the cutoff on the sum to the lattice scale and accounting for the dispersion relation.}
    \label{fig:unimproved dispersion}
\end{figure}

\begin{figure}[hbt]
    \scalebox{0.9}{\input{figure/ere-contact-fitted_a-inv_-5.0_zeta_dispersion_projector_a1g_n-eigs_200.pgf}}
    \caption{The same as \Figref{unimproved spherical}, but tuned and subsequently analyzed using the appropriate latticized \Luscher function, matching the cutoff on the sum to the lattice scale and accounting for the dispersion relation for finite scattering lenght.}
    \label{fig:unimproved dispersion finite a}
\end{figure}

In \Figref{dispersion running of strength} we show how the strength of the contact interaction runs with the lattice scale.  According to \todo{something we know} it should be \todo{some formula that depends on cutoff}.

\begin{figure}
    \input{figure/contact-scaling-contact-fitted_a-inv_+0.0_zeta_dispersion_projector_a1g_n-eigs_200.pgf}
    \caption{
        Scaling of the contact interaction strength $C(\epsilon)$ fitted using the dispersion method at unitarity.
        Data points are fitted values, solid lines are analytical scaling predictions following $C(\epsilon) = \frac{2}{\mu} \frac{1}{\mathcal L^{\dispersion n_\mathrm{step}}} \epsilon $ and the dashed line corresponds to the spherical predictions.
        Bar diagrams below present the absolute error between prediction and extracted value.
    }
    \label{fig:dispersion running of strength}
\end{figure}

\clearpage


%!TEX root =  ../master.tex
\section{One Dimension}\label{sec:1D}

Here we consider L\"uscher's formula in one dimension with a contact interaction.
Since the sum in the quantization condition \eqref{general luscher}or the one-loop finite-volume sum \eqref{I0 FV} with $D=1$ is convergent, the counterterm $\counterterm_{D=1}^\spherical=0$ and things are particularly simple,
\begin{align}
    C(\Lambda)
        &=
            -\frac{1}{\mu a_{10}}
    &
    \frac{a_{10}}{L}
        &=
            \frac{1}{2 \pi^{2}}
            \sum_{n=-\infty}^{\infty} \frac{1}{n^{2}-x}
        \equiv
            \frac{1}{2 \pi^{2}}
            S^\bigcirc_{1}\left(x\right)
    &
    \Bigg(x
        &=
            \frac{2\mu E L^2}{4\pi^2}\Bigg)
\end{align}
and $E$ must be a finite-volume energy on a torus of circumference $L$.
In one dimension the sum in the zeta function is well behaved and has a compact form,
\begin{equation}\label{eq:1d luscher}
S^\bigcirc_{1}(x) \equiv \sum_{n=-\infty}^{\infty} \frac{1}{n^{2}-x}=-\pi \frac{\cot (\pi \sqrt{x})}{\sqrt{x}}\ ,
\end{equation}
which gives a closed form expression for L\"uscher's formula,
\begin{equation}\label{eq:1d luscher}
\frac{a_{10}}{L} =-\frac{1}{pL}\cot\left(\frac{pL}{2}\right),
\end{equation}
consistent with those found in \cite{}.

Since there is no counterterm in one dimension, the dispersion form of L\"uscher's formula is straightforward to obtain.  If one identifies the lattice spacing as the cutoff, then the sum in the zeta function is restricted to the Brillouin zone and one has
\begin{equation}
    \frac{a_{10}}{L}
    =
    \frac{1}{2 \pi^{2}} \sum_{n=-\frac{N}{2}}^{\frac{N}{2}-1} \frac{1}{\tilde{K}^N_{nn}-x}
    =
    \frac{1}{2 \pi^{2}} \sum_{n=-\frac{N}{2}}^{\frac{N}{2}-1} \frac{1}{\frac{N^2}{4\pi^2}\left(\sum_{s}\gamma_s^{(\nstep)}\cos\frac{2\pi n s}{N}\right)-x}
    \equiv
    \frac{1}{2 \pi^{2}} S^{\dispersion}_{1}\left(x\right),\label{eq:1d dispersion luscher}
\end{equation}
where we explicitly show that the dispersion function $S^{\dispersion}$ depends on $\nstep$ and $N$ but not on $L$ or $\epsilon$ explicitly.


As stressed in the previous section, only continuum-extrapolated energies should be used in the quantization condition \eqref{1d luscher}, or induced momentum dependence terms will result.
For example, in \Figref{luescher1d} we show the induced momentum dependence terms when non-continuum eigenvalues are inserted into $S^{\spherical}_1(x)$ (colored points) for lattice sizes of $N=4$, 10, 12, and 14 and $\nstep=\infty$.
Against expectations for a contact interaction, the extracted amplitude is not flat.
However, we also show the scattering data determined through $S^{\dispersion}_1(x)$ (black points), and these all lie on a flat line, as expected.

\begin{figure}
\center
    \center
    \input{figure/1d.pgf}
%\includegraphics[width=.65\textwidth]{figure/1d.pdf}
    \caption{
        (Top panel)
        Results in 1-D using finite-spacing eigenvalues $x=2\mu EL^2/4\pi^2$ of the Schr\"odinger equation with $N=4$, 10, 12, and 14, (triangles, squares, diamonds, and hexagons, respectively) tuned so that $\tilde a_{20}/L=.1$ (closed symbols) and $\tilde a_{20}/L=0$ (open symbols).
        The colored points are obtained using $S^\spherical_1(x)$ for analysis; red, green, blue, and purple corresponding to $N=4$, 10, 12, and 14 respectively.
        The thin colored lines are the derived induced momentum-dependent terms for each $N$ as given in \Tabref{induced terms in 1 d}.
        The black points are obtained using the $N$-appropriate $S^{\dispersion}_1(x)$ and exhibit the correct flat-line behavior.
        The dashed gray line is $S_1^\spherical$, as in the bottom panel.
        (Bottom panel)
        Two one-dimensional zeta functions, the spherical function $S_1^{\spherical}$ (light gray) given in \eqref{1d luscher} and $S_1^{\dispersion}$ (red) given in \eqref{1d dispersion luscher} with $N=4$ and $\nstep=\infty$.
        The difference between the dispersion and spherical curves is responsible for moving the red triangles to the black triangles in the top panel.
        }
        \label{fig:luescher1d}
\end{figure}

We can derive the functional form of these induced momentum-dependent terms by extracting the contribution to the dispersion L\"uscher from the continuum L\"uscher.
\begin{align*}
 \frac{1}{2 \pi^{2}} S^\bigcirc_{1}\left(x\right)=\frac{1}{2 \pi^{2}} \sum_{n=-\infty}^{\infty} \frac{1}{n^{2}-x} &=\frac{1}{2 \pi^{2}} \sum_{n\in \operatorname{B.Z.}} \frac{1}{n^{2}-x} +\frac{1}{2 \pi^{2}} \sum_{n\notin \operatorname{B.Z.}} \frac{1}{n^{2}-x} \\
 &= \frac{1}{2 \pi^{2}} S^{\dispersion}_{1}\left(x\right)+\frac{1}{2 \pi^{2}} \sum_{n\notin \operatorname{B.Z.}} \frac{1}{n^{2}-x}\\
&=\frac{a_{10}}{L}+\frac{1}{2 \pi^{2}} \sum_{n\notin \operatorname{B.Z.}} \frac{1}{n^{2}-x} \ .
\end{align*}
where the energy in $x$ is a finite-spacing finite-volume $\nstep=\infty$ energy.
In the second term the sum is restricted to values of $n$ outside the Brillouin zone.
Assuming $n^2\gg x$, we can expand in powers or $x$,
\begin{equation}
    \frac{1}{2 \pi^{2}} \sum_{n=-\infty}^{\infty} \frac{1}{n^{2}-x}
        =
    \frac{a_{10}}{L}+\frac{1}{2 \pi^{2}} \sum_{n\notin \operatorname{B.Z.}} \frac{1}{n^{2}-x}
        =
    \frac{a_{10}}{L}+\frac{1}{2 \pi^{2}} \sum_{n\notin \operatorname{B.Z.}} \left(\frac{1}{n^2}+\frac{x}{n^4}+\frac{x^2}{n^6}+\mathcal{O}(x^3)\right)
\end{equation}
We make the replacement $\sum_{n\notin\operatorname{B.Z.}}\mapsto\left(\sum_{n=-\infty}^\infty-\sum_{n\in\operatorname{B.Z.}}\right)$, which allows each summation in the expansion to be determined term by term to arbitrary precision.

\begin{table}
    \caption{
    The induced momentum-dependent terms to order $x^2$ due to a contact interaction using $S^\bigcirc_1(x)$ as a function of discretization $N$.
    Here $x$ is determined by a finite-spacing finite-volume $\nstep=\infty$ eigenenergy.}
    \label{tab:induced terms in 1 d}
    \begin{tabular}{c|c}
    $N$
        &
            $\frac{1}{2 \pi^{2}} \sum_{n\notin \operatorname{B.Z.}} \left(\frac{1}{n^2}+\frac{x}{n^4}+\frac{x^2}{n^6}\right)$       \\
        \hline
      4 &    $0.052680335 + 0.00517480049 x + 9.6564782\times10^{-4} x^2$\\
    10  &   $0.020398280 + 0.00028079187 x + 7.1106444\times10^{-6} x^2$    \\
    12  &   $0.016964618 + 0.00016064496 x + 2.7825342\times10^{-6} x^2$    \\
    14  &   $0.014523490 + 0.00010045521 x + 1.2660940\times10^{-6} x^2$    \\
    \end{tabular}
\end{table}

\Tabref{induced terms in 1 d} shows these terms for $N=4$, 10, 12, and 14.
The thin colored lines in \Figref{luescher1d} correspond to the functions given in this table.  In the limit $N\goesto\infty$ all states are included in the Brillouin zone so all terms vanish and one recovers the flat, momentum-independent behavior.
Of course, analyzing the finite-spacing dispersion zeta function $S^{\dispersion}_1$ with $\nstep=\infty$ produces the flat behavior all the way through.
This demonstrates that it was not the contact operator that caused the momentum dependence, but that the dependence was induced by leveraging the continuum finite-volume formalism itself.

Finally, we draw the reader's attention to the structure of $S^\dispersion$ at any finite $N$ in the bottom panel of \Figref{luescher1d}, where the $N=4$ dispersion zeta is shown.
Note that any flat function of $x$ can only ever intercept the zeta function three times---which makes sense, as there are only three states in the parity-even sector of a one-dimensional $N=4$ lattice, $\ket{0}$, $(\ket{-1}+\ket{+1})/\sqrt{2}$ and $\ket{2}$.
If one tunes a contact interaction so that the scattering amplitude vanishes, one will see one state with $0<x<1$, one with $1<x<4$, and one state with $\abs{x}$ very large and a sign depending on whether one is slightly above or below zero numerically.
The finiteness of the parity-even sector puts constraints on the interactions that can be faithfully put onto such a small lattice: one cannot create any interaction at all whatsoever where the scattering amplitude intersects $S^{\dispersion}$ four times, because that would entail too many finite-volume states.
Of course, this is a generic feature in any number of dimensions, and the constraint ultimately vanishes in the continuum limit $N\goesto\infty$, though as $N$ increases the number of accessible $n^2$ shells grows in a dimension-dependent way.

\section{Two dimensions}
\todo{@TL: Stuff}
In 2-d we have 
\begin{equation}\label{eqn:Q1}
\cot \delta_{0}(p)-i=\frac{4}{m} \left(\frac{-1}{C_{0}(\Lambda)}+I_{0}(p, \Lambda)\right)\ ,
\end{equation}
and
\begin{equation}\label{eqn:Q2}
\cot\delta_0(p)=\frac{2}{\pi}\log(p a_0)\ ,
\end{equation}
where $a_0$ is the s-wave scattering length.  I assume a hard cutoff $\Lambda$, which can be most easily identified with the inverse lattice spacing of any numerical lattice calculation.  The term $I_0$ is
\begin{equation}\label{eqn:I0}
I_0(p,\Lambda)=\frac{1}{(2 \pi)^{2}} \int^{\Lambda} \mathrm{d}q_x\mathrm{d}q_y\ \left[\mathcal{P}\left(\frac{1}{\frac{p^2}{m}-\frac{\bm{q}^{2}}{m}}\right)-i \pi m\delta(\bm q^2-p^2)\right]\ .
\end{equation}

\subsection{Result in polar coordinates}
We first do the calculation in polar coordinates.  In this case eq.~\eqref{eqn:I0} becomes
\begin{align}
I_0(p,\Lambda)&=\frac{m}{(2 \pi)} \int^{\Lambda}_0 \mathrm{d}q \ q\ \left[\mathcal{P}\left(\frac{1}{p^2-q^{2}}\right)-i \pi \delta(q^2-p^2)\right]\\
&=\frac{m}{(2 \pi)} \int^{1}_0 \mathrm{d}\tilde q\  \tilde q\ \left[\mathcal{P}\left(\frac{1}{\tilde p^2-\tilde q^{2}}\right)-i \frac{\pi}{2|\tilde q|} \delta( \tilde q-\tilde p)\right]\\
&=\frac{m}{(2 \pi)} \int^{1}_0 \mathrm{d}\tilde q\ \mathcal{P}\left(\frac{\tilde q}{\tilde p^2-\tilde q^{2}}\right)-i \frac{m}{4}  \,\label{eqn:Q3}
\end{align}
where $\tilde q=q/\Lambda$ and $\tilde p=p/\Lambda$.  \texttt{Mathematica} tells me that this integral can be done,
\begin{align}
\frac{m}{(2 \pi)} \int^{1}_0 \mathrm{d}\tilde q\ \mathcal{P}\left(\frac{\tilde q}{\tilde p^2-\tilde q^{2}}\right)&=
\frac{m}{(2 \pi)} \log\left(\frac{\tilde p}{\sqrt{1-\tilde p^2}}\right)\\
&=\frac{m}{2 \pi} \log\left(\frac{p}{\Lambda}\right)+\frac{m}{4 \pi}\tilde p^2+\frac{m}{8 \pi}\tilde p^4+\mathcal{O}(\tilde p^6)\ ,
\end{align}
where in the second line I expanded in small $\tilde p$. Combining this result with eqs.~\eqref{eqn:Q1} and~\eqref{eqn:Q2} gives
\begin{align}
\frac{2}{\pi}\log(p a_0)&=\frac{4}{m} \left(\frac{-1}{C_{0}(\Lambda)}+\frac{m}{2 \pi} \log\left(\frac{p}{\Lambda}\right)+\mathcal{O}(\tilde p^2)\right)\nonumber\\
&=\frac{4}{m} \frac{-1}{C_{0}(\Lambda)}+\frac{2}{\pi}\log\left(\frac{p}{\Lambda}\right)+\mathcal{O}(\tilde p^2)\ ,
\end{align}
which implies that
\begin{equation}
 \frac{-1}{C_{0}(\Lambda)}=\frac{m}{2\pi}\log\left(a_0\Lambda\right)\ .
 \end{equation}

  
 \subsection{Result in cartesian coordinates}
We again start with eq.~\eqref{eqn:Q3} but express the integral in cartesian coordinates,
 \begin{equation}
 \frac{m}{(2 \pi)^2}4 \int^{1}_0 \mathrm{d}\tilde q_x\mathrm{d}\tilde q_y\ \mathcal{P}\left(\frac{1}{\tilde p^2-\tilde q_x^{2}-\tilde q_y^{2}}\right)-i \frac{m}{4} \ .
 \end{equation}
 The factor of 4 comes from using the symmetry of the kernal under $\tilde q_i\mapsto-\tilde q_i$, and therefore reducing the integration region
 \begin{displaymath}
 \int_{-1}^1d\tilde q_i\mapsto2\int_0^1d\tilde q_i\ .
 \end{displaymath}
 
The logarithmic divergence is present whether one does cartesian or polar integration.  The issue is the constant offset when doing the cartesian integration.  This offset comes from the corners of the integration of the Brillouin zone.  In 2-d, this offset is a finite number, and so we can just try to directly calculate the integral in these corners.  A little monkeying around shows that this offset is given by
\begin{equation}
\text{offset} = \left.8\int_0^{\pi/4}d\phi\int_1^{R(\phi)}dr\frac{r}{\tilde p^2-r^2}\right|_{\tilde p=0}\ .
\end{equation}
where $R(\phi)=1/\cos(\phi)$.  This expression represents the area of the box \emph{less} the contribution from the unit circle integration.  The factor of 8 is due to the fact that there are 8 or these regions that must be summed.  {\bf I should make a figure!}  Doing the first integral gives
\begin{equation}
-4\int_0^{\pi/4}d\phi \log\left(\frac{\tilde p^2-\cos(\phi)^{-2}}{\tilde p^2-1}\right)\ .
\end{equation}
Even though we cannot directly evaluate this integral, there is nothing pathologically wrong with it (i.e. no divergences).  We can do a series expansion of the kernal and integrate the expansion and get
\begin{equation}
4 \left(G-\frac{\pi}{2}\log(2)\right) - \frac{1}{2} \tilde p^2 (-2 + \pi) - 
 \frac{1}{16}\tilde p^4 (-8 + 5 \pi) +\mathcal{O}(\tilde p^6)\ .
\end{equation}
Setting $\tilde p=0$ gives us the offset.

\subsection{Determing $C_0(\Lambda)$}
Now that we have a stable counterterm, we can combine everything into the quantization condition (making sure I account for factors of $m$ and $\pi^2$, etc.) and get
\begin{align}
\frac{2}{\pi}\log(p a_0)&=\frac{4}{m} \left(\frac{-1}{C_{0}(\Lambda)}+\frac{m}{2 \pi} \log\left(\frac{p}{\Lambda}\right)+\frac{m}{\pi^2}\left(G-\frac{\pi}{2}\log(2)\right)+\mathcal{O}(\tilde p^2)\right)\nonumber\\
&=\frac{4}{m} \frac{-1}{C_{0}(\Lambda)}+\frac{2}{\pi}\log\left(\frac{p}{\Lambda}\right)+\frac{4}{\pi^2}\left(G-\frac{\pi}{2}\log(2)\right)+\mathcal{O}(\tilde p^2)\ .
\end{align}
For this equality to hold, we must have
\begin{equation}\label{eqn:ta da}
\boxed{
 \frac{-1}{C_{0}(\Lambda)}=\frac{m}{2\pi}\log\left(a_0\Lambda\right)-\frac{m}{\pi^2}\left(G-\frac{\pi}{2}\log(2)\right)
 }\ .
 \end{equation}
We display the coefficient to 15 decimal places,
\begin{displaymath}
G-\frac{\pi}{2}\log(2)=-0.1728274509745820501957409\ldots
\end{displaymath}

\section{Cartesian $S_2$}
We have
\begin{equation}
 \frac{-1}{C_{0}(\Lambda)}+\frac{1}{L^2}\sum_{q_i\in\ (-\Lambda,\Lambda]}\frac { 1 } { E - \frac{\bm{q}^2}{m} }=0\ .
 \end{equation}
 Using eq.~\eqref{eqn:ta da} gives
\begin{multline}
\frac{m}{2\pi}\log\left(a_0\Lambda\right)-\frac{m}{\pi^2}\left(G-\frac{\pi }{2}\log(2)\right)=-\frac{1}{L^2}\sum_{q_i\in\ (-\Lambda,\Lambda]} \frac { 1 } { E - \frac{\bm{q}^2}{m} }\\
\implies
\frac{2}{\pi}\log (pa_0)+\frac{2}{\pi}\log(\Lambda/p)=-\frac{4}{mL^2}\sum_{q_i\in\ (-\Lambda,\Lambda]}  \frac { 1 } { E - \frac{\bm{q}^2}{m} }+\frac{4}{\pi^2}\left(G-\frac{\pi}{2}\log(2)\right)\\
\implies
\frac{2}{\pi}\log (pa_0)=-\frac{4}{mL^2}\sum_{q_i\in\ (-\Lambda,\Lambda]}  \frac { 1 } { E - \frac{\bm{q}^2}{m} }-\frac{2}{\pi}\log(\Lambda/p)+\frac{4}{\pi^2}\left(G-\frac{\pi}{2}\log(2)\right)
\end{multline}
If we use the phase shift relation for 2-d, $\cot\delta(p)=\frac{2}{\pi}\log (pa_0)$ and replace $E=p^2/m$, then we have
\begin{align}
\cot\delta(p)&=-\frac{4}{mL^2}\sum_{q_i\in\ (-\Lambda,\Lambda]}  \frac { 1 } { \frac{p^2}{m} - \frac{\bm{q}^2}{m} }-\frac{2}{\pi}\log\left(\frac{\Lambda}{p}\right)+\frac{4}{\pi^2}\left(G-\frac{\pi }{2}\log(2)\right)\\
&=-\frac{4}{L^2}\sum_{q_i\in\ (-\Lambda,\Lambda]}  \frac { 1 } {p^2 - \bm{q}^2 }-\frac{2}{\pi}\log\left(\frac{\Lambda}{p}\right)+\frac{4}{\pi^2}\left(G-\frac{\pi }{2}\log(2)\right)\\
&=4\sum_{q_i\in\ (-\Lambda,\Lambda]}  \frac { 1 } {(\bm{q}L)^2-(pL)^2 }-\frac{2}{\pi}\log\left(\frac{\Lambda L}{pL}\right)+\frac{4}{\pi^2}\left(G-\frac{\pi }{2}\log(2)\right)\\
&=\frac{1}{\pi^2}\sum_{q_i\in\ (-\Lambda,\Lambda]}  \frac { 1 } {\left(\frac{\bm{q}L}{2\pi}\right)^2-\left(\frac{pL}{2\pi}\right)^2 }-\frac{2}{\pi}\log\left(\frac{\Lambda L}{pL}\right)+\frac{4}{\pi^2}\left(G-\frac{\pi }{2}\log(2)\right).
\end{align}
We now define the \emph{cartesian} $S_2$\label{eqn:S2}\footnote{For comparison, the \emph{spherical} (polar) $S_2$ function is
\begin{equation}\label{eqn:S2 polar}
S_2(x)\equiv\sum_{|\bm n|\le\Lambda'}\frac { 1 } { \bm{n}^2 -x}-2\pi\log\left(\Lambda'\right)\ .
\end{equation}}
\begin{align}
S_2(x)\equiv&\sum_{n_i\in\ (-\Lambda',\Lambda']}\frac { 1 } { \bm{n}^2 -x}-2\pi\log\left(\Lambda'\right)+4\left(G-\frac{\pi }{2}\log(2)\right)\\
=&\sum_{n_i\in\ (-\Lambda',\Lambda']}\frac { 1 } { \bm{n}^2 -x}-2\pi\log\left(\mathcal{L}_\square\Lambda'\right)\ ,
\end{align}
where $\Lambda'=\frac{\Lambda L}{2\pi} \in\ \mathbb{N}$ and
\begin{equation}
\mathcal{L}_\square=\exp\left(\log(2)-G\frac{2}{\pi}\right)=1.116306393581637659468497\ldots
\end{equation}
So we finally have
\begin{equation}\label{eqn:S2}
\cot\delta(p)=\frac{1}{\pi^2}S_2\left(\left(\frac{p L}{2\pi}\right)^2\right)+\frac{2}{\pi}\log\left(\frac{pL}{2\pi}\right)\ .
\end{equation}
Because of the logarithmic dependence in $p$ of the phase shift relation in 2-d, it is instead more convenient to move the logarithmic term on the RHS of the above equation to the LHS, which cancels the logarithmic dependence on $p$,
\begin{equation}
\frac{2}{\pi}\log\left(\frac{2\pi a_0 }{L}\right)=\frac{1}{\pi^2}S_2\left(\left(\frac{p L}{2\pi}\right)^2\right)\ .
\end{equation}

\subsection{Zero-range contact interaction}

\subsection{Finite-range interaction}
Here we use a separable potential of the form
\begin{equation}
V(\bm p',\bm p)=\frac{4}{m}\mathcal{C}f(\bm p')f(\bm p)\ .
\end{equation}
One can show that the discrete box eigenvalues $x=\frac{mL^2E}{4\pi^2}$ satisfy the self-consistency equation
\begin{equation}\label{eqn:SE}
\frac{1}{\mathcal{C}}+\frac{1}{\pi^2}\sum_{\bm n\in\mathrm{B.Z.}}\frac{f\left(\frac{2\pi}{L}\bm n\right)^2}{\bm n^2-x}=0\ .
\end{equation}
where $\bm n\in\mathrm{B.Z.}$ represents $\bm n\in(-N/2,N/2]^2$ and $N$ is the number of sites per side of the box.  Note that $\mathcal{C}$ is dimensionless in 2-d.  With a little math and manipulation, the phase shift can be determined with this separable interaction,
\begin{equation}\label{eqn:cot d}
\cot \delta(p) = \frac{1}{f(p)^2}\left(-\frac{1}{\mathcal{C}}+\frac{2}{\pi}\int_0^\infty dq \ \mathcal{P}\frac{qf(q)^2}{p^2-q^2}\right)\ .
\end{equation}

\subsubsection{Specific potentials}
Our first example uses the following separable potential,
\begin{equation}\label{eqn:potential1}
f(p)=\frac{1}{\sqrt{1+(p/\Lambda)^2}}\ .
\end{equation}
In this case one has
\begin{equation}
\frac{2}{\pi}\int_0^\infty dq \ \mathcal{P}\frac{qf(q)^2}{p^2-q^2}=\frac{2}{\pi}\int_0^\infty dq \ \mathcal{P}\frac{q}{(p^2-q^2)(1+(p/\Lambda)^2)}=\frac{2}{\pi}\frac{\log\left(\frac{p}{\Lambda}\right)}{1+(p/\Lambda)^2}\ .
\end{equation}
Equation~\eqref{eqn:cot d} becomes
\begin{equation}\label{eqn:first phase shift}
\cot\delta(p)=-\frac{1+(p/\Lambda)^2}{\mathcal{C}}+\frac{2}{\pi}\log\left(\frac{p}{\Lambda}\right)\ .
\end{equation}
Without loss of generality, we trade the dimensionless coefficient $\mathcal{C}$ with a dimensional (length) parameter $a_0$ via the relation
\begin{equation}
-\frac{1}{\mathcal{C}}=\frac{2}{\pi}\log(a_0\Lambda)\ .
\end{equation} 
Equation~\eqref{eqn:first phase shift} becomes
\begin{equation}
\cot\delta(p)=\frac{2}{\pi}\log(a_0p)+\frac{p^2}{2}\left[\frac{4\log(a_0\Lambda)}{\pi\Lambda^2}\right]\ .
\end{equation}

So we tested this potential and L\"uscher's formula is working.  Here we use the parameters
\begin{align*}
a_0&=2 \\
L&=10 \\
N&=100\\
\Lambda&= 5\\
\cot\delta(p)&=\frac{2}{\pi} \log (2 p)+0.0586348 \ p^2\ .
\end{align*}
With these parameters we determined the eigenenergies $x$ by numerically finding the roots of eq.~\eqref{eqn:SE} and then feeding these values through eq.~\eqref{eqn:S2}. The left panel of \autoref{fig:cotd1} shows these results.  The right panel uses $a_0=1$, which in this case supports negative $x$ solution.
\begin{figure}
\center
\includegraphics[width=.4925\textwidth]{figure/cotd1.pdf}\includegraphics[width=.5075\textwidth]{figure/cotd4.pdf}
\caption{Dashed line is the exact result, the red dots are numerical results.  The potential in this case is given by eq.~\eqref{eqn:potential1}.  Left panel does not support a negative energy solution ($2\pi a_0/L>1$), right panel does ($2\pi a_0/L<1$).\label{fig:cotd1}}
\end{figure}
The agreement between L\"uscher and exact is pretty good.  We have confirmed that as I increase $N$, the points converge to the exact line.  

Our second example uses
\begin{equation}\label{eqn:potential2}
f(p)=\frac{1}{(1+(p/\Lambda)^2)^2}\ .
\end{equation}
Setting
\begin{equation}
\frac{-1}{\mathcal{C}}=\frac{2 \log (a \Lambda )}{\pi }-\frac{11}{6 \pi }\ ,
\end{equation}
we find 
\begin{equation}
\cot \delta(p)=\frac{2 \log (a p)}{\pi }+\frac{p^2 (24 \log (a \Lambda )-13)}{3 \pi  \Lambda ^2}+\frac{p^4 (24 \log (a \Lambda )-19)}{2 \pi  \Lambda
   ^4}
  +\frac{p^6 (8 \log (a \Lambda
   )-7)}{\pi  \Lambda ^6} -\frac{p^8 (11-12 \log (a \Lambda ))}{6 \pi  \Lambda ^8}\ .
\end{equation}
Using the following parameters,
\begin{align*}
a_0&=2 \\
L&=10 \\
N&=100\\
\Lambda&= 5\ ,
\end{align*}
We find good agreement between L\"uscher and the exact result, as shown in the left panel of~\autoref{fig:cotd23}.
\begin{figure}
\center
\includegraphics[width=.485\textwidth]{figure/cotd2.pdf}\includegraphics[width=.515\textwidth]{figure/cotd3.pdf}
\caption{Dashed line is the exact result, the red dots are numerical results.  The potential in this case is given by eq.~\eqref{eqn:potential2}.  Left panel does not support a negative energy solution ($2\pi a_0/L>1$), right panel does ($2\pi a_0/L<1$).\label{fig:cotd23}}
\end{figure}
We also reduce the scattering length such that it supports a negative energy solution, $a_0=1$.  The right panel of~\autoref{fig:cotd23} shows this case.  Again, there is good agreement between exact and L\"uscher.

\section{Conclusion}\label{sec:conclusion}

We are smart.

LQCD people should care about cartesian \Luscher.  Does it explain HALQCD against the world?

Calculations of the Bertsch parameter should take advantage of the tuning that can be done with the dispersion function.

\section*{Acknowledgements}\label{sec:acknowledgements}

The authors thank
Tom Cohen,
Ben H\"{o}rz,
Colin Morningstar,
Andr\'{e} Walker-Loud,
Ken McElvain
and
Jan-Lukas Wynen
for stimulating discussion, feedback, and technical help during the course of this work.
C.K. gratefully acknowledges funding through the Alexander von Humboldt Foundation through a Feodor Lynen Research Fellowship.
This work was done in part through financial support from the Deutsche Forschungsgemeinschaft (Sino-German CRC 110).
\todo{Thanks for computing time at \$FACILITY. NERSC?}


\appendix
\section{Misc}

We can put the derivation of the counter terms etc. in appendices.

Do we want to put tables?  Make numerical data directly available?  We should be good, reproducible scientists.

\section{Dispersion Relation Coefficients}\label{sec:coefficients}

In \eqref{gamma definition} and \eqref{gamma determination} we give the definition and how to determine the $\gamma_s^{(\nstep)}$ coefficients that give us finite difference formulas.  Here we collect and list them for convenience.

\begin{table}[ht]
    \caption{Values for $\gamma_s^{(\nstep)}$ for a variety of different $\nstep$s that give the optimal approximation $\omega^{(\nstep)}(p,\epsilon) = (\epsilon p)^2\left[1+ \order{(\epsilon p)^{2 \nstep}}\right]$.}
    \label{tab:dispersion coefficients}
    \begin{tabular}{cccccc}
        $\gamma_s^{(\nstep)}$   &   $s=0$   &   $s=1$   &   $s=2$   &   $s=3$       &   $s=4$   \\
        $\nstep=1$              &   $2$     &   $-2$    &           &               &           \\
        $\nstep=2$              &   $5/2$   &   $-8/3$  &   $1/6$   &               &           \\
        $\nstep=3$              &   $49/18$ &   $-3$    &   $3/10$  &   $-1/45$     &           \\
        $\nstep=4$              &   $205/72$&   $-16/5$ &   $2/5$   &   $-16/315$   &   $1/280$
    \end{tabular}
\end{table}

\section{The Usual, Spherical \Luscher's Formula}\label{sec:spherical}


Here we present a $D$-dimensional derivation of \Luscher's formula that roughly follows \Ref{Beane:2003da}, although the technology and sophistication of the finite-volume formalism has grown substantially \todo{cite cite cite}.  Assuming an interaction given by an tower of derivative contact operators
\begin{equation}
    V(p) = +\sum_n C_{2n}(\Lambda) p^{2n}
\end{equation}
where the interaction strengths depend on the regulator and carry spatial-dimension-dependent units.
The scattering amplitude is given by the bubble sum depicted in \Figref{bubbleSum}.

\begin{figure}[ht!]
\center
\includegraphics[width=.8\columnwidth]{figure/bubbleSum.pdf}
\caption{Bubble sum. Each line represents a propagator, each vertex represents $-i \sum_n C_{2n}(\Lambda) p^{2n}$, and the bubble is given by $I_0$ (see also \Figref{I0}).\label{fig:bubbleSum}}
\end{figure}

This bubble sum is a geometric series and gives\cite{Kaplan:1998we,Beane:2003da}
\begin{equation}\label{eq:scattering amplitude}
\amplitude = \frac{-\sum_n C_{2n}(\Lambda) p^{2n}}{1-I_0(p,\Lambda) \sum_n C_{2n}(\Lambda) p^{2n}},
\end{equation}
where $p$ is the relative momentum,  and $I_0(p,\Lambda)$ is a $D$-dependent function that arises from integrating the loop shown in \Figref{I0},
\begin{align}
    I_0(p)
    &=-i\int^{\Lambda/2}
        \frac { \mathrm {d}q_0}{2\pi}\ \frac{\mathrm { d } ^ { D } \vec{ q } } { (2\pi)^ { D } }
        \left( \frac { i } { \frac{E}{2} + q _ { 0 } - \frac{\vec{q}^2}{2m} + i \epsilon } \right)
        \left( \frac { i } { \frac{E}{2} - q _ { 0 } - \frac{\vec{q}^2}{2m} + i \epsilon } \right)
    \nonumber\\
    &=\frac{\Omega_D}{(2\pi)^D}\int^{\Lambda/2}  \mathrm { d } q \ q^{D-1}\left[\mathcal{P} \left( \frac { 1 } { E - \frac{\vec{q}^2}{m} } \right)
-i\frac{\pi m}{2q}\delta(q-\sqrt{mE})\right]
    \\
    &=\frac{\Omega_D}{(2\pi)^2}\frac{m}{L^{D-2}}\int^{\Lambda L/4\pi}  \mathrm { d } n \ n^{D-1}\left[\mathcal{P} \left( \frac { 1 } { \left(\frac{pL}{2\pi}\right)^2 - n^2 } \right)
-i\frac{\pi^2}{L n}\delta\left(\frac{2\pi}{L}n -p\right)\right]
    \label{eq:I0}
\end{align}
where $\mathcal{P}$ refers to Principle (Cauchy) Value, we have used the on-shell condition $mE=p^2$, and the geometric factor
\begin{equation}
\Omega_D=\frac{2\pi^{D/2}}{\Gamma(D/2)}=
    \begin{cases}
        4\pi    &   (D=3)\\
        2\pi    &   (D=2)\\
        2       &   (D=1)
    \end{cases}\ ,
\end{equation}

\begin{figure}[h!]
\center
\includegraphics[width=.35\columnwidth]{figure/I0.eps}
\caption{Loop diagram contributing to $I_0$.\label{fig:I0} \todo{Make it a tower of interactions!  Also this figure is way too big cf the font.}}
\end{figure}

In the $s$-wave, the momentum-dependent scattering amplitude is related to the phase shift $\delta_0(p)$ by
\begin{equation}\label{eq:cot delta}
    \amplitude = \frac{4}{m}\F_d\frac{1}{\cot \delta_0(p)-i}\ ,
\end{equation}
where
\begin{equation}
    \F_D
    =
    \begin{cases}
        \pi/p   & (D=3)\\
        1       & (D=2)\\
        p/2     & (D=1)
\end{cases}
\end{equation}
is a dimension-dependent kinematic factor.
This fixes the coefficients $C(\Lambda)$ as a function of the scattering data,
\begin{equation}
    \frac{1}{\sum_n C_{2n}(\Lambda) p^{2n}}
    =
    I_0(p) + \frac{m}{4 \F_D}\left(\cot \delta_0(p) - i\right)
\end{equation}

In a finite volume, the energy eigenstates cause the amplitude to diverge, so that
\begin{equation}
    \frac{1}{\sum_n C_{2n}(\Lambda) p^{2n}} - I_{0,\FV}(p,L) = 0
\end{equation}
and the infinite-volume integral $I_0$ has been replaced by the matching finite-volume sum,
\begin{align}
I_{0,\FV}(p,L)
    &=-i\int \frac { \mathrm {d}q_0}{2\pi} \frac{1}{L^D}\sum_{\vec{q}}^{q < \Lambda/2} \left( \frac { i } { \frac{E}{2} + q _ { 0 } - \frac{\vec{q}^2}{2m} + i \epsilon } \right) \left( \frac { i } { \frac{E}{2} - q _ { 0 } - \frac{\vec{q}^2}{2m} + i \epsilon } \right)
    \\
    &=\frac{1}{L^D}\sum_{\vec{q}}^{q < \Lambda/2} \frac { 1 } { E - \frac{\vec{q}^2}{m} }
    =\frac{m}{(2\pi)^2 L^{D-2}} \sum_{\vec{n}}^{n < \frac{\Lambda L}{4\pi}} \frac{1}{x-n^2}
    &
    x &= \left( \frac{pL}{2\pi}\right)^2
\end{align}
where we have used the on-shell condition $mE=p^2$.

Combining the infinite-volume result with the finite-volume result, one finds
\begin{equation}
    \frac{1}{4\F_D}\left(\cot \delta_0(p) - i\right) = \frac{1}{(2\pi)^2 L^{D-2}}\left[ \left(\sum_n- \int_n\right) \frac{1}{x-n^2} + \frac{-i \pi^2\Omega_D}{L} \int \mathrm{d}n\ n^{D-2} \delta\left(\frac{2\pi}{L}n - p\right) \right]
\end{equation}
where both the sum and integral are cut off by a restriction on the magnitude of $n$, $n^2 < (\Lambda L / 4\pi)^2$, and the integral implicitly carries a factor of $\Omega_D n^{D-1}$.
In a seemingly miraculous (but required) cancellation, the imaginary part on the left hand side exactly cancels the last term in the sum on the right, and we are left with
\begin{equation}
    p \cot \delta_0(p) = \frac{\F_D\ p}{\pi^2 L^{D-2}} \left(\sum_n-\int_n\right) \frac{1}{x-n^2}
\end{equation}
where $x=(pL/2\pi)^2$.
Because we cut off the sum and the integral in exactly the same way, in dimensions where $I_0$ diverges with $\Lambda$, the divergence cancels against the divergence in the sum.
Let $N=\Lambda L/2\pi$.
Then, defining, with a finite cutoff $N$,
\begin{equation}\label{eq:spherical cutoff S}
    S^{\spherical N}_D(x) = \left(\sum_n- \int_n\right) \frac{1}{x-n^2}
\end{equation}
where the $\spherical$ superscript reminds us that we cut off our sum and integral in a spherical way, based on the magnitude of $n<N$, we recover the usual \Luscher zeta functions by taking
\begin{equation}
    S^\spherical_D(x)
    =
    \lim_{N\goesto\infty} S^{\spherical N}_D(x)
    =
    \lim_{N\rightarrow\infty}\left( \sum_n^{n < N/2} \frac{1}{x-n^2} - \counterterm_D^\spherical \left(\frac{N}{2}\right)^{D-2}\right)
\end{equation}
where dimension-dependent counterterm $\counterterm_D^\spherical$ comes from the integral; we evaluate said counterterms in \Appref{counterterm/spherical}.
\todo{I MUST HAVE LOST A SIGN SOMEWHERE, IT SHOULD BE $n^2-x$.}
Finally,
\begin{equation}\label{eq:spherical quantization}
    p \cot \delta_0(p) = \frac{\F_D\ p}{\pi^2 L^{D-2}} S^\spherical_D(x).
\end{equation}
This is the usual \Luscher finite-volume quantization condition.

\todo{STILL REMAINING:}
\begin{itemize}
    \item ERE / matching
    \item Describe tuning
    \item Show results in 2, 3D
    \item Something is wrong :(?  More like :D because we are smart!
\end{itemize}

\section{The Cartesian \Luscher Finite-Volume Counterterm}\label{sec:counterterm/cartesian}

The regulation of the sum in \eqref{cartesian S} is provided by
\begin{equation}
    \label{eq:cartesian S counterterm}
    \int_{-N/2}^{N/2} \mathrm{d}^Dn \frac{1}{n^2-x}
    =
    \left(\frac{N}{2}\right)^{D-2} \int_{-1}^{+1} \mathrm{d}^D\nu \frac{1}{\nu^2 - \tilde{x}}
\end{equation}
where we rescaled and $\tilde{x}=x/(N/2)$, and we can numerically evaluate the right-hand side with ease to achieve a complete improvement of the $S_D^{\cartesian N}$ function.

We can determine the counterterm in the three-dimensional Cartesian volume,evaluating the large-$N$ limit of the integral
\begin{equation}\label{eq:cartesian integral}
    \int_{-N/2}^{+N/2} \mathrm{d}n_x\ \mathrm{d}n_y\ \mathrm{d}n_z \frac{1}{n^2}
\end{equation}
where $n^2 = n_x^2+n_y^2+n_z^2$ and we henceforth suppress the $(x=0)$ argument.
We can evaluate it as follows.  Consider the integral
\begin{equation}
	J^{\cartesian N}_0(x, m) = \int_{-N/2}^{+N/2} dn_x\ dn_y\ dn_z\ \frac{e^{-m n^2}}{n^2}.
\end{equation}
Then, we know $\lim_{m\goesto\infty} J^{\cartesian N}_0(m) = 0$ and $J^{\cartesian N}_0(0)$ is the integral on the right-hand-side of \eqref{cartesian integral}.
We can take advantage of the fundamental theorem of calculus,
\begin{align}
	\int_{-N/2}^{+N/2} \mathrm{d}^3n \frac{1}{n^2}
    &=
    J^{\cartesian N}_0(0) - J^{\cartesian N}_0(\infty)
		&&= 	\int_{\infty}^{0} dm\ \partial_m J^{\cartesian N}_0(m)
		% \\
		&&=	\int_{\infty}^{0} dm\ \int_{-N/2}^{+N/2} \mathrm{d}^3n -e^{-m n^2}
		\nonumber\\
		&&&=	\int_0^\infty dm\ \left(\int_{-N/2}^{+N/2} dn\ e^{-m n^2}\right)^3
		% \\
		&&=	\int_0^\infty dm\ \left(\sqrt{\frac{\pi}{m}} \erf(\sqrt{m N^2/4})\right)^3
\end{align}
where we exchanged the order of partial differentiation by $m$ and momenta integration, recognize the resulting integral as three identical copies of the same integral (as long as the volume is a cube), and evaluate it, yielding the error function.
Changing variables to isolate the dependence on $N$, $n^2 = m N^2/4$, we find
\begin{align}
    \int_{-N/2}^{+N/2} \mathrm{d}^3n \frac{1}{n^2}
    &=
    \pi^{3/2}\ \frac{N}{2} \int_{0}^\infty 2n\ \mathrm{d}n\ \left(\frac{\erf(n)}{n}\right)^3
\end{align}
and the integral can be easily evaluated numerically, yielding $2.75634$ so that
\begin{equation}
    \label{eq:cartesian counterterm}
    \counterterm^\cartesian_3 \left(\frac{N}{2}\right) = \lim_{N\goesto\infty}\int_{-N/2}^{+N/2} \mathrm{d}^3n \frac{1}{n^2} = 15.34824844488746404710\ \left(\frac{N}{2}\right)
\end{equation}
and more digits are readily available; this constant appears in \eqref{cartesian S}.
This can be compared to the spherical counterterm, which we can read off from \eqref{improved spherical S}, where the counter term can be seen to be $4\pi \approx 12.6 $; that the Cartesian result is larger reflects the fact that more of the momentum space is included in the integration domain for a fixed $N$, as discussed in \Secref{cartesian}.

This method of evaluation can be repeated for higher spatial dimensions.
For $D\geq3$ spatial dimensions one finds
\begin{equation}
    \pi^{D/2} \left(\frac{N}{2}\right)^{D-2} \int_0^\infty 2\mu\ \mathrm{d}\mu\ \left(\frac{\erf(\mu)}{\mu}\right)^D
\end{equation}
For four dimensions one finds $17.14741624920737 (N/2)^2$, $24.49922817921121 (N/2)^3$ for five dimensions, for six dimensions $38.50096808074375 (N/2)^4$, and so on.  However, in these higher dimensions we must also subtract subleading divergences, which requires the determination of counterterms we do not here compute.

The logarithmic divergence in two dimensions must be handled especially carefully.
\todo{TOM'S MAGIC INTEGRAL AND CATALAN'S CONSTANT}.

In fact, we can execute a similar construction in three (and higher) dimensions.  In three dimensions, \todo{TOM'S G+POLYLOG MAGIC}.

\subsubsection{Dispersion L\"{u}scher removes induced artifacts}

In this section, we again attempt to tune our contact interaction to unitarity by matching the first zero of the \Luscher zeta function.
However, the difference is that at each lattice spacing we tune to that spacing's respective $S^{\dispersion N}_D$, leveraging the dispersion relation for that derivative.
Then, when we extract finite-volume and finite-spacing energy levels, we put them through the dispersion equation \todo{eqref} using the same $S$ function.\footnote{We actually make the replacement $\F_D^\epsilon\goesto\F_D$ to avoid numerically evaluating $F_D^\epsilon$.  With a controlled continuum limit the error from this change obviously vanishes.}
The numerical results of said procedure are shown in \Figref{unimproved dispersion}.
Note that the results for $p\cot\delta$ are now flat across the spectrum, matching the known result for a contact interaction.
Moreover, comparing the scale to that in, for example, \Figref{unimproved spherical}, there the deviations were of order~1, while here the results remain within $10^{-8}$ of zero, with the value entirely reflecting how well the contact interaction was tuned.

\begin{figure}[htb]
    \scalebox{0.9}{\input{figure/ere-contact-fitted_a-inv_+0.0_zeta_dispersion_projector_a1g_n-eigs_200.pgf}}
    \caption{The same as \Figref{unimproved spherical}, but tuned and subsequently analyzed using the appropriate latticized \Luscher function, matching the cutoff on the sum to the lattice scale and accounting for the dispersion relation.}
    \label{fig:unimproved dispersion}
\end{figure}

\begin{figure}[hbt]
    \scalebox{0.9}{\input{figure/ere-contact-fitted_a-inv_-5.0_zeta_dispersion_projector_a1g_n-eigs_200.pgf}}
    \caption{The same as \Figref{unimproved spherical}, but tuned and subsequently analyzed using the appropriate latticized \Luscher function, matching the cutoff on the sum to the lattice scale and accounting for the dispersion relation for finite scattering lenght.}
    \label{fig:unimproved dispersion finite a}
\end{figure}

In \Figref{dispersion running of strength} we show how the strength of the contact interaction runs with the lattice scale.  According to \todo{something we know} it should be \todo{some formula that depends on cutoff}.

\begin{figure}
    \input{figure/contact-scaling-contact-fitted_a-inv_+0.0_zeta_dispersion_projector_a1g_n-eigs_200.pgf}
    \caption{
        Scaling of the contact interaction strength $C(\epsilon)$ fitted using the dispersion method at unitarity.
        Data points are fitted values, solid lines are analytical scaling predictions following $C(\epsilon) = \frac{2}{\mu} \frac{1}{\mathcal L^{\dispersion n_\mathrm{step}}} \epsilon $ and the dashed line corresponds to the spherical predictions.
        Bar diagrams below present the absolute error between prediction and extracted value.
    }
    \label{fig:dispersion running of strength}
\end{figure}

\clearpage

\section{Comparison in Two Dimensions}\label{sec:two-d}

Here we do the whole thing for 2D.

%% Creator: Matplotlib, PGF backend
%%
%% To include the figure in your LaTeX document, write
%%   \input{<filename>.pgf}
%%
%% Make sure the required packages are loaded in your preamble
%%   \usepackage{pgf}
%%
%% Figures using additional raster images can only be included by \input if
%% they are in the same directory as the main LaTeX file. For loading figures
%% from other directories you can use the `import` package
%%   \usepackage{import}
%% and then include the figures with
%%   \import{<path to file>}{<filename>.pgf}
%%
%% Matplotlib used the following preamble
%%   \usepackage{fontspec}
%%
\begingroup%
\makeatletter%
\begin{pgfpicture}%
\pgfpathrectangle{\pgfpointorigin}{\pgfqpoint{6.522725in}{7.116719in}}%
\pgfusepath{use as bounding box, clip}%
\begin{pgfscope}%
\pgfsetbuttcap%
\pgfsetmiterjoin%
\definecolor{currentfill}{rgb}{1.000000,1.000000,1.000000}%
\pgfsetfillcolor{currentfill}%
\pgfsetlinewidth{0.000000pt}%
\definecolor{currentstroke}{rgb}{1.000000,1.000000,1.000000}%
\pgfsetstrokecolor{currentstroke}%
\pgfsetdash{}{0pt}%
\pgfpathmoveto{\pgfqpoint{0.000000in}{-0.000000in}}%
\pgfpathlineto{\pgfqpoint{6.522725in}{-0.000000in}}%
\pgfpathlineto{\pgfqpoint{6.522725in}{7.116719in}}%
\pgfpathlineto{\pgfqpoint{0.000000in}{7.116719in}}%
\pgfpathclose%
\pgfusepath{fill}%
\end{pgfscope}%
\begin{pgfscope}%
\pgfsetbuttcap%
\pgfsetmiterjoin%
\definecolor{currentfill}{rgb}{1.000000,1.000000,1.000000}%
\pgfsetfillcolor{currentfill}%
\pgfsetlinewidth{0.000000pt}%
\definecolor{currentstroke}{rgb}{0.000000,0.000000,0.000000}%
\pgfsetstrokecolor{currentstroke}%
\pgfsetstrokeopacity{0.000000}%
\pgfsetdash{}{0pt}%
\pgfpathmoveto{\pgfqpoint{0.702340in}{6.271734in}}%
\pgfpathlineto{\pgfqpoint{1.925444in}{6.271734in}}%
\pgfpathlineto{\pgfqpoint{1.925444in}{6.879682in}}%
\pgfpathlineto{\pgfqpoint{0.702340in}{6.879682in}}%
\pgfpathclose%
\pgfusepath{fill}%
\end{pgfscope}%
\begin{pgfscope}%
\pgfpathrectangle{\pgfqpoint{0.702340in}{6.271734in}}{\pgfqpoint{1.223103in}{0.607948in}}%
\pgfusepath{clip}%
\pgfsetbuttcap%
\pgfsetmiterjoin%
\definecolor{currentfill}{rgb}{0.000000,0.000000,1.000000}%
\pgfsetfillcolor{currentfill}%
\pgfsetfillopacity{0.100000}%
\pgfsetlinewidth{0.803000pt}%
\definecolor{currentstroke}{rgb}{0.000000,0.000000,1.000000}%
\pgfsetstrokecolor{currentstroke}%
\pgfsetstrokeopacity{0.100000}%
\pgfsetdash{}{0pt}%
\pgfpathmoveto{\pgfqpoint{0.702340in}{6.495896in}}%
\pgfpathlineto{\pgfqpoint{0.702340in}{6.698953in}}%
\pgfpathlineto{\pgfqpoint{1.925444in}{6.698953in}}%
\pgfpathlineto{\pgfqpoint{1.925444in}{6.495896in}}%
\pgfpathclose%
\pgfusepath{stroke,fill}%
\end{pgfscope}%
\begin{pgfscope}%
\pgfpathrectangle{\pgfqpoint{0.702340in}{6.271734in}}{\pgfqpoint{1.223103in}{0.607948in}}%
\pgfusepath{clip}%
\pgfsetbuttcap%
\pgfsetroundjoin%
\definecolor{currentfill}{rgb}{0.000000,0.501961,0.000000}%
\pgfsetfillcolor{currentfill}%
\pgfsetfillopacity{0.500000}%
\pgfsetlinewidth{0.803000pt}%
\definecolor{currentstroke}{rgb}{0.000000,0.501961,0.000000}%
\pgfsetstrokecolor{currentstroke}%
\pgfsetstrokeopacity{0.500000}%
\pgfsetdash{}{0pt}%
\pgfpathmoveto{\pgfqpoint{0.702340in}{6.689068in}}%
\pgfpathlineto{\pgfqpoint{0.702340in}{6.501252in}}%
\pgfpathlineto{\pgfqpoint{0.943050in}{6.508598in}}%
\pgfpathlineto{\pgfqpoint{1.054581in}{6.511566in}}%
\pgfpathlineto{\pgfqpoint{1.127977in}{6.510440in}}%
\pgfpathlineto{\pgfqpoint{1.182772in}{6.505456in}}%
\pgfpathlineto{\pgfqpoint{1.226515in}{6.496801in}}%
\pgfpathlineto{\pgfqpoint{1.262923in}{6.484628in}}%
\pgfpathlineto{\pgfqpoint{1.294107in}{6.469052in}}%
\pgfpathlineto{\pgfqpoint{1.321380in}{6.450160in}}%
\pgfpathlineto{\pgfqpoint{1.345614in}{6.428015in}}%
\pgfpathlineto{\pgfqpoint{1.367419in}{6.402657in}}%
\pgfpathlineto{\pgfqpoint{1.387238in}{6.372843in}}%
\pgfpathlineto{\pgfqpoint{1.405403in}{6.334025in}}%
\pgfpathlineto{\pgfqpoint{1.422168in}{6.293362in}}%
\pgfpathlineto{\pgfqpoint{1.437735in}{6.250820in}}%
\pgfpathlineto{\pgfqpoint{1.452262in}{6.206339in}}%
\pgfpathlineto{\pgfqpoint{1.465881in}{6.159845in}}%
\pgfpathlineto{\pgfqpoint{1.478698in}{6.111253in}}%
\pgfpathlineto{\pgfqpoint{1.490802in}{6.060468in}}%
\pgfpathlineto{\pgfqpoint{1.502269in}{6.007388in}}%
\pgfpathlineto{\pgfqpoint{1.513162in}{5.951898in}}%
\pgfpathlineto{\pgfqpoint{1.523536in}{5.893877in}}%
\pgfpathlineto{\pgfqpoint{1.533438in}{5.833193in}}%
\pgfpathlineto{\pgfqpoint{1.542909in}{5.769710in}}%
\pgfpathlineto{\pgfqpoint{1.551986in}{5.703273in}}%
\pgfpathlineto{\pgfqpoint{1.560699in}{5.633343in}}%
\pgfpathlineto{\pgfqpoint{1.569077in}{5.555548in}}%
\pgfpathlineto{\pgfqpoint{1.577145in}{5.473675in}}%
\pgfpathlineto{\pgfqpoint{1.584924in}{5.388164in}}%
\pgfpathlineto{\pgfqpoint{1.592435in}{5.298985in}}%
\pgfpathlineto{\pgfqpoint{1.599696in}{5.206099in}}%
\pgfpathlineto{\pgfqpoint{1.606722in}{5.109466in}}%
\pgfpathlineto{\pgfqpoint{1.613528in}{5.009042in}}%
\pgfpathlineto{\pgfqpoint{1.620129in}{4.904784in}}%
\pgfpathlineto{\pgfqpoint{1.626535in}{4.796648in}}%
\pgfpathlineto{\pgfqpoint{1.632758in}{4.684593in}}%
\pgfpathlineto{\pgfqpoint{1.638808in}{4.568575in}}%
\pgfpathlineto{\pgfqpoint{1.644694in}{4.448554in}}%
\pgfpathlineto{\pgfqpoint{1.650426in}{4.324493in}}%
\pgfpathlineto{\pgfqpoint{1.656011in}{4.196359in}}%
\pgfpathlineto{\pgfqpoint{1.661456in}{4.064123in}}%
\pgfpathlineto{\pgfqpoint{1.666768in}{3.927767in}}%
\pgfpathlineto{\pgfqpoint{1.671953in}{3.787280in}}%
\pgfpathlineto{\pgfqpoint{1.677018in}{3.642664in}}%
\pgfpathlineto{\pgfqpoint{1.681968in}{3.493937in}}%
\pgfpathlineto{\pgfqpoint{1.686808in}{3.341130in}}%
\pgfpathlineto{\pgfqpoint{1.691543in}{3.184295in}}%
\pgfpathlineto{\pgfqpoint{1.696176in}{3.023475in}}%
\pgfpathlineto{\pgfqpoint{1.700714in}{2.858465in}}%
\pgfpathlineto{\pgfqpoint{1.705158in}{2.682727in}}%
\pgfpathlineto{\pgfqpoint{1.705158in}{2.691832in}}%
\pgfpathlineto{\pgfqpoint{1.705158in}{2.691832in}}%
\pgfpathlineto{\pgfqpoint{1.700714in}{2.866721in}}%
\pgfpathlineto{\pgfqpoint{1.696176in}{3.042794in}}%
\pgfpathlineto{\pgfqpoint{1.691543in}{3.213475in}}%
\pgfpathlineto{\pgfqpoint{1.686808in}{3.378542in}}%
\pgfpathlineto{\pgfqpoint{1.681968in}{3.538044in}}%
\pgfpathlineto{\pgfqpoint{1.677018in}{3.692056in}}%
\pgfpathlineto{\pgfqpoint{1.671953in}{3.840656in}}%
\pgfpathlineto{\pgfqpoint{1.666768in}{3.983924in}}%
\pgfpathlineto{\pgfqpoint{1.661456in}{4.121941in}}%
\pgfpathlineto{\pgfqpoint{1.656011in}{4.254793in}}%
\pgfpathlineto{\pgfqpoint{1.650426in}{4.382568in}}%
\pgfpathlineto{\pgfqpoint{1.644694in}{4.505362in}}%
\pgfpathlineto{\pgfqpoint{1.638808in}{4.623277in}}%
\pgfpathlineto{\pgfqpoint{1.632758in}{4.736421in}}%
\pgfpathlineto{\pgfqpoint{1.626535in}{4.844910in}}%
\pgfpathlineto{\pgfqpoint{1.620129in}{4.948867in}}%
\pgfpathlineto{\pgfqpoint{1.613528in}{5.048419in}}%
\pgfpathlineto{\pgfqpoint{1.606722in}{5.143702in}}%
\pgfpathlineto{\pgfqpoint{1.599696in}{5.234857in}}%
\pgfpathlineto{\pgfqpoint{1.592435in}{5.322027in}}%
\pgfpathlineto{\pgfqpoint{1.584924in}{5.405363in}}%
\pgfpathlineto{\pgfqpoint{1.577145in}{5.485016in}}%
\pgfpathlineto{\pgfqpoint{1.569077in}{5.561152in}}%
\pgfpathlineto{\pgfqpoint{1.560699in}{5.634399in}}%
\pgfpathlineto{\pgfqpoint{1.551986in}{5.708737in}}%
\pgfpathlineto{\pgfqpoint{1.542909in}{5.779980in}}%
\pgfpathlineto{\pgfqpoint{1.533438in}{5.847772in}}%
\pgfpathlineto{\pgfqpoint{1.523536in}{5.912136in}}%
\pgfpathlineto{\pgfqpoint{1.513162in}{5.973107in}}%
\pgfpathlineto{\pgfqpoint{1.502269in}{6.030717in}}%
\pgfpathlineto{\pgfqpoint{1.490802in}{6.084999in}}%
\pgfpathlineto{\pgfqpoint{1.478698in}{6.135986in}}%
\pgfpathlineto{\pgfqpoint{1.465881in}{6.183705in}}%
\pgfpathlineto{\pgfqpoint{1.452262in}{6.228186in}}%
\pgfpathlineto{\pgfqpoint{1.437735in}{6.269450in}}%
\pgfpathlineto{\pgfqpoint{1.422168in}{6.307520in}}%
\pgfpathlineto{\pgfqpoint{1.405403in}{6.342410in}}%
\pgfpathlineto{\pgfqpoint{1.387238in}{6.374148in}}%
\pgfpathlineto{\pgfqpoint{1.367419in}{6.409938in}}%
\pgfpathlineto{\pgfqpoint{1.345614in}{6.445252in}}%
\pgfpathlineto{\pgfqpoint{1.321380in}{6.478852in}}%
\pgfpathlineto{\pgfqpoint{1.294107in}{6.510761in}}%
\pgfpathlineto{\pgfqpoint{1.262923in}{6.540998in}}%
\pgfpathlineto{\pgfqpoint{1.226515in}{6.569586in}}%
\pgfpathlineto{\pgfqpoint{1.182772in}{6.596551in}}%
\pgfpathlineto{\pgfqpoint{1.127977in}{6.621927in}}%
\pgfpathlineto{\pgfqpoint{1.054581in}{6.645761in}}%
\pgfpathlineto{\pgfqpoint{0.943050in}{6.668114in}}%
\pgfpathlineto{\pgfqpoint{0.702340in}{6.689068in}}%
\pgfpathclose%
\pgfusepath{stroke,fill}%
\end{pgfscope}%
\begin{pgfscope}%
\pgfpathrectangle{\pgfqpoint{0.702340in}{6.271734in}}{\pgfqpoint{1.223103in}{0.607948in}}%
\pgfusepath{clip}%
\pgfsetroundcap%
\pgfsetroundjoin%
\pgfsetlinewidth{0.501875pt}%
\definecolor{currentstroke}{rgb}{0.000000,0.000000,1.000000}%
\pgfsetstrokecolor{currentstroke}%
\pgfsetstrokeopacity{0.800000}%
\pgfsetdash{}{0pt}%
\pgfpathmoveto{\pgfqpoint{0.702340in}{6.597425in}}%
\pgfpathlineto{\pgfqpoint{1.925444in}{6.597425in}}%
\pgfusepath{stroke}%
\end{pgfscope}%
\begin{pgfscope}%
\pgfpathrectangle{\pgfqpoint{0.702340in}{6.271734in}}{\pgfqpoint{1.223103in}{0.607948in}}%
\pgfusepath{clip}%
\pgfsetbuttcap%
\pgfsetroundjoin%
\pgfsetlinewidth{1.003750pt}%
\definecolor{currentstroke}{rgb}{0.000000,0.000000,0.000000}%
\pgfsetstrokecolor{currentstroke}%
\pgfsetdash{{3.700000pt}{1.600000pt}}{0.000000pt}%
\pgfpathmoveto{\pgfqpoint{0.702340in}{6.595492in}}%
\pgfpathlineto{\pgfqpoint{1.925444in}{6.595492in}}%
\pgfusepath{stroke}%
\end{pgfscope}%
\begin{pgfscope}%
\pgfsetroundcap%
\pgfsetroundjoin%
\pgfsetlinewidth{0.501875pt}%
\definecolor{currentstroke}{rgb}{0.000000,0.000000,1.000000}%
\pgfsetstrokecolor{currentstroke}%
\pgfsetstrokeopacity{0.800000}%
\pgfsetdash{}{0pt}%
\pgfpathmoveto{\pgfqpoint{1.503033in}{6.713929in}}%
\pgfpathquadraticcurveto{\pgfqpoint{1.439785in}{6.664256in}}{\pgfqpoint{1.376536in}{6.614583in}}%
\pgfusepath{stroke}%
\end{pgfscope}%
\begin{pgfscope}%
\pgfsetfillopacity{0.800000}%
\pgfsetstrokeopacity{0.800000}%
\definecolor{textcolor}{rgb}{0.000000,0.000000,1.000000}%
\pgfsetstrokecolor{textcolor}%
\pgfsetfillcolor{textcolor}%
\pgftext[x=1.442982in,y=6.779809in,left,base]{\color{textcolor}\sffamily\fontsize{5.647059}{6.776471}\selectfont 17.54(17)}%
\end{pgfscope}%
\begin{pgfscope}%
\pgfsetbuttcap%
\pgfsetroundjoin%
\definecolor{currentfill}{rgb}{0.150000,0.150000,0.150000}%
\pgfsetfillcolor{currentfill}%
\pgfsetlinewidth{1.003750pt}%
\definecolor{currentstroke}{rgb}{0.150000,0.150000,0.150000}%
\pgfsetstrokecolor{currentstroke}%
\pgfsetdash{}{0pt}%
\pgfsys@defobject{currentmarker}{\pgfqpoint{0.000000in}{-0.066667in}}{\pgfqpoint{0.000000in}{0.000000in}}{%
\pgfpathmoveto{\pgfqpoint{0.000000in}{0.000000in}}%
\pgfpathlineto{\pgfqpoint{0.000000in}{-0.066667in}}%
\pgfusepath{stroke,fill}%
}%
\begin{pgfscope}%
\pgfsys@transformshift{0.702340in}{6.271734in}%
\pgfsys@useobject{currentmarker}{}%
\end{pgfscope}%
\end{pgfscope}%
\begin{pgfscope}%
\pgfsetbuttcap%
\pgfsetroundjoin%
\definecolor{currentfill}{rgb}{0.150000,0.150000,0.150000}%
\pgfsetfillcolor{currentfill}%
\pgfsetlinewidth{1.003750pt}%
\definecolor{currentstroke}{rgb}{0.150000,0.150000,0.150000}%
\pgfsetstrokecolor{currentstroke}%
\pgfsetdash{}{0pt}%
\pgfsys@defobject{currentmarker}{\pgfqpoint{0.000000in}{-0.066667in}}{\pgfqpoint{0.000000in}{0.000000in}}{%
\pgfpathmoveto{\pgfqpoint{0.000000in}{0.000000in}}%
\pgfpathlineto{\pgfqpoint{0.000000in}{-0.066667in}}%
\pgfusepath{stroke,fill}%
}%
\begin{pgfscope}%
\pgfsys@transformshift{1.203749in}{6.271734in}%
\pgfsys@useobject{currentmarker}{}%
\end{pgfscope}%
\end{pgfscope}%
\begin{pgfscope}%
\pgfsetbuttcap%
\pgfsetroundjoin%
\definecolor{currentfill}{rgb}{0.150000,0.150000,0.150000}%
\pgfsetfillcolor{currentfill}%
\pgfsetlinewidth{1.003750pt}%
\definecolor{currentstroke}{rgb}{0.150000,0.150000,0.150000}%
\pgfsetstrokecolor{currentstroke}%
\pgfsetdash{}{0pt}%
\pgfsys@defobject{currentmarker}{\pgfqpoint{0.000000in}{-0.066667in}}{\pgfqpoint{0.000000in}{0.000000in}}{%
\pgfpathmoveto{\pgfqpoint{0.000000in}{0.000000in}}%
\pgfpathlineto{\pgfqpoint{0.000000in}{-0.066667in}}%
\pgfusepath{stroke,fill}%
}%
\begin{pgfscope}%
\pgfsys@transformshift{1.705158in}{6.271734in}%
\pgfsys@useobject{currentmarker}{}%
\end{pgfscope}%
\end{pgfscope}%
\begin{pgfscope}%
\pgfsetbuttcap%
\pgfsetroundjoin%
\definecolor{currentfill}{rgb}{0.150000,0.150000,0.150000}%
\pgfsetfillcolor{currentfill}%
\pgfsetlinewidth{0.803000pt}%
\definecolor{currentstroke}{rgb}{0.150000,0.150000,0.150000}%
\pgfsetstrokecolor{currentstroke}%
\pgfsetdash{}{0pt}%
\pgfsys@defobject{currentmarker}{\pgfqpoint{0.000000in}{-0.044444in}}{\pgfqpoint{0.000000in}{0.000000in}}{%
\pgfpathmoveto{\pgfqpoint{0.000000in}{0.000000in}}%
\pgfpathlineto{\pgfqpoint{0.000000in}{-0.044444in}}%
\pgfusepath{stroke,fill}%
}%
\begin{pgfscope}%
\pgfsys@transformshift{0.853280in}{6.271734in}%
\pgfsys@useobject{currentmarker}{}%
\end{pgfscope}%
\end{pgfscope}%
\begin{pgfscope}%
\pgfsetbuttcap%
\pgfsetroundjoin%
\definecolor{currentfill}{rgb}{0.150000,0.150000,0.150000}%
\pgfsetfillcolor{currentfill}%
\pgfsetlinewidth{0.803000pt}%
\definecolor{currentstroke}{rgb}{0.150000,0.150000,0.150000}%
\pgfsetstrokecolor{currentstroke}%
\pgfsetdash{}{0pt}%
\pgfsys@defobject{currentmarker}{\pgfqpoint{0.000000in}{-0.044444in}}{\pgfqpoint{0.000000in}{0.000000in}}{%
\pgfpathmoveto{\pgfqpoint{0.000000in}{0.000000in}}%
\pgfpathlineto{\pgfqpoint{0.000000in}{-0.044444in}}%
\pgfusepath{stroke,fill}%
}%
\begin{pgfscope}%
\pgfsys@transformshift{0.941573in}{6.271734in}%
\pgfsys@useobject{currentmarker}{}%
\end{pgfscope}%
\end{pgfscope}%
\begin{pgfscope}%
\pgfsetbuttcap%
\pgfsetroundjoin%
\definecolor{currentfill}{rgb}{0.150000,0.150000,0.150000}%
\pgfsetfillcolor{currentfill}%
\pgfsetlinewidth{0.803000pt}%
\definecolor{currentstroke}{rgb}{0.150000,0.150000,0.150000}%
\pgfsetstrokecolor{currentstroke}%
\pgfsetdash{}{0pt}%
\pgfsys@defobject{currentmarker}{\pgfqpoint{0.000000in}{-0.044444in}}{\pgfqpoint{0.000000in}{0.000000in}}{%
\pgfpathmoveto{\pgfqpoint{0.000000in}{0.000000in}}%
\pgfpathlineto{\pgfqpoint{0.000000in}{-0.044444in}}%
\pgfusepath{stroke,fill}%
}%
\begin{pgfscope}%
\pgfsys@transformshift{1.004219in}{6.271734in}%
\pgfsys@useobject{currentmarker}{}%
\end{pgfscope}%
\end{pgfscope}%
\begin{pgfscope}%
\pgfsetbuttcap%
\pgfsetroundjoin%
\definecolor{currentfill}{rgb}{0.150000,0.150000,0.150000}%
\pgfsetfillcolor{currentfill}%
\pgfsetlinewidth{0.803000pt}%
\definecolor{currentstroke}{rgb}{0.150000,0.150000,0.150000}%
\pgfsetstrokecolor{currentstroke}%
\pgfsetdash{}{0pt}%
\pgfsys@defobject{currentmarker}{\pgfqpoint{0.000000in}{-0.044444in}}{\pgfqpoint{0.000000in}{0.000000in}}{%
\pgfpathmoveto{\pgfqpoint{0.000000in}{0.000000in}}%
\pgfpathlineto{\pgfqpoint{0.000000in}{-0.044444in}}%
\pgfusepath{stroke,fill}%
}%
\begin{pgfscope}%
\pgfsys@transformshift{1.052810in}{6.271734in}%
\pgfsys@useobject{currentmarker}{}%
\end{pgfscope}%
\end{pgfscope}%
\begin{pgfscope}%
\pgfsetbuttcap%
\pgfsetroundjoin%
\definecolor{currentfill}{rgb}{0.150000,0.150000,0.150000}%
\pgfsetfillcolor{currentfill}%
\pgfsetlinewidth{0.803000pt}%
\definecolor{currentstroke}{rgb}{0.150000,0.150000,0.150000}%
\pgfsetstrokecolor{currentstroke}%
\pgfsetdash{}{0pt}%
\pgfsys@defobject{currentmarker}{\pgfqpoint{0.000000in}{-0.044444in}}{\pgfqpoint{0.000000in}{0.000000in}}{%
\pgfpathmoveto{\pgfqpoint{0.000000in}{0.000000in}}%
\pgfpathlineto{\pgfqpoint{0.000000in}{-0.044444in}}%
\pgfusepath{stroke,fill}%
}%
\begin{pgfscope}%
\pgfsys@transformshift{1.092512in}{6.271734in}%
\pgfsys@useobject{currentmarker}{}%
\end{pgfscope}%
\end{pgfscope}%
\begin{pgfscope}%
\pgfsetbuttcap%
\pgfsetroundjoin%
\definecolor{currentfill}{rgb}{0.150000,0.150000,0.150000}%
\pgfsetfillcolor{currentfill}%
\pgfsetlinewidth{0.803000pt}%
\definecolor{currentstroke}{rgb}{0.150000,0.150000,0.150000}%
\pgfsetstrokecolor{currentstroke}%
\pgfsetdash{}{0pt}%
\pgfsys@defobject{currentmarker}{\pgfqpoint{0.000000in}{-0.044444in}}{\pgfqpoint{0.000000in}{0.000000in}}{%
\pgfpathmoveto{\pgfqpoint{0.000000in}{0.000000in}}%
\pgfpathlineto{\pgfqpoint{0.000000in}{-0.044444in}}%
\pgfusepath{stroke,fill}%
}%
\begin{pgfscope}%
\pgfsys@transformshift{1.126080in}{6.271734in}%
\pgfsys@useobject{currentmarker}{}%
\end{pgfscope}%
\end{pgfscope}%
\begin{pgfscope}%
\pgfsetbuttcap%
\pgfsetroundjoin%
\definecolor{currentfill}{rgb}{0.150000,0.150000,0.150000}%
\pgfsetfillcolor{currentfill}%
\pgfsetlinewidth{0.803000pt}%
\definecolor{currentstroke}{rgb}{0.150000,0.150000,0.150000}%
\pgfsetstrokecolor{currentstroke}%
\pgfsetdash{}{0pt}%
\pgfsys@defobject{currentmarker}{\pgfqpoint{0.000000in}{-0.044444in}}{\pgfqpoint{0.000000in}{0.000000in}}{%
\pgfpathmoveto{\pgfqpoint{0.000000in}{0.000000in}}%
\pgfpathlineto{\pgfqpoint{0.000000in}{-0.044444in}}%
\pgfusepath{stroke,fill}%
}%
\begin{pgfscope}%
\pgfsys@transformshift{1.155158in}{6.271734in}%
\pgfsys@useobject{currentmarker}{}%
\end{pgfscope}%
\end{pgfscope}%
\begin{pgfscope}%
\pgfsetbuttcap%
\pgfsetroundjoin%
\definecolor{currentfill}{rgb}{0.150000,0.150000,0.150000}%
\pgfsetfillcolor{currentfill}%
\pgfsetlinewidth{0.803000pt}%
\definecolor{currentstroke}{rgb}{0.150000,0.150000,0.150000}%
\pgfsetstrokecolor{currentstroke}%
\pgfsetdash{}{0pt}%
\pgfsys@defobject{currentmarker}{\pgfqpoint{0.000000in}{-0.044444in}}{\pgfqpoint{0.000000in}{0.000000in}}{%
\pgfpathmoveto{\pgfqpoint{0.000000in}{0.000000in}}%
\pgfpathlineto{\pgfqpoint{0.000000in}{-0.044444in}}%
\pgfusepath{stroke,fill}%
}%
\begin{pgfscope}%
\pgfsys@transformshift{1.180806in}{6.271734in}%
\pgfsys@useobject{currentmarker}{}%
\end{pgfscope}%
\end{pgfscope}%
\begin{pgfscope}%
\pgfsetbuttcap%
\pgfsetroundjoin%
\definecolor{currentfill}{rgb}{0.150000,0.150000,0.150000}%
\pgfsetfillcolor{currentfill}%
\pgfsetlinewidth{0.803000pt}%
\definecolor{currentstroke}{rgb}{0.150000,0.150000,0.150000}%
\pgfsetstrokecolor{currentstroke}%
\pgfsetdash{}{0pt}%
\pgfsys@defobject{currentmarker}{\pgfqpoint{0.000000in}{-0.044444in}}{\pgfqpoint{0.000000in}{0.000000in}}{%
\pgfpathmoveto{\pgfqpoint{0.000000in}{0.000000in}}%
\pgfpathlineto{\pgfqpoint{0.000000in}{-0.044444in}}%
\pgfusepath{stroke,fill}%
}%
\begin{pgfscope}%
\pgfsys@transformshift{1.354689in}{6.271734in}%
\pgfsys@useobject{currentmarker}{}%
\end{pgfscope}%
\end{pgfscope}%
\begin{pgfscope}%
\pgfsetbuttcap%
\pgfsetroundjoin%
\definecolor{currentfill}{rgb}{0.150000,0.150000,0.150000}%
\pgfsetfillcolor{currentfill}%
\pgfsetlinewidth{0.803000pt}%
\definecolor{currentstroke}{rgb}{0.150000,0.150000,0.150000}%
\pgfsetstrokecolor{currentstroke}%
\pgfsetdash{}{0pt}%
\pgfsys@defobject{currentmarker}{\pgfqpoint{0.000000in}{-0.044444in}}{\pgfqpoint{0.000000in}{0.000000in}}{%
\pgfpathmoveto{\pgfqpoint{0.000000in}{0.000000in}}%
\pgfpathlineto{\pgfqpoint{0.000000in}{-0.044444in}}%
\pgfusepath{stroke,fill}%
}%
\begin{pgfscope}%
\pgfsys@transformshift{1.442982in}{6.271734in}%
\pgfsys@useobject{currentmarker}{}%
\end{pgfscope}%
\end{pgfscope}%
\begin{pgfscope}%
\pgfsetbuttcap%
\pgfsetroundjoin%
\definecolor{currentfill}{rgb}{0.150000,0.150000,0.150000}%
\pgfsetfillcolor{currentfill}%
\pgfsetlinewidth{0.803000pt}%
\definecolor{currentstroke}{rgb}{0.150000,0.150000,0.150000}%
\pgfsetstrokecolor{currentstroke}%
\pgfsetdash{}{0pt}%
\pgfsys@defobject{currentmarker}{\pgfqpoint{0.000000in}{-0.044444in}}{\pgfqpoint{0.000000in}{0.000000in}}{%
\pgfpathmoveto{\pgfqpoint{0.000000in}{0.000000in}}%
\pgfpathlineto{\pgfqpoint{0.000000in}{-0.044444in}}%
\pgfusepath{stroke,fill}%
}%
\begin{pgfscope}%
\pgfsys@transformshift{1.505628in}{6.271734in}%
\pgfsys@useobject{currentmarker}{}%
\end{pgfscope}%
\end{pgfscope}%
\begin{pgfscope}%
\pgfsetbuttcap%
\pgfsetroundjoin%
\definecolor{currentfill}{rgb}{0.150000,0.150000,0.150000}%
\pgfsetfillcolor{currentfill}%
\pgfsetlinewidth{0.803000pt}%
\definecolor{currentstroke}{rgb}{0.150000,0.150000,0.150000}%
\pgfsetstrokecolor{currentstroke}%
\pgfsetdash{}{0pt}%
\pgfsys@defobject{currentmarker}{\pgfqpoint{0.000000in}{-0.044444in}}{\pgfqpoint{0.000000in}{0.000000in}}{%
\pgfpathmoveto{\pgfqpoint{0.000000in}{0.000000in}}%
\pgfpathlineto{\pgfqpoint{0.000000in}{-0.044444in}}%
\pgfusepath{stroke,fill}%
}%
\begin{pgfscope}%
\pgfsys@transformshift{1.554219in}{6.271734in}%
\pgfsys@useobject{currentmarker}{}%
\end{pgfscope}%
\end{pgfscope}%
\begin{pgfscope}%
\pgfsetbuttcap%
\pgfsetroundjoin%
\definecolor{currentfill}{rgb}{0.150000,0.150000,0.150000}%
\pgfsetfillcolor{currentfill}%
\pgfsetlinewidth{0.803000pt}%
\definecolor{currentstroke}{rgb}{0.150000,0.150000,0.150000}%
\pgfsetstrokecolor{currentstroke}%
\pgfsetdash{}{0pt}%
\pgfsys@defobject{currentmarker}{\pgfqpoint{0.000000in}{-0.044444in}}{\pgfqpoint{0.000000in}{0.000000in}}{%
\pgfpathmoveto{\pgfqpoint{0.000000in}{0.000000in}}%
\pgfpathlineto{\pgfqpoint{0.000000in}{-0.044444in}}%
\pgfusepath{stroke,fill}%
}%
\begin{pgfscope}%
\pgfsys@transformshift{1.593921in}{6.271734in}%
\pgfsys@useobject{currentmarker}{}%
\end{pgfscope}%
\end{pgfscope}%
\begin{pgfscope}%
\pgfsetbuttcap%
\pgfsetroundjoin%
\definecolor{currentfill}{rgb}{0.150000,0.150000,0.150000}%
\pgfsetfillcolor{currentfill}%
\pgfsetlinewidth{0.803000pt}%
\definecolor{currentstroke}{rgb}{0.150000,0.150000,0.150000}%
\pgfsetstrokecolor{currentstroke}%
\pgfsetdash{}{0pt}%
\pgfsys@defobject{currentmarker}{\pgfqpoint{0.000000in}{-0.044444in}}{\pgfqpoint{0.000000in}{0.000000in}}{%
\pgfpathmoveto{\pgfqpoint{0.000000in}{0.000000in}}%
\pgfpathlineto{\pgfqpoint{0.000000in}{-0.044444in}}%
\pgfusepath{stroke,fill}%
}%
\begin{pgfscope}%
\pgfsys@transformshift{1.627489in}{6.271734in}%
\pgfsys@useobject{currentmarker}{}%
\end{pgfscope}%
\end{pgfscope}%
\begin{pgfscope}%
\pgfsetbuttcap%
\pgfsetroundjoin%
\definecolor{currentfill}{rgb}{0.150000,0.150000,0.150000}%
\pgfsetfillcolor{currentfill}%
\pgfsetlinewidth{0.803000pt}%
\definecolor{currentstroke}{rgb}{0.150000,0.150000,0.150000}%
\pgfsetstrokecolor{currentstroke}%
\pgfsetdash{}{0pt}%
\pgfsys@defobject{currentmarker}{\pgfqpoint{0.000000in}{-0.044444in}}{\pgfqpoint{0.000000in}{0.000000in}}{%
\pgfpathmoveto{\pgfqpoint{0.000000in}{0.000000in}}%
\pgfpathlineto{\pgfqpoint{0.000000in}{-0.044444in}}%
\pgfusepath{stroke,fill}%
}%
\begin{pgfscope}%
\pgfsys@transformshift{1.656567in}{6.271734in}%
\pgfsys@useobject{currentmarker}{}%
\end{pgfscope}%
\end{pgfscope}%
\begin{pgfscope}%
\pgfsetbuttcap%
\pgfsetroundjoin%
\definecolor{currentfill}{rgb}{0.150000,0.150000,0.150000}%
\pgfsetfillcolor{currentfill}%
\pgfsetlinewidth{0.803000pt}%
\definecolor{currentstroke}{rgb}{0.150000,0.150000,0.150000}%
\pgfsetstrokecolor{currentstroke}%
\pgfsetdash{}{0pt}%
\pgfsys@defobject{currentmarker}{\pgfqpoint{0.000000in}{-0.044444in}}{\pgfqpoint{0.000000in}{0.000000in}}{%
\pgfpathmoveto{\pgfqpoint{0.000000in}{0.000000in}}%
\pgfpathlineto{\pgfqpoint{0.000000in}{-0.044444in}}%
\pgfusepath{stroke,fill}%
}%
\begin{pgfscope}%
\pgfsys@transformshift{1.682215in}{6.271734in}%
\pgfsys@useobject{currentmarker}{}%
\end{pgfscope}%
\end{pgfscope}%
\begin{pgfscope}%
\pgfsetbuttcap%
\pgfsetroundjoin%
\definecolor{currentfill}{rgb}{0.150000,0.150000,0.150000}%
\pgfsetfillcolor{currentfill}%
\pgfsetlinewidth{0.803000pt}%
\definecolor{currentstroke}{rgb}{0.150000,0.150000,0.150000}%
\pgfsetstrokecolor{currentstroke}%
\pgfsetdash{}{0pt}%
\pgfsys@defobject{currentmarker}{\pgfqpoint{0.000000in}{-0.044444in}}{\pgfqpoint{0.000000in}{0.000000in}}{%
\pgfpathmoveto{\pgfqpoint{0.000000in}{0.000000in}}%
\pgfpathlineto{\pgfqpoint{0.000000in}{-0.044444in}}%
\pgfusepath{stroke,fill}%
}%
\begin{pgfscope}%
\pgfsys@transformshift{1.856098in}{6.271734in}%
\pgfsys@useobject{currentmarker}{}%
\end{pgfscope}%
\end{pgfscope}%
\begin{pgfscope}%
\pgfsetbuttcap%
\pgfsetroundjoin%
\definecolor{currentfill}{rgb}{0.150000,0.150000,0.150000}%
\pgfsetfillcolor{currentfill}%
\pgfsetlinewidth{1.003750pt}%
\definecolor{currentstroke}{rgb}{0.150000,0.150000,0.150000}%
\pgfsetstrokecolor{currentstroke}%
\pgfsetdash{}{0pt}%
\pgfsys@defobject{currentmarker}{\pgfqpoint{-0.066667in}{0.000000in}}{\pgfqpoint{0.000000in}{0.000000in}}{%
\pgfpathmoveto{\pgfqpoint{0.000000in}{0.000000in}}%
\pgfpathlineto{\pgfqpoint{-0.066667in}{0.000000in}}%
\pgfusepath{stroke,fill}%
}%
\begin{pgfscope}%
\pgfsys@transformshift{0.702340in}{6.271734in}%
\pgfsys@useobject{currentmarker}{}%
\end{pgfscope}%
\end{pgfscope}%
\begin{pgfscope}%
\definecolor{textcolor}{rgb}{0.150000,0.150000,0.150000}%
\pgfsetstrokecolor{textcolor}%
\pgfsetfillcolor{textcolor}%
\pgftext[x=0.374971in,y=6.246787in,left,base]{\color{textcolor}\sffamily\fontsize{5.176471}{6.211765}\selectfont 17.000}%
\end{pgfscope}%
\begin{pgfscope}%
\pgfsetbuttcap%
\pgfsetroundjoin%
\definecolor{currentfill}{rgb}{0.150000,0.150000,0.150000}%
\pgfsetfillcolor{currentfill}%
\pgfsetlinewidth{1.003750pt}%
\definecolor{currentstroke}{rgb}{0.150000,0.150000,0.150000}%
\pgfsetstrokecolor{currentstroke}%
\pgfsetdash{}{0pt}%
\pgfsys@defobject{currentmarker}{\pgfqpoint{-0.066667in}{0.000000in}}{\pgfqpoint{0.000000in}{0.000000in}}{%
\pgfpathmoveto{\pgfqpoint{0.000000in}{0.000000in}}%
\pgfpathlineto{\pgfqpoint{-0.066667in}{0.000000in}}%
\pgfusepath{stroke,fill}%
}%
\begin{pgfscope}%
\pgfsys@transformshift{0.702340in}{6.595492in}%
\pgfsys@useobject{currentmarker}{}%
\end{pgfscope}%
\end{pgfscope}%
\begin{pgfscope}%
\definecolor{textcolor}{rgb}{0.150000,0.150000,0.150000}%
\pgfsetstrokecolor{textcolor}%
\pgfsetfillcolor{textcolor}%
\pgftext[x=0.374971in,y=6.570544in,left,base]{\color{textcolor}\sffamily\fontsize{5.176471}{6.211765}\selectfont 17.533}%
\end{pgfscope}%
\begin{pgfscope}%
\pgfsetbuttcap%
\pgfsetroundjoin%
\definecolor{currentfill}{rgb}{0.150000,0.150000,0.150000}%
\pgfsetfillcolor{currentfill}%
\pgfsetlinewidth{1.003750pt}%
\definecolor{currentstroke}{rgb}{0.150000,0.150000,0.150000}%
\pgfsetstrokecolor{currentstroke}%
\pgfsetdash{}{0pt}%
\pgfsys@defobject{currentmarker}{\pgfqpoint{-0.066667in}{0.000000in}}{\pgfqpoint{0.000000in}{0.000000in}}{%
\pgfpathmoveto{\pgfqpoint{0.000000in}{0.000000in}}%
\pgfpathlineto{\pgfqpoint{-0.066667in}{0.000000in}}%
\pgfusepath{stroke,fill}%
}%
\begin{pgfscope}%
\pgfsys@transformshift{0.702340in}{6.879682in}%
\pgfsys@useobject{currentmarker}{}%
\end{pgfscope}%
\end{pgfscope}%
\begin{pgfscope}%
\definecolor{textcolor}{rgb}{0.150000,0.150000,0.150000}%
\pgfsetstrokecolor{textcolor}%
\pgfsetfillcolor{textcolor}%
\pgftext[x=0.374971in,y=6.854735in,left,base]{\color{textcolor}\sffamily\fontsize{5.176471}{6.211765}\selectfont 18.000}%
\end{pgfscope}%
\begin{pgfscope}%
\definecolor{textcolor}{rgb}{0.150000,0.150000,0.150000}%
\pgfsetstrokecolor{textcolor}%
\pgfsetfillcolor{textcolor}%
\pgftext[x=0.319416in,y=6.575708in,,bottom,rotate=90.000000]{\color{textcolor}\sffamily\fontsize{5.647059}{6.776471}\selectfont \(\displaystyle x = \frac{2 \mu E L^2}{4 \pi^2}\)}%
\end{pgfscope}%
\begin{pgfscope}%
\pgfpathrectangle{\pgfqpoint{0.702340in}{6.271734in}}{\pgfqpoint{1.223103in}{0.607948in}}%
\pgfusepath{clip}%
\pgfsetroundcap%
\pgfsetroundjoin%
\pgfsetlinewidth{1.204500pt}%
\definecolor{currentstroke}{rgb}{0.000000,0.501961,0.000000}%
\pgfsetstrokecolor{currentstroke}%
\pgfsetdash{}{0pt}%
\pgfpathmoveto{\pgfqpoint{0.702340in}{6.595160in}}%
\pgfpathlineto{\pgfqpoint{0.943050in}{6.588356in}}%
\pgfpathlineto{\pgfqpoint{1.054581in}{6.578664in}}%
\pgfpathlineto{\pgfqpoint{1.127977in}{6.566184in}}%
\pgfpathlineto{\pgfqpoint{1.182772in}{6.551003in}}%
\pgfpathlineto{\pgfqpoint{1.226515in}{6.533194in}}%
\pgfpathlineto{\pgfqpoint{1.262923in}{6.512813in}}%
\pgfpathlineto{\pgfqpoint{1.294107in}{6.489906in}}%
\pgfpathlineto{\pgfqpoint{1.321380in}{6.464506in}}%
\pgfpathlineto{\pgfqpoint{1.345614in}{6.436634in}}%
\pgfpathlineto{\pgfqpoint{1.367419in}{6.406297in}}%
\pgfpathlineto{\pgfqpoint{1.387238in}{6.373496in}}%
\pgfpathlineto{\pgfqpoint{1.405403in}{6.338218in}}%
\pgfpathlineto{\pgfqpoint{1.422168in}{6.300441in}}%
\pgfpathlineto{\pgfqpoint{1.434543in}{6.268401in}}%
\pgfusepath{stroke}%
\end{pgfscope}%
\begin{pgfscope}%
\pgfsetrectcap%
\pgfsetmiterjoin%
\pgfsetlinewidth{1.003750pt}%
\definecolor{currentstroke}{rgb}{0.150000,0.150000,0.150000}%
\pgfsetstrokecolor{currentstroke}%
\pgfsetdash{}{0pt}%
\pgfpathmoveto{\pgfqpoint{0.702340in}{6.271734in}}%
\pgfpathlineto{\pgfqpoint{0.702340in}{6.879682in}}%
\pgfusepath{stroke}%
\end{pgfscope}%
\begin{pgfscope}%
\pgfsetrectcap%
\pgfsetmiterjoin%
\pgfsetlinewidth{1.003750pt}%
\definecolor{currentstroke}{rgb}{0.150000,0.150000,0.150000}%
\pgfsetstrokecolor{currentstroke}%
\pgfsetdash{}{0pt}%
\pgfpathmoveto{\pgfqpoint{0.702340in}{6.271734in}}%
\pgfpathlineto{\pgfqpoint{1.925444in}{6.271734in}}%
\pgfusepath{stroke}%
\end{pgfscope}%
\begin{pgfscope}%
\pgfpathrectangle{\pgfqpoint{0.702340in}{6.271734in}}{\pgfqpoint{1.223103in}{0.607948in}}%
\pgfusepath{clip}%
\pgfsetbuttcap%
\pgfsetroundjoin%
\definecolor{currentfill}{rgb}{0.000000,0.000000,0.000000}%
\pgfsetfillcolor{currentfill}%
\pgfsetlinewidth{1.003750pt}%
\definecolor{currentstroke}{rgb}{0.000000,0.000000,0.000000}%
\pgfsetstrokecolor{currentstroke}%
\pgfsetdash{}{0pt}%
\pgfsys@defobject{currentmarker}{\pgfqpoint{-0.013889in}{-0.013889in}}{\pgfqpoint{0.013889in}{0.013889in}}{%
\pgfpathmoveto{\pgfqpoint{0.000000in}{-0.013889in}}%
\pgfpathcurveto{\pgfqpoint{0.003683in}{-0.013889in}}{\pgfqpoint{0.007216in}{-0.012425in}}{\pgfqpoint{0.009821in}{-0.009821in}}%
\pgfpathcurveto{\pgfqpoint{0.012425in}{-0.007216in}}{\pgfqpoint{0.013889in}{-0.003683in}}{\pgfqpoint{0.013889in}{0.000000in}}%
\pgfpathcurveto{\pgfqpoint{0.013889in}{0.003683in}}{\pgfqpoint{0.012425in}{0.007216in}}{\pgfqpoint{0.009821in}{0.009821in}}%
\pgfpathcurveto{\pgfqpoint{0.007216in}{0.012425in}}{\pgfqpoint{0.003683in}{0.013889in}}{\pgfqpoint{0.000000in}{0.013889in}}%
\pgfpathcurveto{\pgfqpoint{-0.003683in}{0.013889in}}{\pgfqpoint{-0.007216in}{0.012425in}}{\pgfqpoint{-0.009821in}{0.009821in}}%
\pgfpathcurveto{\pgfqpoint{-0.012425in}{0.007216in}}{\pgfqpoint{-0.013889in}{0.003683in}}{\pgfqpoint{-0.013889in}{0.000000in}}%
\pgfpathcurveto{\pgfqpoint{-0.013889in}{-0.003683in}}{\pgfqpoint{-0.012425in}{-0.007216in}}{\pgfqpoint{-0.009821in}{-0.009821in}}%
\pgfpathcurveto{\pgfqpoint{-0.007216in}{-0.012425in}}{\pgfqpoint{-0.003683in}{-0.013889in}}{\pgfqpoint{0.000000in}{-0.013889in}}%
\pgfpathclose%
\pgfusepath{stroke,fill}%
}%
\begin{pgfscope}%
\pgfsys@transformshift{1.705158in}{2.684627in}%
\pgfsys@useobject{currentmarker}{}%
\end{pgfscope}%
\begin{pgfscope}%
\pgfsys@transformshift{1.616865in}{4.990348in}%
\pgfsys@useobject{currentmarker}{}%
\end{pgfscope}%
\begin{pgfscope}%
\pgfsys@transformshift{1.554219in}{5.687046in}%
\pgfsys@useobject{currentmarker}{}%
\end{pgfscope}%
\begin{pgfscope}%
\pgfsys@transformshift{1.505628in}{5.994838in}%
\pgfsys@useobject{currentmarker}{}%
\end{pgfscope}%
\begin{pgfscope}%
\pgfsys@transformshift{1.465926in}{6.164017in}%
\pgfsys@useobject{currentmarker}{}%
\end{pgfscope}%
\begin{pgfscope}%
\pgfsys@transformshift{1.432358in}{6.269852in}%
\pgfsys@useobject{currentmarker}{}%
\end{pgfscope}%
\begin{pgfscope}%
\pgfsys@transformshift{1.403280in}{6.340954in}%
\pgfsys@useobject{currentmarker}{}%
\end{pgfscope}%
\begin{pgfscope}%
\pgfsys@transformshift{1.397903in}{6.352351in}%
\pgfsys@useobject{currentmarker}{}%
\end{pgfscope}%
\begin{pgfscope}%
\pgfsys@transformshift{1.392656in}{6.362998in}%
\pgfsys@useobject{currentmarker}{}%
\end{pgfscope}%
\begin{pgfscope}%
\pgfsys@transformshift{1.387532in}{6.372959in}%
\pgfsys@useobject{currentmarker}{}%
\end{pgfscope}%
\begin{pgfscope}%
\pgfsys@transformshift{1.382525in}{6.382291in}%
\pgfsys@useobject{currentmarker}{}%
\end{pgfscope}%
\begin{pgfscope}%
\pgfsys@transformshift{1.377632in}{6.391045in}%
\pgfsys@useobject{currentmarker}{}%
\end{pgfscope}%
\begin{pgfscope}%
\pgfsys@transformshift{1.372846in}{6.399268in}%
\pgfsys@useobject{currentmarker}{}%
\end{pgfscope}%
\begin{pgfscope}%
\pgfsys@transformshift{1.368163in}{6.407002in}%
\pgfsys@useobject{currentmarker}{}%
\end{pgfscope}%
\begin{pgfscope}%
\pgfsys@transformshift{1.363578in}{6.414284in}%
\pgfsys@useobject{currentmarker}{}%
\end{pgfscope}%
\begin{pgfscope}%
\pgfsys@transformshift{1.359088in}{6.421150in}%
\pgfsys@useobject{currentmarker}{}%
\end{pgfscope}%
\begin{pgfscope}%
\pgfsys@transformshift{1.354689in}{6.427629in}%
\pgfsys@useobject{currentmarker}{}%
\end{pgfscope}%
\end{pgfscope}%
\begin{pgfscope}%
\definecolor{textcolor}{rgb}{0.150000,0.150000,0.150000}%
\pgfsetstrokecolor{textcolor}%
\pgfsetfillcolor{textcolor}%
\pgftext[x=1.313892in,y=6.963016in,,base]{\color{textcolor}\sffamily\fontsize{5.647059}{6.776471}\selectfont \(\displaystyle n_{\mathrm{step}} = 1\)}%
\end{pgfscope}%
\begin{pgfscope}%
\pgfsetbuttcap%
\pgfsetmiterjoin%
\definecolor{currentfill}{rgb}{1.000000,1.000000,1.000000}%
\pgfsetfillcolor{currentfill}%
\pgfsetlinewidth{0.000000pt}%
\definecolor{currentstroke}{rgb}{0.000000,0.000000,0.000000}%
\pgfsetstrokecolor{currentstroke}%
\pgfsetstrokeopacity{0.000000}%
\pgfsetdash{}{0pt}%
\pgfpathmoveto{\pgfqpoint{2.170064in}{6.271734in}}%
\pgfpathlineto{\pgfqpoint{3.393168in}{6.271734in}}%
\pgfpathlineto{\pgfqpoint{3.393168in}{6.879682in}}%
\pgfpathlineto{\pgfqpoint{2.170064in}{6.879682in}}%
\pgfpathclose%
\pgfusepath{fill}%
\end{pgfscope}%
\begin{pgfscope}%
\pgfpathrectangle{\pgfqpoint{2.170064in}{6.271734in}}{\pgfqpoint{1.223103in}{0.607948in}}%
\pgfusepath{clip}%
\pgfsetbuttcap%
\pgfsetmiterjoin%
\definecolor{currentfill}{rgb}{0.000000,0.000000,1.000000}%
\pgfsetfillcolor{currentfill}%
\pgfsetfillopacity{0.100000}%
\pgfsetlinewidth{0.803000pt}%
\definecolor{currentstroke}{rgb}{0.000000,0.000000,1.000000}%
\pgfsetstrokecolor{currentstroke}%
\pgfsetstrokeopacity{0.100000}%
\pgfsetdash{}{0pt}%
\pgfpathmoveto{\pgfqpoint{2.170064in}{6.273870in}}%
\pgfpathlineto{\pgfqpoint{2.170064in}{6.768412in}}%
\pgfpathlineto{\pgfqpoint{3.393168in}{6.768412in}}%
\pgfpathlineto{\pgfqpoint{3.393168in}{6.273870in}}%
\pgfpathclose%
\pgfusepath{stroke,fill}%
\end{pgfscope}%
\begin{pgfscope}%
\pgfpathrectangle{\pgfqpoint{2.170064in}{6.271734in}}{\pgfqpoint{1.223103in}{0.607948in}}%
\pgfusepath{clip}%
\pgfsetbuttcap%
\pgfsetroundjoin%
\definecolor{currentfill}{rgb}{0.000000,0.501961,0.000000}%
\pgfsetfillcolor{currentfill}%
\pgfsetfillopacity{0.500000}%
\pgfsetlinewidth{0.803000pt}%
\definecolor{currentstroke}{rgb}{0.000000,0.501961,0.000000}%
\pgfsetstrokecolor{currentstroke}%
\pgfsetstrokeopacity{0.500000}%
\pgfsetdash{}{0pt}%
\pgfpathmoveto{\pgfqpoint{2.170064in}{6.753237in}}%
\pgfpathlineto{\pgfqpoint{2.170064in}{6.296287in}}%
\pgfpathlineto{\pgfqpoint{2.410774in}{6.338970in}}%
\pgfpathlineto{\pgfqpoint{2.522305in}{6.378280in}}%
\pgfpathlineto{\pgfqpoint{2.595701in}{6.414334in}}%
\pgfpathlineto{\pgfqpoint{2.650497in}{6.447223in}}%
\pgfpathlineto{\pgfqpoint{2.694239in}{6.477017in}}%
\pgfpathlineto{\pgfqpoint{2.730647in}{6.503765in}}%
\pgfpathlineto{\pgfqpoint{2.761831in}{6.527502in}}%
\pgfpathlineto{\pgfqpoint{2.789104in}{6.548249in}}%
\pgfpathlineto{\pgfqpoint{2.813338in}{6.566015in}}%
\pgfpathlineto{\pgfqpoint{2.835143in}{6.580791in}}%
\pgfpathlineto{\pgfqpoint{2.854962in}{6.591918in}}%
\pgfpathlineto{\pgfqpoint{2.873127in}{6.585890in}}%
\pgfpathlineto{\pgfqpoint{2.889892in}{6.578983in}}%
\pgfpathlineto{\pgfqpoint{2.905459in}{6.571962in}}%
\pgfpathlineto{\pgfqpoint{2.919986in}{6.564645in}}%
\pgfpathlineto{\pgfqpoint{2.933605in}{6.556847in}}%
\pgfpathlineto{\pgfqpoint{2.946422in}{6.548383in}}%
\pgfpathlineto{\pgfqpoint{2.958526in}{6.539071in}}%
\pgfpathlineto{\pgfqpoint{2.969993in}{6.528720in}}%
\pgfpathlineto{\pgfqpoint{2.980886in}{6.517133in}}%
\pgfpathlineto{\pgfqpoint{2.991260in}{6.504090in}}%
\pgfpathlineto{\pgfqpoint{3.001162in}{6.489324in}}%
\pgfpathlineto{\pgfqpoint{3.010633in}{6.472413in}}%
\pgfpathlineto{\pgfqpoint{3.019710in}{6.452334in}}%
\pgfpathlineto{\pgfqpoint{3.028423in}{6.426194in}}%
\pgfpathlineto{\pgfqpoint{3.036801in}{6.392420in}}%
\pgfpathlineto{\pgfqpoint{3.044869in}{6.353309in}}%
\pgfpathlineto{\pgfqpoint{3.052648in}{6.309931in}}%
\pgfpathlineto{\pgfqpoint{3.060159in}{6.262496in}}%
\pgfpathlineto{\pgfqpoint{3.067420in}{6.210999in}}%
\pgfpathlineto{\pgfqpoint{3.074446in}{6.155365in}}%
\pgfpathlineto{\pgfqpoint{3.081253in}{6.095493in}}%
\pgfpathlineto{\pgfqpoint{3.087853in}{6.031264in}}%
\pgfpathlineto{\pgfqpoint{3.094259in}{5.962553in}}%
\pgfpathlineto{\pgfqpoint{3.100482in}{5.889228in}}%
\pgfpathlineto{\pgfqpoint{3.106532in}{5.811151in}}%
\pgfpathlineto{\pgfqpoint{3.112419in}{5.728183in}}%
\pgfpathlineto{\pgfqpoint{3.118150in}{5.640185in}}%
\pgfpathlineto{\pgfqpoint{3.123735in}{5.547016in}}%
\pgfpathlineto{\pgfqpoint{3.129180in}{5.448540in}}%
\pgfpathlineto{\pgfqpoint{3.134492in}{5.344621in}}%
\pgfpathlineto{\pgfqpoint{3.139677in}{5.235130in}}%
\pgfpathlineto{\pgfqpoint{3.144742in}{5.119940in}}%
\pgfpathlineto{\pgfqpoint{3.149692in}{4.998930in}}%
\pgfpathlineto{\pgfqpoint{3.154532in}{4.871983in}}%
\pgfpathlineto{\pgfqpoint{3.159267in}{4.738974in}}%
\pgfpathlineto{\pgfqpoint{3.163901in}{4.599721in}}%
\pgfpathlineto{\pgfqpoint{3.168438in}{4.453398in}}%
\pgfpathlineto{\pgfqpoint{3.172883in}{4.270301in}}%
\pgfpathlineto{\pgfqpoint{3.172883in}{4.305790in}}%
\pgfpathlineto{\pgfqpoint{3.172883in}{4.305790in}}%
\pgfpathlineto{\pgfqpoint{3.168438in}{4.488727in}}%
\pgfpathlineto{\pgfqpoint{3.163901in}{4.684111in}}%
\pgfpathlineto{\pgfqpoint{3.159267in}{4.863683in}}%
\pgfpathlineto{\pgfqpoint{3.154532in}{5.027984in}}%
\pgfpathlineto{\pgfqpoint{3.149692in}{5.178126in}}%
\pgfpathlineto{\pgfqpoint{3.144742in}{5.315210in}}%
\pgfpathlineto{\pgfqpoint{3.139677in}{5.440273in}}%
\pgfpathlineto{\pgfqpoint{3.134492in}{5.554285in}}%
\pgfpathlineto{\pgfqpoint{3.129180in}{5.658145in}}%
\pgfpathlineto{\pgfqpoint{3.123735in}{5.752689in}}%
\pgfpathlineto{\pgfqpoint{3.118150in}{5.838688in}}%
\pgfpathlineto{\pgfqpoint{3.112419in}{5.916855in}}%
\pgfpathlineto{\pgfqpoint{3.106532in}{5.987846in}}%
\pgfpathlineto{\pgfqpoint{3.100482in}{6.052265in}}%
\pgfpathlineto{\pgfqpoint{3.094259in}{6.110669in}}%
\pgfpathlineto{\pgfqpoint{3.087853in}{6.163568in}}%
\pgfpathlineto{\pgfqpoint{3.081253in}{6.211435in}}%
\pgfpathlineto{\pgfqpoint{3.074446in}{6.254704in}}%
\pgfpathlineto{\pgfqpoint{3.067420in}{6.293780in}}%
\pgfpathlineto{\pgfqpoint{3.060159in}{6.329045in}}%
\pgfpathlineto{\pgfqpoint{3.052648in}{6.360874in}}%
\pgfpathlineto{\pgfqpoint{3.044869in}{6.389680in}}%
\pgfpathlineto{\pgfqpoint{3.036801in}{6.416058in}}%
\pgfpathlineto{\pgfqpoint{3.028423in}{6.441437in}}%
\pgfpathlineto{\pgfqpoint{3.019710in}{6.468449in}}%
\pgfpathlineto{\pgfqpoint{3.010633in}{6.495829in}}%
\pgfpathlineto{\pgfqpoint{3.001162in}{6.520975in}}%
\pgfpathlineto{\pgfqpoint{2.991260in}{6.543135in}}%
\pgfpathlineto{\pgfqpoint{2.980886in}{6.562143in}}%
\pgfpathlineto{\pgfqpoint{2.969993in}{6.577972in}}%
\pgfpathlineto{\pgfqpoint{2.958526in}{6.590634in}}%
\pgfpathlineto{\pgfqpoint{2.946422in}{6.600153in}}%
\pgfpathlineto{\pgfqpoint{2.933605in}{6.606554in}}%
\pgfpathlineto{\pgfqpoint{2.919986in}{6.609866in}}%
\pgfpathlineto{\pgfqpoint{2.905459in}{6.610113in}}%
\pgfpathlineto{\pgfqpoint{2.889892in}{6.607321in}}%
\pgfpathlineto{\pgfqpoint{2.873127in}{6.601522in}}%
\pgfpathlineto{\pgfqpoint{2.854962in}{6.593698in}}%
\pgfpathlineto{\pgfqpoint{2.835143in}{6.600355in}}%
\pgfpathlineto{\pgfqpoint{2.813338in}{6.608220in}}%
\pgfpathlineto{\pgfqpoint{2.789104in}{6.616886in}}%
\pgfpathlineto{\pgfqpoint{2.761831in}{6.626609in}}%
\pgfpathlineto{\pgfqpoint{2.730647in}{6.637679in}}%
\pgfpathlineto{\pgfqpoint{2.694239in}{6.650419in}}%
\pgfpathlineto{\pgfqpoint{2.650497in}{6.665187in}}%
\pgfpathlineto{\pgfqpoint{2.595701in}{6.682382in}}%
\pgfpathlineto{\pgfqpoint{2.522305in}{6.702449in}}%
\pgfpathlineto{\pgfqpoint{2.410774in}{6.725883in}}%
\pgfpathlineto{\pgfqpoint{2.170064in}{6.753237in}}%
\pgfpathclose%
\pgfusepath{stroke,fill}%
\end{pgfscope}%
\begin{pgfscope}%
\pgfpathrectangle{\pgfqpoint{2.170064in}{6.271734in}}{\pgfqpoint{1.223103in}{0.607948in}}%
\pgfusepath{clip}%
\pgfsetroundcap%
\pgfsetroundjoin%
\pgfsetlinewidth{0.501875pt}%
\definecolor{currentstroke}{rgb}{0.000000,0.000000,1.000000}%
\pgfsetstrokecolor{currentstroke}%
\pgfsetstrokeopacity{0.800000}%
\pgfsetdash{}{0pt}%
\pgfpathmoveto{\pgfqpoint{2.170064in}{6.521141in}}%
\pgfpathlineto{\pgfqpoint{3.393168in}{6.521141in}}%
\pgfusepath{stroke}%
\end{pgfscope}%
\begin{pgfscope}%
\pgfpathrectangle{\pgfqpoint{2.170064in}{6.271734in}}{\pgfqpoint{1.223103in}{0.607948in}}%
\pgfusepath{clip}%
\pgfsetbuttcap%
\pgfsetroundjoin%
\pgfsetlinewidth{1.003750pt}%
\definecolor{currentstroke}{rgb}{0.000000,0.000000,0.000000}%
\pgfsetstrokecolor{currentstroke}%
\pgfsetdash{{3.700000pt}{1.600000pt}}{0.000000pt}%
\pgfpathmoveto{\pgfqpoint{2.170064in}{6.595492in}}%
\pgfpathlineto{\pgfqpoint{3.393168in}{6.595492in}}%
\pgfusepath{stroke}%
\end{pgfscope}%
\begin{pgfscope}%
\pgfsetroundcap%
\pgfsetroundjoin%
\pgfsetlinewidth{0.501875pt}%
\definecolor{currentstroke}{rgb}{0.000000,0.000000,1.000000}%
\pgfsetstrokecolor{currentstroke}%
\pgfsetstrokeopacity{0.800000}%
\pgfsetdash{}{0pt}%
\pgfpathmoveto{\pgfqpoint{3.008824in}{6.758634in}}%
\pgfpathquadraticcurveto{\pgfqpoint{2.924194in}{6.650814in}}{\pgfqpoint{2.839564in}{6.542993in}}%
\pgfusepath{stroke}%
\end{pgfscope}%
\begin{pgfscope}%
\pgfsetfillopacity{0.800000}%
\pgfsetstrokeopacity{0.800000}%
\definecolor{textcolor}{rgb}{0.000000,0.000000,1.000000}%
\pgfsetstrokecolor{textcolor}%
\pgfsetfillcolor{textcolor}%
\pgftext[x=2.910706in,y=6.829207in,left,base]{\color{textcolor}\sffamily\fontsize{5.647059}{6.776471}\selectfont 17.41(41)}%
\end{pgfscope}%
\begin{pgfscope}%
\pgfsetbuttcap%
\pgfsetroundjoin%
\definecolor{currentfill}{rgb}{0.150000,0.150000,0.150000}%
\pgfsetfillcolor{currentfill}%
\pgfsetlinewidth{1.003750pt}%
\definecolor{currentstroke}{rgb}{0.150000,0.150000,0.150000}%
\pgfsetstrokecolor{currentstroke}%
\pgfsetdash{}{0pt}%
\pgfsys@defobject{currentmarker}{\pgfqpoint{0.000000in}{-0.066667in}}{\pgfqpoint{0.000000in}{0.000000in}}{%
\pgfpathmoveto{\pgfqpoint{0.000000in}{0.000000in}}%
\pgfpathlineto{\pgfqpoint{0.000000in}{-0.066667in}}%
\pgfusepath{stroke,fill}%
}%
\begin{pgfscope}%
\pgfsys@transformshift{2.170064in}{6.271734in}%
\pgfsys@useobject{currentmarker}{}%
\end{pgfscope}%
\end{pgfscope}%
\begin{pgfscope}%
\pgfsetbuttcap%
\pgfsetroundjoin%
\definecolor{currentfill}{rgb}{0.150000,0.150000,0.150000}%
\pgfsetfillcolor{currentfill}%
\pgfsetlinewidth{1.003750pt}%
\definecolor{currentstroke}{rgb}{0.150000,0.150000,0.150000}%
\pgfsetstrokecolor{currentstroke}%
\pgfsetdash{}{0pt}%
\pgfsys@defobject{currentmarker}{\pgfqpoint{0.000000in}{-0.066667in}}{\pgfqpoint{0.000000in}{0.000000in}}{%
\pgfpathmoveto{\pgfqpoint{0.000000in}{0.000000in}}%
\pgfpathlineto{\pgfqpoint{0.000000in}{-0.066667in}}%
\pgfusepath{stroke,fill}%
}%
\begin{pgfscope}%
\pgfsys@transformshift{2.671473in}{6.271734in}%
\pgfsys@useobject{currentmarker}{}%
\end{pgfscope}%
\end{pgfscope}%
\begin{pgfscope}%
\pgfsetbuttcap%
\pgfsetroundjoin%
\definecolor{currentfill}{rgb}{0.150000,0.150000,0.150000}%
\pgfsetfillcolor{currentfill}%
\pgfsetlinewidth{1.003750pt}%
\definecolor{currentstroke}{rgb}{0.150000,0.150000,0.150000}%
\pgfsetstrokecolor{currentstroke}%
\pgfsetdash{}{0pt}%
\pgfsys@defobject{currentmarker}{\pgfqpoint{0.000000in}{-0.066667in}}{\pgfqpoint{0.000000in}{0.000000in}}{%
\pgfpathmoveto{\pgfqpoint{0.000000in}{0.000000in}}%
\pgfpathlineto{\pgfqpoint{0.000000in}{-0.066667in}}%
\pgfusepath{stroke,fill}%
}%
\begin{pgfscope}%
\pgfsys@transformshift{3.172883in}{6.271734in}%
\pgfsys@useobject{currentmarker}{}%
\end{pgfscope}%
\end{pgfscope}%
\begin{pgfscope}%
\pgfsetbuttcap%
\pgfsetroundjoin%
\definecolor{currentfill}{rgb}{0.150000,0.150000,0.150000}%
\pgfsetfillcolor{currentfill}%
\pgfsetlinewidth{0.803000pt}%
\definecolor{currentstroke}{rgb}{0.150000,0.150000,0.150000}%
\pgfsetstrokecolor{currentstroke}%
\pgfsetdash{}{0pt}%
\pgfsys@defobject{currentmarker}{\pgfqpoint{0.000000in}{-0.044444in}}{\pgfqpoint{0.000000in}{0.000000in}}{%
\pgfpathmoveto{\pgfqpoint{0.000000in}{0.000000in}}%
\pgfpathlineto{\pgfqpoint{0.000000in}{-0.044444in}}%
\pgfusepath{stroke,fill}%
}%
\begin{pgfscope}%
\pgfsys@transformshift{2.321004in}{6.271734in}%
\pgfsys@useobject{currentmarker}{}%
\end{pgfscope}%
\end{pgfscope}%
\begin{pgfscope}%
\pgfsetbuttcap%
\pgfsetroundjoin%
\definecolor{currentfill}{rgb}{0.150000,0.150000,0.150000}%
\pgfsetfillcolor{currentfill}%
\pgfsetlinewidth{0.803000pt}%
\definecolor{currentstroke}{rgb}{0.150000,0.150000,0.150000}%
\pgfsetstrokecolor{currentstroke}%
\pgfsetdash{}{0pt}%
\pgfsys@defobject{currentmarker}{\pgfqpoint{0.000000in}{-0.044444in}}{\pgfqpoint{0.000000in}{0.000000in}}{%
\pgfpathmoveto{\pgfqpoint{0.000000in}{0.000000in}}%
\pgfpathlineto{\pgfqpoint{0.000000in}{-0.044444in}}%
\pgfusepath{stroke,fill}%
}%
\begin{pgfscope}%
\pgfsys@transformshift{2.409297in}{6.271734in}%
\pgfsys@useobject{currentmarker}{}%
\end{pgfscope}%
\end{pgfscope}%
\begin{pgfscope}%
\pgfsetbuttcap%
\pgfsetroundjoin%
\definecolor{currentfill}{rgb}{0.150000,0.150000,0.150000}%
\pgfsetfillcolor{currentfill}%
\pgfsetlinewidth{0.803000pt}%
\definecolor{currentstroke}{rgb}{0.150000,0.150000,0.150000}%
\pgfsetstrokecolor{currentstroke}%
\pgfsetdash{}{0pt}%
\pgfsys@defobject{currentmarker}{\pgfqpoint{0.000000in}{-0.044444in}}{\pgfqpoint{0.000000in}{0.000000in}}{%
\pgfpathmoveto{\pgfqpoint{0.000000in}{0.000000in}}%
\pgfpathlineto{\pgfqpoint{0.000000in}{-0.044444in}}%
\pgfusepath{stroke,fill}%
}%
\begin{pgfscope}%
\pgfsys@transformshift{2.471943in}{6.271734in}%
\pgfsys@useobject{currentmarker}{}%
\end{pgfscope}%
\end{pgfscope}%
\begin{pgfscope}%
\pgfsetbuttcap%
\pgfsetroundjoin%
\definecolor{currentfill}{rgb}{0.150000,0.150000,0.150000}%
\pgfsetfillcolor{currentfill}%
\pgfsetlinewidth{0.803000pt}%
\definecolor{currentstroke}{rgb}{0.150000,0.150000,0.150000}%
\pgfsetstrokecolor{currentstroke}%
\pgfsetdash{}{0pt}%
\pgfsys@defobject{currentmarker}{\pgfqpoint{0.000000in}{-0.044444in}}{\pgfqpoint{0.000000in}{0.000000in}}{%
\pgfpathmoveto{\pgfqpoint{0.000000in}{0.000000in}}%
\pgfpathlineto{\pgfqpoint{0.000000in}{-0.044444in}}%
\pgfusepath{stroke,fill}%
}%
\begin{pgfscope}%
\pgfsys@transformshift{2.520534in}{6.271734in}%
\pgfsys@useobject{currentmarker}{}%
\end{pgfscope}%
\end{pgfscope}%
\begin{pgfscope}%
\pgfsetbuttcap%
\pgfsetroundjoin%
\definecolor{currentfill}{rgb}{0.150000,0.150000,0.150000}%
\pgfsetfillcolor{currentfill}%
\pgfsetlinewidth{0.803000pt}%
\definecolor{currentstroke}{rgb}{0.150000,0.150000,0.150000}%
\pgfsetstrokecolor{currentstroke}%
\pgfsetdash{}{0pt}%
\pgfsys@defobject{currentmarker}{\pgfqpoint{0.000000in}{-0.044444in}}{\pgfqpoint{0.000000in}{0.000000in}}{%
\pgfpathmoveto{\pgfqpoint{0.000000in}{0.000000in}}%
\pgfpathlineto{\pgfqpoint{0.000000in}{-0.044444in}}%
\pgfusepath{stroke,fill}%
}%
\begin{pgfscope}%
\pgfsys@transformshift{2.560237in}{6.271734in}%
\pgfsys@useobject{currentmarker}{}%
\end{pgfscope}%
\end{pgfscope}%
\begin{pgfscope}%
\pgfsetbuttcap%
\pgfsetroundjoin%
\definecolor{currentfill}{rgb}{0.150000,0.150000,0.150000}%
\pgfsetfillcolor{currentfill}%
\pgfsetlinewidth{0.803000pt}%
\definecolor{currentstroke}{rgb}{0.150000,0.150000,0.150000}%
\pgfsetstrokecolor{currentstroke}%
\pgfsetdash{}{0pt}%
\pgfsys@defobject{currentmarker}{\pgfqpoint{0.000000in}{-0.044444in}}{\pgfqpoint{0.000000in}{0.000000in}}{%
\pgfpathmoveto{\pgfqpoint{0.000000in}{0.000000in}}%
\pgfpathlineto{\pgfqpoint{0.000000in}{-0.044444in}}%
\pgfusepath{stroke,fill}%
}%
\begin{pgfscope}%
\pgfsys@transformshift{2.593804in}{6.271734in}%
\pgfsys@useobject{currentmarker}{}%
\end{pgfscope}%
\end{pgfscope}%
\begin{pgfscope}%
\pgfsetbuttcap%
\pgfsetroundjoin%
\definecolor{currentfill}{rgb}{0.150000,0.150000,0.150000}%
\pgfsetfillcolor{currentfill}%
\pgfsetlinewidth{0.803000pt}%
\definecolor{currentstroke}{rgb}{0.150000,0.150000,0.150000}%
\pgfsetstrokecolor{currentstroke}%
\pgfsetdash{}{0pt}%
\pgfsys@defobject{currentmarker}{\pgfqpoint{0.000000in}{-0.044444in}}{\pgfqpoint{0.000000in}{0.000000in}}{%
\pgfpathmoveto{\pgfqpoint{0.000000in}{0.000000in}}%
\pgfpathlineto{\pgfqpoint{0.000000in}{-0.044444in}}%
\pgfusepath{stroke,fill}%
}%
\begin{pgfscope}%
\pgfsys@transformshift{2.622882in}{6.271734in}%
\pgfsys@useobject{currentmarker}{}%
\end{pgfscope}%
\end{pgfscope}%
\begin{pgfscope}%
\pgfsetbuttcap%
\pgfsetroundjoin%
\definecolor{currentfill}{rgb}{0.150000,0.150000,0.150000}%
\pgfsetfillcolor{currentfill}%
\pgfsetlinewidth{0.803000pt}%
\definecolor{currentstroke}{rgb}{0.150000,0.150000,0.150000}%
\pgfsetstrokecolor{currentstroke}%
\pgfsetdash{}{0pt}%
\pgfsys@defobject{currentmarker}{\pgfqpoint{0.000000in}{-0.044444in}}{\pgfqpoint{0.000000in}{0.000000in}}{%
\pgfpathmoveto{\pgfqpoint{0.000000in}{0.000000in}}%
\pgfpathlineto{\pgfqpoint{0.000000in}{-0.044444in}}%
\pgfusepath{stroke,fill}%
}%
\begin{pgfscope}%
\pgfsys@transformshift{2.648530in}{6.271734in}%
\pgfsys@useobject{currentmarker}{}%
\end{pgfscope}%
\end{pgfscope}%
\begin{pgfscope}%
\pgfsetbuttcap%
\pgfsetroundjoin%
\definecolor{currentfill}{rgb}{0.150000,0.150000,0.150000}%
\pgfsetfillcolor{currentfill}%
\pgfsetlinewidth{0.803000pt}%
\definecolor{currentstroke}{rgb}{0.150000,0.150000,0.150000}%
\pgfsetstrokecolor{currentstroke}%
\pgfsetdash{}{0pt}%
\pgfsys@defobject{currentmarker}{\pgfqpoint{0.000000in}{-0.044444in}}{\pgfqpoint{0.000000in}{0.000000in}}{%
\pgfpathmoveto{\pgfqpoint{0.000000in}{0.000000in}}%
\pgfpathlineto{\pgfqpoint{0.000000in}{-0.044444in}}%
\pgfusepath{stroke,fill}%
}%
\begin{pgfscope}%
\pgfsys@transformshift{2.822413in}{6.271734in}%
\pgfsys@useobject{currentmarker}{}%
\end{pgfscope}%
\end{pgfscope}%
\begin{pgfscope}%
\pgfsetbuttcap%
\pgfsetroundjoin%
\definecolor{currentfill}{rgb}{0.150000,0.150000,0.150000}%
\pgfsetfillcolor{currentfill}%
\pgfsetlinewidth{0.803000pt}%
\definecolor{currentstroke}{rgb}{0.150000,0.150000,0.150000}%
\pgfsetstrokecolor{currentstroke}%
\pgfsetdash{}{0pt}%
\pgfsys@defobject{currentmarker}{\pgfqpoint{0.000000in}{-0.044444in}}{\pgfqpoint{0.000000in}{0.000000in}}{%
\pgfpathmoveto{\pgfqpoint{0.000000in}{0.000000in}}%
\pgfpathlineto{\pgfqpoint{0.000000in}{-0.044444in}}%
\pgfusepath{stroke,fill}%
}%
\begin{pgfscope}%
\pgfsys@transformshift{2.910706in}{6.271734in}%
\pgfsys@useobject{currentmarker}{}%
\end{pgfscope}%
\end{pgfscope}%
\begin{pgfscope}%
\pgfsetbuttcap%
\pgfsetroundjoin%
\definecolor{currentfill}{rgb}{0.150000,0.150000,0.150000}%
\pgfsetfillcolor{currentfill}%
\pgfsetlinewidth{0.803000pt}%
\definecolor{currentstroke}{rgb}{0.150000,0.150000,0.150000}%
\pgfsetstrokecolor{currentstroke}%
\pgfsetdash{}{0pt}%
\pgfsys@defobject{currentmarker}{\pgfqpoint{0.000000in}{-0.044444in}}{\pgfqpoint{0.000000in}{0.000000in}}{%
\pgfpathmoveto{\pgfqpoint{0.000000in}{0.000000in}}%
\pgfpathlineto{\pgfqpoint{0.000000in}{-0.044444in}}%
\pgfusepath{stroke,fill}%
}%
\begin{pgfscope}%
\pgfsys@transformshift{2.973352in}{6.271734in}%
\pgfsys@useobject{currentmarker}{}%
\end{pgfscope}%
\end{pgfscope}%
\begin{pgfscope}%
\pgfsetbuttcap%
\pgfsetroundjoin%
\definecolor{currentfill}{rgb}{0.150000,0.150000,0.150000}%
\pgfsetfillcolor{currentfill}%
\pgfsetlinewidth{0.803000pt}%
\definecolor{currentstroke}{rgb}{0.150000,0.150000,0.150000}%
\pgfsetstrokecolor{currentstroke}%
\pgfsetdash{}{0pt}%
\pgfsys@defobject{currentmarker}{\pgfqpoint{0.000000in}{-0.044444in}}{\pgfqpoint{0.000000in}{0.000000in}}{%
\pgfpathmoveto{\pgfqpoint{0.000000in}{0.000000in}}%
\pgfpathlineto{\pgfqpoint{0.000000in}{-0.044444in}}%
\pgfusepath{stroke,fill}%
}%
\begin{pgfscope}%
\pgfsys@transformshift{3.021943in}{6.271734in}%
\pgfsys@useobject{currentmarker}{}%
\end{pgfscope}%
\end{pgfscope}%
\begin{pgfscope}%
\pgfsetbuttcap%
\pgfsetroundjoin%
\definecolor{currentfill}{rgb}{0.150000,0.150000,0.150000}%
\pgfsetfillcolor{currentfill}%
\pgfsetlinewidth{0.803000pt}%
\definecolor{currentstroke}{rgb}{0.150000,0.150000,0.150000}%
\pgfsetstrokecolor{currentstroke}%
\pgfsetdash{}{0pt}%
\pgfsys@defobject{currentmarker}{\pgfqpoint{0.000000in}{-0.044444in}}{\pgfqpoint{0.000000in}{0.000000in}}{%
\pgfpathmoveto{\pgfqpoint{0.000000in}{0.000000in}}%
\pgfpathlineto{\pgfqpoint{0.000000in}{-0.044444in}}%
\pgfusepath{stroke,fill}%
}%
\begin{pgfscope}%
\pgfsys@transformshift{3.061646in}{6.271734in}%
\pgfsys@useobject{currentmarker}{}%
\end{pgfscope}%
\end{pgfscope}%
\begin{pgfscope}%
\pgfsetbuttcap%
\pgfsetroundjoin%
\definecolor{currentfill}{rgb}{0.150000,0.150000,0.150000}%
\pgfsetfillcolor{currentfill}%
\pgfsetlinewidth{0.803000pt}%
\definecolor{currentstroke}{rgb}{0.150000,0.150000,0.150000}%
\pgfsetstrokecolor{currentstroke}%
\pgfsetdash{}{0pt}%
\pgfsys@defobject{currentmarker}{\pgfqpoint{0.000000in}{-0.044444in}}{\pgfqpoint{0.000000in}{0.000000in}}{%
\pgfpathmoveto{\pgfqpoint{0.000000in}{0.000000in}}%
\pgfpathlineto{\pgfqpoint{0.000000in}{-0.044444in}}%
\pgfusepath{stroke,fill}%
}%
\begin{pgfscope}%
\pgfsys@transformshift{3.095213in}{6.271734in}%
\pgfsys@useobject{currentmarker}{}%
\end{pgfscope}%
\end{pgfscope}%
\begin{pgfscope}%
\pgfsetbuttcap%
\pgfsetroundjoin%
\definecolor{currentfill}{rgb}{0.150000,0.150000,0.150000}%
\pgfsetfillcolor{currentfill}%
\pgfsetlinewidth{0.803000pt}%
\definecolor{currentstroke}{rgb}{0.150000,0.150000,0.150000}%
\pgfsetstrokecolor{currentstroke}%
\pgfsetdash{}{0pt}%
\pgfsys@defobject{currentmarker}{\pgfqpoint{0.000000in}{-0.044444in}}{\pgfqpoint{0.000000in}{0.000000in}}{%
\pgfpathmoveto{\pgfqpoint{0.000000in}{0.000000in}}%
\pgfpathlineto{\pgfqpoint{0.000000in}{-0.044444in}}%
\pgfusepath{stroke,fill}%
}%
\begin{pgfscope}%
\pgfsys@transformshift{3.124291in}{6.271734in}%
\pgfsys@useobject{currentmarker}{}%
\end{pgfscope}%
\end{pgfscope}%
\begin{pgfscope}%
\pgfsetbuttcap%
\pgfsetroundjoin%
\definecolor{currentfill}{rgb}{0.150000,0.150000,0.150000}%
\pgfsetfillcolor{currentfill}%
\pgfsetlinewidth{0.803000pt}%
\definecolor{currentstroke}{rgb}{0.150000,0.150000,0.150000}%
\pgfsetstrokecolor{currentstroke}%
\pgfsetdash{}{0pt}%
\pgfsys@defobject{currentmarker}{\pgfqpoint{0.000000in}{-0.044444in}}{\pgfqpoint{0.000000in}{0.000000in}}{%
\pgfpathmoveto{\pgfqpoint{0.000000in}{0.000000in}}%
\pgfpathlineto{\pgfqpoint{0.000000in}{-0.044444in}}%
\pgfusepath{stroke,fill}%
}%
\begin{pgfscope}%
\pgfsys@transformshift{3.149939in}{6.271734in}%
\pgfsys@useobject{currentmarker}{}%
\end{pgfscope}%
\end{pgfscope}%
\begin{pgfscope}%
\pgfsetbuttcap%
\pgfsetroundjoin%
\definecolor{currentfill}{rgb}{0.150000,0.150000,0.150000}%
\pgfsetfillcolor{currentfill}%
\pgfsetlinewidth{0.803000pt}%
\definecolor{currentstroke}{rgb}{0.150000,0.150000,0.150000}%
\pgfsetstrokecolor{currentstroke}%
\pgfsetdash{}{0pt}%
\pgfsys@defobject{currentmarker}{\pgfqpoint{0.000000in}{-0.044444in}}{\pgfqpoint{0.000000in}{0.000000in}}{%
\pgfpathmoveto{\pgfqpoint{0.000000in}{0.000000in}}%
\pgfpathlineto{\pgfqpoint{0.000000in}{-0.044444in}}%
\pgfusepath{stroke,fill}%
}%
\begin{pgfscope}%
\pgfsys@transformshift{3.323822in}{6.271734in}%
\pgfsys@useobject{currentmarker}{}%
\end{pgfscope}%
\end{pgfscope}%
\begin{pgfscope}%
\pgfsetbuttcap%
\pgfsetroundjoin%
\definecolor{currentfill}{rgb}{0.150000,0.150000,0.150000}%
\pgfsetfillcolor{currentfill}%
\pgfsetlinewidth{1.003750pt}%
\definecolor{currentstroke}{rgb}{0.150000,0.150000,0.150000}%
\pgfsetstrokecolor{currentstroke}%
\pgfsetdash{}{0pt}%
\pgfsys@defobject{currentmarker}{\pgfqpoint{-0.066667in}{0.000000in}}{\pgfqpoint{0.000000in}{0.000000in}}{%
\pgfpathmoveto{\pgfqpoint{0.000000in}{0.000000in}}%
\pgfpathlineto{\pgfqpoint{-0.066667in}{0.000000in}}%
\pgfusepath{stroke,fill}%
}%
\begin{pgfscope}%
\pgfsys@transformshift{2.170064in}{6.271734in}%
\pgfsys@useobject{currentmarker}{}%
\end{pgfscope}%
\end{pgfscope}%
\begin{pgfscope}%
\pgfsetbuttcap%
\pgfsetroundjoin%
\definecolor{currentfill}{rgb}{0.150000,0.150000,0.150000}%
\pgfsetfillcolor{currentfill}%
\pgfsetlinewidth{1.003750pt}%
\definecolor{currentstroke}{rgb}{0.150000,0.150000,0.150000}%
\pgfsetstrokecolor{currentstroke}%
\pgfsetdash{}{0pt}%
\pgfsys@defobject{currentmarker}{\pgfqpoint{-0.066667in}{0.000000in}}{\pgfqpoint{0.000000in}{0.000000in}}{%
\pgfpathmoveto{\pgfqpoint{0.000000in}{0.000000in}}%
\pgfpathlineto{\pgfqpoint{-0.066667in}{0.000000in}}%
\pgfusepath{stroke,fill}%
}%
\begin{pgfscope}%
\pgfsys@transformshift{2.170064in}{6.595492in}%
\pgfsys@useobject{currentmarker}{}%
\end{pgfscope}%
\end{pgfscope}%
\begin{pgfscope}%
\pgfsetbuttcap%
\pgfsetroundjoin%
\definecolor{currentfill}{rgb}{0.150000,0.150000,0.150000}%
\pgfsetfillcolor{currentfill}%
\pgfsetlinewidth{1.003750pt}%
\definecolor{currentstroke}{rgb}{0.150000,0.150000,0.150000}%
\pgfsetstrokecolor{currentstroke}%
\pgfsetdash{}{0pt}%
\pgfsys@defobject{currentmarker}{\pgfqpoint{-0.066667in}{0.000000in}}{\pgfqpoint{0.000000in}{0.000000in}}{%
\pgfpathmoveto{\pgfqpoint{0.000000in}{0.000000in}}%
\pgfpathlineto{\pgfqpoint{-0.066667in}{0.000000in}}%
\pgfusepath{stroke,fill}%
}%
\begin{pgfscope}%
\pgfsys@transformshift{2.170064in}{6.879682in}%
\pgfsys@useobject{currentmarker}{}%
\end{pgfscope}%
\end{pgfscope}%
\begin{pgfscope}%
\pgfpathrectangle{\pgfqpoint{2.170064in}{6.271734in}}{\pgfqpoint{1.223103in}{0.607948in}}%
\pgfusepath{clip}%
\pgfsetroundcap%
\pgfsetroundjoin%
\pgfsetlinewidth{1.204500pt}%
\definecolor{currentstroke}{rgb}{0.000000,0.501961,0.000000}%
\pgfsetstrokecolor{currentstroke}%
\pgfsetdash{}{0pt}%
\pgfpathmoveto{\pgfqpoint{2.170064in}{6.524762in}}%
\pgfpathlineto{\pgfqpoint{2.410774in}{6.532426in}}%
\pgfpathlineto{\pgfqpoint{2.522305in}{6.540365in}}%
\pgfpathlineto{\pgfqpoint{2.595701in}{6.548358in}}%
\pgfpathlineto{\pgfqpoint{2.650497in}{6.556205in}}%
\pgfpathlineto{\pgfqpoint{2.694239in}{6.563718in}}%
\pgfpathlineto{\pgfqpoint{2.730647in}{6.570722in}}%
\pgfpathlineto{\pgfqpoint{2.761831in}{6.577056in}}%
\pgfpathlineto{\pgfqpoint{2.789104in}{6.582568in}}%
\pgfpathlineto{\pgfqpoint{2.813338in}{6.587117in}}%
\pgfpathlineto{\pgfqpoint{2.835143in}{6.590573in}}%
\pgfpathlineto{\pgfqpoint{2.854962in}{6.592808in}}%
\pgfpathlineto{\pgfqpoint{2.873127in}{6.593706in}}%
\pgfpathlineto{\pgfqpoint{2.889892in}{6.593152in}}%
\pgfpathlineto{\pgfqpoint{2.905459in}{6.591037in}}%
\pgfpathlineto{\pgfqpoint{2.919986in}{6.587255in}}%
\pgfpathlineto{\pgfqpoint{2.933605in}{6.581700in}}%
\pgfpathlineto{\pgfqpoint{2.946422in}{6.574268in}}%
\pgfpathlineto{\pgfqpoint{2.958526in}{6.564852in}}%
\pgfpathlineto{\pgfqpoint{2.969993in}{6.553346in}}%
\pgfpathlineto{\pgfqpoint{2.980886in}{6.539638in}}%
\pgfpathlineto{\pgfqpoint{2.991260in}{6.523613in}}%
\pgfpathlineto{\pgfqpoint{3.001162in}{6.505150in}}%
\pgfpathlineto{\pgfqpoint{3.010633in}{6.484121in}}%
\pgfpathlineto{\pgfqpoint{3.019710in}{6.460391in}}%
\pgfpathlineto{\pgfqpoint{3.028423in}{6.433816in}}%
\pgfpathlineto{\pgfqpoint{3.036801in}{6.404239in}}%
\pgfpathlineto{\pgfqpoint{3.044869in}{6.371494in}}%
\pgfpathlineto{\pgfqpoint{3.052648in}{6.335403in}}%
\pgfpathlineto{\pgfqpoint{3.060159in}{6.295771in}}%
\pgfpathlineto{\pgfqpoint{3.064740in}{6.268401in}}%
\pgfusepath{stroke}%
\end{pgfscope}%
\begin{pgfscope}%
\pgfsetrectcap%
\pgfsetmiterjoin%
\pgfsetlinewidth{1.003750pt}%
\definecolor{currentstroke}{rgb}{0.150000,0.150000,0.150000}%
\pgfsetstrokecolor{currentstroke}%
\pgfsetdash{}{0pt}%
\pgfpathmoveto{\pgfqpoint{2.170064in}{6.271734in}}%
\pgfpathlineto{\pgfqpoint{2.170064in}{6.879682in}}%
\pgfusepath{stroke}%
\end{pgfscope}%
\begin{pgfscope}%
\pgfsetrectcap%
\pgfsetmiterjoin%
\pgfsetlinewidth{1.003750pt}%
\definecolor{currentstroke}{rgb}{0.150000,0.150000,0.150000}%
\pgfsetstrokecolor{currentstroke}%
\pgfsetdash{}{0pt}%
\pgfpathmoveto{\pgfqpoint{2.170064in}{6.271734in}}%
\pgfpathlineto{\pgfqpoint{3.393168in}{6.271734in}}%
\pgfusepath{stroke}%
\end{pgfscope}%
\begin{pgfscope}%
\pgfpathrectangle{\pgfqpoint{2.170064in}{6.271734in}}{\pgfqpoint{1.223103in}{0.607948in}}%
\pgfusepath{clip}%
\pgfsetbuttcap%
\pgfsetroundjoin%
\definecolor{currentfill}{rgb}{0.000000,0.000000,0.000000}%
\pgfsetfillcolor{currentfill}%
\pgfsetlinewidth{1.003750pt}%
\definecolor{currentstroke}{rgb}{0.000000,0.000000,0.000000}%
\pgfsetstrokecolor{currentstroke}%
\pgfsetdash{}{0pt}%
\pgfsys@defobject{currentmarker}{\pgfqpoint{-0.013889in}{-0.013889in}}{\pgfqpoint{0.013889in}{0.013889in}}{%
\pgfpathmoveto{\pgfqpoint{0.000000in}{-0.013889in}}%
\pgfpathcurveto{\pgfqpoint{0.003683in}{-0.013889in}}{\pgfqpoint{0.007216in}{-0.012425in}}{\pgfqpoint{0.009821in}{-0.009821in}}%
\pgfpathcurveto{\pgfqpoint{0.012425in}{-0.007216in}}{\pgfqpoint{0.013889in}{-0.003683in}}{\pgfqpoint{0.013889in}{0.000000in}}%
\pgfpathcurveto{\pgfqpoint{0.013889in}{0.003683in}}{\pgfqpoint{0.012425in}{0.007216in}}{\pgfqpoint{0.009821in}{0.009821in}}%
\pgfpathcurveto{\pgfqpoint{0.007216in}{0.012425in}}{\pgfqpoint{0.003683in}{0.013889in}}{\pgfqpoint{0.000000in}{0.013889in}}%
\pgfpathcurveto{\pgfqpoint{-0.003683in}{0.013889in}}{\pgfqpoint{-0.007216in}{0.012425in}}{\pgfqpoint{-0.009821in}{0.009821in}}%
\pgfpathcurveto{\pgfqpoint{-0.012425in}{0.007216in}}{\pgfqpoint{-0.013889in}{0.003683in}}{\pgfqpoint{-0.013889in}{0.000000in}}%
\pgfpathcurveto{\pgfqpoint{-0.013889in}{-0.003683in}}{\pgfqpoint{-0.012425in}{-0.007216in}}{\pgfqpoint{-0.009821in}{-0.009821in}}%
\pgfpathcurveto{\pgfqpoint{-0.007216in}{-0.012425in}}{\pgfqpoint{-0.003683in}{-0.013889in}}{\pgfqpoint{0.000000in}{-0.013889in}}%
\pgfpathclose%
\pgfusepath{stroke,fill}%
}%
\begin{pgfscope}%
\pgfsys@transformshift{3.172883in}{4.277394in}%
\pgfsys@useobject{currentmarker}{}%
\end{pgfscope}%
\begin{pgfscope}%
\pgfsys@transformshift{3.084589in}{6.165529in}%
\pgfsys@useobject{currentmarker}{}%
\end{pgfscope}%
\begin{pgfscope}%
\pgfsys@transformshift{3.021943in}{6.445076in}%
\pgfsys@useobject{currentmarker}{}%
\end{pgfscope}%
\begin{pgfscope}%
\pgfsys@transformshift{2.973352in}{6.535009in}%
\pgfsys@useobject{currentmarker}{}%
\end{pgfscope}%
\begin{pgfscope}%
\pgfsys@transformshift{2.933650in}{6.569136in}%
\pgfsys@useobject{currentmarker}{}%
\end{pgfscope}%
\begin{pgfscope}%
\pgfsys@transformshift{2.900082in}{6.583894in}%
\pgfsys@useobject{currentmarker}{}%
\end{pgfscope}%
\begin{pgfscope}%
\pgfsys@transformshift{2.871004in}{6.590913in}%
\pgfsys@useobject{currentmarker}{}%
\end{pgfscope}%
\begin{pgfscope}%
\pgfsys@transformshift{2.865627in}{6.591824in}%
\pgfsys@useobject{currentmarker}{}%
\end{pgfscope}%
\begin{pgfscope}%
\pgfsys@transformshift{2.860380in}{6.592622in}%
\pgfsys@useobject{currentmarker}{}%
\end{pgfscope}%
\begin{pgfscope}%
\pgfsys@transformshift{2.855256in}{6.593322in}%
\pgfsys@useobject{currentmarker}{}%
\end{pgfscope}%
\begin{pgfscope}%
\pgfsys@transformshift{2.850250in}{6.593938in}%
\pgfsys@useobject{currentmarker}{}%
\end{pgfscope}%
\begin{pgfscope}%
\pgfsys@transformshift{2.845356in}{6.594480in}%
\pgfsys@useobject{currentmarker}{}%
\end{pgfscope}%
\begin{pgfscope}%
\pgfsys@transformshift{2.840570in}{6.594957in}%
\pgfsys@useobject{currentmarker}{}%
\end{pgfscope}%
\begin{pgfscope}%
\pgfsys@transformshift{2.835887in}{6.595379in}%
\pgfsys@useobject{currentmarker}{}%
\end{pgfscope}%
\begin{pgfscope}%
\pgfsys@transformshift{2.831302in}{6.595751in}%
\pgfsys@useobject{currentmarker}{}%
\end{pgfscope}%
\begin{pgfscope}%
\pgfsys@transformshift{2.826812in}{6.596081in}%
\pgfsys@useobject{currentmarker}{}%
\end{pgfscope}%
\begin{pgfscope}%
\pgfsys@transformshift{2.822413in}{6.596372in}%
\pgfsys@useobject{currentmarker}{}%
\end{pgfscope}%
\end{pgfscope}%
\begin{pgfscope}%
\definecolor{textcolor}{rgb}{0.150000,0.150000,0.150000}%
\pgfsetstrokecolor{textcolor}%
\pgfsetfillcolor{textcolor}%
\pgftext[x=2.781616in,y=6.963016in,,base]{\color{textcolor}\sffamily\fontsize{5.647059}{6.776471}\selectfont \(\displaystyle n_{\mathrm{step}} = 2\)}%
\end{pgfscope}%
\begin{pgfscope}%
\pgfsetbuttcap%
\pgfsetmiterjoin%
\definecolor{currentfill}{rgb}{1.000000,1.000000,1.000000}%
\pgfsetfillcolor{currentfill}%
\pgfsetlinewidth{0.000000pt}%
\definecolor{currentstroke}{rgb}{0.000000,0.000000,0.000000}%
\pgfsetstrokecolor{currentstroke}%
\pgfsetstrokeopacity{0.000000}%
\pgfsetdash{}{0pt}%
\pgfpathmoveto{\pgfqpoint{3.637789in}{6.271734in}}%
\pgfpathlineto{\pgfqpoint{4.860892in}{6.271734in}}%
\pgfpathlineto{\pgfqpoint{4.860892in}{6.879682in}}%
\pgfpathlineto{\pgfqpoint{3.637789in}{6.879682in}}%
\pgfpathclose%
\pgfusepath{fill}%
\end{pgfscope}%
\begin{pgfscope}%
\pgfpathrectangle{\pgfqpoint{3.637789in}{6.271734in}}{\pgfqpoint{1.223103in}{0.607948in}}%
\pgfusepath{clip}%
\pgfsetbuttcap%
\pgfsetmiterjoin%
\definecolor{currentfill}{rgb}{0.000000,0.000000,1.000000}%
\pgfsetfillcolor{currentfill}%
\pgfsetfillopacity{0.100000}%
\pgfsetlinewidth{0.803000pt}%
\definecolor{currentstroke}{rgb}{0.000000,0.000000,1.000000}%
\pgfsetstrokecolor{currentstroke}%
\pgfsetstrokeopacity{0.100000}%
\pgfsetdash{}{0pt}%
\pgfpathmoveto{\pgfqpoint{3.637789in}{6.434257in}}%
\pgfpathlineto{\pgfqpoint{3.637789in}{6.685605in}}%
\pgfpathlineto{\pgfqpoint{4.860892in}{6.685605in}}%
\pgfpathlineto{\pgfqpoint{4.860892in}{6.434257in}}%
\pgfpathclose%
\pgfusepath{stroke,fill}%
\end{pgfscope}%
\begin{pgfscope}%
\pgfpathrectangle{\pgfqpoint{3.637789in}{6.271734in}}{\pgfqpoint{1.223103in}{0.607948in}}%
\pgfusepath{clip}%
\pgfsetbuttcap%
\pgfsetroundjoin%
\definecolor{currentfill}{rgb}{0.000000,0.501961,0.000000}%
\pgfsetfillcolor{currentfill}%
\pgfsetfillopacity{0.500000}%
\pgfsetlinewidth{0.803000pt}%
\definecolor{currentstroke}{rgb}{0.000000,0.501961,0.000000}%
\pgfsetstrokecolor{currentstroke}%
\pgfsetstrokeopacity{0.500000}%
\pgfsetdash{}{0pt}%
\pgfpathmoveto{\pgfqpoint{3.637789in}{6.679345in}}%
\pgfpathlineto{\pgfqpoint{3.637789in}{6.445459in}}%
\pgfpathlineto{\pgfqpoint{3.878498in}{6.466869in}}%
\pgfpathlineto{\pgfqpoint{3.990029in}{6.486726in}}%
\pgfpathlineto{\pgfqpoint{4.063425in}{6.505122in}}%
\pgfpathlineto{\pgfqpoint{4.118221in}{6.522137in}}%
\pgfpathlineto{\pgfqpoint{4.161963in}{6.537843in}}%
\pgfpathlineto{\pgfqpoint{4.198371in}{6.552303in}}%
\pgfpathlineto{\pgfqpoint{4.229555in}{6.565569in}}%
\pgfpathlineto{\pgfqpoint{4.256828in}{6.577685in}}%
\pgfpathlineto{\pgfqpoint{4.281062in}{6.588687in}}%
\pgfpathlineto{\pgfqpoint{4.302867in}{6.598596in}}%
\pgfpathlineto{\pgfqpoint{4.322686in}{6.606040in}}%
\pgfpathlineto{\pgfqpoint{4.340851in}{6.604653in}}%
\pgfpathlineto{\pgfqpoint{4.357617in}{6.603731in}}%
\pgfpathlineto{\pgfqpoint{4.373183in}{6.603346in}}%
\pgfpathlineto{\pgfqpoint{4.387711in}{6.603431in}}%
\pgfpathlineto{\pgfqpoint{4.401329in}{6.603907in}}%
\pgfpathlineto{\pgfqpoint{4.414146in}{6.604691in}}%
\pgfpathlineto{\pgfqpoint{4.426250in}{6.605692in}}%
\pgfpathlineto{\pgfqpoint{4.437717in}{6.606814in}}%
\pgfpathlineto{\pgfqpoint{4.448610in}{6.607950in}}%
\pgfpathlineto{\pgfqpoint{4.458984in}{6.608988in}}%
\pgfpathlineto{\pgfqpoint{4.468886in}{6.609803in}}%
\pgfpathlineto{\pgfqpoint{4.478357in}{6.610256in}}%
\pgfpathlineto{\pgfqpoint{4.487434in}{6.610171in}}%
\pgfpathlineto{\pgfqpoint{4.496147in}{6.609249in}}%
\pgfpathlineto{\pgfqpoint{4.504525in}{6.606356in}}%
\pgfpathlineto{\pgfqpoint{4.512593in}{6.598780in}}%
\pgfpathlineto{\pgfqpoint{4.520372in}{6.588090in}}%
\pgfpathlineto{\pgfqpoint{4.527883in}{6.575575in}}%
\pgfpathlineto{\pgfqpoint{4.535144in}{6.561402in}}%
\pgfpathlineto{\pgfqpoint{4.542170in}{6.545569in}}%
\pgfpathlineto{\pgfqpoint{4.548977in}{6.528037in}}%
\pgfpathlineto{\pgfqpoint{4.555577in}{6.508756in}}%
\pgfpathlineto{\pgfqpoint{4.561983in}{6.487669in}}%
\pgfpathlineto{\pgfqpoint{4.568206in}{6.464720in}}%
\pgfpathlineto{\pgfqpoint{4.574256in}{6.439847in}}%
\pgfpathlineto{\pgfqpoint{4.580143in}{6.412989in}}%
\pgfpathlineto{\pgfqpoint{4.585874in}{6.384083in}}%
\pgfpathlineto{\pgfqpoint{4.591459in}{6.353064in}}%
\pgfpathlineto{\pgfqpoint{4.596904in}{6.319868in}}%
\pgfpathlineto{\pgfqpoint{4.602216in}{6.284428in}}%
\pgfpathlineto{\pgfqpoint{4.607401in}{6.246678in}}%
\pgfpathlineto{\pgfqpoint{4.612466in}{6.206550in}}%
\pgfpathlineto{\pgfqpoint{4.617416in}{6.163975in}}%
\pgfpathlineto{\pgfqpoint{4.622256in}{6.118879in}}%
\pgfpathlineto{\pgfqpoint{4.626991in}{6.071183in}}%
\pgfpathlineto{\pgfqpoint{4.631625in}{6.020767in}}%
\pgfpathlineto{\pgfqpoint{4.636162in}{5.966784in}}%
\pgfpathlineto{\pgfqpoint{4.640607in}{5.891473in}}%
\pgfpathlineto{\pgfqpoint{4.640607in}{5.912609in}}%
\pgfpathlineto{\pgfqpoint{4.640607in}{5.912609in}}%
\pgfpathlineto{\pgfqpoint{4.636162in}{5.974192in}}%
\pgfpathlineto{\pgfqpoint{4.631625in}{6.047520in}}%
\pgfpathlineto{\pgfqpoint{4.626991in}{6.115274in}}%
\pgfpathlineto{\pgfqpoint{4.622256in}{6.177043in}}%
\pgfpathlineto{\pgfqpoint{4.617416in}{6.233129in}}%
\pgfpathlineto{\pgfqpoint{4.612466in}{6.283863in}}%
\pgfpathlineto{\pgfqpoint{4.607401in}{6.329574in}}%
\pgfpathlineto{\pgfqpoint{4.602216in}{6.370582in}}%
\pgfpathlineto{\pgfqpoint{4.596904in}{6.407198in}}%
\pgfpathlineto{\pgfqpoint{4.591459in}{6.439724in}}%
\pgfpathlineto{\pgfqpoint{4.585874in}{6.468450in}}%
\pgfpathlineto{\pgfqpoint{4.580143in}{6.493658in}}%
\pgfpathlineto{\pgfqpoint{4.574256in}{6.515619in}}%
\pgfpathlineto{\pgfqpoint{4.568206in}{6.534594in}}%
\pgfpathlineto{\pgfqpoint{4.561983in}{6.550835in}}%
\pgfpathlineto{\pgfqpoint{4.555577in}{6.564584in}}%
\pgfpathlineto{\pgfqpoint{4.548977in}{6.576075in}}%
\pgfpathlineto{\pgfqpoint{4.542170in}{6.585533in}}%
\pgfpathlineto{\pgfqpoint{4.535144in}{6.593178in}}%
\pgfpathlineto{\pgfqpoint{4.527883in}{6.599231in}}%
\pgfpathlineto{\pgfqpoint{4.520372in}{6.603938in}}%
\pgfpathlineto{\pgfqpoint{4.512593in}{6.607703in}}%
\pgfpathlineto{\pgfqpoint{4.504525in}{6.612043in}}%
\pgfpathlineto{\pgfqpoint{4.496147in}{6.618741in}}%
\pgfpathlineto{\pgfqpoint{4.487434in}{6.625293in}}%
\pgfpathlineto{\pgfqpoint{4.478357in}{6.630758in}}%
\pgfpathlineto{\pgfqpoint{4.468886in}{6.635021in}}%
\pgfpathlineto{\pgfqpoint{4.458984in}{6.638080in}}%
\pgfpathlineto{\pgfqpoint{4.448610in}{6.639956in}}%
\pgfpathlineto{\pgfqpoint{4.437717in}{6.640678in}}%
\pgfpathlineto{\pgfqpoint{4.426250in}{6.640272in}}%
\pgfpathlineto{\pgfqpoint{4.414146in}{6.638764in}}%
\pgfpathlineto{\pgfqpoint{4.401329in}{6.636176in}}%
\pgfpathlineto{\pgfqpoint{4.387711in}{6.632526in}}%
\pgfpathlineto{\pgfqpoint{4.373183in}{6.627828in}}%
\pgfpathlineto{\pgfqpoint{4.357617in}{6.622091in}}%
\pgfpathlineto{\pgfqpoint{4.340851in}{6.615325in}}%
\pgfpathlineto{\pgfqpoint{4.322686in}{6.607668in}}%
\pgfpathlineto{\pgfqpoint{4.302867in}{6.608470in}}%
\pgfpathlineto{\pgfqpoint{4.281062in}{6.611410in}}%
\pgfpathlineto{\pgfqpoint{4.256828in}{6.615152in}}%
\pgfpathlineto{\pgfqpoint{4.229555in}{6.619739in}}%
\pgfpathlineto{\pgfqpoint{4.198371in}{6.625219in}}%
\pgfpathlineto{\pgfqpoint{4.161963in}{6.631639in}}%
\pgfpathlineto{\pgfqpoint{4.118221in}{6.639043in}}%
\pgfpathlineto{\pgfqpoint{4.063425in}{6.647474in}}%
\pgfpathlineto{\pgfqpoint{3.990029in}{6.656974in}}%
\pgfpathlineto{\pgfqpoint{3.878498in}{6.667584in}}%
\pgfpathlineto{\pgfqpoint{3.637789in}{6.679345in}}%
\pgfpathclose%
\pgfusepath{stroke,fill}%
\end{pgfscope}%
\begin{pgfscope}%
\pgfpathrectangle{\pgfqpoint{3.637789in}{6.271734in}}{\pgfqpoint{1.223103in}{0.607948in}}%
\pgfusepath{clip}%
\pgfsetroundcap%
\pgfsetroundjoin%
\pgfsetlinewidth{0.501875pt}%
\definecolor{currentstroke}{rgb}{0.000000,0.000000,1.000000}%
\pgfsetstrokecolor{currentstroke}%
\pgfsetstrokeopacity{0.800000}%
\pgfsetdash{}{0pt}%
\pgfpathmoveto{\pgfqpoint{3.637789in}{6.559931in}}%
\pgfpathlineto{\pgfqpoint{4.860892in}{6.559931in}}%
\pgfusepath{stroke}%
\end{pgfscope}%
\begin{pgfscope}%
\pgfpathrectangle{\pgfqpoint{3.637789in}{6.271734in}}{\pgfqpoint{1.223103in}{0.607948in}}%
\pgfusepath{clip}%
\pgfsetbuttcap%
\pgfsetroundjoin%
\pgfsetlinewidth{1.003750pt}%
\definecolor{currentstroke}{rgb}{0.000000,0.000000,0.000000}%
\pgfsetstrokecolor{currentstroke}%
\pgfsetdash{{3.700000pt}{1.600000pt}}{0.000000pt}%
\pgfpathmoveto{\pgfqpoint{3.637789in}{6.595492in}}%
\pgfpathlineto{\pgfqpoint{4.860892in}{6.595492in}}%
\pgfusepath{stroke}%
\end{pgfscope}%
\begin{pgfscope}%
\pgfsetroundcap%
\pgfsetroundjoin%
\pgfsetlinewidth{0.501875pt}%
\definecolor{currentstroke}{rgb}{0.000000,0.000000,1.000000}%
\pgfsetstrokecolor{currentstroke}%
\pgfsetstrokeopacity{0.800000}%
\pgfsetdash{}{0pt}%
\pgfpathmoveto{\pgfqpoint{4.440385in}{6.680317in}}%
\pgfpathquadraticcurveto{\pgfqpoint{4.376096in}{6.628805in}}{\pgfqpoint{4.311806in}{6.577294in}}%
\pgfusepath{stroke}%
\end{pgfscope}%
\begin{pgfscope}%
\pgfsetfillopacity{0.800000}%
\pgfsetstrokeopacity{0.800000}%
\definecolor{textcolor}{rgb}{0.000000,0.000000,1.000000}%
\pgfsetstrokecolor{textcolor}%
\pgfsetfillcolor{textcolor}%
\pgftext[x=4.378430in,y=6.746400in,left,base]{\color{textcolor}\sffamily\fontsize{5.647059}{6.776471}\selectfont 17.47(21)}%
\end{pgfscope}%
\begin{pgfscope}%
\pgfsetbuttcap%
\pgfsetroundjoin%
\definecolor{currentfill}{rgb}{0.150000,0.150000,0.150000}%
\pgfsetfillcolor{currentfill}%
\pgfsetlinewidth{1.003750pt}%
\definecolor{currentstroke}{rgb}{0.150000,0.150000,0.150000}%
\pgfsetstrokecolor{currentstroke}%
\pgfsetdash{}{0pt}%
\pgfsys@defobject{currentmarker}{\pgfqpoint{0.000000in}{-0.066667in}}{\pgfqpoint{0.000000in}{0.000000in}}{%
\pgfpathmoveto{\pgfqpoint{0.000000in}{0.000000in}}%
\pgfpathlineto{\pgfqpoint{0.000000in}{-0.066667in}}%
\pgfusepath{stroke,fill}%
}%
\begin{pgfscope}%
\pgfsys@transformshift{3.637789in}{6.271734in}%
\pgfsys@useobject{currentmarker}{}%
\end{pgfscope}%
\end{pgfscope}%
\begin{pgfscope}%
\pgfsetbuttcap%
\pgfsetroundjoin%
\definecolor{currentfill}{rgb}{0.150000,0.150000,0.150000}%
\pgfsetfillcolor{currentfill}%
\pgfsetlinewidth{1.003750pt}%
\definecolor{currentstroke}{rgb}{0.150000,0.150000,0.150000}%
\pgfsetstrokecolor{currentstroke}%
\pgfsetdash{}{0pt}%
\pgfsys@defobject{currentmarker}{\pgfqpoint{0.000000in}{-0.066667in}}{\pgfqpoint{0.000000in}{0.000000in}}{%
\pgfpathmoveto{\pgfqpoint{0.000000in}{0.000000in}}%
\pgfpathlineto{\pgfqpoint{0.000000in}{-0.066667in}}%
\pgfusepath{stroke,fill}%
}%
\begin{pgfscope}%
\pgfsys@transformshift{4.139198in}{6.271734in}%
\pgfsys@useobject{currentmarker}{}%
\end{pgfscope}%
\end{pgfscope}%
\begin{pgfscope}%
\pgfsetbuttcap%
\pgfsetroundjoin%
\definecolor{currentfill}{rgb}{0.150000,0.150000,0.150000}%
\pgfsetfillcolor{currentfill}%
\pgfsetlinewidth{1.003750pt}%
\definecolor{currentstroke}{rgb}{0.150000,0.150000,0.150000}%
\pgfsetstrokecolor{currentstroke}%
\pgfsetdash{}{0pt}%
\pgfsys@defobject{currentmarker}{\pgfqpoint{0.000000in}{-0.066667in}}{\pgfqpoint{0.000000in}{0.000000in}}{%
\pgfpathmoveto{\pgfqpoint{0.000000in}{0.000000in}}%
\pgfpathlineto{\pgfqpoint{0.000000in}{-0.066667in}}%
\pgfusepath{stroke,fill}%
}%
\begin{pgfscope}%
\pgfsys@transformshift{4.640607in}{6.271734in}%
\pgfsys@useobject{currentmarker}{}%
\end{pgfscope}%
\end{pgfscope}%
\begin{pgfscope}%
\pgfsetbuttcap%
\pgfsetroundjoin%
\definecolor{currentfill}{rgb}{0.150000,0.150000,0.150000}%
\pgfsetfillcolor{currentfill}%
\pgfsetlinewidth{0.803000pt}%
\definecolor{currentstroke}{rgb}{0.150000,0.150000,0.150000}%
\pgfsetstrokecolor{currentstroke}%
\pgfsetdash{}{0pt}%
\pgfsys@defobject{currentmarker}{\pgfqpoint{0.000000in}{-0.044444in}}{\pgfqpoint{0.000000in}{0.000000in}}{%
\pgfpathmoveto{\pgfqpoint{0.000000in}{0.000000in}}%
\pgfpathlineto{\pgfqpoint{0.000000in}{-0.044444in}}%
\pgfusepath{stroke,fill}%
}%
\begin{pgfscope}%
\pgfsys@transformshift{3.788728in}{6.271734in}%
\pgfsys@useobject{currentmarker}{}%
\end{pgfscope}%
\end{pgfscope}%
\begin{pgfscope}%
\pgfsetbuttcap%
\pgfsetroundjoin%
\definecolor{currentfill}{rgb}{0.150000,0.150000,0.150000}%
\pgfsetfillcolor{currentfill}%
\pgfsetlinewidth{0.803000pt}%
\definecolor{currentstroke}{rgb}{0.150000,0.150000,0.150000}%
\pgfsetstrokecolor{currentstroke}%
\pgfsetdash{}{0pt}%
\pgfsys@defobject{currentmarker}{\pgfqpoint{0.000000in}{-0.044444in}}{\pgfqpoint{0.000000in}{0.000000in}}{%
\pgfpathmoveto{\pgfqpoint{0.000000in}{0.000000in}}%
\pgfpathlineto{\pgfqpoint{0.000000in}{-0.044444in}}%
\pgfusepath{stroke,fill}%
}%
\begin{pgfscope}%
\pgfsys@transformshift{3.877021in}{6.271734in}%
\pgfsys@useobject{currentmarker}{}%
\end{pgfscope}%
\end{pgfscope}%
\begin{pgfscope}%
\pgfsetbuttcap%
\pgfsetroundjoin%
\definecolor{currentfill}{rgb}{0.150000,0.150000,0.150000}%
\pgfsetfillcolor{currentfill}%
\pgfsetlinewidth{0.803000pt}%
\definecolor{currentstroke}{rgb}{0.150000,0.150000,0.150000}%
\pgfsetstrokecolor{currentstroke}%
\pgfsetdash{}{0pt}%
\pgfsys@defobject{currentmarker}{\pgfqpoint{0.000000in}{-0.044444in}}{\pgfqpoint{0.000000in}{0.000000in}}{%
\pgfpathmoveto{\pgfqpoint{0.000000in}{0.000000in}}%
\pgfpathlineto{\pgfqpoint{0.000000in}{-0.044444in}}%
\pgfusepath{stroke,fill}%
}%
\begin{pgfscope}%
\pgfsys@transformshift{3.939667in}{6.271734in}%
\pgfsys@useobject{currentmarker}{}%
\end{pgfscope}%
\end{pgfscope}%
\begin{pgfscope}%
\pgfsetbuttcap%
\pgfsetroundjoin%
\definecolor{currentfill}{rgb}{0.150000,0.150000,0.150000}%
\pgfsetfillcolor{currentfill}%
\pgfsetlinewidth{0.803000pt}%
\definecolor{currentstroke}{rgb}{0.150000,0.150000,0.150000}%
\pgfsetstrokecolor{currentstroke}%
\pgfsetdash{}{0pt}%
\pgfsys@defobject{currentmarker}{\pgfqpoint{0.000000in}{-0.044444in}}{\pgfqpoint{0.000000in}{0.000000in}}{%
\pgfpathmoveto{\pgfqpoint{0.000000in}{0.000000in}}%
\pgfpathlineto{\pgfqpoint{0.000000in}{-0.044444in}}%
\pgfusepath{stroke,fill}%
}%
\begin{pgfscope}%
\pgfsys@transformshift{3.988258in}{6.271734in}%
\pgfsys@useobject{currentmarker}{}%
\end{pgfscope}%
\end{pgfscope}%
\begin{pgfscope}%
\pgfsetbuttcap%
\pgfsetroundjoin%
\definecolor{currentfill}{rgb}{0.150000,0.150000,0.150000}%
\pgfsetfillcolor{currentfill}%
\pgfsetlinewidth{0.803000pt}%
\definecolor{currentstroke}{rgb}{0.150000,0.150000,0.150000}%
\pgfsetstrokecolor{currentstroke}%
\pgfsetdash{}{0pt}%
\pgfsys@defobject{currentmarker}{\pgfqpoint{0.000000in}{-0.044444in}}{\pgfqpoint{0.000000in}{0.000000in}}{%
\pgfpathmoveto{\pgfqpoint{0.000000in}{0.000000in}}%
\pgfpathlineto{\pgfqpoint{0.000000in}{-0.044444in}}%
\pgfusepath{stroke,fill}%
}%
\begin{pgfscope}%
\pgfsys@transformshift{4.027961in}{6.271734in}%
\pgfsys@useobject{currentmarker}{}%
\end{pgfscope}%
\end{pgfscope}%
\begin{pgfscope}%
\pgfsetbuttcap%
\pgfsetroundjoin%
\definecolor{currentfill}{rgb}{0.150000,0.150000,0.150000}%
\pgfsetfillcolor{currentfill}%
\pgfsetlinewidth{0.803000pt}%
\definecolor{currentstroke}{rgb}{0.150000,0.150000,0.150000}%
\pgfsetstrokecolor{currentstroke}%
\pgfsetdash{}{0pt}%
\pgfsys@defobject{currentmarker}{\pgfqpoint{0.000000in}{-0.044444in}}{\pgfqpoint{0.000000in}{0.000000in}}{%
\pgfpathmoveto{\pgfqpoint{0.000000in}{0.000000in}}%
\pgfpathlineto{\pgfqpoint{0.000000in}{-0.044444in}}%
\pgfusepath{stroke,fill}%
}%
\begin{pgfscope}%
\pgfsys@transformshift{4.061528in}{6.271734in}%
\pgfsys@useobject{currentmarker}{}%
\end{pgfscope}%
\end{pgfscope}%
\begin{pgfscope}%
\pgfsetbuttcap%
\pgfsetroundjoin%
\definecolor{currentfill}{rgb}{0.150000,0.150000,0.150000}%
\pgfsetfillcolor{currentfill}%
\pgfsetlinewidth{0.803000pt}%
\definecolor{currentstroke}{rgb}{0.150000,0.150000,0.150000}%
\pgfsetstrokecolor{currentstroke}%
\pgfsetdash{}{0pt}%
\pgfsys@defobject{currentmarker}{\pgfqpoint{0.000000in}{-0.044444in}}{\pgfqpoint{0.000000in}{0.000000in}}{%
\pgfpathmoveto{\pgfqpoint{0.000000in}{0.000000in}}%
\pgfpathlineto{\pgfqpoint{0.000000in}{-0.044444in}}%
\pgfusepath{stroke,fill}%
}%
\begin{pgfscope}%
\pgfsys@transformshift{4.090606in}{6.271734in}%
\pgfsys@useobject{currentmarker}{}%
\end{pgfscope}%
\end{pgfscope}%
\begin{pgfscope}%
\pgfsetbuttcap%
\pgfsetroundjoin%
\definecolor{currentfill}{rgb}{0.150000,0.150000,0.150000}%
\pgfsetfillcolor{currentfill}%
\pgfsetlinewidth{0.803000pt}%
\definecolor{currentstroke}{rgb}{0.150000,0.150000,0.150000}%
\pgfsetstrokecolor{currentstroke}%
\pgfsetdash{}{0pt}%
\pgfsys@defobject{currentmarker}{\pgfqpoint{0.000000in}{-0.044444in}}{\pgfqpoint{0.000000in}{0.000000in}}{%
\pgfpathmoveto{\pgfqpoint{0.000000in}{0.000000in}}%
\pgfpathlineto{\pgfqpoint{0.000000in}{-0.044444in}}%
\pgfusepath{stroke,fill}%
}%
\begin{pgfscope}%
\pgfsys@transformshift{4.116254in}{6.271734in}%
\pgfsys@useobject{currentmarker}{}%
\end{pgfscope}%
\end{pgfscope}%
\begin{pgfscope}%
\pgfsetbuttcap%
\pgfsetroundjoin%
\definecolor{currentfill}{rgb}{0.150000,0.150000,0.150000}%
\pgfsetfillcolor{currentfill}%
\pgfsetlinewidth{0.803000pt}%
\definecolor{currentstroke}{rgb}{0.150000,0.150000,0.150000}%
\pgfsetstrokecolor{currentstroke}%
\pgfsetdash{}{0pt}%
\pgfsys@defobject{currentmarker}{\pgfqpoint{0.000000in}{-0.044444in}}{\pgfqpoint{0.000000in}{0.000000in}}{%
\pgfpathmoveto{\pgfqpoint{0.000000in}{0.000000in}}%
\pgfpathlineto{\pgfqpoint{0.000000in}{-0.044444in}}%
\pgfusepath{stroke,fill}%
}%
\begin{pgfscope}%
\pgfsys@transformshift{4.290137in}{6.271734in}%
\pgfsys@useobject{currentmarker}{}%
\end{pgfscope}%
\end{pgfscope}%
\begin{pgfscope}%
\pgfsetbuttcap%
\pgfsetroundjoin%
\definecolor{currentfill}{rgb}{0.150000,0.150000,0.150000}%
\pgfsetfillcolor{currentfill}%
\pgfsetlinewidth{0.803000pt}%
\definecolor{currentstroke}{rgb}{0.150000,0.150000,0.150000}%
\pgfsetstrokecolor{currentstroke}%
\pgfsetdash{}{0pt}%
\pgfsys@defobject{currentmarker}{\pgfqpoint{0.000000in}{-0.044444in}}{\pgfqpoint{0.000000in}{0.000000in}}{%
\pgfpathmoveto{\pgfqpoint{0.000000in}{0.000000in}}%
\pgfpathlineto{\pgfqpoint{0.000000in}{-0.044444in}}%
\pgfusepath{stroke,fill}%
}%
\begin{pgfscope}%
\pgfsys@transformshift{4.378430in}{6.271734in}%
\pgfsys@useobject{currentmarker}{}%
\end{pgfscope}%
\end{pgfscope}%
\begin{pgfscope}%
\pgfsetbuttcap%
\pgfsetroundjoin%
\definecolor{currentfill}{rgb}{0.150000,0.150000,0.150000}%
\pgfsetfillcolor{currentfill}%
\pgfsetlinewidth{0.803000pt}%
\definecolor{currentstroke}{rgb}{0.150000,0.150000,0.150000}%
\pgfsetstrokecolor{currentstroke}%
\pgfsetdash{}{0pt}%
\pgfsys@defobject{currentmarker}{\pgfqpoint{0.000000in}{-0.044444in}}{\pgfqpoint{0.000000in}{0.000000in}}{%
\pgfpathmoveto{\pgfqpoint{0.000000in}{0.000000in}}%
\pgfpathlineto{\pgfqpoint{0.000000in}{-0.044444in}}%
\pgfusepath{stroke,fill}%
}%
\begin{pgfscope}%
\pgfsys@transformshift{4.441076in}{6.271734in}%
\pgfsys@useobject{currentmarker}{}%
\end{pgfscope}%
\end{pgfscope}%
\begin{pgfscope}%
\pgfsetbuttcap%
\pgfsetroundjoin%
\definecolor{currentfill}{rgb}{0.150000,0.150000,0.150000}%
\pgfsetfillcolor{currentfill}%
\pgfsetlinewidth{0.803000pt}%
\definecolor{currentstroke}{rgb}{0.150000,0.150000,0.150000}%
\pgfsetstrokecolor{currentstroke}%
\pgfsetdash{}{0pt}%
\pgfsys@defobject{currentmarker}{\pgfqpoint{0.000000in}{-0.044444in}}{\pgfqpoint{0.000000in}{0.000000in}}{%
\pgfpathmoveto{\pgfqpoint{0.000000in}{0.000000in}}%
\pgfpathlineto{\pgfqpoint{0.000000in}{-0.044444in}}%
\pgfusepath{stroke,fill}%
}%
\begin{pgfscope}%
\pgfsys@transformshift{4.489667in}{6.271734in}%
\pgfsys@useobject{currentmarker}{}%
\end{pgfscope}%
\end{pgfscope}%
\begin{pgfscope}%
\pgfsetbuttcap%
\pgfsetroundjoin%
\definecolor{currentfill}{rgb}{0.150000,0.150000,0.150000}%
\pgfsetfillcolor{currentfill}%
\pgfsetlinewidth{0.803000pt}%
\definecolor{currentstroke}{rgb}{0.150000,0.150000,0.150000}%
\pgfsetstrokecolor{currentstroke}%
\pgfsetdash{}{0pt}%
\pgfsys@defobject{currentmarker}{\pgfqpoint{0.000000in}{-0.044444in}}{\pgfqpoint{0.000000in}{0.000000in}}{%
\pgfpathmoveto{\pgfqpoint{0.000000in}{0.000000in}}%
\pgfpathlineto{\pgfqpoint{0.000000in}{-0.044444in}}%
\pgfusepath{stroke,fill}%
}%
\begin{pgfscope}%
\pgfsys@transformshift{4.529370in}{6.271734in}%
\pgfsys@useobject{currentmarker}{}%
\end{pgfscope}%
\end{pgfscope}%
\begin{pgfscope}%
\pgfsetbuttcap%
\pgfsetroundjoin%
\definecolor{currentfill}{rgb}{0.150000,0.150000,0.150000}%
\pgfsetfillcolor{currentfill}%
\pgfsetlinewidth{0.803000pt}%
\definecolor{currentstroke}{rgb}{0.150000,0.150000,0.150000}%
\pgfsetstrokecolor{currentstroke}%
\pgfsetdash{}{0pt}%
\pgfsys@defobject{currentmarker}{\pgfqpoint{0.000000in}{-0.044444in}}{\pgfqpoint{0.000000in}{0.000000in}}{%
\pgfpathmoveto{\pgfqpoint{0.000000in}{0.000000in}}%
\pgfpathlineto{\pgfqpoint{0.000000in}{-0.044444in}}%
\pgfusepath{stroke,fill}%
}%
\begin{pgfscope}%
\pgfsys@transformshift{4.562937in}{6.271734in}%
\pgfsys@useobject{currentmarker}{}%
\end{pgfscope}%
\end{pgfscope}%
\begin{pgfscope}%
\pgfsetbuttcap%
\pgfsetroundjoin%
\definecolor{currentfill}{rgb}{0.150000,0.150000,0.150000}%
\pgfsetfillcolor{currentfill}%
\pgfsetlinewidth{0.803000pt}%
\definecolor{currentstroke}{rgb}{0.150000,0.150000,0.150000}%
\pgfsetstrokecolor{currentstroke}%
\pgfsetdash{}{0pt}%
\pgfsys@defobject{currentmarker}{\pgfqpoint{0.000000in}{-0.044444in}}{\pgfqpoint{0.000000in}{0.000000in}}{%
\pgfpathmoveto{\pgfqpoint{0.000000in}{0.000000in}}%
\pgfpathlineto{\pgfqpoint{0.000000in}{-0.044444in}}%
\pgfusepath{stroke,fill}%
}%
\begin{pgfscope}%
\pgfsys@transformshift{4.592015in}{6.271734in}%
\pgfsys@useobject{currentmarker}{}%
\end{pgfscope}%
\end{pgfscope}%
\begin{pgfscope}%
\pgfsetbuttcap%
\pgfsetroundjoin%
\definecolor{currentfill}{rgb}{0.150000,0.150000,0.150000}%
\pgfsetfillcolor{currentfill}%
\pgfsetlinewidth{0.803000pt}%
\definecolor{currentstroke}{rgb}{0.150000,0.150000,0.150000}%
\pgfsetstrokecolor{currentstroke}%
\pgfsetdash{}{0pt}%
\pgfsys@defobject{currentmarker}{\pgfqpoint{0.000000in}{-0.044444in}}{\pgfqpoint{0.000000in}{0.000000in}}{%
\pgfpathmoveto{\pgfqpoint{0.000000in}{0.000000in}}%
\pgfpathlineto{\pgfqpoint{0.000000in}{-0.044444in}}%
\pgfusepath{stroke,fill}%
}%
\begin{pgfscope}%
\pgfsys@transformshift{4.617663in}{6.271734in}%
\pgfsys@useobject{currentmarker}{}%
\end{pgfscope}%
\end{pgfscope}%
\begin{pgfscope}%
\pgfsetbuttcap%
\pgfsetroundjoin%
\definecolor{currentfill}{rgb}{0.150000,0.150000,0.150000}%
\pgfsetfillcolor{currentfill}%
\pgfsetlinewidth{0.803000pt}%
\definecolor{currentstroke}{rgb}{0.150000,0.150000,0.150000}%
\pgfsetstrokecolor{currentstroke}%
\pgfsetdash{}{0pt}%
\pgfsys@defobject{currentmarker}{\pgfqpoint{0.000000in}{-0.044444in}}{\pgfqpoint{0.000000in}{0.000000in}}{%
\pgfpathmoveto{\pgfqpoint{0.000000in}{0.000000in}}%
\pgfpathlineto{\pgfqpoint{0.000000in}{-0.044444in}}%
\pgfusepath{stroke,fill}%
}%
\begin{pgfscope}%
\pgfsys@transformshift{4.791546in}{6.271734in}%
\pgfsys@useobject{currentmarker}{}%
\end{pgfscope}%
\end{pgfscope}%
\begin{pgfscope}%
\pgfsetbuttcap%
\pgfsetroundjoin%
\definecolor{currentfill}{rgb}{0.150000,0.150000,0.150000}%
\pgfsetfillcolor{currentfill}%
\pgfsetlinewidth{1.003750pt}%
\definecolor{currentstroke}{rgb}{0.150000,0.150000,0.150000}%
\pgfsetstrokecolor{currentstroke}%
\pgfsetdash{}{0pt}%
\pgfsys@defobject{currentmarker}{\pgfqpoint{-0.066667in}{0.000000in}}{\pgfqpoint{0.000000in}{0.000000in}}{%
\pgfpathmoveto{\pgfqpoint{0.000000in}{0.000000in}}%
\pgfpathlineto{\pgfqpoint{-0.066667in}{0.000000in}}%
\pgfusepath{stroke,fill}%
}%
\begin{pgfscope}%
\pgfsys@transformshift{3.637789in}{6.271734in}%
\pgfsys@useobject{currentmarker}{}%
\end{pgfscope}%
\end{pgfscope}%
\begin{pgfscope}%
\pgfsetbuttcap%
\pgfsetroundjoin%
\definecolor{currentfill}{rgb}{0.150000,0.150000,0.150000}%
\pgfsetfillcolor{currentfill}%
\pgfsetlinewidth{1.003750pt}%
\definecolor{currentstroke}{rgb}{0.150000,0.150000,0.150000}%
\pgfsetstrokecolor{currentstroke}%
\pgfsetdash{}{0pt}%
\pgfsys@defobject{currentmarker}{\pgfqpoint{-0.066667in}{0.000000in}}{\pgfqpoint{0.000000in}{0.000000in}}{%
\pgfpathmoveto{\pgfqpoint{0.000000in}{0.000000in}}%
\pgfpathlineto{\pgfqpoint{-0.066667in}{0.000000in}}%
\pgfusepath{stroke,fill}%
}%
\begin{pgfscope}%
\pgfsys@transformshift{3.637789in}{6.595492in}%
\pgfsys@useobject{currentmarker}{}%
\end{pgfscope}%
\end{pgfscope}%
\begin{pgfscope}%
\pgfsetbuttcap%
\pgfsetroundjoin%
\definecolor{currentfill}{rgb}{0.150000,0.150000,0.150000}%
\pgfsetfillcolor{currentfill}%
\pgfsetlinewidth{1.003750pt}%
\definecolor{currentstroke}{rgb}{0.150000,0.150000,0.150000}%
\pgfsetstrokecolor{currentstroke}%
\pgfsetdash{}{0pt}%
\pgfsys@defobject{currentmarker}{\pgfqpoint{-0.066667in}{0.000000in}}{\pgfqpoint{0.000000in}{0.000000in}}{%
\pgfpathmoveto{\pgfqpoint{0.000000in}{0.000000in}}%
\pgfpathlineto{\pgfqpoint{-0.066667in}{0.000000in}}%
\pgfusepath{stroke,fill}%
}%
\begin{pgfscope}%
\pgfsys@transformshift{3.637789in}{6.879682in}%
\pgfsys@useobject{currentmarker}{}%
\end{pgfscope}%
\end{pgfscope}%
\begin{pgfscope}%
\pgfpathrectangle{\pgfqpoint{3.637789in}{6.271734in}}{\pgfqpoint{1.223103in}{0.607948in}}%
\pgfusepath{clip}%
\pgfsetroundcap%
\pgfsetroundjoin%
\pgfsetlinewidth{1.204500pt}%
\definecolor{currentstroke}{rgb}{0.000000,0.501961,0.000000}%
\pgfsetstrokecolor{currentstroke}%
\pgfsetdash{}{0pt}%
\pgfpathmoveto{\pgfqpoint{3.637789in}{6.562402in}}%
\pgfpathlineto{\pgfqpoint{3.878498in}{6.567227in}}%
\pgfpathlineto{\pgfqpoint{3.990029in}{6.571850in}}%
\pgfpathlineto{\pgfqpoint{4.063425in}{6.576298in}}%
\pgfpathlineto{\pgfqpoint{4.118221in}{6.580590in}}%
\pgfpathlineto{\pgfqpoint{4.161963in}{6.584741in}}%
\pgfpathlineto{\pgfqpoint{4.198371in}{6.588761in}}%
\pgfpathlineto{\pgfqpoint{4.229555in}{6.592654in}}%
\pgfpathlineto{\pgfqpoint{4.256828in}{6.596419in}}%
\pgfpathlineto{\pgfqpoint{4.281062in}{6.600049in}}%
\pgfpathlineto{\pgfqpoint{4.302867in}{6.603533in}}%
\pgfpathlineto{\pgfqpoint{4.322686in}{6.606854in}}%
\pgfpathlineto{\pgfqpoint{4.340851in}{6.609989in}}%
\pgfpathlineto{\pgfqpoint{4.357617in}{6.612911in}}%
\pgfpathlineto{\pgfqpoint{4.373183in}{6.615587in}}%
\pgfpathlineto{\pgfqpoint{4.387711in}{6.617978in}}%
\pgfpathlineto{\pgfqpoint{4.401329in}{6.620042in}}%
\pgfpathlineto{\pgfqpoint{4.414146in}{6.621728in}}%
\pgfpathlineto{\pgfqpoint{4.426250in}{6.622982in}}%
\pgfpathlineto{\pgfqpoint{4.437717in}{6.623746in}}%
\pgfpathlineto{\pgfqpoint{4.448610in}{6.623953in}}%
\pgfpathlineto{\pgfqpoint{4.458984in}{6.623534in}}%
\pgfpathlineto{\pgfqpoint{4.468886in}{6.622412in}}%
\pgfpathlineto{\pgfqpoint{4.478357in}{6.620507in}}%
\pgfpathlineto{\pgfqpoint{4.487434in}{6.617732in}}%
\pgfpathlineto{\pgfqpoint{4.496147in}{6.613995in}}%
\pgfpathlineto{\pgfqpoint{4.504525in}{6.609199in}}%
\pgfpathlineto{\pgfqpoint{4.512593in}{6.603241in}}%
\pgfpathlineto{\pgfqpoint{4.520372in}{6.596014in}}%
\pgfpathlineto{\pgfqpoint{4.527883in}{6.587403in}}%
\pgfpathlineto{\pgfqpoint{4.535144in}{6.577290in}}%
\pgfpathlineto{\pgfqpoint{4.542170in}{6.565551in}}%
\pgfpathlineto{\pgfqpoint{4.548977in}{6.552056in}}%
\pgfpathlineto{\pgfqpoint{4.555577in}{6.536670in}}%
\pgfpathlineto{\pgfqpoint{4.561983in}{6.519252in}}%
\pgfpathlineto{\pgfqpoint{4.568206in}{6.499657in}}%
\pgfpathlineto{\pgfqpoint{4.574256in}{6.477733in}}%
\pgfpathlineto{\pgfqpoint{4.580143in}{6.453324in}}%
\pgfpathlineto{\pgfqpoint{4.585874in}{6.426267in}}%
\pgfpathlineto{\pgfqpoint{4.591459in}{6.396394in}}%
\pgfpathlineto{\pgfqpoint{4.596904in}{6.363533in}}%
\pgfpathlineto{\pgfqpoint{4.602216in}{6.327505in}}%
\pgfpathlineto{\pgfqpoint{4.607401in}{6.288126in}}%
\pgfpathlineto{\pgfqpoint{4.609729in}{6.268401in}}%
\pgfusepath{stroke}%
\end{pgfscope}%
\begin{pgfscope}%
\pgfsetrectcap%
\pgfsetmiterjoin%
\pgfsetlinewidth{1.003750pt}%
\definecolor{currentstroke}{rgb}{0.150000,0.150000,0.150000}%
\pgfsetstrokecolor{currentstroke}%
\pgfsetdash{}{0pt}%
\pgfpathmoveto{\pgfqpoint{3.637789in}{6.271734in}}%
\pgfpathlineto{\pgfqpoint{3.637789in}{6.879682in}}%
\pgfusepath{stroke}%
\end{pgfscope}%
\begin{pgfscope}%
\pgfsetrectcap%
\pgfsetmiterjoin%
\pgfsetlinewidth{1.003750pt}%
\definecolor{currentstroke}{rgb}{0.150000,0.150000,0.150000}%
\pgfsetstrokecolor{currentstroke}%
\pgfsetdash{}{0pt}%
\pgfpathmoveto{\pgfqpoint{3.637789in}{6.271734in}}%
\pgfpathlineto{\pgfqpoint{4.860892in}{6.271734in}}%
\pgfusepath{stroke}%
\end{pgfscope}%
\begin{pgfscope}%
\pgfpathrectangle{\pgfqpoint{3.637789in}{6.271734in}}{\pgfqpoint{1.223103in}{0.607948in}}%
\pgfusepath{clip}%
\pgfsetbuttcap%
\pgfsetroundjoin%
\definecolor{currentfill}{rgb}{0.000000,0.000000,0.000000}%
\pgfsetfillcolor{currentfill}%
\pgfsetlinewidth{1.003750pt}%
\definecolor{currentstroke}{rgb}{0.000000,0.000000,0.000000}%
\pgfsetstrokecolor{currentstroke}%
\pgfsetdash{}{0pt}%
\pgfsys@defobject{currentmarker}{\pgfqpoint{-0.013889in}{-0.013889in}}{\pgfqpoint{0.013889in}{0.013889in}}{%
\pgfpathmoveto{\pgfqpoint{0.000000in}{-0.013889in}}%
\pgfpathcurveto{\pgfqpoint{0.003683in}{-0.013889in}}{\pgfqpoint{0.007216in}{-0.012425in}}{\pgfqpoint{0.009821in}{-0.009821in}}%
\pgfpathcurveto{\pgfqpoint{0.012425in}{-0.007216in}}{\pgfqpoint{0.013889in}{-0.003683in}}{\pgfqpoint{0.013889in}{0.000000in}}%
\pgfpathcurveto{\pgfqpoint{0.013889in}{0.003683in}}{\pgfqpoint{0.012425in}{0.007216in}}{\pgfqpoint{0.009821in}{0.009821in}}%
\pgfpathcurveto{\pgfqpoint{0.007216in}{0.012425in}}{\pgfqpoint{0.003683in}{0.013889in}}{\pgfqpoint{0.000000in}{0.013889in}}%
\pgfpathcurveto{\pgfqpoint{-0.003683in}{0.013889in}}{\pgfqpoint{-0.007216in}{0.012425in}}{\pgfqpoint{-0.009821in}{0.009821in}}%
\pgfpathcurveto{\pgfqpoint{-0.012425in}{0.007216in}}{\pgfqpoint{-0.013889in}{0.003683in}}{\pgfqpoint{-0.013889in}{0.000000in}}%
\pgfpathcurveto{\pgfqpoint{-0.013889in}{-0.003683in}}{\pgfqpoint{-0.012425in}{-0.007216in}}{\pgfqpoint{-0.009821in}{-0.009821in}}%
\pgfpathcurveto{\pgfqpoint{-0.007216in}{-0.012425in}}{\pgfqpoint{-0.003683in}{-0.013889in}}{\pgfqpoint{0.000000in}{-0.013889in}}%
\pgfpathclose%
\pgfusepath{stroke,fill}%
}%
\begin{pgfscope}%
\pgfsys@transformshift{4.640607in}{5.895759in}%
\pgfsys@useobject{currentmarker}{}%
\end{pgfscope}%
\begin{pgfscope}%
\pgfsys@transformshift{4.552313in}{6.560647in}%
\pgfsys@useobject{currentmarker}{}%
\end{pgfscope}%
\begin{pgfscope}%
\pgfsys@transformshift{4.290137in}{6.605446in}%
\pgfsys@useobject{currentmarker}{}%
\end{pgfscope}%
\begin{pgfscope}%
\pgfsys@transformshift{4.294536in}{6.605646in}%
\pgfsys@useobject{currentmarker}{}%
\end{pgfscope}%
\begin{pgfscope}%
\pgfsys@transformshift{4.299026in}{6.605854in}%
\pgfsys@useobject{currentmarker}{}%
\end{pgfscope}%
\begin{pgfscope}%
\pgfsys@transformshift{4.303611in}{6.606070in}%
\pgfsys@useobject{currentmarker}{}%
\end{pgfscope}%
\begin{pgfscope}%
\pgfsys@transformshift{4.308294in}{6.606295in}%
\pgfsys@useobject{currentmarker}{}%
\end{pgfscope}%
\begin{pgfscope}%
\pgfsys@transformshift{4.313080in}{6.606530in}%
\pgfsys@useobject{currentmarker}{}%
\end{pgfscope}%
\begin{pgfscope}%
\pgfsys@transformshift{4.317974in}{6.606774in}%
\pgfsys@useobject{currentmarker}{}%
\end{pgfscope}%
\begin{pgfscope}%
\pgfsys@transformshift{4.322980in}{6.607029in}%
\pgfsys@useobject{currentmarker}{}%
\end{pgfscope}%
\begin{pgfscope}%
\pgfsys@transformshift{4.328104in}{6.607295in}%
\pgfsys@useobject{currentmarker}{}%
\end{pgfscope}%
\begin{pgfscope}%
\pgfsys@transformshift{4.333351in}{6.607572in}%
\pgfsys@useobject{currentmarker}{}%
\end{pgfscope}%
\begin{pgfscope}%
\pgfsys@transformshift{4.338728in}{6.607862in}%
\pgfsys@useobject{currentmarker}{}%
\end{pgfscope}%
\begin{pgfscope}%
\pgfsys@transformshift{4.367806in}{6.609515in}%
\pgfsys@useobject{currentmarker}{}%
\end{pgfscope}%
\begin{pgfscope}%
\pgfsys@transformshift{4.489667in}{6.610977in}%
\pgfsys@useobject{currentmarker}{}%
\end{pgfscope}%
\begin{pgfscope}%
\pgfsys@transformshift{4.401374in}{6.611512in}%
\pgfsys@useobject{currentmarker}{}%
\end{pgfscope}%
\begin{pgfscope}%
\pgfsys@transformshift{4.441076in}{6.613409in}%
\pgfsys@useobject{currentmarker}{}%
\end{pgfscope}%
\end{pgfscope}%
\begin{pgfscope}%
\definecolor{textcolor}{rgb}{0.150000,0.150000,0.150000}%
\pgfsetstrokecolor{textcolor}%
\pgfsetfillcolor{textcolor}%
\pgftext[x=4.249340in,y=6.963016in,,base]{\color{textcolor}\sffamily\fontsize{5.647059}{6.776471}\selectfont \(\displaystyle n_{\mathrm{step}} = 4\)}%
\end{pgfscope}%
\begin{pgfscope}%
\pgfsetbuttcap%
\pgfsetmiterjoin%
\definecolor{currentfill}{rgb}{1.000000,1.000000,1.000000}%
\pgfsetfillcolor{currentfill}%
\pgfsetlinewidth{0.000000pt}%
\definecolor{currentstroke}{rgb}{0.000000,0.000000,0.000000}%
\pgfsetstrokecolor{currentstroke}%
\pgfsetstrokeopacity{0.000000}%
\pgfsetdash{}{0pt}%
\pgfpathmoveto{\pgfqpoint{5.105513in}{6.271734in}}%
\pgfpathlineto{\pgfqpoint{6.328616in}{6.271734in}}%
\pgfpathlineto{\pgfqpoint{6.328616in}{6.879682in}}%
\pgfpathlineto{\pgfqpoint{5.105513in}{6.879682in}}%
\pgfpathclose%
\pgfusepath{fill}%
\end{pgfscope}%
\begin{pgfscope}%
\pgfpathrectangle{\pgfqpoint{5.105513in}{6.271734in}}{\pgfqpoint{1.223103in}{0.607948in}}%
\pgfusepath{clip}%
\pgfsetbuttcap%
\pgfsetmiterjoin%
\definecolor{currentfill}{rgb}{0.000000,0.000000,1.000000}%
\pgfsetfillcolor{currentfill}%
\pgfsetfillopacity{0.100000}%
\pgfsetlinewidth{0.803000pt}%
\definecolor{currentstroke}{rgb}{0.000000,0.000000,1.000000}%
\pgfsetstrokecolor{currentstroke}%
\pgfsetstrokeopacity{0.100000}%
\pgfsetdash{}{0pt}%
\pgfpathmoveto{\pgfqpoint{5.105513in}{6.593655in}}%
\pgfpathlineto{\pgfqpoint{5.105513in}{6.597568in}}%
\pgfpathlineto{\pgfqpoint{6.328616in}{6.597568in}}%
\pgfpathlineto{\pgfqpoint{6.328616in}{6.593655in}}%
\pgfpathclose%
\pgfusepath{stroke,fill}%
\end{pgfscope}%
\begin{pgfscope}%
\pgfpathrectangle{\pgfqpoint{5.105513in}{6.271734in}}{\pgfqpoint{1.223103in}{0.607948in}}%
\pgfusepath{clip}%
\pgfsetbuttcap%
\pgfsetroundjoin%
\definecolor{currentfill}{rgb}{0.000000,0.501961,0.000000}%
\pgfsetfillcolor{currentfill}%
\pgfsetfillopacity{0.500000}%
\pgfsetlinewidth{0.803000pt}%
\definecolor{currentstroke}{rgb}{0.000000,0.501961,0.000000}%
\pgfsetstrokecolor{currentstroke}%
\pgfsetstrokeopacity{0.500000}%
\pgfsetdash{}{0pt}%
\pgfpathmoveto{\pgfqpoint{5.105513in}{6.598009in}}%
\pgfpathlineto{\pgfqpoint{5.105513in}{6.594459in}}%
\pgfpathlineto{\pgfqpoint{5.346222in}{6.596030in}}%
\pgfpathlineto{\pgfqpoint{5.457753in}{6.597535in}}%
\pgfpathlineto{\pgfqpoint{5.531149in}{6.598988in}}%
\pgfpathlineto{\pgfqpoint{5.585945in}{6.600396in}}%
\pgfpathlineto{\pgfqpoint{5.629687in}{6.601768in}}%
\pgfpathlineto{\pgfqpoint{5.666095in}{6.603111in}}%
\pgfpathlineto{\pgfqpoint{5.697279in}{6.604432in}}%
\pgfpathlineto{\pgfqpoint{5.724552in}{6.605736in}}%
\pgfpathlineto{\pgfqpoint{5.748786in}{6.607026in}}%
\pgfpathlineto{\pgfqpoint{5.770591in}{6.608308in}}%
\pgfpathlineto{\pgfqpoint{5.790410in}{6.609568in}}%
\pgfpathlineto{\pgfqpoint{5.808575in}{6.610744in}}%
\pgfpathlineto{\pgfqpoint{5.825341in}{6.611933in}}%
\pgfpathlineto{\pgfqpoint{5.840907in}{6.613140in}}%
\pgfpathlineto{\pgfqpoint{5.855435in}{6.614364in}}%
\pgfpathlineto{\pgfqpoint{5.869053in}{6.615605in}}%
\pgfpathlineto{\pgfqpoint{5.881870in}{6.616863in}}%
\pgfpathlineto{\pgfqpoint{5.893974in}{6.618139in}}%
\pgfpathlineto{\pgfqpoint{5.905441in}{6.619431in}}%
\pgfpathlineto{\pgfqpoint{5.916334in}{6.620741in}}%
\pgfpathlineto{\pgfqpoint{5.926708in}{6.622067in}}%
\pgfpathlineto{\pgfqpoint{5.936610in}{6.623410in}}%
\pgfpathlineto{\pgfqpoint{5.946082in}{6.624769in}}%
\pgfpathlineto{\pgfqpoint{5.955158in}{6.626141in}}%
\pgfpathlineto{\pgfqpoint{5.963871in}{6.627522in}}%
\pgfpathlineto{\pgfqpoint{5.972249in}{6.628903in}}%
\pgfpathlineto{\pgfqpoint{5.980317in}{6.630278in}}%
\pgfpathlineto{\pgfqpoint{5.988096in}{6.631653in}}%
\pgfpathlineto{\pgfqpoint{5.995607in}{6.633035in}}%
\pgfpathlineto{\pgfqpoint{6.002868in}{6.634429in}}%
\pgfpathlineto{\pgfqpoint{6.009894in}{6.635837in}}%
\pgfpathlineto{\pgfqpoint{6.016701in}{6.637263in}}%
\pgfpathlineto{\pgfqpoint{6.023301in}{6.638708in}}%
\pgfpathlineto{\pgfqpoint{6.029707in}{6.640176in}}%
\pgfpathlineto{\pgfqpoint{6.035930in}{6.641671in}}%
\pgfpathlineto{\pgfqpoint{6.041980in}{6.643195in}}%
\pgfpathlineto{\pgfqpoint{6.047867in}{6.644755in}}%
\pgfpathlineto{\pgfqpoint{6.053598in}{6.646354in}}%
\pgfpathlineto{\pgfqpoint{6.059183in}{6.648000in}}%
\pgfpathlineto{\pgfqpoint{6.064628in}{6.649699in}}%
\pgfpathlineto{\pgfqpoint{6.069940in}{6.651459in}}%
\pgfpathlineto{\pgfqpoint{6.075126in}{6.653289in}}%
\pgfpathlineto{\pgfqpoint{6.080191in}{6.655198in}}%
\pgfpathlineto{\pgfqpoint{6.085140in}{6.657197in}}%
\pgfpathlineto{\pgfqpoint{6.089980in}{6.659297in}}%
\pgfpathlineto{\pgfqpoint{6.094715in}{6.661512in}}%
\pgfpathlineto{\pgfqpoint{6.099349in}{6.663855in}}%
\pgfpathlineto{\pgfqpoint{6.103886in}{6.666334in}}%
\pgfpathlineto{\pgfqpoint{6.108331in}{6.668765in}}%
\pgfpathlineto{\pgfqpoint{6.108331in}{6.669033in}}%
\pgfpathlineto{\pgfqpoint{6.108331in}{6.669033in}}%
\pgfpathlineto{\pgfqpoint{6.103886in}{6.666742in}}%
\pgfpathlineto{\pgfqpoint{6.099349in}{6.664692in}}%
\pgfpathlineto{\pgfqpoint{6.094715in}{6.662680in}}%
\pgfpathlineto{\pgfqpoint{6.089980in}{6.660703in}}%
\pgfpathlineto{\pgfqpoint{6.085140in}{6.658759in}}%
\pgfpathlineto{\pgfqpoint{6.080191in}{6.656849in}}%
\pgfpathlineto{\pgfqpoint{6.075126in}{6.654972in}}%
\pgfpathlineto{\pgfqpoint{6.069940in}{6.653129in}}%
\pgfpathlineto{\pgfqpoint{6.064628in}{6.651318in}}%
\pgfpathlineto{\pgfqpoint{6.059183in}{6.649541in}}%
\pgfpathlineto{\pgfqpoint{6.053598in}{6.647796in}}%
\pgfpathlineto{\pgfqpoint{6.047867in}{6.646082in}}%
\pgfpathlineto{\pgfqpoint{6.041980in}{6.644398in}}%
\pgfpathlineto{\pgfqpoint{6.035930in}{6.642744in}}%
\pgfpathlineto{\pgfqpoint{6.029707in}{6.641119in}}%
\pgfpathlineto{\pgfqpoint{6.023301in}{6.639521in}}%
\pgfpathlineto{\pgfqpoint{6.016701in}{6.637950in}}%
\pgfpathlineto{\pgfqpoint{6.009894in}{6.636405in}}%
\pgfpathlineto{\pgfqpoint{6.002868in}{6.634884in}}%
\pgfpathlineto{\pgfqpoint{5.995607in}{6.633387in}}%
\pgfpathlineto{\pgfqpoint{5.988096in}{6.631915in}}%
\pgfpathlineto{\pgfqpoint{5.980317in}{6.630470in}}%
\pgfpathlineto{\pgfqpoint{5.972249in}{6.629056in}}%
\pgfpathlineto{\pgfqpoint{5.963871in}{6.627678in}}%
\pgfpathlineto{\pgfqpoint{5.955158in}{6.626330in}}%
\pgfpathlineto{\pgfqpoint{5.946082in}{6.624999in}}%
\pgfpathlineto{\pgfqpoint{5.936610in}{6.623682in}}%
\pgfpathlineto{\pgfqpoint{5.926708in}{6.622374in}}%
\pgfpathlineto{\pgfqpoint{5.916334in}{6.621075in}}%
\pgfpathlineto{\pgfqpoint{5.905441in}{6.619782in}}%
\pgfpathlineto{\pgfqpoint{5.893974in}{6.618496in}}%
\pgfpathlineto{\pgfqpoint{5.881870in}{6.617215in}}%
\pgfpathlineto{\pgfqpoint{5.869053in}{6.615939in}}%
\pgfpathlineto{\pgfqpoint{5.855435in}{6.614665in}}%
\pgfpathlineto{\pgfqpoint{5.840907in}{6.613395in}}%
\pgfpathlineto{\pgfqpoint{5.825341in}{6.612125in}}%
\pgfpathlineto{\pgfqpoint{5.808575in}{6.610856in}}%
\pgfpathlineto{\pgfqpoint{5.790410in}{6.609588in}}%
\pgfpathlineto{\pgfqpoint{5.770591in}{6.608418in}}%
\pgfpathlineto{\pgfqpoint{5.748786in}{6.607280in}}%
\pgfpathlineto{\pgfqpoint{5.724552in}{6.606160in}}%
\pgfpathlineto{\pgfqpoint{5.697279in}{6.605059in}}%
\pgfpathlineto{\pgfqpoint{5.666095in}{6.603978in}}%
\pgfpathlineto{\pgfqpoint{5.629687in}{6.602917in}}%
\pgfpathlineto{\pgfqpoint{5.585945in}{6.601880in}}%
\pgfpathlineto{\pgfqpoint{5.531149in}{6.600867in}}%
\pgfpathlineto{\pgfqpoint{5.457753in}{6.599882in}}%
\pgfpathlineto{\pgfqpoint{5.346222in}{6.598928in}}%
\pgfpathlineto{\pgfqpoint{5.105513in}{6.598009in}}%
\pgfpathclose%
\pgfusepath{stroke,fill}%
\end{pgfscope}%
\begin{pgfscope}%
\pgfpathrectangle{\pgfqpoint{5.105513in}{6.271734in}}{\pgfqpoint{1.223103in}{0.607948in}}%
\pgfusepath{clip}%
\pgfsetroundcap%
\pgfsetroundjoin%
\pgfsetlinewidth{0.501875pt}%
\definecolor{currentstroke}{rgb}{0.000000,0.000000,1.000000}%
\pgfsetstrokecolor{currentstroke}%
\pgfsetstrokeopacity{0.800000}%
\pgfsetdash{}{0pt}%
\pgfpathmoveto{\pgfqpoint{5.105513in}{6.595612in}}%
\pgfpathlineto{\pgfqpoint{6.328616in}{6.595612in}}%
\pgfusepath{stroke}%
\end{pgfscope}%
\begin{pgfscope}%
\pgfpathrectangle{\pgfqpoint{5.105513in}{6.271734in}}{\pgfqpoint{1.223103in}{0.607948in}}%
\pgfusepath{clip}%
\pgfsetbuttcap%
\pgfsetroundjoin%
\pgfsetlinewidth{1.003750pt}%
\definecolor{currentstroke}{rgb}{0.000000,0.000000,0.000000}%
\pgfsetstrokecolor{currentstroke}%
\pgfsetdash{{3.700000pt}{1.600000pt}}{0.000000pt}%
\pgfpathmoveto{\pgfqpoint{5.105513in}{6.595492in}}%
\pgfpathlineto{\pgfqpoint{6.328616in}{6.595492in}}%
\pgfusepath{stroke}%
\end{pgfscope}%
\begin{pgfscope}%
\pgfsetroundcap%
\pgfsetroundjoin%
\pgfsetlinewidth{0.501875pt}%
\definecolor{currentstroke}{rgb}{0.000000,0.000000,1.000000}%
\pgfsetstrokecolor{currentstroke}%
\pgfsetstrokeopacity{0.800000}%
\pgfsetdash{}{0pt}%
\pgfpathmoveto{\pgfqpoint{5.932592in}{6.713715in}}%
\pgfpathquadraticcurveto{\pgfqpoint{5.856728in}{6.662438in}}{\pgfqpoint{5.780865in}{6.611160in}}%
\pgfusepath{stroke}%
\end{pgfscope}%
\begin{pgfscope}%
\pgfsetfillopacity{0.800000}%
\pgfsetstrokeopacity{0.800000}%
\definecolor{textcolor}{rgb}{0.000000,0.000000,1.000000}%
\pgfsetstrokecolor{textcolor}%
\pgfsetfillcolor{textcolor}%
\pgftext[x=5.846155in,y=6.777996in,left,base]{\color{textcolor}\sffamily\fontsize{5.647059}{6.776471}\selectfont 17.5327(32)}%
\end{pgfscope}%
\begin{pgfscope}%
\pgfsetbuttcap%
\pgfsetroundjoin%
\definecolor{currentfill}{rgb}{0.150000,0.150000,0.150000}%
\pgfsetfillcolor{currentfill}%
\pgfsetlinewidth{1.003750pt}%
\definecolor{currentstroke}{rgb}{0.150000,0.150000,0.150000}%
\pgfsetstrokecolor{currentstroke}%
\pgfsetdash{}{0pt}%
\pgfsys@defobject{currentmarker}{\pgfqpoint{0.000000in}{-0.066667in}}{\pgfqpoint{0.000000in}{0.000000in}}{%
\pgfpathmoveto{\pgfqpoint{0.000000in}{0.000000in}}%
\pgfpathlineto{\pgfqpoint{0.000000in}{-0.066667in}}%
\pgfusepath{stroke,fill}%
}%
\begin{pgfscope}%
\pgfsys@transformshift{5.105513in}{6.271734in}%
\pgfsys@useobject{currentmarker}{}%
\end{pgfscope}%
\end{pgfscope}%
\begin{pgfscope}%
\pgfsetbuttcap%
\pgfsetroundjoin%
\definecolor{currentfill}{rgb}{0.150000,0.150000,0.150000}%
\pgfsetfillcolor{currentfill}%
\pgfsetlinewidth{1.003750pt}%
\definecolor{currentstroke}{rgb}{0.150000,0.150000,0.150000}%
\pgfsetstrokecolor{currentstroke}%
\pgfsetdash{}{0pt}%
\pgfsys@defobject{currentmarker}{\pgfqpoint{0.000000in}{-0.066667in}}{\pgfqpoint{0.000000in}{0.000000in}}{%
\pgfpathmoveto{\pgfqpoint{0.000000in}{0.000000in}}%
\pgfpathlineto{\pgfqpoint{0.000000in}{-0.066667in}}%
\pgfusepath{stroke,fill}%
}%
\begin{pgfscope}%
\pgfsys@transformshift{5.606922in}{6.271734in}%
\pgfsys@useobject{currentmarker}{}%
\end{pgfscope}%
\end{pgfscope}%
\begin{pgfscope}%
\pgfsetbuttcap%
\pgfsetroundjoin%
\definecolor{currentfill}{rgb}{0.150000,0.150000,0.150000}%
\pgfsetfillcolor{currentfill}%
\pgfsetlinewidth{1.003750pt}%
\definecolor{currentstroke}{rgb}{0.150000,0.150000,0.150000}%
\pgfsetstrokecolor{currentstroke}%
\pgfsetdash{}{0pt}%
\pgfsys@defobject{currentmarker}{\pgfqpoint{0.000000in}{-0.066667in}}{\pgfqpoint{0.000000in}{0.000000in}}{%
\pgfpathmoveto{\pgfqpoint{0.000000in}{0.000000in}}%
\pgfpathlineto{\pgfqpoint{0.000000in}{-0.066667in}}%
\pgfusepath{stroke,fill}%
}%
\begin{pgfscope}%
\pgfsys@transformshift{6.108331in}{6.271734in}%
\pgfsys@useobject{currentmarker}{}%
\end{pgfscope}%
\end{pgfscope}%
\begin{pgfscope}%
\pgfsetbuttcap%
\pgfsetroundjoin%
\definecolor{currentfill}{rgb}{0.150000,0.150000,0.150000}%
\pgfsetfillcolor{currentfill}%
\pgfsetlinewidth{0.803000pt}%
\definecolor{currentstroke}{rgb}{0.150000,0.150000,0.150000}%
\pgfsetstrokecolor{currentstroke}%
\pgfsetdash{}{0pt}%
\pgfsys@defobject{currentmarker}{\pgfqpoint{0.000000in}{-0.044444in}}{\pgfqpoint{0.000000in}{0.000000in}}{%
\pgfpathmoveto{\pgfqpoint{0.000000in}{0.000000in}}%
\pgfpathlineto{\pgfqpoint{0.000000in}{-0.044444in}}%
\pgfusepath{stroke,fill}%
}%
\begin{pgfscope}%
\pgfsys@transformshift{5.256452in}{6.271734in}%
\pgfsys@useobject{currentmarker}{}%
\end{pgfscope}%
\end{pgfscope}%
\begin{pgfscope}%
\pgfsetbuttcap%
\pgfsetroundjoin%
\definecolor{currentfill}{rgb}{0.150000,0.150000,0.150000}%
\pgfsetfillcolor{currentfill}%
\pgfsetlinewidth{0.803000pt}%
\definecolor{currentstroke}{rgb}{0.150000,0.150000,0.150000}%
\pgfsetstrokecolor{currentstroke}%
\pgfsetdash{}{0pt}%
\pgfsys@defobject{currentmarker}{\pgfqpoint{0.000000in}{-0.044444in}}{\pgfqpoint{0.000000in}{0.000000in}}{%
\pgfpathmoveto{\pgfqpoint{0.000000in}{0.000000in}}%
\pgfpathlineto{\pgfqpoint{0.000000in}{-0.044444in}}%
\pgfusepath{stroke,fill}%
}%
\begin{pgfscope}%
\pgfsys@transformshift{5.344746in}{6.271734in}%
\pgfsys@useobject{currentmarker}{}%
\end{pgfscope}%
\end{pgfscope}%
\begin{pgfscope}%
\pgfsetbuttcap%
\pgfsetroundjoin%
\definecolor{currentfill}{rgb}{0.150000,0.150000,0.150000}%
\pgfsetfillcolor{currentfill}%
\pgfsetlinewidth{0.803000pt}%
\definecolor{currentstroke}{rgb}{0.150000,0.150000,0.150000}%
\pgfsetstrokecolor{currentstroke}%
\pgfsetdash{}{0pt}%
\pgfsys@defobject{currentmarker}{\pgfqpoint{0.000000in}{-0.044444in}}{\pgfqpoint{0.000000in}{0.000000in}}{%
\pgfpathmoveto{\pgfqpoint{0.000000in}{0.000000in}}%
\pgfpathlineto{\pgfqpoint{0.000000in}{-0.044444in}}%
\pgfusepath{stroke,fill}%
}%
\begin{pgfscope}%
\pgfsys@transformshift{5.407391in}{6.271734in}%
\pgfsys@useobject{currentmarker}{}%
\end{pgfscope}%
\end{pgfscope}%
\begin{pgfscope}%
\pgfsetbuttcap%
\pgfsetroundjoin%
\definecolor{currentfill}{rgb}{0.150000,0.150000,0.150000}%
\pgfsetfillcolor{currentfill}%
\pgfsetlinewidth{0.803000pt}%
\definecolor{currentstroke}{rgb}{0.150000,0.150000,0.150000}%
\pgfsetstrokecolor{currentstroke}%
\pgfsetdash{}{0pt}%
\pgfsys@defobject{currentmarker}{\pgfqpoint{0.000000in}{-0.044444in}}{\pgfqpoint{0.000000in}{0.000000in}}{%
\pgfpathmoveto{\pgfqpoint{0.000000in}{0.000000in}}%
\pgfpathlineto{\pgfqpoint{0.000000in}{-0.044444in}}%
\pgfusepath{stroke,fill}%
}%
\begin{pgfscope}%
\pgfsys@transformshift{5.455982in}{6.271734in}%
\pgfsys@useobject{currentmarker}{}%
\end{pgfscope}%
\end{pgfscope}%
\begin{pgfscope}%
\pgfsetbuttcap%
\pgfsetroundjoin%
\definecolor{currentfill}{rgb}{0.150000,0.150000,0.150000}%
\pgfsetfillcolor{currentfill}%
\pgfsetlinewidth{0.803000pt}%
\definecolor{currentstroke}{rgb}{0.150000,0.150000,0.150000}%
\pgfsetstrokecolor{currentstroke}%
\pgfsetdash{}{0pt}%
\pgfsys@defobject{currentmarker}{\pgfqpoint{0.000000in}{-0.044444in}}{\pgfqpoint{0.000000in}{0.000000in}}{%
\pgfpathmoveto{\pgfqpoint{0.000000in}{0.000000in}}%
\pgfpathlineto{\pgfqpoint{0.000000in}{-0.044444in}}%
\pgfusepath{stroke,fill}%
}%
\begin{pgfscope}%
\pgfsys@transformshift{5.495685in}{6.271734in}%
\pgfsys@useobject{currentmarker}{}%
\end{pgfscope}%
\end{pgfscope}%
\begin{pgfscope}%
\pgfsetbuttcap%
\pgfsetroundjoin%
\definecolor{currentfill}{rgb}{0.150000,0.150000,0.150000}%
\pgfsetfillcolor{currentfill}%
\pgfsetlinewidth{0.803000pt}%
\definecolor{currentstroke}{rgb}{0.150000,0.150000,0.150000}%
\pgfsetstrokecolor{currentstroke}%
\pgfsetdash{}{0pt}%
\pgfsys@defobject{currentmarker}{\pgfqpoint{0.000000in}{-0.044444in}}{\pgfqpoint{0.000000in}{0.000000in}}{%
\pgfpathmoveto{\pgfqpoint{0.000000in}{0.000000in}}%
\pgfpathlineto{\pgfqpoint{0.000000in}{-0.044444in}}%
\pgfusepath{stroke,fill}%
}%
\begin{pgfscope}%
\pgfsys@transformshift{5.529252in}{6.271734in}%
\pgfsys@useobject{currentmarker}{}%
\end{pgfscope}%
\end{pgfscope}%
\begin{pgfscope}%
\pgfsetbuttcap%
\pgfsetroundjoin%
\definecolor{currentfill}{rgb}{0.150000,0.150000,0.150000}%
\pgfsetfillcolor{currentfill}%
\pgfsetlinewidth{0.803000pt}%
\definecolor{currentstroke}{rgb}{0.150000,0.150000,0.150000}%
\pgfsetstrokecolor{currentstroke}%
\pgfsetdash{}{0pt}%
\pgfsys@defobject{currentmarker}{\pgfqpoint{0.000000in}{-0.044444in}}{\pgfqpoint{0.000000in}{0.000000in}}{%
\pgfpathmoveto{\pgfqpoint{0.000000in}{0.000000in}}%
\pgfpathlineto{\pgfqpoint{0.000000in}{-0.044444in}}%
\pgfusepath{stroke,fill}%
}%
\begin{pgfscope}%
\pgfsys@transformshift{5.558330in}{6.271734in}%
\pgfsys@useobject{currentmarker}{}%
\end{pgfscope}%
\end{pgfscope}%
\begin{pgfscope}%
\pgfsetbuttcap%
\pgfsetroundjoin%
\definecolor{currentfill}{rgb}{0.150000,0.150000,0.150000}%
\pgfsetfillcolor{currentfill}%
\pgfsetlinewidth{0.803000pt}%
\definecolor{currentstroke}{rgb}{0.150000,0.150000,0.150000}%
\pgfsetstrokecolor{currentstroke}%
\pgfsetdash{}{0pt}%
\pgfsys@defobject{currentmarker}{\pgfqpoint{0.000000in}{-0.044444in}}{\pgfqpoint{0.000000in}{0.000000in}}{%
\pgfpathmoveto{\pgfqpoint{0.000000in}{0.000000in}}%
\pgfpathlineto{\pgfqpoint{0.000000in}{-0.044444in}}%
\pgfusepath{stroke,fill}%
}%
\begin{pgfscope}%
\pgfsys@transformshift{5.583978in}{6.271734in}%
\pgfsys@useobject{currentmarker}{}%
\end{pgfscope}%
\end{pgfscope}%
\begin{pgfscope}%
\pgfsetbuttcap%
\pgfsetroundjoin%
\definecolor{currentfill}{rgb}{0.150000,0.150000,0.150000}%
\pgfsetfillcolor{currentfill}%
\pgfsetlinewidth{0.803000pt}%
\definecolor{currentstroke}{rgb}{0.150000,0.150000,0.150000}%
\pgfsetstrokecolor{currentstroke}%
\pgfsetdash{}{0pt}%
\pgfsys@defobject{currentmarker}{\pgfqpoint{0.000000in}{-0.044444in}}{\pgfqpoint{0.000000in}{0.000000in}}{%
\pgfpathmoveto{\pgfqpoint{0.000000in}{0.000000in}}%
\pgfpathlineto{\pgfqpoint{0.000000in}{-0.044444in}}%
\pgfusepath{stroke,fill}%
}%
\begin{pgfscope}%
\pgfsys@transformshift{5.757861in}{6.271734in}%
\pgfsys@useobject{currentmarker}{}%
\end{pgfscope}%
\end{pgfscope}%
\begin{pgfscope}%
\pgfsetbuttcap%
\pgfsetroundjoin%
\definecolor{currentfill}{rgb}{0.150000,0.150000,0.150000}%
\pgfsetfillcolor{currentfill}%
\pgfsetlinewidth{0.803000pt}%
\definecolor{currentstroke}{rgb}{0.150000,0.150000,0.150000}%
\pgfsetstrokecolor{currentstroke}%
\pgfsetdash{}{0pt}%
\pgfsys@defobject{currentmarker}{\pgfqpoint{0.000000in}{-0.044444in}}{\pgfqpoint{0.000000in}{0.000000in}}{%
\pgfpathmoveto{\pgfqpoint{0.000000in}{0.000000in}}%
\pgfpathlineto{\pgfqpoint{0.000000in}{-0.044444in}}%
\pgfusepath{stroke,fill}%
}%
\begin{pgfscope}%
\pgfsys@transformshift{5.846155in}{6.271734in}%
\pgfsys@useobject{currentmarker}{}%
\end{pgfscope}%
\end{pgfscope}%
\begin{pgfscope}%
\pgfsetbuttcap%
\pgfsetroundjoin%
\definecolor{currentfill}{rgb}{0.150000,0.150000,0.150000}%
\pgfsetfillcolor{currentfill}%
\pgfsetlinewidth{0.803000pt}%
\definecolor{currentstroke}{rgb}{0.150000,0.150000,0.150000}%
\pgfsetstrokecolor{currentstroke}%
\pgfsetdash{}{0pt}%
\pgfsys@defobject{currentmarker}{\pgfqpoint{0.000000in}{-0.044444in}}{\pgfqpoint{0.000000in}{0.000000in}}{%
\pgfpathmoveto{\pgfqpoint{0.000000in}{0.000000in}}%
\pgfpathlineto{\pgfqpoint{0.000000in}{-0.044444in}}%
\pgfusepath{stroke,fill}%
}%
\begin{pgfscope}%
\pgfsys@transformshift{5.908800in}{6.271734in}%
\pgfsys@useobject{currentmarker}{}%
\end{pgfscope}%
\end{pgfscope}%
\begin{pgfscope}%
\pgfsetbuttcap%
\pgfsetroundjoin%
\definecolor{currentfill}{rgb}{0.150000,0.150000,0.150000}%
\pgfsetfillcolor{currentfill}%
\pgfsetlinewidth{0.803000pt}%
\definecolor{currentstroke}{rgb}{0.150000,0.150000,0.150000}%
\pgfsetstrokecolor{currentstroke}%
\pgfsetdash{}{0pt}%
\pgfsys@defobject{currentmarker}{\pgfqpoint{0.000000in}{-0.044444in}}{\pgfqpoint{0.000000in}{0.000000in}}{%
\pgfpathmoveto{\pgfqpoint{0.000000in}{0.000000in}}%
\pgfpathlineto{\pgfqpoint{0.000000in}{-0.044444in}}%
\pgfusepath{stroke,fill}%
}%
\begin{pgfscope}%
\pgfsys@transformshift{5.957391in}{6.271734in}%
\pgfsys@useobject{currentmarker}{}%
\end{pgfscope}%
\end{pgfscope}%
\begin{pgfscope}%
\pgfsetbuttcap%
\pgfsetroundjoin%
\definecolor{currentfill}{rgb}{0.150000,0.150000,0.150000}%
\pgfsetfillcolor{currentfill}%
\pgfsetlinewidth{0.803000pt}%
\definecolor{currentstroke}{rgb}{0.150000,0.150000,0.150000}%
\pgfsetstrokecolor{currentstroke}%
\pgfsetdash{}{0pt}%
\pgfsys@defobject{currentmarker}{\pgfqpoint{0.000000in}{-0.044444in}}{\pgfqpoint{0.000000in}{0.000000in}}{%
\pgfpathmoveto{\pgfqpoint{0.000000in}{0.000000in}}%
\pgfpathlineto{\pgfqpoint{0.000000in}{-0.044444in}}%
\pgfusepath{stroke,fill}%
}%
\begin{pgfscope}%
\pgfsys@transformshift{5.997094in}{6.271734in}%
\pgfsys@useobject{currentmarker}{}%
\end{pgfscope}%
\end{pgfscope}%
\begin{pgfscope}%
\pgfsetbuttcap%
\pgfsetroundjoin%
\definecolor{currentfill}{rgb}{0.150000,0.150000,0.150000}%
\pgfsetfillcolor{currentfill}%
\pgfsetlinewidth{0.803000pt}%
\definecolor{currentstroke}{rgb}{0.150000,0.150000,0.150000}%
\pgfsetstrokecolor{currentstroke}%
\pgfsetdash{}{0pt}%
\pgfsys@defobject{currentmarker}{\pgfqpoint{0.000000in}{-0.044444in}}{\pgfqpoint{0.000000in}{0.000000in}}{%
\pgfpathmoveto{\pgfqpoint{0.000000in}{0.000000in}}%
\pgfpathlineto{\pgfqpoint{0.000000in}{-0.044444in}}%
\pgfusepath{stroke,fill}%
}%
\begin{pgfscope}%
\pgfsys@transformshift{6.030661in}{6.271734in}%
\pgfsys@useobject{currentmarker}{}%
\end{pgfscope}%
\end{pgfscope}%
\begin{pgfscope}%
\pgfsetbuttcap%
\pgfsetroundjoin%
\definecolor{currentfill}{rgb}{0.150000,0.150000,0.150000}%
\pgfsetfillcolor{currentfill}%
\pgfsetlinewidth{0.803000pt}%
\definecolor{currentstroke}{rgb}{0.150000,0.150000,0.150000}%
\pgfsetstrokecolor{currentstroke}%
\pgfsetdash{}{0pt}%
\pgfsys@defobject{currentmarker}{\pgfqpoint{0.000000in}{-0.044444in}}{\pgfqpoint{0.000000in}{0.000000in}}{%
\pgfpathmoveto{\pgfqpoint{0.000000in}{0.000000in}}%
\pgfpathlineto{\pgfqpoint{0.000000in}{-0.044444in}}%
\pgfusepath{stroke,fill}%
}%
\begin{pgfscope}%
\pgfsys@transformshift{6.059739in}{6.271734in}%
\pgfsys@useobject{currentmarker}{}%
\end{pgfscope}%
\end{pgfscope}%
\begin{pgfscope}%
\pgfsetbuttcap%
\pgfsetroundjoin%
\definecolor{currentfill}{rgb}{0.150000,0.150000,0.150000}%
\pgfsetfillcolor{currentfill}%
\pgfsetlinewidth{0.803000pt}%
\definecolor{currentstroke}{rgb}{0.150000,0.150000,0.150000}%
\pgfsetstrokecolor{currentstroke}%
\pgfsetdash{}{0pt}%
\pgfsys@defobject{currentmarker}{\pgfqpoint{0.000000in}{-0.044444in}}{\pgfqpoint{0.000000in}{0.000000in}}{%
\pgfpathmoveto{\pgfqpoint{0.000000in}{0.000000in}}%
\pgfpathlineto{\pgfqpoint{0.000000in}{-0.044444in}}%
\pgfusepath{stroke,fill}%
}%
\begin{pgfscope}%
\pgfsys@transformshift{6.085387in}{6.271734in}%
\pgfsys@useobject{currentmarker}{}%
\end{pgfscope}%
\end{pgfscope}%
\begin{pgfscope}%
\pgfsetbuttcap%
\pgfsetroundjoin%
\definecolor{currentfill}{rgb}{0.150000,0.150000,0.150000}%
\pgfsetfillcolor{currentfill}%
\pgfsetlinewidth{0.803000pt}%
\definecolor{currentstroke}{rgb}{0.150000,0.150000,0.150000}%
\pgfsetstrokecolor{currentstroke}%
\pgfsetdash{}{0pt}%
\pgfsys@defobject{currentmarker}{\pgfqpoint{0.000000in}{-0.044444in}}{\pgfqpoint{0.000000in}{0.000000in}}{%
\pgfpathmoveto{\pgfqpoint{0.000000in}{0.000000in}}%
\pgfpathlineto{\pgfqpoint{0.000000in}{-0.044444in}}%
\pgfusepath{stroke,fill}%
}%
\begin{pgfscope}%
\pgfsys@transformshift{6.259270in}{6.271734in}%
\pgfsys@useobject{currentmarker}{}%
\end{pgfscope}%
\end{pgfscope}%
\begin{pgfscope}%
\pgfsetbuttcap%
\pgfsetroundjoin%
\definecolor{currentfill}{rgb}{0.150000,0.150000,0.150000}%
\pgfsetfillcolor{currentfill}%
\pgfsetlinewidth{1.003750pt}%
\definecolor{currentstroke}{rgb}{0.150000,0.150000,0.150000}%
\pgfsetstrokecolor{currentstroke}%
\pgfsetdash{}{0pt}%
\pgfsys@defobject{currentmarker}{\pgfqpoint{-0.066667in}{0.000000in}}{\pgfqpoint{0.000000in}{0.000000in}}{%
\pgfpathmoveto{\pgfqpoint{0.000000in}{0.000000in}}%
\pgfpathlineto{\pgfqpoint{-0.066667in}{0.000000in}}%
\pgfusepath{stroke,fill}%
}%
\begin{pgfscope}%
\pgfsys@transformshift{5.105513in}{6.271734in}%
\pgfsys@useobject{currentmarker}{}%
\end{pgfscope}%
\end{pgfscope}%
\begin{pgfscope}%
\pgfsetbuttcap%
\pgfsetroundjoin%
\definecolor{currentfill}{rgb}{0.150000,0.150000,0.150000}%
\pgfsetfillcolor{currentfill}%
\pgfsetlinewidth{1.003750pt}%
\definecolor{currentstroke}{rgb}{0.150000,0.150000,0.150000}%
\pgfsetstrokecolor{currentstroke}%
\pgfsetdash{}{0pt}%
\pgfsys@defobject{currentmarker}{\pgfqpoint{-0.066667in}{0.000000in}}{\pgfqpoint{0.000000in}{0.000000in}}{%
\pgfpathmoveto{\pgfqpoint{0.000000in}{0.000000in}}%
\pgfpathlineto{\pgfqpoint{-0.066667in}{0.000000in}}%
\pgfusepath{stroke,fill}%
}%
\begin{pgfscope}%
\pgfsys@transformshift{5.105513in}{6.595492in}%
\pgfsys@useobject{currentmarker}{}%
\end{pgfscope}%
\end{pgfscope}%
\begin{pgfscope}%
\pgfsetbuttcap%
\pgfsetroundjoin%
\definecolor{currentfill}{rgb}{0.150000,0.150000,0.150000}%
\pgfsetfillcolor{currentfill}%
\pgfsetlinewidth{1.003750pt}%
\definecolor{currentstroke}{rgb}{0.150000,0.150000,0.150000}%
\pgfsetstrokecolor{currentstroke}%
\pgfsetdash{}{0pt}%
\pgfsys@defobject{currentmarker}{\pgfqpoint{-0.066667in}{0.000000in}}{\pgfqpoint{0.000000in}{0.000000in}}{%
\pgfpathmoveto{\pgfqpoint{0.000000in}{0.000000in}}%
\pgfpathlineto{\pgfqpoint{-0.066667in}{0.000000in}}%
\pgfusepath{stroke,fill}%
}%
\begin{pgfscope}%
\pgfsys@transformshift{5.105513in}{6.879682in}%
\pgfsys@useobject{currentmarker}{}%
\end{pgfscope}%
\end{pgfscope}%
\begin{pgfscope}%
\pgfpathrectangle{\pgfqpoint{5.105513in}{6.271734in}}{\pgfqpoint{1.223103in}{0.607948in}}%
\pgfusepath{clip}%
\pgfsetroundcap%
\pgfsetroundjoin%
\pgfsetlinewidth{1.204500pt}%
\definecolor{currentstroke}{rgb}{0.000000,0.501961,0.000000}%
\pgfsetstrokecolor{currentstroke}%
\pgfsetdash{}{0pt}%
\pgfpathmoveto{\pgfqpoint{5.105513in}{6.596234in}}%
\pgfpathlineto{\pgfqpoint{5.346222in}{6.597479in}}%
\pgfpathlineto{\pgfqpoint{5.457753in}{6.598709in}}%
\pgfpathlineto{\pgfqpoint{5.531149in}{6.599927in}}%
\pgfpathlineto{\pgfqpoint{5.585945in}{6.601138in}}%
\pgfpathlineto{\pgfqpoint{5.629687in}{6.602343in}}%
\pgfpathlineto{\pgfqpoint{5.666095in}{6.603545in}}%
\pgfpathlineto{\pgfqpoint{5.697279in}{6.604746in}}%
\pgfpathlineto{\pgfqpoint{5.724552in}{6.605948in}}%
\pgfpathlineto{\pgfqpoint{5.748786in}{6.607153in}}%
\pgfpathlineto{\pgfqpoint{5.770591in}{6.608363in}}%
\pgfpathlineto{\pgfqpoint{5.790410in}{6.609578in}}%
\pgfpathlineto{\pgfqpoint{5.808575in}{6.610800in}}%
\pgfpathlineto{\pgfqpoint{5.825341in}{6.612029in}}%
\pgfpathlineto{\pgfqpoint{5.840907in}{6.613267in}}%
\pgfpathlineto{\pgfqpoint{5.855435in}{6.614515in}}%
\pgfpathlineto{\pgfqpoint{5.869053in}{6.615772in}}%
\pgfpathlineto{\pgfqpoint{5.881870in}{6.617039in}}%
\pgfpathlineto{\pgfqpoint{5.893974in}{6.618317in}}%
\pgfpathlineto{\pgfqpoint{5.905441in}{6.619607in}}%
\pgfpathlineto{\pgfqpoint{5.916334in}{6.620908in}}%
\pgfpathlineto{\pgfqpoint{5.926708in}{6.622221in}}%
\pgfpathlineto{\pgfqpoint{5.936610in}{6.623546in}}%
\pgfpathlineto{\pgfqpoint{5.946082in}{6.624884in}}%
\pgfpathlineto{\pgfqpoint{5.955158in}{6.626235in}}%
\pgfpathlineto{\pgfqpoint{5.963871in}{6.627600in}}%
\pgfpathlineto{\pgfqpoint{5.972249in}{6.628979in}}%
\pgfpathlineto{\pgfqpoint{5.980317in}{6.630374in}}%
\pgfpathlineto{\pgfqpoint{5.988096in}{6.631784in}}%
\pgfpathlineto{\pgfqpoint{5.995607in}{6.633211in}}%
\pgfpathlineto{\pgfqpoint{6.002868in}{6.634656in}}%
\pgfpathlineto{\pgfqpoint{6.009894in}{6.636121in}}%
\pgfpathlineto{\pgfqpoint{6.016701in}{6.637607in}}%
\pgfpathlineto{\pgfqpoint{6.023301in}{6.639115in}}%
\pgfpathlineto{\pgfqpoint{6.029707in}{6.640648in}}%
\pgfpathlineto{\pgfqpoint{6.035930in}{6.642207in}}%
\pgfpathlineto{\pgfqpoint{6.041980in}{6.643797in}}%
\pgfpathlineto{\pgfqpoint{6.047867in}{6.645418in}}%
\pgfpathlineto{\pgfqpoint{6.053598in}{6.647075in}}%
\pgfpathlineto{\pgfqpoint{6.059183in}{6.648771in}}%
\pgfpathlineto{\pgfqpoint{6.064628in}{6.650509in}}%
\pgfpathlineto{\pgfqpoint{6.069940in}{6.652294in}}%
\pgfpathlineto{\pgfqpoint{6.075126in}{6.654130in}}%
\pgfpathlineto{\pgfqpoint{6.080191in}{6.656023in}}%
\pgfpathlineto{\pgfqpoint{6.085140in}{6.657978in}}%
\pgfpathlineto{\pgfqpoint{6.089980in}{6.660000in}}%
\pgfpathlineto{\pgfqpoint{6.094715in}{6.662096in}}%
\pgfpathlineto{\pgfqpoint{6.099349in}{6.664273in}}%
\pgfpathlineto{\pgfqpoint{6.103886in}{6.666538in}}%
\pgfpathlineto{\pgfqpoint{6.108331in}{6.668899in}}%
\pgfusepath{stroke}%
\end{pgfscope}%
\begin{pgfscope}%
\pgfsetrectcap%
\pgfsetmiterjoin%
\pgfsetlinewidth{1.003750pt}%
\definecolor{currentstroke}{rgb}{0.150000,0.150000,0.150000}%
\pgfsetstrokecolor{currentstroke}%
\pgfsetdash{}{0pt}%
\pgfpathmoveto{\pgfqpoint{5.105513in}{6.271734in}}%
\pgfpathlineto{\pgfqpoint{5.105513in}{6.879682in}}%
\pgfusepath{stroke}%
\end{pgfscope}%
\begin{pgfscope}%
\pgfsetrectcap%
\pgfsetmiterjoin%
\pgfsetlinewidth{1.003750pt}%
\definecolor{currentstroke}{rgb}{0.150000,0.150000,0.150000}%
\pgfsetstrokecolor{currentstroke}%
\pgfsetdash{}{0pt}%
\pgfpathmoveto{\pgfqpoint{5.105513in}{6.271734in}}%
\pgfpathlineto{\pgfqpoint{6.328616in}{6.271734in}}%
\pgfusepath{stroke}%
\end{pgfscope}%
\begin{pgfscope}%
\pgfpathrectangle{\pgfqpoint{5.105513in}{6.271734in}}{\pgfqpoint{1.223103in}{0.607948in}}%
\pgfusepath{clip}%
\pgfsetbuttcap%
\pgfsetroundjoin%
\definecolor{currentfill}{rgb}{0.000000,0.000000,0.000000}%
\pgfsetfillcolor{currentfill}%
\pgfsetlinewidth{1.003750pt}%
\definecolor{currentstroke}{rgb}{0.000000,0.000000,0.000000}%
\pgfsetstrokecolor{currentstroke}%
\pgfsetdash{}{0pt}%
\pgfsys@defobject{currentmarker}{\pgfqpoint{-0.013889in}{-0.013889in}}{\pgfqpoint{0.013889in}{0.013889in}}{%
\pgfpathmoveto{\pgfqpoint{0.000000in}{-0.013889in}}%
\pgfpathcurveto{\pgfqpoint{0.003683in}{-0.013889in}}{\pgfqpoint{0.007216in}{-0.012425in}}{\pgfqpoint{0.009821in}{-0.009821in}}%
\pgfpathcurveto{\pgfqpoint{0.012425in}{-0.007216in}}{\pgfqpoint{0.013889in}{-0.003683in}}{\pgfqpoint{0.013889in}{0.000000in}}%
\pgfpathcurveto{\pgfqpoint{0.013889in}{0.003683in}}{\pgfqpoint{0.012425in}{0.007216in}}{\pgfqpoint{0.009821in}{0.009821in}}%
\pgfpathcurveto{\pgfqpoint{0.007216in}{0.012425in}}{\pgfqpoint{0.003683in}{0.013889in}}{\pgfqpoint{0.000000in}{0.013889in}}%
\pgfpathcurveto{\pgfqpoint{-0.003683in}{0.013889in}}{\pgfqpoint{-0.007216in}{0.012425in}}{\pgfqpoint{-0.009821in}{0.009821in}}%
\pgfpathcurveto{\pgfqpoint{-0.012425in}{0.007216in}}{\pgfqpoint{-0.013889in}{0.003683in}}{\pgfqpoint{-0.013889in}{0.000000in}}%
\pgfpathcurveto{\pgfqpoint{-0.013889in}{-0.003683in}}{\pgfqpoint{-0.012425in}{-0.007216in}}{\pgfqpoint{-0.009821in}{-0.009821in}}%
\pgfpathcurveto{\pgfqpoint{-0.007216in}{-0.012425in}}{\pgfqpoint{-0.003683in}{-0.013889in}}{\pgfqpoint{0.000000in}{-0.013889in}}%
\pgfpathclose%
\pgfusepath{stroke,fill}%
}%
\begin{pgfscope}%
\pgfsys@transformshift{5.757861in}{6.607603in}%
\pgfsys@useobject{currentmarker}{}%
\end{pgfscope}%
\begin{pgfscope}%
\pgfsys@transformshift{5.762260in}{6.607845in}%
\pgfsys@useobject{currentmarker}{}%
\end{pgfscope}%
\begin{pgfscope}%
\pgfsys@transformshift{5.766750in}{6.608113in}%
\pgfsys@useobject{currentmarker}{}%
\end{pgfscope}%
\begin{pgfscope}%
\pgfsys@transformshift{5.771335in}{6.608376in}%
\pgfsys@useobject{currentmarker}{}%
\end{pgfscope}%
\begin{pgfscope}%
\pgfsys@transformshift{5.776018in}{6.608669in}%
\pgfsys@useobject{currentmarker}{}%
\end{pgfscope}%
\begin{pgfscope}%
\pgfsys@transformshift{5.780804in}{6.608955in}%
\pgfsys@useobject{currentmarker}{}%
\end{pgfscope}%
\begin{pgfscope}%
\pgfsys@transformshift{5.785698in}{6.609276in}%
\pgfsys@useobject{currentmarker}{}%
\end{pgfscope}%
\begin{pgfscope}%
\pgfsys@transformshift{5.790704in}{6.609589in}%
\pgfsys@useobject{currentmarker}{}%
\end{pgfscope}%
\begin{pgfscope}%
\pgfsys@transformshift{5.795828in}{6.609942in}%
\pgfsys@useobject{currentmarker}{}%
\end{pgfscope}%
\begin{pgfscope}%
\pgfsys@transformshift{5.801075in}{6.610286in}%
\pgfsys@useobject{currentmarker}{}%
\end{pgfscope}%
\begin{pgfscope}%
\pgfsys@transformshift{5.806452in}{6.610676in}%
\pgfsys@useobject{currentmarker}{}%
\end{pgfscope}%
\begin{pgfscope}%
\pgfsys@transformshift{5.835530in}{6.612869in}%
\pgfsys@useobject{currentmarker}{}%
\end{pgfscope}%
\begin{pgfscope}%
\pgfsys@transformshift{5.869098in}{6.615873in}%
\pgfsys@useobject{currentmarker}{}%
\end{pgfscope}%
\begin{pgfscope}%
\pgfsys@transformshift{5.908800in}{6.620060in}%
\pgfsys@useobject{currentmarker}{}%
\end{pgfscope}%
\begin{pgfscope}%
\pgfsys@transformshift{5.957391in}{6.626700in}%
\pgfsys@useobject{currentmarker}{}%
\end{pgfscope}%
\begin{pgfscope}%
\pgfsys@transformshift{6.020037in}{6.638107in}%
\pgfsys@useobject{currentmarker}{}%
\end{pgfscope}%
\begin{pgfscope}%
\pgfsys@transformshift{6.108331in}{6.668981in}%
\pgfsys@useobject{currentmarker}{}%
\end{pgfscope}%
\end{pgfscope}%
\begin{pgfscope}%
\pgfsetbuttcap%
\pgfsetmiterjoin%
\definecolor{currentfill}{rgb}{1.000000,1.000000,1.000000}%
\pgfsetfillcolor{currentfill}%
\pgfsetlinewidth{0.803000pt}%
\definecolor{currentstroke}{rgb}{1.000000,1.000000,1.000000}%
\pgfsetstrokecolor{currentstroke}%
\pgfsetdash{}{0pt}%
\pgfpathmoveto{\pgfqpoint{6.297392in}{6.830806in}}%
\pgfpathlineto{\pgfqpoint{6.297392in}{6.320610in}}%
\pgfpathlineto{\pgfqpoint{6.478411in}{6.320610in}}%
\pgfpathlineto{\pgfqpoint{6.478411in}{6.830806in}}%
\pgfpathclose%
\pgfusepath{stroke,fill}%
\end{pgfscope}%
\begin{pgfscope}%
\definecolor{textcolor}{rgb}{0.150000,0.150000,0.150000}%
\pgfsetstrokecolor{textcolor}%
\pgfsetfillcolor{textcolor}%
\pgftext[x=6.368294in,y=6.775120in,left,base,rotate=270.000000]{\color{textcolor}\sffamily\fontsize{5.647059}{6.776471}\selectfont nlevel = 18}%
\end{pgfscope}%
\begin{pgfscope}%
\pgfsetbuttcap%
\pgfsetmiterjoin%
\definecolor{currentfill}{rgb}{1.000000,1.000000,1.000000}%
\pgfsetfillcolor{currentfill}%
\pgfsetlinewidth{0.803000pt}%
\definecolor{currentstroke}{rgb}{1.000000,1.000000,1.000000}%
\pgfsetstrokecolor{currentstroke}%
\pgfsetdash{}{0pt}%
\pgfpathmoveto{\pgfqpoint{6.297392in}{6.830806in}}%
\pgfpathlineto{\pgfqpoint{6.297392in}{6.320610in}}%
\pgfpathlineto{\pgfqpoint{6.478411in}{6.320610in}}%
\pgfpathlineto{\pgfqpoint{6.478411in}{6.830806in}}%
\pgfpathclose%
\pgfusepath{stroke,fill}%
\end{pgfscope}%
\begin{pgfscope}%
\definecolor{textcolor}{rgb}{0.150000,0.150000,0.150000}%
\pgfsetstrokecolor{textcolor}%
\pgfsetfillcolor{textcolor}%
\pgftext[x=6.368294in,y=6.775120in,left,base,rotate=270.000000]{\color{textcolor}\sffamily\fontsize{5.647059}{6.776471}\selectfont nlevel = 18}%
\end{pgfscope}%
\begin{pgfscope}%
\definecolor{textcolor}{rgb}{0.150000,0.150000,0.150000}%
\pgfsetstrokecolor{textcolor}%
\pgfsetfillcolor{textcolor}%
\pgftext[x=5.717064in,y=6.963016in,,base]{\color{textcolor}\sffamily\fontsize{5.647059}{6.776471}\selectfont \(\displaystyle n_{\mathrm{step}} = -1\)}%
\end{pgfscope}%
\begin{pgfscope}%
\pgfsetbuttcap%
\pgfsetmiterjoin%
\definecolor{currentfill}{rgb}{1.000000,1.000000,1.000000}%
\pgfsetfillcolor{currentfill}%
\pgfsetlinewidth{0.000000pt}%
\definecolor{currentstroke}{rgb}{0.000000,0.000000,0.000000}%
\pgfsetstrokecolor{currentstroke}%
\pgfsetstrokeopacity{0.000000}%
\pgfsetdash{}{0pt}%
\pgfpathmoveto{\pgfqpoint{0.702340in}{5.542197in}}%
\pgfpathlineto{\pgfqpoint{1.925444in}{5.542197in}}%
\pgfpathlineto{\pgfqpoint{1.925444in}{6.150145in}}%
\pgfpathlineto{\pgfqpoint{0.702340in}{6.150145in}}%
\pgfpathclose%
\pgfusepath{fill}%
\end{pgfscope}%
\begin{pgfscope}%
\pgfpathrectangle{\pgfqpoint{0.702340in}{5.542197in}}{\pgfqpoint{1.223103in}{0.607948in}}%
\pgfusepath{clip}%
\pgfsetbuttcap%
\pgfsetmiterjoin%
\definecolor{currentfill}{rgb}{0.000000,0.000000,1.000000}%
\pgfsetfillcolor{currentfill}%
\pgfsetfillopacity{0.100000}%
\pgfsetlinewidth{0.803000pt}%
\definecolor{currentstroke}{rgb}{0.000000,0.000000,1.000000}%
\pgfsetstrokecolor{currentstroke}%
\pgfsetstrokeopacity{0.100000}%
\pgfsetdash{}{0pt}%
\pgfpathmoveto{\pgfqpoint{0.702340in}{5.527672in}}%
\pgfpathlineto{\pgfqpoint{0.702340in}{5.788060in}}%
\pgfpathlineto{\pgfqpoint{1.925444in}{5.788060in}}%
\pgfpathlineto{\pgfqpoint{1.925444in}{5.527672in}}%
\pgfpathclose%
\pgfusepath{stroke,fill}%
\end{pgfscope}%
\begin{pgfscope}%
\pgfpathrectangle{\pgfqpoint{0.702340in}{5.542197in}}{\pgfqpoint{1.223103in}{0.607948in}}%
\pgfusepath{clip}%
\pgfsetbuttcap%
\pgfsetroundjoin%
\definecolor{currentfill}{rgb}{0.000000,0.501961,0.000000}%
\pgfsetfillcolor{currentfill}%
\pgfsetfillopacity{0.500000}%
\pgfsetlinewidth{0.803000pt}%
\definecolor{currentstroke}{rgb}{0.000000,0.501961,0.000000}%
\pgfsetstrokecolor{currentstroke}%
\pgfsetstrokeopacity{0.500000}%
\pgfsetdash{}{0pt}%
\pgfpathmoveto{\pgfqpoint{0.702340in}{5.773937in}}%
\pgfpathlineto{\pgfqpoint{0.702340in}{5.536956in}}%
\pgfpathlineto{\pgfqpoint{0.943050in}{5.550295in}}%
\pgfpathlineto{\pgfqpoint{1.054581in}{5.556667in}}%
\pgfpathlineto{\pgfqpoint{1.127977in}{5.556400in}}%
\pgfpathlineto{\pgfqpoint{1.182772in}{5.549818in}}%
\pgfpathlineto{\pgfqpoint{1.226515in}{5.537233in}}%
\pgfpathlineto{\pgfqpoint{1.262923in}{5.518957in}}%
\pgfpathlineto{\pgfqpoint{1.294107in}{5.495292in}}%
\pgfpathlineto{\pgfqpoint{1.321380in}{5.466534in}}%
\pgfpathlineto{\pgfqpoint{1.345614in}{5.432977in}}%
\pgfpathlineto{\pgfqpoint{1.367419in}{5.394894in}}%
\pgfpathlineto{\pgfqpoint{1.387238in}{5.349875in}}%
\pgfpathlineto{\pgfqpoint{1.405403in}{5.299786in}}%
\pgfpathlineto{\pgfqpoint{1.422168in}{5.247697in}}%
\pgfpathlineto{\pgfqpoint{1.437735in}{5.193623in}}%
\pgfpathlineto{\pgfqpoint{1.452262in}{5.137563in}}%
\pgfpathlineto{\pgfqpoint{1.465881in}{5.079507in}}%
\pgfpathlineto{\pgfqpoint{1.478698in}{5.019402in}}%
\pgfpathlineto{\pgfqpoint{1.490802in}{4.957045in}}%
\pgfpathlineto{\pgfqpoint{1.502269in}{4.891880in}}%
\pgfpathlineto{\pgfqpoint{1.513162in}{4.823767in}}%
\pgfpathlineto{\pgfqpoint{1.523536in}{4.753406in}}%
\pgfpathlineto{\pgfqpoint{1.533438in}{4.681261in}}%
\pgfpathlineto{\pgfqpoint{1.542909in}{4.607584in}}%
\pgfpathlineto{\pgfqpoint{1.551986in}{4.532556in}}%
\pgfpathlineto{\pgfqpoint{1.560699in}{4.456320in}}%
\pgfpathlineto{\pgfqpoint{1.569077in}{4.378992in}}%
\pgfpathlineto{\pgfqpoint{1.577145in}{4.300663in}}%
\pgfpathlineto{\pgfqpoint{1.584924in}{4.221395in}}%
\pgfpathlineto{\pgfqpoint{1.592435in}{4.141214in}}%
\pgfpathlineto{\pgfqpoint{1.599696in}{4.060085in}}%
\pgfpathlineto{\pgfqpoint{1.606722in}{3.977822in}}%
\pgfpathlineto{\pgfqpoint{1.613528in}{3.893663in}}%
\pgfpathlineto{\pgfqpoint{1.620129in}{3.806128in}}%
\pgfpathlineto{\pgfqpoint{1.626535in}{3.715572in}}%
\pgfpathlineto{\pgfqpoint{1.632758in}{3.622858in}}%
\pgfpathlineto{\pgfqpoint{1.638808in}{3.528254in}}%
\pgfpathlineto{\pgfqpoint{1.644694in}{3.431847in}}%
\pgfpathlineto{\pgfqpoint{1.650426in}{3.333681in}}%
\pgfpathlineto{\pgfqpoint{1.656011in}{3.233786in}}%
\pgfpathlineto{\pgfqpoint{1.661456in}{3.132187in}}%
\pgfpathlineto{\pgfqpoint{1.666768in}{3.028914in}}%
\pgfpathlineto{\pgfqpoint{1.671953in}{2.923996in}}%
\pgfpathlineto{\pgfqpoint{1.677018in}{2.817467in}}%
\pgfpathlineto{\pgfqpoint{1.681968in}{2.709365in}}%
\pgfpathlineto{\pgfqpoint{1.686808in}{2.599733in}}%
\pgfpathlineto{\pgfqpoint{1.691543in}{2.488618in}}%
\pgfpathlineto{\pgfqpoint{1.696176in}{2.376074in}}%
\pgfpathlineto{\pgfqpoint{1.700714in}{2.262156in}}%
\pgfpathlineto{\pgfqpoint{1.705158in}{2.146305in}}%
\pgfpathlineto{\pgfqpoint{1.705158in}{2.147981in}}%
\pgfpathlineto{\pgfqpoint{1.705158in}{2.147981in}}%
\pgfpathlineto{\pgfqpoint{1.700714in}{2.286793in}}%
\pgfpathlineto{\pgfqpoint{1.696176in}{2.418643in}}%
\pgfpathlineto{\pgfqpoint{1.691543in}{2.543702in}}%
\pgfpathlineto{\pgfqpoint{1.686808in}{2.662700in}}%
\pgfpathlineto{\pgfqpoint{1.681968in}{2.776318in}}%
\pgfpathlineto{\pgfqpoint{1.677018in}{2.885179in}}%
\pgfpathlineto{\pgfqpoint{1.671953in}{2.989855in}}%
\pgfpathlineto{\pgfqpoint{1.666768in}{3.090863in}}%
\pgfpathlineto{\pgfqpoint{1.661456in}{3.188674in}}%
\pgfpathlineto{\pgfqpoint{1.656011in}{3.283708in}}%
\pgfpathlineto{\pgfqpoint{1.650426in}{3.376343in}}%
\pgfpathlineto{\pgfqpoint{1.644694in}{3.466913in}}%
\pgfpathlineto{\pgfqpoint{1.638808in}{3.555716in}}%
\pgfpathlineto{\pgfqpoint{1.632758in}{3.643024in}}%
\pgfpathlineto{\pgfqpoint{1.626535in}{3.729115in}}%
\pgfpathlineto{\pgfqpoint{1.620129in}{3.814417in}}%
\pgfpathlineto{\pgfqpoint{1.613528in}{3.899912in}}%
\pgfpathlineto{\pgfqpoint{1.606722in}{3.986048in}}%
\pgfpathlineto{\pgfqpoint{1.599696in}{4.071400in}}%
\pgfpathlineto{\pgfqpoint{1.592435in}{4.155237in}}%
\pgfpathlineto{\pgfqpoint{1.584924in}{4.237371in}}%
\pgfpathlineto{\pgfqpoint{1.577145in}{4.317741in}}%
\pgfpathlineto{\pgfqpoint{1.569077in}{4.396318in}}%
\pgfpathlineto{\pgfqpoint{1.560699in}{4.473087in}}%
\pgfpathlineto{\pgfqpoint{1.551986in}{4.548038in}}%
\pgfpathlineto{\pgfqpoint{1.542909in}{4.621167in}}%
\pgfpathlineto{\pgfqpoint{1.533438in}{4.692474in}}%
\pgfpathlineto{\pgfqpoint{1.523536in}{4.761978in}}%
\pgfpathlineto{\pgfqpoint{1.513162in}{4.829753in}}%
\pgfpathlineto{\pgfqpoint{1.502269in}{4.896070in}}%
\pgfpathlineto{\pgfqpoint{1.490802in}{4.961419in}}%
\pgfpathlineto{\pgfqpoint{1.478698in}{5.025441in}}%
\pgfpathlineto{\pgfqpoint{1.465881in}{5.087343in}}%
\pgfpathlineto{\pgfqpoint{1.452262in}{5.146681in}}%
\pgfpathlineto{\pgfqpoint{1.437735in}{5.203150in}}%
\pgfpathlineto{\pgfqpoint{1.422168in}{5.256478in}}%
\pgfpathlineto{\pgfqpoint{1.405403in}{5.306400in}}%
\pgfpathlineto{\pgfqpoint{1.387238in}{5.352660in}}%
\pgfpathlineto{\pgfqpoint{1.367419in}{5.398053in}}%
\pgfpathlineto{\pgfqpoint{1.345614in}{5.444171in}}%
\pgfpathlineto{\pgfqpoint{1.321380in}{5.488327in}}%
\pgfpathlineto{\pgfqpoint{1.294107in}{5.530521in}}%
\pgfpathlineto{\pgfqpoint{1.262923in}{5.570772in}}%
\pgfpathlineto{\pgfqpoint{1.226515in}{5.609108in}}%
\pgfpathlineto{\pgfqpoint{1.182772in}{5.645570in}}%
\pgfpathlineto{\pgfqpoint{1.127977in}{5.680209in}}%
\pgfpathlineto{\pgfqpoint{1.054581in}{5.713091in}}%
\pgfpathlineto{\pgfqpoint{0.943050in}{5.744301in}}%
\pgfpathlineto{\pgfqpoint{0.702340in}{5.773937in}}%
\pgfpathclose%
\pgfusepath{stroke,fill}%
\end{pgfscope}%
\begin{pgfscope}%
\pgfpathrectangle{\pgfqpoint{0.702340in}{5.542197in}}{\pgfqpoint{1.223103in}{0.607948in}}%
\pgfusepath{clip}%
\pgfsetroundcap%
\pgfsetroundjoin%
\pgfsetlinewidth{0.501875pt}%
\definecolor{currentstroke}{rgb}{0.000000,0.000000,1.000000}%
\pgfsetstrokecolor{currentstroke}%
\pgfsetstrokeopacity{0.800000}%
\pgfsetdash{}{0pt}%
\pgfpathmoveto{\pgfqpoint{0.702340in}{5.657866in}}%
\pgfpathlineto{\pgfqpoint{1.925444in}{5.657866in}}%
\pgfusepath{stroke}%
\end{pgfscope}%
\begin{pgfscope}%
\pgfpathrectangle{\pgfqpoint{0.702340in}{5.542197in}}{\pgfqpoint{1.223103in}{0.607948in}}%
\pgfusepath{clip}%
\pgfsetbuttcap%
\pgfsetroundjoin%
\pgfsetlinewidth{1.003750pt}%
\definecolor{currentstroke}{rgb}{0.000000,0.000000,0.000000}%
\pgfsetstrokecolor{currentstroke}%
\pgfsetdash{{3.700000pt}{1.600000pt}}{0.000000pt}%
\pgfpathmoveto{\pgfqpoint{0.702340in}{5.616260in}}%
\pgfpathlineto{\pgfqpoint{1.925444in}{5.616260in}}%
\pgfusepath{stroke}%
\end{pgfscope}%
\begin{pgfscope}%
\pgfsetroundcap%
\pgfsetroundjoin%
\pgfsetlinewidth{0.501875pt}%
\definecolor{currentstroke}{rgb}{0.000000,0.000000,1.000000}%
\pgfsetstrokecolor{currentstroke}%
\pgfsetstrokeopacity{0.800000}%
\pgfsetdash{}{0pt}%
\pgfpathmoveto{\pgfqpoint{1.506945in}{5.782537in}}%
\pgfpathquadraticcurveto{\pgfqpoint{1.441564in}{5.729001in}}{\pgfqpoint{1.376183in}{5.675466in}}%
\pgfusepath{stroke}%
\end{pgfscope}%
\begin{pgfscope}%
\pgfsetfillopacity{0.800000}%
\pgfsetstrokeopacity{0.800000}%
\definecolor{textcolor}{rgb}{0.000000,0.000000,1.000000}%
\pgfsetstrokecolor{textcolor}%
\pgfsetfillcolor{textcolor}%
\pgftext[x=1.442982in,y=5.848855in,left,base]{\color{textcolor}\sffamily\fontsize{5.647059}{6.776471}\selectfont 16.19(21)}%
\end{pgfscope}%
\begin{pgfscope}%
\pgfsetbuttcap%
\pgfsetroundjoin%
\definecolor{currentfill}{rgb}{0.150000,0.150000,0.150000}%
\pgfsetfillcolor{currentfill}%
\pgfsetlinewidth{1.003750pt}%
\definecolor{currentstroke}{rgb}{0.150000,0.150000,0.150000}%
\pgfsetstrokecolor{currentstroke}%
\pgfsetdash{}{0pt}%
\pgfsys@defobject{currentmarker}{\pgfqpoint{0.000000in}{-0.066667in}}{\pgfqpoint{0.000000in}{0.000000in}}{%
\pgfpathmoveto{\pgfqpoint{0.000000in}{0.000000in}}%
\pgfpathlineto{\pgfqpoint{0.000000in}{-0.066667in}}%
\pgfusepath{stroke,fill}%
}%
\begin{pgfscope}%
\pgfsys@transformshift{0.702340in}{5.542197in}%
\pgfsys@useobject{currentmarker}{}%
\end{pgfscope}%
\end{pgfscope}%
\begin{pgfscope}%
\pgfsetbuttcap%
\pgfsetroundjoin%
\definecolor{currentfill}{rgb}{0.150000,0.150000,0.150000}%
\pgfsetfillcolor{currentfill}%
\pgfsetlinewidth{1.003750pt}%
\definecolor{currentstroke}{rgb}{0.150000,0.150000,0.150000}%
\pgfsetstrokecolor{currentstroke}%
\pgfsetdash{}{0pt}%
\pgfsys@defobject{currentmarker}{\pgfqpoint{0.000000in}{-0.066667in}}{\pgfqpoint{0.000000in}{0.000000in}}{%
\pgfpathmoveto{\pgfqpoint{0.000000in}{0.000000in}}%
\pgfpathlineto{\pgfqpoint{0.000000in}{-0.066667in}}%
\pgfusepath{stroke,fill}%
}%
\begin{pgfscope}%
\pgfsys@transformshift{1.203749in}{5.542197in}%
\pgfsys@useobject{currentmarker}{}%
\end{pgfscope}%
\end{pgfscope}%
\begin{pgfscope}%
\pgfsetbuttcap%
\pgfsetroundjoin%
\definecolor{currentfill}{rgb}{0.150000,0.150000,0.150000}%
\pgfsetfillcolor{currentfill}%
\pgfsetlinewidth{1.003750pt}%
\definecolor{currentstroke}{rgb}{0.150000,0.150000,0.150000}%
\pgfsetstrokecolor{currentstroke}%
\pgfsetdash{}{0pt}%
\pgfsys@defobject{currentmarker}{\pgfqpoint{0.000000in}{-0.066667in}}{\pgfqpoint{0.000000in}{0.000000in}}{%
\pgfpathmoveto{\pgfqpoint{0.000000in}{0.000000in}}%
\pgfpathlineto{\pgfqpoint{0.000000in}{-0.066667in}}%
\pgfusepath{stroke,fill}%
}%
\begin{pgfscope}%
\pgfsys@transformshift{1.705158in}{5.542197in}%
\pgfsys@useobject{currentmarker}{}%
\end{pgfscope}%
\end{pgfscope}%
\begin{pgfscope}%
\pgfsetbuttcap%
\pgfsetroundjoin%
\definecolor{currentfill}{rgb}{0.150000,0.150000,0.150000}%
\pgfsetfillcolor{currentfill}%
\pgfsetlinewidth{0.803000pt}%
\definecolor{currentstroke}{rgb}{0.150000,0.150000,0.150000}%
\pgfsetstrokecolor{currentstroke}%
\pgfsetdash{}{0pt}%
\pgfsys@defobject{currentmarker}{\pgfqpoint{0.000000in}{-0.044444in}}{\pgfqpoint{0.000000in}{0.000000in}}{%
\pgfpathmoveto{\pgfqpoint{0.000000in}{0.000000in}}%
\pgfpathlineto{\pgfqpoint{0.000000in}{-0.044444in}}%
\pgfusepath{stroke,fill}%
}%
\begin{pgfscope}%
\pgfsys@transformshift{0.853280in}{5.542197in}%
\pgfsys@useobject{currentmarker}{}%
\end{pgfscope}%
\end{pgfscope}%
\begin{pgfscope}%
\pgfsetbuttcap%
\pgfsetroundjoin%
\definecolor{currentfill}{rgb}{0.150000,0.150000,0.150000}%
\pgfsetfillcolor{currentfill}%
\pgfsetlinewidth{0.803000pt}%
\definecolor{currentstroke}{rgb}{0.150000,0.150000,0.150000}%
\pgfsetstrokecolor{currentstroke}%
\pgfsetdash{}{0pt}%
\pgfsys@defobject{currentmarker}{\pgfqpoint{0.000000in}{-0.044444in}}{\pgfqpoint{0.000000in}{0.000000in}}{%
\pgfpathmoveto{\pgfqpoint{0.000000in}{0.000000in}}%
\pgfpathlineto{\pgfqpoint{0.000000in}{-0.044444in}}%
\pgfusepath{stroke,fill}%
}%
\begin{pgfscope}%
\pgfsys@transformshift{0.941573in}{5.542197in}%
\pgfsys@useobject{currentmarker}{}%
\end{pgfscope}%
\end{pgfscope}%
\begin{pgfscope}%
\pgfsetbuttcap%
\pgfsetroundjoin%
\definecolor{currentfill}{rgb}{0.150000,0.150000,0.150000}%
\pgfsetfillcolor{currentfill}%
\pgfsetlinewidth{0.803000pt}%
\definecolor{currentstroke}{rgb}{0.150000,0.150000,0.150000}%
\pgfsetstrokecolor{currentstroke}%
\pgfsetdash{}{0pt}%
\pgfsys@defobject{currentmarker}{\pgfqpoint{0.000000in}{-0.044444in}}{\pgfqpoint{0.000000in}{0.000000in}}{%
\pgfpathmoveto{\pgfqpoint{0.000000in}{0.000000in}}%
\pgfpathlineto{\pgfqpoint{0.000000in}{-0.044444in}}%
\pgfusepath{stroke,fill}%
}%
\begin{pgfscope}%
\pgfsys@transformshift{1.004219in}{5.542197in}%
\pgfsys@useobject{currentmarker}{}%
\end{pgfscope}%
\end{pgfscope}%
\begin{pgfscope}%
\pgfsetbuttcap%
\pgfsetroundjoin%
\definecolor{currentfill}{rgb}{0.150000,0.150000,0.150000}%
\pgfsetfillcolor{currentfill}%
\pgfsetlinewidth{0.803000pt}%
\definecolor{currentstroke}{rgb}{0.150000,0.150000,0.150000}%
\pgfsetstrokecolor{currentstroke}%
\pgfsetdash{}{0pt}%
\pgfsys@defobject{currentmarker}{\pgfqpoint{0.000000in}{-0.044444in}}{\pgfqpoint{0.000000in}{0.000000in}}{%
\pgfpathmoveto{\pgfqpoint{0.000000in}{0.000000in}}%
\pgfpathlineto{\pgfqpoint{0.000000in}{-0.044444in}}%
\pgfusepath{stroke,fill}%
}%
\begin{pgfscope}%
\pgfsys@transformshift{1.052810in}{5.542197in}%
\pgfsys@useobject{currentmarker}{}%
\end{pgfscope}%
\end{pgfscope}%
\begin{pgfscope}%
\pgfsetbuttcap%
\pgfsetroundjoin%
\definecolor{currentfill}{rgb}{0.150000,0.150000,0.150000}%
\pgfsetfillcolor{currentfill}%
\pgfsetlinewidth{0.803000pt}%
\definecolor{currentstroke}{rgb}{0.150000,0.150000,0.150000}%
\pgfsetstrokecolor{currentstroke}%
\pgfsetdash{}{0pt}%
\pgfsys@defobject{currentmarker}{\pgfqpoint{0.000000in}{-0.044444in}}{\pgfqpoint{0.000000in}{0.000000in}}{%
\pgfpathmoveto{\pgfqpoint{0.000000in}{0.000000in}}%
\pgfpathlineto{\pgfqpoint{0.000000in}{-0.044444in}}%
\pgfusepath{stroke,fill}%
}%
\begin{pgfscope}%
\pgfsys@transformshift{1.092512in}{5.542197in}%
\pgfsys@useobject{currentmarker}{}%
\end{pgfscope}%
\end{pgfscope}%
\begin{pgfscope}%
\pgfsetbuttcap%
\pgfsetroundjoin%
\definecolor{currentfill}{rgb}{0.150000,0.150000,0.150000}%
\pgfsetfillcolor{currentfill}%
\pgfsetlinewidth{0.803000pt}%
\definecolor{currentstroke}{rgb}{0.150000,0.150000,0.150000}%
\pgfsetstrokecolor{currentstroke}%
\pgfsetdash{}{0pt}%
\pgfsys@defobject{currentmarker}{\pgfqpoint{0.000000in}{-0.044444in}}{\pgfqpoint{0.000000in}{0.000000in}}{%
\pgfpathmoveto{\pgfqpoint{0.000000in}{0.000000in}}%
\pgfpathlineto{\pgfqpoint{0.000000in}{-0.044444in}}%
\pgfusepath{stroke,fill}%
}%
\begin{pgfscope}%
\pgfsys@transformshift{1.126080in}{5.542197in}%
\pgfsys@useobject{currentmarker}{}%
\end{pgfscope}%
\end{pgfscope}%
\begin{pgfscope}%
\pgfsetbuttcap%
\pgfsetroundjoin%
\definecolor{currentfill}{rgb}{0.150000,0.150000,0.150000}%
\pgfsetfillcolor{currentfill}%
\pgfsetlinewidth{0.803000pt}%
\definecolor{currentstroke}{rgb}{0.150000,0.150000,0.150000}%
\pgfsetstrokecolor{currentstroke}%
\pgfsetdash{}{0pt}%
\pgfsys@defobject{currentmarker}{\pgfqpoint{0.000000in}{-0.044444in}}{\pgfqpoint{0.000000in}{0.000000in}}{%
\pgfpathmoveto{\pgfqpoint{0.000000in}{0.000000in}}%
\pgfpathlineto{\pgfqpoint{0.000000in}{-0.044444in}}%
\pgfusepath{stroke,fill}%
}%
\begin{pgfscope}%
\pgfsys@transformshift{1.155158in}{5.542197in}%
\pgfsys@useobject{currentmarker}{}%
\end{pgfscope}%
\end{pgfscope}%
\begin{pgfscope}%
\pgfsetbuttcap%
\pgfsetroundjoin%
\definecolor{currentfill}{rgb}{0.150000,0.150000,0.150000}%
\pgfsetfillcolor{currentfill}%
\pgfsetlinewidth{0.803000pt}%
\definecolor{currentstroke}{rgb}{0.150000,0.150000,0.150000}%
\pgfsetstrokecolor{currentstroke}%
\pgfsetdash{}{0pt}%
\pgfsys@defobject{currentmarker}{\pgfqpoint{0.000000in}{-0.044444in}}{\pgfqpoint{0.000000in}{0.000000in}}{%
\pgfpathmoveto{\pgfqpoint{0.000000in}{0.000000in}}%
\pgfpathlineto{\pgfqpoint{0.000000in}{-0.044444in}}%
\pgfusepath{stroke,fill}%
}%
\begin{pgfscope}%
\pgfsys@transformshift{1.180806in}{5.542197in}%
\pgfsys@useobject{currentmarker}{}%
\end{pgfscope}%
\end{pgfscope}%
\begin{pgfscope}%
\pgfsetbuttcap%
\pgfsetroundjoin%
\definecolor{currentfill}{rgb}{0.150000,0.150000,0.150000}%
\pgfsetfillcolor{currentfill}%
\pgfsetlinewidth{0.803000pt}%
\definecolor{currentstroke}{rgb}{0.150000,0.150000,0.150000}%
\pgfsetstrokecolor{currentstroke}%
\pgfsetdash{}{0pt}%
\pgfsys@defobject{currentmarker}{\pgfqpoint{0.000000in}{-0.044444in}}{\pgfqpoint{0.000000in}{0.000000in}}{%
\pgfpathmoveto{\pgfqpoint{0.000000in}{0.000000in}}%
\pgfpathlineto{\pgfqpoint{0.000000in}{-0.044444in}}%
\pgfusepath{stroke,fill}%
}%
\begin{pgfscope}%
\pgfsys@transformshift{1.354689in}{5.542197in}%
\pgfsys@useobject{currentmarker}{}%
\end{pgfscope}%
\end{pgfscope}%
\begin{pgfscope}%
\pgfsetbuttcap%
\pgfsetroundjoin%
\definecolor{currentfill}{rgb}{0.150000,0.150000,0.150000}%
\pgfsetfillcolor{currentfill}%
\pgfsetlinewidth{0.803000pt}%
\definecolor{currentstroke}{rgb}{0.150000,0.150000,0.150000}%
\pgfsetstrokecolor{currentstroke}%
\pgfsetdash{}{0pt}%
\pgfsys@defobject{currentmarker}{\pgfqpoint{0.000000in}{-0.044444in}}{\pgfqpoint{0.000000in}{0.000000in}}{%
\pgfpathmoveto{\pgfqpoint{0.000000in}{0.000000in}}%
\pgfpathlineto{\pgfqpoint{0.000000in}{-0.044444in}}%
\pgfusepath{stroke,fill}%
}%
\begin{pgfscope}%
\pgfsys@transformshift{1.442982in}{5.542197in}%
\pgfsys@useobject{currentmarker}{}%
\end{pgfscope}%
\end{pgfscope}%
\begin{pgfscope}%
\pgfsetbuttcap%
\pgfsetroundjoin%
\definecolor{currentfill}{rgb}{0.150000,0.150000,0.150000}%
\pgfsetfillcolor{currentfill}%
\pgfsetlinewidth{0.803000pt}%
\definecolor{currentstroke}{rgb}{0.150000,0.150000,0.150000}%
\pgfsetstrokecolor{currentstroke}%
\pgfsetdash{}{0pt}%
\pgfsys@defobject{currentmarker}{\pgfqpoint{0.000000in}{-0.044444in}}{\pgfqpoint{0.000000in}{0.000000in}}{%
\pgfpathmoveto{\pgfqpoint{0.000000in}{0.000000in}}%
\pgfpathlineto{\pgfqpoint{0.000000in}{-0.044444in}}%
\pgfusepath{stroke,fill}%
}%
\begin{pgfscope}%
\pgfsys@transformshift{1.505628in}{5.542197in}%
\pgfsys@useobject{currentmarker}{}%
\end{pgfscope}%
\end{pgfscope}%
\begin{pgfscope}%
\pgfsetbuttcap%
\pgfsetroundjoin%
\definecolor{currentfill}{rgb}{0.150000,0.150000,0.150000}%
\pgfsetfillcolor{currentfill}%
\pgfsetlinewidth{0.803000pt}%
\definecolor{currentstroke}{rgb}{0.150000,0.150000,0.150000}%
\pgfsetstrokecolor{currentstroke}%
\pgfsetdash{}{0pt}%
\pgfsys@defobject{currentmarker}{\pgfqpoint{0.000000in}{-0.044444in}}{\pgfqpoint{0.000000in}{0.000000in}}{%
\pgfpathmoveto{\pgfqpoint{0.000000in}{0.000000in}}%
\pgfpathlineto{\pgfqpoint{0.000000in}{-0.044444in}}%
\pgfusepath{stroke,fill}%
}%
\begin{pgfscope}%
\pgfsys@transformshift{1.554219in}{5.542197in}%
\pgfsys@useobject{currentmarker}{}%
\end{pgfscope}%
\end{pgfscope}%
\begin{pgfscope}%
\pgfsetbuttcap%
\pgfsetroundjoin%
\definecolor{currentfill}{rgb}{0.150000,0.150000,0.150000}%
\pgfsetfillcolor{currentfill}%
\pgfsetlinewidth{0.803000pt}%
\definecolor{currentstroke}{rgb}{0.150000,0.150000,0.150000}%
\pgfsetstrokecolor{currentstroke}%
\pgfsetdash{}{0pt}%
\pgfsys@defobject{currentmarker}{\pgfqpoint{0.000000in}{-0.044444in}}{\pgfqpoint{0.000000in}{0.000000in}}{%
\pgfpathmoveto{\pgfqpoint{0.000000in}{0.000000in}}%
\pgfpathlineto{\pgfqpoint{0.000000in}{-0.044444in}}%
\pgfusepath{stroke,fill}%
}%
\begin{pgfscope}%
\pgfsys@transformshift{1.593921in}{5.542197in}%
\pgfsys@useobject{currentmarker}{}%
\end{pgfscope}%
\end{pgfscope}%
\begin{pgfscope}%
\pgfsetbuttcap%
\pgfsetroundjoin%
\definecolor{currentfill}{rgb}{0.150000,0.150000,0.150000}%
\pgfsetfillcolor{currentfill}%
\pgfsetlinewidth{0.803000pt}%
\definecolor{currentstroke}{rgb}{0.150000,0.150000,0.150000}%
\pgfsetstrokecolor{currentstroke}%
\pgfsetdash{}{0pt}%
\pgfsys@defobject{currentmarker}{\pgfqpoint{0.000000in}{-0.044444in}}{\pgfqpoint{0.000000in}{0.000000in}}{%
\pgfpathmoveto{\pgfqpoint{0.000000in}{0.000000in}}%
\pgfpathlineto{\pgfqpoint{0.000000in}{-0.044444in}}%
\pgfusepath{stroke,fill}%
}%
\begin{pgfscope}%
\pgfsys@transformshift{1.627489in}{5.542197in}%
\pgfsys@useobject{currentmarker}{}%
\end{pgfscope}%
\end{pgfscope}%
\begin{pgfscope}%
\pgfsetbuttcap%
\pgfsetroundjoin%
\definecolor{currentfill}{rgb}{0.150000,0.150000,0.150000}%
\pgfsetfillcolor{currentfill}%
\pgfsetlinewidth{0.803000pt}%
\definecolor{currentstroke}{rgb}{0.150000,0.150000,0.150000}%
\pgfsetstrokecolor{currentstroke}%
\pgfsetdash{}{0pt}%
\pgfsys@defobject{currentmarker}{\pgfqpoint{0.000000in}{-0.044444in}}{\pgfqpoint{0.000000in}{0.000000in}}{%
\pgfpathmoveto{\pgfqpoint{0.000000in}{0.000000in}}%
\pgfpathlineto{\pgfqpoint{0.000000in}{-0.044444in}}%
\pgfusepath{stroke,fill}%
}%
\begin{pgfscope}%
\pgfsys@transformshift{1.656567in}{5.542197in}%
\pgfsys@useobject{currentmarker}{}%
\end{pgfscope}%
\end{pgfscope}%
\begin{pgfscope}%
\pgfsetbuttcap%
\pgfsetroundjoin%
\definecolor{currentfill}{rgb}{0.150000,0.150000,0.150000}%
\pgfsetfillcolor{currentfill}%
\pgfsetlinewidth{0.803000pt}%
\definecolor{currentstroke}{rgb}{0.150000,0.150000,0.150000}%
\pgfsetstrokecolor{currentstroke}%
\pgfsetdash{}{0pt}%
\pgfsys@defobject{currentmarker}{\pgfqpoint{0.000000in}{-0.044444in}}{\pgfqpoint{0.000000in}{0.000000in}}{%
\pgfpathmoveto{\pgfqpoint{0.000000in}{0.000000in}}%
\pgfpathlineto{\pgfqpoint{0.000000in}{-0.044444in}}%
\pgfusepath{stroke,fill}%
}%
\begin{pgfscope}%
\pgfsys@transformshift{1.682215in}{5.542197in}%
\pgfsys@useobject{currentmarker}{}%
\end{pgfscope}%
\end{pgfscope}%
\begin{pgfscope}%
\pgfsetbuttcap%
\pgfsetroundjoin%
\definecolor{currentfill}{rgb}{0.150000,0.150000,0.150000}%
\pgfsetfillcolor{currentfill}%
\pgfsetlinewidth{0.803000pt}%
\definecolor{currentstroke}{rgb}{0.150000,0.150000,0.150000}%
\pgfsetstrokecolor{currentstroke}%
\pgfsetdash{}{0pt}%
\pgfsys@defobject{currentmarker}{\pgfqpoint{0.000000in}{-0.044444in}}{\pgfqpoint{0.000000in}{0.000000in}}{%
\pgfpathmoveto{\pgfqpoint{0.000000in}{0.000000in}}%
\pgfpathlineto{\pgfqpoint{0.000000in}{-0.044444in}}%
\pgfusepath{stroke,fill}%
}%
\begin{pgfscope}%
\pgfsys@transformshift{1.856098in}{5.542197in}%
\pgfsys@useobject{currentmarker}{}%
\end{pgfscope}%
\end{pgfscope}%
\begin{pgfscope}%
\pgfsetbuttcap%
\pgfsetroundjoin%
\definecolor{currentfill}{rgb}{0.150000,0.150000,0.150000}%
\pgfsetfillcolor{currentfill}%
\pgfsetlinewidth{1.003750pt}%
\definecolor{currentstroke}{rgb}{0.150000,0.150000,0.150000}%
\pgfsetstrokecolor{currentstroke}%
\pgfsetdash{}{0pt}%
\pgfsys@defobject{currentmarker}{\pgfqpoint{-0.066667in}{0.000000in}}{\pgfqpoint{0.000000in}{0.000000in}}{%
\pgfpathmoveto{\pgfqpoint{0.000000in}{0.000000in}}%
\pgfpathlineto{\pgfqpoint{-0.066667in}{0.000000in}}%
\pgfusepath{stroke,fill}%
}%
\begin{pgfscope}%
\pgfsys@transformshift{0.702340in}{5.542197in}%
\pgfsys@useobject{currentmarker}{}%
\end{pgfscope}%
\end{pgfscope}%
\begin{pgfscope}%
\definecolor{textcolor}{rgb}{0.150000,0.150000,0.150000}%
\pgfsetstrokecolor{textcolor}%
\pgfsetfillcolor{textcolor}%
\pgftext[x=0.374971in,y=5.517249in,left,base]{\color{textcolor}\sffamily\fontsize{5.176471}{6.211765}\selectfont 16.000}%
\end{pgfscope}%
\begin{pgfscope}%
\pgfsetbuttcap%
\pgfsetroundjoin%
\definecolor{currentfill}{rgb}{0.150000,0.150000,0.150000}%
\pgfsetfillcolor{currentfill}%
\pgfsetlinewidth{1.003750pt}%
\definecolor{currentstroke}{rgb}{0.150000,0.150000,0.150000}%
\pgfsetstrokecolor{currentstroke}%
\pgfsetdash{}{0pt}%
\pgfsys@defobject{currentmarker}{\pgfqpoint{-0.066667in}{0.000000in}}{\pgfqpoint{0.000000in}{0.000000in}}{%
\pgfpathmoveto{\pgfqpoint{0.000000in}{0.000000in}}%
\pgfpathlineto{\pgfqpoint{-0.066667in}{0.000000in}}%
\pgfusepath{stroke,fill}%
}%
\begin{pgfscope}%
\pgfsys@transformshift{0.702340in}{5.616260in}%
\pgfsys@useobject{currentmarker}{}%
\end{pgfscope}%
\end{pgfscope}%
\begin{pgfscope}%
\definecolor{textcolor}{rgb}{0.150000,0.150000,0.150000}%
\pgfsetstrokecolor{textcolor}%
\pgfsetfillcolor{textcolor}%
\pgftext[x=0.374971in,y=5.591312in,left,base]{\color{textcolor}\sffamily\fontsize{5.176471}{6.211765}\selectfont 16.122}%
\end{pgfscope}%
\begin{pgfscope}%
\pgfsetbuttcap%
\pgfsetroundjoin%
\definecolor{currentfill}{rgb}{0.150000,0.150000,0.150000}%
\pgfsetfillcolor{currentfill}%
\pgfsetlinewidth{1.003750pt}%
\definecolor{currentstroke}{rgb}{0.150000,0.150000,0.150000}%
\pgfsetstrokecolor{currentstroke}%
\pgfsetdash{}{0pt}%
\pgfsys@defobject{currentmarker}{\pgfqpoint{-0.066667in}{0.000000in}}{\pgfqpoint{0.000000in}{0.000000in}}{%
\pgfpathmoveto{\pgfqpoint{0.000000in}{0.000000in}}%
\pgfpathlineto{\pgfqpoint{-0.066667in}{0.000000in}}%
\pgfusepath{stroke,fill}%
}%
\begin{pgfscope}%
\pgfsys@transformshift{0.702340in}{6.150145in}%
\pgfsys@useobject{currentmarker}{}%
\end{pgfscope}%
\end{pgfscope}%
\begin{pgfscope}%
\definecolor{textcolor}{rgb}{0.150000,0.150000,0.150000}%
\pgfsetstrokecolor{textcolor}%
\pgfsetfillcolor{textcolor}%
\pgftext[x=0.374971in,y=6.125197in,left,base]{\color{textcolor}\sffamily\fontsize{5.176471}{6.211765}\selectfont 17.000}%
\end{pgfscope}%
\begin{pgfscope}%
\definecolor{textcolor}{rgb}{0.150000,0.150000,0.150000}%
\pgfsetstrokecolor{textcolor}%
\pgfsetfillcolor{textcolor}%
\pgftext[x=0.319416in,y=5.846171in,,bottom,rotate=90.000000]{\color{textcolor}\sffamily\fontsize{5.647059}{6.776471}\selectfont \(\displaystyle x = \frac{2 \mu E L^2}{4 \pi^2}\)}%
\end{pgfscope}%
\begin{pgfscope}%
\pgfpathrectangle{\pgfqpoint{0.702340in}{5.542197in}}{\pgfqpoint{1.223103in}{0.607948in}}%
\pgfusepath{clip}%
\pgfsetroundcap%
\pgfsetroundjoin%
\pgfsetlinewidth{1.204500pt}%
\definecolor{currentstroke}{rgb}{0.000000,0.501961,0.000000}%
\pgfsetstrokecolor{currentstroke}%
\pgfsetdash{}{0pt}%
\pgfpathmoveto{\pgfqpoint{0.702340in}{5.655446in}}%
\pgfpathlineto{\pgfqpoint{0.943050in}{5.647298in}}%
\pgfpathlineto{\pgfqpoint{1.054581in}{5.634879in}}%
\pgfpathlineto{\pgfqpoint{1.127977in}{5.618305in}}%
\pgfpathlineto{\pgfqpoint{1.182772in}{5.597694in}}%
\pgfpathlineto{\pgfqpoint{1.226515in}{5.573171in}}%
\pgfpathlineto{\pgfqpoint{1.262923in}{5.544864in}}%
\pgfpathlineto{\pgfqpoint{1.268779in}{5.538863in}}%
\pgfusepath{stroke}%
\end{pgfscope}%
\begin{pgfscope}%
\pgfsetrectcap%
\pgfsetmiterjoin%
\pgfsetlinewidth{1.003750pt}%
\definecolor{currentstroke}{rgb}{0.150000,0.150000,0.150000}%
\pgfsetstrokecolor{currentstroke}%
\pgfsetdash{}{0pt}%
\pgfpathmoveto{\pgfqpoint{0.702340in}{5.542197in}}%
\pgfpathlineto{\pgfqpoint{0.702340in}{6.150145in}}%
\pgfusepath{stroke}%
\end{pgfscope}%
\begin{pgfscope}%
\pgfsetrectcap%
\pgfsetmiterjoin%
\pgfsetlinewidth{1.003750pt}%
\definecolor{currentstroke}{rgb}{0.150000,0.150000,0.150000}%
\pgfsetstrokecolor{currentstroke}%
\pgfsetdash{}{0pt}%
\pgfpathmoveto{\pgfqpoint{0.702340in}{5.542197in}}%
\pgfpathlineto{\pgfqpoint{1.925444in}{5.542197in}}%
\pgfusepath{stroke}%
\end{pgfscope}%
\begin{pgfscope}%
\pgfpathrectangle{\pgfqpoint{0.702340in}{5.542197in}}{\pgfqpoint{1.223103in}{0.607948in}}%
\pgfusepath{clip}%
\pgfsetbuttcap%
\pgfsetroundjoin%
\definecolor{currentfill}{rgb}{0.000000,0.000000,0.000000}%
\pgfsetfillcolor{currentfill}%
\pgfsetlinewidth{1.003750pt}%
\definecolor{currentstroke}{rgb}{0.000000,0.000000,0.000000}%
\pgfsetstrokecolor{currentstroke}%
\pgfsetdash{}{0pt}%
\pgfsys@defobject{currentmarker}{\pgfqpoint{-0.013889in}{-0.013889in}}{\pgfqpoint{0.013889in}{0.013889in}}{%
\pgfpathmoveto{\pgfqpoint{0.000000in}{-0.013889in}}%
\pgfpathcurveto{\pgfqpoint{0.003683in}{-0.013889in}}{\pgfqpoint{0.007216in}{-0.012425in}}{\pgfqpoint{0.009821in}{-0.009821in}}%
\pgfpathcurveto{\pgfqpoint{0.012425in}{-0.007216in}}{\pgfqpoint{0.013889in}{-0.003683in}}{\pgfqpoint{0.013889in}{0.000000in}}%
\pgfpathcurveto{\pgfqpoint{0.013889in}{0.003683in}}{\pgfqpoint{0.012425in}{0.007216in}}{\pgfqpoint{0.009821in}{0.009821in}}%
\pgfpathcurveto{\pgfqpoint{0.007216in}{0.012425in}}{\pgfqpoint{0.003683in}{0.013889in}}{\pgfqpoint{0.000000in}{0.013889in}}%
\pgfpathcurveto{\pgfqpoint{-0.003683in}{0.013889in}}{\pgfqpoint{-0.007216in}{0.012425in}}{\pgfqpoint{-0.009821in}{0.009821in}}%
\pgfpathcurveto{\pgfqpoint{-0.012425in}{0.007216in}}{\pgfqpoint{-0.013889in}{0.003683in}}{\pgfqpoint{-0.013889in}{0.000000in}}%
\pgfpathcurveto{\pgfqpoint{-0.013889in}{-0.003683in}}{\pgfqpoint{-0.012425in}{-0.007216in}}{\pgfqpoint{-0.009821in}{-0.009821in}}%
\pgfpathcurveto{\pgfqpoint{-0.007216in}{-0.012425in}}{\pgfqpoint{-0.003683in}{-0.013889in}}{\pgfqpoint{0.000000in}{-0.013889in}}%
\pgfpathclose%
\pgfusepath{stroke,fill}%
}%
\begin{pgfscope}%
\pgfsys@transformshift{1.705158in}{2.147393in}%
\pgfsys@useobject{currentmarker}{}%
\end{pgfscope}%
\begin{pgfscope}%
\pgfsys@transformshift{1.616865in}{3.854464in}%
\pgfsys@useobject{currentmarker}{}%
\end{pgfscope}%
\begin{pgfscope}%
\pgfsys@transformshift{1.554219in}{4.512667in}%
\pgfsys@useobject{currentmarker}{}%
\end{pgfscope}%
\begin{pgfscope}%
\pgfsys@transformshift{1.505628in}{4.872659in}%
\pgfsys@useobject{currentmarker}{}%
\end{pgfscope}%
\begin{pgfscope}%
\pgfsys@transformshift{1.465926in}{5.087413in}%
\pgfsys@useobject{currentmarker}{}%
\end{pgfscope}%
\begin{pgfscope}%
\pgfsys@transformshift{1.432358in}{5.222323in}%
\pgfsys@useobject{currentmarker}{}%
\end{pgfscope}%
\begin{pgfscope}%
\pgfsys@transformshift{1.403280in}{5.311795in}%
\pgfsys@useobject{currentmarker}{}%
\end{pgfscope}%
\begin{pgfscope}%
\pgfsys@transformshift{1.397903in}{5.326008in}%
\pgfsys@useobject{currentmarker}{}%
\end{pgfscope}%
\begin{pgfscope}%
\pgfsys@transformshift{1.392656in}{5.339249in}%
\pgfsys@useobject{currentmarker}{}%
\end{pgfscope}%
\begin{pgfscope}%
\pgfsys@transformshift{1.387532in}{5.351603in}%
\pgfsys@useobject{currentmarker}{}%
\end{pgfscope}%
\begin{pgfscope}%
\pgfsys@transformshift{1.382525in}{5.363148in}%
\pgfsys@useobject{currentmarker}{}%
\end{pgfscope}%
\begin{pgfscope}%
\pgfsys@transformshift{1.377632in}{5.373952in}%
\pgfsys@useobject{currentmarker}{}%
\end{pgfscope}%
\begin{pgfscope}%
\pgfsys@transformshift{1.372846in}{5.384077in}%
\pgfsys@useobject{currentmarker}{}%
\end{pgfscope}%
\begin{pgfscope}%
\pgfsys@transformshift{1.368163in}{5.393578in}%
\pgfsys@useobject{currentmarker}{}%
\end{pgfscope}%
\begin{pgfscope}%
\pgfsys@transformshift{1.363578in}{5.402505in}%
\pgfsys@useobject{currentmarker}{}%
\end{pgfscope}%
\begin{pgfscope}%
\pgfsys@transformshift{1.359088in}{5.410903in}%
\pgfsys@useobject{currentmarker}{}%
\end{pgfscope}%
\begin{pgfscope}%
\pgfsys@transformshift{1.354689in}{5.418814in}%
\pgfsys@useobject{currentmarker}{}%
\end{pgfscope}%
\end{pgfscope}%
\begin{pgfscope}%
\pgfsetbuttcap%
\pgfsetmiterjoin%
\definecolor{currentfill}{rgb}{1.000000,1.000000,1.000000}%
\pgfsetfillcolor{currentfill}%
\pgfsetlinewidth{0.000000pt}%
\definecolor{currentstroke}{rgb}{0.000000,0.000000,0.000000}%
\pgfsetstrokecolor{currentstroke}%
\pgfsetstrokeopacity{0.000000}%
\pgfsetdash{}{0pt}%
\pgfpathmoveto{\pgfqpoint{2.170064in}{5.542197in}}%
\pgfpathlineto{\pgfqpoint{3.393168in}{5.542197in}}%
\pgfpathlineto{\pgfqpoint{3.393168in}{6.150145in}}%
\pgfpathlineto{\pgfqpoint{2.170064in}{6.150145in}}%
\pgfpathclose%
\pgfusepath{fill}%
\end{pgfscope}%
\begin{pgfscope}%
\pgfpathrectangle{\pgfqpoint{2.170064in}{5.542197in}}{\pgfqpoint{1.223103in}{0.607948in}}%
\pgfusepath{clip}%
\pgfsetbuttcap%
\pgfsetmiterjoin%
\definecolor{currentfill}{rgb}{0.000000,0.000000,1.000000}%
\pgfsetfillcolor{currentfill}%
\pgfsetfillopacity{0.100000}%
\pgfsetlinewidth{0.803000pt}%
\definecolor{currentstroke}{rgb}{0.000000,0.000000,1.000000}%
\pgfsetstrokecolor{currentstroke}%
\pgfsetstrokeopacity{0.100000}%
\pgfsetdash{}{0pt}%
\pgfpathmoveto{\pgfqpoint{2.170064in}{5.438945in}}%
\pgfpathlineto{\pgfqpoint{2.170064in}{5.665876in}}%
\pgfpathlineto{\pgfqpoint{3.393168in}{5.665876in}}%
\pgfpathlineto{\pgfqpoint{3.393168in}{5.438945in}}%
\pgfpathclose%
\pgfusepath{stroke,fill}%
\end{pgfscope}%
\begin{pgfscope}%
\pgfpathrectangle{\pgfqpoint{2.170064in}{5.542197in}}{\pgfqpoint{1.223103in}{0.607948in}}%
\pgfusepath{clip}%
\pgfsetbuttcap%
\pgfsetroundjoin%
\definecolor{currentfill}{rgb}{0.000000,0.501961,0.000000}%
\pgfsetfillcolor{currentfill}%
\pgfsetfillopacity{0.500000}%
\pgfsetlinewidth{0.803000pt}%
\definecolor{currentstroke}{rgb}{0.000000,0.501961,0.000000}%
\pgfsetstrokecolor{currentstroke}%
\pgfsetstrokeopacity{0.500000}%
\pgfsetdash{}{0pt}%
\pgfpathmoveto{\pgfqpoint{2.170064in}{5.660565in}}%
\pgfpathlineto{\pgfqpoint{2.170064in}{5.452117in}}%
\pgfpathlineto{\pgfqpoint{2.410774in}{5.477237in}}%
\pgfpathlineto{\pgfqpoint{2.522305in}{5.500330in}}%
\pgfpathlineto{\pgfqpoint{2.595701in}{5.521361in}}%
\pgfpathlineto{\pgfqpoint{2.650497in}{5.540294in}}%
\pgfpathlineto{\pgfqpoint{2.694239in}{5.557094in}}%
\pgfpathlineto{\pgfqpoint{2.730647in}{5.571725in}}%
\pgfpathlineto{\pgfqpoint{2.761831in}{5.584153in}}%
\pgfpathlineto{\pgfqpoint{2.789104in}{5.594341in}}%
\pgfpathlineto{\pgfqpoint{2.813338in}{5.602252in}}%
\pgfpathlineto{\pgfqpoint{2.835143in}{5.607845in}}%
\pgfpathlineto{\pgfqpoint{2.854962in}{5.609669in}}%
\pgfpathlineto{\pgfqpoint{2.873127in}{5.605065in}}%
\pgfpathlineto{\pgfqpoint{2.889892in}{5.599420in}}%
\pgfpathlineto{\pgfqpoint{2.905459in}{5.592588in}}%
\pgfpathlineto{\pgfqpoint{2.919986in}{5.584383in}}%
\pgfpathlineto{\pgfqpoint{2.933605in}{5.574627in}}%
\pgfpathlineto{\pgfqpoint{2.946422in}{5.563151in}}%
\pgfpathlineto{\pgfqpoint{2.958526in}{5.549791in}}%
\pgfpathlineto{\pgfqpoint{2.969993in}{5.534392in}}%
\pgfpathlineto{\pgfqpoint{2.980886in}{5.516803in}}%
\pgfpathlineto{\pgfqpoint{2.991260in}{5.496875in}}%
\pgfpathlineto{\pgfqpoint{3.001162in}{5.474445in}}%
\pgfpathlineto{\pgfqpoint{3.010633in}{5.449298in}}%
\pgfpathlineto{\pgfqpoint{3.019710in}{5.421038in}}%
\pgfpathlineto{\pgfqpoint{3.028423in}{5.389093in}}%
\pgfpathlineto{\pgfqpoint{3.036801in}{5.353523in}}%
\pgfpathlineto{\pgfqpoint{3.044869in}{5.314742in}}%
\pgfpathlineto{\pgfqpoint{3.052648in}{5.272915in}}%
\pgfpathlineto{\pgfqpoint{3.060159in}{5.228081in}}%
\pgfpathlineto{\pgfqpoint{3.067420in}{5.180238in}}%
\pgfpathlineto{\pgfqpoint{3.074446in}{5.129372in}}%
\pgfpathlineto{\pgfqpoint{3.081253in}{5.075465in}}%
\pgfpathlineto{\pgfqpoint{3.087853in}{5.018498in}}%
\pgfpathlineto{\pgfqpoint{3.094259in}{4.958456in}}%
\pgfpathlineto{\pgfqpoint{3.100482in}{4.895325in}}%
\pgfpathlineto{\pgfqpoint{3.106532in}{4.829094in}}%
\pgfpathlineto{\pgfqpoint{3.112419in}{4.759758in}}%
\pgfpathlineto{\pgfqpoint{3.118150in}{4.687314in}}%
\pgfpathlineto{\pgfqpoint{3.123735in}{4.611765in}}%
\pgfpathlineto{\pgfqpoint{3.129180in}{4.533122in}}%
\pgfpathlineto{\pgfqpoint{3.134492in}{4.451398in}}%
\pgfpathlineto{\pgfqpoint{3.139677in}{4.366612in}}%
\pgfpathlineto{\pgfqpoint{3.144742in}{4.278788in}}%
\pgfpathlineto{\pgfqpoint{3.149692in}{4.187945in}}%
\pgfpathlineto{\pgfqpoint{3.154532in}{4.094086in}}%
\pgfpathlineto{\pgfqpoint{3.159267in}{3.997145in}}%
\pgfpathlineto{\pgfqpoint{3.163901in}{3.896790in}}%
\pgfpathlineto{\pgfqpoint{3.168438in}{3.791923in}}%
\pgfpathlineto{\pgfqpoint{3.172883in}{3.681600in}}%
\pgfpathlineto{\pgfqpoint{3.172883in}{3.694635in}}%
\pgfpathlineto{\pgfqpoint{3.172883in}{3.694635in}}%
\pgfpathlineto{\pgfqpoint{3.168438in}{3.800900in}}%
\pgfpathlineto{\pgfqpoint{3.163901in}{3.906273in}}%
\pgfpathlineto{\pgfqpoint{3.159267in}{4.009749in}}%
\pgfpathlineto{\pgfqpoint{3.154532in}{4.110179in}}%
\pgfpathlineto{\pgfqpoint{3.149692in}{4.207182in}}%
\pgfpathlineto{\pgfqpoint{3.144742in}{4.300655in}}%
\pgfpathlineto{\pgfqpoint{3.139677in}{4.390572in}}%
\pgfpathlineto{\pgfqpoint{3.134492in}{4.476930in}}%
\pgfpathlineto{\pgfqpoint{3.129180in}{4.559739in}}%
\pgfpathlineto{\pgfqpoint{3.123735in}{4.639013in}}%
\pgfpathlineto{\pgfqpoint{3.118150in}{4.714767in}}%
\pgfpathlineto{\pgfqpoint{3.112419in}{4.787022in}}%
\pgfpathlineto{\pgfqpoint{3.106532in}{4.855798in}}%
\pgfpathlineto{\pgfqpoint{3.100482in}{4.921122in}}%
\pgfpathlineto{\pgfqpoint{3.094259in}{4.983021in}}%
\pgfpathlineto{\pgfqpoint{3.087853in}{5.041528in}}%
\pgfpathlineto{\pgfqpoint{3.081253in}{5.096680in}}%
\pgfpathlineto{\pgfqpoint{3.074446in}{5.148522in}}%
\pgfpathlineto{\pgfqpoint{3.067420in}{5.197101in}}%
\pgfpathlineto{\pgfqpoint{3.060159in}{5.242477in}}%
\pgfpathlineto{\pgfqpoint{3.052648in}{5.284719in}}%
\pgfpathlineto{\pgfqpoint{3.044869in}{5.323919in}}%
\pgfpathlineto{\pgfqpoint{3.036801in}{5.360214in}}%
\pgfpathlineto{\pgfqpoint{3.028423in}{5.393883in}}%
\pgfpathlineto{\pgfqpoint{3.019710in}{5.425456in}}%
\pgfpathlineto{\pgfqpoint{3.010633in}{5.455118in}}%
\pgfpathlineto{\pgfqpoint{3.001162in}{5.482435in}}%
\pgfpathlineto{\pgfqpoint{2.991260in}{5.507152in}}%
\pgfpathlineto{\pgfqpoint{2.980886in}{5.529207in}}%
\pgfpathlineto{\pgfqpoint{2.969993in}{5.548598in}}%
\pgfpathlineto{\pgfqpoint{2.958526in}{5.565345in}}%
\pgfpathlineto{\pgfqpoint{2.946422in}{5.579476in}}%
\pgfpathlineto{\pgfqpoint{2.933605in}{5.591020in}}%
\pgfpathlineto{\pgfqpoint{2.919986in}{5.600013in}}%
\pgfpathlineto{\pgfqpoint{2.905459in}{5.606486in}}%
\pgfpathlineto{\pgfqpoint{2.889892in}{5.610475in}}%
\pgfpathlineto{\pgfqpoint{2.873127in}{5.612017in}}%
\pgfpathlineto{\pgfqpoint{2.854962in}{5.611195in}}%
\pgfpathlineto{\pgfqpoint{2.835143in}{5.613637in}}%
\pgfpathlineto{\pgfqpoint{2.813338in}{5.616932in}}%
\pgfpathlineto{\pgfqpoint{2.789104in}{5.619885in}}%
\pgfpathlineto{\pgfqpoint{2.761831in}{5.622723in}}%
\pgfpathlineto{\pgfqpoint{2.730647in}{5.625682in}}%
\pgfpathlineto{\pgfqpoint{2.694239in}{5.629013in}}%
\pgfpathlineto{\pgfqpoint{2.650497in}{5.632971in}}%
\pgfpathlineto{\pgfqpoint{2.595701in}{5.637825in}}%
\pgfpathlineto{\pgfqpoint{2.522305in}{5.643851in}}%
\pgfpathlineto{\pgfqpoint{2.410774in}{5.651333in}}%
\pgfpathlineto{\pgfqpoint{2.170064in}{5.660565in}}%
\pgfpathclose%
\pgfusepath{stroke,fill}%
\end{pgfscope}%
\begin{pgfscope}%
\pgfpathrectangle{\pgfqpoint{2.170064in}{5.542197in}}{\pgfqpoint{1.223103in}{0.607948in}}%
\pgfusepath{clip}%
\pgfsetroundcap%
\pgfsetroundjoin%
\pgfsetlinewidth{0.501875pt}%
\definecolor{currentstroke}{rgb}{0.000000,0.000000,1.000000}%
\pgfsetstrokecolor{currentstroke}%
\pgfsetstrokeopacity{0.800000}%
\pgfsetdash{}{0pt}%
\pgfpathmoveto{\pgfqpoint{2.170064in}{5.552411in}}%
\pgfpathlineto{\pgfqpoint{3.393168in}{5.552411in}}%
\pgfusepath{stroke}%
\end{pgfscope}%
\begin{pgfscope}%
\pgfpathrectangle{\pgfqpoint{2.170064in}{5.542197in}}{\pgfqpoint{1.223103in}{0.607948in}}%
\pgfusepath{clip}%
\pgfsetbuttcap%
\pgfsetroundjoin%
\pgfsetlinewidth{1.003750pt}%
\definecolor{currentstroke}{rgb}{0.000000,0.000000,0.000000}%
\pgfsetstrokecolor{currentstroke}%
\pgfsetdash{{3.700000pt}{1.600000pt}}{0.000000pt}%
\pgfpathmoveto{\pgfqpoint{2.170064in}{5.616260in}}%
\pgfpathlineto{\pgfqpoint{3.393168in}{5.616260in}}%
\pgfusepath{stroke}%
\end{pgfscope}%
\begin{pgfscope}%
\pgfsetroundcap%
\pgfsetroundjoin%
\pgfsetlinewidth{0.501875pt}%
\definecolor{currentstroke}{rgb}{0.000000,0.000000,1.000000}%
\pgfsetstrokecolor{currentstroke}%
\pgfsetstrokeopacity{0.800000}%
\pgfsetdash{}{0pt}%
\pgfpathmoveto{\pgfqpoint{2.970757in}{5.668915in}}%
\pgfpathquadraticcurveto{\pgfqpoint{2.907509in}{5.619242in}}{\pgfqpoint{2.844261in}{5.569569in}}%
\pgfusepath{stroke}%
\end{pgfscope}%
\begin{pgfscope}%
\pgfsetfillopacity{0.800000}%
\pgfsetstrokeopacity{0.800000}%
\definecolor{textcolor}{rgb}{0.000000,0.000000,1.000000}%
\pgfsetstrokecolor{textcolor}%
\pgfsetfillcolor{textcolor}%
\pgftext[x=2.910706in,y=5.734795in,left,base]{\color{textcolor}\sffamily\fontsize{5.647059}{6.776471}\selectfont 16.02(19)}%
\end{pgfscope}%
\begin{pgfscope}%
\pgfsetbuttcap%
\pgfsetroundjoin%
\definecolor{currentfill}{rgb}{0.150000,0.150000,0.150000}%
\pgfsetfillcolor{currentfill}%
\pgfsetlinewidth{1.003750pt}%
\definecolor{currentstroke}{rgb}{0.150000,0.150000,0.150000}%
\pgfsetstrokecolor{currentstroke}%
\pgfsetdash{}{0pt}%
\pgfsys@defobject{currentmarker}{\pgfqpoint{0.000000in}{-0.066667in}}{\pgfqpoint{0.000000in}{0.000000in}}{%
\pgfpathmoveto{\pgfqpoint{0.000000in}{0.000000in}}%
\pgfpathlineto{\pgfqpoint{0.000000in}{-0.066667in}}%
\pgfusepath{stroke,fill}%
}%
\begin{pgfscope}%
\pgfsys@transformshift{2.170064in}{5.542197in}%
\pgfsys@useobject{currentmarker}{}%
\end{pgfscope}%
\end{pgfscope}%
\begin{pgfscope}%
\pgfsetbuttcap%
\pgfsetroundjoin%
\definecolor{currentfill}{rgb}{0.150000,0.150000,0.150000}%
\pgfsetfillcolor{currentfill}%
\pgfsetlinewidth{1.003750pt}%
\definecolor{currentstroke}{rgb}{0.150000,0.150000,0.150000}%
\pgfsetstrokecolor{currentstroke}%
\pgfsetdash{}{0pt}%
\pgfsys@defobject{currentmarker}{\pgfqpoint{0.000000in}{-0.066667in}}{\pgfqpoint{0.000000in}{0.000000in}}{%
\pgfpathmoveto{\pgfqpoint{0.000000in}{0.000000in}}%
\pgfpathlineto{\pgfqpoint{0.000000in}{-0.066667in}}%
\pgfusepath{stroke,fill}%
}%
\begin{pgfscope}%
\pgfsys@transformshift{2.671473in}{5.542197in}%
\pgfsys@useobject{currentmarker}{}%
\end{pgfscope}%
\end{pgfscope}%
\begin{pgfscope}%
\pgfsetbuttcap%
\pgfsetroundjoin%
\definecolor{currentfill}{rgb}{0.150000,0.150000,0.150000}%
\pgfsetfillcolor{currentfill}%
\pgfsetlinewidth{1.003750pt}%
\definecolor{currentstroke}{rgb}{0.150000,0.150000,0.150000}%
\pgfsetstrokecolor{currentstroke}%
\pgfsetdash{}{0pt}%
\pgfsys@defobject{currentmarker}{\pgfqpoint{0.000000in}{-0.066667in}}{\pgfqpoint{0.000000in}{0.000000in}}{%
\pgfpathmoveto{\pgfqpoint{0.000000in}{0.000000in}}%
\pgfpathlineto{\pgfqpoint{0.000000in}{-0.066667in}}%
\pgfusepath{stroke,fill}%
}%
\begin{pgfscope}%
\pgfsys@transformshift{3.172883in}{5.542197in}%
\pgfsys@useobject{currentmarker}{}%
\end{pgfscope}%
\end{pgfscope}%
\begin{pgfscope}%
\pgfsetbuttcap%
\pgfsetroundjoin%
\definecolor{currentfill}{rgb}{0.150000,0.150000,0.150000}%
\pgfsetfillcolor{currentfill}%
\pgfsetlinewidth{0.803000pt}%
\definecolor{currentstroke}{rgb}{0.150000,0.150000,0.150000}%
\pgfsetstrokecolor{currentstroke}%
\pgfsetdash{}{0pt}%
\pgfsys@defobject{currentmarker}{\pgfqpoint{0.000000in}{-0.044444in}}{\pgfqpoint{0.000000in}{0.000000in}}{%
\pgfpathmoveto{\pgfqpoint{0.000000in}{0.000000in}}%
\pgfpathlineto{\pgfqpoint{0.000000in}{-0.044444in}}%
\pgfusepath{stroke,fill}%
}%
\begin{pgfscope}%
\pgfsys@transformshift{2.321004in}{5.542197in}%
\pgfsys@useobject{currentmarker}{}%
\end{pgfscope}%
\end{pgfscope}%
\begin{pgfscope}%
\pgfsetbuttcap%
\pgfsetroundjoin%
\definecolor{currentfill}{rgb}{0.150000,0.150000,0.150000}%
\pgfsetfillcolor{currentfill}%
\pgfsetlinewidth{0.803000pt}%
\definecolor{currentstroke}{rgb}{0.150000,0.150000,0.150000}%
\pgfsetstrokecolor{currentstroke}%
\pgfsetdash{}{0pt}%
\pgfsys@defobject{currentmarker}{\pgfqpoint{0.000000in}{-0.044444in}}{\pgfqpoint{0.000000in}{0.000000in}}{%
\pgfpathmoveto{\pgfqpoint{0.000000in}{0.000000in}}%
\pgfpathlineto{\pgfqpoint{0.000000in}{-0.044444in}}%
\pgfusepath{stroke,fill}%
}%
\begin{pgfscope}%
\pgfsys@transformshift{2.409297in}{5.542197in}%
\pgfsys@useobject{currentmarker}{}%
\end{pgfscope}%
\end{pgfscope}%
\begin{pgfscope}%
\pgfsetbuttcap%
\pgfsetroundjoin%
\definecolor{currentfill}{rgb}{0.150000,0.150000,0.150000}%
\pgfsetfillcolor{currentfill}%
\pgfsetlinewidth{0.803000pt}%
\definecolor{currentstroke}{rgb}{0.150000,0.150000,0.150000}%
\pgfsetstrokecolor{currentstroke}%
\pgfsetdash{}{0pt}%
\pgfsys@defobject{currentmarker}{\pgfqpoint{0.000000in}{-0.044444in}}{\pgfqpoint{0.000000in}{0.000000in}}{%
\pgfpathmoveto{\pgfqpoint{0.000000in}{0.000000in}}%
\pgfpathlineto{\pgfqpoint{0.000000in}{-0.044444in}}%
\pgfusepath{stroke,fill}%
}%
\begin{pgfscope}%
\pgfsys@transformshift{2.471943in}{5.542197in}%
\pgfsys@useobject{currentmarker}{}%
\end{pgfscope}%
\end{pgfscope}%
\begin{pgfscope}%
\pgfsetbuttcap%
\pgfsetroundjoin%
\definecolor{currentfill}{rgb}{0.150000,0.150000,0.150000}%
\pgfsetfillcolor{currentfill}%
\pgfsetlinewidth{0.803000pt}%
\definecolor{currentstroke}{rgb}{0.150000,0.150000,0.150000}%
\pgfsetstrokecolor{currentstroke}%
\pgfsetdash{}{0pt}%
\pgfsys@defobject{currentmarker}{\pgfqpoint{0.000000in}{-0.044444in}}{\pgfqpoint{0.000000in}{0.000000in}}{%
\pgfpathmoveto{\pgfqpoint{0.000000in}{0.000000in}}%
\pgfpathlineto{\pgfqpoint{0.000000in}{-0.044444in}}%
\pgfusepath{stroke,fill}%
}%
\begin{pgfscope}%
\pgfsys@transformshift{2.520534in}{5.542197in}%
\pgfsys@useobject{currentmarker}{}%
\end{pgfscope}%
\end{pgfscope}%
\begin{pgfscope}%
\pgfsetbuttcap%
\pgfsetroundjoin%
\definecolor{currentfill}{rgb}{0.150000,0.150000,0.150000}%
\pgfsetfillcolor{currentfill}%
\pgfsetlinewidth{0.803000pt}%
\definecolor{currentstroke}{rgb}{0.150000,0.150000,0.150000}%
\pgfsetstrokecolor{currentstroke}%
\pgfsetdash{}{0pt}%
\pgfsys@defobject{currentmarker}{\pgfqpoint{0.000000in}{-0.044444in}}{\pgfqpoint{0.000000in}{0.000000in}}{%
\pgfpathmoveto{\pgfqpoint{0.000000in}{0.000000in}}%
\pgfpathlineto{\pgfqpoint{0.000000in}{-0.044444in}}%
\pgfusepath{stroke,fill}%
}%
\begin{pgfscope}%
\pgfsys@transformshift{2.560237in}{5.542197in}%
\pgfsys@useobject{currentmarker}{}%
\end{pgfscope}%
\end{pgfscope}%
\begin{pgfscope}%
\pgfsetbuttcap%
\pgfsetroundjoin%
\definecolor{currentfill}{rgb}{0.150000,0.150000,0.150000}%
\pgfsetfillcolor{currentfill}%
\pgfsetlinewidth{0.803000pt}%
\definecolor{currentstroke}{rgb}{0.150000,0.150000,0.150000}%
\pgfsetstrokecolor{currentstroke}%
\pgfsetdash{}{0pt}%
\pgfsys@defobject{currentmarker}{\pgfqpoint{0.000000in}{-0.044444in}}{\pgfqpoint{0.000000in}{0.000000in}}{%
\pgfpathmoveto{\pgfqpoint{0.000000in}{0.000000in}}%
\pgfpathlineto{\pgfqpoint{0.000000in}{-0.044444in}}%
\pgfusepath{stroke,fill}%
}%
\begin{pgfscope}%
\pgfsys@transformshift{2.593804in}{5.542197in}%
\pgfsys@useobject{currentmarker}{}%
\end{pgfscope}%
\end{pgfscope}%
\begin{pgfscope}%
\pgfsetbuttcap%
\pgfsetroundjoin%
\definecolor{currentfill}{rgb}{0.150000,0.150000,0.150000}%
\pgfsetfillcolor{currentfill}%
\pgfsetlinewidth{0.803000pt}%
\definecolor{currentstroke}{rgb}{0.150000,0.150000,0.150000}%
\pgfsetstrokecolor{currentstroke}%
\pgfsetdash{}{0pt}%
\pgfsys@defobject{currentmarker}{\pgfqpoint{0.000000in}{-0.044444in}}{\pgfqpoint{0.000000in}{0.000000in}}{%
\pgfpathmoveto{\pgfqpoint{0.000000in}{0.000000in}}%
\pgfpathlineto{\pgfqpoint{0.000000in}{-0.044444in}}%
\pgfusepath{stroke,fill}%
}%
\begin{pgfscope}%
\pgfsys@transformshift{2.622882in}{5.542197in}%
\pgfsys@useobject{currentmarker}{}%
\end{pgfscope}%
\end{pgfscope}%
\begin{pgfscope}%
\pgfsetbuttcap%
\pgfsetroundjoin%
\definecolor{currentfill}{rgb}{0.150000,0.150000,0.150000}%
\pgfsetfillcolor{currentfill}%
\pgfsetlinewidth{0.803000pt}%
\definecolor{currentstroke}{rgb}{0.150000,0.150000,0.150000}%
\pgfsetstrokecolor{currentstroke}%
\pgfsetdash{}{0pt}%
\pgfsys@defobject{currentmarker}{\pgfqpoint{0.000000in}{-0.044444in}}{\pgfqpoint{0.000000in}{0.000000in}}{%
\pgfpathmoveto{\pgfqpoint{0.000000in}{0.000000in}}%
\pgfpathlineto{\pgfqpoint{0.000000in}{-0.044444in}}%
\pgfusepath{stroke,fill}%
}%
\begin{pgfscope}%
\pgfsys@transformshift{2.648530in}{5.542197in}%
\pgfsys@useobject{currentmarker}{}%
\end{pgfscope}%
\end{pgfscope}%
\begin{pgfscope}%
\pgfsetbuttcap%
\pgfsetroundjoin%
\definecolor{currentfill}{rgb}{0.150000,0.150000,0.150000}%
\pgfsetfillcolor{currentfill}%
\pgfsetlinewidth{0.803000pt}%
\definecolor{currentstroke}{rgb}{0.150000,0.150000,0.150000}%
\pgfsetstrokecolor{currentstroke}%
\pgfsetdash{}{0pt}%
\pgfsys@defobject{currentmarker}{\pgfqpoint{0.000000in}{-0.044444in}}{\pgfqpoint{0.000000in}{0.000000in}}{%
\pgfpathmoveto{\pgfqpoint{0.000000in}{0.000000in}}%
\pgfpathlineto{\pgfqpoint{0.000000in}{-0.044444in}}%
\pgfusepath{stroke,fill}%
}%
\begin{pgfscope}%
\pgfsys@transformshift{2.822413in}{5.542197in}%
\pgfsys@useobject{currentmarker}{}%
\end{pgfscope}%
\end{pgfscope}%
\begin{pgfscope}%
\pgfsetbuttcap%
\pgfsetroundjoin%
\definecolor{currentfill}{rgb}{0.150000,0.150000,0.150000}%
\pgfsetfillcolor{currentfill}%
\pgfsetlinewidth{0.803000pt}%
\definecolor{currentstroke}{rgb}{0.150000,0.150000,0.150000}%
\pgfsetstrokecolor{currentstroke}%
\pgfsetdash{}{0pt}%
\pgfsys@defobject{currentmarker}{\pgfqpoint{0.000000in}{-0.044444in}}{\pgfqpoint{0.000000in}{0.000000in}}{%
\pgfpathmoveto{\pgfqpoint{0.000000in}{0.000000in}}%
\pgfpathlineto{\pgfqpoint{0.000000in}{-0.044444in}}%
\pgfusepath{stroke,fill}%
}%
\begin{pgfscope}%
\pgfsys@transformshift{2.910706in}{5.542197in}%
\pgfsys@useobject{currentmarker}{}%
\end{pgfscope}%
\end{pgfscope}%
\begin{pgfscope}%
\pgfsetbuttcap%
\pgfsetroundjoin%
\definecolor{currentfill}{rgb}{0.150000,0.150000,0.150000}%
\pgfsetfillcolor{currentfill}%
\pgfsetlinewidth{0.803000pt}%
\definecolor{currentstroke}{rgb}{0.150000,0.150000,0.150000}%
\pgfsetstrokecolor{currentstroke}%
\pgfsetdash{}{0pt}%
\pgfsys@defobject{currentmarker}{\pgfqpoint{0.000000in}{-0.044444in}}{\pgfqpoint{0.000000in}{0.000000in}}{%
\pgfpathmoveto{\pgfqpoint{0.000000in}{0.000000in}}%
\pgfpathlineto{\pgfqpoint{0.000000in}{-0.044444in}}%
\pgfusepath{stroke,fill}%
}%
\begin{pgfscope}%
\pgfsys@transformshift{2.973352in}{5.542197in}%
\pgfsys@useobject{currentmarker}{}%
\end{pgfscope}%
\end{pgfscope}%
\begin{pgfscope}%
\pgfsetbuttcap%
\pgfsetroundjoin%
\definecolor{currentfill}{rgb}{0.150000,0.150000,0.150000}%
\pgfsetfillcolor{currentfill}%
\pgfsetlinewidth{0.803000pt}%
\definecolor{currentstroke}{rgb}{0.150000,0.150000,0.150000}%
\pgfsetstrokecolor{currentstroke}%
\pgfsetdash{}{0pt}%
\pgfsys@defobject{currentmarker}{\pgfqpoint{0.000000in}{-0.044444in}}{\pgfqpoint{0.000000in}{0.000000in}}{%
\pgfpathmoveto{\pgfqpoint{0.000000in}{0.000000in}}%
\pgfpathlineto{\pgfqpoint{0.000000in}{-0.044444in}}%
\pgfusepath{stroke,fill}%
}%
\begin{pgfscope}%
\pgfsys@transformshift{3.021943in}{5.542197in}%
\pgfsys@useobject{currentmarker}{}%
\end{pgfscope}%
\end{pgfscope}%
\begin{pgfscope}%
\pgfsetbuttcap%
\pgfsetroundjoin%
\definecolor{currentfill}{rgb}{0.150000,0.150000,0.150000}%
\pgfsetfillcolor{currentfill}%
\pgfsetlinewidth{0.803000pt}%
\definecolor{currentstroke}{rgb}{0.150000,0.150000,0.150000}%
\pgfsetstrokecolor{currentstroke}%
\pgfsetdash{}{0pt}%
\pgfsys@defobject{currentmarker}{\pgfqpoint{0.000000in}{-0.044444in}}{\pgfqpoint{0.000000in}{0.000000in}}{%
\pgfpathmoveto{\pgfqpoint{0.000000in}{0.000000in}}%
\pgfpathlineto{\pgfqpoint{0.000000in}{-0.044444in}}%
\pgfusepath{stroke,fill}%
}%
\begin{pgfscope}%
\pgfsys@transformshift{3.061646in}{5.542197in}%
\pgfsys@useobject{currentmarker}{}%
\end{pgfscope}%
\end{pgfscope}%
\begin{pgfscope}%
\pgfsetbuttcap%
\pgfsetroundjoin%
\definecolor{currentfill}{rgb}{0.150000,0.150000,0.150000}%
\pgfsetfillcolor{currentfill}%
\pgfsetlinewidth{0.803000pt}%
\definecolor{currentstroke}{rgb}{0.150000,0.150000,0.150000}%
\pgfsetstrokecolor{currentstroke}%
\pgfsetdash{}{0pt}%
\pgfsys@defobject{currentmarker}{\pgfqpoint{0.000000in}{-0.044444in}}{\pgfqpoint{0.000000in}{0.000000in}}{%
\pgfpathmoveto{\pgfqpoint{0.000000in}{0.000000in}}%
\pgfpathlineto{\pgfqpoint{0.000000in}{-0.044444in}}%
\pgfusepath{stroke,fill}%
}%
\begin{pgfscope}%
\pgfsys@transformshift{3.095213in}{5.542197in}%
\pgfsys@useobject{currentmarker}{}%
\end{pgfscope}%
\end{pgfscope}%
\begin{pgfscope}%
\pgfsetbuttcap%
\pgfsetroundjoin%
\definecolor{currentfill}{rgb}{0.150000,0.150000,0.150000}%
\pgfsetfillcolor{currentfill}%
\pgfsetlinewidth{0.803000pt}%
\definecolor{currentstroke}{rgb}{0.150000,0.150000,0.150000}%
\pgfsetstrokecolor{currentstroke}%
\pgfsetdash{}{0pt}%
\pgfsys@defobject{currentmarker}{\pgfqpoint{0.000000in}{-0.044444in}}{\pgfqpoint{0.000000in}{0.000000in}}{%
\pgfpathmoveto{\pgfqpoint{0.000000in}{0.000000in}}%
\pgfpathlineto{\pgfqpoint{0.000000in}{-0.044444in}}%
\pgfusepath{stroke,fill}%
}%
\begin{pgfscope}%
\pgfsys@transformshift{3.124291in}{5.542197in}%
\pgfsys@useobject{currentmarker}{}%
\end{pgfscope}%
\end{pgfscope}%
\begin{pgfscope}%
\pgfsetbuttcap%
\pgfsetroundjoin%
\definecolor{currentfill}{rgb}{0.150000,0.150000,0.150000}%
\pgfsetfillcolor{currentfill}%
\pgfsetlinewidth{0.803000pt}%
\definecolor{currentstroke}{rgb}{0.150000,0.150000,0.150000}%
\pgfsetstrokecolor{currentstroke}%
\pgfsetdash{}{0pt}%
\pgfsys@defobject{currentmarker}{\pgfqpoint{0.000000in}{-0.044444in}}{\pgfqpoint{0.000000in}{0.000000in}}{%
\pgfpathmoveto{\pgfqpoint{0.000000in}{0.000000in}}%
\pgfpathlineto{\pgfqpoint{0.000000in}{-0.044444in}}%
\pgfusepath{stroke,fill}%
}%
\begin{pgfscope}%
\pgfsys@transformshift{3.149939in}{5.542197in}%
\pgfsys@useobject{currentmarker}{}%
\end{pgfscope}%
\end{pgfscope}%
\begin{pgfscope}%
\pgfsetbuttcap%
\pgfsetroundjoin%
\definecolor{currentfill}{rgb}{0.150000,0.150000,0.150000}%
\pgfsetfillcolor{currentfill}%
\pgfsetlinewidth{0.803000pt}%
\definecolor{currentstroke}{rgb}{0.150000,0.150000,0.150000}%
\pgfsetstrokecolor{currentstroke}%
\pgfsetdash{}{0pt}%
\pgfsys@defobject{currentmarker}{\pgfqpoint{0.000000in}{-0.044444in}}{\pgfqpoint{0.000000in}{0.000000in}}{%
\pgfpathmoveto{\pgfqpoint{0.000000in}{0.000000in}}%
\pgfpathlineto{\pgfqpoint{0.000000in}{-0.044444in}}%
\pgfusepath{stroke,fill}%
}%
\begin{pgfscope}%
\pgfsys@transformshift{3.323822in}{5.542197in}%
\pgfsys@useobject{currentmarker}{}%
\end{pgfscope}%
\end{pgfscope}%
\begin{pgfscope}%
\pgfsetbuttcap%
\pgfsetroundjoin%
\definecolor{currentfill}{rgb}{0.150000,0.150000,0.150000}%
\pgfsetfillcolor{currentfill}%
\pgfsetlinewidth{1.003750pt}%
\definecolor{currentstroke}{rgb}{0.150000,0.150000,0.150000}%
\pgfsetstrokecolor{currentstroke}%
\pgfsetdash{}{0pt}%
\pgfsys@defobject{currentmarker}{\pgfqpoint{-0.066667in}{0.000000in}}{\pgfqpoint{0.000000in}{0.000000in}}{%
\pgfpathmoveto{\pgfqpoint{0.000000in}{0.000000in}}%
\pgfpathlineto{\pgfqpoint{-0.066667in}{0.000000in}}%
\pgfusepath{stroke,fill}%
}%
\begin{pgfscope}%
\pgfsys@transformshift{2.170064in}{5.542197in}%
\pgfsys@useobject{currentmarker}{}%
\end{pgfscope}%
\end{pgfscope}%
\begin{pgfscope}%
\pgfsetbuttcap%
\pgfsetroundjoin%
\definecolor{currentfill}{rgb}{0.150000,0.150000,0.150000}%
\pgfsetfillcolor{currentfill}%
\pgfsetlinewidth{1.003750pt}%
\definecolor{currentstroke}{rgb}{0.150000,0.150000,0.150000}%
\pgfsetstrokecolor{currentstroke}%
\pgfsetdash{}{0pt}%
\pgfsys@defobject{currentmarker}{\pgfqpoint{-0.066667in}{0.000000in}}{\pgfqpoint{0.000000in}{0.000000in}}{%
\pgfpathmoveto{\pgfqpoint{0.000000in}{0.000000in}}%
\pgfpathlineto{\pgfqpoint{-0.066667in}{0.000000in}}%
\pgfusepath{stroke,fill}%
}%
\begin{pgfscope}%
\pgfsys@transformshift{2.170064in}{5.616260in}%
\pgfsys@useobject{currentmarker}{}%
\end{pgfscope}%
\end{pgfscope}%
\begin{pgfscope}%
\pgfsetbuttcap%
\pgfsetroundjoin%
\definecolor{currentfill}{rgb}{0.150000,0.150000,0.150000}%
\pgfsetfillcolor{currentfill}%
\pgfsetlinewidth{1.003750pt}%
\definecolor{currentstroke}{rgb}{0.150000,0.150000,0.150000}%
\pgfsetstrokecolor{currentstroke}%
\pgfsetdash{}{0pt}%
\pgfsys@defobject{currentmarker}{\pgfqpoint{-0.066667in}{0.000000in}}{\pgfqpoint{0.000000in}{0.000000in}}{%
\pgfpathmoveto{\pgfqpoint{0.000000in}{0.000000in}}%
\pgfpathlineto{\pgfqpoint{-0.066667in}{0.000000in}}%
\pgfusepath{stroke,fill}%
}%
\begin{pgfscope}%
\pgfsys@transformshift{2.170064in}{6.150145in}%
\pgfsys@useobject{currentmarker}{}%
\end{pgfscope}%
\end{pgfscope}%
\begin{pgfscope}%
\pgfpathrectangle{\pgfqpoint{2.170064in}{5.542197in}}{\pgfqpoint{1.223103in}{0.607948in}}%
\pgfusepath{clip}%
\pgfsetroundcap%
\pgfsetroundjoin%
\pgfsetlinewidth{1.204500pt}%
\definecolor{currentstroke}{rgb}{0.000000,0.501961,0.000000}%
\pgfsetstrokecolor{currentstroke}%
\pgfsetdash{}{0pt}%
\pgfpathmoveto{\pgfqpoint{2.170064in}{5.556341in}}%
\pgfpathlineto{\pgfqpoint{2.410774in}{5.564285in}}%
\pgfpathlineto{\pgfqpoint{2.522305in}{5.572090in}}%
\pgfpathlineto{\pgfqpoint{2.595701in}{5.579593in}}%
\pgfpathlineto{\pgfqpoint{2.650497in}{5.586633in}}%
\pgfpathlineto{\pgfqpoint{2.694239in}{5.593053in}}%
\pgfpathlineto{\pgfqpoint{2.730647in}{5.598704in}}%
\pgfpathlineto{\pgfqpoint{2.761831in}{5.603438in}}%
\pgfpathlineto{\pgfqpoint{2.789104in}{5.607113in}}%
\pgfpathlineto{\pgfqpoint{2.813338in}{5.609592in}}%
\pgfpathlineto{\pgfqpoint{2.835143in}{5.610741in}}%
\pgfpathlineto{\pgfqpoint{2.854962in}{5.610432in}}%
\pgfpathlineto{\pgfqpoint{2.873127in}{5.608541in}}%
\pgfpathlineto{\pgfqpoint{2.889892in}{5.604947in}}%
\pgfpathlineto{\pgfqpoint{2.905459in}{5.599537in}}%
\pgfpathlineto{\pgfqpoint{2.919986in}{5.592198in}}%
\pgfpathlineto{\pgfqpoint{2.933605in}{5.582824in}}%
\pgfpathlineto{\pgfqpoint{2.946422in}{5.571313in}}%
\pgfpathlineto{\pgfqpoint{2.958526in}{5.557568in}}%
\pgfpathlineto{\pgfqpoint{2.969993in}{5.541495in}}%
\pgfpathlineto{\pgfqpoint{2.971543in}{5.538863in}}%
\pgfusepath{stroke}%
\end{pgfscope}%
\begin{pgfscope}%
\pgfsetrectcap%
\pgfsetmiterjoin%
\pgfsetlinewidth{1.003750pt}%
\definecolor{currentstroke}{rgb}{0.150000,0.150000,0.150000}%
\pgfsetstrokecolor{currentstroke}%
\pgfsetdash{}{0pt}%
\pgfpathmoveto{\pgfqpoint{2.170064in}{5.542197in}}%
\pgfpathlineto{\pgfqpoint{2.170064in}{6.150145in}}%
\pgfusepath{stroke}%
\end{pgfscope}%
\begin{pgfscope}%
\pgfsetrectcap%
\pgfsetmiterjoin%
\pgfsetlinewidth{1.003750pt}%
\definecolor{currentstroke}{rgb}{0.150000,0.150000,0.150000}%
\pgfsetstrokecolor{currentstroke}%
\pgfsetdash{}{0pt}%
\pgfpathmoveto{\pgfqpoint{2.170064in}{5.542197in}}%
\pgfpathlineto{\pgfqpoint{3.393168in}{5.542197in}}%
\pgfusepath{stroke}%
\end{pgfscope}%
\begin{pgfscope}%
\pgfpathrectangle{\pgfqpoint{2.170064in}{5.542197in}}{\pgfqpoint{1.223103in}{0.607948in}}%
\pgfusepath{clip}%
\pgfsetbuttcap%
\pgfsetroundjoin%
\definecolor{currentfill}{rgb}{0.000000,0.000000,0.000000}%
\pgfsetfillcolor{currentfill}%
\pgfsetlinewidth{1.003750pt}%
\definecolor{currentstroke}{rgb}{0.000000,0.000000,0.000000}%
\pgfsetstrokecolor{currentstroke}%
\pgfsetdash{}{0pt}%
\pgfsys@defobject{currentmarker}{\pgfqpoint{-0.013889in}{-0.013889in}}{\pgfqpoint{0.013889in}{0.013889in}}{%
\pgfpathmoveto{\pgfqpoint{0.000000in}{-0.013889in}}%
\pgfpathcurveto{\pgfqpoint{0.003683in}{-0.013889in}}{\pgfqpoint{0.007216in}{-0.012425in}}{\pgfqpoint{0.009821in}{-0.009821in}}%
\pgfpathcurveto{\pgfqpoint{0.012425in}{-0.007216in}}{\pgfqpoint{0.013889in}{-0.003683in}}{\pgfqpoint{0.013889in}{0.000000in}}%
\pgfpathcurveto{\pgfqpoint{0.013889in}{0.003683in}}{\pgfqpoint{0.012425in}{0.007216in}}{\pgfqpoint{0.009821in}{0.009821in}}%
\pgfpathcurveto{\pgfqpoint{0.007216in}{0.012425in}}{\pgfqpoint{0.003683in}{0.013889in}}{\pgfqpoint{0.000000in}{0.013889in}}%
\pgfpathcurveto{\pgfqpoint{-0.003683in}{0.013889in}}{\pgfqpoint{-0.007216in}{0.012425in}}{\pgfqpoint{-0.009821in}{0.009821in}}%
\pgfpathcurveto{\pgfqpoint{-0.012425in}{0.007216in}}{\pgfqpoint{-0.013889in}{0.003683in}}{\pgfqpoint{-0.013889in}{0.000000in}}%
\pgfpathcurveto{\pgfqpoint{-0.013889in}{-0.003683in}}{\pgfqpoint{-0.012425in}{-0.007216in}}{\pgfqpoint{-0.009821in}{-0.009821in}}%
\pgfpathcurveto{\pgfqpoint{-0.007216in}{-0.012425in}}{\pgfqpoint{-0.003683in}{-0.013889in}}{\pgfqpoint{0.000000in}{-0.013889in}}%
\pgfpathclose%
\pgfusepath{stroke,fill}%
}%
\begin{pgfscope}%
\pgfsys@transformshift{3.172883in}{3.684553in}%
\pgfsys@useobject{currentmarker}{}%
\end{pgfscope}%
\begin{pgfscope}%
\pgfsys@transformshift{3.084589in}{5.063858in}%
\pgfsys@useobject{currentmarker}{}%
\end{pgfscope}%
\begin{pgfscope}%
\pgfsys@transformshift{3.021943in}{5.417056in}%
\pgfsys@useobject{currentmarker}{}%
\end{pgfscope}%
\begin{pgfscope}%
\pgfsys@transformshift{2.973352in}{5.532901in}%
\pgfsys@useobject{currentmarker}{}%
\end{pgfscope}%
\begin{pgfscope}%
\pgfsys@transformshift{2.933650in}{5.577353in}%
\pgfsys@useobject{currentmarker}{}%
\end{pgfscope}%
\begin{pgfscope}%
\pgfsys@transformshift{2.900082in}{5.596873in}%
\pgfsys@useobject{currentmarker}{}%
\end{pgfscope}%
\begin{pgfscope}%
\pgfsys@transformshift{2.871004in}{5.606358in}%
\pgfsys@useobject{currentmarker}{}%
\end{pgfscope}%
\begin{pgfscope}%
\pgfsys@transformshift{2.865627in}{5.607612in}%
\pgfsys@useobject{currentmarker}{}%
\end{pgfscope}%
\begin{pgfscope}%
\pgfsys@transformshift{2.860380in}{5.608717in}%
\pgfsys@useobject{currentmarker}{}%
\end{pgfscope}%
\begin{pgfscope}%
\pgfsys@transformshift{2.855256in}{5.609694in}%
\pgfsys@useobject{currentmarker}{}%
\end{pgfscope}%
\begin{pgfscope}%
\pgfsys@transformshift{2.850250in}{5.610559in}%
\pgfsys@useobject{currentmarker}{}%
\end{pgfscope}%
\begin{pgfscope}%
\pgfsys@transformshift{2.845356in}{5.611327in}%
\pgfsys@useobject{currentmarker}{}%
\end{pgfscope}%
\begin{pgfscope}%
\pgfsys@transformshift{2.840570in}{5.612009in}%
\pgfsys@useobject{currentmarker}{}%
\end{pgfscope}%
\begin{pgfscope}%
\pgfsys@transformshift{2.835887in}{5.612617in}%
\pgfsys@useobject{currentmarker}{}%
\end{pgfscope}%
\begin{pgfscope}%
\pgfsys@transformshift{2.831302in}{5.613160in}%
\pgfsys@useobject{currentmarker}{}%
\end{pgfscope}%
\begin{pgfscope}%
\pgfsys@transformshift{2.826812in}{5.613644in}%
\pgfsys@useobject{currentmarker}{}%
\end{pgfscope}%
\begin{pgfscope}%
\pgfsys@transformshift{2.822413in}{5.614078in}%
\pgfsys@useobject{currentmarker}{}%
\end{pgfscope}%
\end{pgfscope}%
\begin{pgfscope}%
\pgfsetbuttcap%
\pgfsetmiterjoin%
\definecolor{currentfill}{rgb}{1.000000,1.000000,1.000000}%
\pgfsetfillcolor{currentfill}%
\pgfsetlinewidth{0.000000pt}%
\definecolor{currentstroke}{rgb}{0.000000,0.000000,0.000000}%
\pgfsetstrokecolor{currentstroke}%
\pgfsetstrokeopacity{0.000000}%
\pgfsetdash{}{0pt}%
\pgfpathmoveto{\pgfqpoint{3.637789in}{5.542197in}}%
\pgfpathlineto{\pgfqpoint{4.860892in}{5.542197in}}%
\pgfpathlineto{\pgfqpoint{4.860892in}{6.150145in}}%
\pgfpathlineto{\pgfqpoint{3.637789in}{6.150145in}}%
\pgfpathclose%
\pgfusepath{fill}%
\end{pgfscope}%
\begin{pgfscope}%
\pgfpathrectangle{\pgfqpoint{3.637789in}{5.542197in}}{\pgfqpoint{1.223103in}{0.607948in}}%
\pgfusepath{clip}%
\pgfsetbuttcap%
\pgfsetmiterjoin%
\definecolor{currentfill}{rgb}{0.000000,0.000000,1.000000}%
\pgfsetfillcolor{currentfill}%
\pgfsetfillopacity{0.100000}%
\pgfsetlinewidth{0.803000pt}%
\definecolor{currentstroke}{rgb}{0.000000,0.000000,1.000000}%
\pgfsetstrokecolor{currentstroke}%
\pgfsetstrokeopacity{0.100000}%
\pgfsetdash{}{0pt}%
\pgfpathmoveto{\pgfqpoint{3.637789in}{5.434295in}}%
\pgfpathlineto{\pgfqpoint{3.637789in}{5.717701in}}%
\pgfpathlineto{\pgfqpoint{4.860892in}{5.717701in}}%
\pgfpathlineto{\pgfqpoint{4.860892in}{5.434295in}}%
\pgfpathclose%
\pgfusepath{stroke,fill}%
\end{pgfscope}%
\begin{pgfscope}%
\pgfpathrectangle{\pgfqpoint{3.637789in}{5.542197in}}{\pgfqpoint{1.223103in}{0.607948in}}%
\pgfusepath{clip}%
\pgfsetbuttcap%
\pgfsetroundjoin%
\definecolor{currentfill}{rgb}{0.000000,0.501961,0.000000}%
\pgfsetfillcolor{currentfill}%
\pgfsetfillopacity{0.500000}%
\pgfsetlinewidth{0.803000pt}%
\definecolor{currentstroke}{rgb}{0.000000,0.501961,0.000000}%
\pgfsetstrokecolor{currentstroke}%
\pgfsetstrokeopacity{0.500000}%
\pgfsetdash{}{0pt}%
\pgfpathmoveto{\pgfqpoint{3.637789in}{5.710341in}}%
\pgfpathlineto{\pgfqpoint{3.637789in}{5.446860in}}%
\pgfpathlineto{\pgfqpoint{3.878498in}{5.470844in}}%
\pgfpathlineto{\pgfqpoint{3.990029in}{5.493044in}}%
\pgfpathlineto{\pgfqpoint{4.063425in}{5.513565in}}%
\pgfpathlineto{\pgfqpoint{4.118221in}{5.532499in}}%
\pgfpathlineto{\pgfqpoint{4.161963in}{5.549924in}}%
\pgfpathlineto{\pgfqpoint{4.198371in}{5.565912in}}%
\pgfpathlineto{\pgfqpoint{4.229555in}{5.580520in}}%
\pgfpathlineto{\pgfqpoint{4.256828in}{5.593796in}}%
\pgfpathlineto{\pgfqpoint{4.281062in}{5.605780in}}%
\pgfpathlineto{\pgfqpoint{4.302867in}{5.616496in}}%
\pgfpathlineto{\pgfqpoint{4.322686in}{5.623653in}}%
\pgfpathlineto{\pgfqpoint{4.340851in}{5.621792in}}%
\pgfpathlineto{\pgfqpoint{4.357617in}{5.620582in}}%
\pgfpathlineto{\pgfqpoint{4.373183in}{5.619993in}}%
\pgfpathlineto{\pgfqpoint{4.387711in}{5.619940in}}%
\pgfpathlineto{\pgfqpoint{4.401329in}{5.620328in}}%
\pgfpathlineto{\pgfqpoint{4.414146in}{5.621055in}}%
\pgfpathlineto{\pgfqpoint{4.426250in}{5.622010in}}%
\pgfpathlineto{\pgfqpoint{4.437717in}{5.623076in}}%
\pgfpathlineto{\pgfqpoint{4.448610in}{5.624122in}}%
\pgfpathlineto{\pgfqpoint{4.458984in}{5.625009in}}%
\pgfpathlineto{\pgfqpoint{4.468886in}{5.625575in}}%
\pgfpathlineto{\pgfqpoint{4.478357in}{5.625620in}}%
\pgfpathlineto{\pgfqpoint{4.487434in}{5.624773in}}%
\pgfpathlineto{\pgfqpoint{4.496147in}{5.621527in}}%
\pgfpathlineto{\pgfqpoint{4.504525in}{5.612833in}}%
\pgfpathlineto{\pgfqpoint{4.512593in}{5.600830in}}%
\pgfpathlineto{\pgfqpoint{4.520372in}{5.586787in}}%
\pgfpathlineto{\pgfqpoint{4.527883in}{5.570855in}}%
\pgfpathlineto{\pgfqpoint{4.535144in}{5.553024in}}%
\pgfpathlineto{\pgfqpoint{4.542170in}{5.533247in}}%
\pgfpathlineto{\pgfqpoint{4.548977in}{5.511468in}}%
\pgfpathlineto{\pgfqpoint{4.555577in}{5.487621in}}%
\pgfpathlineto{\pgfqpoint{4.561983in}{5.461640in}}%
\pgfpathlineto{\pgfqpoint{4.568206in}{5.433455in}}%
\pgfpathlineto{\pgfqpoint{4.574256in}{5.402996in}}%
\pgfpathlineto{\pgfqpoint{4.580143in}{5.370190in}}%
\pgfpathlineto{\pgfqpoint{4.585874in}{5.334962in}}%
\pgfpathlineto{\pgfqpoint{4.591459in}{5.297240in}}%
\pgfpathlineto{\pgfqpoint{4.596904in}{5.256946in}}%
\pgfpathlineto{\pgfqpoint{4.602216in}{5.214004in}}%
\pgfpathlineto{\pgfqpoint{4.607401in}{5.168337in}}%
\pgfpathlineto{\pgfqpoint{4.612466in}{5.119868in}}%
\pgfpathlineto{\pgfqpoint{4.617416in}{5.068515in}}%
\pgfpathlineto{\pgfqpoint{4.622256in}{5.014200in}}%
\pgfpathlineto{\pgfqpoint{4.626991in}{4.956839in}}%
\pgfpathlineto{\pgfqpoint{4.631625in}{4.896338in}}%
\pgfpathlineto{\pgfqpoint{4.636162in}{4.832500in}}%
\pgfpathlineto{\pgfqpoint{4.640607in}{4.750395in}}%
\pgfpathlineto{\pgfqpoint{4.640607in}{4.766104in}}%
\pgfpathlineto{\pgfqpoint{4.640607in}{4.766104in}}%
\pgfpathlineto{\pgfqpoint{4.636162in}{4.847969in}}%
\pgfpathlineto{\pgfqpoint{4.631625in}{4.936840in}}%
\pgfpathlineto{\pgfqpoint{4.626991in}{5.018295in}}%
\pgfpathlineto{\pgfqpoint{4.622256in}{5.092637in}}%
\pgfpathlineto{\pgfqpoint{4.617416in}{5.160258in}}%
\pgfpathlineto{\pgfqpoint{4.612466in}{5.221549in}}%
\pgfpathlineto{\pgfqpoint{4.607401in}{5.276892in}}%
\pgfpathlineto{\pgfqpoint{4.602216in}{5.326660in}}%
\pgfpathlineto{\pgfqpoint{4.596904in}{5.371213in}}%
\pgfpathlineto{\pgfqpoint{4.591459in}{5.410901in}}%
\pgfpathlineto{\pgfqpoint{4.585874in}{5.446062in}}%
\pgfpathlineto{\pgfqpoint{4.580143in}{5.477024in}}%
\pgfpathlineto{\pgfqpoint{4.574256in}{5.504102in}}%
\pgfpathlineto{\pgfqpoint{4.568206in}{5.527603in}}%
\pgfpathlineto{\pgfqpoint{4.561983in}{5.547819in}}%
\pgfpathlineto{\pgfqpoint{4.555577in}{5.565036in}}%
\pgfpathlineto{\pgfqpoint{4.548977in}{5.579526in}}%
\pgfpathlineto{\pgfqpoint{4.542170in}{5.591552in}}%
\pgfpathlineto{\pgfqpoint{4.535144in}{5.601369in}}%
\pgfpathlineto{\pgfqpoint{4.527883in}{5.609225in}}%
\pgfpathlineto{\pgfqpoint{4.520372in}{5.615367in}}%
\pgfpathlineto{\pgfqpoint{4.512593in}{5.620068in}}%
\pgfpathlineto{\pgfqpoint{4.504525in}{5.623747in}}%
\pgfpathlineto{\pgfqpoint{4.496147in}{5.627930in}}%
\pgfpathlineto{\pgfqpoint{4.487434in}{5.635001in}}%
\pgfpathlineto{\pgfqpoint{4.478357in}{5.642145in}}%
\pgfpathlineto{\pgfqpoint{4.468886in}{5.648072in}}%
\pgfpathlineto{\pgfqpoint{4.458984in}{5.652621in}}%
\pgfpathlineto{\pgfqpoint{4.448610in}{5.655785in}}%
\pgfpathlineto{\pgfqpoint{4.437717in}{5.657587in}}%
\pgfpathlineto{\pgfqpoint{4.426250in}{5.658057in}}%
\pgfpathlineto{\pgfqpoint{4.414146in}{5.657223in}}%
\pgfpathlineto{\pgfqpoint{4.401329in}{5.655111in}}%
\pgfpathlineto{\pgfqpoint{4.387711in}{5.651746in}}%
\pgfpathlineto{\pgfqpoint{4.373183in}{5.647144in}}%
\pgfpathlineto{\pgfqpoint{4.357617in}{5.641319in}}%
\pgfpathlineto{\pgfqpoint{4.340851in}{5.634279in}}%
\pgfpathlineto{\pgfqpoint{4.322686in}{5.626078in}}%
\pgfpathlineto{\pgfqpoint{4.302867in}{5.626456in}}%
\pgfpathlineto{\pgfqpoint{4.281062in}{5.630008in}}%
\pgfpathlineto{\pgfqpoint{4.256828in}{5.634491in}}%
\pgfpathlineto{\pgfqpoint{4.229555in}{5.639958in}}%
\pgfpathlineto{\pgfqpoint{4.198371in}{5.646469in}}%
\pgfpathlineto{\pgfqpoint{4.161963in}{5.654078in}}%
\pgfpathlineto{\pgfqpoint{4.118221in}{5.662836in}}%
\pgfpathlineto{\pgfqpoint{4.063425in}{5.672794in}}%
\pgfpathlineto{\pgfqpoint{3.990029in}{5.684000in}}%
\pgfpathlineto{\pgfqpoint{3.878498in}{5.696501in}}%
\pgfpathlineto{\pgfqpoint{3.637789in}{5.710341in}}%
\pgfpathclose%
\pgfusepath{stroke,fill}%
\end{pgfscope}%
\begin{pgfscope}%
\pgfpathrectangle{\pgfqpoint{3.637789in}{5.542197in}}{\pgfqpoint{1.223103in}{0.607948in}}%
\pgfusepath{clip}%
\pgfsetroundcap%
\pgfsetroundjoin%
\pgfsetlinewidth{0.501875pt}%
\definecolor{currentstroke}{rgb}{0.000000,0.000000,1.000000}%
\pgfsetstrokecolor{currentstroke}%
\pgfsetstrokeopacity{0.800000}%
\pgfsetdash{}{0pt}%
\pgfpathmoveto{\pgfqpoint{3.637789in}{5.575998in}}%
\pgfpathlineto{\pgfqpoint{4.860892in}{5.575998in}}%
\pgfusepath{stroke}%
\end{pgfscope}%
\begin{pgfscope}%
\pgfpathrectangle{\pgfqpoint{3.637789in}{5.542197in}}{\pgfqpoint{1.223103in}{0.607948in}}%
\pgfusepath{clip}%
\pgfsetbuttcap%
\pgfsetroundjoin%
\pgfsetlinewidth{1.003750pt}%
\definecolor{currentstroke}{rgb}{0.000000,0.000000,0.000000}%
\pgfsetstrokecolor{currentstroke}%
\pgfsetdash{{3.700000pt}{1.600000pt}}{0.000000pt}%
\pgfpathmoveto{\pgfqpoint{3.637789in}{5.616260in}}%
\pgfpathlineto{\pgfqpoint{4.860892in}{5.616260in}}%
\pgfusepath{stroke}%
\end{pgfscope}%
\begin{pgfscope}%
\pgfsetroundcap%
\pgfsetroundjoin%
\pgfsetlinewidth{0.501875pt}%
\definecolor{currentstroke}{rgb}{0.000000,0.000000,1.000000}%
\pgfsetstrokecolor{currentstroke}%
\pgfsetstrokeopacity{0.800000}%
\pgfsetdash{}{0pt}%
\pgfpathmoveto{\pgfqpoint{4.447176in}{5.711612in}}%
\pgfpathquadraticcurveto{\pgfqpoint{4.379166in}{5.652881in}}{\pgfqpoint{4.311156in}{5.594150in}}%
\pgfusepath{stroke}%
\end{pgfscope}%
\begin{pgfscope}%
\pgfsetfillopacity{0.800000}%
\pgfsetstrokeopacity{0.800000}%
\definecolor{textcolor}{rgb}{0.000000,0.000000,1.000000}%
\pgfsetstrokecolor{textcolor}%
\pgfsetfillcolor{textcolor}%
\pgftext[x=4.378430in,y=5.778496in,left,base]{\color{textcolor}\sffamily\fontsize{5.647059}{6.776471}\selectfont 16.06(23)}%
\end{pgfscope}%
\begin{pgfscope}%
\pgfsetbuttcap%
\pgfsetroundjoin%
\definecolor{currentfill}{rgb}{0.150000,0.150000,0.150000}%
\pgfsetfillcolor{currentfill}%
\pgfsetlinewidth{1.003750pt}%
\definecolor{currentstroke}{rgb}{0.150000,0.150000,0.150000}%
\pgfsetstrokecolor{currentstroke}%
\pgfsetdash{}{0pt}%
\pgfsys@defobject{currentmarker}{\pgfqpoint{0.000000in}{-0.066667in}}{\pgfqpoint{0.000000in}{0.000000in}}{%
\pgfpathmoveto{\pgfqpoint{0.000000in}{0.000000in}}%
\pgfpathlineto{\pgfqpoint{0.000000in}{-0.066667in}}%
\pgfusepath{stroke,fill}%
}%
\begin{pgfscope}%
\pgfsys@transformshift{3.637789in}{5.542197in}%
\pgfsys@useobject{currentmarker}{}%
\end{pgfscope}%
\end{pgfscope}%
\begin{pgfscope}%
\pgfsetbuttcap%
\pgfsetroundjoin%
\definecolor{currentfill}{rgb}{0.150000,0.150000,0.150000}%
\pgfsetfillcolor{currentfill}%
\pgfsetlinewidth{1.003750pt}%
\definecolor{currentstroke}{rgb}{0.150000,0.150000,0.150000}%
\pgfsetstrokecolor{currentstroke}%
\pgfsetdash{}{0pt}%
\pgfsys@defobject{currentmarker}{\pgfqpoint{0.000000in}{-0.066667in}}{\pgfqpoint{0.000000in}{0.000000in}}{%
\pgfpathmoveto{\pgfqpoint{0.000000in}{0.000000in}}%
\pgfpathlineto{\pgfqpoint{0.000000in}{-0.066667in}}%
\pgfusepath{stroke,fill}%
}%
\begin{pgfscope}%
\pgfsys@transformshift{4.139198in}{5.542197in}%
\pgfsys@useobject{currentmarker}{}%
\end{pgfscope}%
\end{pgfscope}%
\begin{pgfscope}%
\pgfsetbuttcap%
\pgfsetroundjoin%
\definecolor{currentfill}{rgb}{0.150000,0.150000,0.150000}%
\pgfsetfillcolor{currentfill}%
\pgfsetlinewidth{1.003750pt}%
\definecolor{currentstroke}{rgb}{0.150000,0.150000,0.150000}%
\pgfsetstrokecolor{currentstroke}%
\pgfsetdash{}{0pt}%
\pgfsys@defobject{currentmarker}{\pgfqpoint{0.000000in}{-0.066667in}}{\pgfqpoint{0.000000in}{0.000000in}}{%
\pgfpathmoveto{\pgfqpoint{0.000000in}{0.000000in}}%
\pgfpathlineto{\pgfqpoint{0.000000in}{-0.066667in}}%
\pgfusepath{stroke,fill}%
}%
\begin{pgfscope}%
\pgfsys@transformshift{4.640607in}{5.542197in}%
\pgfsys@useobject{currentmarker}{}%
\end{pgfscope}%
\end{pgfscope}%
\begin{pgfscope}%
\pgfsetbuttcap%
\pgfsetroundjoin%
\definecolor{currentfill}{rgb}{0.150000,0.150000,0.150000}%
\pgfsetfillcolor{currentfill}%
\pgfsetlinewidth{0.803000pt}%
\definecolor{currentstroke}{rgb}{0.150000,0.150000,0.150000}%
\pgfsetstrokecolor{currentstroke}%
\pgfsetdash{}{0pt}%
\pgfsys@defobject{currentmarker}{\pgfqpoint{0.000000in}{-0.044444in}}{\pgfqpoint{0.000000in}{0.000000in}}{%
\pgfpathmoveto{\pgfqpoint{0.000000in}{0.000000in}}%
\pgfpathlineto{\pgfqpoint{0.000000in}{-0.044444in}}%
\pgfusepath{stroke,fill}%
}%
\begin{pgfscope}%
\pgfsys@transformshift{3.788728in}{5.542197in}%
\pgfsys@useobject{currentmarker}{}%
\end{pgfscope}%
\end{pgfscope}%
\begin{pgfscope}%
\pgfsetbuttcap%
\pgfsetroundjoin%
\definecolor{currentfill}{rgb}{0.150000,0.150000,0.150000}%
\pgfsetfillcolor{currentfill}%
\pgfsetlinewidth{0.803000pt}%
\definecolor{currentstroke}{rgb}{0.150000,0.150000,0.150000}%
\pgfsetstrokecolor{currentstroke}%
\pgfsetdash{}{0pt}%
\pgfsys@defobject{currentmarker}{\pgfqpoint{0.000000in}{-0.044444in}}{\pgfqpoint{0.000000in}{0.000000in}}{%
\pgfpathmoveto{\pgfqpoint{0.000000in}{0.000000in}}%
\pgfpathlineto{\pgfqpoint{0.000000in}{-0.044444in}}%
\pgfusepath{stroke,fill}%
}%
\begin{pgfscope}%
\pgfsys@transformshift{3.877021in}{5.542197in}%
\pgfsys@useobject{currentmarker}{}%
\end{pgfscope}%
\end{pgfscope}%
\begin{pgfscope}%
\pgfsetbuttcap%
\pgfsetroundjoin%
\definecolor{currentfill}{rgb}{0.150000,0.150000,0.150000}%
\pgfsetfillcolor{currentfill}%
\pgfsetlinewidth{0.803000pt}%
\definecolor{currentstroke}{rgb}{0.150000,0.150000,0.150000}%
\pgfsetstrokecolor{currentstroke}%
\pgfsetdash{}{0pt}%
\pgfsys@defobject{currentmarker}{\pgfqpoint{0.000000in}{-0.044444in}}{\pgfqpoint{0.000000in}{0.000000in}}{%
\pgfpathmoveto{\pgfqpoint{0.000000in}{0.000000in}}%
\pgfpathlineto{\pgfqpoint{0.000000in}{-0.044444in}}%
\pgfusepath{stroke,fill}%
}%
\begin{pgfscope}%
\pgfsys@transformshift{3.939667in}{5.542197in}%
\pgfsys@useobject{currentmarker}{}%
\end{pgfscope}%
\end{pgfscope}%
\begin{pgfscope}%
\pgfsetbuttcap%
\pgfsetroundjoin%
\definecolor{currentfill}{rgb}{0.150000,0.150000,0.150000}%
\pgfsetfillcolor{currentfill}%
\pgfsetlinewidth{0.803000pt}%
\definecolor{currentstroke}{rgb}{0.150000,0.150000,0.150000}%
\pgfsetstrokecolor{currentstroke}%
\pgfsetdash{}{0pt}%
\pgfsys@defobject{currentmarker}{\pgfqpoint{0.000000in}{-0.044444in}}{\pgfqpoint{0.000000in}{0.000000in}}{%
\pgfpathmoveto{\pgfqpoint{0.000000in}{0.000000in}}%
\pgfpathlineto{\pgfqpoint{0.000000in}{-0.044444in}}%
\pgfusepath{stroke,fill}%
}%
\begin{pgfscope}%
\pgfsys@transformshift{3.988258in}{5.542197in}%
\pgfsys@useobject{currentmarker}{}%
\end{pgfscope}%
\end{pgfscope}%
\begin{pgfscope}%
\pgfsetbuttcap%
\pgfsetroundjoin%
\definecolor{currentfill}{rgb}{0.150000,0.150000,0.150000}%
\pgfsetfillcolor{currentfill}%
\pgfsetlinewidth{0.803000pt}%
\definecolor{currentstroke}{rgb}{0.150000,0.150000,0.150000}%
\pgfsetstrokecolor{currentstroke}%
\pgfsetdash{}{0pt}%
\pgfsys@defobject{currentmarker}{\pgfqpoint{0.000000in}{-0.044444in}}{\pgfqpoint{0.000000in}{0.000000in}}{%
\pgfpathmoveto{\pgfqpoint{0.000000in}{0.000000in}}%
\pgfpathlineto{\pgfqpoint{0.000000in}{-0.044444in}}%
\pgfusepath{stroke,fill}%
}%
\begin{pgfscope}%
\pgfsys@transformshift{4.027961in}{5.542197in}%
\pgfsys@useobject{currentmarker}{}%
\end{pgfscope}%
\end{pgfscope}%
\begin{pgfscope}%
\pgfsetbuttcap%
\pgfsetroundjoin%
\definecolor{currentfill}{rgb}{0.150000,0.150000,0.150000}%
\pgfsetfillcolor{currentfill}%
\pgfsetlinewidth{0.803000pt}%
\definecolor{currentstroke}{rgb}{0.150000,0.150000,0.150000}%
\pgfsetstrokecolor{currentstroke}%
\pgfsetdash{}{0pt}%
\pgfsys@defobject{currentmarker}{\pgfqpoint{0.000000in}{-0.044444in}}{\pgfqpoint{0.000000in}{0.000000in}}{%
\pgfpathmoveto{\pgfqpoint{0.000000in}{0.000000in}}%
\pgfpathlineto{\pgfqpoint{0.000000in}{-0.044444in}}%
\pgfusepath{stroke,fill}%
}%
\begin{pgfscope}%
\pgfsys@transformshift{4.061528in}{5.542197in}%
\pgfsys@useobject{currentmarker}{}%
\end{pgfscope}%
\end{pgfscope}%
\begin{pgfscope}%
\pgfsetbuttcap%
\pgfsetroundjoin%
\definecolor{currentfill}{rgb}{0.150000,0.150000,0.150000}%
\pgfsetfillcolor{currentfill}%
\pgfsetlinewidth{0.803000pt}%
\definecolor{currentstroke}{rgb}{0.150000,0.150000,0.150000}%
\pgfsetstrokecolor{currentstroke}%
\pgfsetdash{}{0pt}%
\pgfsys@defobject{currentmarker}{\pgfqpoint{0.000000in}{-0.044444in}}{\pgfqpoint{0.000000in}{0.000000in}}{%
\pgfpathmoveto{\pgfqpoint{0.000000in}{0.000000in}}%
\pgfpathlineto{\pgfqpoint{0.000000in}{-0.044444in}}%
\pgfusepath{stroke,fill}%
}%
\begin{pgfscope}%
\pgfsys@transformshift{4.090606in}{5.542197in}%
\pgfsys@useobject{currentmarker}{}%
\end{pgfscope}%
\end{pgfscope}%
\begin{pgfscope}%
\pgfsetbuttcap%
\pgfsetroundjoin%
\definecolor{currentfill}{rgb}{0.150000,0.150000,0.150000}%
\pgfsetfillcolor{currentfill}%
\pgfsetlinewidth{0.803000pt}%
\definecolor{currentstroke}{rgb}{0.150000,0.150000,0.150000}%
\pgfsetstrokecolor{currentstroke}%
\pgfsetdash{}{0pt}%
\pgfsys@defobject{currentmarker}{\pgfqpoint{0.000000in}{-0.044444in}}{\pgfqpoint{0.000000in}{0.000000in}}{%
\pgfpathmoveto{\pgfqpoint{0.000000in}{0.000000in}}%
\pgfpathlineto{\pgfqpoint{0.000000in}{-0.044444in}}%
\pgfusepath{stroke,fill}%
}%
\begin{pgfscope}%
\pgfsys@transformshift{4.116254in}{5.542197in}%
\pgfsys@useobject{currentmarker}{}%
\end{pgfscope}%
\end{pgfscope}%
\begin{pgfscope}%
\pgfsetbuttcap%
\pgfsetroundjoin%
\definecolor{currentfill}{rgb}{0.150000,0.150000,0.150000}%
\pgfsetfillcolor{currentfill}%
\pgfsetlinewidth{0.803000pt}%
\definecolor{currentstroke}{rgb}{0.150000,0.150000,0.150000}%
\pgfsetstrokecolor{currentstroke}%
\pgfsetdash{}{0pt}%
\pgfsys@defobject{currentmarker}{\pgfqpoint{0.000000in}{-0.044444in}}{\pgfqpoint{0.000000in}{0.000000in}}{%
\pgfpathmoveto{\pgfqpoint{0.000000in}{0.000000in}}%
\pgfpathlineto{\pgfqpoint{0.000000in}{-0.044444in}}%
\pgfusepath{stroke,fill}%
}%
\begin{pgfscope}%
\pgfsys@transformshift{4.290137in}{5.542197in}%
\pgfsys@useobject{currentmarker}{}%
\end{pgfscope}%
\end{pgfscope}%
\begin{pgfscope}%
\pgfsetbuttcap%
\pgfsetroundjoin%
\definecolor{currentfill}{rgb}{0.150000,0.150000,0.150000}%
\pgfsetfillcolor{currentfill}%
\pgfsetlinewidth{0.803000pt}%
\definecolor{currentstroke}{rgb}{0.150000,0.150000,0.150000}%
\pgfsetstrokecolor{currentstroke}%
\pgfsetdash{}{0pt}%
\pgfsys@defobject{currentmarker}{\pgfqpoint{0.000000in}{-0.044444in}}{\pgfqpoint{0.000000in}{0.000000in}}{%
\pgfpathmoveto{\pgfqpoint{0.000000in}{0.000000in}}%
\pgfpathlineto{\pgfqpoint{0.000000in}{-0.044444in}}%
\pgfusepath{stroke,fill}%
}%
\begin{pgfscope}%
\pgfsys@transformshift{4.378430in}{5.542197in}%
\pgfsys@useobject{currentmarker}{}%
\end{pgfscope}%
\end{pgfscope}%
\begin{pgfscope}%
\pgfsetbuttcap%
\pgfsetroundjoin%
\definecolor{currentfill}{rgb}{0.150000,0.150000,0.150000}%
\pgfsetfillcolor{currentfill}%
\pgfsetlinewidth{0.803000pt}%
\definecolor{currentstroke}{rgb}{0.150000,0.150000,0.150000}%
\pgfsetstrokecolor{currentstroke}%
\pgfsetdash{}{0pt}%
\pgfsys@defobject{currentmarker}{\pgfqpoint{0.000000in}{-0.044444in}}{\pgfqpoint{0.000000in}{0.000000in}}{%
\pgfpathmoveto{\pgfqpoint{0.000000in}{0.000000in}}%
\pgfpathlineto{\pgfqpoint{0.000000in}{-0.044444in}}%
\pgfusepath{stroke,fill}%
}%
\begin{pgfscope}%
\pgfsys@transformshift{4.441076in}{5.542197in}%
\pgfsys@useobject{currentmarker}{}%
\end{pgfscope}%
\end{pgfscope}%
\begin{pgfscope}%
\pgfsetbuttcap%
\pgfsetroundjoin%
\definecolor{currentfill}{rgb}{0.150000,0.150000,0.150000}%
\pgfsetfillcolor{currentfill}%
\pgfsetlinewidth{0.803000pt}%
\definecolor{currentstroke}{rgb}{0.150000,0.150000,0.150000}%
\pgfsetstrokecolor{currentstroke}%
\pgfsetdash{}{0pt}%
\pgfsys@defobject{currentmarker}{\pgfqpoint{0.000000in}{-0.044444in}}{\pgfqpoint{0.000000in}{0.000000in}}{%
\pgfpathmoveto{\pgfqpoint{0.000000in}{0.000000in}}%
\pgfpathlineto{\pgfqpoint{0.000000in}{-0.044444in}}%
\pgfusepath{stroke,fill}%
}%
\begin{pgfscope}%
\pgfsys@transformshift{4.489667in}{5.542197in}%
\pgfsys@useobject{currentmarker}{}%
\end{pgfscope}%
\end{pgfscope}%
\begin{pgfscope}%
\pgfsetbuttcap%
\pgfsetroundjoin%
\definecolor{currentfill}{rgb}{0.150000,0.150000,0.150000}%
\pgfsetfillcolor{currentfill}%
\pgfsetlinewidth{0.803000pt}%
\definecolor{currentstroke}{rgb}{0.150000,0.150000,0.150000}%
\pgfsetstrokecolor{currentstroke}%
\pgfsetdash{}{0pt}%
\pgfsys@defobject{currentmarker}{\pgfqpoint{0.000000in}{-0.044444in}}{\pgfqpoint{0.000000in}{0.000000in}}{%
\pgfpathmoveto{\pgfqpoint{0.000000in}{0.000000in}}%
\pgfpathlineto{\pgfqpoint{0.000000in}{-0.044444in}}%
\pgfusepath{stroke,fill}%
}%
\begin{pgfscope}%
\pgfsys@transformshift{4.529370in}{5.542197in}%
\pgfsys@useobject{currentmarker}{}%
\end{pgfscope}%
\end{pgfscope}%
\begin{pgfscope}%
\pgfsetbuttcap%
\pgfsetroundjoin%
\definecolor{currentfill}{rgb}{0.150000,0.150000,0.150000}%
\pgfsetfillcolor{currentfill}%
\pgfsetlinewidth{0.803000pt}%
\definecolor{currentstroke}{rgb}{0.150000,0.150000,0.150000}%
\pgfsetstrokecolor{currentstroke}%
\pgfsetdash{}{0pt}%
\pgfsys@defobject{currentmarker}{\pgfqpoint{0.000000in}{-0.044444in}}{\pgfqpoint{0.000000in}{0.000000in}}{%
\pgfpathmoveto{\pgfqpoint{0.000000in}{0.000000in}}%
\pgfpathlineto{\pgfqpoint{0.000000in}{-0.044444in}}%
\pgfusepath{stroke,fill}%
}%
\begin{pgfscope}%
\pgfsys@transformshift{4.562937in}{5.542197in}%
\pgfsys@useobject{currentmarker}{}%
\end{pgfscope}%
\end{pgfscope}%
\begin{pgfscope}%
\pgfsetbuttcap%
\pgfsetroundjoin%
\definecolor{currentfill}{rgb}{0.150000,0.150000,0.150000}%
\pgfsetfillcolor{currentfill}%
\pgfsetlinewidth{0.803000pt}%
\definecolor{currentstroke}{rgb}{0.150000,0.150000,0.150000}%
\pgfsetstrokecolor{currentstroke}%
\pgfsetdash{}{0pt}%
\pgfsys@defobject{currentmarker}{\pgfqpoint{0.000000in}{-0.044444in}}{\pgfqpoint{0.000000in}{0.000000in}}{%
\pgfpathmoveto{\pgfqpoint{0.000000in}{0.000000in}}%
\pgfpathlineto{\pgfqpoint{0.000000in}{-0.044444in}}%
\pgfusepath{stroke,fill}%
}%
\begin{pgfscope}%
\pgfsys@transformshift{4.592015in}{5.542197in}%
\pgfsys@useobject{currentmarker}{}%
\end{pgfscope}%
\end{pgfscope}%
\begin{pgfscope}%
\pgfsetbuttcap%
\pgfsetroundjoin%
\definecolor{currentfill}{rgb}{0.150000,0.150000,0.150000}%
\pgfsetfillcolor{currentfill}%
\pgfsetlinewidth{0.803000pt}%
\definecolor{currentstroke}{rgb}{0.150000,0.150000,0.150000}%
\pgfsetstrokecolor{currentstroke}%
\pgfsetdash{}{0pt}%
\pgfsys@defobject{currentmarker}{\pgfqpoint{0.000000in}{-0.044444in}}{\pgfqpoint{0.000000in}{0.000000in}}{%
\pgfpathmoveto{\pgfqpoint{0.000000in}{0.000000in}}%
\pgfpathlineto{\pgfqpoint{0.000000in}{-0.044444in}}%
\pgfusepath{stroke,fill}%
}%
\begin{pgfscope}%
\pgfsys@transformshift{4.617663in}{5.542197in}%
\pgfsys@useobject{currentmarker}{}%
\end{pgfscope}%
\end{pgfscope}%
\begin{pgfscope}%
\pgfsetbuttcap%
\pgfsetroundjoin%
\definecolor{currentfill}{rgb}{0.150000,0.150000,0.150000}%
\pgfsetfillcolor{currentfill}%
\pgfsetlinewidth{0.803000pt}%
\definecolor{currentstroke}{rgb}{0.150000,0.150000,0.150000}%
\pgfsetstrokecolor{currentstroke}%
\pgfsetdash{}{0pt}%
\pgfsys@defobject{currentmarker}{\pgfqpoint{0.000000in}{-0.044444in}}{\pgfqpoint{0.000000in}{0.000000in}}{%
\pgfpathmoveto{\pgfqpoint{0.000000in}{0.000000in}}%
\pgfpathlineto{\pgfqpoint{0.000000in}{-0.044444in}}%
\pgfusepath{stroke,fill}%
}%
\begin{pgfscope}%
\pgfsys@transformshift{4.791546in}{5.542197in}%
\pgfsys@useobject{currentmarker}{}%
\end{pgfscope}%
\end{pgfscope}%
\begin{pgfscope}%
\pgfsetbuttcap%
\pgfsetroundjoin%
\definecolor{currentfill}{rgb}{0.150000,0.150000,0.150000}%
\pgfsetfillcolor{currentfill}%
\pgfsetlinewidth{1.003750pt}%
\definecolor{currentstroke}{rgb}{0.150000,0.150000,0.150000}%
\pgfsetstrokecolor{currentstroke}%
\pgfsetdash{}{0pt}%
\pgfsys@defobject{currentmarker}{\pgfqpoint{-0.066667in}{0.000000in}}{\pgfqpoint{0.000000in}{0.000000in}}{%
\pgfpathmoveto{\pgfqpoint{0.000000in}{0.000000in}}%
\pgfpathlineto{\pgfqpoint{-0.066667in}{0.000000in}}%
\pgfusepath{stroke,fill}%
}%
\begin{pgfscope}%
\pgfsys@transformshift{3.637789in}{5.542197in}%
\pgfsys@useobject{currentmarker}{}%
\end{pgfscope}%
\end{pgfscope}%
\begin{pgfscope}%
\pgfsetbuttcap%
\pgfsetroundjoin%
\definecolor{currentfill}{rgb}{0.150000,0.150000,0.150000}%
\pgfsetfillcolor{currentfill}%
\pgfsetlinewidth{1.003750pt}%
\definecolor{currentstroke}{rgb}{0.150000,0.150000,0.150000}%
\pgfsetstrokecolor{currentstroke}%
\pgfsetdash{}{0pt}%
\pgfsys@defobject{currentmarker}{\pgfqpoint{-0.066667in}{0.000000in}}{\pgfqpoint{0.000000in}{0.000000in}}{%
\pgfpathmoveto{\pgfqpoint{0.000000in}{0.000000in}}%
\pgfpathlineto{\pgfqpoint{-0.066667in}{0.000000in}}%
\pgfusepath{stroke,fill}%
}%
\begin{pgfscope}%
\pgfsys@transformshift{3.637789in}{5.616260in}%
\pgfsys@useobject{currentmarker}{}%
\end{pgfscope}%
\end{pgfscope}%
\begin{pgfscope}%
\pgfsetbuttcap%
\pgfsetroundjoin%
\definecolor{currentfill}{rgb}{0.150000,0.150000,0.150000}%
\pgfsetfillcolor{currentfill}%
\pgfsetlinewidth{1.003750pt}%
\definecolor{currentstroke}{rgb}{0.150000,0.150000,0.150000}%
\pgfsetstrokecolor{currentstroke}%
\pgfsetdash{}{0pt}%
\pgfsys@defobject{currentmarker}{\pgfqpoint{-0.066667in}{0.000000in}}{\pgfqpoint{0.000000in}{0.000000in}}{%
\pgfpathmoveto{\pgfqpoint{0.000000in}{0.000000in}}%
\pgfpathlineto{\pgfqpoint{-0.066667in}{0.000000in}}%
\pgfusepath{stroke,fill}%
}%
\begin{pgfscope}%
\pgfsys@transformshift{3.637789in}{6.150145in}%
\pgfsys@useobject{currentmarker}{}%
\end{pgfscope}%
\end{pgfscope}%
\begin{pgfscope}%
\pgfpathrectangle{\pgfqpoint{3.637789in}{5.542197in}}{\pgfqpoint{1.223103in}{0.607948in}}%
\pgfusepath{clip}%
\pgfsetroundcap%
\pgfsetroundjoin%
\pgfsetlinewidth{1.204500pt}%
\definecolor{currentstroke}{rgb}{0.000000,0.501961,0.000000}%
\pgfsetstrokecolor{currentstroke}%
\pgfsetdash{}{0pt}%
\pgfpathmoveto{\pgfqpoint{3.637789in}{5.578601in}}%
\pgfpathlineto{\pgfqpoint{3.878498in}{5.583672in}}%
\pgfpathlineto{\pgfqpoint{3.990029in}{5.588522in}}%
\pgfpathlineto{\pgfqpoint{4.063425in}{5.593180in}}%
\pgfpathlineto{\pgfqpoint{4.118221in}{5.597667in}}%
\pgfpathlineto{\pgfqpoint{4.161963in}{5.602001in}}%
\pgfpathlineto{\pgfqpoint{4.198371in}{5.606190in}}%
\pgfpathlineto{\pgfqpoint{4.229555in}{5.610239in}}%
\pgfpathlineto{\pgfqpoint{4.256828in}{5.614143in}}%
\pgfpathlineto{\pgfqpoint{4.281062in}{5.617894in}}%
\pgfpathlineto{\pgfqpoint{4.302867in}{5.621476in}}%
\pgfpathlineto{\pgfqpoint{4.322686in}{5.624866in}}%
\pgfpathlineto{\pgfqpoint{4.340851in}{5.628035in}}%
\pgfpathlineto{\pgfqpoint{4.357617in}{5.630950in}}%
\pgfpathlineto{\pgfqpoint{4.373183in}{5.633569in}}%
\pgfpathlineto{\pgfqpoint{4.387711in}{5.635843in}}%
\pgfpathlineto{\pgfqpoint{4.401329in}{5.637720in}}%
\pgfpathlineto{\pgfqpoint{4.414146in}{5.639139in}}%
\pgfpathlineto{\pgfqpoint{4.426250in}{5.640034in}}%
\pgfpathlineto{\pgfqpoint{4.437717in}{5.640331in}}%
\pgfpathlineto{\pgfqpoint{4.448610in}{5.639954in}}%
\pgfpathlineto{\pgfqpoint{4.458984in}{5.638815in}}%
\pgfpathlineto{\pgfqpoint{4.468886in}{5.636823in}}%
\pgfpathlineto{\pgfqpoint{4.478357in}{5.633882in}}%
\pgfpathlineto{\pgfqpoint{4.487434in}{5.629887in}}%
\pgfpathlineto{\pgfqpoint{4.496147in}{5.624729in}}%
\pgfpathlineto{\pgfqpoint{4.504525in}{5.618290in}}%
\pgfpathlineto{\pgfqpoint{4.512593in}{5.610449in}}%
\pgfpathlineto{\pgfqpoint{4.520372in}{5.601077in}}%
\pgfpathlineto{\pgfqpoint{4.527883in}{5.590040in}}%
\pgfpathlineto{\pgfqpoint{4.535144in}{5.577196in}}%
\pgfpathlineto{\pgfqpoint{4.542170in}{5.562400in}}%
\pgfpathlineto{\pgfqpoint{4.548977in}{5.545497in}}%
\pgfpathlineto{\pgfqpoint{4.551261in}{5.538863in}}%
\pgfusepath{stroke}%
\end{pgfscope}%
\begin{pgfscope}%
\pgfsetrectcap%
\pgfsetmiterjoin%
\pgfsetlinewidth{1.003750pt}%
\definecolor{currentstroke}{rgb}{0.150000,0.150000,0.150000}%
\pgfsetstrokecolor{currentstroke}%
\pgfsetdash{}{0pt}%
\pgfpathmoveto{\pgfqpoint{3.637789in}{5.542197in}}%
\pgfpathlineto{\pgfqpoint{3.637789in}{6.150145in}}%
\pgfusepath{stroke}%
\end{pgfscope}%
\begin{pgfscope}%
\pgfsetrectcap%
\pgfsetmiterjoin%
\pgfsetlinewidth{1.003750pt}%
\definecolor{currentstroke}{rgb}{0.150000,0.150000,0.150000}%
\pgfsetstrokecolor{currentstroke}%
\pgfsetdash{}{0pt}%
\pgfpathmoveto{\pgfqpoint{3.637789in}{5.542197in}}%
\pgfpathlineto{\pgfqpoint{4.860892in}{5.542197in}}%
\pgfusepath{stroke}%
\end{pgfscope}%
\begin{pgfscope}%
\pgfpathrectangle{\pgfqpoint{3.637789in}{5.542197in}}{\pgfqpoint{1.223103in}{0.607948in}}%
\pgfusepath{clip}%
\pgfsetbuttcap%
\pgfsetroundjoin%
\definecolor{currentfill}{rgb}{0.000000,0.000000,0.000000}%
\pgfsetfillcolor{currentfill}%
\pgfsetlinewidth{1.003750pt}%
\definecolor{currentstroke}{rgb}{0.000000,0.000000,0.000000}%
\pgfsetstrokecolor{currentstroke}%
\pgfsetdash{}{0pt}%
\pgfsys@defobject{currentmarker}{\pgfqpoint{-0.013889in}{-0.013889in}}{\pgfqpoint{0.013889in}{0.013889in}}{%
\pgfpathmoveto{\pgfqpoint{0.000000in}{-0.013889in}}%
\pgfpathcurveto{\pgfqpoint{0.003683in}{-0.013889in}}{\pgfqpoint{0.007216in}{-0.012425in}}{\pgfqpoint{0.009821in}{-0.009821in}}%
\pgfpathcurveto{\pgfqpoint{0.012425in}{-0.007216in}}{\pgfqpoint{0.013889in}{-0.003683in}}{\pgfqpoint{0.013889in}{0.000000in}}%
\pgfpathcurveto{\pgfqpoint{0.013889in}{0.003683in}}{\pgfqpoint{0.012425in}{0.007216in}}{\pgfqpoint{0.009821in}{0.009821in}}%
\pgfpathcurveto{\pgfqpoint{0.007216in}{0.012425in}}{\pgfqpoint{0.003683in}{0.013889in}}{\pgfqpoint{0.000000in}{0.013889in}}%
\pgfpathcurveto{\pgfqpoint{-0.003683in}{0.013889in}}{\pgfqpoint{-0.007216in}{0.012425in}}{\pgfqpoint{-0.009821in}{0.009821in}}%
\pgfpathcurveto{\pgfqpoint{-0.012425in}{0.007216in}}{\pgfqpoint{-0.013889in}{0.003683in}}{\pgfqpoint{-0.013889in}{0.000000in}}%
\pgfpathcurveto{\pgfqpoint{-0.013889in}{-0.003683in}}{\pgfqpoint{-0.012425in}{-0.007216in}}{\pgfqpoint{-0.009821in}{-0.009821in}}%
\pgfpathcurveto{\pgfqpoint{-0.007216in}{-0.012425in}}{\pgfqpoint{-0.003683in}{-0.013889in}}{\pgfqpoint{0.000000in}{-0.013889in}}%
\pgfpathclose%
\pgfusepath{stroke,fill}%
}%
\begin{pgfscope}%
\pgfsys@transformshift{4.640607in}{4.753556in}%
\pgfsys@useobject{currentmarker}{}%
\end{pgfscope}%
\begin{pgfscope}%
\pgfsys@transformshift{4.552313in}{5.558981in}%
\pgfsys@useobject{currentmarker}{}%
\end{pgfscope}%
\begin{pgfscope}%
\pgfsys@transformshift{4.290137in}{5.623491in}%
\pgfsys@useobject{currentmarker}{}%
\end{pgfscope}%
\begin{pgfscope}%
\pgfsys@transformshift{4.294536in}{5.623647in}%
\pgfsys@useobject{currentmarker}{}%
\end{pgfscope}%
\begin{pgfscope}%
\pgfsys@transformshift{4.299026in}{5.623809in}%
\pgfsys@useobject{currentmarker}{}%
\end{pgfscope}%
\begin{pgfscope}%
\pgfsys@transformshift{4.303611in}{5.623979in}%
\pgfsys@useobject{currentmarker}{}%
\end{pgfscope}%
\begin{pgfscope}%
\pgfsys@transformshift{4.308294in}{5.624156in}%
\pgfsys@useobject{currentmarker}{}%
\end{pgfscope}%
\begin{pgfscope}%
\pgfsys@transformshift{4.313080in}{5.624340in}%
\pgfsys@useobject{currentmarker}{}%
\end{pgfscope}%
\begin{pgfscope}%
\pgfsys@transformshift{4.489667in}{5.624392in}%
\pgfsys@useobject{currentmarker}{}%
\end{pgfscope}%
\begin{pgfscope}%
\pgfsys@transformshift{4.317974in}{5.624533in}%
\pgfsys@useobject{currentmarker}{}%
\end{pgfscope}%
\begin{pgfscope}%
\pgfsys@transformshift{4.322980in}{5.624735in}%
\pgfsys@useobject{currentmarker}{}%
\end{pgfscope}%
\begin{pgfscope}%
\pgfsys@transformshift{4.328104in}{5.624946in}%
\pgfsys@useobject{currentmarker}{}%
\end{pgfscope}%
\begin{pgfscope}%
\pgfsys@transformshift{4.333351in}{5.625167in}%
\pgfsys@useobject{currentmarker}{}%
\end{pgfscope}%
\begin{pgfscope}%
\pgfsys@transformshift{4.338728in}{5.625398in}%
\pgfsys@useobject{currentmarker}{}%
\end{pgfscope}%
\begin{pgfscope}%
\pgfsys@transformshift{4.367806in}{5.626718in}%
\pgfsys@useobject{currentmarker}{}%
\end{pgfscope}%
\begin{pgfscope}%
\pgfsys@transformshift{4.401374in}{5.628273in}%
\pgfsys@useobject{currentmarker}{}%
\end{pgfscope}%
\begin{pgfscope}%
\pgfsys@transformshift{4.441076in}{5.629388in}%
\pgfsys@useobject{currentmarker}{}%
\end{pgfscope}%
\end{pgfscope}%
\begin{pgfscope}%
\pgfsetbuttcap%
\pgfsetmiterjoin%
\definecolor{currentfill}{rgb}{1.000000,1.000000,1.000000}%
\pgfsetfillcolor{currentfill}%
\pgfsetlinewidth{0.000000pt}%
\definecolor{currentstroke}{rgb}{0.000000,0.000000,0.000000}%
\pgfsetstrokecolor{currentstroke}%
\pgfsetstrokeopacity{0.000000}%
\pgfsetdash{}{0pt}%
\pgfpathmoveto{\pgfqpoint{5.105513in}{5.542197in}}%
\pgfpathlineto{\pgfqpoint{6.328616in}{5.542197in}}%
\pgfpathlineto{\pgfqpoint{6.328616in}{6.150145in}}%
\pgfpathlineto{\pgfqpoint{5.105513in}{6.150145in}}%
\pgfpathclose%
\pgfusepath{fill}%
\end{pgfscope}%
\begin{pgfscope}%
\pgfpathrectangle{\pgfqpoint{5.105513in}{5.542197in}}{\pgfqpoint{1.223103in}{0.607948in}}%
\pgfusepath{clip}%
\pgfsetbuttcap%
\pgfsetmiterjoin%
\definecolor{currentfill}{rgb}{0.000000,0.000000,1.000000}%
\pgfsetfillcolor{currentfill}%
\pgfsetfillopacity{0.100000}%
\pgfsetlinewidth{0.803000pt}%
\definecolor{currentstroke}{rgb}{0.000000,0.000000,1.000000}%
\pgfsetstrokecolor{currentstroke}%
\pgfsetstrokeopacity{0.100000}%
\pgfsetdash{}{0pt}%
\pgfpathmoveto{\pgfqpoint{5.105513in}{5.613163in}}%
\pgfpathlineto{\pgfqpoint{5.105513in}{5.620589in}}%
\pgfpathlineto{\pgfqpoint{6.328616in}{5.620589in}}%
\pgfpathlineto{\pgfqpoint{6.328616in}{5.613163in}}%
\pgfpathclose%
\pgfusepath{stroke,fill}%
\end{pgfscope}%
\begin{pgfscope}%
\pgfpathrectangle{\pgfqpoint{5.105513in}{5.542197in}}{\pgfqpoint{1.223103in}{0.607948in}}%
\pgfusepath{clip}%
\pgfsetbuttcap%
\pgfsetroundjoin%
\definecolor{currentfill}{rgb}{0.000000,0.501961,0.000000}%
\pgfsetfillcolor{currentfill}%
\pgfsetfillopacity{0.500000}%
\pgfsetlinewidth{0.803000pt}%
\definecolor{currentstroke}{rgb}{0.000000,0.501961,0.000000}%
\pgfsetstrokecolor{currentstroke}%
\pgfsetstrokeopacity{0.500000}%
\pgfsetdash{}{0pt}%
\pgfpathmoveto{\pgfqpoint{5.105513in}{5.620695in}}%
\pgfpathlineto{\pgfqpoint{5.105513in}{5.613873in}}%
\pgfpathlineto{\pgfqpoint{5.346222in}{5.615247in}}%
\pgfpathlineto{\pgfqpoint{5.457753in}{5.616552in}}%
\pgfpathlineto{\pgfqpoint{5.531149in}{5.617799in}}%
\pgfpathlineto{\pgfqpoint{5.585945in}{5.619001in}}%
\pgfpathlineto{\pgfqpoint{5.629687in}{5.620167in}}%
\pgfpathlineto{\pgfqpoint{5.666095in}{5.621306in}}%
\pgfpathlineto{\pgfqpoint{5.697279in}{5.622425in}}%
\pgfpathlineto{\pgfqpoint{5.724552in}{5.623532in}}%
\pgfpathlineto{\pgfqpoint{5.748786in}{5.624631in}}%
\pgfpathlineto{\pgfqpoint{5.770591in}{5.625727in}}%
\pgfpathlineto{\pgfqpoint{5.790410in}{5.626813in}}%
\pgfpathlineto{\pgfqpoint{5.808575in}{5.627704in}}%
\pgfpathlineto{\pgfqpoint{5.825341in}{5.628634in}}%
\pgfpathlineto{\pgfqpoint{5.840907in}{5.629616in}}%
\pgfpathlineto{\pgfqpoint{5.855435in}{5.630651in}}%
\pgfpathlineto{\pgfqpoint{5.869053in}{5.631737in}}%
\pgfpathlineto{\pgfqpoint{5.881870in}{5.632876in}}%
\pgfpathlineto{\pgfqpoint{5.893974in}{5.634068in}}%
\pgfpathlineto{\pgfqpoint{5.905441in}{5.635313in}}%
\pgfpathlineto{\pgfqpoint{5.916334in}{5.636611in}}%
\pgfpathlineto{\pgfqpoint{5.926708in}{5.637963in}}%
\pgfpathlineto{\pgfqpoint{5.936610in}{5.639367in}}%
\pgfpathlineto{\pgfqpoint{5.946082in}{5.640823in}}%
\pgfpathlineto{\pgfqpoint{5.955158in}{5.642327in}}%
\pgfpathlineto{\pgfqpoint{5.963871in}{5.643861in}}%
\pgfpathlineto{\pgfqpoint{5.972249in}{5.645394in}}%
\pgfpathlineto{\pgfqpoint{5.980317in}{5.646924in}}%
\pgfpathlineto{\pgfqpoint{5.988096in}{5.648482in}}%
\pgfpathlineto{\pgfqpoint{5.995607in}{5.650082in}}%
\pgfpathlineto{\pgfqpoint{6.002868in}{5.651733in}}%
\pgfpathlineto{\pgfqpoint{6.009894in}{5.653440in}}%
\pgfpathlineto{\pgfqpoint{6.016701in}{5.655209in}}%
\pgfpathlineto{\pgfqpoint{6.023301in}{5.657045in}}%
\pgfpathlineto{\pgfqpoint{6.029707in}{5.658955in}}%
\pgfpathlineto{\pgfqpoint{6.035930in}{5.660945in}}%
\pgfpathlineto{\pgfqpoint{6.041980in}{5.663024in}}%
\pgfpathlineto{\pgfqpoint{6.047867in}{5.665199in}}%
\pgfpathlineto{\pgfqpoint{6.053598in}{5.667481in}}%
\pgfpathlineto{\pgfqpoint{6.059183in}{5.669879in}}%
\pgfpathlineto{\pgfqpoint{6.064628in}{5.672405in}}%
\pgfpathlineto{\pgfqpoint{6.069940in}{5.675073in}}%
\pgfpathlineto{\pgfqpoint{6.075126in}{5.677896in}}%
\pgfpathlineto{\pgfqpoint{6.080191in}{5.680890in}}%
\pgfpathlineto{\pgfqpoint{6.085140in}{5.684071in}}%
\pgfpathlineto{\pgfqpoint{6.089980in}{5.687459in}}%
\pgfpathlineto{\pgfqpoint{6.094715in}{5.691072in}}%
\pgfpathlineto{\pgfqpoint{6.099349in}{5.694931in}}%
\pgfpathlineto{\pgfqpoint{6.103886in}{5.699055in}}%
\pgfpathlineto{\pgfqpoint{6.108331in}{5.703129in}}%
\pgfpathlineto{\pgfqpoint{6.108331in}{5.703535in}}%
\pgfpathlineto{\pgfqpoint{6.108331in}{5.703535in}}%
\pgfpathlineto{\pgfqpoint{6.103886in}{5.699743in}}%
\pgfpathlineto{\pgfqpoint{6.099349in}{5.696389in}}%
\pgfpathlineto{\pgfqpoint{6.094715in}{5.693134in}}%
\pgfpathlineto{\pgfqpoint{6.089980in}{5.689972in}}%
\pgfpathlineto{\pgfqpoint{6.085140in}{5.686902in}}%
\pgfpathlineto{\pgfqpoint{6.080191in}{5.683922in}}%
\pgfpathlineto{\pgfqpoint{6.075126in}{5.681032in}}%
\pgfpathlineto{\pgfqpoint{6.069940in}{5.678231in}}%
\pgfpathlineto{\pgfqpoint{6.064628in}{5.675516in}}%
\pgfpathlineto{\pgfqpoint{6.059183in}{5.672887in}}%
\pgfpathlineto{\pgfqpoint{6.053598in}{5.670342in}}%
\pgfpathlineto{\pgfqpoint{6.047867in}{5.667879in}}%
\pgfpathlineto{\pgfqpoint{6.041980in}{5.665496in}}%
\pgfpathlineto{\pgfqpoint{6.035930in}{5.663191in}}%
\pgfpathlineto{\pgfqpoint{6.029707in}{5.660961in}}%
\pgfpathlineto{\pgfqpoint{6.023301in}{5.658806in}}%
\pgfpathlineto{\pgfqpoint{6.016701in}{5.656722in}}%
\pgfpathlineto{\pgfqpoint{6.009894in}{5.654709in}}%
\pgfpathlineto{\pgfqpoint{6.002868in}{5.652764in}}%
\pgfpathlineto{\pgfqpoint{5.995607in}{5.650886in}}%
\pgfpathlineto{\pgfqpoint{5.988096in}{5.649075in}}%
\pgfpathlineto{\pgfqpoint{5.980317in}{5.647333in}}%
\pgfpathlineto{\pgfqpoint{5.972249in}{5.645672in}}%
\pgfpathlineto{\pgfqpoint{5.963871in}{5.644114in}}%
\pgfpathlineto{\pgfqpoint{5.955158in}{5.642657in}}%
\pgfpathlineto{\pgfqpoint{5.946082in}{5.641264in}}%
\pgfpathlineto{\pgfqpoint{5.936610in}{5.639915in}}%
\pgfpathlineto{\pgfqpoint{5.926708in}{5.638602in}}%
\pgfpathlineto{\pgfqpoint{5.916334in}{5.637322in}}%
\pgfpathlineto{\pgfqpoint{5.905441in}{5.636070in}}%
\pgfpathlineto{\pgfqpoint{5.893974in}{5.634846in}}%
\pgfpathlineto{\pgfqpoint{5.881870in}{5.633646in}}%
\pgfpathlineto{\pgfqpoint{5.869053in}{5.632469in}}%
\pgfpathlineto{\pgfqpoint{5.855435in}{5.631311in}}%
\pgfpathlineto{\pgfqpoint{5.840907in}{5.630171in}}%
\pgfpathlineto{\pgfqpoint{5.825341in}{5.629045in}}%
\pgfpathlineto{\pgfqpoint{5.808575in}{5.627932in}}%
\pgfpathlineto{\pgfqpoint{5.790410in}{5.626840in}}%
\pgfpathlineto{\pgfqpoint{5.770591in}{5.626001in}}%
\pgfpathlineto{\pgfqpoint{5.748786in}{5.625226in}}%
\pgfpathlineto{\pgfqpoint{5.724552in}{5.624503in}}%
\pgfpathlineto{\pgfqpoint{5.697279in}{5.623832in}}%
\pgfpathlineto{\pgfqpoint{5.666095in}{5.623214in}}%
\pgfpathlineto{\pgfqpoint{5.629687in}{5.622649in}}%
\pgfpathlineto{\pgfqpoint{5.585945in}{5.622139in}}%
\pgfpathlineto{\pgfqpoint{5.531149in}{5.621686in}}%
\pgfpathlineto{\pgfqpoint{5.457753in}{5.621292in}}%
\pgfpathlineto{\pgfqpoint{5.346222in}{5.620961in}}%
\pgfpathlineto{\pgfqpoint{5.105513in}{5.620695in}}%
\pgfpathclose%
\pgfusepath{stroke,fill}%
\end{pgfscope}%
\begin{pgfscope}%
\pgfpathrectangle{\pgfqpoint{5.105513in}{5.542197in}}{\pgfqpoint{1.223103in}{0.607948in}}%
\pgfusepath{clip}%
\pgfsetroundcap%
\pgfsetroundjoin%
\pgfsetlinewidth{0.501875pt}%
\definecolor{currentstroke}{rgb}{0.000000,0.000000,1.000000}%
\pgfsetstrokecolor{currentstroke}%
\pgfsetstrokeopacity{0.800000}%
\pgfsetdash{}{0pt}%
\pgfpathmoveto{\pgfqpoint{5.105513in}{5.616876in}}%
\pgfpathlineto{\pgfqpoint{6.328616in}{5.616876in}}%
\pgfusepath{stroke}%
\end{pgfscope}%
\begin{pgfscope}%
\pgfpathrectangle{\pgfqpoint{5.105513in}{5.542197in}}{\pgfqpoint{1.223103in}{0.607948in}}%
\pgfusepath{clip}%
\pgfsetbuttcap%
\pgfsetroundjoin%
\pgfsetlinewidth{1.003750pt}%
\definecolor{currentstroke}{rgb}{0.000000,0.000000,0.000000}%
\pgfsetstrokecolor{currentstroke}%
\pgfsetdash{{3.700000pt}{1.600000pt}}{0.000000pt}%
\pgfpathmoveto{\pgfqpoint{5.105513in}{5.616260in}}%
\pgfpathlineto{\pgfqpoint{6.328616in}{5.616260in}}%
\pgfusepath{stroke}%
\end{pgfscope}%
\begin{pgfscope}%
\pgfsetroundcap%
\pgfsetroundjoin%
\pgfsetlinewidth{0.501875pt}%
\definecolor{currentstroke}{rgb}{0.000000,0.000000,1.000000}%
\pgfsetstrokecolor{currentstroke}%
\pgfsetstrokeopacity{0.800000}%
\pgfsetdash{}{0pt}%
\pgfpathmoveto{\pgfqpoint{5.932592in}{5.734979in}}%
\pgfpathquadraticcurveto{\pgfqpoint{5.856728in}{5.683702in}}{\pgfqpoint{5.780865in}{5.632424in}}%
\pgfusepath{stroke}%
\end{pgfscope}%
\begin{pgfscope}%
\pgfsetfillopacity{0.800000}%
\pgfsetstrokeopacity{0.800000}%
\definecolor{textcolor}{rgb}{0.000000,0.000000,1.000000}%
\pgfsetstrokecolor{textcolor}%
\pgfsetfillcolor{textcolor}%
\pgftext[x=5.846155in,y=5.799260in,left,base]{\color{textcolor}\sffamily\fontsize{5.647059}{6.776471}\selectfont 16.1228(61)}%
\end{pgfscope}%
\begin{pgfscope}%
\pgfsetbuttcap%
\pgfsetroundjoin%
\definecolor{currentfill}{rgb}{0.150000,0.150000,0.150000}%
\pgfsetfillcolor{currentfill}%
\pgfsetlinewidth{1.003750pt}%
\definecolor{currentstroke}{rgb}{0.150000,0.150000,0.150000}%
\pgfsetstrokecolor{currentstroke}%
\pgfsetdash{}{0pt}%
\pgfsys@defobject{currentmarker}{\pgfqpoint{0.000000in}{-0.066667in}}{\pgfqpoint{0.000000in}{0.000000in}}{%
\pgfpathmoveto{\pgfqpoint{0.000000in}{0.000000in}}%
\pgfpathlineto{\pgfqpoint{0.000000in}{-0.066667in}}%
\pgfusepath{stroke,fill}%
}%
\begin{pgfscope}%
\pgfsys@transformshift{5.105513in}{5.542197in}%
\pgfsys@useobject{currentmarker}{}%
\end{pgfscope}%
\end{pgfscope}%
\begin{pgfscope}%
\pgfsetbuttcap%
\pgfsetroundjoin%
\definecolor{currentfill}{rgb}{0.150000,0.150000,0.150000}%
\pgfsetfillcolor{currentfill}%
\pgfsetlinewidth{1.003750pt}%
\definecolor{currentstroke}{rgb}{0.150000,0.150000,0.150000}%
\pgfsetstrokecolor{currentstroke}%
\pgfsetdash{}{0pt}%
\pgfsys@defobject{currentmarker}{\pgfqpoint{0.000000in}{-0.066667in}}{\pgfqpoint{0.000000in}{0.000000in}}{%
\pgfpathmoveto{\pgfqpoint{0.000000in}{0.000000in}}%
\pgfpathlineto{\pgfqpoint{0.000000in}{-0.066667in}}%
\pgfusepath{stroke,fill}%
}%
\begin{pgfscope}%
\pgfsys@transformshift{5.606922in}{5.542197in}%
\pgfsys@useobject{currentmarker}{}%
\end{pgfscope}%
\end{pgfscope}%
\begin{pgfscope}%
\pgfsetbuttcap%
\pgfsetroundjoin%
\definecolor{currentfill}{rgb}{0.150000,0.150000,0.150000}%
\pgfsetfillcolor{currentfill}%
\pgfsetlinewidth{1.003750pt}%
\definecolor{currentstroke}{rgb}{0.150000,0.150000,0.150000}%
\pgfsetstrokecolor{currentstroke}%
\pgfsetdash{}{0pt}%
\pgfsys@defobject{currentmarker}{\pgfqpoint{0.000000in}{-0.066667in}}{\pgfqpoint{0.000000in}{0.000000in}}{%
\pgfpathmoveto{\pgfqpoint{0.000000in}{0.000000in}}%
\pgfpathlineto{\pgfqpoint{0.000000in}{-0.066667in}}%
\pgfusepath{stroke,fill}%
}%
\begin{pgfscope}%
\pgfsys@transformshift{6.108331in}{5.542197in}%
\pgfsys@useobject{currentmarker}{}%
\end{pgfscope}%
\end{pgfscope}%
\begin{pgfscope}%
\pgfsetbuttcap%
\pgfsetroundjoin%
\definecolor{currentfill}{rgb}{0.150000,0.150000,0.150000}%
\pgfsetfillcolor{currentfill}%
\pgfsetlinewidth{0.803000pt}%
\definecolor{currentstroke}{rgb}{0.150000,0.150000,0.150000}%
\pgfsetstrokecolor{currentstroke}%
\pgfsetdash{}{0pt}%
\pgfsys@defobject{currentmarker}{\pgfqpoint{0.000000in}{-0.044444in}}{\pgfqpoint{0.000000in}{0.000000in}}{%
\pgfpathmoveto{\pgfqpoint{0.000000in}{0.000000in}}%
\pgfpathlineto{\pgfqpoint{0.000000in}{-0.044444in}}%
\pgfusepath{stroke,fill}%
}%
\begin{pgfscope}%
\pgfsys@transformshift{5.256452in}{5.542197in}%
\pgfsys@useobject{currentmarker}{}%
\end{pgfscope}%
\end{pgfscope}%
\begin{pgfscope}%
\pgfsetbuttcap%
\pgfsetroundjoin%
\definecolor{currentfill}{rgb}{0.150000,0.150000,0.150000}%
\pgfsetfillcolor{currentfill}%
\pgfsetlinewidth{0.803000pt}%
\definecolor{currentstroke}{rgb}{0.150000,0.150000,0.150000}%
\pgfsetstrokecolor{currentstroke}%
\pgfsetdash{}{0pt}%
\pgfsys@defobject{currentmarker}{\pgfqpoint{0.000000in}{-0.044444in}}{\pgfqpoint{0.000000in}{0.000000in}}{%
\pgfpathmoveto{\pgfqpoint{0.000000in}{0.000000in}}%
\pgfpathlineto{\pgfqpoint{0.000000in}{-0.044444in}}%
\pgfusepath{stroke,fill}%
}%
\begin{pgfscope}%
\pgfsys@transformshift{5.344746in}{5.542197in}%
\pgfsys@useobject{currentmarker}{}%
\end{pgfscope}%
\end{pgfscope}%
\begin{pgfscope}%
\pgfsetbuttcap%
\pgfsetroundjoin%
\definecolor{currentfill}{rgb}{0.150000,0.150000,0.150000}%
\pgfsetfillcolor{currentfill}%
\pgfsetlinewidth{0.803000pt}%
\definecolor{currentstroke}{rgb}{0.150000,0.150000,0.150000}%
\pgfsetstrokecolor{currentstroke}%
\pgfsetdash{}{0pt}%
\pgfsys@defobject{currentmarker}{\pgfqpoint{0.000000in}{-0.044444in}}{\pgfqpoint{0.000000in}{0.000000in}}{%
\pgfpathmoveto{\pgfqpoint{0.000000in}{0.000000in}}%
\pgfpathlineto{\pgfqpoint{0.000000in}{-0.044444in}}%
\pgfusepath{stroke,fill}%
}%
\begin{pgfscope}%
\pgfsys@transformshift{5.407391in}{5.542197in}%
\pgfsys@useobject{currentmarker}{}%
\end{pgfscope}%
\end{pgfscope}%
\begin{pgfscope}%
\pgfsetbuttcap%
\pgfsetroundjoin%
\definecolor{currentfill}{rgb}{0.150000,0.150000,0.150000}%
\pgfsetfillcolor{currentfill}%
\pgfsetlinewidth{0.803000pt}%
\definecolor{currentstroke}{rgb}{0.150000,0.150000,0.150000}%
\pgfsetstrokecolor{currentstroke}%
\pgfsetdash{}{0pt}%
\pgfsys@defobject{currentmarker}{\pgfqpoint{0.000000in}{-0.044444in}}{\pgfqpoint{0.000000in}{0.000000in}}{%
\pgfpathmoveto{\pgfqpoint{0.000000in}{0.000000in}}%
\pgfpathlineto{\pgfqpoint{0.000000in}{-0.044444in}}%
\pgfusepath{stroke,fill}%
}%
\begin{pgfscope}%
\pgfsys@transformshift{5.455982in}{5.542197in}%
\pgfsys@useobject{currentmarker}{}%
\end{pgfscope}%
\end{pgfscope}%
\begin{pgfscope}%
\pgfsetbuttcap%
\pgfsetroundjoin%
\definecolor{currentfill}{rgb}{0.150000,0.150000,0.150000}%
\pgfsetfillcolor{currentfill}%
\pgfsetlinewidth{0.803000pt}%
\definecolor{currentstroke}{rgb}{0.150000,0.150000,0.150000}%
\pgfsetstrokecolor{currentstroke}%
\pgfsetdash{}{0pt}%
\pgfsys@defobject{currentmarker}{\pgfqpoint{0.000000in}{-0.044444in}}{\pgfqpoint{0.000000in}{0.000000in}}{%
\pgfpathmoveto{\pgfqpoint{0.000000in}{0.000000in}}%
\pgfpathlineto{\pgfqpoint{0.000000in}{-0.044444in}}%
\pgfusepath{stroke,fill}%
}%
\begin{pgfscope}%
\pgfsys@transformshift{5.495685in}{5.542197in}%
\pgfsys@useobject{currentmarker}{}%
\end{pgfscope}%
\end{pgfscope}%
\begin{pgfscope}%
\pgfsetbuttcap%
\pgfsetroundjoin%
\definecolor{currentfill}{rgb}{0.150000,0.150000,0.150000}%
\pgfsetfillcolor{currentfill}%
\pgfsetlinewidth{0.803000pt}%
\definecolor{currentstroke}{rgb}{0.150000,0.150000,0.150000}%
\pgfsetstrokecolor{currentstroke}%
\pgfsetdash{}{0pt}%
\pgfsys@defobject{currentmarker}{\pgfqpoint{0.000000in}{-0.044444in}}{\pgfqpoint{0.000000in}{0.000000in}}{%
\pgfpathmoveto{\pgfqpoint{0.000000in}{0.000000in}}%
\pgfpathlineto{\pgfqpoint{0.000000in}{-0.044444in}}%
\pgfusepath{stroke,fill}%
}%
\begin{pgfscope}%
\pgfsys@transformshift{5.529252in}{5.542197in}%
\pgfsys@useobject{currentmarker}{}%
\end{pgfscope}%
\end{pgfscope}%
\begin{pgfscope}%
\pgfsetbuttcap%
\pgfsetroundjoin%
\definecolor{currentfill}{rgb}{0.150000,0.150000,0.150000}%
\pgfsetfillcolor{currentfill}%
\pgfsetlinewidth{0.803000pt}%
\definecolor{currentstroke}{rgb}{0.150000,0.150000,0.150000}%
\pgfsetstrokecolor{currentstroke}%
\pgfsetdash{}{0pt}%
\pgfsys@defobject{currentmarker}{\pgfqpoint{0.000000in}{-0.044444in}}{\pgfqpoint{0.000000in}{0.000000in}}{%
\pgfpathmoveto{\pgfqpoint{0.000000in}{0.000000in}}%
\pgfpathlineto{\pgfqpoint{0.000000in}{-0.044444in}}%
\pgfusepath{stroke,fill}%
}%
\begin{pgfscope}%
\pgfsys@transformshift{5.558330in}{5.542197in}%
\pgfsys@useobject{currentmarker}{}%
\end{pgfscope}%
\end{pgfscope}%
\begin{pgfscope}%
\pgfsetbuttcap%
\pgfsetroundjoin%
\definecolor{currentfill}{rgb}{0.150000,0.150000,0.150000}%
\pgfsetfillcolor{currentfill}%
\pgfsetlinewidth{0.803000pt}%
\definecolor{currentstroke}{rgb}{0.150000,0.150000,0.150000}%
\pgfsetstrokecolor{currentstroke}%
\pgfsetdash{}{0pt}%
\pgfsys@defobject{currentmarker}{\pgfqpoint{0.000000in}{-0.044444in}}{\pgfqpoint{0.000000in}{0.000000in}}{%
\pgfpathmoveto{\pgfqpoint{0.000000in}{0.000000in}}%
\pgfpathlineto{\pgfqpoint{0.000000in}{-0.044444in}}%
\pgfusepath{stroke,fill}%
}%
\begin{pgfscope}%
\pgfsys@transformshift{5.583978in}{5.542197in}%
\pgfsys@useobject{currentmarker}{}%
\end{pgfscope}%
\end{pgfscope}%
\begin{pgfscope}%
\pgfsetbuttcap%
\pgfsetroundjoin%
\definecolor{currentfill}{rgb}{0.150000,0.150000,0.150000}%
\pgfsetfillcolor{currentfill}%
\pgfsetlinewidth{0.803000pt}%
\definecolor{currentstroke}{rgb}{0.150000,0.150000,0.150000}%
\pgfsetstrokecolor{currentstroke}%
\pgfsetdash{}{0pt}%
\pgfsys@defobject{currentmarker}{\pgfqpoint{0.000000in}{-0.044444in}}{\pgfqpoint{0.000000in}{0.000000in}}{%
\pgfpathmoveto{\pgfqpoint{0.000000in}{0.000000in}}%
\pgfpathlineto{\pgfqpoint{0.000000in}{-0.044444in}}%
\pgfusepath{stroke,fill}%
}%
\begin{pgfscope}%
\pgfsys@transformshift{5.757861in}{5.542197in}%
\pgfsys@useobject{currentmarker}{}%
\end{pgfscope}%
\end{pgfscope}%
\begin{pgfscope}%
\pgfsetbuttcap%
\pgfsetroundjoin%
\definecolor{currentfill}{rgb}{0.150000,0.150000,0.150000}%
\pgfsetfillcolor{currentfill}%
\pgfsetlinewidth{0.803000pt}%
\definecolor{currentstroke}{rgb}{0.150000,0.150000,0.150000}%
\pgfsetstrokecolor{currentstroke}%
\pgfsetdash{}{0pt}%
\pgfsys@defobject{currentmarker}{\pgfqpoint{0.000000in}{-0.044444in}}{\pgfqpoint{0.000000in}{0.000000in}}{%
\pgfpathmoveto{\pgfqpoint{0.000000in}{0.000000in}}%
\pgfpathlineto{\pgfqpoint{0.000000in}{-0.044444in}}%
\pgfusepath{stroke,fill}%
}%
\begin{pgfscope}%
\pgfsys@transformshift{5.846155in}{5.542197in}%
\pgfsys@useobject{currentmarker}{}%
\end{pgfscope}%
\end{pgfscope}%
\begin{pgfscope}%
\pgfsetbuttcap%
\pgfsetroundjoin%
\definecolor{currentfill}{rgb}{0.150000,0.150000,0.150000}%
\pgfsetfillcolor{currentfill}%
\pgfsetlinewidth{0.803000pt}%
\definecolor{currentstroke}{rgb}{0.150000,0.150000,0.150000}%
\pgfsetstrokecolor{currentstroke}%
\pgfsetdash{}{0pt}%
\pgfsys@defobject{currentmarker}{\pgfqpoint{0.000000in}{-0.044444in}}{\pgfqpoint{0.000000in}{0.000000in}}{%
\pgfpathmoveto{\pgfqpoint{0.000000in}{0.000000in}}%
\pgfpathlineto{\pgfqpoint{0.000000in}{-0.044444in}}%
\pgfusepath{stroke,fill}%
}%
\begin{pgfscope}%
\pgfsys@transformshift{5.908800in}{5.542197in}%
\pgfsys@useobject{currentmarker}{}%
\end{pgfscope}%
\end{pgfscope}%
\begin{pgfscope}%
\pgfsetbuttcap%
\pgfsetroundjoin%
\definecolor{currentfill}{rgb}{0.150000,0.150000,0.150000}%
\pgfsetfillcolor{currentfill}%
\pgfsetlinewidth{0.803000pt}%
\definecolor{currentstroke}{rgb}{0.150000,0.150000,0.150000}%
\pgfsetstrokecolor{currentstroke}%
\pgfsetdash{}{0pt}%
\pgfsys@defobject{currentmarker}{\pgfqpoint{0.000000in}{-0.044444in}}{\pgfqpoint{0.000000in}{0.000000in}}{%
\pgfpathmoveto{\pgfqpoint{0.000000in}{0.000000in}}%
\pgfpathlineto{\pgfqpoint{0.000000in}{-0.044444in}}%
\pgfusepath{stroke,fill}%
}%
\begin{pgfscope}%
\pgfsys@transformshift{5.957391in}{5.542197in}%
\pgfsys@useobject{currentmarker}{}%
\end{pgfscope}%
\end{pgfscope}%
\begin{pgfscope}%
\pgfsetbuttcap%
\pgfsetroundjoin%
\definecolor{currentfill}{rgb}{0.150000,0.150000,0.150000}%
\pgfsetfillcolor{currentfill}%
\pgfsetlinewidth{0.803000pt}%
\definecolor{currentstroke}{rgb}{0.150000,0.150000,0.150000}%
\pgfsetstrokecolor{currentstroke}%
\pgfsetdash{}{0pt}%
\pgfsys@defobject{currentmarker}{\pgfqpoint{0.000000in}{-0.044444in}}{\pgfqpoint{0.000000in}{0.000000in}}{%
\pgfpathmoveto{\pgfqpoint{0.000000in}{0.000000in}}%
\pgfpathlineto{\pgfqpoint{0.000000in}{-0.044444in}}%
\pgfusepath{stroke,fill}%
}%
\begin{pgfscope}%
\pgfsys@transformshift{5.997094in}{5.542197in}%
\pgfsys@useobject{currentmarker}{}%
\end{pgfscope}%
\end{pgfscope}%
\begin{pgfscope}%
\pgfsetbuttcap%
\pgfsetroundjoin%
\definecolor{currentfill}{rgb}{0.150000,0.150000,0.150000}%
\pgfsetfillcolor{currentfill}%
\pgfsetlinewidth{0.803000pt}%
\definecolor{currentstroke}{rgb}{0.150000,0.150000,0.150000}%
\pgfsetstrokecolor{currentstroke}%
\pgfsetdash{}{0pt}%
\pgfsys@defobject{currentmarker}{\pgfqpoint{0.000000in}{-0.044444in}}{\pgfqpoint{0.000000in}{0.000000in}}{%
\pgfpathmoveto{\pgfqpoint{0.000000in}{0.000000in}}%
\pgfpathlineto{\pgfqpoint{0.000000in}{-0.044444in}}%
\pgfusepath{stroke,fill}%
}%
\begin{pgfscope}%
\pgfsys@transformshift{6.030661in}{5.542197in}%
\pgfsys@useobject{currentmarker}{}%
\end{pgfscope}%
\end{pgfscope}%
\begin{pgfscope}%
\pgfsetbuttcap%
\pgfsetroundjoin%
\definecolor{currentfill}{rgb}{0.150000,0.150000,0.150000}%
\pgfsetfillcolor{currentfill}%
\pgfsetlinewidth{0.803000pt}%
\definecolor{currentstroke}{rgb}{0.150000,0.150000,0.150000}%
\pgfsetstrokecolor{currentstroke}%
\pgfsetdash{}{0pt}%
\pgfsys@defobject{currentmarker}{\pgfqpoint{0.000000in}{-0.044444in}}{\pgfqpoint{0.000000in}{0.000000in}}{%
\pgfpathmoveto{\pgfqpoint{0.000000in}{0.000000in}}%
\pgfpathlineto{\pgfqpoint{0.000000in}{-0.044444in}}%
\pgfusepath{stroke,fill}%
}%
\begin{pgfscope}%
\pgfsys@transformshift{6.059739in}{5.542197in}%
\pgfsys@useobject{currentmarker}{}%
\end{pgfscope}%
\end{pgfscope}%
\begin{pgfscope}%
\pgfsetbuttcap%
\pgfsetroundjoin%
\definecolor{currentfill}{rgb}{0.150000,0.150000,0.150000}%
\pgfsetfillcolor{currentfill}%
\pgfsetlinewidth{0.803000pt}%
\definecolor{currentstroke}{rgb}{0.150000,0.150000,0.150000}%
\pgfsetstrokecolor{currentstroke}%
\pgfsetdash{}{0pt}%
\pgfsys@defobject{currentmarker}{\pgfqpoint{0.000000in}{-0.044444in}}{\pgfqpoint{0.000000in}{0.000000in}}{%
\pgfpathmoveto{\pgfqpoint{0.000000in}{0.000000in}}%
\pgfpathlineto{\pgfqpoint{0.000000in}{-0.044444in}}%
\pgfusepath{stroke,fill}%
}%
\begin{pgfscope}%
\pgfsys@transformshift{6.085387in}{5.542197in}%
\pgfsys@useobject{currentmarker}{}%
\end{pgfscope}%
\end{pgfscope}%
\begin{pgfscope}%
\pgfsetbuttcap%
\pgfsetroundjoin%
\definecolor{currentfill}{rgb}{0.150000,0.150000,0.150000}%
\pgfsetfillcolor{currentfill}%
\pgfsetlinewidth{0.803000pt}%
\definecolor{currentstroke}{rgb}{0.150000,0.150000,0.150000}%
\pgfsetstrokecolor{currentstroke}%
\pgfsetdash{}{0pt}%
\pgfsys@defobject{currentmarker}{\pgfqpoint{0.000000in}{-0.044444in}}{\pgfqpoint{0.000000in}{0.000000in}}{%
\pgfpathmoveto{\pgfqpoint{0.000000in}{0.000000in}}%
\pgfpathlineto{\pgfqpoint{0.000000in}{-0.044444in}}%
\pgfusepath{stroke,fill}%
}%
\begin{pgfscope}%
\pgfsys@transformshift{6.259270in}{5.542197in}%
\pgfsys@useobject{currentmarker}{}%
\end{pgfscope}%
\end{pgfscope}%
\begin{pgfscope}%
\pgfsetbuttcap%
\pgfsetroundjoin%
\definecolor{currentfill}{rgb}{0.150000,0.150000,0.150000}%
\pgfsetfillcolor{currentfill}%
\pgfsetlinewidth{1.003750pt}%
\definecolor{currentstroke}{rgb}{0.150000,0.150000,0.150000}%
\pgfsetstrokecolor{currentstroke}%
\pgfsetdash{}{0pt}%
\pgfsys@defobject{currentmarker}{\pgfqpoint{-0.066667in}{0.000000in}}{\pgfqpoint{0.000000in}{0.000000in}}{%
\pgfpathmoveto{\pgfqpoint{0.000000in}{0.000000in}}%
\pgfpathlineto{\pgfqpoint{-0.066667in}{0.000000in}}%
\pgfusepath{stroke,fill}%
}%
\begin{pgfscope}%
\pgfsys@transformshift{5.105513in}{5.542197in}%
\pgfsys@useobject{currentmarker}{}%
\end{pgfscope}%
\end{pgfscope}%
\begin{pgfscope}%
\pgfsetbuttcap%
\pgfsetroundjoin%
\definecolor{currentfill}{rgb}{0.150000,0.150000,0.150000}%
\pgfsetfillcolor{currentfill}%
\pgfsetlinewidth{1.003750pt}%
\definecolor{currentstroke}{rgb}{0.150000,0.150000,0.150000}%
\pgfsetstrokecolor{currentstroke}%
\pgfsetdash{}{0pt}%
\pgfsys@defobject{currentmarker}{\pgfqpoint{-0.066667in}{0.000000in}}{\pgfqpoint{0.000000in}{0.000000in}}{%
\pgfpathmoveto{\pgfqpoint{0.000000in}{0.000000in}}%
\pgfpathlineto{\pgfqpoint{-0.066667in}{0.000000in}}%
\pgfusepath{stroke,fill}%
}%
\begin{pgfscope}%
\pgfsys@transformshift{5.105513in}{5.616260in}%
\pgfsys@useobject{currentmarker}{}%
\end{pgfscope}%
\end{pgfscope}%
\begin{pgfscope}%
\pgfsetbuttcap%
\pgfsetroundjoin%
\definecolor{currentfill}{rgb}{0.150000,0.150000,0.150000}%
\pgfsetfillcolor{currentfill}%
\pgfsetlinewidth{1.003750pt}%
\definecolor{currentstroke}{rgb}{0.150000,0.150000,0.150000}%
\pgfsetstrokecolor{currentstroke}%
\pgfsetdash{}{0pt}%
\pgfsys@defobject{currentmarker}{\pgfqpoint{-0.066667in}{0.000000in}}{\pgfqpoint{0.000000in}{0.000000in}}{%
\pgfpathmoveto{\pgfqpoint{0.000000in}{0.000000in}}%
\pgfpathlineto{\pgfqpoint{-0.066667in}{0.000000in}}%
\pgfusepath{stroke,fill}%
}%
\begin{pgfscope}%
\pgfsys@transformshift{5.105513in}{6.150145in}%
\pgfsys@useobject{currentmarker}{}%
\end{pgfscope}%
\end{pgfscope}%
\begin{pgfscope}%
\pgfpathrectangle{\pgfqpoint{5.105513in}{5.542197in}}{\pgfqpoint{1.223103in}{0.607948in}}%
\pgfusepath{clip}%
\pgfsetroundcap%
\pgfsetroundjoin%
\pgfsetlinewidth{1.204500pt}%
\definecolor{currentstroke}{rgb}{0.000000,0.501961,0.000000}%
\pgfsetstrokecolor{currentstroke}%
\pgfsetdash{}{0pt}%
\pgfpathmoveto{\pgfqpoint{5.105513in}{5.617284in}}%
\pgfpathlineto{\pgfqpoint{5.346222in}{5.618104in}}%
\pgfpathlineto{\pgfqpoint{5.457753in}{5.618922in}}%
\pgfpathlineto{\pgfqpoint{5.531149in}{5.619743in}}%
\pgfpathlineto{\pgfqpoint{5.585945in}{5.620570in}}%
\pgfpathlineto{\pgfqpoint{5.629687in}{5.621408in}}%
\pgfpathlineto{\pgfqpoint{5.666095in}{5.622260in}}%
\pgfpathlineto{\pgfqpoint{5.697279in}{5.623129in}}%
\pgfpathlineto{\pgfqpoint{5.724552in}{5.624017in}}%
\pgfpathlineto{\pgfqpoint{5.748786in}{5.624928in}}%
\pgfpathlineto{\pgfqpoint{5.770591in}{5.625864in}}%
\pgfpathlineto{\pgfqpoint{5.790410in}{5.626827in}}%
\pgfpathlineto{\pgfqpoint{5.808575in}{5.627818in}}%
\pgfpathlineto{\pgfqpoint{5.825341in}{5.628840in}}%
\pgfpathlineto{\pgfqpoint{5.840907in}{5.629893in}}%
\pgfpathlineto{\pgfqpoint{5.855435in}{5.630981in}}%
\pgfpathlineto{\pgfqpoint{5.869053in}{5.632103in}}%
\pgfpathlineto{\pgfqpoint{5.881870in}{5.633261in}}%
\pgfpathlineto{\pgfqpoint{5.893974in}{5.634457in}}%
\pgfpathlineto{\pgfqpoint{5.905441in}{5.635692in}}%
\pgfpathlineto{\pgfqpoint{5.916334in}{5.636966in}}%
\pgfpathlineto{\pgfqpoint{5.926708in}{5.638282in}}%
\pgfpathlineto{\pgfqpoint{5.936610in}{5.639641in}}%
\pgfpathlineto{\pgfqpoint{5.946082in}{5.641044in}}%
\pgfpathlineto{\pgfqpoint{5.955158in}{5.642492in}}%
\pgfpathlineto{\pgfqpoint{5.963871in}{5.643988in}}%
\pgfpathlineto{\pgfqpoint{5.972249in}{5.645533in}}%
\pgfpathlineto{\pgfqpoint{5.980317in}{5.647129in}}%
\pgfpathlineto{\pgfqpoint{5.988096in}{5.648778in}}%
\pgfpathlineto{\pgfqpoint{5.995607in}{5.650484in}}%
\pgfpathlineto{\pgfqpoint{6.002868in}{5.652248in}}%
\pgfpathlineto{\pgfqpoint{6.009894in}{5.654074in}}%
\pgfpathlineto{\pgfqpoint{6.016701in}{5.655965in}}%
\pgfpathlineto{\pgfqpoint{6.023301in}{5.657925in}}%
\pgfpathlineto{\pgfqpoint{6.029707in}{5.659958in}}%
\pgfpathlineto{\pgfqpoint{6.035930in}{5.662068in}}%
\pgfpathlineto{\pgfqpoint{6.041980in}{5.664260in}}%
\pgfpathlineto{\pgfqpoint{6.047867in}{5.666539in}}%
\pgfpathlineto{\pgfqpoint{6.053598in}{5.668911in}}%
\pgfpathlineto{\pgfqpoint{6.059183in}{5.671383in}}%
\pgfpathlineto{\pgfqpoint{6.064628in}{5.673961in}}%
\pgfpathlineto{\pgfqpoint{6.069940in}{5.676652in}}%
\pgfpathlineto{\pgfqpoint{6.075126in}{5.679464in}}%
\pgfpathlineto{\pgfqpoint{6.080191in}{5.682406in}}%
\pgfpathlineto{\pgfqpoint{6.085140in}{5.685487in}}%
\pgfpathlineto{\pgfqpoint{6.089980in}{5.688715in}}%
\pgfpathlineto{\pgfqpoint{6.094715in}{5.692103in}}%
\pgfpathlineto{\pgfqpoint{6.099349in}{5.695660in}}%
\pgfpathlineto{\pgfqpoint{6.103886in}{5.699399in}}%
\pgfpathlineto{\pgfqpoint{6.108331in}{5.703332in}}%
\pgfusepath{stroke}%
\end{pgfscope}%
\begin{pgfscope}%
\pgfsetrectcap%
\pgfsetmiterjoin%
\pgfsetlinewidth{1.003750pt}%
\definecolor{currentstroke}{rgb}{0.150000,0.150000,0.150000}%
\pgfsetstrokecolor{currentstroke}%
\pgfsetdash{}{0pt}%
\pgfpathmoveto{\pgfqpoint{5.105513in}{5.542197in}}%
\pgfpathlineto{\pgfqpoint{5.105513in}{6.150145in}}%
\pgfusepath{stroke}%
\end{pgfscope}%
\begin{pgfscope}%
\pgfsetrectcap%
\pgfsetmiterjoin%
\pgfsetlinewidth{1.003750pt}%
\definecolor{currentstroke}{rgb}{0.150000,0.150000,0.150000}%
\pgfsetstrokecolor{currentstroke}%
\pgfsetdash{}{0pt}%
\pgfpathmoveto{\pgfqpoint{5.105513in}{5.542197in}}%
\pgfpathlineto{\pgfqpoint{6.328616in}{5.542197in}}%
\pgfusepath{stroke}%
\end{pgfscope}%
\begin{pgfscope}%
\pgfpathrectangle{\pgfqpoint{5.105513in}{5.542197in}}{\pgfqpoint{1.223103in}{0.607948in}}%
\pgfusepath{clip}%
\pgfsetbuttcap%
\pgfsetroundjoin%
\definecolor{currentfill}{rgb}{0.000000,0.000000,0.000000}%
\pgfsetfillcolor{currentfill}%
\pgfsetlinewidth{1.003750pt}%
\definecolor{currentstroke}{rgb}{0.000000,0.000000,0.000000}%
\pgfsetstrokecolor{currentstroke}%
\pgfsetdash{}{0pt}%
\pgfsys@defobject{currentmarker}{\pgfqpoint{-0.013889in}{-0.013889in}}{\pgfqpoint{0.013889in}{0.013889in}}{%
\pgfpathmoveto{\pgfqpoint{0.000000in}{-0.013889in}}%
\pgfpathcurveto{\pgfqpoint{0.003683in}{-0.013889in}}{\pgfqpoint{0.007216in}{-0.012425in}}{\pgfqpoint{0.009821in}{-0.009821in}}%
\pgfpathcurveto{\pgfqpoint{0.012425in}{-0.007216in}}{\pgfqpoint{0.013889in}{-0.003683in}}{\pgfqpoint{0.013889in}{0.000000in}}%
\pgfpathcurveto{\pgfqpoint{0.013889in}{0.003683in}}{\pgfqpoint{0.012425in}{0.007216in}}{\pgfqpoint{0.009821in}{0.009821in}}%
\pgfpathcurveto{\pgfqpoint{0.007216in}{0.012425in}}{\pgfqpoint{0.003683in}{0.013889in}}{\pgfqpoint{0.000000in}{0.013889in}}%
\pgfpathcurveto{\pgfqpoint{-0.003683in}{0.013889in}}{\pgfqpoint{-0.007216in}{0.012425in}}{\pgfqpoint{-0.009821in}{0.009821in}}%
\pgfpathcurveto{\pgfqpoint{-0.012425in}{0.007216in}}{\pgfqpoint{-0.013889in}{0.003683in}}{\pgfqpoint{-0.013889in}{0.000000in}}%
\pgfpathcurveto{\pgfqpoint{-0.013889in}{-0.003683in}}{\pgfqpoint{-0.012425in}{-0.007216in}}{\pgfqpoint{-0.009821in}{-0.009821in}}%
\pgfpathcurveto{\pgfqpoint{-0.007216in}{-0.012425in}}{\pgfqpoint{-0.003683in}{-0.013889in}}{\pgfqpoint{0.000000in}{-0.013889in}}%
\pgfpathclose%
\pgfusepath{stroke,fill}%
}%
\begin{pgfscope}%
\pgfsys@transformshift{5.757861in}{5.625202in}%
\pgfsys@useobject{currentmarker}{}%
\end{pgfscope}%
\begin{pgfscope}%
\pgfsys@transformshift{5.762260in}{5.625398in}%
\pgfsys@useobject{currentmarker}{}%
\end{pgfscope}%
\begin{pgfscope}%
\pgfsys@transformshift{5.766750in}{5.625616in}%
\pgfsys@useobject{currentmarker}{}%
\end{pgfscope}%
\begin{pgfscope}%
\pgfsys@transformshift{5.771335in}{5.625830in}%
\pgfsys@useobject{currentmarker}{}%
\end{pgfscope}%
\begin{pgfscope}%
\pgfsys@transformshift{5.776018in}{5.626069in}%
\pgfsys@useobject{currentmarker}{}%
\end{pgfscope}%
\begin{pgfscope}%
\pgfsys@transformshift{5.780804in}{5.626305in}%
\pgfsys@useobject{currentmarker}{}%
\end{pgfscope}%
\begin{pgfscope}%
\pgfsys@transformshift{5.785698in}{5.626569in}%
\pgfsys@useobject{currentmarker}{}%
\end{pgfscope}%
\begin{pgfscope}%
\pgfsys@transformshift{5.790704in}{5.626829in}%
\pgfsys@useobject{currentmarker}{}%
\end{pgfscope}%
\begin{pgfscope}%
\pgfsys@transformshift{5.795828in}{5.627123in}%
\pgfsys@useobject{currentmarker}{}%
\end{pgfscope}%
\begin{pgfscope}%
\pgfsys@transformshift{5.801075in}{5.627412in}%
\pgfsys@useobject{currentmarker}{}%
\end{pgfscope}%
\begin{pgfscope}%
\pgfsys@transformshift{5.806452in}{5.627740in}%
\pgfsys@useobject{currentmarker}{}%
\end{pgfscope}%
\begin{pgfscope}%
\pgfsys@transformshift{5.835530in}{5.629624in}%
\pgfsys@useobject{currentmarker}{}%
\end{pgfscope}%
\begin{pgfscope}%
\pgfsys@transformshift{5.869098in}{5.632309in}%
\pgfsys@useobject{currentmarker}{}%
\end{pgfscope}%
\begin{pgfscope}%
\pgfsys@transformshift{5.908800in}{5.636260in}%
\pgfsys@useobject{currentmarker}{}%
\end{pgfscope}%
\begin{pgfscope}%
\pgfsys@transformshift{5.957391in}{5.643058in}%
\pgfsys@useobject{currentmarker}{}%
\end{pgfscope}%
\begin{pgfscope}%
\pgfsys@transformshift{6.020037in}{5.656406in}%
\pgfsys@useobject{currentmarker}{}%
\end{pgfscope}%
\begin{pgfscope}%
\pgfsys@transformshift{6.108331in}{5.703455in}%
\pgfsys@useobject{currentmarker}{}%
\end{pgfscope}%
\end{pgfscope}%
\begin{pgfscope}%
\pgfsetbuttcap%
\pgfsetmiterjoin%
\definecolor{currentfill}{rgb}{1.000000,1.000000,1.000000}%
\pgfsetfillcolor{currentfill}%
\pgfsetlinewidth{0.803000pt}%
\definecolor{currentstroke}{rgb}{1.000000,1.000000,1.000000}%
\pgfsetstrokecolor{currentstroke}%
\pgfsetdash{}{0pt}%
\pgfpathmoveto{\pgfqpoint{6.297392in}{6.101269in}}%
\pgfpathlineto{\pgfqpoint{6.297392in}{5.591073in}}%
\pgfpathlineto{\pgfqpoint{6.478411in}{5.591073in}}%
\pgfpathlineto{\pgfqpoint{6.478411in}{6.101269in}}%
\pgfpathclose%
\pgfusepath{stroke,fill}%
\end{pgfscope}%
\begin{pgfscope}%
\definecolor{textcolor}{rgb}{0.150000,0.150000,0.150000}%
\pgfsetstrokecolor{textcolor}%
\pgfsetfillcolor{textcolor}%
\pgftext[x=6.368294in,y=6.045582in,left,base,rotate=270.000000]{\color{textcolor}\sffamily\fontsize{5.647059}{6.776471}\selectfont nlevel = 16}%
\end{pgfscope}%
\begin{pgfscope}%
\pgfsetbuttcap%
\pgfsetmiterjoin%
\definecolor{currentfill}{rgb}{1.000000,1.000000,1.000000}%
\pgfsetfillcolor{currentfill}%
\pgfsetlinewidth{0.803000pt}%
\definecolor{currentstroke}{rgb}{1.000000,1.000000,1.000000}%
\pgfsetstrokecolor{currentstroke}%
\pgfsetdash{}{0pt}%
\pgfpathmoveto{\pgfqpoint{6.297392in}{6.101269in}}%
\pgfpathlineto{\pgfqpoint{6.297392in}{5.591073in}}%
\pgfpathlineto{\pgfqpoint{6.478411in}{5.591073in}}%
\pgfpathlineto{\pgfqpoint{6.478411in}{6.101269in}}%
\pgfpathclose%
\pgfusepath{stroke,fill}%
\end{pgfscope}%
\begin{pgfscope}%
\definecolor{textcolor}{rgb}{0.150000,0.150000,0.150000}%
\pgfsetstrokecolor{textcolor}%
\pgfsetfillcolor{textcolor}%
\pgftext[x=6.368294in,y=6.045582in,left,base,rotate=270.000000]{\color{textcolor}\sffamily\fontsize{5.647059}{6.776471}\selectfont nlevel = 16}%
\end{pgfscope}%
\begin{pgfscope}%
\pgfsetbuttcap%
\pgfsetmiterjoin%
\definecolor{currentfill}{rgb}{1.000000,1.000000,1.000000}%
\pgfsetfillcolor{currentfill}%
\pgfsetlinewidth{0.000000pt}%
\definecolor{currentstroke}{rgb}{0.000000,0.000000,0.000000}%
\pgfsetstrokecolor{currentstroke}%
\pgfsetstrokeopacity{0.000000}%
\pgfsetdash{}{0pt}%
\pgfpathmoveto{\pgfqpoint{0.702340in}{4.812659in}}%
\pgfpathlineto{\pgfqpoint{1.925444in}{4.812659in}}%
\pgfpathlineto{\pgfqpoint{1.925444in}{5.420607in}}%
\pgfpathlineto{\pgfqpoint{0.702340in}{5.420607in}}%
\pgfpathclose%
\pgfusepath{fill}%
\end{pgfscope}%
\begin{pgfscope}%
\pgfpathrectangle{\pgfqpoint{0.702340in}{4.812659in}}{\pgfqpoint{1.223103in}{0.607948in}}%
\pgfusepath{clip}%
\pgfsetbuttcap%
\pgfsetmiterjoin%
\definecolor{currentfill}{rgb}{0.000000,0.000000,1.000000}%
\pgfsetfillcolor{currentfill}%
\pgfsetfillopacity{0.100000}%
\pgfsetlinewidth{0.803000pt}%
\definecolor{currentstroke}{rgb}{0.000000,0.000000,1.000000}%
\pgfsetstrokecolor{currentstroke}%
\pgfsetstrokeopacity{0.100000}%
\pgfsetdash{}{0pt}%
\pgfpathmoveto{\pgfqpoint{0.702340in}{5.025354in}}%
\pgfpathlineto{\pgfqpoint{0.702340in}{5.066082in}}%
\pgfpathlineto{\pgfqpoint{1.925444in}{5.066082in}}%
\pgfpathlineto{\pgfqpoint{1.925444in}{5.025354in}}%
\pgfpathclose%
\pgfusepath{stroke,fill}%
\end{pgfscope}%
\begin{pgfscope}%
\pgfpathrectangle{\pgfqpoint{0.702340in}{4.812659in}}{\pgfqpoint{1.223103in}{0.607948in}}%
\pgfusepath{clip}%
\pgfsetbuttcap%
\pgfsetroundjoin%
\definecolor{currentfill}{rgb}{0.000000,0.501961,0.000000}%
\pgfsetfillcolor{currentfill}%
\pgfsetfillopacity{0.500000}%
\pgfsetlinewidth{0.803000pt}%
\definecolor{currentstroke}{rgb}{0.000000,0.501961,0.000000}%
\pgfsetstrokecolor{currentstroke}%
\pgfsetstrokeopacity{0.500000}%
\pgfsetdash{}{0pt}%
\pgfpathmoveto{\pgfqpoint{0.702340in}{5.063611in}}%
\pgfpathlineto{\pgfqpoint{0.702340in}{5.026301in}}%
\pgfpathlineto{\pgfqpoint{0.943050in}{5.026724in}}%
\pgfpathlineto{\pgfqpoint{1.054581in}{5.025189in}}%
\pgfpathlineto{\pgfqpoint{1.127977in}{5.021737in}}%
\pgfpathlineto{\pgfqpoint{1.182772in}{5.016410in}}%
\pgfpathlineto{\pgfqpoint{1.226515in}{5.009245in}}%
\pgfpathlineto{\pgfqpoint{1.262923in}{5.000277in}}%
\pgfpathlineto{\pgfqpoint{1.294107in}{4.989542in}}%
\pgfpathlineto{\pgfqpoint{1.321380in}{4.977069in}}%
\pgfpathlineto{\pgfqpoint{1.345614in}{4.962888in}}%
\pgfpathlineto{\pgfqpoint{1.367419in}{4.947019in}}%
\pgfpathlineto{\pgfqpoint{1.387238in}{4.929360in}}%
\pgfpathlineto{\pgfqpoint{1.405403in}{4.909234in}}%
\pgfpathlineto{\pgfqpoint{1.422168in}{4.887553in}}%
\pgfpathlineto{\pgfqpoint{1.437735in}{4.864448in}}%
\pgfpathlineto{\pgfqpoint{1.452262in}{4.839919in}}%
\pgfpathlineto{\pgfqpoint{1.465881in}{4.813967in}}%
\pgfpathlineto{\pgfqpoint{1.478698in}{4.786588in}}%
\pgfpathlineto{\pgfqpoint{1.490802in}{4.757783in}}%
\pgfpathlineto{\pgfqpoint{1.502269in}{4.727551in}}%
\pgfpathlineto{\pgfqpoint{1.513162in}{4.695893in}}%
\pgfpathlineto{\pgfqpoint{1.523536in}{4.662805in}}%
\pgfpathlineto{\pgfqpoint{1.533438in}{4.628285in}}%
\pgfpathlineto{\pgfqpoint{1.542909in}{4.592326in}}%
\pgfpathlineto{\pgfqpoint{1.551986in}{4.554911in}}%
\pgfpathlineto{\pgfqpoint{1.560699in}{4.516009in}}%
\pgfpathlineto{\pgfqpoint{1.569077in}{4.475542in}}%
\pgfpathlineto{\pgfqpoint{1.577145in}{4.433356in}}%
\pgfpathlineto{\pgfqpoint{1.584924in}{4.389343in}}%
\pgfpathlineto{\pgfqpoint{1.592435in}{4.343566in}}%
\pgfpathlineto{\pgfqpoint{1.599696in}{4.296099in}}%
\pgfpathlineto{\pgfqpoint{1.606722in}{4.246970in}}%
\pgfpathlineto{\pgfqpoint{1.613528in}{4.196184in}}%
\pgfpathlineto{\pgfqpoint{1.620129in}{4.143739in}}%
\pgfpathlineto{\pgfqpoint{1.626535in}{4.089625in}}%
\pgfpathlineto{\pgfqpoint{1.632758in}{4.033835in}}%
\pgfpathlineto{\pgfqpoint{1.638808in}{3.976357in}}%
\pgfpathlineto{\pgfqpoint{1.644694in}{3.917183in}}%
\pgfpathlineto{\pgfqpoint{1.650426in}{3.856301in}}%
\pgfpathlineto{\pgfqpoint{1.656011in}{3.793702in}}%
\pgfpathlineto{\pgfqpoint{1.661456in}{3.729375in}}%
\pgfpathlineto{\pgfqpoint{1.666768in}{3.663311in}}%
\pgfpathlineto{\pgfqpoint{1.671953in}{3.595501in}}%
\pgfpathlineto{\pgfqpoint{1.677018in}{3.525935in}}%
\pgfpathlineto{\pgfqpoint{1.681968in}{3.454606in}}%
\pgfpathlineto{\pgfqpoint{1.686808in}{3.381505in}}%
\pgfpathlineto{\pgfqpoint{1.691543in}{3.306622in}}%
\pgfpathlineto{\pgfqpoint{1.696176in}{3.229945in}}%
\pgfpathlineto{\pgfqpoint{1.700714in}{3.151409in}}%
\pgfpathlineto{\pgfqpoint{1.705158in}{3.068576in}}%
\pgfpathlineto{\pgfqpoint{1.705158in}{3.071589in}}%
\pgfpathlineto{\pgfqpoint{1.705158in}{3.071589in}}%
\pgfpathlineto{\pgfqpoint{1.700714in}{3.155454in}}%
\pgfpathlineto{\pgfqpoint{1.696176in}{3.238742in}}%
\pgfpathlineto{\pgfqpoint{1.691543in}{3.319175in}}%
\pgfpathlineto{\pgfqpoint{1.686808in}{3.396838in}}%
\pgfpathlineto{\pgfqpoint{1.681968in}{3.471860in}}%
\pgfpathlineto{\pgfqpoint{1.677018in}{3.544365in}}%
\pgfpathlineto{\pgfqpoint{1.671953in}{3.614470in}}%
\pgfpathlineto{\pgfqpoint{1.666768in}{3.682285in}}%
\pgfpathlineto{\pgfqpoint{1.661456in}{3.747909in}}%
\pgfpathlineto{\pgfqpoint{1.656011in}{3.811437in}}%
\pgfpathlineto{\pgfqpoint{1.650426in}{3.872954in}}%
\pgfpathlineto{\pgfqpoint{1.644694in}{3.932539in}}%
\pgfpathlineto{\pgfqpoint{1.638808in}{3.990263in}}%
\pgfpathlineto{\pgfqpoint{1.632758in}{4.046191in}}%
\pgfpathlineto{\pgfqpoint{1.626535in}{4.100382in}}%
\pgfpathlineto{\pgfqpoint{1.620129in}{4.152890in}}%
\pgfpathlineto{\pgfqpoint{1.613528in}{4.203762in}}%
\pgfpathlineto{\pgfqpoint{1.606722in}{4.253043in}}%
\pgfpathlineto{\pgfqpoint{1.599696in}{4.300775in}}%
\pgfpathlineto{\pgfqpoint{1.592435in}{4.347006in}}%
\pgfpathlineto{\pgfqpoint{1.584924in}{4.391800in}}%
\pgfpathlineto{\pgfqpoint{1.577145in}{4.435264in}}%
\pgfpathlineto{\pgfqpoint{1.569077in}{4.477486in}}%
\pgfpathlineto{\pgfqpoint{1.560699in}{4.518383in}}%
\pgfpathlineto{\pgfqpoint{1.551986in}{4.557821in}}%
\pgfpathlineto{\pgfqpoint{1.542909in}{4.595739in}}%
\pgfpathlineto{\pgfqpoint{1.533438in}{4.632114in}}%
\pgfpathlineto{\pgfqpoint{1.523536in}{4.666942in}}%
\pgfpathlineto{\pgfqpoint{1.513162in}{4.700221in}}%
\pgfpathlineto{\pgfqpoint{1.502269in}{4.731951in}}%
\pgfpathlineto{\pgfqpoint{1.490802in}{4.762132in}}%
\pgfpathlineto{\pgfqpoint{1.478698in}{4.790760in}}%
\pgfpathlineto{\pgfqpoint{1.465881in}{4.817832in}}%
\pgfpathlineto{\pgfqpoint{1.452262in}{4.843342in}}%
\pgfpathlineto{\pgfqpoint{1.437735in}{4.867282in}}%
\pgfpathlineto{\pgfqpoint{1.422168in}{4.889641in}}%
\pgfpathlineto{\pgfqpoint{1.405403in}{4.910411in}}%
\pgfpathlineto{\pgfqpoint{1.387238in}{4.929713in}}%
\pgfpathlineto{\pgfqpoint{1.367419in}{4.948447in}}%
\pgfpathlineto{\pgfqpoint{1.345614in}{4.965923in}}%
\pgfpathlineto{\pgfqpoint{1.321380in}{4.982028in}}%
\pgfpathlineto{\pgfqpoint{1.294107in}{4.996771in}}%
\pgfpathlineto{\pgfqpoint{1.262923in}{5.010168in}}%
\pgfpathlineto{\pgfqpoint{1.226515in}{5.022238in}}%
\pgfpathlineto{\pgfqpoint{1.182772in}{5.033005in}}%
\pgfpathlineto{\pgfqpoint{1.127977in}{5.042495in}}%
\pgfpathlineto{\pgfqpoint{1.054581in}{5.050737in}}%
\pgfpathlineto{\pgfqpoint{0.943050in}{5.057763in}}%
\pgfpathlineto{\pgfqpoint{0.702340in}{5.063611in}}%
\pgfpathclose%
\pgfusepath{stroke,fill}%
\end{pgfscope}%
\begin{pgfscope}%
\pgfpathrectangle{\pgfqpoint{0.702340in}{4.812659in}}{\pgfqpoint{1.223103in}{0.607948in}}%
\pgfusepath{clip}%
\pgfsetroundcap%
\pgfsetroundjoin%
\pgfsetlinewidth{0.501875pt}%
\definecolor{currentstroke}{rgb}{0.000000,0.000000,1.000000}%
\pgfsetstrokecolor{currentstroke}%
\pgfsetstrokeopacity{0.800000}%
\pgfsetdash{}{0pt}%
\pgfpathmoveto{\pgfqpoint{0.702340in}{5.045718in}}%
\pgfpathlineto{\pgfqpoint{1.925444in}{5.045718in}}%
\pgfusepath{stroke}%
\end{pgfscope}%
\begin{pgfscope}%
\pgfpathrectangle{\pgfqpoint{0.702340in}{4.812659in}}{\pgfqpoint{1.223103in}{0.607948in}}%
\pgfusepath{clip}%
\pgfsetbuttcap%
\pgfsetroundjoin%
\pgfsetlinewidth{1.003750pt}%
\definecolor{currentstroke}{rgb}{0.000000,0.000000,0.000000}%
\pgfsetstrokecolor{currentstroke}%
\pgfsetdash{{3.700000pt}{1.600000pt}}{0.000000pt}%
\pgfpathmoveto{\pgfqpoint{0.702340in}{5.045573in}}%
\pgfpathlineto{\pgfqpoint{1.925444in}{5.045573in}}%
\pgfusepath{stroke}%
\end{pgfscope}%
\begin{pgfscope}%
\pgfsetroundcap%
\pgfsetroundjoin%
\pgfsetlinewidth{0.501875pt}%
\definecolor{currentstroke}{rgb}{0.000000,0.000000,1.000000}%
\pgfsetstrokecolor{currentstroke}%
\pgfsetstrokeopacity{0.800000}%
\pgfsetdash{}{0pt}%
\pgfpathmoveto{\pgfqpoint{1.516189in}{5.163054in}}%
\pgfpathquadraticcurveto{\pgfqpoint{1.446671in}{5.112547in}}{\pgfqpoint{1.377153in}{5.062039in}}%
\pgfusepath{stroke}%
\end{pgfscope}%
\begin{pgfscope}%
\pgfsetfillopacity{0.800000}%
\pgfsetstrokeopacity{0.800000}%
\definecolor{textcolor}{rgb}{0.000000,0.000000,1.000000}%
\pgfsetstrokecolor{textcolor}%
\pgfsetfillcolor{textcolor}%
\pgftext[x=1.442982in,y=5.228102in,left,base]{\color{textcolor}\sffamily\fontsize{5.647059}{6.776471}\selectfont 13.383(33)}%
\end{pgfscope}%
\begin{pgfscope}%
\pgfsetbuttcap%
\pgfsetroundjoin%
\definecolor{currentfill}{rgb}{0.150000,0.150000,0.150000}%
\pgfsetfillcolor{currentfill}%
\pgfsetlinewidth{1.003750pt}%
\definecolor{currentstroke}{rgb}{0.150000,0.150000,0.150000}%
\pgfsetstrokecolor{currentstroke}%
\pgfsetdash{}{0pt}%
\pgfsys@defobject{currentmarker}{\pgfqpoint{0.000000in}{-0.066667in}}{\pgfqpoint{0.000000in}{0.000000in}}{%
\pgfpathmoveto{\pgfqpoint{0.000000in}{0.000000in}}%
\pgfpathlineto{\pgfqpoint{0.000000in}{-0.066667in}}%
\pgfusepath{stroke,fill}%
}%
\begin{pgfscope}%
\pgfsys@transformshift{0.702340in}{4.812659in}%
\pgfsys@useobject{currentmarker}{}%
\end{pgfscope}%
\end{pgfscope}%
\begin{pgfscope}%
\pgfsetbuttcap%
\pgfsetroundjoin%
\definecolor{currentfill}{rgb}{0.150000,0.150000,0.150000}%
\pgfsetfillcolor{currentfill}%
\pgfsetlinewidth{1.003750pt}%
\definecolor{currentstroke}{rgb}{0.150000,0.150000,0.150000}%
\pgfsetstrokecolor{currentstroke}%
\pgfsetdash{}{0pt}%
\pgfsys@defobject{currentmarker}{\pgfqpoint{0.000000in}{-0.066667in}}{\pgfqpoint{0.000000in}{0.000000in}}{%
\pgfpathmoveto{\pgfqpoint{0.000000in}{0.000000in}}%
\pgfpathlineto{\pgfqpoint{0.000000in}{-0.066667in}}%
\pgfusepath{stroke,fill}%
}%
\begin{pgfscope}%
\pgfsys@transformshift{1.203749in}{4.812659in}%
\pgfsys@useobject{currentmarker}{}%
\end{pgfscope}%
\end{pgfscope}%
\begin{pgfscope}%
\pgfsetbuttcap%
\pgfsetroundjoin%
\definecolor{currentfill}{rgb}{0.150000,0.150000,0.150000}%
\pgfsetfillcolor{currentfill}%
\pgfsetlinewidth{1.003750pt}%
\definecolor{currentstroke}{rgb}{0.150000,0.150000,0.150000}%
\pgfsetstrokecolor{currentstroke}%
\pgfsetdash{}{0pt}%
\pgfsys@defobject{currentmarker}{\pgfqpoint{0.000000in}{-0.066667in}}{\pgfqpoint{0.000000in}{0.000000in}}{%
\pgfpathmoveto{\pgfqpoint{0.000000in}{0.000000in}}%
\pgfpathlineto{\pgfqpoint{0.000000in}{-0.066667in}}%
\pgfusepath{stroke,fill}%
}%
\begin{pgfscope}%
\pgfsys@transformshift{1.705158in}{4.812659in}%
\pgfsys@useobject{currentmarker}{}%
\end{pgfscope}%
\end{pgfscope}%
\begin{pgfscope}%
\pgfsetbuttcap%
\pgfsetroundjoin%
\definecolor{currentfill}{rgb}{0.150000,0.150000,0.150000}%
\pgfsetfillcolor{currentfill}%
\pgfsetlinewidth{0.803000pt}%
\definecolor{currentstroke}{rgb}{0.150000,0.150000,0.150000}%
\pgfsetstrokecolor{currentstroke}%
\pgfsetdash{}{0pt}%
\pgfsys@defobject{currentmarker}{\pgfqpoint{0.000000in}{-0.044444in}}{\pgfqpoint{0.000000in}{0.000000in}}{%
\pgfpathmoveto{\pgfqpoint{0.000000in}{0.000000in}}%
\pgfpathlineto{\pgfqpoint{0.000000in}{-0.044444in}}%
\pgfusepath{stroke,fill}%
}%
\begin{pgfscope}%
\pgfsys@transformshift{0.853280in}{4.812659in}%
\pgfsys@useobject{currentmarker}{}%
\end{pgfscope}%
\end{pgfscope}%
\begin{pgfscope}%
\pgfsetbuttcap%
\pgfsetroundjoin%
\definecolor{currentfill}{rgb}{0.150000,0.150000,0.150000}%
\pgfsetfillcolor{currentfill}%
\pgfsetlinewidth{0.803000pt}%
\definecolor{currentstroke}{rgb}{0.150000,0.150000,0.150000}%
\pgfsetstrokecolor{currentstroke}%
\pgfsetdash{}{0pt}%
\pgfsys@defobject{currentmarker}{\pgfqpoint{0.000000in}{-0.044444in}}{\pgfqpoint{0.000000in}{0.000000in}}{%
\pgfpathmoveto{\pgfqpoint{0.000000in}{0.000000in}}%
\pgfpathlineto{\pgfqpoint{0.000000in}{-0.044444in}}%
\pgfusepath{stroke,fill}%
}%
\begin{pgfscope}%
\pgfsys@transformshift{0.941573in}{4.812659in}%
\pgfsys@useobject{currentmarker}{}%
\end{pgfscope}%
\end{pgfscope}%
\begin{pgfscope}%
\pgfsetbuttcap%
\pgfsetroundjoin%
\definecolor{currentfill}{rgb}{0.150000,0.150000,0.150000}%
\pgfsetfillcolor{currentfill}%
\pgfsetlinewidth{0.803000pt}%
\definecolor{currentstroke}{rgb}{0.150000,0.150000,0.150000}%
\pgfsetstrokecolor{currentstroke}%
\pgfsetdash{}{0pt}%
\pgfsys@defobject{currentmarker}{\pgfqpoint{0.000000in}{-0.044444in}}{\pgfqpoint{0.000000in}{0.000000in}}{%
\pgfpathmoveto{\pgfqpoint{0.000000in}{0.000000in}}%
\pgfpathlineto{\pgfqpoint{0.000000in}{-0.044444in}}%
\pgfusepath{stroke,fill}%
}%
\begin{pgfscope}%
\pgfsys@transformshift{1.004219in}{4.812659in}%
\pgfsys@useobject{currentmarker}{}%
\end{pgfscope}%
\end{pgfscope}%
\begin{pgfscope}%
\pgfsetbuttcap%
\pgfsetroundjoin%
\definecolor{currentfill}{rgb}{0.150000,0.150000,0.150000}%
\pgfsetfillcolor{currentfill}%
\pgfsetlinewidth{0.803000pt}%
\definecolor{currentstroke}{rgb}{0.150000,0.150000,0.150000}%
\pgfsetstrokecolor{currentstroke}%
\pgfsetdash{}{0pt}%
\pgfsys@defobject{currentmarker}{\pgfqpoint{0.000000in}{-0.044444in}}{\pgfqpoint{0.000000in}{0.000000in}}{%
\pgfpathmoveto{\pgfqpoint{0.000000in}{0.000000in}}%
\pgfpathlineto{\pgfqpoint{0.000000in}{-0.044444in}}%
\pgfusepath{stroke,fill}%
}%
\begin{pgfscope}%
\pgfsys@transformshift{1.052810in}{4.812659in}%
\pgfsys@useobject{currentmarker}{}%
\end{pgfscope}%
\end{pgfscope}%
\begin{pgfscope}%
\pgfsetbuttcap%
\pgfsetroundjoin%
\definecolor{currentfill}{rgb}{0.150000,0.150000,0.150000}%
\pgfsetfillcolor{currentfill}%
\pgfsetlinewidth{0.803000pt}%
\definecolor{currentstroke}{rgb}{0.150000,0.150000,0.150000}%
\pgfsetstrokecolor{currentstroke}%
\pgfsetdash{}{0pt}%
\pgfsys@defobject{currentmarker}{\pgfqpoint{0.000000in}{-0.044444in}}{\pgfqpoint{0.000000in}{0.000000in}}{%
\pgfpathmoveto{\pgfqpoint{0.000000in}{0.000000in}}%
\pgfpathlineto{\pgfqpoint{0.000000in}{-0.044444in}}%
\pgfusepath{stroke,fill}%
}%
\begin{pgfscope}%
\pgfsys@transformshift{1.092512in}{4.812659in}%
\pgfsys@useobject{currentmarker}{}%
\end{pgfscope}%
\end{pgfscope}%
\begin{pgfscope}%
\pgfsetbuttcap%
\pgfsetroundjoin%
\definecolor{currentfill}{rgb}{0.150000,0.150000,0.150000}%
\pgfsetfillcolor{currentfill}%
\pgfsetlinewidth{0.803000pt}%
\definecolor{currentstroke}{rgb}{0.150000,0.150000,0.150000}%
\pgfsetstrokecolor{currentstroke}%
\pgfsetdash{}{0pt}%
\pgfsys@defobject{currentmarker}{\pgfqpoint{0.000000in}{-0.044444in}}{\pgfqpoint{0.000000in}{0.000000in}}{%
\pgfpathmoveto{\pgfqpoint{0.000000in}{0.000000in}}%
\pgfpathlineto{\pgfqpoint{0.000000in}{-0.044444in}}%
\pgfusepath{stroke,fill}%
}%
\begin{pgfscope}%
\pgfsys@transformshift{1.126080in}{4.812659in}%
\pgfsys@useobject{currentmarker}{}%
\end{pgfscope}%
\end{pgfscope}%
\begin{pgfscope}%
\pgfsetbuttcap%
\pgfsetroundjoin%
\definecolor{currentfill}{rgb}{0.150000,0.150000,0.150000}%
\pgfsetfillcolor{currentfill}%
\pgfsetlinewidth{0.803000pt}%
\definecolor{currentstroke}{rgb}{0.150000,0.150000,0.150000}%
\pgfsetstrokecolor{currentstroke}%
\pgfsetdash{}{0pt}%
\pgfsys@defobject{currentmarker}{\pgfqpoint{0.000000in}{-0.044444in}}{\pgfqpoint{0.000000in}{0.000000in}}{%
\pgfpathmoveto{\pgfqpoint{0.000000in}{0.000000in}}%
\pgfpathlineto{\pgfqpoint{0.000000in}{-0.044444in}}%
\pgfusepath{stroke,fill}%
}%
\begin{pgfscope}%
\pgfsys@transformshift{1.155158in}{4.812659in}%
\pgfsys@useobject{currentmarker}{}%
\end{pgfscope}%
\end{pgfscope}%
\begin{pgfscope}%
\pgfsetbuttcap%
\pgfsetroundjoin%
\definecolor{currentfill}{rgb}{0.150000,0.150000,0.150000}%
\pgfsetfillcolor{currentfill}%
\pgfsetlinewidth{0.803000pt}%
\definecolor{currentstroke}{rgb}{0.150000,0.150000,0.150000}%
\pgfsetstrokecolor{currentstroke}%
\pgfsetdash{}{0pt}%
\pgfsys@defobject{currentmarker}{\pgfqpoint{0.000000in}{-0.044444in}}{\pgfqpoint{0.000000in}{0.000000in}}{%
\pgfpathmoveto{\pgfqpoint{0.000000in}{0.000000in}}%
\pgfpathlineto{\pgfqpoint{0.000000in}{-0.044444in}}%
\pgfusepath{stroke,fill}%
}%
\begin{pgfscope}%
\pgfsys@transformshift{1.180806in}{4.812659in}%
\pgfsys@useobject{currentmarker}{}%
\end{pgfscope}%
\end{pgfscope}%
\begin{pgfscope}%
\pgfsetbuttcap%
\pgfsetroundjoin%
\definecolor{currentfill}{rgb}{0.150000,0.150000,0.150000}%
\pgfsetfillcolor{currentfill}%
\pgfsetlinewidth{0.803000pt}%
\definecolor{currentstroke}{rgb}{0.150000,0.150000,0.150000}%
\pgfsetstrokecolor{currentstroke}%
\pgfsetdash{}{0pt}%
\pgfsys@defobject{currentmarker}{\pgfqpoint{0.000000in}{-0.044444in}}{\pgfqpoint{0.000000in}{0.000000in}}{%
\pgfpathmoveto{\pgfqpoint{0.000000in}{0.000000in}}%
\pgfpathlineto{\pgfqpoint{0.000000in}{-0.044444in}}%
\pgfusepath{stroke,fill}%
}%
\begin{pgfscope}%
\pgfsys@transformshift{1.354689in}{4.812659in}%
\pgfsys@useobject{currentmarker}{}%
\end{pgfscope}%
\end{pgfscope}%
\begin{pgfscope}%
\pgfsetbuttcap%
\pgfsetroundjoin%
\definecolor{currentfill}{rgb}{0.150000,0.150000,0.150000}%
\pgfsetfillcolor{currentfill}%
\pgfsetlinewidth{0.803000pt}%
\definecolor{currentstroke}{rgb}{0.150000,0.150000,0.150000}%
\pgfsetstrokecolor{currentstroke}%
\pgfsetdash{}{0pt}%
\pgfsys@defobject{currentmarker}{\pgfqpoint{0.000000in}{-0.044444in}}{\pgfqpoint{0.000000in}{0.000000in}}{%
\pgfpathmoveto{\pgfqpoint{0.000000in}{0.000000in}}%
\pgfpathlineto{\pgfqpoint{0.000000in}{-0.044444in}}%
\pgfusepath{stroke,fill}%
}%
\begin{pgfscope}%
\pgfsys@transformshift{1.442982in}{4.812659in}%
\pgfsys@useobject{currentmarker}{}%
\end{pgfscope}%
\end{pgfscope}%
\begin{pgfscope}%
\pgfsetbuttcap%
\pgfsetroundjoin%
\definecolor{currentfill}{rgb}{0.150000,0.150000,0.150000}%
\pgfsetfillcolor{currentfill}%
\pgfsetlinewidth{0.803000pt}%
\definecolor{currentstroke}{rgb}{0.150000,0.150000,0.150000}%
\pgfsetstrokecolor{currentstroke}%
\pgfsetdash{}{0pt}%
\pgfsys@defobject{currentmarker}{\pgfqpoint{0.000000in}{-0.044444in}}{\pgfqpoint{0.000000in}{0.000000in}}{%
\pgfpathmoveto{\pgfqpoint{0.000000in}{0.000000in}}%
\pgfpathlineto{\pgfqpoint{0.000000in}{-0.044444in}}%
\pgfusepath{stroke,fill}%
}%
\begin{pgfscope}%
\pgfsys@transformshift{1.505628in}{4.812659in}%
\pgfsys@useobject{currentmarker}{}%
\end{pgfscope}%
\end{pgfscope}%
\begin{pgfscope}%
\pgfsetbuttcap%
\pgfsetroundjoin%
\definecolor{currentfill}{rgb}{0.150000,0.150000,0.150000}%
\pgfsetfillcolor{currentfill}%
\pgfsetlinewidth{0.803000pt}%
\definecolor{currentstroke}{rgb}{0.150000,0.150000,0.150000}%
\pgfsetstrokecolor{currentstroke}%
\pgfsetdash{}{0pt}%
\pgfsys@defobject{currentmarker}{\pgfqpoint{0.000000in}{-0.044444in}}{\pgfqpoint{0.000000in}{0.000000in}}{%
\pgfpathmoveto{\pgfqpoint{0.000000in}{0.000000in}}%
\pgfpathlineto{\pgfqpoint{0.000000in}{-0.044444in}}%
\pgfusepath{stroke,fill}%
}%
\begin{pgfscope}%
\pgfsys@transformshift{1.554219in}{4.812659in}%
\pgfsys@useobject{currentmarker}{}%
\end{pgfscope}%
\end{pgfscope}%
\begin{pgfscope}%
\pgfsetbuttcap%
\pgfsetroundjoin%
\definecolor{currentfill}{rgb}{0.150000,0.150000,0.150000}%
\pgfsetfillcolor{currentfill}%
\pgfsetlinewidth{0.803000pt}%
\definecolor{currentstroke}{rgb}{0.150000,0.150000,0.150000}%
\pgfsetstrokecolor{currentstroke}%
\pgfsetdash{}{0pt}%
\pgfsys@defobject{currentmarker}{\pgfqpoint{0.000000in}{-0.044444in}}{\pgfqpoint{0.000000in}{0.000000in}}{%
\pgfpathmoveto{\pgfqpoint{0.000000in}{0.000000in}}%
\pgfpathlineto{\pgfqpoint{0.000000in}{-0.044444in}}%
\pgfusepath{stroke,fill}%
}%
\begin{pgfscope}%
\pgfsys@transformshift{1.593921in}{4.812659in}%
\pgfsys@useobject{currentmarker}{}%
\end{pgfscope}%
\end{pgfscope}%
\begin{pgfscope}%
\pgfsetbuttcap%
\pgfsetroundjoin%
\definecolor{currentfill}{rgb}{0.150000,0.150000,0.150000}%
\pgfsetfillcolor{currentfill}%
\pgfsetlinewidth{0.803000pt}%
\definecolor{currentstroke}{rgb}{0.150000,0.150000,0.150000}%
\pgfsetstrokecolor{currentstroke}%
\pgfsetdash{}{0pt}%
\pgfsys@defobject{currentmarker}{\pgfqpoint{0.000000in}{-0.044444in}}{\pgfqpoint{0.000000in}{0.000000in}}{%
\pgfpathmoveto{\pgfqpoint{0.000000in}{0.000000in}}%
\pgfpathlineto{\pgfqpoint{0.000000in}{-0.044444in}}%
\pgfusepath{stroke,fill}%
}%
\begin{pgfscope}%
\pgfsys@transformshift{1.627489in}{4.812659in}%
\pgfsys@useobject{currentmarker}{}%
\end{pgfscope}%
\end{pgfscope}%
\begin{pgfscope}%
\pgfsetbuttcap%
\pgfsetroundjoin%
\definecolor{currentfill}{rgb}{0.150000,0.150000,0.150000}%
\pgfsetfillcolor{currentfill}%
\pgfsetlinewidth{0.803000pt}%
\definecolor{currentstroke}{rgb}{0.150000,0.150000,0.150000}%
\pgfsetstrokecolor{currentstroke}%
\pgfsetdash{}{0pt}%
\pgfsys@defobject{currentmarker}{\pgfqpoint{0.000000in}{-0.044444in}}{\pgfqpoint{0.000000in}{0.000000in}}{%
\pgfpathmoveto{\pgfqpoint{0.000000in}{0.000000in}}%
\pgfpathlineto{\pgfqpoint{0.000000in}{-0.044444in}}%
\pgfusepath{stroke,fill}%
}%
\begin{pgfscope}%
\pgfsys@transformshift{1.656567in}{4.812659in}%
\pgfsys@useobject{currentmarker}{}%
\end{pgfscope}%
\end{pgfscope}%
\begin{pgfscope}%
\pgfsetbuttcap%
\pgfsetroundjoin%
\definecolor{currentfill}{rgb}{0.150000,0.150000,0.150000}%
\pgfsetfillcolor{currentfill}%
\pgfsetlinewidth{0.803000pt}%
\definecolor{currentstroke}{rgb}{0.150000,0.150000,0.150000}%
\pgfsetstrokecolor{currentstroke}%
\pgfsetdash{}{0pt}%
\pgfsys@defobject{currentmarker}{\pgfqpoint{0.000000in}{-0.044444in}}{\pgfqpoint{0.000000in}{0.000000in}}{%
\pgfpathmoveto{\pgfqpoint{0.000000in}{0.000000in}}%
\pgfpathlineto{\pgfqpoint{0.000000in}{-0.044444in}}%
\pgfusepath{stroke,fill}%
}%
\begin{pgfscope}%
\pgfsys@transformshift{1.682215in}{4.812659in}%
\pgfsys@useobject{currentmarker}{}%
\end{pgfscope}%
\end{pgfscope}%
\begin{pgfscope}%
\pgfsetbuttcap%
\pgfsetroundjoin%
\definecolor{currentfill}{rgb}{0.150000,0.150000,0.150000}%
\pgfsetfillcolor{currentfill}%
\pgfsetlinewidth{0.803000pt}%
\definecolor{currentstroke}{rgb}{0.150000,0.150000,0.150000}%
\pgfsetstrokecolor{currentstroke}%
\pgfsetdash{}{0pt}%
\pgfsys@defobject{currentmarker}{\pgfqpoint{0.000000in}{-0.044444in}}{\pgfqpoint{0.000000in}{0.000000in}}{%
\pgfpathmoveto{\pgfqpoint{0.000000in}{0.000000in}}%
\pgfpathlineto{\pgfqpoint{0.000000in}{-0.044444in}}%
\pgfusepath{stroke,fill}%
}%
\begin{pgfscope}%
\pgfsys@transformshift{1.856098in}{4.812659in}%
\pgfsys@useobject{currentmarker}{}%
\end{pgfscope}%
\end{pgfscope}%
\begin{pgfscope}%
\pgfsetbuttcap%
\pgfsetroundjoin%
\definecolor{currentfill}{rgb}{0.150000,0.150000,0.150000}%
\pgfsetfillcolor{currentfill}%
\pgfsetlinewidth{1.003750pt}%
\definecolor{currentstroke}{rgb}{0.150000,0.150000,0.150000}%
\pgfsetstrokecolor{currentstroke}%
\pgfsetdash{}{0pt}%
\pgfsys@defobject{currentmarker}{\pgfqpoint{-0.066667in}{0.000000in}}{\pgfqpoint{0.000000in}{0.000000in}}{%
\pgfpathmoveto{\pgfqpoint{0.000000in}{0.000000in}}%
\pgfpathlineto{\pgfqpoint{-0.066667in}{0.000000in}}%
\pgfusepath{stroke,fill}%
}%
\begin{pgfscope}%
\pgfsys@transformshift{0.702340in}{4.812659in}%
\pgfsys@useobject{currentmarker}{}%
\end{pgfscope}%
\end{pgfscope}%
\begin{pgfscope}%
\definecolor{textcolor}{rgb}{0.150000,0.150000,0.150000}%
\pgfsetstrokecolor{textcolor}%
\pgfsetfillcolor{textcolor}%
\pgftext[x=0.374971in,y=4.787711in,left,base]{\color{textcolor}\sffamily\fontsize{5.176471}{6.211765}\selectfont 13.000}%
\end{pgfscope}%
\begin{pgfscope}%
\pgfsetbuttcap%
\pgfsetroundjoin%
\definecolor{currentfill}{rgb}{0.150000,0.150000,0.150000}%
\pgfsetfillcolor{currentfill}%
\pgfsetlinewidth{1.003750pt}%
\definecolor{currentstroke}{rgb}{0.150000,0.150000,0.150000}%
\pgfsetstrokecolor{currentstroke}%
\pgfsetdash{}{0pt}%
\pgfsys@defobject{currentmarker}{\pgfqpoint{-0.066667in}{0.000000in}}{\pgfqpoint{0.000000in}{0.000000in}}{%
\pgfpathmoveto{\pgfqpoint{0.000000in}{0.000000in}}%
\pgfpathlineto{\pgfqpoint{-0.066667in}{0.000000in}}%
\pgfusepath{stroke,fill}%
}%
\begin{pgfscope}%
\pgfsys@transformshift{0.702340in}{5.045573in}%
\pgfsys@useobject{currentmarker}{}%
\end{pgfscope}%
\end{pgfscope}%
\begin{pgfscope}%
\definecolor{textcolor}{rgb}{0.150000,0.150000,0.150000}%
\pgfsetstrokecolor{textcolor}%
\pgfsetfillcolor{textcolor}%
\pgftext[x=0.374971in,y=5.020625in,left,base]{\color{textcolor}\sffamily\fontsize{5.176471}{6.211765}\selectfont 13.383}%
\end{pgfscope}%
\begin{pgfscope}%
\pgfsetbuttcap%
\pgfsetroundjoin%
\definecolor{currentfill}{rgb}{0.150000,0.150000,0.150000}%
\pgfsetfillcolor{currentfill}%
\pgfsetlinewidth{1.003750pt}%
\definecolor{currentstroke}{rgb}{0.150000,0.150000,0.150000}%
\pgfsetstrokecolor{currentstroke}%
\pgfsetdash{}{0pt}%
\pgfsys@defobject{currentmarker}{\pgfqpoint{-0.066667in}{0.000000in}}{\pgfqpoint{0.000000in}{0.000000in}}{%
\pgfpathmoveto{\pgfqpoint{0.000000in}{0.000000in}}%
\pgfpathlineto{\pgfqpoint{-0.066667in}{0.000000in}}%
\pgfusepath{stroke,fill}%
}%
\begin{pgfscope}%
\pgfsys@transformshift{0.702340in}{5.420607in}%
\pgfsys@useobject{currentmarker}{}%
\end{pgfscope}%
\end{pgfscope}%
\begin{pgfscope}%
\definecolor{textcolor}{rgb}{0.150000,0.150000,0.150000}%
\pgfsetstrokecolor{textcolor}%
\pgfsetfillcolor{textcolor}%
\pgftext[x=0.374971in,y=5.395659in,left,base]{\color{textcolor}\sffamily\fontsize{5.176471}{6.211765}\selectfont 14.000}%
\end{pgfscope}%
\begin{pgfscope}%
\definecolor{textcolor}{rgb}{0.150000,0.150000,0.150000}%
\pgfsetstrokecolor{textcolor}%
\pgfsetfillcolor{textcolor}%
\pgftext[x=0.319416in,y=5.116633in,,bottom,rotate=90.000000]{\color{textcolor}\sffamily\fontsize{5.647059}{6.776471}\selectfont \(\displaystyle x = \frac{2 \mu E L^2}{4 \pi^2}\)}%
\end{pgfscope}%
\begin{pgfscope}%
\pgfpathrectangle{\pgfqpoint{0.702340in}{4.812659in}}{\pgfqpoint{1.223103in}{0.607948in}}%
\pgfusepath{clip}%
\pgfsetroundcap%
\pgfsetroundjoin%
\pgfsetlinewidth{1.204500pt}%
\definecolor{currentstroke}{rgb}{0.000000,0.501961,0.000000}%
\pgfsetstrokecolor{currentstroke}%
\pgfsetdash{}{0pt}%
\pgfpathmoveto{\pgfqpoint{0.702340in}{5.044956in}}%
\pgfpathlineto{\pgfqpoint{0.943050in}{5.042244in}}%
\pgfpathlineto{\pgfqpoint{1.054581in}{5.037963in}}%
\pgfpathlineto{\pgfqpoint{1.127977in}{5.032116in}}%
\pgfpathlineto{\pgfqpoint{1.182772in}{5.024708in}}%
\pgfpathlineto{\pgfqpoint{1.226515in}{5.015741in}}%
\pgfpathlineto{\pgfqpoint{1.262923in}{5.005223in}}%
\pgfpathlineto{\pgfqpoint{1.294107in}{4.993156in}}%
\pgfpathlineto{\pgfqpoint{1.321380in}{4.979549in}}%
\pgfpathlineto{\pgfqpoint{1.345614in}{4.964406in}}%
\pgfpathlineto{\pgfqpoint{1.367419in}{4.947733in}}%
\pgfpathlineto{\pgfqpoint{1.387238in}{4.929537in}}%
\pgfpathlineto{\pgfqpoint{1.405403in}{4.909823in}}%
\pgfpathlineto{\pgfqpoint{1.422168in}{4.888597in}}%
\pgfpathlineto{\pgfqpoint{1.437735in}{4.865865in}}%
\pgfpathlineto{\pgfqpoint{1.452262in}{4.841631in}}%
\pgfpathlineto{\pgfqpoint{1.465881in}{4.815899in}}%
\pgfpathlineto{\pgfqpoint{1.468976in}{4.809326in}}%
\pgfusepath{stroke}%
\end{pgfscope}%
\begin{pgfscope}%
\pgfsetrectcap%
\pgfsetmiterjoin%
\pgfsetlinewidth{1.003750pt}%
\definecolor{currentstroke}{rgb}{0.150000,0.150000,0.150000}%
\pgfsetstrokecolor{currentstroke}%
\pgfsetdash{}{0pt}%
\pgfpathmoveto{\pgfqpoint{0.702340in}{4.812659in}}%
\pgfpathlineto{\pgfqpoint{0.702340in}{5.420607in}}%
\pgfusepath{stroke}%
\end{pgfscope}%
\begin{pgfscope}%
\pgfsetrectcap%
\pgfsetmiterjoin%
\pgfsetlinewidth{1.003750pt}%
\definecolor{currentstroke}{rgb}{0.150000,0.150000,0.150000}%
\pgfsetstrokecolor{currentstroke}%
\pgfsetdash{}{0pt}%
\pgfpathmoveto{\pgfqpoint{0.702340in}{4.812659in}}%
\pgfpathlineto{\pgfqpoint{1.925444in}{4.812659in}}%
\pgfusepath{stroke}%
\end{pgfscope}%
\begin{pgfscope}%
\pgfpathrectangle{\pgfqpoint{0.702340in}{4.812659in}}{\pgfqpoint{1.223103in}{0.607948in}}%
\pgfusepath{clip}%
\pgfsetbuttcap%
\pgfsetroundjoin%
\definecolor{currentfill}{rgb}{0.000000,0.000000,0.000000}%
\pgfsetfillcolor{currentfill}%
\pgfsetlinewidth{1.003750pt}%
\definecolor{currentstroke}{rgb}{0.000000,0.000000,0.000000}%
\pgfsetstrokecolor{currentstroke}%
\pgfsetdash{}{0pt}%
\pgfsys@defobject{currentmarker}{\pgfqpoint{-0.013889in}{-0.013889in}}{\pgfqpoint{0.013889in}{0.013889in}}{%
\pgfpathmoveto{\pgfqpoint{0.000000in}{-0.013889in}}%
\pgfpathcurveto{\pgfqpoint{0.003683in}{-0.013889in}}{\pgfqpoint{0.007216in}{-0.012425in}}{\pgfqpoint{0.009821in}{-0.009821in}}%
\pgfpathcurveto{\pgfqpoint{0.012425in}{-0.007216in}}{\pgfqpoint{0.013889in}{-0.003683in}}{\pgfqpoint{0.013889in}{0.000000in}}%
\pgfpathcurveto{\pgfqpoint{0.013889in}{0.003683in}}{\pgfqpoint{0.012425in}{0.007216in}}{\pgfqpoint{0.009821in}{0.009821in}}%
\pgfpathcurveto{\pgfqpoint{0.007216in}{0.012425in}}{\pgfqpoint{0.003683in}{0.013889in}}{\pgfqpoint{0.000000in}{0.013889in}}%
\pgfpathcurveto{\pgfqpoint{-0.003683in}{0.013889in}}{\pgfqpoint{-0.007216in}{0.012425in}}{\pgfqpoint{-0.009821in}{0.009821in}}%
\pgfpathcurveto{\pgfqpoint{-0.012425in}{0.007216in}}{\pgfqpoint{-0.013889in}{0.003683in}}{\pgfqpoint{-0.013889in}{0.000000in}}%
\pgfpathcurveto{\pgfqpoint{-0.013889in}{-0.003683in}}{\pgfqpoint{-0.012425in}{-0.007216in}}{\pgfqpoint{-0.009821in}{-0.009821in}}%
\pgfpathcurveto{\pgfqpoint{-0.007216in}{-0.012425in}}{\pgfqpoint{-0.003683in}{-0.013889in}}{\pgfqpoint{0.000000in}{-0.013889in}}%
\pgfpathclose%
\pgfusepath{stroke,fill}%
}%
\begin{pgfscope}%
\pgfsys@transformshift{1.705158in}{3.069136in}%
\pgfsys@useobject{currentmarker}{}%
\end{pgfscope}%
\begin{pgfscope}%
\pgfsys@transformshift{1.616865in}{4.177188in}%
\pgfsys@useobject{currentmarker}{}%
\end{pgfscope}%
\begin{pgfscope}%
\pgfsys@transformshift{1.554219in}{4.545174in}%
\pgfsys@useobject{currentmarker}{}%
\end{pgfscope}%
\begin{pgfscope}%
\pgfsys@transformshift{1.505628in}{4.719002in}%
\pgfsys@useobject{currentmarker}{}%
\end{pgfscope}%
\begin{pgfscope}%
\pgfsys@transformshift{1.465926in}{4.815100in}%
\pgfsys@useobject{currentmarker}{}%
\end{pgfscope}%
\begin{pgfscope}%
\pgfsys@transformshift{1.432358in}{4.873804in}%
\pgfsys@useobject{currentmarker}{}%
\end{pgfscope}%
\begin{pgfscope}%
\pgfsys@transformshift{1.403280in}{4.912307in}%
\pgfsys@useobject{currentmarker}{}%
\end{pgfscope}%
\begin{pgfscope}%
\pgfsys@transformshift{1.397903in}{4.918403in}%
\pgfsys@useobject{currentmarker}{}%
\end{pgfscope}%
\begin{pgfscope}%
\pgfsys@transformshift{1.392656in}{4.924079in}%
\pgfsys@useobject{currentmarker}{}%
\end{pgfscope}%
\begin{pgfscope}%
\pgfsys@transformshift{1.387532in}{4.929372in}%
\pgfsys@useobject{currentmarker}{}%
\end{pgfscope}%
\begin{pgfscope}%
\pgfsys@transformshift{1.382525in}{4.934317in}%
\pgfsys@useobject{currentmarker}{}%
\end{pgfscope}%
\begin{pgfscope}%
\pgfsys@transformshift{1.377632in}{4.938944in}%
\pgfsys@useobject{currentmarker}{}%
\end{pgfscope}%
\begin{pgfscope}%
\pgfsys@transformshift{1.372846in}{4.943278in}%
\pgfsys@useobject{currentmarker}{}%
\end{pgfscope}%
\begin{pgfscope}%
\pgfsys@transformshift{1.368163in}{4.947346in}%
\pgfsys@useobject{currentmarker}{}%
\end{pgfscope}%
\begin{pgfscope}%
\pgfsys@transformshift{1.363578in}{4.951168in}%
\pgfsys@useobject{currentmarker}{}%
\end{pgfscope}%
\begin{pgfscope}%
\pgfsys@transformshift{1.359088in}{4.954764in}%
\pgfsys@useobject{currentmarker}{}%
\end{pgfscope}%
\begin{pgfscope}%
\pgfsys@transformshift{1.354689in}{4.958151in}%
\pgfsys@useobject{currentmarker}{}%
\end{pgfscope}%
\end{pgfscope}%
\begin{pgfscope}%
\pgfsetbuttcap%
\pgfsetmiterjoin%
\definecolor{currentfill}{rgb}{1.000000,1.000000,1.000000}%
\pgfsetfillcolor{currentfill}%
\pgfsetlinewidth{0.000000pt}%
\definecolor{currentstroke}{rgb}{0.000000,0.000000,0.000000}%
\pgfsetstrokecolor{currentstroke}%
\pgfsetstrokeopacity{0.000000}%
\pgfsetdash{}{0pt}%
\pgfpathmoveto{\pgfqpoint{2.170064in}{4.812659in}}%
\pgfpathlineto{\pgfqpoint{3.393168in}{4.812659in}}%
\pgfpathlineto{\pgfqpoint{3.393168in}{5.420607in}}%
\pgfpathlineto{\pgfqpoint{2.170064in}{5.420607in}}%
\pgfpathclose%
\pgfusepath{fill}%
\end{pgfscope}%
\begin{pgfscope}%
\pgfpathrectangle{\pgfqpoint{2.170064in}{4.812659in}}{\pgfqpoint{1.223103in}{0.607948in}}%
\pgfusepath{clip}%
\pgfsetbuttcap%
\pgfsetmiterjoin%
\definecolor{currentfill}{rgb}{0.000000,0.000000,1.000000}%
\pgfsetfillcolor{currentfill}%
\pgfsetfillopacity{0.100000}%
\pgfsetlinewidth{0.803000pt}%
\definecolor{currentstroke}{rgb}{0.000000,0.000000,1.000000}%
\pgfsetstrokecolor{currentstroke}%
\pgfsetstrokeopacity{0.100000}%
\pgfsetdash{}{0pt}%
\pgfpathmoveto{\pgfqpoint{2.170064in}{4.912654in}}%
\pgfpathlineto{\pgfqpoint{2.170064in}{5.116352in}}%
\pgfpathlineto{\pgfqpoint{3.393168in}{5.116352in}}%
\pgfpathlineto{\pgfqpoint{3.393168in}{4.912654in}}%
\pgfpathclose%
\pgfusepath{stroke,fill}%
\end{pgfscope}%
\begin{pgfscope}%
\pgfpathrectangle{\pgfqpoint{2.170064in}{4.812659in}}{\pgfqpoint{1.223103in}{0.607948in}}%
\pgfusepath{clip}%
\pgfsetbuttcap%
\pgfsetroundjoin%
\definecolor{currentfill}{rgb}{0.000000,0.501961,0.000000}%
\pgfsetfillcolor{currentfill}%
\pgfsetfillopacity{0.500000}%
\pgfsetlinewidth{0.803000pt}%
\definecolor{currentstroke}{rgb}{0.000000,0.501961,0.000000}%
\pgfsetstrokecolor{currentstroke}%
\pgfsetstrokeopacity{0.500000}%
\pgfsetdash{}{0pt}%
\pgfpathmoveto{\pgfqpoint{2.170064in}{5.110752in}}%
\pgfpathlineto{\pgfqpoint{2.170064in}{4.921975in}}%
\pgfpathlineto{\pgfqpoint{2.410774in}{4.939809in}}%
\pgfpathlineto{\pgfqpoint{2.522305in}{4.956339in}}%
\pgfpathlineto{\pgfqpoint{2.595701in}{4.971598in}}%
\pgfpathlineto{\pgfqpoint{2.650497in}{4.985612in}}%
\pgfpathlineto{\pgfqpoint{2.694239in}{4.998403in}}%
\pgfpathlineto{\pgfqpoint{2.730647in}{5.009988in}}%
\pgfpathlineto{\pgfqpoint{2.761831in}{5.020381in}}%
\pgfpathlineto{\pgfqpoint{2.789104in}{5.029591in}}%
\pgfpathlineto{\pgfqpoint{2.813338in}{5.037625in}}%
\pgfpathlineto{\pgfqpoint{2.835143in}{5.044482in}}%
\pgfpathlineto{\pgfqpoint{2.854962in}{5.048052in}}%
\pgfpathlineto{\pgfqpoint{2.873127in}{5.045631in}}%
\pgfpathlineto{\pgfqpoint{2.889892in}{5.043350in}}%
\pgfpathlineto{\pgfqpoint{2.905459in}{5.041157in}}%
\pgfpathlineto{\pgfqpoint{2.919986in}{5.038980in}}%
\pgfpathlineto{\pgfqpoint{2.933605in}{5.036743in}}%
\pgfpathlineto{\pgfqpoint{2.946422in}{5.034370in}}%
\pgfpathlineto{\pgfqpoint{2.958526in}{5.031783in}}%
\pgfpathlineto{\pgfqpoint{2.969993in}{5.028902in}}%
\pgfpathlineto{\pgfqpoint{2.980886in}{5.025643in}}%
\pgfpathlineto{\pgfqpoint{2.991260in}{5.021918in}}%
\pgfpathlineto{\pgfqpoint{3.001162in}{5.017627in}}%
\pgfpathlineto{\pgfqpoint{3.010633in}{5.012651in}}%
\pgfpathlineto{\pgfqpoint{3.019710in}{5.006785in}}%
\pgfpathlineto{\pgfqpoint{3.028423in}{4.999300in}}%
\pgfpathlineto{\pgfqpoint{3.036801in}{4.988003in}}%
\pgfpathlineto{\pgfqpoint{3.044869in}{4.973706in}}%
\pgfpathlineto{\pgfqpoint{3.052648in}{4.957610in}}%
\pgfpathlineto{\pgfqpoint{3.060159in}{4.939883in}}%
\pgfpathlineto{\pgfqpoint{3.067420in}{4.920529in}}%
\pgfpathlineto{\pgfqpoint{3.074446in}{4.899519in}}%
\pgfpathlineto{\pgfqpoint{3.081253in}{4.876809in}}%
\pgfpathlineto{\pgfqpoint{3.087853in}{4.852353in}}%
\pgfpathlineto{\pgfqpoint{3.094259in}{4.826099in}}%
\pgfpathlineto{\pgfqpoint{3.100482in}{4.797996in}}%
\pgfpathlineto{\pgfqpoint{3.106532in}{4.767989in}}%
\pgfpathlineto{\pgfqpoint{3.112419in}{4.736021in}}%
\pgfpathlineto{\pgfqpoint{3.118150in}{4.702037in}}%
\pgfpathlineto{\pgfqpoint{3.123735in}{4.665979in}}%
\pgfpathlineto{\pgfqpoint{3.129180in}{4.627790in}}%
\pgfpathlineto{\pgfqpoint{3.134492in}{4.587411in}}%
\pgfpathlineto{\pgfqpoint{3.139677in}{4.544787in}}%
\pgfpathlineto{\pgfqpoint{3.144742in}{4.499858in}}%
\pgfpathlineto{\pgfqpoint{3.149692in}{4.452569in}}%
\pgfpathlineto{\pgfqpoint{3.154532in}{4.402862in}}%
\pgfpathlineto{\pgfqpoint{3.159267in}{4.350674in}}%
\pgfpathlineto{\pgfqpoint{3.163901in}{4.295916in}}%
\pgfpathlineto{\pgfqpoint{3.168438in}{4.238196in}}%
\pgfpathlineto{\pgfqpoint{3.172883in}{4.164325in}}%
\pgfpathlineto{\pgfqpoint{3.172883in}{4.179717in}}%
\pgfpathlineto{\pgfqpoint{3.172883in}{4.179717in}}%
\pgfpathlineto{\pgfqpoint{3.168438in}{4.249905in}}%
\pgfpathlineto{\pgfqpoint{3.163901in}{4.326724in}}%
\pgfpathlineto{\pgfqpoint{3.159267in}{4.397506in}}%
\pgfpathlineto{\pgfqpoint{3.154532in}{4.462352in}}%
\pgfpathlineto{\pgfqpoint{3.149692in}{4.521649in}}%
\pgfpathlineto{\pgfqpoint{3.144742in}{4.575783in}}%
\pgfpathlineto{\pgfqpoint{3.139677in}{4.625127in}}%
\pgfpathlineto{\pgfqpoint{3.134492in}{4.670032in}}%
\pgfpathlineto{\pgfqpoint{3.129180in}{4.710829in}}%
\pgfpathlineto{\pgfqpoint{3.123735in}{4.747828in}}%
\pgfpathlineto{\pgfqpoint{3.118150in}{4.781323in}}%
\pgfpathlineto{\pgfqpoint{3.112419in}{4.811587in}}%
\pgfpathlineto{\pgfqpoint{3.106532in}{4.838876in}}%
\pgfpathlineto{\pgfqpoint{3.100482in}{4.863431in}}%
\pgfpathlineto{\pgfqpoint{3.094259in}{4.885475in}}%
\pgfpathlineto{\pgfqpoint{3.087853in}{4.905219in}}%
\pgfpathlineto{\pgfqpoint{3.081253in}{4.922859in}}%
\pgfpathlineto{\pgfqpoint{3.074446in}{4.938580in}}%
\pgfpathlineto{\pgfqpoint{3.067420in}{4.952556in}}%
\pgfpathlineto{\pgfqpoint{3.060159in}{4.964953in}}%
\pgfpathlineto{\pgfqpoint{3.052648in}{4.975935in}}%
\pgfpathlineto{\pgfqpoint{3.044869in}{4.985690in}}%
\pgfpathlineto{\pgfqpoint{3.036801in}{4.994561in}}%
\pgfpathlineto{\pgfqpoint{3.028423in}{5.003907in}}%
\pgfpathlineto{\pgfqpoint{3.019710in}{5.014695in}}%
\pgfpathlineto{\pgfqpoint{3.010633in}{5.024873in}}%
\pgfpathlineto{\pgfqpoint{3.001162in}{5.033846in}}%
\pgfpathlineto{\pgfqpoint{2.991260in}{5.041537in}}%
\pgfpathlineto{\pgfqpoint{2.980886in}{5.047942in}}%
\pgfpathlineto{\pgfqpoint{2.969993in}{5.053077in}}%
\pgfpathlineto{\pgfqpoint{2.958526in}{5.056957in}}%
\pgfpathlineto{\pgfqpoint{2.946422in}{5.059599in}}%
\pgfpathlineto{\pgfqpoint{2.933605in}{5.061019in}}%
\pgfpathlineto{\pgfqpoint{2.919986in}{5.061230in}}%
\pgfpathlineto{\pgfqpoint{2.905459in}{5.060245in}}%
\pgfpathlineto{\pgfqpoint{2.889892in}{5.058074in}}%
\pgfpathlineto{\pgfqpoint{2.873127in}{5.054726in}}%
\pgfpathlineto{\pgfqpoint{2.854962in}{5.050232in}}%
\pgfpathlineto{\pgfqpoint{2.835143in}{5.050802in}}%
\pgfpathlineto{\pgfqpoint{2.813338in}{5.053814in}}%
\pgfpathlineto{\pgfqpoint{2.789104in}{5.057237in}}%
\pgfpathlineto{\pgfqpoint{2.761831in}{5.061154in}}%
\pgfpathlineto{\pgfqpoint{2.730647in}{5.065653in}}%
\pgfpathlineto{\pgfqpoint{2.694239in}{5.070831in}}%
\pgfpathlineto{\pgfqpoint{2.650497in}{5.076788in}}%
\pgfpathlineto{\pgfqpoint{2.595701in}{5.083636in}}%
\pgfpathlineto{\pgfqpoint{2.522305in}{5.091492in}}%
\pgfpathlineto{\pgfqpoint{2.410774in}{5.100484in}}%
\pgfpathlineto{\pgfqpoint{2.170064in}{5.110752in}}%
\pgfpathclose%
\pgfusepath{stroke,fill}%
\end{pgfscope}%
\begin{pgfscope}%
\pgfpathrectangle{\pgfqpoint{2.170064in}{4.812659in}}{\pgfqpoint{1.223103in}{0.607948in}}%
\pgfusepath{clip}%
\pgfsetroundcap%
\pgfsetroundjoin%
\pgfsetlinewidth{0.501875pt}%
\definecolor{currentstroke}{rgb}{0.000000,0.000000,1.000000}%
\pgfsetstrokecolor{currentstroke}%
\pgfsetstrokeopacity{0.800000}%
\pgfsetdash{}{0pt}%
\pgfpathmoveto{\pgfqpoint{2.170064in}{5.014503in}}%
\pgfpathlineto{\pgfqpoint{3.393168in}{5.014503in}}%
\pgfusepath{stroke}%
\end{pgfscope}%
\begin{pgfscope}%
\pgfpathrectangle{\pgfqpoint{2.170064in}{4.812659in}}{\pgfqpoint{1.223103in}{0.607948in}}%
\pgfusepath{clip}%
\pgfsetbuttcap%
\pgfsetroundjoin%
\pgfsetlinewidth{1.003750pt}%
\definecolor{currentstroke}{rgb}{0.000000,0.000000,0.000000}%
\pgfsetstrokecolor{currentstroke}%
\pgfsetdash{{3.700000pt}{1.600000pt}}{0.000000pt}%
\pgfpathmoveto{\pgfqpoint{2.170064in}{5.045573in}}%
\pgfpathlineto{\pgfqpoint{3.393168in}{5.045573in}}%
\pgfusepath{stroke}%
\end{pgfscope}%
\begin{pgfscope}%
\pgfsetroundcap%
\pgfsetroundjoin%
\pgfsetlinewidth{0.501875pt}%
\definecolor{currentstroke}{rgb}{0.000000,0.000000,1.000000}%
\pgfsetstrokecolor{currentstroke}%
\pgfsetstrokeopacity{0.800000}%
\pgfsetdash{}{0pt}%
\pgfpathmoveto{\pgfqpoint{2.970757in}{5.131008in}}%
\pgfpathquadraticcurveto{\pgfqpoint{2.907509in}{5.081335in}}{\pgfqpoint{2.844261in}{5.031661in}}%
\pgfusepath{stroke}%
\end{pgfscope}%
\begin{pgfscope}%
\pgfsetfillopacity{0.800000}%
\pgfsetstrokeopacity{0.800000}%
\definecolor{textcolor}{rgb}{0.000000,0.000000,1.000000}%
\pgfsetstrokecolor{textcolor}%
\pgfsetfillcolor{textcolor}%
\pgftext[x=2.910706in,y=5.196887in,left,base]{\color{textcolor}\sffamily\fontsize{5.647059}{6.776471}\selectfont 13.33(17)}%
\end{pgfscope}%
\begin{pgfscope}%
\pgfsetbuttcap%
\pgfsetroundjoin%
\definecolor{currentfill}{rgb}{0.150000,0.150000,0.150000}%
\pgfsetfillcolor{currentfill}%
\pgfsetlinewidth{1.003750pt}%
\definecolor{currentstroke}{rgb}{0.150000,0.150000,0.150000}%
\pgfsetstrokecolor{currentstroke}%
\pgfsetdash{}{0pt}%
\pgfsys@defobject{currentmarker}{\pgfqpoint{0.000000in}{-0.066667in}}{\pgfqpoint{0.000000in}{0.000000in}}{%
\pgfpathmoveto{\pgfqpoint{0.000000in}{0.000000in}}%
\pgfpathlineto{\pgfqpoint{0.000000in}{-0.066667in}}%
\pgfusepath{stroke,fill}%
}%
\begin{pgfscope}%
\pgfsys@transformshift{2.170064in}{4.812659in}%
\pgfsys@useobject{currentmarker}{}%
\end{pgfscope}%
\end{pgfscope}%
\begin{pgfscope}%
\pgfsetbuttcap%
\pgfsetroundjoin%
\definecolor{currentfill}{rgb}{0.150000,0.150000,0.150000}%
\pgfsetfillcolor{currentfill}%
\pgfsetlinewidth{1.003750pt}%
\definecolor{currentstroke}{rgb}{0.150000,0.150000,0.150000}%
\pgfsetstrokecolor{currentstroke}%
\pgfsetdash{}{0pt}%
\pgfsys@defobject{currentmarker}{\pgfqpoint{0.000000in}{-0.066667in}}{\pgfqpoint{0.000000in}{0.000000in}}{%
\pgfpathmoveto{\pgfqpoint{0.000000in}{0.000000in}}%
\pgfpathlineto{\pgfqpoint{0.000000in}{-0.066667in}}%
\pgfusepath{stroke,fill}%
}%
\begin{pgfscope}%
\pgfsys@transformshift{2.671473in}{4.812659in}%
\pgfsys@useobject{currentmarker}{}%
\end{pgfscope}%
\end{pgfscope}%
\begin{pgfscope}%
\pgfsetbuttcap%
\pgfsetroundjoin%
\definecolor{currentfill}{rgb}{0.150000,0.150000,0.150000}%
\pgfsetfillcolor{currentfill}%
\pgfsetlinewidth{1.003750pt}%
\definecolor{currentstroke}{rgb}{0.150000,0.150000,0.150000}%
\pgfsetstrokecolor{currentstroke}%
\pgfsetdash{}{0pt}%
\pgfsys@defobject{currentmarker}{\pgfqpoint{0.000000in}{-0.066667in}}{\pgfqpoint{0.000000in}{0.000000in}}{%
\pgfpathmoveto{\pgfqpoint{0.000000in}{0.000000in}}%
\pgfpathlineto{\pgfqpoint{0.000000in}{-0.066667in}}%
\pgfusepath{stroke,fill}%
}%
\begin{pgfscope}%
\pgfsys@transformshift{3.172883in}{4.812659in}%
\pgfsys@useobject{currentmarker}{}%
\end{pgfscope}%
\end{pgfscope}%
\begin{pgfscope}%
\pgfsetbuttcap%
\pgfsetroundjoin%
\definecolor{currentfill}{rgb}{0.150000,0.150000,0.150000}%
\pgfsetfillcolor{currentfill}%
\pgfsetlinewidth{0.803000pt}%
\definecolor{currentstroke}{rgb}{0.150000,0.150000,0.150000}%
\pgfsetstrokecolor{currentstroke}%
\pgfsetdash{}{0pt}%
\pgfsys@defobject{currentmarker}{\pgfqpoint{0.000000in}{-0.044444in}}{\pgfqpoint{0.000000in}{0.000000in}}{%
\pgfpathmoveto{\pgfqpoint{0.000000in}{0.000000in}}%
\pgfpathlineto{\pgfqpoint{0.000000in}{-0.044444in}}%
\pgfusepath{stroke,fill}%
}%
\begin{pgfscope}%
\pgfsys@transformshift{2.321004in}{4.812659in}%
\pgfsys@useobject{currentmarker}{}%
\end{pgfscope}%
\end{pgfscope}%
\begin{pgfscope}%
\pgfsetbuttcap%
\pgfsetroundjoin%
\definecolor{currentfill}{rgb}{0.150000,0.150000,0.150000}%
\pgfsetfillcolor{currentfill}%
\pgfsetlinewidth{0.803000pt}%
\definecolor{currentstroke}{rgb}{0.150000,0.150000,0.150000}%
\pgfsetstrokecolor{currentstroke}%
\pgfsetdash{}{0pt}%
\pgfsys@defobject{currentmarker}{\pgfqpoint{0.000000in}{-0.044444in}}{\pgfqpoint{0.000000in}{0.000000in}}{%
\pgfpathmoveto{\pgfqpoint{0.000000in}{0.000000in}}%
\pgfpathlineto{\pgfqpoint{0.000000in}{-0.044444in}}%
\pgfusepath{stroke,fill}%
}%
\begin{pgfscope}%
\pgfsys@transformshift{2.409297in}{4.812659in}%
\pgfsys@useobject{currentmarker}{}%
\end{pgfscope}%
\end{pgfscope}%
\begin{pgfscope}%
\pgfsetbuttcap%
\pgfsetroundjoin%
\definecolor{currentfill}{rgb}{0.150000,0.150000,0.150000}%
\pgfsetfillcolor{currentfill}%
\pgfsetlinewidth{0.803000pt}%
\definecolor{currentstroke}{rgb}{0.150000,0.150000,0.150000}%
\pgfsetstrokecolor{currentstroke}%
\pgfsetdash{}{0pt}%
\pgfsys@defobject{currentmarker}{\pgfqpoint{0.000000in}{-0.044444in}}{\pgfqpoint{0.000000in}{0.000000in}}{%
\pgfpathmoveto{\pgfqpoint{0.000000in}{0.000000in}}%
\pgfpathlineto{\pgfqpoint{0.000000in}{-0.044444in}}%
\pgfusepath{stroke,fill}%
}%
\begin{pgfscope}%
\pgfsys@transformshift{2.471943in}{4.812659in}%
\pgfsys@useobject{currentmarker}{}%
\end{pgfscope}%
\end{pgfscope}%
\begin{pgfscope}%
\pgfsetbuttcap%
\pgfsetroundjoin%
\definecolor{currentfill}{rgb}{0.150000,0.150000,0.150000}%
\pgfsetfillcolor{currentfill}%
\pgfsetlinewidth{0.803000pt}%
\definecolor{currentstroke}{rgb}{0.150000,0.150000,0.150000}%
\pgfsetstrokecolor{currentstroke}%
\pgfsetdash{}{0pt}%
\pgfsys@defobject{currentmarker}{\pgfqpoint{0.000000in}{-0.044444in}}{\pgfqpoint{0.000000in}{0.000000in}}{%
\pgfpathmoveto{\pgfqpoint{0.000000in}{0.000000in}}%
\pgfpathlineto{\pgfqpoint{0.000000in}{-0.044444in}}%
\pgfusepath{stroke,fill}%
}%
\begin{pgfscope}%
\pgfsys@transformshift{2.520534in}{4.812659in}%
\pgfsys@useobject{currentmarker}{}%
\end{pgfscope}%
\end{pgfscope}%
\begin{pgfscope}%
\pgfsetbuttcap%
\pgfsetroundjoin%
\definecolor{currentfill}{rgb}{0.150000,0.150000,0.150000}%
\pgfsetfillcolor{currentfill}%
\pgfsetlinewidth{0.803000pt}%
\definecolor{currentstroke}{rgb}{0.150000,0.150000,0.150000}%
\pgfsetstrokecolor{currentstroke}%
\pgfsetdash{}{0pt}%
\pgfsys@defobject{currentmarker}{\pgfqpoint{0.000000in}{-0.044444in}}{\pgfqpoint{0.000000in}{0.000000in}}{%
\pgfpathmoveto{\pgfqpoint{0.000000in}{0.000000in}}%
\pgfpathlineto{\pgfqpoint{0.000000in}{-0.044444in}}%
\pgfusepath{stroke,fill}%
}%
\begin{pgfscope}%
\pgfsys@transformshift{2.560237in}{4.812659in}%
\pgfsys@useobject{currentmarker}{}%
\end{pgfscope}%
\end{pgfscope}%
\begin{pgfscope}%
\pgfsetbuttcap%
\pgfsetroundjoin%
\definecolor{currentfill}{rgb}{0.150000,0.150000,0.150000}%
\pgfsetfillcolor{currentfill}%
\pgfsetlinewidth{0.803000pt}%
\definecolor{currentstroke}{rgb}{0.150000,0.150000,0.150000}%
\pgfsetstrokecolor{currentstroke}%
\pgfsetdash{}{0pt}%
\pgfsys@defobject{currentmarker}{\pgfqpoint{0.000000in}{-0.044444in}}{\pgfqpoint{0.000000in}{0.000000in}}{%
\pgfpathmoveto{\pgfqpoint{0.000000in}{0.000000in}}%
\pgfpathlineto{\pgfqpoint{0.000000in}{-0.044444in}}%
\pgfusepath{stroke,fill}%
}%
\begin{pgfscope}%
\pgfsys@transformshift{2.593804in}{4.812659in}%
\pgfsys@useobject{currentmarker}{}%
\end{pgfscope}%
\end{pgfscope}%
\begin{pgfscope}%
\pgfsetbuttcap%
\pgfsetroundjoin%
\definecolor{currentfill}{rgb}{0.150000,0.150000,0.150000}%
\pgfsetfillcolor{currentfill}%
\pgfsetlinewidth{0.803000pt}%
\definecolor{currentstroke}{rgb}{0.150000,0.150000,0.150000}%
\pgfsetstrokecolor{currentstroke}%
\pgfsetdash{}{0pt}%
\pgfsys@defobject{currentmarker}{\pgfqpoint{0.000000in}{-0.044444in}}{\pgfqpoint{0.000000in}{0.000000in}}{%
\pgfpathmoveto{\pgfqpoint{0.000000in}{0.000000in}}%
\pgfpathlineto{\pgfqpoint{0.000000in}{-0.044444in}}%
\pgfusepath{stroke,fill}%
}%
\begin{pgfscope}%
\pgfsys@transformshift{2.622882in}{4.812659in}%
\pgfsys@useobject{currentmarker}{}%
\end{pgfscope}%
\end{pgfscope}%
\begin{pgfscope}%
\pgfsetbuttcap%
\pgfsetroundjoin%
\definecolor{currentfill}{rgb}{0.150000,0.150000,0.150000}%
\pgfsetfillcolor{currentfill}%
\pgfsetlinewidth{0.803000pt}%
\definecolor{currentstroke}{rgb}{0.150000,0.150000,0.150000}%
\pgfsetstrokecolor{currentstroke}%
\pgfsetdash{}{0pt}%
\pgfsys@defobject{currentmarker}{\pgfqpoint{0.000000in}{-0.044444in}}{\pgfqpoint{0.000000in}{0.000000in}}{%
\pgfpathmoveto{\pgfqpoint{0.000000in}{0.000000in}}%
\pgfpathlineto{\pgfqpoint{0.000000in}{-0.044444in}}%
\pgfusepath{stroke,fill}%
}%
\begin{pgfscope}%
\pgfsys@transformshift{2.648530in}{4.812659in}%
\pgfsys@useobject{currentmarker}{}%
\end{pgfscope}%
\end{pgfscope}%
\begin{pgfscope}%
\pgfsetbuttcap%
\pgfsetroundjoin%
\definecolor{currentfill}{rgb}{0.150000,0.150000,0.150000}%
\pgfsetfillcolor{currentfill}%
\pgfsetlinewidth{0.803000pt}%
\definecolor{currentstroke}{rgb}{0.150000,0.150000,0.150000}%
\pgfsetstrokecolor{currentstroke}%
\pgfsetdash{}{0pt}%
\pgfsys@defobject{currentmarker}{\pgfqpoint{0.000000in}{-0.044444in}}{\pgfqpoint{0.000000in}{0.000000in}}{%
\pgfpathmoveto{\pgfqpoint{0.000000in}{0.000000in}}%
\pgfpathlineto{\pgfqpoint{0.000000in}{-0.044444in}}%
\pgfusepath{stroke,fill}%
}%
\begin{pgfscope}%
\pgfsys@transformshift{2.822413in}{4.812659in}%
\pgfsys@useobject{currentmarker}{}%
\end{pgfscope}%
\end{pgfscope}%
\begin{pgfscope}%
\pgfsetbuttcap%
\pgfsetroundjoin%
\definecolor{currentfill}{rgb}{0.150000,0.150000,0.150000}%
\pgfsetfillcolor{currentfill}%
\pgfsetlinewidth{0.803000pt}%
\definecolor{currentstroke}{rgb}{0.150000,0.150000,0.150000}%
\pgfsetstrokecolor{currentstroke}%
\pgfsetdash{}{0pt}%
\pgfsys@defobject{currentmarker}{\pgfqpoint{0.000000in}{-0.044444in}}{\pgfqpoint{0.000000in}{0.000000in}}{%
\pgfpathmoveto{\pgfqpoint{0.000000in}{0.000000in}}%
\pgfpathlineto{\pgfqpoint{0.000000in}{-0.044444in}}%
\pgfusepath{stroke,fill}%
}%
\begin{pgfscope}%
\pgfsys@transformshift{2.910706in}{4.812659in}%
\pgfsys@useobject{currentmarker}{}%
\end{pgfscope}%
\end{pgfscope}%
\begin{pgfscope}%
\pgfsetbuttcap%
\pgfsetroundjoin%
\definecolor{currentfill}{rgb}{0.150000,0.150000,0.150000}%
\pgfsetfillcolor{currentfill}%
\pgfsetlinewidth{0.803000pt}%
\definecolor{currentstroke}{rgb}{0.150000,0.150000,0.150000}%
\pgfsetstrokecolor{currentstroke}%
\pgfsetdash{}{0pt}%
\pgfsys@defobject{currentmarker}{\pgfqpoint{0.000000in}{-0.044444in}}{\pgfqpoint{0.000000in}{0.000000in}}{%
\pgfpathmoveto{\pgfqpoint{0.000000in}{0.000000in}}%
\pgfpathlineto{\pgfqpoint{0.000000in}{-0.044444in}}%
\pgfusepath{stroke,fill}%
}%
\begin{pgfscope}%
\pgfsys@transformshift{2.973352in}{4.812659in}%
\pgfsys@useobject{currentmarker}{}%
\end{pgfscope}%
\end{pgfscope}%
\begin{pgfscope}%
\pgfsetbuttcap%
\pgfsetroundjoin%
\definecolor{currentfill}{rgb}{0.150000,0.150000,0.150000}%
\pgfsetfillcolor{currentfill}%
\pgfsetlinewidth{0.803000pt}%
\definecolor{currentstroke}{rgb}{0.150000,0.150000,0.150000}%
\pgfsetstrokecolor{currentstroke}%
\pgfsetdash{}{0pt}%
\pgfsys@defobject{currentmarker}{\pgfqpoint{0.000000in}{-0.044444in}}{\pgfqpoint{0.000000in}{0.000000in}}{%
\pgfpathmoveto{\pgfqpoint{0.000000in}{0.000000in}}%
\pgfpathlineto{\pgfqpoint{0.000000in}{-0.044444in}}%
\pgfusepath{stroke,fill}%
}%
\begin{pgfscope}%
\pgfsys@transformshift{3.021943in}{4.812659in}%
\pgfsys@useobject{currentmarker}{}%
\end{pgfscope}%
\end{pgfscope}%
\begin{pgfscope}%
\pgfsetbuttcap%
\pgfsetroundjoin%
\definecolor{currentfill}{rgb}{0.150000,0.150000,0.150000}%
\pgfsetfillcolor{currentfill}%
\pgfsetlinewidth{0.803000pt}%
\definecolor{currentstroke}{rgb}{0.150000,0.150000,0.150000}%
\pgfsetstrokecolor{currentstroke}%
\pgfsetdash{}{0pt}%
\pgfsys@defobject{currentmarker}{\pgfqpoint{0.000000in}{-0.044444in}}{\pgfqpoint{0.000000in}{0.000000in}}{%
\pgfpathmoveto{\pgfqpoint{0.000000in}{0.000000in}}%
\pgfpathlineto{\pgfqpoint{0.000000in}{-0.044444in}}%
\pgfusepath{stroke,fill}%
}%
\begin{pgfscope}%
\pgfsys@transformshift{3.061646in}{4.812659in}%
\pgfsys@useobject{currentmarker}{}%
\end{pgfscope}%
\end{pgfscope}%
\begin{pgfscope}%
\pgfsetbuttcap%
\pgfsetroundjoin%
\definecolor{currentfill}{rgb}{0.150000,0.150000,0.150000}%
\pgfsetfillcolor{currentfill}%
\pgfsetlinewidth{0.803000pt}%
\definecolor{currentstroke}{rgb}{0.150000,0.150000,0.150000}%
\pgfsetstrokecolor{currentstroke}%
\pgfsetdash{}{0pt}%
\pgfsys@defobject{currentmarker}{\pgfqpoint{0.000000in}{-0.044444in}}{\pgfqpoint{0.000000in}{0.000000in}}{%
\pgfpathmoveto{\pgfqpoint{0.000000in}{0.000000in}}%
\pgfpathlineto{\pgfqpoint{0.000000in}{-0.044444in}}%
\pgfusepath{stroke,fill}%
}%
\begin{pgfscope}%
\pgfsys@transformshift{3.095213in}{4.812659in}%
\pgfsys@useobject{currentmarker}{}%
\end{pgfscope}%
\end{pgfscope}%
\begin{pgfscope}%
\pgfsetbuttcap%
\pgfsetroundjoin%
\definecolor{currentfill}{rgb}{0.150000,0.150000,0.150000}%
\pgfsetfillcolor{currentfill}%
\pgfsetlinewidth{0.803000pt}%
\definecolor{currentstroke}{rgb}{0.150000,0.150000,0.150000}%
\pgfsetstrokecolor{currentstroke}%
\pgfsetdash{}{0pt}%
\pgfsys@defobject{currentmarker}{\pgfqpoint{0.000000in}{-0.044444in}}{\pgfqpoint{0.000000in}{0.000000in}}{%
\pgfpathmoveto{\pgfqpoint{0.000000in}{0.000000in}}%
\pgfpathlineto{\pgfqpoint{0.000000in}{-0.044444in}}%
\pgfusepath{stroke,fill}%
}%
\begin{pgfscope}%
\pgfsys@transformshift{3.124291in}{4.812659in}%
\pgfsys@useobject{currentmarker}{}%
\end{pgfscope}%
\end{pgfscope}%
\begin{pgfscope}%
\pgfsetbuttcap%
\pgfsetroundjoin%
\definecolor{currentfill}{rgb}{0.150000,0.150000,0.150000}%
\pgfsetfillcolor{currentfill}%
\pgfsetlinewidth{0.803000pt}%
\definecolor{currentstroke}{rgb}{0.150000,0.150000,0.150000}%
\pgfsetstrokecolor{currentstroke}%
\pgfsetdash{}{0pt}%
\pgfsys@defobject{currentmarker}{\pgfqpoint{0.000000in}{-0.044444in}}{\pgfqpoint{0.000000in}{0.000000in}}{%
\pgfpathmoveto{\pgfqpoint{0.000000in}{0.000000in}}%
\pgfpathlineto{\pgfqpoint{0.000000in}{-0.044444in}}%
\pgfusepath{stroke,fill}%
}%
\begin{pgfscope}%
\pgfsys@transformshift{3.149939in}{4.812659in}%
\pgfsys@useobject{currentmarker}{}%
\end{pgfscope}%
\end{pgfscope}%
\begin{pgfscope}%
\pgfsetbuttcap%
\pgfsetroundjoin%
\definecolor{currentfill}{rgb}{0.150000,0.150000,0.150000}%
\pgfsetfillcolor{currentfill}%
\pgfsetlinewidth{0.803000pt}%
\definecolor{currentstroke}{rgb}{0.150000,0.150000,0.150000}%
\pgfsetstrokecolor{currentstroke}%
\pgfsetdash{}{0pt}%
\pgfsys@defobject{currentmarker}{\pgfqpoint{0.000000in}{-0.044444in}}{\pgfqpoint{0.000000in}{0.000000in}}{%
\pgfpathmoveto{\pgfqpoint{0.000000in}{0.000000in}}%
\pgfpathlineto{\pgfqpoint{0.000000in}{-0.044444in}}%
\pgfusepath{stroke,fill}%
}%
\begin{pgfscope}%
\pgfsys@transformshift{3.323822in}{4.812659in}%
\pgfsys@useobject{currentmarker}{}%
\end{pgfscope}%
\end{pgfscope}%
\begin{pgfscope}%
\pgfsetbuttcap%
\pgfsetroundjoin%
\definecolor{currentfill}{rgb}{0.150000,0.150000,0.150000}%
\pgfsetfillcolor{currentfill}%
\pgfsetlinewidth{1.003750pt}%
\definecolor{currentstroke}{rgb}{0.150000,0.150000,0.150000}%
\pgfsetstrokecolor{currentstroke}%
\pgfsetdash{}{0pt}%
\pgfsys@defobject{currentmarker}{\pgfqpoint{-0.066667in}{0.000000in}}{\pgfqpoint{0.000000in}{0.000000in}}{%
\pgfpathmoveto{\pgfqpoint{0.000000in}{0.000000in}}%
\pgfpathlineto{\pgfqpoint{-0.066667in}{0.000000in}}%
\pgfusepath{stroke,fill}%
}%
\begin{pgfscope}%
\pgfsys@transformshift{2.170064in}{4.812659in}%
\pgfsys@useobject{currentmarker}{}%
\end{pgfscope}%
\end{pgfscope}%
\begin{pgfscope}%
\pgfsetbuttcap%
\pgfsetroundjoin%
\definecolor{currentfill}{rgb}{0.150000,0.150000,0.150000}%
\pgfsetfillcolor{currentfill}%
\pgfsetlinewidth{1.003750pt}%
\definecolor{currentstroke}{rgb}{0.150000,0.150000,0.150000}%
\pgfsetstrokecolor{currentstroke}%
\pgfsetdash{}{0pt}%
\pgfsys@defobject{currentmarker}{\pgfqpoint{-0.066667in}{0.000000in}}{\pgfqpoint{0.000000in}{0.000000in}}{%
\pgfpathmoveto{\pgfqpoint{0.000000in}{0.000000in}}%
\pgfpathlineto{\pgfqpoint{-0.066667in}{0.000000in}}%
\pgfusepath{stroke,fill}%
}%
\begin{pgfscope}%
\pgfsys@transformshift{2.170064in}{5.045573in}%
\pgfsys@useobject{currentmarker}{}%
\end{pgfscope}%
\end{pgfscope}%
\begin{pgfscope}%
\pgfsetbuttcap%
\pgfsetroundjoin%
\definecolor{currentfill}{rgb}{0.150000,0.150000,0.150000}%
\pgfsetfillcolor{currentfill}%
\pgfsetlinewidth{1.003750pt}%
\definecolor{currentstroke}{rgb}{0.150000,0.150000,0.150000}%
\pgfsetstrokecolor{currentstroke}%
\pgfsetdash{}{0pt}%
\pgfsys@defobject{currentmarker}{\pgfqpoint{-0.066667in}{0.000000in}}{\pgfqpoint{0.000000in}{0.000000in}}{%
\pgfpathmoveto{\pgfqpoint{0.000000in}{0.000000in}}%
\pgfpathlineto{\pgfqpoint{-0.066667in}{0.000000in}}%
\pgfusepath{stroke,fill}%
}%
\begin{pgfscope}%
\pgfsys@transformshift{2.170064in}{5.420607in}%
\pgfsys@useobject{currentmarker}{}%
\end{pgfscope}%
\end{pgfscope}%
\begin{pgfscope}%
\pgfpathrectangle{\pgfqpoint{2.170064in}{4.812659in}}{\pgfqpoint{1.223103in}{0.607948in}}%
\pgfusepath{clip}%
\pgfsetroundcap%
\pgfsetroundjoin%
\pgfsetlinewidth{1.204500pt}%
\definecolor{currentstroke}{rgb}{0.000000,0.501961,0.000000}%
\pgfsetstrokecolor{currentstroke}%
\pgfsetdash{}{0pt}%
\pgfpathmoveto{\pgfqpoint{2.170064in}{5.016363in}}%
\pgfpathlineto{\pgfqpoint{2.410774in}{5.020147in}}%
\pgfpathlineto{\pgfqpoint{2.522305in}{5.023916in}}%
\pgfpathlineto{\pgfqpoint{2.595701in}{5.027617in}}%
\pgfpathlineto{\pgfqpoint{2.650497in}{5.031200in}}%
\pgfpathlineto{\pgfqpoint{2.694239in}{5.034617in}}%
\pgfpathlineto{\pgfqpoint{2.730647in}{5.037821in}}%
\pgfpathlineto{\pgfqpoint{2.761831in}{5.040767in}}%
\pgfpathlineto{\pgfqpoint{2.789104in}{5.043414in}}%
\pgfpathlineto{\pgfqpoint{2.813338in}{5.045719in}}%
\pgfpathlineto{\pgfqpoint{2.835143in}{5.047642in}}%
\pgfpathlineto{\pgfqpoint{2.854962in}{5.049142in}}%
\pgfpathlineto{\pgfqpoint{2.873127in}{5.050179in}}%
\pgfpathlineto{\pgfqpoint{2.889892in}{5.050712in}}%
\pgfpathlineto{\pgfqpoint{2.905459in}{5.050701in}}%
\pgfpathlineto{\pgfqpoint{2.919986in}{5.050105in}}%
\pgfpathlineto{\pgfqpoint{2.933605in}{5.048881in}}%
\pgfpathlineto{\pgfqpoint{2.946422in}{5.046984in}}%
\pgfpathlineto{\pgfqpoint{2.958526in}{5.044370in}}%
\pgfpathlineto{\pgfqpoint{2.969993in}{5.040989in}}%
\pgfpathlineto{\pgfqpoint{2.980886in}{5.036793in}}%
\pgfpathlineto{\pgfqpoint{2.991260in}{5.031727in}}%
\pgfpathlineto{\pgfqpoint{3.001162in}{5.025737in}}%
\pgfpathlineto{\pgfqpoint{3.010633in}{5.018762in}}%
\pgfpathlineto{\pgfqpoint{3.019710in}{5.010740in}}%
\pgfpathlineto{\pgfqpoint{3.028423in}{5.001604in}}%
\pgfpathlineto{\pgfqpoint{3.036801in}{4.991282in}}%
\pgfpathlineto{\pgfqpoint{3.044869in}{4.979698in}}%
\pgfpathlineto{\pgfqpoint{3.052648in}{4.966772in}}%
\pgfpathlineto{\pgfqpoint{3.060159in}{4.952418in}}%
\pgfpathlineto{\pgfqpoint{3.067420in}{4.936543in}}%
\pgfpathlineto{\pgfqpoint{3.074446in}{4.919050in}}%
\pgfpathlineto{\pgfqpoint{3.081253in}{4.899834in}}%
\pgfpathlineto{\pgfqpoint{3.087853in}{4.878786in}}%
\pgfpathlineto{\pgfqpoint{3.094259in}{4.855787in}}%
\pgfpathlineto{\pgfqpoint{3.100482in}{4.830713in}}%
\pgfpathlineto{\pgfqpoint{3.105225in}{4.809326in}}%
\pgfusepath{stroke}%
\end{pgfscope}%
\begin{pgfscope}%
\pgfsetrectcap%
\pgfsetmiterjoin%
\pgfsetlinewidth{1.003750pt}%
\definecolor{currentstroke}{rgb}{0.150000,0.150000,0.150000}%
\pgfsetstrokecolor{currentstroke}%
\pgfsetdash{}{0pt}%
\pgfpathmoveto{\pgfqpoint{2.170064in}{4.812659in}}%
\pgfpathlineto{\pgfqpoint{2.170064in}{5.420607in}}%
\pgfusepath{stroke}%
\end{pgfscope}%
\begin{pgfscope}%
\pgfsetrectcap%
\pgfsetmiterjoin%
\pgfsetlinewidth{1.003750pt}%
\definecolor{currentstroke}{rgb}{0.150000,0.150000,0.150000}%
\pgfsetstrokecolor{currentstroke}%
\pgfsetdash{}{0pt}%
\pgfpathmoveto{\pgfqpoint{2.170064in}{4.812659in}}%
\pgfpathlineto{\pgfqpoint{3.393168in}{4.812659in}}%
\pgfusepath{stroke}%
\end{pgfscope}%
\begin{pgfscope}%
\pgfpathrectangle{\pgfqpoint{2.170064in}{4.812659in}}{\pgfqpoint{1.223103in}{0.607948in}}%
\pgfusepath{clip}%
\pgfsetbuttcap%
\pgfsetroundjoin%
\definecolor{currentfill}{rgb}{0.000000,0.000000,0.000000}%
\pgfsetfillcolor{currentfill}%
\pgfsetlinewidth{1.003750pt}%
\definecolor{currentstroke}{rgb}{0.000000,0.000000,0.000000}%
\pgfsetstrokecolor{currentstroke}%
\pgfsetdash{}{0pt}%
\pgfsys@defobject{currentmarker}{\pgfqpoint{-0.013889in}{-0.013889in}}{\pgfqpoint{0.013889in}{0.013889in}}{%
\pgfpathmoveto{\pgfqpoint{0.000000in}{-0.013889in}}%
\pgfpathcurveto{\pgfqpoint{0.003683in}{-0.013889in}}{\pgfqpoint{0.007216in}{-0.012425in}}{\pgfqpoint{0.009821in}{-0.009821in}}%
\pgfpathcurveto{\pgfqpoint{0.012425in}{-0.007216in}}{\pgfqpoint{0.013889in}{-0.003683in}}{\pgfqpoint{0.013889in}{0.000000in}}%
\pgfpathcurveto{\pgfqpoint{0.013889in}{0.003683in}}{\pgfqpoint{0.012425in}{0.007216in}}{\pgfqpoint{0.009821in}{0.009821in}}%
\pgfpathcurveto{\pgfqpoint{0.007216in}{0.012425in}}{\pgfqpoint{0.003683in}{0.013889in}}{\pgfqpoint{0.000000in}{0.013889in}}%
\pgfpathcurveto{\pgfqpoint{-0.003683in}{0.013889in}}{\pgfqpoint{-0.007216in}{0.012425in}}{\pgfqpoint{-0.009821in}{0.009821in}}%
\pgfpathcurveto{\pgfqpoint{-0.012425in}{0.007216in}}{\pgfqpoint{-0.013889in}{0.003683in}}{\pgfqpoint{-0.013889in}{0.000000in}}%
\pgfpathcurveto{\pgfqpoint{-0.013889in}{-0.003683in}}{\pgfqpoint{-0.012425in}{-0.007216in}}{\pgfqpoint{-0.009821in}{-0.009821in}}%
\pgfpathcurveto{\pgfqpoint{-0.007216in}{-0.012425in}}{\pgfqpoint{-0.003683in}{-0.013889in}}{\pgfqpoint{0.000000in}{-0.013889in}}%
\pgfpathclose%
\pgfusepath{stroke,fill}%
}%
\begin{pgfscope}%
\pgfsys@transformshift{3.172883in}{4.167361in}%
\pgfsys@useobject{currentmarker}{}%
\end{pgfscope}%
\begin{pgfscope}%
\pgfsys@transformshift{3.084589in}{4.905580in}%
\pgfsys@useobject{currentmarker}{}%
\end{pgfscope}%
\begin{pgfscope}%
\pgfsys@transformshift{3.021943in}{5.004976in}%
\pgfsys@useobject{currentmarker}{}%
\end{pgfscope}%
\begin{pgfscope}%
\pgfsys@transformshift{2.973352in}{5.033054in}%
\pgfsys@useobject{currentmarker}{}%
\end{pgfscope}%
\begin{pgfscope}%
\pgfsys@transformshift{2.933650in}{5.042883in}%
\pgfsys@useobject{currentmarker}{}%
\end{pgfscope}%
\begin{pgfscope}%
\pgfsys@transformshift{2.900082in}{5.046769in}%
\pgfsys@useobject{currentmarker}{}%
\end{pgfscope}%
\begin{pgfscope}%
\pgfsys@transformshift{2.871004in}{5.048389in}%
\pgfsys@useobject{currentmarker}{}%
\end{pgfscope}%
\begin{pgfscope}%
\pgfsys@transformshift{2.865627in}{5.048575in}%
\pgfsys@useobject{currentmarker}{}%
\end{pgfscope}%
\begin{pgfscope}%
\pgfsys@transformshift{2.860380in}{5.048729in}%
\pgfsys@useobject{currentmarker}{}%
\end{pgfscope}%
\begin{pgfscope}%
\pgfsys@transformshift{2.855256in}{5.048858in}%
\pgfsys@useobject{currentmarker}{}%
\end{pgfscope}%
\begin{pgfscope}%
\pgfsys@transformshift{2.850250in}{5.048963in}%
\pgfsys@useobject{currentmarker}{}%
\end{pgfscope}%
\begin{pgfscope}%
\pgfsys@transformshift{2.845356in}{5.049050in}%
\pgfsys@useobject{currentmarker}{}%
\end{pgfscope}%
\begin{pgfscope}%
\pgfsys@transformshift{2.840570in}{5.049120in}%
\pgfsys@useobject{currentmarker}{}%
\end{pgfscope}%
\begin{pgfscope}%
\pgfsys@transformshift{2.835887in}{5.049175in}%
\pgfsys@useobject{currentmarker}{}%
\end{pgfscope}%
\begin{pgfscope}%
\pgfsys@transformshift{2.831302in}{5.049218in}%
\pgfsys@useobject{currentmarker}{}%
\end{pgfscope}%
\begin{pgfscope}%
\pgfsys@transformshift{2.826812in}{5.049251in}%
\pgfsys@useobject{currentmarker}{}%
\end{pgfscope}%
\begin{pgfscope}%
\pgfsys@transformshift{2.822413in}{5.049274in}%
\pgfsys@useobject{currentmarker}{}%
\end{pgfscope}%
\end{pgfscope}%
\begin{pgfscope}%
\pgfsetbuttcap%
\pgfsetmiterjoin%
\definecolor{currentfill}{rgb}{1.000000,1.000000,1.000000}%
\pgfsetfillcolor{currentfill}%
\pgfsetlinewidth{0.000000pt}%
\definecolor{currentstroke}{rgb}{0.000000,0.000000,0.000000}%
\pgfsetstrokecolor{currentstroke}%
\pgfsetstrokeopacity{0.000000}%
\pgfsetdash{}{0pt}%
\pgfpathmoveto{\pgfqpoint{3.637789in}{4.812659in}}%
\pgfpathlineto{\pgfqpoint{4.860892in}{4.812659in}}%
\pgfpathlineto{\pgfqpoint{4.860892in}{5.420607in}}%
\pgfpathlineto{\pgfqpoint{3.637789in}{5.420607in}}%
\pgfpathclose%
\pgfusepath{fill}%
\end{pgfscope}%
\begin{pgfscope}%
\pgfpathrectangle{\pgfqpoint{3.637789in}{4.812659in}}{\pgfqpoint{1.223103in}{0.607948in}}%
\pgfusepath{clip}%
\pgfsetbuttcap%
\pgfsetmiterjoin%
\definecolor{currentfill}{rgb}{0.000000,0.000000,1.000000}%
\pgfsetfillcolor{currentfill}%
\pgfsetfillopacity{0.100000}%
\pgfsetlinewidth{0.803000pt}%
\definecolor{currentstroke}{rgb}{0.000000,0.000000,1.000000}%
\pgfsetstrokecolor{currentstroke}%
\pgfsetstrokeopacity{0.100000}%
\pgfsetdash{}{0pt}%
\pgfpathmoveto{\pgfqpoint{3.637789in}{5.008910in}}%
\pgfpathlineto{\pgfqpoint{3.637789in}{5.067721in}}%
\pgfpathlineto{\pgfqpoint{4.860892in}{5.067721in}}%
\pgfpathlineto{\pgfqpoint{4.860892in}{5.008910in}}%
\pgfpathclose%
\pgfusepath{stroke,fill}%
\end{pgfscope}%
\begin{pgfscope}%
\pgfpathrectangle{\pgfqpoint{3.637789in}{4.812659in}}{\pgfqpoint{1.223103in}{0.607948in}}%
\pgfusepath{clip}%
\pgfsetbuttcap%
\pgfsetroundjoin%
\definecolor{currentfill}{rgb}{0.000000,0.501961,0.000000}%
\pgfsetfillcolor{currentfill}%
\pgfsetfillopacity{0.500000}%
\pgfsetlinewidth{0.803000pt}%
\definecolor{currentstroke}{rgb}{0.000000,0.501961,0.000000}%
\pgfsetstrokecolor{currentstroke}%
\pgfsetstrokeopacity{0.500000}%
\pgfsetdash{}{0pt}%
\pgfpathmoveto{\pgfqpoint{3.637789in}{5.066421in}}%
\pgfpathlineto{\pgfqpoint{3.637789in}{5.011818in}}%
\pgfpathlineto{\pgfqpoint{3.878498in}{5.017406in}}%
\pgfpathlineto{\pgfqpoint{3.990029in}{5.022629in}}%
\pgfpathlineto{\pgfqpoint{4.063425in}{5.027510in}}%
\pgfpathlineto{\pgfqpoint{4.118221in}{5.032067in}}%
\pgfpathlineto{\pgfqpoint{4.161963in}{5.036318in}}%
\pgfpathlineto{\pgfqpoint{4.198371in}{5.040278in}}%
\pgfpathlineto{\pgfqpoint{4.229555in}{5.043962in}}%
\pgfpathlineto{\pgfqpoint{4.256828in}{5.047382in}}%
\pgfpathlineto{\pgfqpoint{4.281062in}{5.050548in}}%
\pgfpathlineto{\pgfqpoint{4.302867in}{5.053466in}}%
\pgfpathlineto{\pgfqpoint{4.322686in}{5.055872in}}%
\pgfpathlineto{\pgfqpoint{4.340851in}{5.056246in}}%
\pgfpathlineto{\pgfqpoint{4.357617in}{5.056717in}}%
\pgfpathlineto{\pgfqpoint{4.373183in}{5.057315in}}%
\pgfpathlineto{\pgfqpoint{4.387711in}{5.058026in}}%
\pgfpathlineto{\pgfqpoint{4.401329in}{5.058836in}}%
\pgfpathlineto{\pgfqpoint{4.414146in}{5.059729in}}%
\pgfpathlineto{\pgfqpoint{4.426250in}{5.060686in}}%
\pgfpathlineto{\pgfqpoint{4.437717in}{5.061691in}}%
\pgfpathlineto{\pgfqpoint{4.448610in}{5.062724in}}%
\pgfpathlineto{\pgfqpoint{4.458984in}{5.063761in}}%
\pgfpathlineto{\pgfqpoint{4.468886in}{5.064780in}}%
\pgfpathlineto{\pgfqpoint{4.478357in}{5.065750in}}%
\pgfpathlineto{\pgfqpoint{4.487434in}{5.066628in}}%
\pgfpathlineto{\pgfqpoint{4.496147in}{5.067334in}}%
\pgfpathlineto{\pgfqpoint{4.504525in}{5.067620in}}%
\pgfpathlineto{\pgfqpoint{4.512593in}{5.067031in}}%
\pgfpathlineto{\pgfqpoint{4.520372in}{5.065724in}}%
\pgfpathlineto{\pgfqpoint{4.527883in}{5.063975in}}%
\pgfpathlineto{\pgfqpoint{4.535144in}{5.061853in}}%
\pgfpathlineto{\pgfqpoint{4.542170in}{5.059367in}}%
\pgfpathlineto{\pgfqpoint{4.548977in}{5.056511in}}%
\pgfpathlineto{\pgfqpoint{4.555577in}{5.053273in}}%
\pgfpathlineto{\pgfqpoint{4.561983in}{5.049640in}}%
\pgfpathlineto{\pgfqpoint{4.568206in}{5.045598in}}%
\pgfpathlineto{\pgfqpoint{4.574256in}{5.041131in}}%
\pgfpathlineto{\pgfqpoint{4.580143in}{5.036223in}}%
\pgfpathlineto{\pgfqpoint{4.585874in}{5.030858in}}%
\pgfpathlineto{\pgfqpoint{4.591459in}{5.025020in}}%
\pgfpathlineto{\pgfqpoint{4.596904in}{5.018692in}}%
\pgfpathlineto{\pgfqpoint{4.602216in}{5.011858in}}%
\pgfpathlineto{\pgfqpoint{4.607401in}{5.004502in}}%
\pgfpathlineto{\pgfqpoint{4.612466in}{4.996605in}}%
\pgfpathlineto{\pgfqpoint{4.617416in}{4.988152in}}%
\pgfpathlineto{\pgfqpoint{4.622256in}{4.979127in}}%
\pgfpathlineto{\pgfqpoint{4.626991in}{4.969509in}}%
\pgfpathlineto{\pgfqpoint{4.631625in}{4.959277in}}%
\pgfpathlineto{\pgfqpoint{4.636162in}{4.948345in}}%
\pgfpathlineto{\pgfqpoint{4.640607in}{4.933268in}}%
\pgfpathlineto{\pgfqpoint{4.640607in}{4.937209in}}%
\pgfpathlineto{\pgfqpoint{4.640607in}{4.937209in}}%
\pgfpathlineto{\pgfqpoint{4.636162in}{4.952109in}}%
\pgfpathlineto{\pgfqpoint{4.631625in}{4.968624in}}%
\pgfpathlineto{\pgfqpoint{4.626991in}{4.983459in}}%
\pgfpathlineto{\pgfqpoint{4.622256in}{4.996676in}}%
\pgfpathlineto{\pgfqpoint{4.617416in}{5.008391in}}%
\pgfpathlineto{\pgfqpoint{4.612466in}{5.018719in}}%
\pgfpathlineto{\pgfqpoint{4.607401in}{5.027771in}}%
\pgfpathlineto{\pgfqpoint{4.602216in}{5.035651in}}%
\pgfpathlineto{\pgfqpoint{4.596904in}{5.042458in}}%
\pgfpathlineto{\pgfqpoint{4.591459in}{5.048286in}}%
\pgfpathlineto{\pgfqpoint{4.585874in}{5.053225in}}%
\pgfpathlineto{\pgfqpoint{4.580143in}{5.057355in}}%
\pgfpathlineto{\pgfqpoint{4.574256in}{5.060757in}}%
\pgfpathlineto{\pgfqpoint{4.568206in}{5.063503in}}%
\pgfpathlineto{\pgfqpoint{4.561983in}{5.065662in}}%
\pgfpathlineto{\pgfqpoint{4.555577in}{5.067299in}}%
\pgfpathlineto{\pgfqpoint{4.548977in}{5.068474in}}%
\pgfpathlineto{\pgfqpoint{4.542170in}{5.069245in}}%
\pgfpathlineto{\pgfqpoint{4.535144in}{5.069668in}}%
\pgfpathlineto{\pgfqpoint{4.527883in}{5.069798in}}%
\pgfpathlineto{\pgfqpoint{4.520372in}{5.069703in}}%
\pgfpathlineto{\pgfqpoint{4.512593in}{5.069506in}}%
\pgfpathlineto{\pgfqpoint{4.504525in}{5.069535in}}%
\pgfpathlineto{\pgfqpoint{4.496147in}{5.069993in}}%
\pgfpathlineto{\pgfqpoint{4.487434in}{5.070469in}}%
\pgfpathlineto{\pgfqpoint{4.478357in}{5.070758in}}%
\pgfpathlineto{\pgfqpoint{4.468886in}{5.070813in}}%
\pgfpathlineto{\pgfqpoint{4.458984in}{5.070628in}}%
\pgfpathlineto{\pgfqpoint{4.448610in}{5.070204in}}%
\pgfpathlineto{\pgfqpoint{4.437717in}{5.069545in}}%
\pgfpathlineto{\pgfqpoint{4.426250in}{5.068655in}}%
\pgfpathlineto{\pgfqpoint{4.414146in}{5.067538in}}%
\pgfpathlineto{\pgfqpoint{4.401329in}{5.066197in}}%
\pgfpathlineto{\pgfqpoint{4.387711in}{5.064635in}}%
\pgfpathlineto{\pgfqpoint{4.373183in}{5.062851in}}%
\pgfpathlineto{\pgfqpoint{4.357617in}{5.060847in}}%
\pgfpathlineto{\pgfqpoint{4.340851in}{5.058621in}}%
\pgfpathlineto{\pgfqpoint{4.322686in}{5.056217in}}%
\pgfpathlineto{\pgfqpoint{4.302867in}{5.055772in}}%
\pgfpathlineto{\pgfqpoint{4.281062in}{5.055777in}}%
\pgfpathlineto{\pgfqpoint{4.256828in}{5.055973in}}%
\pgfpathlineto{\pgfqpoint{4.229555in}{5.056374in}}%
\pgfpathlineto{\pgfqpoint{4.198371in}{5.056997in}}%
\pgfpathlineto{\pgfqpoint{4.161963in}{5.057858in}}%
\pgfpathlineto{\pgfqpoint{4.118221in}{5.058975in}}%
\pgfpathlineto{\pgfqpoint{4.063425in}{5.060368in}}%
\pgfpathlineto{\pgfqpoint{3.990029in}{5.062059in}}%
\pgfpathlineto{\pgfqpoint{3.878498in}{5.064068in}}%
\pgfpathlineto{\pgfqpoint{3.637789in}{5.066421in}}%
\pgfpathclose%
\pgfusepath{stroke,fill}%
\end{pgfscope}%
\begin{pgfscope}%
\pgfpathrectangle{\pgfqpoint{3.637789in}{4.812659in}}{\pgfqpoint{1.223103in}{0.607948in}}%
\pgfusepath{clip}%
\pgfsetroundcap%
\pgfsetroundjoin%
\pgfsetlinewidth{0.501875pt}%
\definecolor{currentstroke}{rgb}{0.000000,0.000000,1.000000}%
\pgfsetstrokecolor{currentstroke}%
\pgfsetstrokeopacity{0.800000}%
\pgfsetdash{}{0pt}%
\pgfpathmoveto{\pgfqpoint{3.637789in}{5.038316in}}%
\pgfpathlineto{\pgfqpoint{4.860892in}{5.038316in}}%
\pgfusepath{stroke}%
\end{pgfscope}%
\begin{pgfscope}%
\pgfpathrectangle{\pgfqpoint{3.637789in}{4.812659in}}{\pgfqpoint{1.223103in}{0.607948in}}%
\pgfusepath{clip}%
\pgfsetbuttcap%
\pgfsetroundjoin%
\pgfsetlinewidth{1.003750pt}%
\definecolor{currentstroke}{rgb}{0.000000,0.000000,0.000000}%
\pgfsetstrokecolor{currentstroke}%
\pgfsetdash{{3.700000pt}{1.600000pt}}{0.000000pt}%
\pgfpathmoveto{\pgfqpoint{3.637789in}{5.045573in}}%
\pgfpathlineto{\pgfqpoint{4.860892in}{5.045573in}}%
\pgfusepath{stroke}%
\end{pgfscope}%
\begin{pgfscope}%
\pgfsetroundcap%
\pgfsetroundjoin%
\pgfsetlinewidth{0.501875pt}%
\definecolor{currentstroke}{rgb}{0.000000,0.000000,1.000000}%
\pgfsetstrokecolor{currentstroke}%
\pgfsetstrokeopacity{0.800000}%
\pgfsetdash{}{0pt}%
\pgfpathmoveto{\pgfqpoint{4.451637in}{5.155653in}}%
\pgfpathquadraticcurveto{\pgfqpoint{4.382119in}{5.105145in}}{\pgfqpoint{4.312601in}{5.054637in}}%
\pgfusepath{stroke}%
\end{pgfscope}%
\begin{pgfscope}%
\pgfsetfillopacity{0.800000}%
\pgfsetstrokeopacity{0.800000}%
\definecolor{textcolor}{rgb}{0.000000,0.000000,1.000000}%
\pgfsetstrokecolor{textcolor}%
\pgfsetfillcolor{textcolor}%
\pgftext[x=4.378430in,y=5.220700in,left,base]{\color{textcolor}\sffamily\fontsize{5.647059}{6.776471}\selectfont 13.371(48)}%
\end{pgfscope}%
\begin{pgfscope}%
\pgfsetbuttcap%
\pgfsetroundjoin%
\definecolor{currentfill}{rgb}{0.150000,0.150000,0.150000}%
\pgfsetfillcolor{currentfill}%
\pgfsetlinewidth{1.003750pt}%
\definecolor{currentstroke}{rgb}{0.150000,0.150000,0.150000}%
\pgfsetstrokecolor{currentstroke}%
\pgfsetdash{}{0pt}%
\pgfsys@defobject{currentmarker}{\pgfqpoint{0.000000in}{-0.066667in}}{\pgfqpoint{0.000000in}{0.000000in}}{%
\pgfpathmoveto{\pgfqpoint{0.000000in}{0.000000in}}%
\pgfpathlineto{\pgfqpoint{0.000000in}{-0.066667in}}%
\pgfusepath{stroke,fill}%
}%
\begin{pgfscope}%
\pgfsys@transformshift{3.637789in}{4.812659in}%
\pgfsys@useobject{currentmarker}{}%
\end{pgfscope}%
\end{pgfscope}%
\begin{pgfscope}%
\pgfsetbuttcap%
\pgfsetroundjoin%
\definecolor{currentfill}{rgb}{0.150000,0.150000,0.150000}%
\pgfsetfillcolor{currentfill}%
\pgfsetlinewidth{1.003750pt}%
\definecolor{currentstroke}{rgb}{0.150000,0.150000,0.150000}%
\pgfsetstrokecolor{currentstroke}%
\pgfsetdash{}{0pt}%
\pgfsys@defobject{currentmarker}{\pgfqpoint{0.000000in}{-0.066667in}}{\pgfqpoint{0.000000in}{0.000000in}}{%
\pgfpathmoveto{\pgfqpoint{0.000000in}{0.000000in}}%
\pgfpathlineto{\pgfqpoint{0.000000in}{-0.066667in}}%
\pgfusepath{stroke,fill}%
}%
\begin{pgfscope}%
\pgfsys@transformshift{4.139198in}{4.812659in}%
\pgfsys@useobject{currentmarker}{}%
\end{pgfscope}%
\end{pgfscope}%
\begin{pgfscope}%
\pgfsetbuttcap%
\pgfsetroundjoin%
\definecolor{currentfill}{rgb}{0.150000,0.150000,0.150000}%
\pgfsetfillcolor{currentfill}%
\pgfsetlinewidth{1.003750pt}%
\definecolor{currentstroke}{rgb}{0.150000,0.150000,0.150000}%
\pgfsetstrokecolor{currentstroke}%
\pgfsetdash{}{0pt}%
\pgfsys@defobject{currentmarker}{\pgfqpoint{0.000000in}{-0.066667in}}{\pgfqpoint{0.000000in}{0.000000in}}{%
\pgfpathmoveto{\pgfqpoint{0.000000in}{0.000000in}}%
\pgfpathlineto{\pgfqpoint{0.000000in}{-0.066667in}}%
\pgfusepath{stroke,fill}%
}%
\begin{pgfscope}%
\pgfsys@transformshift{4.640607in}{4.812659in}%
\pgfsys@useobject{currentmarker}{}%
\end{pgfscope}%
\end{pgfscope}%
\begin{pgfscope}%
\pgfsetbuttcap%
\pgfsetroundjoin%
\definecolor{currentfill}{rgb}{0.150000,0.150000,0.150000}%
\pgfsetfillcolor{currentfill}%
\pgfsetlinewidth{0.803000pt}%
\definecolor{currentstroke}{rgb}{0.150000,0.150000,0.150000}%
\pgfsetstrokecolor{currentstroke}%
\pgfsetdash{}{0pt}%
\pgfsys@defobject{currentmarker}{\pgfqpoint{0.000000in}{-0.044444in}}{\pgfqpoint{0.000000in}{0.000000in}}{%
\pgfpathmoveto{\pgfqpoint{0.000000in}{0.000000in}}%
\pgfpathlineto{\pgfqpoint{0.000000in}{-0.044444in}}%
\pgfusepath{stroke,fill}%
}%
\begin{pgfscope}%
\pgfsys@transformshift{3.788728in}{4.812659in}%
\pgfsys@useobject{currentmarker}{}%
\end{pgfscope}%
\end{pgfscope}%
\begin{pgfscope}%
\pgfsetbuttcap%
\pgfsetroundjoin%
\definecolor{currentfill}{rgb}{0.150000,0.150000,0.150000}%
\pgfsetfillcolor{currentfill}%
\pgfsetlinewidth{0.803000pt}%
\definecolor{currentstroke}{rgb}{0.150000,0.150000,0.150000}%
\pgfsetstrokecolor{currentstroke}%
\pgfsetdash{}{0pt}%
\pgfsys@defobject{currentmarker}{\pgfqpoint{0.000000in}{-0.044444in}}{\pgfqpoint{0.000000in}{0.000000in}}{%
\pgfpathmoveto{\pgfqpoint{0.000000in}{0.000000in}}%
\pgfpathlineto{\pgfqpoint{0.000000in}{-0.044444in}}%
\pgfusepath{stroke,fill}%
}%
\begin{pgfscope}%
\pgfsys@transformshift{3.877021in}{4.812659in}%
\pgfsys@useobject{currentmarker}{}%
\end{pgfscope}%
\end{pgfscope}%
\begin{pgfscope}%
\pgfsetbuttcap%
\pgfsetroundjoin%
\definecolor{currentfill}{rgb}{0.150000,0.150000,0.150000}%
\pgfsetfillcolor{currentfill}%
\pgfsetlinewidth{0.803000pt}%
\definecolor{currentstroke}{rgb}{0.150000,0.150000,0.150000}%
\pgfsetstrokecolor{currentstroke}%
\pgfsetdash{}{0pt}%
\pgfsys@defobject{currentmarker}{\pgfqpoint{0.000000in}{-0.044444in}}{\pgfqpoint{0.000000in}{0.000000in}}{%
\pgfpathmoveto{\pgfqpoint{0.000000in}{0.000000in}}%
\pgfpathlineto{\pgfqpoint{0.000000in}{-0.044444in}}%
\pgfusepath{stroke,fill}%
}%
\begin{pgfscope}%
\pgfsys@transformshift{3.939667in}{4.812659in}%
\pgfsys@useobject{currentmarker}{}%
\end{pgfscope}%
\end{pgfscope}%
\begin{pgfscope}%
\pgfsetbuttcap%
\pgfsetroundjoin%
\definecolor{currentfill}{rgb}{0.150000,0.150000,0.150000}%
\pgfsetfillcolor{currentfill}%
\pgfsetlinewidth{0.803000pt}%
\definecolor{currentstroke}{rgb}{0.150000,0.150000,0.150000}%
\pgfsetstrokecolor{currentstroke}%
\pgfsetdash{}{0pt}%
\pgfsys@defobject{currentmarker}{\pgfqpoint{0.000000in}{-0.044444in}}{\pgfqpoint{0.000000in}{0.000000in}}{%
\pgfpathmoveto{\pgfqpoint{0.000000in}{0.000000in}}%
\pgfpathlineto{\pgfqpoint{0.000000in}{-0.044444in}}%
\pgfusepath{stroke,fill}%
}%
\begin{pgfscope}%
\pgfsys@transformshift{3.988258in}{4.812659in}%
\pgfsys@useobject{currentmarker}{}%
\end{pgfscope}%
\end{pgfscope}%
\begin{pgfscope}%
\pgfsetbuttcap%
\pgfsetroundjoin%
\definecolor{currentfill}{rgb}{0.150000,0.150000,0.150000}%
\pgfsetfillcolor{currentfill}%
\pgfsetlinewidth{0.803000pt}%
\definecolor{currentstroke}{rgb}{0.150000,0.150000,0.150000}%
\pgfsetstrokecolor{currentstroke}%
\pgfsetdash{}{0pt}%
\pgfsys@defobject{currentmarker}{\pgfqpoint{0.000000in}{-0.044444in}}{\pgfqpoint{0.000000in}{0.000000in}}{%
\pgfpathmoveto{\pgfqpoint{0.000000in}{0.000000in}}%
\pgfpathlineto{\pgfqpoint{0.000000in}{-0.044444in}}%
\pgfusepath{stroke,fill}%
}%
\begin{pgfscope}%
\pgfsys@transformshift{4.027961in}{4.812659in}%
\pgfsys@useobject{currentmarker}{}%
\end{pgfscope}%
\end{pgfscope}%
\begin{pgfscope}%
\pgfsetbuttcap%
\pgfsetroundjoin%
\definecolor{currentfill}{rgb}{0.150000,0.150000,0.150000}%
\pgfsetfillcolor{currentfill}%
\pgfsetlinewidth{0.803000pt}%
\definecolor{currentstroke}{rgb}{0.150000,0.150000,0.150000}%
\pgfsetstrokecolor{currentstroke}%
\pgfsetdash{}{0pt}%
\pgfsys@defobject{currentmarker}{\pgfqpoint{0.000000in}{-0.044444in}}{\pgfqpoint{0.000000in}{0.000000in}}{%
\pgfpathmoveto{\pgfqpoint{0.000000in}{0.000000in}}%
\pgfpathlineto{\pgfqpoint{0.000000in}{-0.044444in}}%
\pgfusepath{stroke,fill}%
}%
\begin{pgfscope}%
\pgfsys@transformshift{4.061528in}{4.812659in}%
\pgfsys@useobject{currentmarker}{}%
\end{pgfscope}%
\end{pgfscope}%
\begin{pgfscope}%
\pgfsetbuttcap%
\pgfsetroundjoin%
\definecolor{currentfill}{rgb}{0.150000,0.150000,0.150000}%
\pgfsetfillcolor{currentfill}%
\pgfsetlinewidth{0.803000pt}%
\definecolor{currentstroke}{rgb}{0.150000,0.150000,0.150000}%
\pgfsetstrokecolor{currentstroke}%
\pgfsetdash{}{0pt}%
\pgfsys@defobject{currentmarker}{\pgfqpoint{0.000000in}{-0.044444in}}{\pgfqpoint{0.000000in}{0.000000in}}{%
\pgfpathmoveto{\pgfqpoint{0.000000in}{0.000000in}}%
\pgfpathlineto{\pgfqpoint{0.000000in}{-0.044444in}}%
\pgfusepath{stroke,fill}%
}%
\begin{pgfscope}%
\pgfsys@transformshift{4.090606in}{4.812659in}%
\pgfsys@useobject{currentmarker}{}%
\end{pgfscope}%
\end{pgfscope}%
\begin{pgfscope}%
\pgfsetbuttcap%
\pgfsetroundjoin%
\definecolor{currentfill}{rgb}{0.150000,0.150000,0.150000}%
\pgfsetfillcolor{currentfill}%
\pgfsetlinewidth{0.803000pt}%
\definecolor{currentstroke}{rgb}{0.150000,0.150000,0.150000}%
\pgfsetstrokecolor{currentstroke}%
\pgfsetdash{}{0pt}%
\pgfsys@defobject{currentmarker}{\pgfqpoint{0.000000in}{-0.044444in}}{\pgfqpoint{0.000000in}{0.000000in}}{%
\pgfpathmoveto{\pgfqpoint{0.000000in}{0.000000in}}%
\pgfpathlineto{\pgfqpoint{0.000000in}{-0.044444in}}%
\pgfusepath{stroke,fill}%
}%
\begin{pgfscope}%
\pgfsys@transformshift{4.116254in}{4.812659in}%
\pgfsys@useobject{currentmarker}{}%
\end{pgfscope}%
\end{pgfscope}%
\begin{pgfscope}%
\pgfsetbuttcap%
\pgfsetroundjoin%
\definecolor{currentfill}{rgb}{0.150000,0.150000,0.150000}%
\pgfsetfillcolor{currentfill}%
\pgfsetlinewidth{0.803000pt}%
\definecolor{currentstroke}{rgb}{0.150000,0.150000,0.150000}%
\pgfsetstrokecolor{currentstroke}%
\pgfsetdash{}{0pt}%
\pgfsys@defobject{currentmarker}{\pgfqpoint{0.000000in}{-0.044444in}}{\pgfqpoint{0.000000in}{0.000000in}}{%
\pgfpathmoveto{\pgfqpoint{0.000000in}{0.000000in}}%
\pgfpathlineto{\pgfqpoint{0.000000in}{-0.044444in}}%
\pgfusepath{stroke,fill}%
}%
\begin{pgfscope}%
\pgfsys@transformshift{4.290137in}{4.812659in}%
\pgfsys@useobject{currentmarker}{}%
\end{pgfscope}%
\end{pgfscope}%
\begin{pgfscope}%
\pgfsetbuttcap%
\pgfsetroundjoin%
\definecolor{currentfill}{rgb}{0.150000,0.150000,0.150000}%
\pgfsetfillcolor{currentfill}%
\pgfsetlinewidth{0.803000pt}%
\definecolor{currentstroke}{rgb}{0.150000,0.150000,0.150000}%
\pgfsetstrokecolor{currentstroke}%
\pgfsetdash{}{0pt}%
\pgfsys@defobject{currentmarker}{\pgfqpoint{0.000000in}{-0.044444in}}{\pgfqpoint{0.000000in}{0.000000in}}{%
\pgfpathmoveto{\pgfqpoint{0.000000in}{0.000000in}}%
\pgfpathlineto{\pgfqpoint{0.000000in}{-0.044444in}}%
\pgfusepath{stroke,fill}%
}%
\begin{pgfscope}%
\pgfsys@transformshift{4.378430in}{4.812659in}%
\pgfsys@useobject{currentmarker}{}%
\end{pgfscope}%
\end{pgfscope}%
\begin{pgfscope}%
\pgfsetbuttcap%
\pgfsetroundjoin%
\definecolor{currentfill}{rgb}{0.150000,0.150000,0.150000}%
\pgfsetfillcolor{currentfill}%
\pgfsetlinewidth{0.803000pt}%
\definecolor{currentstroke}{rgb}{0.150000,0.150000,0.150000}%
\pgfsetstrokecolor{currentstroke}%
\pgfsetdash{}{0pt}%
\pgfsys@defobject{currentmarker}{\pgfqpoint{0.000000in}{-0.044444in}}{\pgfqpoint{0.000000in}{0.000000in}}{%
\pgfpathmoveto{\pgfqpoint{0.000000in}{0.000000in}}%
\pgfpathlineto{\pgfqpoint{0.000000in}{-0.044444in}}%
\pgfusepath{stroke,fill}%
}%
\begin{pgfscope}%
\pgfsys@transformshift{4.441076in}{4.812659in}%
\pgfsys@useobject{currentmarker}{}%
\end{pgfscope}%
\end{pgfscope}%
\begin{pgfscope}%
\pgfsetbuttcap%
\pgfsetroundjoin%
\definecolor{currentfill}{rgb}{0.150000,0.150000,0.150000}%
\pgfsetfillcolor{currentfill}%
\pgfsetlinewidth{0.803000pt}%
\definecolor{currentstroke}{rgb}{0.150000,0.150000,0.150000}%
\pgfsetstrokecolor{currentstroke}%
\pgfsetdash{}{0pt}%
\pgfsys@defobject{currentmarker}{\pgfqpoint{0.000000in}{-0.044444in}}{\pgfqpoint{0.000000in}{0.000000in}}{%
\pgfpathmoveto{\pgfqpoint{0.000000in}{0.000000in}}%
\pgfpathlineto{\pgfqpoint{0.000000in}{-0.044444in}}%
\pgfusepath{stroke,fill}%
}%
\begin{pgfscope}%
\pgfsys@transformshift{4.489667in}{4.812659in}%
\pgfsys@useobject{currentmarker}{}%
\end{pgfscope}%
\end{pgfscope}%
\begin{pgfscope}%
\pgfsetbuttcap%
\pgfsetroundjoin%
\definecolor{currentfill}{rgb}{0.150000,0.150000,0.150000}%
\pgfsetfillcolor{currentfill}%
\pgfsetlinewidth{0.803000pt}%
\definecolor{currentstroke}{rgb}{0.150000,0.150000,0.150000}%
\pgfsetstrokecolor{currentstroke}%
\pgfsetdash{}{0pt}%
\pgfsys@defobject{currentmarker}{\pgfqpoint{0.000000in}{-0.044444in}}{\pgfqpoint{0.000000in}{0.000000in}}{%
\pgfpathmoveto{\pgfqpoint{0.000000in}{0.000000in}}%
\pgfpathlineto{\pgfqpoint{0.000000in}{-0.044444in}}%
\pgfusepath{stroke,fill}%
}%
\begin{pgfscope}%
\pgfsys@transformshift{4.529370in}{4.812659in}%
\pgfsys@useobject{currentmarker}{}%
\end{pgfscope}%
\end{pgfscope}%
\begin{pgfscope}%
\pgfsetbuttcap%
\pgfsetroundjoin%
\definecolor{currentfill}{rgb}{0.150000,0.150000,0.150000}%
\pgfsetfillcolor{currentfill}%
\pgfsetlinewidth{0.803000pt}%
\definecolor{currentstroke}{rgb}{0.150000,0.150000,0.150000}%
\pgfsetstrokecolor{currentstroke}%
\pgfsetdash{}{0pt}%
\pgfsys@defobject{currentmarker}{\pgfqpoint{0.000000in}{-0.044444in}}{\pgfqpoint{0.000000in}{0.000000in}}{%
\pgfpathmoveto{\pgfqpoint{0.000000in}{0.000000in}}%
\pgfpathlineto{\pgfqpoint{0.000000in}{-0.044444in}}%
\pgfusepath{stroke,fill}%
}%
\begin{pgfscope}%
\pgfsys@transformshift{4.562937in}{4.812659in}%
\pgfsys@useobject{currentmarker}{}%
\end{pgfscope}%
\end{pgfscope}%
\begin{pgfscope}%
\pgfsetbuttcap%
\pgfsetroundjoin%
\definecolor{currentfill}{rgb}{0.150000,0.150000,0.150000}%
\pgfsetfillcolor{currentfill}%
\pgfsetlinewidth{0.803000pt}%
\definecolor{currentstroke}{rgb}{0.150000,0.150000,0.150000}%
\pgfsetstrokecolor{currentstroke}%
\pgfsetdash{}{0pt}%
\pgfsys@defobject{currentmarker}{\pgfqpoint{0.000000in}{-0.044444in}}{\pgfqpoint{0.000000in}{0.000000in}}{%
\pgfpathmoveto{\pgfqpoint{0.000000in}{0.000000in}}%
\pgfpathlineto{\pgfqpoint{0.000000in}{-0.044444in}}%
\pgfusepath{stroke,fill}%
}%
\begin{pgfscope}%
\pgfsys@transformshift{4.592015in}{4.812659in}%
\pgfsys@useobject{currentmarker}{}%
\end{pgfscope}%
\end{pgfscope}%
\begin{pgfscope}%
\pgfsetbuttcap%
\pgfsetroundjoin%
\definecolor{currentfill}{rgb}{0.150000,0.150000,0.150000}%
\pgfsetfillcolor{currentfill}%
\pgfsetlinewidth{0.803000pt}%
\definecolor{currentstroke}{rgb}{0.150000,0.150000,0.150000}%
\pgfsetstrokecolor{currentstroke}%
\pgfsetdash{}{0pt}%
\pgfsys@defobject{currentmarker}{\pgfqpoint{0.000000in}{-0.044444in}}{\pgfqpoint{0.000000in}{0.000000in}}{%
\pgfpathmoveto{\pgfqpoint{0.000000in}{0.000000in}}%
\pgfpathlineto{\pgfqpoint{0.000000in}{-0.044444in}}%
\pgfusepath{stroke,fill}%
}%
\begin{pgfscope}%
\pgfsys@transformshift{4.617663in}{4.812659in}%
\pgfsys@useobject{currentmarker}{}%
\end{pgfscope}%
\end{pgfscope}%
\begin{pgfscope}%
\pgfsetbuttcap%
\pgfsetroundjoin%
\definecolor{currentfill}{rgb}{0.150000,0.150000,0.150000}%
\pgfsetfillcolor{currentfill}%
\pgfsetlinewidth{0.803000pt}%
\definecolor{currentstroke}{rgb}{0.150000,0.150000,0.150000}%
\pgfsetstrokecolor{currentstroke}%
\pgfsetdash{}{0pt}%
\pgfsys@defobject{currentmarker}{\pgfqpoint{0.000000in}{-0.044444in}}{\pgfqpoint{0.000000in}{0.000000in}}{%
\pgfpathmoveto{\pgfqpoint{0.000000in}{0.000000in}}%
\pgfpathlineto{\pgfqpoint{0.000000in}{-0.044444in}}%
\pgfusepath{stroke,fill}%
}%
\begin{pgfscope}%
\pgfsys@transformshift{4.791546in}{4.812659in}%
\pgfsys@useobject{currentmarker}{}%
\end{pgfscope}%
\end{pgfscope}%
\begin{pgfscope}%
\pgfsetbuttcap%
\pgfsetroundjoin%
\definecolor{currentfill}{rgb}{0.150000,0.150000,0.150000}%
\pgfsetfillcolor{currentfill}%
\pgfsetlinewidth{1.003750pt}%
\definecolor{currentstroke}{rgb}{0.150000,0.150000,0.150000}%
\pgfsetstrokecolor{currentstroke}%
\pgfsetdash{}{0pt}%
\pgfsys@defobject{currentmarker}{\pgfqpoint{-0.066667in}{0.000000in}}{\pgfqpoint{0.000000in}{0.000000in}}{%
\pgfpathmoveto{\pgfqpoint{0.000000in}{0.000000in}}%
\pgfpathlineto{\pgfqpoint{-0.066667in}{0.000000in}}%
\pgfusepath{stroke,fill}%
}%
\begin{pgfscope}%
\pgfsys@transformshift{3.637789in}{4.812659in}%
\pgfsys@useobject{currentmarker}{}%
\end{pgfscope}%
\end{pgfscope}%
\begin{pgfscope}%
\pgfsetbuttcap%
\pgfsetroundjoin%
\definecolor{currentfill}{rgb}{0.150000,0.150000,0.150000}%
\pgfsetfillcolor{currentfill}%
\pgfsetlinewidth{1.003750pt}%
\definecolor{currentstroke}{rgb}{0.150000,0.150000,0.150000}%
\pgfsetstrokecolor{currentstroke}%
\pgfsetdash{}{0pt}%
\pgfsys@defobject{currentmarker}{\pgfqpoint{-0.066667in}{0.000000in}}{\pgfqpoint{0.000000in}{0.000000in}}{%
\pgfpathmoveto{\pgfqpoint{0.000000in}{0.000000in}}%
\pgfpathlineto{\pgfqpoint{-0.066667in}{0.000000in}}%
\pgfusepath{stroke,fill}%
}%
\begin{pgfscope}%
\pgfsys@transformshift{3.637789in}{5.045573in}%
\pgfsys@useobject{currentmarker}{}%
\end{pgfscope}%
\end{pgfscope}%
\begin{pgfscope}%
\pgfsetbuttcap%
\pgfsetroundjoin%
\definecolor{currentfill}{rgb}{0.150000,0.150000,0.150000}%
\pgfsetfillcolor{currentfill}%
\pgfsetlinewidth{1.003750pt}%
\definecolor{currentstroke}{rgb}{0.150000,0.150000,0.150000}%
\pgfsetstrokecolor{currentstroke}%
\pgfsetdash{}{0pt}%
\pgfsys@defobject{currentmarker}{\pgfqpoint{-0.066667in}{0.000000in}}{\pgfqpoint{0.000000in}{0.000000in}}{%
\pgfpathmoveto{\pgfqpoint{0.000000in}{0.000000in}}%
\pgfpathlineto{\pgfqpoint{-0.066667in}{0.000000in}}%
\pgfusepath{stroke,fill}%
}%
\begin{pgfscope}%
\pgfsys@transformshift{3.637789in}{5.420607in}%
\pgfsys@useobject{currentmarker}{}%
\end{pgfscope}%
\end{pgfscope}%
\begin{pgfscope}%
\pgfpathrectangle{\pgfqpoint{3.637789in}{4.812659in}}{\pgfqpoint{1.223103in}{0.607948in}}%
\pgfusepath{clip}%
\pgfsetroundcap%
\pgfsetroundjoin%
\pgfsetlinewidth{1.204500pt}%
\definecolor{currentstroke}{rgb}{0.000000,0.501961,0.000000}%
\pgfsetstrokecolor{currentstroke}%
\pgfsetdash{}{0pt}%
\pgfpathmoveto{\pgfqpoint{3.637789in}{5.039120in}}%
\pgfpathlineto{\pgfqpoint{3.878498in}{5.040737in}}%
\pgfpathlineto{\pgfqpoint{3.990029in}{5.042344in}}%
\pgfpathlineto{\pgfqpoint{4.063425in}{5.043939in}}%
\pgfpathlineto{\pgfqpoint{4.118221in}{5.045521in}}%
\pgfpathlineto{\pgfqpoint{4.161963in}{5.047088in}}%
\pgfpathlineto{\pgfqpoint{4.198371in}{5.048638in}}%
\pgfpathlineto{\pgfqpoint{4.229555in}{5.050168in}}%
\pgfpathlineto{\pgfqpoint{4.256828in}{5.051678in}}%
\pgfpathlineto{\pgfqpoint{4.281062in}{5.053162in}}%
\pgfpathlineto{\pgfqpoint{4.302867in}{5.054619in}}%
\pgfpathlineto{\pgfqpoint{4.322686in}{5.056045in}}%
\pgfpathlineto{\pgfqpoint{4.340851in}{5.057434in}}%
\pgfpathlineto{\pgfqpoint{4.357617in}{5.058782in}}%
\pgfpathlineto{\pgfqpoint{4.373183in}{5.060083in}}%
\pgfpathlineto{\pgfqpoint{4.387711in}{5.061331in}}%
\pgfpathlineto{\pgfqpoint{4.401329in}{5.062517in}}%
\pgfpathlineto{\pgfqpoint{4.414146in}{5.063633in}}%
\pgfpathlineto{\pgfqpoint{4.426250in}{5.064671in}}%
\pgfpathlineto{\pgfqpoint{4.437717in}{5.065618in}}%
\pgfpathlineto{\pgfqpoint{4.448610in}{5.066464in}}%
\pgfpathlineto{\pgfqpoint{4.458984in}{5.067195in}}%
\pgfpathlineto{\pgfqpoint{4.468886in}{5.067796in}}%
\pgfpathlineto{\pgfqpoint{4.478357in}{5.068254in}}%
\pgfpathlineto{\pgfqpoint{4.487434in}{5.068549in}}%
\pgfpathlineto{\pgfqpoint{4.496147in}{5.068663in}}%
\pgfpathlineto{\pgfqpoint{4.504525in}{5.068577in}}%
\pgfpathlineto{\pgfqpoint{4.512593in}{5.068269in}}%
\pgfpathlineto{\pgfqpoint{4.520372in}{5.067713in}}%
\pgfpathlineto{\pgfqpoint{4.527883in}{5.066887in}}%
\pgfpathlineto{\pgfqpoint{4.535144in}{5.065761in}}%
\pgfpathlineto{\pgfqpoint{4.542170in}{5.064306in}}%
\pgfpathlineto{\pgfqpoint{4.548977in}{5.062493in}}%
\pgfpathlineto{\pgfqpoint{4.555577in}{5.060286in}}%
\pgfpathlineto{\pgfqpoint{4.561983in}{5.057651in}}%
\pgfpathlineto{\pgfqpoint{4.568206in}{5.054551in}}%
\pgfpathlineto{\pgfqpoint{4.574256in}{5.050944in}}%
\pgfpathlineto{\pgfqpoint{4.580143in}{5.046789in}}%
\pgfpathlineto{\pgfqpoint{4.585874in}{5.042041in}}%
\pgfpathlineto{\pgfqpoint{4.591459in}{5.036653in}}%
\pgfpathlineto{\pgfqpoint{4.596904in}{5.030575in}}%
\pgfpathlineto{\pgfqpoint{4.602216in}{5.023754in}}%
\pgfpathlineto{\pgfqpoint{4.607401in}{5.016136in}}%
\pgfpathlineto{\pgfqpoint{4.612466in}{5.007662in}}%
\pgfpathlineto{\pgfqpoint{4.617416in}{4.998272in}}%
\pgfpathlineto{\pgfqpoint{4.622256in}{4.987902in}}%
\pgfpathlineto{\pgfqpoint{4.626991in}{4.976484in}}%
\pgfpathlineto{\pgfqpoint{4.631625in}{4.963950in}}%
\pgfpathlineto{\pgfqpoint{4.636162in}{4.950227in}}%
\pgfpathlineto{\pgfqpoint{4.640607in}{4.935238in}}%
\pgfusepath{stroke}%
\end{pgfscope}%
\begin{pgfscope}%
\pgfsetrectcap%
\pgfsetmiterjoin%
\pgfsetlinewidth{1.003750pt}%
\definecolor{currentstroke}{rgb}{0.150000,0.150000,0.150000}%
\pgfsetstrokecolor{currentstroke}%
\pgfsetdash{}{0pt}%
\pgfpathmoveto{\pgfqpoint{3.637789in}{4.812659in}}%
\pgfpathlineto{\pgfqpoint{3.637789in}{5.420607in}}%
\pgfusepath{stroke}%
\end{pgfscope}%
\begin{pgfscope}%
\pgfsetrectcap%
\pgfsetmiterjoin%
\pgfsetlinewidth{1.003750pt}%
\definecolor{currentstroke}{rgb}{0.150000,0.150000,0.150000}%
\pgfsetstrokecolor{currentstroke}%
\pgfsetdash{}{0pt}%
\pgfpathmoveto{\pgfqpoint{3.637789in}{4.812659in}}%
\pgfpathlineto{\pgfqpoint{4.860892in}{4.812659in}}%
\pgfusepath{stroke}%
\end{pgfscope}%
\begin{pgfscope}%
\pgfpathrectangle{\pgfqpoint{3.637789in}{4.812659in}}{\pgfqpoint{1.223103in}{0.607948in}}%
\pgfusepath{clip}%
\pgfsetbuttcap%
\pgfsetroundjoin%
\definecolor{currentfill}{rgb}{0.000000,0.000000,0.000000}%
\pgfsetfillcolor{currentfill}%
\pgfsetlinewidth{1.003750pt}%
\definecolor{currentstroke}{rgb}{0.000000,0.000000,0.000000}%
\pgfsetstrokecolor{currentstroke}%
\pgfsetdash{}{0pt}%
\pgfsys@defobject{currentmarker}{\pgfqpoint{-0.013889in}{-0.013889in}}{\pgfqpoint{0.013889in}{0.013889in}}{%
\pgfpathmoveto{\pgfqpoint{0.000000in}{-0.013889in}}%
\pgfpathcurveto{\pgfqpoint{0.003683in}{-0.013889in}}{\pgfqpoint{0.007216in}{-0.012425in}}{\pgfqpoint{0.009821in}{-0.009821in}}%
\pgfpathcurveto{\pgfqpoint{0.012425in}{-0.007216in}}{\pgfqpoint{0.013889in}{-0.003683in}}{\pgfqpoint{0.013889in}{0.000000in}}%
\pgfpathcurveto{\pgfqpoint{0.013889in}{0.003683in}}{\pgfqpoint{0.012425in}{0.007216in}}{\pgfqpoint{0.009821in}{0.009821in}}%
\pgfpathcurveto{\pgfqpoint{0.007216in}{0.012425in}}{\pgfqpoint{0.003683in}{0.013889in}}{\pgfqpoint{0.000000in}{0.013889in}}%
\pgfpathcurveto{\pgfqpoint{-0.003683in}{0.013889in}}{\pgfqpoint{-0.007216in}{0.012425in}}{\pgfqpoint{-0.009821in}{0.009821in}}%
\pgfpathcurveto{\pgfqpoint{-0.012425in}{0.007216in}}{\pgfqpoint{-0.013889in}{0.003683in}}{\pgfqpoint{-0.013889in}{0.000000in}}%
\pgfpathcurveto{\pgfqpoint{-0.013889in}{-0.003683in}}{\pgfqpoint{-0.012425in}{-0.007216in}}{\pgfqpoint{-0.009821in}{-0.009821in}}%
\pgfpathcurveto{\pgfqpoint{-0.007216in}{-0.012425in}}{\pgfqpoint{-0.003683in}{-0.013889in}}{\pgfqpoint{0.000000in}{-0.013889in}}%
\pgfpathclose%
\pgfusepath{stroke,fill}%
}%
\begin{pgfscope}%
\pgfsys@transformshift{4.640607in}{4.934049in}%
\pgfsys@useobject{currentmarker}{}%
\end{pgfscope}%
\begin{pgfscope}%
\pgfsys@transformshift{4.290137in}{5.054628in}%
\pgfsys@useobject{currentmarker}{}%
\end{pgfscope}%
\begin{pgfscope}%
\pgfsys@transformshift{4.294536in}{5.054813in}%
\pgfsys@useobject{currentmarker}{}%
\end{pgfscope}%
\begin{pgfscope}%
\pgfsys@transformshift{4.299026in}{5.055006in}%
\pgfsys@useobject{currentmarker}{}%
\end{pgfscope}%
\begin{pgfscope}%
\pgfsys@transformshift{4.303611in}{5.055207in}%
\pgfsys@useobject{currentmarker}{}%
\end{pgfscope}%
\begin{pgfscope}%
\pgfsys@transformshift{4.308294in}{5.055418in}%
\pgfsys@useobject{currentmarker}{}%
\end{pgfscope}%
\begin{pgfscope}%
\pgfsys@transformshift{4.313080in}{5.055637in}%
\pgfsys@useobject{currentmarker}{}%
\end{pgfscope}%
\begin{pgfscope}%
\pgfsys@transformshift{4.317974in}{5.055866in}%
\pgfsys@useobject{currentmarker}{}%
\end{pgfscope}%
\begin{pgfscope}%
\pgfsys@transformshift{4.322980in}{5.056106in}%
\pgfsys@useobject{currentmarker}{}%
\end{pgfscope}%
\begin{pgfscope}%
\pgfsys@transformshift{4.328104in}{5.056357in}%
\pgfsys@useobject{currentmarker}{}%
\end{pgfscope}%
\begin{pgfscope}%
\pgfsys@transformshift{4.333351in}{5.056620in}%
\pgfsys@useobject{currentmarker}{}%
\end{pgfscope}%
\begin{pgfscope}%
\pgfsys@transformshift{4.338728in}{5.056896in}%
\pgfsys@useobject{currentmarker}{}%
\end{pgfscope}%
\begin{pgfscope}%
\pgfsys@transformshift{4.367806in}{5.058505in}%
\pgfsys@useobject{currentmarker}{}%
\end{pgfscope}%
\begin{pgfscope}%
\pgfsys@transformshift{4.401374in}{5.060619in}%
\pgfsys@useobject{currentmarker}{}%
\end{pgfscope}%
\begin{pgfscope}%
\pgfsys@transformshift{4.441076in}{5.063439in}%
\pgfsys@useobject{currentmarker}{}%
\end{pgfscope}%
\begin{pgfscope}%
\pgfsys@transformshift{4.552313in}{5.065611in}%
\pgfsys@useobject{currentmarker}{}%
\end{pgfscope}%
\begin{pgfscope}%
\pgfsys@transformshift{4.489667in}{5.066878in}%
\pgfsys@useobject{currentmarker}{}%
\end{pgfscope}%
\end{pgfscope}%
\begin{pgfscope}%
\pgfsetbuttcap%
\pgfsetmiterjoin%
\definecolor{currentfill}{rgb}{1.000000,1.000000,1.000000}%
\pgfsetfillcolor{currentfill}%
\pgfsetlinewidth{0.000000pt}%
\definecolor{currentstroke}{rgb}{0.000000,0.000000,0.000000}%
\pgfsetstrokecolor{currentstroke}%
\pgfsetstrokeopacity{0.000000}%
\pgfsetdash{}{0pt}%
\pgfpathmoveto{\pgfqpoint{5.105513in}{4.812659in}}%
\pgfpathlineto{\pgfqpoint{6.328616in}{4.812659in}}%
\pgfpathlineto{\pgfqpoint{6.328616in}{5.420607in}}%
\pgfpathlineto{\pgfqpoint{5.105513in}{5.420607in}}%
\pgfpathclose%
\pgfusepath{fill}%
\end{pgfscope}%
\begin{pgfscope}%
\pgfpathrectangle{\pgfqpoint{5.105513in}{4.812659in}}{\pgfqpoint{1.223103in}{0.607948in}}%
\pgfusepath{clip}%
\pgfsetbuttcap%
\pgfsetmiterjoin%
\definecolor{currentfill}{rgb}{0.000000,0.000000,1.000000}%
\pgfsetfillcolor{currentfill}%
\pgfsetfillopacity{0.100000}%
\pgfsetlinewidth{0.803000pt}%
\definecolor{currentstroke}{rgb}{0.000000,0.000000,1.000000}%
\pgfsetstrokecolor{currentstroke}%
\pgfsetstrokeopacity{0.100000}%
\pgfsetdash{}{0pt}%
\pgfpathmoveto{\pgfqpoint{5.105513in}{5.044103in}}%
\pgfpathlineto{\pgfqpoint{5.105513in}{5.047101in}}%
\pgfpathlineto{\pgfqpoint{6.328616in}{5.047101in}}%
\pgfpathlineto{\pgfqpoint{6.328616in}{5.044103in}}%
\pgfpathclose%
\pgfusepath{stroke,fill}%
\end{pgfscope}%
\begin{pgfscope}%
\pgfpathrectangle{\pgfqpoint{5.105513in}{4.812659in}}{\pgfqpoint{1.223103in}{0.607948in}}%
\pgfusepath{clip}%
\pgfsetbuttcap%
\pgfsetroundjoin%
\definecolor{currentfill}{rgb}{0.000000,0.501961,0.000000}%
\pgfsetfillcolor{currentfill}%
\pgfsetfillopacity{0.500000}%
\pgfsetlinewidth{0.803000pt}%
\definecolor{currentstroke}{rgb}{0.000000,0.501961,0.000000}%
\pgfsetstrokecolor{currentstroke}%
\pgfsetstrokeopacity{0.500000}%
\pgfsetdash{}{0pt}%
\pgfpathmoveto{\pgfqpoint{5.105513in}{5.047517in}}%
\pgfpathlineto{\pgfqpoint{5.105513in}{5.044814in}}%
\pgfpathlineto{\pgfqpoint{5.346222in}{5.046202in}}%
\pgfpathlineto{\pgfqpoint{5.457753in}{5.047538in}}%
\pgfpathlineto{\pgfqpoint{5.531149in}{5.048831in}}%
\pgfpathlineto{\pgfqpoint{5.585945in}{5.050088in}}%
\pgfpathlineto{\pgfqpoint{5.629687in}{5.051317in}}%
\pgfpathlineto{\pgfqpoint{5.666095in}{5.052524in}}%
\pgfpathlineto{\pgfqpoint{5.697279in}{5.053716in}}%
\pgfpathlineto{\pgfqpoint{5.724552in}{5.054896in}}%
\pgfpathlineto{\pgfqpoint{5.748786in}{5.056070in}}%
\pgfpathlineto{\pgfqpoint{5.770591in}{5.057239in}}%
\pgfpathlineto{\pgfqpoint{5.790410in}{5.058400in}}%
\pgfpathlineto{\pgfqpoint{5.808575in}{5.059507in}}%
\pgfpathlineto{\pgfqpoint{5.825341in}{5.060624in}}%
\pgfpathlineto{\pgfqpoint{5.840907in}{5.061756in}}%
\pgfpathlineto{\pgfqpoint{5.855435in}{5.062903in}}%
\pgfpathlineto{\pgfqpoint{5.869053in}{5.064065in}}%
\pgfpathlineto{\pgfqpoint{5.881870in}{5.065243in}}%
\pgfpathlineto{\pgfqpoint{5.893974in}{5.066437in}}%
\pgfpathlineto{\pgfqpoint{5.905441in}{5.067645in}}%
\pgfpathlineto{\pgfqpoint{5.916334in}{5.068869in}}%
\pgfpathlineto{\pgfqpoint{5.926708in}{5.070107in}}%
\pgfpathlineto{\pgfqpoint{5.936610in}{5.071361in}}%
\pgfpathlineto{\pgfqpoint{5.946082in}{5.072628in}}%
\pgfpathlineto{\pgfqpoint{5.955158in}{5.073908in}}%
\pgfpathlineto{\pgfqpoint{5.963871in}{5.075199in}}%
\pgfpathlineto{\pgfqpoint{5.972249in}{5.076496in}}%
\pgfpathlineto{\pgfqpoint{5.980317in}{5.077794in}}%
\pgfpathlineto{\pgfqpoint{5.988096in}{5.079094in}}%
\pgfpathlineto{\pgfqpoint{5.995607in}{5.080401in}}%
\pgfpathlineto{\pgfqpoint{6.002868in}{5.081716in}}%
\pgfpathlineto{\pgfqpoint{6.009894in}{5.083044in}}%
\pgfpathlineto{\pgfqpoint{6.016701in}{5.084384in}}%
\pgfpathlineto{\pgfqpoint{6.023301in}{5.085740in}}%
\pgfpathlineto{\pgfqpoint{6.029707in}{5.087113in}}%
\pgfpathlineto{\pgfqpoint{6.035930in}{5.088506in}}%
\pgfpathlineto{\pgfqpoint{6.041980in}{5.089921in}}%
\pgfpathlineto{\pgfqpoint{6.047867in}{5.091363in}}%
\pgfpathlineto{\pgfqpoint{6.053598in}{5.092834in}}%
\pgfpathlineto{\pgfqpoint{6.059183in}{5.094341in}}%
\pgfpathlineto{\pgfqpoint{6.064628in}{5.095886in}}%
\pgfpathlineto{\pgfqpoint{6.069940in}{5.097478in}}%
\pgfpathlineto{\pgfqpoint{6.075126in}{5.099122in}}%
\pgfpathlineto{\pgfqpoint{6.080191in}{5.100827in}}%
\pgfpathlineto{\pgfqpoint{6.085140in}{5.102599in}}%
\pgfpathlineto{\pgfqpoint{6.089980in}{5.104449in}}%
\pgfpathlineto{\pgfqpoint{6.094715in}{5.106387in}}%
\pgfpathlineto{\pgfqpoint{6.099349in}{5.108423in}}%
\pgfpathlineto{\pgfqpoint{6.103886in}{5.110566in}}%
\pgfpathlineto{\pgfqpoint{6.108331in}{5.112689in}}%
\pgfpathlineto{\pgfqpoint{6.108331in}{5.112875in}}%
\pgfpathlineto{\pgfqpoint{6.108331in}{5.112875in}}%
\pgfpathlineto{\pgfqpoint{6.103886in}{5.110888in}}%
\pgfpathlineto{\pgfqpoint{6.099349in}{5.109067in}}%
\pgfpathlineto{\pgfqpoint{6.094715in}{5.107274in}}%
\pgfpathlineto{\pgfqpoint{6.089980in}{5.105506in}}%
\pgfpathlineto{\pgfqpoint{6.085140in}{5.103764in}}%
\pgfpathlineto{\pgfqpoint{6.080191in}{5.102047in}}%
\pgfpathlineto{\pgfqpoint{6.075126in}{5.100357in}}%
\pgfpathlineto{\pgfqpoint{6.069940in}{5.098692in}}%
\pgfpathlineto{\pgfqpoint{6.064628in}{5.097053in}}%
\pgfpathlineto{\pgfqpoint{6.059183in}{5.095440in}}%
\pgfpathlineto{\pgfqpoint{6.053598in}{5.093853in}}%
\pgfpathlineto{\pgfqpoint{6.047867in}{5.092291in}}%
\pgfpathlineto{\pgfqpoint{6.041980in}{5.090753in}}%
\pgfpathlineto{\pgfqpoint{6.035930in}{5.089240in}}%
\pgfpathlineto{\pgfqpoint{6.029707in}{5.087750in}}%
\pgfpathlineto{\pgfqpoint{6.023301in}{5.086282in}}%
\pgfpathlineto{\pgfqpoint{6.016701in}{5.084837in}}%
\pgfpathlineto{\pgfqpoint{6.009894in}{5.083412in}}%
\pgfpathlineto{\pgfqpoint{6.002868in}{5.082009in}}%
\pgfpathlineto{\pgfqpoint{5.995607in}{5.080626in}}%
\pgfpathlineto{\pgfqpoint{5.988096in}{5.079263in}}%
\pgfpathlineto{\pgfqpoint{5.980317in}{5.077924in}}%
\pgfpathlineto{\pgfqpoint{5.972249in}{5.076610in}}%
\pgfpathlineto{\pgfqpoint{5.963871in}{5.075322in}}%
\pgfpathlineto{\pgfqpoint{5.955158in}{5.074053in}}%
\pgfpathlineto{\pgfqpoint{5.946082in}{5.072799in}}%
\pgfpathlineto{\pgfqpoint{5.936610in}{5.071557in}}%
\pgfpathlineto{\pgfqpoint{5.926708in}{5.070324in}}%
\pgfpathlineto{\pgfqpoint{5.916334in}{5.069101in}}%
\pgfpathlineto{\pgfqpoint{5.905441in}{5.067887in}}%
\pgfpathlineto{\pgfqpoint{5.893974in}{5.066680in}}%
\pgfpathlineto{\pgfqpoint{5.881870in}{5.065481in}}%
\pgfpathlineto{\pgfqpoint{5.869053in}{5.064289in}}%
\pgfpathlineto{\pgfqpoint{5.855435in}{5.063104in}}%
\pgfpathlineto{\pgfqpoint{5.840907in}{5.061924in}}%
\pgfpathlineto{\pgfqpoint{5.825341in}{5.060749in}}%
\pgfpathlineto{\pgfqpoint{5.808575in}{5.059577in}}%
\pgfpathlineto{\pgfqpoint{5.790410in}{5.058414in}}%
\pgfpathlineto{\pgfqpoint{5.770591in}{5.057321in}}%
\pgfpathlineto{\pgfqpoint{5.748786in}{5.056251in}}%
\pgfpathlineto{\pgfqpoint{5.724552in}{5.055197in}}%
\pgfpathlineto{\pgfqpoint{5.697279in}{5.054160in}}%
\pgfpathlineto{\pgfqpoint{5.666095in}{5.053141in}}%
\pgfpathlineto{\pgfqpoint{5.629687in}{5.052142in}}%
\pgfpathlineto{\pgfqpoint{5.585945in}{5.051164in}}%
\pgfpathlineto{\pgfqpoint{5.531149in}{5.050210in}}%
\pgfpathlineto{\pgfqpoint{5.457753in}{5.049282in}}%
\pgfpathlineto{\pgfqpoint{5.346222in}{5.048383in}}%
\pgfpathlineto{\pgfqpoint{5.105513in}{5.047517in}}%
\pgfpathclose%
\pgfusepath{stroke,fill}%
\end{pgfscope}%
\begin{pgfscope}%
\pgfpathrectangle{\pgfqpoint{5.105513in}{4.812659in}}{\pgfqpoint{1.223103in}{0.607948in}}%
\pgfusepath{clip}%
\pgfsetroundcap%
\pgfsetroundjoin%
\pgfsetlinewidth{0.501875pt}%
\definecolor{currentstroke}{rgb}{0.000000,0.000000,1.000000}%
\pgfsetstrokecolor{currentstroke}%
\pgfsetstrokeopacity{0.800000}%
\pgfsetdash{}{0pt}%
\pgfpathmoveto{\pgfqpoint{5.105513in}{5.045602in}}%
\pgfpathlineto{\pgfqpoint{6.328616in}{5.045602in}}%
\pgfusepath{stroke}%
\end{pgfscope}%
\begin{pgfscope}%
\pgfpathrectangle{\pgfqpoint{5.105513in}{4.812659in}}{\pgfqpoint{1.223103in}{0.607948in}}%
\pgfusepath{clip}%
\pgfsetbuttcap%
\pgfsetroundjoin%
\pgfsetlinewidth{1.003750pt}%
\definecolor{currentstroke}{rgb}{0.000000,0.000000,0.000000}%
\pgfsetstrokecolor{currentstroke}%
\pgfsetdash{{3.700000pt}{1.600000pt}}{0.000000pt}%
\pgfpathmoveto{\pgfqpoint{5.105513in}{5.045573in}}%
\pgfpathlineto{\pgfqpoint{6.328616in}{5.045573in}}%
\pgfusepath{stroke}%
\end{pgfscope}%
\begin{pgfscope}%
\pgfsetroundcap%
\pgfsetroundjoin%
\pgfsetlinewidth{0.501875pt}%
\definecolor{currentstroke}{rgb}{0.000000,0.000000,1.000000}%
\pgfsetstrokecolor{currentstroke}%
\pgfsetstrokeopacity{0.800000}%
\pgfsetdash{}{0pt}%
\pgfpathmoveto{\pgfqpoint{5.932592in}{5.163706in}}%
\pgfpathquadraticcurveto{\pgfqpoint{5.856728in}{5.112429in}}{\pgfqpoint{5.780865in}{5.061151in}}%
\pgfusepath{stroke}%
\end{pgfscope}%
\begin{pgfscope}%
\pgfsetfillopacity{0.800000}%
\pgfsetstrokeopacity{0.800000}%
\definecolor{textcolor}{rgb}{0.000000,0.000000,1.000000}%
\pgfsetstrokecolor{textcolor}%
\pgfsetfillcolor{textcolor}%
\pgftext[x=5.846155in,y=5.227987in,left,base]{\color{textcolor}\sffamily\fontsize{5.647059}{6.776471}\selectfont 13.3832(25)}%
\end{pgfscope}%
\begin{pgfscope}%
\pgfsetbuttcap%
\pgfsetroundjoin%
\definecolor{currentfill}{rgb}{0.150000,0.150000,0.150000}%
\pgfsetfillcolor{currentfill}%
\pgfsetlinewidth{1.003750pt}%
\definecolor{currentstroke}{rgb}{0.150000,0.150000,0.150000}%
\pgfsetstrokecolor{currentstroke}%
\pgfsetdash{}{0pt}%
\pgfsys@defobject{currentmarker}{\pgfqpoint{0.000000in}{-0.066667in}}{\pgfqpoint{0.000000in}{0.000000in}}{%
\pgfpathmoveto{\pgfqpoint{0.000000in}{0.000000in}}%
\pgfpathlineto{\pgfqpoint{0.000000in}{-0.066667in}}%
\pgfusepath{stroke,fill}%
}%
\begin{pgfscope}%
\pgfsys@transformshift{5.105513in}{4.812659in}%
\pgfsys@useobject{currentmarker}{}%
\end{pgfscope}%
\end{pgfscope}%
\begin{pgfscope}%
\pgfsetbuttcap%
\pgfsetroundjoin%
\definecolor{currentfill}{rgb}{0.150000,0.150000,0.150000}%
\pgfsetfillcolor{currentfill}%
\pgfsetlinewidth{1.003750pt}%
\definecolor{currentstroke}{rgb}{0.150000,0.150000,0.150000}%
\pgfsetstrokecolor{currentstroke}%
\pgfsetdash{}{0pt}%
\pgfsys@defobject{currentmarker}{\pgfqpoint{0.000000in}{-0.066667in}}{\pgfqpoint{0.000000in}{0.000000in}}{%
\pgfpathmoveto{\pgfqpoint{0.000000in}{0.000000in}}%
\pgfpathlineto{\pgfqpoint{0.000000in}{-0.066667in}}%
\pgfusepath{stroke,fill}%
}%
\begin{pgfscope}%
\pgfsys@transformshift{5.606922in}{4.812659in}%
\pgfsys@useobject{currentmarker}{}%
\end{pgfscope}%
\end{pgfscope}%
\begin{pgfscope}%
\pgfsetbuttcap%
\pgfsetroundjoin%
\definecolor{currentfill}{rgb}{0.150000,0.150000,0.150000}%
\pgfsetfillcolor{currentfill}%
\pgfsetlinewidth{1.003750pt}%
\definecolor{currentstroke}{rgb}{0.150000,0.150000,0.150000}%
\pgfsetstrokecolor{currentstroke}%
\pgfsetdash{}{0pt}%
\pgfsys@defobject{currentmarker}{\pgfqpoint{0.000000in}{-0.066667in}}{\pgfqpoint{0.000000in}{0.000000in}}{%
\pgfpathmoveto{\pgfqpoint{0.000000in}{0.000000in}}%
\pgfpathlineto{\pgfqpoint{0.000000in}{-0.066667in}}%
\pgfusepath{stroke,fill}%
}%
\begin{pgfscope}%
\pgfsys@transformshift{6.108331in}{4.812659in}%
\pgfsys@useobject{currentmarker}{}%
\end{pgfscope}%
\end{pgfscope}%
\begin{pgfscope}%
\pgfsetbuttcap%
\pgfsetroundjoin%
\definecolor{currentfill}{rgb}{0.150000,0.150000,0.150000}%
\pgfsetfillcolor{currentfill}%
\pgfsetlinewidth{0.803000pt}%
\definecolor{currentstroke}{rgb}{0.150000,0.150000,0.150000}%
\pgfsetstrokecolor{currentstroke}%
\pgfsetdash{}{0pt}%
\pgfsys@defobject{currentmarker}{\pgfqpoint{0.000000in}{-0.044444in}}{\pgfqpoint{0.000000in}{0.000000in}}{%
\pgfpathmoveto{\pgfqpoint{0.000000in}{0.000000in}}%
\pgfpathlineto{\pgfqpoint{0.000000in}{-0.044444in}}%
\pgfusepath{stroke,fill}%
}%
\begin{pgfscope}%
\pgfsys@transformshift{5.256452in}{4.812659in}%
\pgfsys@useobject{currentmarker}{}%
\end{pgfscope}%
\end{pgfscope}%
\begin{pgfscope}%
\pgfsetbuttcap%
\pgfsetroundjoin%
\definecolor{currentfill}{rgb}{0.150000,0.150000,0.150000}%
\pgfsetfillcolor{currentfill}%
\pgfsetlinewidth{0.803000pt}%
\definecolor{currentstroke}{rgb}{0.150000,0.150000,0.150000}%
\pgfsetstrokecolor{currentstroke}%
\pgfsetdash{}{0pt}%
\pgfsys@defobject{currentmarker}{\pgfqpoint{0.000000in}{-0.044444in}}{\pgfqpoint{0.000000in}{0.000000in}}{%
\pgfpathmoveto{\pgfqpoint{0.000000in}{0.000000in}}%
\pgfpathlineto{\pgfqpoint{0.000000in}{-0.044444in}}%
\pgfusepath{stroke,fill}%
}%
\begin{pgfscope}%
\pgfsys@transformshift{5.344746in}{4.812659in}%
\pgfsys@useobject{currentmarker}{}%
\end{pgfscope}%
\end{pgfscope}%
\begin{pgfscope}%
\pgfsetbuttcap%
\pgfsetroundjoin%
\definecolor{currentfill}{rgb}{0.150000,0.150000,0.150000}%
\pgfsetfillcolor{currentfill}%
\pgfsetlinewidth{0.803000pt}%
\definecolor{currentstroke}{rgb}{0.150000,0.150000,0.150000}%
\pgfsetstrokecolor{currentstroke}%
\pgfsetdash{}{0pt}%
\pgfsys@defobject{currentmarker}{\pgfqpoint{0.000000in}{-0.044444in}}{\pgfqpoint{0.000000in}{0.000000in}}{%
\pgfpathmoveto{\pgfqpoint{0.000000in}{0.000000in}}%
\pgfpathlineto{\pgfqpoint{0.000000in}{-0.044444in}}%
\pgfusepath{stroke,fill}%
}%
\begin{pgfscope}%
\pgfsys@transformshift{5.407391in}{4.812659in}%
\pgfsys@useobject{currentmarker}{}%
\end{pgfscope}%
\end{pgfscope}%
\begin{pgfscope}%
\pgfsetbuttcap%
\pgfsetroundjoin%
\definecolor{currentfill}{rgb}{0.150000,0.150000,0.150000}%
\pgfsetfillcolor{currentfill}%
\pgfsetlinewidth{0.803000pt}%
\definecolor{currentstroke}{rgb}{0.150000,0.150000,0.150000}%
\pgfsetstrokecolor{currentstroke}%
\pgfsetdash{}{0pt}%
\pgfsys@defobject{currentmarker}{\pgfqpoint{0.000000in}{-0.044444in}}{\pgfqpoint{0.000000in}{0.000000in}}{%
\pgfpathmoveto{\pgfqpoint{0.000000in}{0.000000in}}%
\pgfpathlineto{\pgfqpoint{0.000000in}{-0.044444in}}%
\pgfusepath{stroke,fill}%
}%
\begin{pgfscope}%
\pgfsys@transformshift{5.455982in}{4.812659in}%
\pgfsys@useobject{currentmarker}{}%
\end{pgfscope}%
\end{pgfscope}%
\begin{pgfscope}%
\pgfsetbuttcap%
\pgfsetroundjoin%
\definecolor{currentfill}{rgb}{0.150000,0.150000,0.150000}%
\pgfsetfillcolor{currentfill}%
\pgfsetlinewidth{0.803000pt}%
\definecolor{currentstroke}{rgb}{0.150000,0.150000,0.150000}%
\pgfsetstrokecolor{currentstroke}%
\pgfsetdash{}{0pt}%
\pgfsys@defobject{currentmarker}{\pgfqpoint{0.000000in}{-0.044444in}}{\pgfqpoint{0.000000in}{0.000000in}}{%
\pgfpathmoveto{\pgfqpoint{0.000000in}{0.000000in}}%
\pgfpathlineto{\pgfqpoint{0.000000in}{-0.044444in}}%
\pgfusepath{stroke,fill}%
}%
\begin{pgfscope}%
\pgfsys@transformshift{5.495685in}{4.812659in}%
\pgfsys@useobject{currentmarker}{}%
\end{pgfscope}%
\end{pgfscope}%
\begin{pgfscope}%
\pgfsetbuttcap%
\pgfsetroundjoin%
\definecolor{currentfill}{rgb}{0.150000,0.150000,0.150000}%
\pgfsetfillcolor{currentfill}%
\pgfsetlinewidth{0.803000pt}%
\definecolor{currentstroke}{rgb}{0.150000,0.150000,0.150000}%
\pgfsetstrokecolor{currentstroke}%
\pgfsetdash{}{0pt}%
\pgfsys@defobject{currentmarker}{\pgfqpoint{0.000000in}{-0.044444in}}{\pgfqpoint{0.000000in}{0.000000in}}{%
\pgfpathmoveto{\pgfqpoint{0.000000in}{0.000000in}}%
\pgfpathlineto{\pgfqpoint{0.000000in}{-0.044444in}}%
\pgfusepath{stroke,fill}%
}%
\begin{pgfscope}%
\pgfsys@transformshift{5.529252in}{4.812659in}%
\pgfsys@useobject{currentmarker}{}%
\end{pgfscope}%
\end{pgfscope}%
\begin{pgfscope}%
\pgfsetbuttcap%
\pgfsetroundjoin%
\definecolor{currentfill}{rgb}{0.150000,0.150000,0.150000}%
\pgfsetfillcolor{currentfill}%
\pgfsetlinewidth{0.803000pt}%
\definecolor{currentstroke}{rgb}{0.150000,0.150000,0.150000}%
\pgfsetstrokecolor{currentstroke}%
\pgfsetdash{}{0pt}%
\pgfsys@defobject{currentmarker}{\pgfqpoint{0.000000in}{-0.044444in}}{\pgfqpoint{0.000000in}{0.000000in}}{%
\pgfpathmoveto{\pgfqpoint{0.000000in}{0.000000in}}%
\pgfpathlineto{\pgfqpoint{0.000000in}{-0.044444in}}%
\pgfusepath{stroke,fill}%
}%
\begin{pgfscope}%
\pgfsys@transformshift{5.558330in}{4.812659in}%
\pgfsys@useobject{currentmarker}{}%
\end{pgfscope}%
\end{pgfscope}%
\begin{pgfscope}%
\pgfsetbuttcap%
\pgfsetroundjoin%
\definecolor{currentfill}{rgb}{0.150000,0.150000,0.150000}%
\pgfsetfillcolor{currentfill}%
\pgfsetlinewidth{0.803000pt}%
\definecolor{currentstroke}{rgb}{0.150000,0.150000,0.150000}%
\pgfsetstrokecolor{currentstroke}%
\pgfsetdash{}{0pt}%
\pgfsys@defobject{currentmarker}{\pgfqpoint{0.000000in}{-0.044444in}}{\pgfqpoint{0.000000in}{0.000000in}}{%
\pgfpathmoveto{\pgfqpoint{0.000000in}{0.000000in}}%
\pgfpathlineto{\pgfqpoint{0.000000in}{-0.044444in}}%
\pgfusepath{stroke,fill}%
}%
\begin{pgfscope}%
\pgfsys@transformshift{5.583978in}{4.812659in}%
\pgfsys@useobject{currentmarker}{}%
\end{pgfscope}%
\end{pgfscope}%
\begin{pgfscope}%
\pgfsetbuttcap%
\pgfsetroundjoin%
\definecolor{currentfill}{rgb}{0.150000,0.150000,0.150000}%
\pgfsetfillcolor{currentfill}%
\pgfsetlinewidth{0.803000pt}%
\definecolor{currentstroke}{rgb}{0.150000,0.150000,0.150000}%
\pgfsetstrokecolor{currentstroke}%
\pgfsetdash{}{0pt}%
\pgfsys@defobject{currentmarker}{\pgfqpoint{0.000000in}{-0.044444in}}{\pgfqpoint{0.000000in}{0.000000in}}{%
\pgfpathmoveto{\pgfqpoint{0.000000in}{0.000000in}}%
\pgfpathlineto{\pgfqpoint{0.000000in}{-0.044444in}}%
\pgfusepath{stroke,fill}%
}%
\begin{pgfscope}%
\pgfsys@transformshift{5.757861in}{4.812659in}%
\pgfsys@useobject{currentmarker}{}%
\end{pgfscope}%
\end{pgfscope}%
\begin{pgfscope}%
\pgfsetbuttcap%
\pgfsetroundjoin%
\definecolor{currentfill}{rgb}{0.150000,0.150000,0.150000}%
\pgfsetfillcolor{currentfill}%
\pgfsetlinewidth{0.803000pt}%
\definecolor{currentstroke}{rgb}{0.150000,0.150000,0.150000}%
\pgfsetstrokecolor{currentstroke}%
\pgfsetdash{}{0pt}%
\pgfsys@defobject{currentmarker}{\pgfqpoint{0.000000in}{-0.044444in}}{\pgfqpoint{0.000000in}{0.000000in}}{%
\pgfpathmoveto{\pgfqpoint{0.000000in}{0.000000in}}%
\pgfpathlineto{\pgfqpoint{0.000000in}{-0.044444in}}%
\pgfusepath{stroke,fill}%
}%
\begin{pgfscope}%
\pgfsys@transformshift{5.846155in}{4.812659in}%
\pgfsys@useobject{currentmarker}{}%
\end{pgfscope}%
\end{pgfscope}%
\begin{pgfscope}%
\pgfsetbuttcap%
\pgfsetroundjoin%
\definecolor{currentfill}{rgb}{0.150000,0.150000,0.150000}%
\pgfsetfillcolor{currentfill}%
\pgfsetlinewidth{0.803000pt}%
\definecolor{currentstroke}{rgb}{0.150000,0.150000,0.150000}%
\pgfsetstrokecolor{currentstroke}%
\pgfsetdash{}{0pt}%
\pgfsys@defobject{currentmarker}{\pgfqpoint{0.000000in}{-0.044444in}}{\pgfqpoint{0.000000in}{0.000000in}}{%
\pgfpathmoveto{\pgfqpoint{0.000000in}{0.000000in}}%
\pgfpathlineto{\pgfqpoint{0.000000in}{-0.044444in}}%
\pgfusepath{stroke,fill}%
}%
\begin{pgfscope}%
\pgfsys@transformshift{5.908800in}{4.812659in}%
\pgfsys@useobject{currentmarker}{}%
\end{pgfscope}%
\end{pgfscope}%
\begin{pgfscope}%
\pgfsetbuttcap%
\pgfsetroundjoin%
\definecolor{currentfill}{rgb}{0.150000,0.150000,0.150000}%
\pgfsetfillcolor{currentfill}%
\pgfsetlinewidth{0.803000pt}%
\definecolor{currentstroke}{rgb}{0.150000,0.150000,0.150000}%
\pgfsetstrokecolor{currentstroke}%
\pgfsetdash{}{0pt}%
\pgfsys@defobject{currentmarker}{\pgfqpoint{0.000000in}{-0.044444in}}{\pgfqpoint{0.000000in}{0.000000in}}{%
\pgfpathmoveto{\pgfqpoint{0.000000in}{0.000000in}}%
\pgfpathlineto{\pgfqpoint{0.000000in}{-0.044444in}}%
\pgfusepath{stroke,fill}%
}%
\begin{pgfscope}%
\pgfsys@transformshift{5.957391in}{4.812659in}%
\pgfsys@useobject{currentmarker}{}%
\end{pgfscope}%
\end{pgfscope}%
\begin{pgfscope}%
\pgfsetbuttcap%
\pgfsetroundjoin%
\definecolor{currentfill}{rgb}{0.150000,0.150000,0.150000}%
\pgfsetfillcolor{currentfill}%
\pgfsetlinewidth{0.803000pt}%
\definecolor{currentstroke}{rgb}{0.150000,0.150000,0.150000}%
\pgfsetstrokecolor{currentstroke}%
\pgfsetdash{}{0pt}%
\pgfsys@defobject{currentmarker}{\pgfqpoint{0.000000in}{-0.044444in}}{\pgfqpoint{0.000000in}{0.000000in}}{%
\pgfpathmoveto{\pgfqpoint{0.000000in}{0.000000in}}%
\pgfpathlineto{\pgfqpoint{0.000000in}{-0.044444in}}%
\pgfusepath{stroke,fill}%
}%
\begin{pgfscope}%
\pgfsys@transformshift{5.997094in}{4.812659in}%
\pgfsys@useobject{currentmarker}{}%
\end{pgfscope}%
\end{pgfscope}%
\begin{pgfscope}%
\pgfsetbuttcap%
\pgfsetroundjoin%
\definecolor{currentfill}{rgb}{0.150000,0.150000,0.150000}%
\pgfsetfillcolor{currentfill}%
\pgfsetlinewidth{0.803000pt}%
\definecolor{currentstroke}{rgb}{0.150000,0.150000,0.150000}%
\pgfsetstrokecolor{currentstroke}%
\pgfsetdash{}{0pt}%
\pgfsys@defobject{currentmarker}{\pgfqpoint{0.000000in}{-0.044444in}}{\pgfqpoint{0.000000in}{0.000000in}}{%
\pgfpathmoveto{\pgfqpoint{0.000000in}{0.000000in}}%
\pgfpathlineto{\pgfqpoint{0.000000in}{-0.044444in}}%
\pgfusepath{stroke,fill}%
}%
\begin{pgfscope}%
\pgfsys@transformshift{6.030661in}{4.812659in}%
\pgfsys@useobject{currentmarker}{}%
\end{pgfscope}%
\end{pgfscope}%
\begin{pgfscope}%
\pgfsetbuttcap%
\pgfsetroundjoin%
\definecolor{currentfill}{rgb}{0.150000,0.150000,0.150000}%
\pgfsetfillcolor{currentfill}%
\pgfsetlinewidth{0.803000pt}%
\definecolor{currentstroke}{rgb}{0.150000,0.150000,0.150000}%
\pgfsetstrokecolor{currentstroke}%
\pgfsetdash{}{0pt}%
\pgfsys@defobject{currentmarker}{\pgfqpoint{0.000000in}{-0.044444in}}{\pgfqpoint{0.000000in}{0.000000in}}{%
\pgfpathmoveto{\pgfqpoint{0.000000in}{0.000000in}}%
\pgfpathlineto{\pgfqpoint{0.000000in}{-0.044444in}}%
\pgfusepath{stroke,fill}%
}%
\begin{pgfscope}%
\pgfsys@transformshift{6.059739in}{4.812659in}%
\pgfsys@useobject{currentmarker}{}%
\end{pgfscope}%
\end{pgfscope}%
\begin{pgfscope}%
\pgfsetbuttcap%
\pgfsetroundjoin%
\definecolor{currentfill}{rgb}{0.150000,0.150000,0.150000}%
\pgfsetfillcolor{currentfill}%
\pgfsetlinewidth{0.803000pt}%
\definecolor{currentstroke}{rgb}{0.150000,0.150000,0.150000}%
\pgfsetstrokecolor{currentstroke}%
\pgfsetdash{}{0pt}%
\pgfsys@defobject{currentmarker}{\pgfqpoint{0.000000in}{-0.044444in}}{\pgfqpoint{0.000000in}{0.000000in}}{%
\pgfpathmoveto{\pgfqpoint{0.000000in}{0.000000in}}%
\pgfpathlineto{\pgfqpoint{0.000000in}{-0.044444in}}%
\pgfusepath{stroke,fill}%
}%
\begin{pgfscope}%
\pgfsys@transformshift{6.085387in}{4.812659in}%
\pgfsys@useobject{currentmarker}{}%
\end{pgfscope}%
\end{pgfscope}%
\begin{pgfscope}%
\pgfsetbuttcap%
\pgfsetroundjoin%
\definecolor{currentfill}{rgb}{0.150000,0.150000,0.150000}%
\pgfsetfillcolor{currentfill}%
\pgfsetlinewidth{0.803000pt}%
\definecolor{currentstroke}{rgb}{0.150000,0.150000,0.150000}%
\pgfsetstrokecolor{currentstroke}%
\pgfsetdash{}{0pt}%
\pgfsys@defobject{currentmarker}{\pgfqpoint{0.000000in}{-0.044444in}}{\pgfqpoint{0.000000in}{0.000000in}}{%
\pgfpathmoveto{\pgfqpoint{0.000000in}{0.000000in}}%
\pgfpathlineto{\pgfqpoint{0.000000in}{-0.044444in}}%
\pgfusepath{stroke,fill}%
}%
\begin{pgfscope}%
\pgfsys@transformshift{6.259270in}{4.812659in}%
\pgfsys@useobject{currentmarker}{}%
\end{pgfscope}%
\end{pgfscope}%
\begin{pgfscope}%
\pgfsetbuttcap%
\pgfsetroundjoin%
\definecolor{currentfill}{rgb}{0.150000,0.150000,0.150000}%
\pgfsetfillcolor{currentfill}%
\pgfsetlinewidth{1.003750pt}%
\definecolor{currentstroke}{rgb}{0.150000,0.150000,0.150000}%
\pgfsetstrokecolor{currentstroke}%
\pgfsetdash{}{0pt}%
\pgfsys@defobject{currentmarker}{\pgfqpoint{-0.066667in}{0.000000in}}{\pgfqpoint{0.000000in}{0.000000in}}{%
\pgfpathmoveto{\pgfqpoint{0.000000in}{0.000000in}}%
\pgfpathlineto{\pgfqpoint{-0.066667in}{0.000000in}}%
\pgfusepath{stroke,fill}%
}%
\begin{pgfscope}%
\pgfsys@transformshift{5.105513in}{4.812659in}%
\pgfsys@useobject{currentmarker}{}%
\end{pgfscope}%
\end{pgfscope}%
\begin{pgfscope}%
\pgfsetbuttcap%
\pgfsetroundjoin%
\definecolor{currentfill}{rgb}{0.150000,0.150000,0.150000}%
\pgfsetfillcolor{currentfill}%
\pgfsetlinewidth{1.003750pt}%
\definecolor{currentstroke}{rgb}{0.150000,0.150000,0.150000}%
\pgfsetstrokecolor{currentstroke}%
\pgfsetdash{}{0pt}%
\pgfsys@defobject{currentmarker}{\pgfqpoint{-0.066667in}{0.000000in}}{\pgfqpoint{0.000000in}{0.000000in}}{%
\pgfpathmoveto{\pgfqpoint{0.000000in}{0.000000in}}%
\pgfpathlineto{\pgfqpoint{-0.066667in}{0.000000in}}%
\pgfusepath{stroke,fill}%
}%
\begin{pgfscope}%
\pgfsys@transformshift{5.105513in}{5.045573in}%
\pgfsys@useobject{currentmarker}{}%
\end{pgfscope}%
\end{pgfscope}%
\begin{pgfscope}%
\pgfsetbuttcap%
\pgfsetroundjoin%
\definecolor{currentfill}{rgb}{0.150000,0.150000,0.150000}%
\pgfsetfillcolor{currentfill}%
\pgfsetlinewidth{1.003750pt}%
\definecolor{currentstroke}{rgb}{0.150000,0.150000,0.150000}%
\pgfsetstrokecolor{currentstroke}%
\pgfsetdash{}{0pt}%
\pgfsys@defobject{currentmarker}{\pgfqpoint{-0.066667in}{0.000000in}}{\pgfqpoint{0.000000in}{0.000000in}}{%
\pgfpathmoveto{\pgfqpoint{0.000000in}{0.000000in}}%
\pgfpathlineto{\pgfqpoint{-0.066667in}{0.000000in}}%
\pgfusepath{stroke,fill}%
}%
\begin{pgfscope}%
\pgfsys@transformshift{5.105513in}{5.420607in}%
\pgfsys@useobject{currentmarker}{}%
\end{pgfscope}%
\end{pgfscope}%
\begin{pgfscope}%
\pgfpathrectangle{\pgfqpoint{5.105513in}{4.812659in}}{\pgfqpoint{1.223103in}{0.607948in}}%
\pgfusepath{clip}%
\pgfsetroundcap%
\pgfsetroundjoin%
\pgfsetlinewidth{1.204500pt}%
\definecolor{currentstroke}{rgb}{0.000000,0.501961,0.000000}%
\pgfsetstrokecolor{currentstroke}%
\pgfsetdash{}{0pt}%
\pgfpathmoveto{\pgfqpoint{5.105513in}{5.046165in}}%
\pgfpathlineto{\pgfqpoint{5.346222in}{5.047293in}}%
\pgfpathlineto{\pgfqpoint{5.457753in}{5.048410in}}%
\pgfpathlineto{\pgfqpoint{5.531149in}{5.049520in}}%
\pgfpathlineto{\pgfqpoint{5.585945in}{5.050626in}}%
\pgfpathlineto{\pgfqpoint{5.629687in}{5.051729in}}%
\pgfpathlineto{\pgfqpoint{5.666095in}{5.052833in}}%
\pgfpathlineto{\pgfqpoint{5.697279in}{5.053938in}}%
\pgfpathlineto{\pgfqpoint{5.724552in}{5.055047in}}%
\pgfpathlineto{\pgfqpoint{5.748786in}{5.056160in}}%
\pgfpathlineto{\pgfqpoint{5.770591in}{5.057280in}}%
\pgfpathlineto{\pgfqpoint{5.790410in}{5.058407in}}%
\pgfpathlineto{\pgfqpoint{5.808575in}{5.059542in}}%
\pgfpathlineto{\pgfqpoint{5.825341in}{5.060686in}}%
\pgfpathlineto{\pgfqpoint{5.840907in}{5.061840in}}%
\pgfpathlineto{\pgfqpoint{5.855435in}{5.063003in}}%
\pgfpathlineto{\pgfqpoint{5.869053in}{5.064177in}}%
\pgfpathlineto{\pgfqpoint{5.881870in}{5.065362in}}%
\pgfpathlineto{\pgfqpoint{5.893974in}{5.066559in}}%
\pgfpathlineto{\pgfqpoint{5.905441in}{5.067766in}}%
\pgfpathlineto{\pgfqpoint{5.916334in}{5.068985in}}%
\pgfpathlineto{\pgfqpoint{5.926708in}{5.070216in}}%
\pgfpathlineto{\pgfqpoint{5.936610in}{5.071459in}}%
\pgfpathlineto{\pgfqpoint{5.946082in}{5.072713in}}%
\pgfpathlineto{\pgfqpoint{5.955158in}{5.073981in}}%
\pgfpathlineto{\pgfqpoint{5.963871in}{5.075260in}}%
\pgfpathlineto{\pgfqpoint{5.972249in}{5.076553in}}%
\pgfpathlineto{\pgfqpoint{5.980317in}{5.077859in}}%
\pgfpathlineto{\pgfqpoint{5.988096in}{5.079179in}}%
\pgfpathlineto{\pgfqpoint{5.995607in}{5.080513in}}%
\pgfpathlineto{\pgfqpoint{6.002868in}{5.081863in}}%
\pgfpathlineto{\pgfqpoint{6.009894in}{5.083228in}}%
\pgfpathlineto{\pgfqpoint{6.016701in}{5.084610in}}%
\pgfpathlineto{\pgfqpoint{6.023301in}{5.086011in}}%
\pgfpathlineto{\pgfqpoint{6.029707in}{5.087431in}}%
\pgfpathlineto{\pgfqpoint{6.035930in}{5.088873in}}%
\pgfpathlineto{\pgfqpoint{6.041980in}{5.090337in}}%
\pgfpathlineto{\pgfqpoint{6.047867in}{5.091827in}}%
\pgfpathlineto{\pgfqpoint{6.053598in}{5.093344in}}%
\pgfpathlineto{\pgfqpoint{6.059183in}{5.094890in}}%
\pgfpathlineto{\pgfqpoint{6.064628in}{5.096470in}}%
\pgfpathlineto{\pgfqpoint{6.069940in}{5.098085in}}%
\pgfpathlineto{\pgfqpoint{6.075126in}{5.099739in}}%
\pgfpathlineto{\pgfqpoint{6.080191in}{5.101437in}}%
\pgfpathlineto{\pgfqpoint{6.085140in}{5.103182in}}%
\pgfpathlineto{\pgfqpoint{6.089980in}{5.104978in}}%
\pgfpathlineto{\pgfqpoint{6.094715in}{5.106831in}}%
\pgfpathlineto{\pgfqpoint{6.099349in}{5.108745in}}%
\pgfpathlineto{\pgfqpoint{6.103886in}{5.110727in}}%
\pgfpathlineto{\pgfqpoint{6.108331in}{5.112782in}}%
\pgfusepath{stroke}%
\end{pgfscope}%
\begin{pgfscope}%
\pgfsetrectcap%
\pgfsetmiterjoin%
\pgfsetlinewidth{1.003750pt}%
\definecolor{currentstroke}{rgb}{0.150000,0.150000,0.150000}%
\pgfsetstrokecolor{currentstroke}%
\pgfsetdash{}{0pt}%
\pgfpathmoveto{\pgfqpoint{5.105513in}{4.812659in}}%
\pgfpathlineto{\pgfqpoint{5.105513in}{5.420607in}}%
\pgfusepath{stroke}%
\end{pgfscope}%
\begin{pgfscope}%
\pgfsetrectcap%
\pgfsetmiterjoin%
\pgfsetlinewidth{1.003750pt}%
\definecolor{currentstroke}{rgb}{0.150000,0.150000,0.150000}%
\pgfsetstrokecolor{currentstroke}%
\pgfsetdash{}{0pt}%
\pgfpathmoveto{\pgfqpoint{5.105513in}{4.812659in}}%
\pgfpathlineto{\pgfqpoint{6.328616in}{4.812659in}}%
\pgfusepath{stroke}%
\end{pgfscope}%
\begin{pgfscope}%
\pgfpathrectangle{\pgfqpoint{5.105513in}{4.812659in}}{\pgfqpoint{1.223103in}{0.607948in}}%
\pgfusepath{clip}%
\pgfsetbuttcap%
\pgfsetroundjoin%
\definecolor{currentfill}{rgb}{0.000000,0.000000,0.000000}%
\pgfsetfillcolor{currentfill}%
\pgfsetlinewidth{1.003750pt}%
\definecolor{currentstroke}{rgb}{0.000000,0.000000,0.000000}%
\pgfsetstrokecolor{currentstroke}%
\pgfsetdash{}{0pt}%
\pgfsys@defobject{currentmarker}{\pgfqpoint{-0.013889in}{-0.013889in}}{\pgfqpoint{0.013889in}{0.013889in}}{%
\pgfpathmoveto{\pgfqpoint{0.000000in}{-0.013889in}}%
\pgfpathcurveto{\pgfqpoint{0.003683in}{-0.013889in}}{\pgfqpoint{0.007216in}{-0.012425in}}{\pgfqpoint{0.009821in}{-0.009821in}}%
\pgfpathcurveto{\pgfqpoint{0.012425in}{-0.007216in}}{\pgfqpoint{0.013889in}{-0.003683in}}{\pgfqpoint{0.013889in}{0.000000in}}%
\pgfpathcurveto{\pgfqpoint{0.013889in}{0.003683in}}{\pgfqpoint{0.012425in}{0.007216in}}{\pgfqpoint{0.009821in}{0.009821in}}%
\pgfpathcurveto{\pgfqpoint{0.007216in}{0.012425in}}{\pgfqpoint{0.003683in}{0.013889in}}{\pgfqpoint{0.000000in}{0.013889in}}%
\pgfpathcurveto{\pgfqpoint{-0.003683in}{0.013889in}}{\pgfqpoint{-0.007216in}{0.012425in}}{\pgfqpoint{-0.009821in}{0.009821in}}%
\pgfpathcurveto{\pgfqpoint{-0.012425in}{0.007216in}}{\pgfqpoint{-0.013889in}{0.003683in}}{\pgfqpoint{-0.013889in}{0.000000in}}%
\pgfpathcurveto{\pgfqpoint{-0.013889in}{-0.003683in}}{\pgfqpoint{-0.012425in}{-0.007216in}}{\pgfqpoint{-0.009821in}{-0.009821in}}%
\pgfpathcurveto{\pgfqpoint{-0.007216in}{-0.012425in}}{\pgfqpoint{-0.003683in}{-0.013889in}}{\pgfqpoint{0.000000in}{-0.013889in}}%
\pgfpathclose%
\pgfusepath{stroke,fill}%
}%
\begin{pgfscope}%
\pgfsys@transformshift{5.757861in}{5.056585in}%
\pgfsys@useobject{currentmarker}{}%
\end{pgfscope}%
\begin{pgfscope}%
\pgfsys@transformshift{5.762260in}{5.056808in}%
\pgfsys@useobject{currentmarker}{}%
\end{pgfscope}%
\begin{pgfscope}%
\pgfsys@transformshift{5.766750in}{5.057055in}%
\pgfsys@useobject{currentmarker}{}%
\end{pgfscope}%
\begin{pgfscope}%
\pgfsys@transformshift{5.771335in}{5.057297in}%
\pgfsys@useobject{currentmarker}{}%
\end{pgfscope}%
\begin{pgfscope}%
\pgfsys@transformshift{5.776018in}{5.057567in}%
\pgfsys@useobject{currentmarker}{}%
\end{pgfscope}%
\begin{pgfscope}%
\pgfsys@transformshift{5.780804in}{5.057831in}%
\pgfsys@useobject{currentmarker}{}%
\end{pgfscope}%
\begin{pgfscope}%
\pgfsys@transformshift{5.785698in}{5.058127in}%
\pgfsys@useobject{currentmarker}{}%
\end{pgfscope}%
\begin{pgfscope}%
\pgfsys@transformshift{5.790704in}{5.058417in}%
\pgfsys@useobject{currentmarker}{}%
\end{pgfscope}%
\begin{pgfscope}%
\pgfsys@transformshift{5.795828in}{5.058742in}%
\pgfsys@useobject{currentmarker}{}%
\end{pgfscope}%
\begin{pgfscope}%
\pgfsys@transformshift{5.801075in}{5.059061in}%
\pgfsys@useobject{currentmarker}{}%
\end{pgfscope}%
\begin{pgfscope}%
\pgfsys@transformshift{5.806452in}{5.059421in}%
\pgfsys@useobject{currentmarker}{}%
\end{pgfscope}%
\begin{pgfscope}%
\pgfsys@transformshift{5.835530in}{5.061454in}%
\pgfsys@useobject{currentmarker}{}%
\end{pgfscope}%
\begin{pgfscope}%
\pgfsys@transformshift{5.869098in}{5.064249in}%
\pgfsys@useobject{currentmarker}{}%
\end{pgfscope}%
\begin{pgfscope}%
\pgfsys@transformshift{5.908800in}{5.068162in}%
\pgfsys@useobject{currentmarker}{}%
\end{pgfscope}%
\begin{pgfscope}%
\pgfsys@transformshift{5.957391in}{5.074396in}%
\pgfsys@useobject{currentmarker}{}%
\end{pgfscope}%
\begin{pgfscope}%
\pgfsys@transformshift{6.020037in}{5.085141in}%
\pgfsys@useobject{currentmarker}{}%
\end{pgfscope}%
\begin{pgfscope}%
\pgfsys@transformshift{6.108331in}{5.112839in}%
\pgfsys@useobject{currentmarker}{}%
\end{pgfscope}%
\end{pgfscope}%
\begin{pgfscope}%
\pgfsetbuttcap%
\pgfsetmiterjoin%
\definecolor{currentfill}{rgb}{1.000000,1.000000,1.000000}%
\pgfsetfillcolor{currentfill}%
\pgfsetlinewidth{0.803000pt}%
\definecolor{currentstroke}{rgb}{1.000000,1.000000,1.000000}%
\pgfsetstrokecolor{currentstroke}%
\pgfsetdash{}{0pt}%
\pgfpathmoveto{\pgfqpoint{6.297392in}{5.371731in}}%
\pgfpathlineto{\pgfqpoint{6.297392in}{4.861535in}}%
\pgfpathlineto{\pgfqpoint{6.478411in}{4.861535in}}%
\pgfpathlineto{\pgfqpoint{6.478411in}{5.371731in}}%
\pgfpathclose%
\pgfusepath{stroke,fill}%
\end{pgfscope}%
\begin{pgfscope}%
\definecolor{textcolor}{rgb}{0.150000,0.150000,0.150000}%
\pgfsetstrokecolor{textcolor}%
\pgfsetfillcolor{textcolor}%
\pgftext[x=6.368294in,y=5.316045in,left,base,rotate=270.000000]{\color{textcolor}\sffamily\fontsize{5.647059}{6.776471}\selectfont nlevel = 14}%
\end{pgfscope}%
\begin{pgfscope}%
\pgfsetbuttcap%
\pgfsetmiterjoin%
\definecolor{currentfill}{rgb}{1.000000,1.000000,1.000000}%
\pgfsetfillcolor{currentfill}%
\pgfsetlinewidth{0.803000pt}%
\definecolor{currentstroke}{rgb}{1.000000,1.000000,1.000000}%
\pgfsetstrokecolor{currentstroke}%
\pgfsetdash{}{0pt}%
\pgfpathmoveto{\pgfqpoint{6.297392in}{5.371731in}}%
\pgfpathlineto{\pgfqpoint{6.297392in}{4.861535in}}%
\pgfpathlineto{\pgfqpoint{6.478411in}{4.861535in}}%
\pgfpathlineto{\pgfqpoint{6.478411in}{5.371731in}}%
\pgfpathclose%
\pgfusepath{stroke,fill}%
\end{pgfscope}%
\begin{pgfscope}%
\definecolor{textcolor}{rgb}{0.150000,0.150000,0.150000}%
\pgfsetstrokecolor{textcolor}%
\pgfsetfillcolor{textcolor}%
\pgftext[x=6.368294in,y=5.316045in,left,base,rotate=270.000000]{\color{textcolor}\sffamily\fontsize{5.647059}{6.776471}\selectfont nlevel = 14}%
\end{pgfscope}%
\begin{pgfscope}%
\pgfsetbuttcap%
\pgfsetmiterjoin%
\definecolor{currentfill}{rgb}{1.000000,1.000000,1.000000}%
\pgfsetfillcolor{currentfill}%
\pgfsetlinewidth{0.000000pt}%
\definecolor{currentstroke}{rgb}{0.000000,0.000000,0.000000}%
\pgfsetstrokecolor{currentstroke}%
\pgfsetstrokeopacity{0.000000}%
\pgfsetdash{}{0pt}%
\pgfpathmoveto{\pgfqpoint{0.702340in}{4.083121in}}%
\pgfpathlineto{\pgfqpoint{1.925444in}{4.083121in}}%
\pgfpathlineto{\pgfqpoint{1.925444in}{4.691069in}}%
\pgfpathlineto{\pgfqpoint{0.702340in}{4.691069in}}%
\pgfpathclose%
\pgfusepath{fill}%
\end{pgfscope}%
\begin{pgfscope}%
\pgfpathrectangle{\pgfqpoint{0.702340in}{4.083121in}}{\pgfqpoint{1.223103in}{0.607948in}}%
\pgfusepath{clip}%
\pgfsetbuttcap%
\pgfsetmiterjoin%
\definecolor{currentfill}{rgb}{0.000000,0.000000,1.000000}%
\pgfsetfillcolor{currentfill}%
\pgfsetfillopacity{0.100000}%
\pgfsetlinewidth{0.803000pt}%
\definecolor{currentstroke}{rgb}{0.000000,0.000000,1.000000}%
\pgfsetstrokecolor{currentstroke}%
\pgfsetstrokeopacity{0.100000}%
\pgfsetdash{}{0pt}%
\pgfpathmoveto{\pgfqpoint{0.702340in}{4.485341in}}%
\pgfpathlineto{\pgfqpoint{0.702340in}{4.523021in}}%
\pgfpathlineto{\pgfqpoint{1.925444in}{4.523021in}}%
\pgfpathlineto{\pgfqpoint{1.925444in}{4.485341in}}%
\pgfpathclose%
\pgfusepath{stroke,fill}%
\end{pgfscope}%
\begin{pgfscope}%
\pgfpathrectangle{\pgfqpoint{0.702340in}{4.083121in}}{\pgfqpoint{1.223103in}{0.607948in}}%
\pgfusepath{clip}%
\pgfsetbuttcap%
\pgfsetroundjoin%
\definecolor{currentfill}{rgb}{0.000000,0.501961,0.000000}%
\pgfsetfillcolor{currentfill}%
\pgfsetfillopacity{0.500000}%
\pgfsetlinewidth{0.803000pt}%
\definecolor{currentstroke}{rgb}{0.000000,0.501961,0.000000}%
\pgfsetstrokecolor{currentstroke}%
\pgfsetstrokeopacity{0.500000}%
\pgfsetdash{}{0pt}%
\pgfpathmoveto{\pgfqpoint{0.702340in}{4.520766in}}%
\pgfpathlineto{\pgfqpoint{0.702340in}{4.486454in}}%
\pgfpathlineto{\pgfqpoint{0.943050in}{4.487533in}}%
\pgfpathlineto{\pgfqpoint{1.054581in}{4.487080in}}%
\pgfpathlineto{\pgfqpoint{1.127977in}{4.485128in}}%
\pgfpathlineto{\pgfqpoint{1.182772in}{4.481707in}}%
\pgfpathlineto{\pgfqpoint{1.226515in}{4.476840in}}%
\pgfpathlineto{\pgfqpoint{1.262923in}{4.470548in}}%
\pgfpathlineto{\pgfqpoint{1.294107in}{4.462845in}}%
\pgfpathlineto{\pgfqpoint{1.321380in}{4.453745in}}%
\pgfpathlineto{\pgfqpoint{1.345614in}{4.443255in}}%
\pgfpathlineto{\pgfqpoint{1.367419in}{4.431381in}}%
\pgfpathlineto{\pgfqpoint{1.387238in}{4.417945in}}%
\pgfpathlineto{\pgfqpoint{1.405403in}{4.402337in}}%
\pgfpathlineto{\pgfqpoint{1.422168in}{4.385523in}}%
\pgfpathlineto{\pgfqpoint{1.437735in}{4.367516in}}%
\pgfpathlineto{\pgfqpoint{1.452262in}{4.348306in}}%
\pgfpathlineto{\pgfqpoint{1.465881in}{4.327882in}}%
\pgfpathlineto{\pgfqpoint{1.478698in}{4.306234in}}%
\pgfpathlineto{\pgfqpoint{1.490802in}{4.283356in}}%
\pgfpathlineto{\pgfqpoint{1.502269in}{4.259237in}}%
\pgfpathlineto{\pgfqpoint{1.513162in}{4.233871in}}%
\pgfpathlineto{\pgfqpoint{1.523536in}{4.207248in}}%
\pgfpathlineto{\pgfqpoint{1.533438in}{4.179357in}}%
\pgfpathlineto{\pgfqpoint{1.542909in}{4.150181in}}%
\pgfpathlineto{\pgfqpoint{1.551986in}{4.119690in}}%
\pgfpathlineto{\pgfqpoint{1.560699in}{4.087806in}}%
\pgfpathlineto{\pgfqpoint{1.569077in}{4.054382in}}%
\pgfpathlineto{\pgfqpoint{1.577145in}{4.019365in}}%
\pgfpathlineto{\pgfqpoint{1.584924in}{3.982844in}}%
\pgfpathlineto{\pgfqpoint{1.592435in}{3.944876in}}%
\pgfpathlineto{\pgfqpoint{1.599696in}{3.905474in}}%
\pgfpathlineto{\pgfqpoint{1.606722in}{3.864640in}}%
\pgfpathlineto{\pgfqpoint{1.613528in}{3.822366in}}%
\pgfpathlineto{\pgfqpoint{1.620129in}{3.778644in}}%
\pgfpathlineto{\pgfqpoint{1.626535in}{3.733464in}}%
\pgfpathlineto{\pgfqpoint{1.632758in}{3.686815in}}%
\pgfpathlineto{\pgfqpoint{1.638808in}{3.638688in}}%
\pgfpathlineto{\pgfqpoint{1.644694in}{3.589072in}}%
\pgfpathlineto{\pgfqpoint{1.650426in}{3.537957in}}%
\pgfpathlineto{\pgfqpoint{1.656011in}{3.485335in}}%
\pgfpathlineto{\pgfqpoint{1.661456in}{3.431197in}}%
\pgfpathlineto{\pgfqpoint{1.666768in}{3.375536in}}%
\pgfpathlineto{\pgfqpoint{1.671953in}{3.318347in}}%
\pgfpathlineto{\pgfqpoint{1.677018in}{3.259624in}}%
\pgfpathlineto{\pgfqpoint{1.681968in}{3.199366in}}%
\pgfpathlineto{\pgfqpoint{1.686808in}{3.137570in}}%
\pgfpathlineto{\pgfqpoint{1.691543in}{3.074237in}}%
\pgfpathlineto{\pgfqpoint{1.696176in}{3.009367in}}%
\pgfpathlineto{\pgfqpoint{1.700714in}{2.942954in}}%
\pgfpathlineto{\pgfqpoint{1.705158in}{2.873800in}}%
\pgfpathlineto{\pgfqpoint{1.705158in}{2.875225in}}%
\pgfpathlineto{\pgfqpoint{1.705158in}{2.875225in}}%
\pgfpathlineto{\pgfqpoint{1.700714in}{2.947543in}}%
\pgfpathlineto{\pgfqpoint{1.696176in}{3.018312in}}%
\pgfpathlineto{\pgfqpoint{1.691543in}{3.086485in}}%
\pgfpathlineto{\pgfqpoint{1.686808in}{3.152198in}}%
\pgfpathlineto{\pgfqpoint{1.681968in}{3.215581in}}%
\pgfpathlineto{\pgfqpoint{1.677018in}{3.276758in}}%
\pgfpathlineto{\pgfqpoint{1.671953in}{3.335841in}}%
\pgfpathlineto{\pgfqpoint{1.666768in}{3.392933in}}%
\pgfpathlineto{\pgfqpoint{1.661456in}{3.448127in}}%
\pgfpathlineto{\pgfqpoint{1.656011in}{3.501508in}}%
\pgfpathlineto{\pgfqpoint{1.650426in}{3.553151in}}%
\pgfpathlineto{\pgfqpoint{1.644694in}{3.603124in}}%
\pgfpathlineto{\pgfqpoint{1.638808in}{3.651488in}}%
\pgfpathlineto{\pgfqpoint{1.632758in}{3.698295in}}%
\pgfpathlineto{\pgfqpoint{1.626535in}{3.743592in}}%
\pgfpathlineto{\pgfqpoint{1.620129in}{3.787421in}}%
\pgfpathlineto{\pgfqpoint{1.613528in}{3.829817in}}%
\pgfpathlineto{\pgfqpoint{1.606722in}{3.870811in}}%
\pgfpathlineto{\pgfqpoint{1.599696in}{3.910431in}}%
\pgfpathlineto{\pgfqpoint{1.592435in}{3.948703in}}%
\pgfpathlineto{\pgfqpoint{1.584924in}{3.985655in}}%
\pgfpathlineto{\pgfqpoint{1.577145in}{4.021326in}}%
\pgfpathlineto{\pgfqpoint{1.569077in}{4.055794in}}%
\pgfpathlineto{\pgfqpoint{1.560699in}{4.089168in}}%
\pgfpathlineto{\pgfqpoint{1.551986in}{4.121409in}}%
\pgfpathlineto{\pgfqpoint{1.542909in}{4.152383in}}%
\pgfpathlineto{\pgfqpoint{1.533438in}{4.182027in}}%
\pgfpathlineto{\pgfqpoint{1.523536in}{4.210318in}}%
\pgfpathlineto{\pgfqpoint{1.513162in}{4.237249in}}%
\pgfpathlineto{\pgfqpoint{1.502269in}{4.262816in}}%
\pgfpathlineto{\pgfqpoint{1.490802in}{4.287019in}}%
\pgfpathlineto{\pgfqpoint{1.478698in}{4.309856in}}%
\pgfpathlineto{\pgfqpoint{1.465881in}{4.331327in}}%
\pgfpathlineto{\pgfqpoint{1.452262in}{4.351429in}}%
\pgfpathlineto{\pgfqpoint{1.437735in}{4.370164in}}%
\pgfpathlineto{\pgfqpoint{1.422168in}{4.387529in}}%
\pgfpathlineto{\pgfqpoint{1.405403in}{4.403525in}}%
\pgfpathlineto{\pgfqpoint{1.387238in}{4.418179in}}%
\pgfpathlineto{\pgfqpoint{1.367419in}{4.432471in}}%
\pgfpathlineto{\pgfqpoint{1.345614in}{4.445808in}}%
\pgfpathlineto{\pgfqpoint{1.321380in}{4.458033in}}%
\pgfpathlineto{\pgfqpoint{1.294107in}{4.469174in}}%
\pgfpathlineto{\pgfqpoint{1.262923in}{4.479266in}}%
\pgfpathlineto{\pgfqpoint{1.226515in}{4.488352in}}%
\pgfpathlineto{\pgfqpoint{1.182772in}{4.496482in}}%
\pgfpathlineto{\pgfqpoint{1.127977in}{4.503716in}}%
\pgfpathlineto{\pgfqpoint{1.054581in}{4.510121in}}%
\pgfpathlineto{\pgfqpoint{0.943050in}{4.515775in}}%
\pgfpathlineto{\pgfqpoint{0.702340in}{4.520766in}}%
\pgfpathclose%
\pgfusepath{stroke,fill}%
\end{pgfscope}%
\begin{pgfscope}%
\pgfpathrectangle{\pgfqpoint{0.702340in}{4.083121in}}{\pgfqpoint{1.223103in}{0.607948in}}%
\pgfusepath{clip}%
\pgfsetroundcap%
\pgfsetroundjoin%
\pgfsetlinewidth{0.501875pt}%
\definecolor{currentstroke}{rgb}{0.000000,0.000000,1.000000}%
\pgfsetstrokecolor{currentstroke}%
\pgfsetstrokeopacity{0.800000}%
\pgfsetdash{}{0pt}%
\pgfpathmoveto{\pgfqpoint{0.702340in}{4.504181in}}%
\pgfpathlineto{\pgfqpoint{1.925444in}{4.504181in}}%
\pgfusepath{stroke}%
\end{pgfscope}%
\begin{pgfscope}%
\pgfpathrectangle{\pgfqpoint{0.702340in}{4.083121in}}{\pgfqpoint{1.223103in}{0.607948in}}%
\pgfusepath{clip}%
\pgfsetbuttcap%
\pgfsetroundjoin%
\pgfsetlinewidth{1.003750pt}%
\definecolor{currentstroke}{rgb}{0.000000,0.000000,0.000000}%
\pgfsetstrokecolor{currentstroke}%
\pgfsetdash{{3.700000pt}{1.600000pt}}{0.000000pt}%
\pgfpathmoveto{\pgfqpoint{0.702340in}{4.509594in}}%
\pgfpathlineto{\pgfqpoint{1.925444in}{4.509594in}}%
\pgfusepath{stroke}%
\end{pgfscope}%
\begin{pgfscope}%
\pgfsetroundcap%
\pgfsetroundjoin%
\pgfsetlinewidth{0.501875pt}%
\definecolor{currentstroke}{rgb}{0.000000,0.000000,1.000000}%
\pgfsetstrokecolor{currentstroke}%
\pgfsetstrokeopacity{0.800000}%
\pgfsetdash{}{0pt}%
\pgfpathmoveto{\pgfqpoint{1.516189in}{4.621518in}}%
\pgfpathquadraticcurveto{\pgfqpoint{1.446671in}{4.571010in}}{\pgfqpoint{1.377153in}{4.520502in}}%
\pgfusepath{stroke}%
\end{pgfscope}%
\begin{pgfscope}%
\pgfsetfillopacity{0.800000}%
\pgfsetstrokeopacity{0.800000}%
\definecolor{textcolor}{rgb}{0.000000,0.000000,1.000000}%
\pgfsetstrokecolor{textcolor}%
\pgfsetfillcolor{textcolor}%
\pgftext[x=1.442982in,y=4.686565in,left,base]{\color{textcolor}\sffamily\fontsize{5.647059}{6.776471}\selectfont 11.693(31)}%
\end{pgfscope}%
\begin{pgfscope}%
\pgfsetbuttcap%
\pgfsetroundjoin%
\definecolor{currentfill}{rgb}{0.150000,0.150000,0.150000}%
\pgfsetfillcolor{currentfill}%
\pgfsetlinewidth{1.003750pt}%
\definecolor{currentstroke}{rgb}{0.150000,0.150000,0.150000}%
\pgfsetstrokecolor{currentstroke}%
\pgfsetdash{}{0pt}%
\pgfsys@defobject{currentmarker}{\pgfqpoint{0.000000in}{-0.066667in}}{\pgfqpoint{0.000000in}{0.000000in}}{%
\pgfpathmoveto{\pgfqpoint{0.000000in}{0.000000in}}%
\pgfpathlineto{\pgfqpoint{0.000000in}{-0.066667in}}%
\pgfusepath{stroke,fill}%
}%
\begin{pgfscope}%
\pgfsys@transformshift{0.702340in}{4.083121in}%
\pgfsys@useobject{currentmarker}{}%
\end{pgfscope}%
\end{pgfscope}%
\begin{pgfscope}%
\pgfsetbuttcap%
\pgfsetroundjoin%
\definecolor{currentfill}{rgb}{0.150000,0.150000,0.150000}%
\pgfsetfillcolor{currentfill}%
\pgfsetlinewidth{1.003750pt}%
\definecolor{currentstroke}{rgb}{0.150000,0.150000,0.150000}%
\pgfsetstrokecolor{currentstroke}%
\pgfsetdash{}{0pt}%
\pgfsys@defobject{currentmarker}{\pgfqpoint{0.000000in}{-0.066667in}}{\pgfqpoint{0.000000in}{0.000000in}}{%
\pgfpathmoveto{\pgfqpoint{0.000000in}{0.000000in}}%
\pgfpathlineto{\pgfqpoint{0.000000in}{-0.066667in}}%
\pgfusepath{stroke,fill}%
}%
\begin{pgfscope}%
\pgfsys@transformshift{1.203749in}{4.083121in}%
\pgfsys@useobject{currentmarker}{}%
\end{pgfscope}%
\end{pgfscope}%
\begin{pgfscope}%
\pgfsetbuttcap%
\pgfsetroundjoin%
\definecolor{currentfill}{rgb}{0.150000,0.150000,0.150000}%
\pgfsetfillcolor{currentfill}%
\pgfsetlinewidth{1.003750pt}%
\definecolor{currentstroke}{rgb}{0.150000,0.150000,0.150000}%
\pgfsetstrokecolor{currentstroke}%
\pgfsetdash{}{0pt}%
\pgfsys@defobject{currentmarker}{\pgfqpoint{0.000000in}{-0.066667in}}{\pgfqpoint{0.000000in}{0.000000in}}{%
\pgfpathmoveto{\pgfqpoint{0.000000in}{0.000000in}}%
\pgfpathlineto{\pgfqpoint{0.000000in}{-0.066667in}}%
\pgfusepath{stroke,fill}%
}%
\begin{pgfscope}%
\pgfsys@transformshift{1.705158in}{4.083121in}%
\pgfsys@useobject{currentmarker}{}%
\end{pgfscope}%
\end{pgfscope}%
\begin{pgfscope}%
\pgfsetbuttcap%
\pgfsetroundjoin%
\definecolor{currentfill}{rgb}{0.150000,0.150000,0.150000}%
\pgfsetfillcolor{currentfill}%
\pgfsetlinewidth{0.803000pt}%
\definecolor{currentstroke}{rgb}{0.150000,0.150000,0.150000}%
\pgfsetstrokecolor{currentstroke}%
\pgfsetdash{}{0pt}%
\pgfsys@defobject{currentmarker}{\pgfqpoint{0.000000in}{-0.044444in}}{\pgfqpoint{0.000000in}{0.000000in}}{%
\pgfpathmoveto{\pgfqpoint{0.000000in}{0.000000in}}%
\pgfpathlineto{\pgfqpoint{0.000000in}{-0.044444in}}%
\pgfusepath{stroke,fill}%
}%
\begin{pgfscope}%
\pgfsys@transformshift{0.853280in}{4.083121in}%
\pgfsys@useobject{currentmarker}{}%
\end{pgfscope}%
\end{pgfscope}%
\begin{pgfscope}%
\pgfsetbuttcap%
\pgfsetroundjoin%
\definecolor{currentfill}{rgb}{0.150000,0.150000,0.150000}%
\pgfsetfillcolor{currentfill}%
\pgfsetlinewidth{0.803000pt}%
\definecolor{currentstroke}{rgb}{0.150000,0.150000,0.150000}%
\pgfsetstrokecolor{currentstroke}%
\pgfsetdash{}{0pt}%
\pgfsys@defobject{currentmarker}{\pgfqpoint{0.000000in}{-0.044444in}}{\pgfqpoint{0.000000in}{0.000000in}}{%
\pgfpathmoveto{\pgfqpoint{0.000000in}{0.000000in}}%
\pgfpathlineto{\pgfqpoint{0.000000in}{-0.044444in}}%
\pgfusepath{stroke,fill}%
}%
\begin{pgfscope}%
\pgfsys@transformshift{0.941573in}{4.083121in}%
\pgfsys@useobject{currentmarker}{}%
\end{pgfscope}%
\end{pgfscope}%
\begin{pgfscope}%
\pgfsetbuttcap%
\pgfsetroundjoin%
\definecolor{currentfill}{rgb}{0.150000,0.150000,0.150000}%
\pgfsetfillcolor{currentfill}%
\pgfsetlinewidth{0.803000pt}%
\definecolor{currentstroke}{rgb}{0.150000,0.150000,0.150000}%
\pgfsetstrokecolor{currentstroke}%
\pgfsetdash{}{0pt}%
\pgfsys@defobject{currentmarker}{\pgfqpoint{0.000000in}{-0.044444in}}{\pgfqpoint{0.000000in}{0.000000in}}{%
\pgfpathmoveto{\pgfqpoint{0.000000in}{0.000000in}}%
\pgfpathlineto{\pgfqpoint{0.000000in}{-0.044444in}}%
\pgfusepath{stroke,fill}%
}%
\begin{pgfscope}%
\pgfsys@transformshift{1.004219in}{4.083121in}%
\pgfsys@useobject{currentmarker}{}%
\end{pgfscope}%
\end{pgfscope}%
\begin{pgfscope}%
\pgfsetbuttcap%
\pgfsetroundjoin%
\definecolor{currentfill}{rgb}{0.150000,0.150000,0.150000}%
\pgfsetfillcolor{currentfill}%
\pgfsetlinewidth{0.803000pt}%
\definecolor{currentstroke}{rgb}{0.150000,0.150000,0.150000}%
\pgfsetstrokecolor{currentstroke}%
\pgfsetdash{}{0pt}%
\pgfsys@defobject{currentmarker}{\pgfqpoint{0.000000in}{-0.044444in}}{\pgfqpoint{0.000000in}{0.000000in}}{%
\pgfpathmoveto{\pgfqpoint{0.000000in}{0.000000in}}%
\pgfpathlineto{\pgfqpoint{0.000000in}{-0.044444in}}%
\pgfusepath{stroke,fill}%
}%
\begin{pgfscope}%
\pgfsys@transformshift{1.052810in}{4.083121in}%
\pgfsys@useobject{currentmarker}{}%
\end{pgfscope}%
\end{pgfscope}%
\begin{pgfscope}%
\pgfsetbuttcap%
\pgfsetroundjoin%
\definecolor{currentfill}{rgb}{0.150000,0.150000,0.150000}%
\pgfsetfillcolor{currentfill}%
\pgfsetlinewidth{0.803000pt}%
\definecolor{currentstroke}{rgb}{0.150000,0.150000,0.150000}%
\pgfsetstrokecolor{currentstroke}%
\pgfsetdash{}{0pt}%
\pgfsys@defobject{currentmarker}{\pgfqpoint{0.000000in}{-0.044444in}}{\pgfqpoint{0.000000in}{0.000000in}}{%
\pgfpathmoveto{\pgfqpoint{0.000000in}{0.000000in}}%
\pgfpathlineto{\pgfqpoint{0.000000in}{-0.044444in}}%
\pgfusepath{stroke,fill}%
}%
\begin{pgfscope}%
\pgfsys@transformshift{1.092512in}{4.083121in}%
\pgfsys@useobject{currentmarker}{}%
\end{pgfscope}%
\end{pgfscope}%
\begin{pgfscope}%
\pgfsetbuttcap%
\pgfsetroundjoin%
\definecolor{currentfill}{rgb}{0.150000,0.150000,0.150000}%
\pgfsetfillcolor{currentfill}%
\pgfsetlinewidth{0.803000pt}%
\definecolor{currentstroke}{rgb}{0.150000,0.150000,0.150000}%
\pgfsetstrokecolor{currentstroke}%
\pgfsetdash{}{0pt}%
\pgfsys@defobject{currentmarker}{\pgfqpoint{0.000000in}{-0.044444in}}{\pgfqpoint{0.000000in}{0.000000in}}{%
\pgfpathmoveto{\pgfqpoint{0.000000in}{0.000000in}}%
\pgfpathlineto{\pgfqpoint{0.000000in}{-0.044444in}}%
\pgfusepath{stroke,fill}%
}%
\begin{pgfscope}%
\pgfsys@transformshift{1.126080in}{4.083121in}%
\pgfsys@useobject{currentmarker}{}%
\end{pgfscope}%
\end{pgfscope}%
\begin{pgfscope}%
\pgfsetbuttcap%
\pgfsetroundjoin%
\definecolor{currentfill}{rgb}{0.150000,0.150000,0.150000}%
\pgfsetfillcolor{currentfill}%
\pgfsetlinewidth{0.803000pt}%
\definecolor{currentstroke}{rgb}{0.150000,0.150000,0.150000}%
\pgfsetstrokecolor{currentstroke}%
\pgfsetdash{}{0pt}%
\pgfsys@defobject{currentmarker}{\pgfqpoint{0.000000in}{-0.044444in}}{\pgfqpoint{0.000000in}{0.000000in}}{%
\pgfpathmoveto{\pgfqpoint{0.000000in}{0.000000in}}%
\pgfpathlineto{\pgfqpoint{0.000000in}{-0.044444in}}%
\pgfusepath{stroke,fill}%
}%
\begin{pgfscope}%
\pgfsys@transformshift{1.155158in}{4.083121in}%
\pgfsys@useobject{currentmarker}{}%
\end{pgfscope}%
\end{pgfscope}%
\begin{pgfscope}%
\pgfsetbuttcap%
\pgfsetroundjoin%
\definecolor{currentfill}{rgb}{0.150000,0.150000,0.150000}%
\pgfsetfillcolor{currentfill}%
\pgfsetlinewidth{0.803000pt}%
\definecolor{currentstroke}{rgb}{0.150000,0.150000,0.150000}%
\pgfsetstrokecolor{currentstroke}%
\pgfsetdash{}{0pt}%
\pgfsys@defobject{currentmarker}{\pgfqpoint{0.000000in}{-0.044444in}}{\pgfqpoint{0.000000in}{0.000000in}}{%
\pgfpathmoveto{\pgfqpoint{0.000000in}{0.000000in}}%
\pgfpathlineto{\pgfqpoint{0.000000in}{-0.044444in}}%
\pgfusepath{stroke,fill}%
}%
\begin{pgfscope}%
\pgfsys@transformshift{1.180806in}{4.083121in}%
\pgfsys@useobject{currentmarker}{}%
\end{pgfscope}%
\end{pgfscope}%
\begin{pgfscope}%
\pgfsetbuttcap%
\pgfsetroundjoin%
\definecolor{currentfill}{rgb}{0.150000,0.150000,0.150000}%
\pgfsetfillcolor{currentfill}%
\pgfsetlinewidth{0.803000pt}%
\definecolor{currentstroke}{rgb}{0.150000,0.150000,0.150000}%
\pgfsetstrokecolor{currentstroke}%
\pgfsetdash{}{0pt}%
\pgfsys@defobject{currentmarker}{\pgfqpoint{0.000000in}{-0.044444in}}{\pgfqpoint{0.000000in}{0.000000in}}{%
\pgfpathmoveto{\pgfqpoint{0.000000in}{0.000000in}}%
\pgfpathlineto{\pgfqpoint{0.000000in}{-0.044444in}}%
\pgfusepath{stroke,fill}%
}%
\begin{pgfscope}%
\pgfsys@transformshift{1.354689in}{4.083121in}%
\pgfsys@useobject{currentmarker}{}%
\end{pgfscope}%
\end{pgfscope}%
\begin{pgfscope}%
\pgfsetbuttcap%
\pgfsetroundjoin%
\definecolor{currentfill}{rgb}{0.150000,0.150000,0.150000}%
\pgfsetfillcolor{currentfill}%
\pgfsetlinewidth{0.803000pt}%
\definecolor{currentstroke}{rgb}{0.150000,0.150000,0.150000}%
\pgfsetstrokecolor{currentstroke}%
\pgfsetdash{}{0pt}%
\pgfsys@defobject{currentmarker}{\pgfqpoint{0.000000in}{-0.044444in}}{\pgfqpoint{0.000000in}{0.000000in}}{%
\pgfpathmoveto{\pgfqpoint{0.000000in}{0.000000in}}%
\pgfpathlineto{\pgfqpoint{0.000000in}{-0.044444in}}%
\pgfusepath{stroke,fill}%
}%
\begin{pgfscope}%
\pgfsys@transformshift{1.442982in}{4.083121in}%
\pgfsys@useobject{currentmarker}{}%
\end{pgfscope}%
\end{pgfscope}%
\begin{pgfscope}%
\pgfsetbuttcap%
\pgfsetroundjoin%
\definecolor{currentfill}{rgb}{0.150000,0.150000,0.150000}%
\pgfsetfillcolor{currentfill}%
\pgfsetlinewidth{0.803000pt}%
\definecolor{currentstroke}{rgb}{0.150000,0.150000,0.150000}%
\pgfsetstrokecolor{currentstroke}%
\pgfsetdash{}{0pt}%
\pgfsys@defobject{currentmarker}{\pgfqpoint{0.000000in}{-0.044444in}}{\pgfqpoint{0.000000in}{0.000000in}}{%
\pgfpathmoveto{\pgfqpoint{0.000000in}{0.000000in}}%
\pgfpathlineto{\pgfqpoint{0.000000in}{-0.044444in}}%
\pgfusepath{stroke,fill}%
}%
\begin{pgfscope}%
\pgfsys@transformshift{1.505628in}{4.083121in}%
\pgfsys@useobject{currentmarker}{}%
\end{pgfscope}%
\end{pgfscope}%
\begin{pgfscope}%
\pgfsetbuttcap%
\pgfsetroundjoin%
\definecolor{currentfill}{rgb}{0.150000,0.150000,0.150000}%
\pgfsetfillcolor{currentfill}%
\pgfsetlinewidth{0.803000pt}%
\definecolor{currentstroke}{rgb}{0.150000,0.150000,0.150000}%
\pgfsetstrokecolor{currentstroke}%
\pgfsetdash{}{0pt}%
\pgfsys@defobject{currentmarker}{\pgfqpoint{0.000000in}{-0.044444in}}{\pgfqpoint{0.000000in}{0.000000in}}{%
\pgfpathmoveto{\pgfqpoint{0.000000in}{0.000000in}}%
\pgfpathlineto{\pgfqpoint{0.000000in}{-0.044444in}}%
\pgfusepath{stroke,fill}%
}%
\begin{pgfscope}%
\pgfsys@transformshift{1.554219in}{4.083121in}%
\pgfsys@useobject{currentmarker}{}%
\end{pgfscope}%
\end{pgfscope}%
\begin{pgfscope}%
\pgfsetbuttcap%
\pgfsetroundjoin%
\definecolor{currentfill}{rgb}{0.150000,0.150000,0.150000}%
\pgfsetfillcolor{currentfill}%
\pgfsetlinewidth{0.803000pt}%
\definecolor{currentstroke}{rgb}{0.150000,0.150000,0.150000}%
\pgfsetstrokecolor{currentstroke}%
\pgfsetdash{}{0pt}%
\pgfsys@defobject{currentmarker}{\pgfqpoint{0.000000in}{-0.044444in}}{\pgfqpoint{0.000000in}{0.000000in}}{%
\pgfpathmoveto{\pgfqpoint{0.000000in}{0.000000in}}%
\pgfpathlineto{\pgfqpoint{0.000000in}{-0.044444in}}%
\pgfusepath{stroke,fill}%
}%
\begin{pgfscope}%
\pgfsys@transformshift{1.593921in}{4.083121in}%
\pgfsys@useobject{currentmarker}{}%
\end{pgfscope}%
\end{pgfscope}%
\begin{pgfscope}%
\pgfsetbuttcap%
\pgfsetroundjoin%
\definecolor{currentfill}{rgb}{0.150000,0.150000,0.150000}%
\pgfsetfillcolor{currentfill}%
\pgfsetlinewidth{0.803000pt}%
\definecolor{currentstroke}{rgb}{0.150000,0.150000,0.150000}%
\pgfsetstrokecolor{currentstroke}%
\pgfsetdash{}{0pt}%
\pgfsys@defobject{currentmarker}{\pgfqpoint{0.000000in}{-0.044444in}}{\pgfqpoint{0.000000in}{0.000000in}}{%
\pgfpathmoveto{\pgfqpoint{0.000000in}{0.000000in}}%
\pgfpathlineto{\pgfqpoint{0.000000in}{-0.044444in}}%
\pgfusepath{stroke,fill}%
}%
\begin{pgfscope}%
\pgfsys@transformshift{1.627489in}{4.083121in}%
\pgfsys@useobject{currentmarker}{}%
\end{pgfscope}%
\end{pgfscope}%
\begin{pgfscope}%
\pgfsetbuttcap%
\pgfsetroundjoin%
\definecolor{currentfill}{rgb}{0.150000,0.150000,0.150000}%
\pgfsetfillcolor{currentfill}%
\pgfsetlinewidth{0.803000pt}%
\definecolor{currentstroke}{rgb}{0.150000,0.150000,0.150000}%
\pgfsetstrokecolor{currentstroke}%
\pgfsetdash{}{0pt}%
\pgfsys@defobject{currentmarker}{\pgfqpoint{0.000000in}{-0.044444in}}{\pgfqpoint{0.000000in}{0.000000in}}{%
\pgfpathmoveto{\pgfqpoint{0.000000in}{0.000000in}}%
\pgfpathlineto{\pgfqpoint{0.000000in}{-0.044444in}}%
\pgfusepath{stroke,fill}%
}%
\begin{pgfscope}%
\pgfsys@transformshift{1.656567in}{4.083121in}%
\pgfsys@useobject{currentmarker}{}%
\end{pgfscope}%
\end{pgfscope}%
\begin{pgfscope}%
\pgfsetbuttcap%
\pgfsetroundjoin%
\definecolor{currentfill}{rgb}{0.150000,0.150000,0.150000}%
\pgfsetfillcolor{currentfill}%
\pgfsetlinewidth{0.803000pt}%
\definecolor{currentstroke}{rgb}{0.150000,0.150000,0.150000}%
\pgfsetstrokecolor{currentstroke}%
\pgfsetdash{}{0pt}%
\pgfsys@defobject{currentmarker}{\pgfqpoint{0.000000in}{-0.044444in}}{\pgfqpoint{0.000000in}{0.000000in}}{%
\pgfpathmoveto{\pgfqpoint{0.000000in}{0.000000in}}%
\pgfpathlineto{\pgfqpoint{0.000000in}{-0.044444in}}%
\pgfusepath{stroke,fill}%
}%
\begin{pgfscope}%
\pgfsys@transformshift{1.682215in}{4.083121in}%
\pgfsys@useobject{currentmarker}{}%
\end{pgfscope}%
\end{pgfscope}%
\begin{pgfscope}%
\pgfsetbuttcap%
\pgfsetroundjoin%
\definecolor{currentfill}{rgb}{0.150000,0.150000,0.150000}%
\pgfsetfillcolor{currentfill}%
\pgfsetlinewidth{0.803000pt}%
\definecolor{currentstroke}{rgb}{0.150000,0.150000,0.150000}%
\pgfsetstrokecolor{currentstroke}%
\pgfsetdash{}{0pt}%
\pgfsys@defobject{currentmarker}{\pgfqpoint{0.000000in}{-0.044444in}}{\pgfqpoint{0.000000in}{0.000000in}}{%
\pgfpathmoveto{\pgfqpoint{0.000000in}{0.000000in}}%
\pgfpathlineto{\pgfqpoint{0.000000in}{-0.044444in}}%
\pgfusepath{stroke,fill}%
}%
\begin{pgfscope}%
\pgfsys@transformshift{1.856098in}{4.083121in}%
\pgfsys@useobject{currentmarker}{}%
\end{pgfscope}%
\end{pgfscope}%
\begin{pgfscope}%
\pgfsetbuttcap%
\pgfsetroundjoin%
\definecolor{currentfill}{rgb}{0.150000,0.150000,0.150000}%
\pgfsetfillcolor{currentfill}%
\pgfsetlinewidth{1.003750pt}%
\definecolor{currentstroke}{rgb}{0.150000,0.150000,0.150000}%
\pgfsetstrokecolor{currentstroke}%
\pgfsetdash{}{0pt}%
\pgfsys@defobject{currentmarker}{\pgfqpoint{-0.066667in}{0.000000in}}{\pgfqpoint{0.000000in}{0.000000in}}{%
\pgfpathmoveto{\pgfqpoint{0.000000in}{0.000000in}}%
\pgfpathlineto{\pgfqpoint{-0.066667in}{0.000000in}}%
\pgfusepath{stroke,fill}%
}%
\begin{pgfscope}%
\pgfsys@transformshift{0.702340in}{4.083121in}%
\pgfsys@useobject{currentmarker}{}%
\end{pgfscope}%
\end{pgfscope}%
\begin{pgfscope}%
\definecolor{textcolor}{rgb}{0.150000,0.150000,0.150000}%
\pgfsetstrokecolor{textcolor}%
\pgfsetfillcolor{textcolor}%
\pgftext[x=0.374971in,y=4.058173in,left,base]{\color{textcolor}\sffamily\fontsize{5.176471}{6.211765}\selectfont 11.000}%
\end{pgfscope}%
\begin{pgfscope}%
\pgfsetbuttcap%
\pgfsetroundjoin%
\definecolor{currentfill}{rgb}{0.150000,0.150000,0.150000}%
\pgfsetfillcolor{currentfill}%
\pgfsetlinewidth{1.003750pt}%
\definecolor{currentstroke}{rgb}{0.150000,0.150000,0.150000}%
\pgfsetstrokecolor{currentstroke}%
\pgfsetdash{}{0pt}%
\pgfsys@defobject{currentmarker}{\pgfqpoint{-0.066667in}{0.000000in}}{\pgfqpoint{0.000000in}{0.000000in}}{%
\pgfpathmoveto{\pgfqpoint{0.000000in}{0.000000in}}%
\pgfpathlineto{\pgfqpoint{-0.066667in}{0.000000in}}%
\pgfusepath{stroke,fill}%
}%
\begin{pgfscope}%
\pgfsys@transformshift{0.702340in}{4.509594in}%
\pgfsys@useobject{currentmarker}{}%
\end{pgfscope}%
\end{pgfscope}%
\begin{pgfscope}%
\definecolor{textcolor}{rgb}{0.150000,0.150000,0.150000}%
\pgfsetstrokecolor{textcolor}%
\pgfsetfillcolor{textcolor}%
\pgftext[x=0.374971in,y=4.484646in,left,base]{\color{textcolor}\sffamily\fontsize{5.176471}{6.211765}\selectfont 11.701}%
\end{pgfscope}%
\begin{pgfscope}%
\pgfsetbuttcap%
\pgfsetroundjoin%
\definecolor{currentfill}{rgb}{0.150000,0.150000,0.150000}%
\pgfsetfillcolor{currentfill}%
\pgfsetlinewidth{1.003750pt}%
\definecolor{currentstroke}{rgb}{0.150000,0.150000,0.150000}%
\pgfsetstrokecolor{currentstroke}%
\pgfsetdash{}{0pt}%
\pgfsys@defobject{currentmarker}{\pgfqpoint{-0.066667in}{0.000000in}}{\pgfqpoint{0.000000in}{0.000000in}}{%
\pgfpathmoveto{\pgfqpoint{0.000000in}{0.000000in}}%
\pgfpathlineto{\pgfqpoint{-0.066667in}{0.000000in}}%
\pgfusepath{stroke,fill}%
}%
\begin{pgfscope}%
\pgfsys@transformshift{0.702340in}{4.691069in}%
\pgfsys@useobject{currentmarker}{}%
\end{pgfscope}%
\end{pgfscope}%
\begin{pgfscope}%
\definecolor{textcolor}{rgb}{0.150000,0.150000,0.150000}%
\pgfsetstrokecolor{textcolor}%
\pgfsetfillcolor{textcolor}%
\pgftext[x=0.374971in,y=4.666122in,left,base]{\color{textcolor}\sffamily\fontsize{5.176471}{6.211765}\selectfont 12.000}%
\end{pgfscope}%
\begin{pgfscope}%
\definecolor{textcolor}{rgb}{0.150000,0.150000,0.150000}%
\pgfsetstrokecolor{textcolor}%
\pgfsetfillcolor{textcolor}%
\pgftext[x=0.319416in,y=4.387095in,,bottom,rotate=90.000000]{\color{textcolor}\sffamily\fontsize{5.647059}{6.776471}\selectfont \(\displaystyle x = \frac{2 \mu E L^2}{4 \pi^2}\)}%
\end{pgfscope}%
\begin{pgfscope}%
\pgfpathrectangle{\pgfqpoint{0.702340in}{4.083121in}}{\pgfqpoint{1.223103in}{0.607948in}}%
\pgfusepath{clip}%
\pgfsetroundcap%
\pgfsetroundjoin%
\pgfsetlinewidth{1.204500pt}%
\definecolor{currentstroke}{rgb}{0.000000,0.501961,0.000000}%
\pgfsetstrokecolor{currentstroke}%
\pgfsetdash{}{0pt}%
\pgfpathmoveto{\pgfqpoint{0.702340in}{4.503610in}}%
\pgfpathlineto{\pgfqpoint{0.943050in}{4.501654in}}%
\pgfpathlineto{\pgfqpoint{1.054581in}{4.498600in}}%
\pgfpathlineto{\pgfqpoint{1.127977in}{4.494422in}}%
\pgfpathlineto{\pgfqpoint{1.182772in}{4.489095in}}%
\pgfpathlineto{\pgfqpoint{1.226515in}{4.482596in}}%
\pgfpathlineto{\pgfqpoint{1.262923in}{4.474907in}}%
\pgfpathlineto{\pgfqpoint{1.294107in}{4.466009in}}%
\pgfpathlineto{\pgfqpoint{1.321380in}{4.455889in}}%
\pgfpathlineto{\pgfqpoint{1.345614in}{4.444532in}}%
\pgfpathlineto{\pgfqpoint{1.367419in}{4.431926in}}%
\pgfpathlineto{\pgfqpoint{1.387238in}{4.418062in}}%
\pgfpathlineto{\pgfqpoint{1.405403in}{4.402931in}}%
\pgfpathlineto{\pgfqpoint{1.422168in}{4.386526in}}%
\pgfpathlineto{\pgfqpoint{1.437735in}{4.368840in}}%
\pgfpathlineto{\pgfqpoint{1.452262in}{4.349868in}}%
\pgfpathlineto{\pgfqpoint{1.465881in}{4.329604in}}%
\pgfpathlineto{\pgfqpoint{1.478698in}{4.308045in}}%
\pgfpathlineto{\pgfqpoint{1.490802in}{4.285187in}}%
\pgfpathlineto{\pgfqpoint{1.502269in}{4.261027in}}%
\pgfpathlineto{\pgfqpoint{1.513162in}{4.235560in}}%
\pgfpathlineto{\pgfqpoint{1.523536in}{4.208783in}}%
\pgfpathlineto{\pgfqpoint{1.533438in}{4.180692in}}%
\pgfpathlineto{\pgfqpoint{1.542909in}{4.151282in}}%
\pgfpathlineto{\pgfqpoint{1.551986in}{4.120549in}}%
\pgfpathlineto{\pgfqpoint{1.560699in}{4.088487in}}%
\pgfpathlineto{\pgfqpoint{1.562881in}{4.079788in}}%
\pgfusepath{stroke}%
\end{pgfscope}%
\begin{pgfscope}%
\pgfsetrectcap%
\pgfsetmiterjoin%
\pgfsetlinewidth{1.003750pt}%
\definecolor{currentstroke}{rgb}{0.150000,0.150000,0.150000}%
\pgfsetstrokecolor{currentstroke}%
\pgfsetdash{}{0pt}%
\pgfpathmoveto{\pgfqpoint{0.702340in}{4.083121in}}%
\pgfpathlineto{\pgfqpoint{0.702340in}{4.691069in}}%
\pgfusepath{stroke}%
\end{pgfscope}%
\begin{pgfscope}%
\pgfsetrectcap%
\pgfsetmiterjoin%
\pgfsetlinewidth{1.003750pt}%
\definecolor{currentstroke}{rgb}{0.150000,0.150000,0.150000}%
\pgfsetstrokecolor{currentstroke}%
\pgfsetdash{}{0pt}%
\pgfpathmoveto{\pgfqpoint{0.702340in}{4.083121in}}%
\pgfpathlineto{\pgfqpoint{1.925444in}{4.083121in}}%
\pgfusepath{stroke}%
\end{pgfscope}%
\begin{pgfscope}%
\pgfpathrectangle{\pgfqpoint{0.702340in}{4.083121in}}{\pgfqpoint{1.223103in}{0.607948in}}%
\pgfusepath{clip}%
\pgfsetbuttcap%
\pgfsetroundjoin%
\definecolor{currentfill}{rgb}{0.000000,0.000000,0.000000}%
\pgfsetfillcolor{currentfill}%
\pgfsetlinewidth{1.003750pt}%
\definecolor{currentstroke}{rgb}{0.000000,0.000000,0.000000}%
\pgfsetstrokecolor{currentstroke}%
\pgfsetdash{}{0pt}%
\pgfsys@defobject{currentmarker}{\pgfqpoint{-0.013889in}{-0.013889in}}{\pgfqpoint{0.013889in}{0.013889in}}{%
\pgfpathmoveto{\pgfqpoint{0.000000in}{-0.013889in}}%
\pgfpathcurveto{\pgfqpoint{0.003683in}{-0.013889in}}{\pgfqpoint{0.007216in}{-0.012425in}}{\pgfqpoint{0.009821in}{-0.009821in}}%
\pgfpathcurveto{\pgfqpoint{0.012425in}{-0.007216in}}{\pgfqpoint{0.013889in}{-0.003683in}}{\pgfqpoint{0.013889in}{0.000000in}}%
\pgfpathcurveto{\pgfqpoint{0.013889in}{0.003683in}}{\pgfqpoint{0.012425in}{0.007216in}}{\pgfqpoint{0.009821in}{0.009821in}}%
\pgfpathcurveto{\pgfqpoint{0.007216in}{0.012425in}}{\pgfqpoint{0.003683in}{0.013889in}}{\pgfqpoint{0.000000in}{0.013889in}}%
\pgfpathcurveto{\pgfqpoint{-0.003683in}{0.013889in}}{\pgfqpoint{-0.007216in}{0.012425in}}{\pgfqpoint{-0.009821in}{0.009821in}}%
\pgfpathcurveto{\pgfqpoint{-0.012425in}{0.007216in}}{\pgfqpoint{-0.013889in}{0.003683in}}{\pgfqpoint{-0.013889in}{0.000000in}}%
\pgfpathcurveto{\pgfqpoint{-0.013889in}{-0.003683in}}{\pgfqpoint{-0.012425in}{-0.007216in}}{\pgfqpoint{-0.009821in}{-0.009821in}}%
\pgfpathcurveto{\pgfqpoint{-0.007216in}{-0.012425in}}{\pgfqpoint{-0.003683in}{-0.013889in}}{\pgfqpoint{0.000000in}{-0.013889in}}%
\pgfpathclose%
\pgfusepath{stroke,fill}%
}%
\begin{pgfscope}%
\pgfsys@transformshift{1.705158in}{2.874071in}%
\pgfsys@useobject{currentmarker}{}%
\end{pgfscope}%
\begin{pgfscope}%
\pgfsys@transformshift{1.616865in}{3.807445in}%
\pgfsys@useobject{currentmarker}{}%
\end{pgfscope}%
\begin{pgfscope}%
\pgfsys@transformshift{1.554219in}{4.111614in}%
\pgfsys@useobject{currentmarker}{}%
\end{pgfscope}%
\begin{pgfscope}%
\pgfsys@transformshift{1.505628in}{4.252564in}%
\pgfsys@useobject{currentmarker}{}%
\end{pgfscope}%
\begin{pgfscope}%
\pgfsys@transformshift{1.465926in}{4.328862in}%
\pgfsys@useobject{currentmarker}{}%
\end{pgfscope}%
\begin{pgfscope}%
\pgfsys@transformshift{1.432358in}{4.374802in}%
\pgfsys@useobject{currentmarker}{}%
\end{pgfscope}%
\begin{pgfscope}%
\pgfsys@transformshift{1.403280in}{4.404663in}%
\pgfsys@useobject{currentmarker}{}%
\end{pgfscope}%
\begin{pgfscope}%
\pgfsys@transformshift{1.397903in}{4.409373in}%
\pgfsys@useobject{currentmarker}{}%
\end{pgfscope}%
\begin{pgfscope}%
\pgfsys@transformshift{1.392656in}{4.413755in}%
\pgfsys@useobject{currentmarker}{}%
\end{pgfscope}%
\begin{pgfscope}%
\pgfsys@transformshift{1.387532in}{4.417838in}%
\pgfsys@useobject{currentmarker}{}%
\end{pgfscope}%
\begin{pgfscope}%
\pgfsys@transformshift{1.382525in}{4.421649in}%
\pgfsys@useobject{currentmarker}{}%
\end{pgfscope}%
\begin{pgfscope}%
\pgfsys@transformshift{1.377632in}{4.425213in}%
\pgfsys@useobject{currentmarker}{}%
\end{pgfscope}%
\begin{pgfscope}%
\pgfsys@transformshift{1.372846in}{4.428551in}%
\pgfsys@useobject{currentmarker}{}%
\end{pgfscope}%
\begin{pgfscope}%
\pgfsys@transformshift{1.368163in}{4.431682in}%
\pgfsys@useobject{currentmarker}{}%
\end{pgfscope}%
\begin{pgfscope}%
\pgfsys@transformshift{1.363578in}{4.434623in}%
\pgfsys@useobject{currentmarker}{}%
\end{pgfscope}%
\begin{pgfscope}%
\pgfsys@transformshift{1.359088in}{4.437389in}%
\pgfsys@useobject{currentmarker}{}%
\end{pgfscope}%
\begin{pgfscope}%
\pgfsys@transformshift{1.354689in}{4.439994in}%
\pgfsys@useobject{currentmarker}{}%
\end{pgfscope}%
\end{pgfscope}%
\begin{pgfscope}%
\pgfsetbuttcap%
\pgfsetmiterjoin%
\definecolor{currentfill}{rgb}{1.000000,1.000000,1.000000}%
\pgfsetfillcolor{currentfill}%
\pgfsetlinewidth{0.000000pt}%
\definecolor{currentstroke}{rgb}{0.000000,0.000000,0.000000}%
\pgfsetstrokecolor{currentstroke}%
\pgfsetstrokeopacity{0.000000}%
\pgfsetdash{}{0pt}%
\pgfpathmoveto{\pgfqpoint{2.170064in}{4.083121in}}%
\pgfpathlineto{\pgfqpoint{3.393168in}{4.083121in}}%
\pgfpathlineto{\pgfqpoint{3.393168in}{4.691069in}}%
\pgfpathlineto{\pgfqpoint{2.170064in}{4.691069in}}%
\pgfpathclose%
\pgfusepath{fill}%
\end{pgfscope}%
\begin{pgfscope}%
\pgfpathrectangle{\pgfqpoint{2.170064in}{4.083121in}}{\pgfqpoint{1.223103in}{0.607948in}}%
\pgfusepath{clip}%
\pgfsetbuttcap%
\pgfsetmiterjoin%
\definecolor{currentfill}{rgb}{0.000000,0.000000,1.000000}%
\pgfsetfillcolor{currentfill}%
\pgfsetfillopacity{0.100000}%
\pgfsetlinewidth{0.803000pt}%
\definecolor{currentstroke}{rgb}{0.000000,0.000000,1.000000}%
\pgfsetstrokecolor{currentstroke}%
\pgfsetstrokeopacity{0.100000}%
\pgfsetdash{}{0pt}%
\pgfpathmoveto{\pgfqpoint{2.170064in}{4.441609in}}%
\pgfpathlineto{\pgfqpoint{2.170064in}{4.540897in}}%
\pgfpathlineto{\pgfqpoint{3.393168in}{4.540897in}}%
\pgfpathlineto{\pgfqpoint{3.393168in}{4.441609in}}%
\pgfpathclose%
\pgfusepath{stroke,fill}%
\end{pgfscope}%
\begin{pgfscope}%
\pgfpathrectangle{\pgfqpoint{2.170064in}{4.083121in}}{\pgfqpoint{1.223103in}{0.607948in}}%
\pgfusepath{clip}%
\pgfsetbuttcap%
\pgfsetroundjoin%
\definecolor{currentfill}{rgb}{0.000000,0.501961,0.000000}%
\pgfsetfillcolor{currentfill}%
\pgfsetfillopacity{0.500000}%
\pgfsetlinewidth{0.803000pt}%
\definecolor{currentstroke}{rgb}{0.000000,0.501961,0.000000}%
\pgfsetstrokecolor{currentstroke}%
\pgfsetstrokeopacity{0.500000}%
\pgfsetdash{}{0pt}%
\pgfpathmoveto{\pgfqpoint{2.170064in}{4.538875in}}%
\pgfpathlineto{\pgfqpoint{2.170064in}{4.446521in}}%
\pgfpathlineto{\pgfqpoint{2.410774in}{4.455982in}}%
\pgfpathlineto{\pgfqpoint{2.522305in}{4.464823in}}%
\pgfpathlineto{\pgfqpoint{2.595701in}{4.473050in}}%
\pgfpathlineto{\pgfqpoint{2.650497in}{4.480664in}}%
\pgfpathlineto{\pgfqpoint{2.694239in}{4.487667in}}%
\pgfpathlineto{\pgfqpoint{2.730647in}{4.494060in}}%
\pgfpathlineto{\pgfqpoint{2.761831in}{4.499843in}}%
\pgfpathlineto{\pgfqpoint{2.789104in}{4.505013in}}%
\pgfpathlineto{\pgfqpoint{2.813338in}{4.509568in}}%
\pgfpathlineto{\pgfqpoint{2.835143in}{4.513503in}}%
\pgfpathlineto{\pgfqpoint{2.854962in}{4.516466in}}%
\pgfpathlineto{\pgfqpoint{2.873127in}{4.515710in}}%
\pgfpathlineto{\pgfqpoint{2.889892in}{4.514949in}}%
\pgfpathlineto{\pgfqpoint{2.905459in}{4.514187in}}%
\pgfpathlineto{\pgfqpoint{2.919986in}{4.513384in}}%
\pgfpathlineto{\pgfqpoint{2.933605in}{4.512498in}}%
\pgfpathlineto{\pgfqpoint{2.946422in}{4.511485in}}%
\pgfpathlineto{\pgfqpoint{2.958526in}{4.510299in}}%
\pgfpathlineto{\pgfqpoint{2.969993in}{4.508895in}}%
\pgfpathlineto{\pgfqpoint{2.980886in}{4.507224in}}%
\pgfpathlineto{\pgfqpoint{2.991260in}{4.505234in}}%
\pgfpathlineto{\pgfqpoint{3.001162in}{4.502870in}}%
\pgfpathlineto{\pgfqpoint{3.010633in}{4.500068in}}%
\pgfpathlineto{\pgfqpoint{3.019710in}{4.496725in}}%
\pgfpathlineto{\pgfqpoint{3.028423in}{4.492365in}}%
\pgfpathlineto{\pgfqpoint{3.036801in}{4.485787in}}%
\pgfpathlineto{\pgfqpoint{3.044869in}{4.477915in}}%
\pgfpathlineto{\pgfqpoint{3.052648in}{4.469145in}}%
\pgfpathlineto{\pgfqpoint{3.060159in}{4.459505in}}%
\pgfpathlineto{\pgfqpoint{3.067420in}{4.448987in}}%
\pgfpathlineto{\pgfqpoint{3.074446in}{4.437576in}}%
\pgfpathlineto{\pgfqpoint{3.081253in}{4.425250in}}%
\pgfpathlineto{\pgfqpoint{3.087853in}{4.411991in}}%
\pgfpathlineto{\pgfqpoint{3.094259in}{4.397775in}}%
\pgfpathlineto{\pgfqpoint{3.100482in}{4.382581in}}%
\pgfpathlineto{\pgfqpoint{3.106532in}{4.366386in}}%
\pgfpathlineto{\pgfqpoint{3.112419in}{4.349167in}}%
\pgfpathlineto{\pgfqpoint{3.118150in}{4.330898in}}%
\pgfpathlineto{\pgfqpoint{3.123735in}{4.311558in}}%
\pgfpathlineto{\pgfqpoint{3.129180in}{4.291119in}}%
\pgfpathlineto{\pgfqpoint{3.134492in}{4.269558in}}%
\pgfpathlineto{\pgfqpoint{3.139677in}{4.246849in}}%
\pgfpathlineto{\pgfqpoint{3.144742in}{4.222966in}}%
\pgfpathlineto{\pgfqpoint{3.149692in}{4.197883in}}%
\pgfpathlineto{\pgfqpoint{3.154532in}{4.171573in}}%
\pgfpathlineto{\pgfqpoint{3.159267in}{4.144009in}}%
\pgfpathlineto{\pgfqpoint{3.163901in}{4.115155in}}%
\pgfpathlineto{\pgfqpoint{3.168438in}{4.084896in}}%
\pgfpathlineto{\pgfqpoint{3.172883in}{4.047254in}}%
\pgfpathlineto{\pgfqpoint{3.172883in}{4.053698in}}%
\pgfpathlineto{\pgfqpoint{3.172883in}{4.053698in}}%
\pgfpathlineto{\pgfqpoint{3.168438in}{4.088590in}}%
\pgfpathlineto{\pgfqpoint{3.163901in}{4.126706in}}%
\pgfpathlineto{\pgfqpoint{3.159267in}{4.162239in}}%
\pgfpathlineto{\pgfqpoint{3.154532in}{4.195235in}}%
\pgfpathlineto{\pgfqpoint{3.149692in}{4.225820in}}%
\pgfpathlineto{\pgfqpoint{3.144742in}{4.254120in}}%
\pgfpathlineto{\pgfqpoint{3.139677in}{4.280256in}}%
\pgfpathlineto{\pgfqpoint{3.134492in}{4.304348in}}%
\pgfpathlineto{\pgfqpoint{3.129180in}{4.326510in}}%
\pgfpathlineto{\pgfqpoint{3.123735in}{4.346851in}}%
\pgfpathlineto{\pgfqpoint{3.118150in}{4.365478in}}%
\pgfpathlineto{\pgfqpoint{3.112419in}{4.382494in}}%
\pgfpathlineto{\pgfqpoint{3.106532in}{4.397996in}}%
\pgfpathlineto{\pgfqpoint{3.100482in}{4.412078in}}%
\pgfpathlineto{\pgfqpoint{3.094259in}{4.424833in}}%
\pgfpathlineto{\pgfqpoint{3.087853in}{4.436348in}}%
\pgfpathlineto{\pgfqpoint{3.081253in}{4.446708in}}%
\pgfpathlineto{\pgfqpoint{3.074446in}{4.455992in}}%
\pgfpathlineto{\pgfqpoint{3.067420in}{4.464281in}}%
\pgfpathlineto{\pgfqpoint{3.060159in}{4.471651in}}%
\pgfpathlineto{\pgfqpoint{3.052648in}{4.478178in}}%
\pgfpathlineto{\pgfqpoint{3.044869in}{4.483942in}}%
\pgfpathlineto{\pgfqpoint{3.036801in}{4.489058in}}%
\pgfpathlineto{\pgfqpoint{3.028423in}{4.494003in}}%
\pgfpathlineto{\pgfqpoint{3.019710in}{4.499781in}}%
\pgfpathlineto{\pgfqpoint{3.010633in}{4.505269in}}%
\pgfpathlineto{\pgfqpoint{3.001162in}{4.510063in}}%
\pgfpathlineto{\pgfqpoint{2.991260in}{4.514130in}}%
\pgfpathlineto{\pgfqpoint{2.980886in}{4.517475in}}%
\pgfpathlineto{\pgfqpoint{2.969993in}{4.520108in}}%
\pgfpathlineto{\pgfqpoint{2.958526in}{4.522039in}}%
\pgfpathlineto{\pgfqpoint{2.946422in}{4.523281in}}%
\pgfpathlineto{\pgfqpoint{2.933605in}{4.523844in}}%
\pgfpathlineto{\pgfqpoint{2.919986in}{4.523740in}}%
\pgfpathlineto{\pgfqpoint{2.905459in}{4.522978in}}%
\pgfpathlineto{\pgfqpoint{2.889892in}{4.521566in}}%
\pgfpathlineto{\pgfqpoint{2.873127in}{4.519515in}}%
\pgfpathlineto{\pgfqpoint{2.854962in}{4.516876in}}%
\pgfpathlineto{\pgfqpoint{2.835143in}{4.517408in}}%
\pgfpathlineto{\pgfqpoint{2.813338in}{4.518408in}}%
\pgfpathlineto{\pgfqpoint{2.789104in}{4.519567in}}%
\pgfpathlineto{\pgfqpoint{2.761831in}{4.520919in}}%
\pgfpathlineto{\pgfqpoint{2.730647in}{4.522503in}}%
\pgfpathlineto{\pgfqpoint{2.694239in}{4.524353in}}%
\pgfpathlineto{\pgfqpoint{2.650497in}{4.526507in}}%
\pgfpathlineto{\pgfqpoint{2.595701in}{4.529000in}}%
\pgfpathlineto{\pgfqpoint{2.522305in}{4.531869in}}%
\pgfpathlineto{\pgfqpoint{2.410774in}{4.535148in}}%
\pgfpathlineto{\pgfqpoint{2.170064in}{4.538875in}}%
\pgfpathclose%
\pgfusepath{stroke,fill}%
\end{pgfscope}%
\begin{pgfscope}%
\pgfpathrectangle{\pgfqpoint{2.170064in}{4.083121in}}{\pgfqpoint{1.223103in}{0.607948in}}%
\pgfusepath{clip}%
\pgfsetroundcap%
\pgfsetroundjoin%
\pgfsetlinewidth{0.501875pt}%
\definecolor{currentstroke}{rgb}{0.000000,0.000000,1.000000}%
\pgfsetstrokecolor{currentstroke}%
\pgfsetstrokeopacity{0.800000}%
\pgfsetdash{}{0pt}%
\pgfpathmoveto{\pgfqpoint{2.170064in}{4.491253in}}%
\pgfpathlineto{\pgfqpoint{3.393168in}{4.491253in}}%
\pgfusepath{stroke}%
\end{pgfscope}%
\begin{pgfscope}%
\pgfpathrectangle{\pgfqpoint{2.170064in}{4.083121in}}{\pgfqpoint{1.223103in}{0.607948in}}%
\pgfusepath{clip}%
\pgfsetbuttcap%
\pgfsetroundjoin%
\pgfsetlinewidth{1.003750pt}%
\definecolor{currentstroke}{rgb}{0.000000,0.000000,0.000000}%
\pgfsetstrokecolor{currentstroke}%
\pgfsetdash{{3.700000pt}{1.600000pt}}{0.000000pt}%
\pgfpathmoveto{\pgfqpoint{2.170064in}{4.509594in}}%
\pgfpathlineto{\pgfqpoint{3.393168in}{4.509594in}}%
\pgfusepath{stroke}%
\end{pgfscope}%
\begin{pgfscope}%
\pgfsetroundcap%
\pgfsetroundjoin%
\pgfsetlinewidth{0.501875pt}%
\definecolor{currentstroke}{rgb}{0.000000,0.000000,1.000000}%
\pgfsetstrokecolor{currentstroke}%
\pgfsetstrokeopacity{0.800000}%
\pgfsetdash{}{0pt}%
\pgfpathmoveto{\pgfqpoint{2.983913in}{4.608590in}}%
\pgfpathquadraticcurveto{\pgfqpoint{2.914395in}{4.558082in}}{\pgfqpoint{2.844877in}{4.507574in}}%
\pgfusepath{stroke}%
\end{pgfscope}%
\begin{pgfscope}%
\pgfsetfillopacity{0.800000}%
\pgfsetstrokeopacity{0.800000}%
\definecolor{textcolor}{rgb}{0.000000,0.000000,1.000000}%
\pgfsetstrokecolor{textcolor}%
\pgfsetfillcolor{textcolor}%
\pgftext[x=2.910706in,y=4.673637in,left,base]{\color{textcolor}\sffamily\fontsize{5.647059}{6.776471}\selectfont 11.671(82)}%
\end{pgfscope}%
\begin{pgfscope}%
\pgfsetbuttcap%
\pgfsetroundjoin%
\definecolor{currentfill}{rgb}{0.150000,0.150000,0.150000}%
\pgfsetfillcolor{currentfill}%
\pgfsetlinewidth{1.003750pt}%
\definecolor{currentstroke}{rgb}{0.150000,0.150000,0.150000}%
\pgfsetstrokecolor{currentstroke}%
\pgfsetdash{}{0pt}%
\pgfsys@defobject{currentmarker}{\pgfqpoint{0.000000in}{-0.066667in}}{\pgfqpoint{0.000000in}{0.000000in}}{%
\pgfpathmoveto{\pgfqpoint{0.000000in}{0.000000in}}%
\pgfpathlineto{\pgfqpoint{0.000000in}{-0.066667in}}%
\pgfusepath{stroke,fill}%
}%
\begin{pgfscope}%
\pgfsys@transformshift{2.170064in}{4.083121in}%
\pgfsys@useobject{currentmarker}{}%
\end{pgfscope}%
\end{pgfscope}%
\begin{pgfscope}%
\pgfsetbuttcap%
\pgfsetroundjoin%
\definecolor{currentfill}{rgb}{0.150000,0.150000,0.150000}%
\pgfsetfillcolor{currentfill}%
\pgfsetlinewidth{1.003750pt}%
\definecolor{currentstroke}{rgb}{0.150000,0.150000,0.150000}%
\pgfsetstrokecolor{currentstroke}%
\pgfsetdash{}{0pt}%
\pgfsys@defobject{currentmarker}{\pgfqpoint{0.000000in}{-0.066667in}}{\pgfqpoint{0.000000in}{0.000000in}}{%
\pgfpathmoveto{\pgfqpoint{0.000000in}{0.000000in}}%
\pgfpathlineto{\pgfqpoint{0.000000in}{-0.066667in}}%
\pgfusepath{stroke,fill}%
}%
\begin{pgfscope}%
\pgfsys@transformshift{2.671473in}{4.083121in}%
\pgfsys@useobject{currentmarker}{}%
\end{pgfscope}%
\end{pgfscope}%
\begin{pgfscope}%
\pgfsetbuttcap%
\pgfsetroundjoin%
\definecolor{currentfill}{rgb}{0.150000,0.150000,0.150000}%
\pgfsetfillcolor{currentfill}%
\pgfsetlinewidth{1.003750pt}%
\definecolor{currentstroke}{rgb}{0.150000,0.150000,0.150000}%
\pgfsetstrokecolor{currentstroke}%
\pgfsetdash{}{0pt}%
\pgfsys@defobject{currentmarker}{\pgfqpoint{0.000000in}{-0.066667in}}{\pgfqpoint{0.000000in}{0.000000in}}{%
\pgfpathmoveto{\pgfqpoint{0.000000in}{0.000000in}}%
\pgfpathlineto{\pgfqpoint{0.000000in}{-0.066667in}}%
\pgfusepath{stroke,fill}%
}%
\begin{pgfscope}%
\pgfsys@transformshift{3.172883in}{4.083121in}%
\pgfsys@useobject{currentmarker}{}%
\end{pgfscope}%
\end{pgfscope}%
\begin{pgfscope}%
\pgfsetbuttcap%
\pgfsetroundjoin%
\definecolor{currentfill}{rgb}{0.150000,0.150000,0.150000}%
\pgfsetfillcolor{currentfill}%
\pgfsetlinewidth{0.803000pt}%
\definecolor{currentstroke}{rgb}{0.150000,0.150000,0.150000}%
\pgfsetstrokecolor{currentstroke}%
\pgfsetdash{}{0pt}%
\pgfsys@defobject{currentmarker}{\pgfqpoint{0.000000in}{-0.044444in}}{\pgfqpoint{0.000000in}{0.000000in}}{%
\pgfpathmoveto{\pgfqpoint{0.000000in}{0.000000in}}%
\pgfpathlineto{\pgfqpoint{0.000000in}{-0.044444in}}%
\pgfusepath{stroke,fill}%
}%
\begin{pgfscope}%
\pgfsys@transformshift{2.321004in}{4.083121in}%
\pgfsys@useobject{currentmarker}{}%
\end{pgfscope}%
\end{pgfscope}%
\begin{pgfscope}%
\pgfsetbuttcap%
\pgfsetroundjoin%
\definecolor{currentfill}{rgb}{0.150000,0.150000,0.150000}%
\pgfsetfillcolor{currentfill}%
\pgfsetlinewidth{0.803000pt}%
\definecolor{currentstroke}{rgb}{0.150000,0.150000,0.150000}%
\pgfsetstrokecolor{currentstroke}%
\pgfsetdash{}{0pt}%
\pgfsys@defobject{currentmarker}{\pgfqpoint{0.000000in}{-0.044444in}}{\pgfqpoint{0.000000in}{0.000000in}}{%
\pgfpathmoveto{\pgfqpoint{0.000000in}{0.000000in}}%
\pgfpathlineto{\pgfqpoint{0.000000in}{-0.044444in}}%
\pgfusepath{stroke,fill}%
}%
\begin{pgfscope}%
\pgfsys@transformshift{2.409297in}{4.083121in}%
\pgfsys@useobject{currentmarker}{}%
\end{pgfscope}%
\end{pgfscope}%
\begin{pgfscope}%
\pgfsetbuttcap%
\pgfsetroundjoin%
\definecolor{currentfill}{rgb}{0.150000,0.150000,0.150000}%
\pgfsetfillcolor{currentfill}%
\pgfsetlinewidth{0.803000pt}%
\definecolor{currentstroke}{rgb}{0.150000,0.150000,0.150000}%
\pgfsetstrokecolor{currentstroke}%
\pgfsetdash{}{0pt}%
\pgfsys@defobject{currentmarker}{\pgfqpoint{0.000000in}{-0.044444in}}{\pgfqpoint{0.000000in}{0.000000in}}{%
\pgfpathmoveto{\pgfqpoint{0.000000in}{0.000000in}}%
\pgfpathlineto{\pgfqpoint{0.000000in}{-0.044444in}}%
\pgfusepath{stroke,fill}%
}%
\begin{pgfscope}%
\pgfsys@transformshift{2.471943in}{4.083121in}%
\pgfsys@useobject{currentmarker}{}%
\end{pgfscope}%
\end{pgfscope}%
\begin{pgfscope}%
\pgfsetbuttcap%
\pgfsetroundjoin%
\definecolor{currentfill}{rgb}{0.150000,0.150000,0.150000}%
\pgfsetfillcolor{currentfill}%
\pgfsetlinewidth{0.803000pt}%
\definecolor{currentstroke}{rgb}{0.150000,0.150000,0.150000}%
\pgfsetstrokecolor{currentstroke}%
\pgfsetdash{}{0pt}%
\pgfsys@defobject{currentmarker}{\pgfqpoint{0.000000in}{-0.044444in}}{\pgfqpoint{0.000000in}{0.000000in}}{%
\pgfpathmoveto{\pgfqpoint{0.000000in}{0.000000in}}%
\pgfpathlineto{\pgfqpoint{0.000000in}{-0.044444in}}%
\pgfusepath{stroke,fill}%
}%
\begin{pgfscope}%
\pgfsys@transformshift{2.520534in}{4.083121in}%
\pgfsys@useobject{currentmarker}{}%
\end{pgfscope}%
\end{pgfscope}%
\begin{pgfscope}%
\pgfsetbuttcap%
\pgfsetroundjoin%
\definecolor{currentfill}{rgb}{0.150000,0.150000,0.150000}%
\pgfsetfillcolor{currentfill}%
\pgfsetlinewidth{0.803000pt}%
\definecolor{currentstroke}{rgb}{0.150000,0.150000,0.150000}%
\pgfsetstrokecolor{currentstroke}%
\pgfsetdash{}{0pt}%
\pgfsys@defobject{currentmarker}{\pgfqpoint{0.000000in}{-0.044444in}}{\pgfqpoint{0.000000in}{0.000000in}}{%
\pgfpathmoveto{\pgfqpoint{0.000000in}{0.000000in}}%
\pgfpathlineto{\pgfqpoint{0.000000in}{-0.044444in}}%
\pgfusepath{stroke,fill}%
}%
\begin{pgfscope}%
\pgfsys@transformshift{2.560237in}{4.083121in}%
\pgfsys@useobject{currentmarker}{}%
\end{pgfscope}%
\end{pgfscope}%
\begin{pgfscope}%
\pgfsetbuttcap%
\pgfsetroundjoin%
\definecolor{currentfill}{rgb}{0.150000,0.150000,0.150000}%
\pgfsetfillcolor{currentfill}%
\pgfsetlinewidth{0.803000pt}%
\definecolor{currentstroke}{rgb}{0.150000,0.150000,0.150000}%
\pgfsetstrokecolor{currentstroke}%
\pgfsetdash{}{0pt}%
\pgfsys@defobject{currentmarker}{\pgfqpoint{0.000000in}{-0.044444in}}{\pgfqpoint{0.000000in}{0.000000in}}{%
\pgfpathmoveto{\pgfqpoint{0.000000in}{0.000000in}}%
\pgfpathlineto{\pgfqpoint{0.000000in}{-0.044444in}}%
\pgfusepath{stroke,fill}%
}%
\begin{pgfscope}%
\pgfsys@transformshift{2.593804in}{4.083121in}%
\pgfsys@useobject{currentmarker}{}%
\end{pgfscope}%
\end{pgfscope}%
\begin{pgfscope}%
\pgfsetbuttcap%
\pgfsetroundjoin%
\definecolor{currentfill}{rgb}{0.150000,0.150000,0.150000}%
\pgfsetfillcolor{currentfill}%
\pgfsetlinewidth{0.803000pt}%
\definecolor{currentstroke}{rgb}{0.150000,0.150000,0.150000}%
\pgfsetstrokecolor{currentstroke}%
\pgfsetdash{}{0pt}%
\pgfsys@defobject{currentmarker}{\pgfqpoint{0.000000in}{-0.044444in}}{\pgfqpoint{0.000000in}{0.000000in}}{%
\pgfpathmoveto{\pgfqpoint{0.000000in}{0.000000in}}%
\pgfpathlineto{\pgfqpoint{0.000000in}{-0.044444in}}%
\pgfusepath{stroke,fill}%
}%
\begin{pgfscope}%
\pgfsys@transformshift{2.622882in}{4.083121in}%
\pgfsys@useobject{currentmarker}{}%
\end{pgfscope}%
\end{pgfscope}%
\begin{pgfscope}%
\pgfsetbuttcap%
\pgfsetroundjoin%
\definecolor{currentfill}{rgb}{0.150000,0.150000,0.150000}%
\pgfsetfillcolor{currentfill}%
\pgfsetlinewidth{0.803000pt}%
\definecolor{currentstroke}{rgb}{0.150000,0.150000,0.150000}%
\pgfsetstrokecolor{currentstroke}%
\pgfsetdash{}{0pt}%
\pgfsys@defobject{currentmarker}{\pgfqpoint{0.000000in}{-0.044444in}}{\pgfqpoint{0.000000in}{0.000000in}}{%
\pgfpathmoveto{\pgfqpoint{0.000000in}{0.000000in}}%
\pgfpathlineto{\pgfqpoint{0.000000in}{-0.044444in}}%
\pgfusepath{stroke,fill}%
}%
\begin{pgfscope}%
\pgfsys@transformshift{2.648530in}{4.083121in}%
\pgfsys@useobject{currentmarker}{}%
\end{pgfscope}%
\end{pgfscope}%
\begin{pgfscope}%
\pgfsetbuttcap%
\pgfsetroundjoin%
\definecolor{currentfill}{rgb}{0.150000,0.150000,0.150000}%
\pgfsetfillcolor{currentfill}%
\pgfsetlinewidth{0.803000pt}%
\definecolor{currentstroke}{rgb}{0.150000,0.150000,0.150000}%
\pgfsetstrokecolor{currentstroke}%
\pgfsetdash{}{0pt}%
\pgfsys@defobject{currentmarker}{\pgfqpoint{0.000000in}{-0.044444in}}{\pgfqpoint{0.000000in}{0.000000in}}{%
\pgfpathmoveto{\pgfqpoint{0.000000in}{0.000000in}}%
\pgfpathlineto{\pgfqpoint{0.000000in}{-0.044444in}}%
\pgfusepath{stroke,fill}%
}%
\begin{pgfscope}%
\pgfsys@transformshift{2.822413in}{4.083121in}%
\pgfsys@useobject{currentmarker}{}%
\end{pgfscope}%
\end{pgfscope}%
\begin{pgfscope}%
\pgfsetbuttcap%
\pgfsetroundjoin%
\definecolor{currentfill}{rgb}{0.150000,0.150000,0.150000}%
\pgfsetfillcolor{currentfill}%
\pgfsetlinewidth{0.803000pt}%
\definecolor{currentstroke}{rgb}{0.150000,0.150000,0.150000}%
\pgfsetstrokecolor{currentstroke}%
\pgfsetdash{}{0pt}%
\pgfsys@defobject{currentmarker}{\pgfqpoint{0.000000in}{-0.044444in}}{\pgfqpoint{0.000000in}{0.000000in}}{%
\pgfpathmoveto{\pgfqpoint{0.000000in}{0.000000in}}%
\pgfpathlineto{\pgfqpoint{0.000000in}{-0.044444in}}%
\pgfusepath{stroke,fill}%
}%
\begin{pgfscope}%
\pgfsys@transformshift{2.910706in}{4.083121in}%
\pgfsys@useobject{currentmarker}{}%
\end{pgfscope}%
\end{pgfscope}%
\begin{pgfscope}%
\pgfsetbuttcap%
\pgfsetroundjoin%
\definecolor{currentfill}{rgb}{0.150000,0.150000,0.150000}%
\pgfsetfillcolor{currentfill}%
\pgfsetlinewidth{0.803000pt}%
\definecolor{currentstroke}{rgb}{0.150000,0.150000,0.150000}%
\pgfsetstrokecolor{currentstroke}%
\pgfsetdash{}{0pt}%
\pgfsys@defobject{currentmarker}{\pgfqpoint{0.000000in}{-0.044444in}}{\pgfqpoint{0.000000in}{0.000000in}}{%
\pgfpathmoveto{\pgfqpoint{0.000000in}{0.000000in}}%
\pgfpathlineto{\pgfqpoint{0.000000in}{-0.044444in}}%
\pgfusepath{stroke,fill}%
}%
\begin{pgfscope}%
\pgfsys@transformshift{2.973352in}{4.083121in}%
\pgfsys@useobject{currentmarker}{}%
\end{pgfscope}%
\end{pgfscope}%
\begin{pgfscope}%
\pgfsetbuttcap%
\pgfsetroundjoin%
\definecolor{currentfill}{rgb}{0.150000,0.150000,0.150000}%
\pgfsetfillcolor{currentfill}%
\pgfsetlinewidth{0.803000pt}%
\definecolor{currentstroke}{rgb}{0.150000,0.150000,0.150000}%
\pgfsetstrokecolor{currentstroke}%
\pgfsetdash{}{0pt}%
\pgfsys@defobject{currentmarker}{\pgfqpoint{0.000000in}{-0.044444in}}{\pgfqpoint{0.000000in}{0.000000in}}{%
\pgfpathmoveto{\pgfqpoint{0.000000in}{0.000000in}}%
\pgfpathlineto{\pgfqpoint{0.000000in}{-0.044444in}}%
\pgfusepath{stroke,fill}%
}%
\begin{pgfscope}%
\pgfsys@transformshift{3.021943in}{4.083121in}%
\pgfsys@useobject{currentmarker}{}%
\end{pgfscope}%
\end{pgfscope}%
\begin{pgfscope}%
\pgfsetbuttcap%
\pgfsetroundjoin%
\definecolor{currentfill}{rgb}{0.150000,0.150000,0.150000}%
\pgfsetfillcolor{currentfill}%
\pgfsetlinewidth{0.803000pt}%
\definecolor{currentstroke}{rgb}{0.150000,0.150000,0.150000}%
\pgfsetstrokecolor{currentstroke}%
\pgfsetdash{}{0pt}%
\pgfsys@defobject{currentmarker}{\pgfqpoint{0.000000in}{-0.044444in}}{\pgfqpoint{0.000000in}{0.000000in}}{%
\pgfpathmoveto{\pgfqpoint{0.000000in}{0.000000in}}%
\pgfpathlineto{\pgfqpoint{0.000000in}{-0.044444in}}%
\pgfusepath{stroke,fill}%
}%
\begin{pgfscope}%
\pgfsys@transformshift{3.061646in}{4.083121in}%
\pgfsys@useobject{currentmarker}{}%
\end{pgfscope}%
\end{pgfscope}%
\begin{pgfscope}%
\pgfsetbuttcap%
\pgfsetroundjoin%
\definecolor{currentfill}{rgb}{0.150000,0.150000,0.150000}%
\pgfsetfillcolor{currentfill}%
\pgfsetlinewidth{0.803000pt}%
\definecolor{currentstroke}{rgb}{0.150000,0.150000,0.150000}%
\pgfsetstrokecolor{currentstroke}%
\pgfsetdash{}{0pt}%
\pgfsys@defobject{currentmarker}{\pgfqpoint{0.000000in}{-0.044444in}}{\pgfqpoint{0.000000in}{0.000000in}}{%
\pgfpathmoveto{\pgfqpoint{0.000000in}{0.000000in}}%
\pgfpathlineto{\pgfqpoint{0.000000in}{-0.044444in}}%
\pgfusepath{stroke,fill}%
}%
\begin{pgfscope}%
\pgfsys@transformshift{3.095213in}{4.083121in}%
\pgfsys@useobject{currentmarker}{}%
\end{pgfscope}%
\end{pgfscope}%
\begin{pgfscope}%
\pgfsetbuttcap%
\pgfsetroundjoin%
\definecolor{currentfill}{rgb}{0.150000,0.150000,0.150000}%
\pgfsetfillcolor{currentfill}%
\pgfsetlinewidth{0.803000pt}%
\definecolor{currentstroke}{rgb}{0.150000,0.150000,0.150000}%
\pgfsetstrokecolor{currentstroke}%
\pgfsetdash{}{0pt}%
\pgfsys@defobject{currentmarker}{\pgfqpoint{0.000000in}{-0.044444in}}{\pgfqpoint{0.000000in}{0.000000in}}{%
\pgfpathmoveto{\pgfqpoint{0.000000in}{0.000000in}}%
\pgfpathlineto{\pgfqpoint{0.000000in}{-0.044444in}}%
\pgfusepath{stroke,fill}%
}%
\begin{pgfscope}%
\pgfsys@transformshift{3.124291in}{4.083121in}%
\pgfsys@useobject{currentmarker}{}%
\end{pgfscope}%
\end{pgfscope}%
\begin{pgfscope}%
\pgfsetbuttcap%
\pgfsetroundjoin%
\definecolor{currentfill}{rgb}{0.150000,0.150000,0.150000}%
\pgfsetfillcolor{currentfill}%
\pgfsetlinewidth{0.803000pt}%
\definecolor{currentstroke}{rgb}{0.150000,0.150000,0.150000}%
\pgfsetstrokecolor{currentstroke}%
\pgfsetdash{}{0pt}%
\pgfsys@defobject{currentmarker}{\pgfqpoint{0.000000in}{-0.044444in}}{\pgfqpoint{0.000000in}{0.000000in}}{%
\pgfpathmoveto{\pgfqpoint{0.000000in}{0.000000in}}%
\pgfpathlineto{\pgfqpoint{0.000000in}{-0.044444in}}%
\pgfusepath{stroke,fill}%
}%
\begin{pgfscope}%
\pgfsys@transformshift{3.149939in}{4.083121in}%
\pgfsys@useobject{currentmarker}{}%
\end{pgfscope}%
\end{pgfscope}%
\begin{pgfscope}%
\pgfsetbuttcap%
\pgfsetroundjoin%
\definecolor{currentfill}{rgb}{0.150000,0.150000,0.150000}%
\pgfsetfillcolor{currentfill}%
\pgfsetlinewidth{0.803000pt}%
\definecolor{currentstroke}{rgb}{0.150000,0.150000,0.150000}%
\pgfsetstrokecolor{currentstroke}%
\pgfsetdash{}{0pt}%
\pgfsys@defobject{currentmarker}{\pgfqpoint{0.000000in}{-0.044444in}}{\pgfqpoint{0.000000in}{0.000000in}}{%
\pgfpathmoveto{\pgfqpoint{0.000000in}{0.000000in}}%
\pgfpathlineto{\pgfqpoint{0.000000in}{-0.044444in}}%
\pgfusepath{stroke,fill}%
}%
\begin{pgfscope}%
\pgfsys@transformshift{3.323822in}{4.083121in}%
\pgfsys@useobject{currentmarker}{}%
\end{pgfscope}%
\end{pgfscope}%
\begin{pgfscope}%
\pgfsetbuttcap%
\pgfsetroundjoin%
\definecolor{currentfill}{rgb}{0.150000,0.150000,0.150000}%
\pgfsetfillcolor{currentfill}%
\pgfsetlinewidth{1.003750pt}%
\definecolor{currentstroke}{rgb}{0.150000,0.150000,0.150000}%
\pgfsetstrokecolor{currentstroke}%
\pgfsetdash{}{0pt}%
\pgfsys@defobject{currentmarker}{\pgfqpoint{-0.066667in}{0.000000in}}{\pgfqpoint{0.000000in}{0.000000in}}{%
\pgfpathmoveto{\pgfqpoint{0.000000in}{0.000000in}}%
\pgfpathlineto{\pgfqpoint{-0.066667in}{0.000000in}}%
\pgfusepath{stroke,fill}%
}%
\begin{pgfscope}%
\pgfsys@transformshift{2.170064in}{4.083121in}%
\pgfsys@useobject{currentmarker}{}%
\end{pgfscope}%
\end{pgfscope}%
\begin{pgfscope}%
\pgfsetbuttcap%
\pgfsetroundjoin%
\definecolor{currentfill}{rgb}{0.150000,0.150000,0.150000}%
\pgfsetfillcolor{currentfill}%
\pgfsetlinewidth{1.003750pt}%
\definecolor{currentstroke}{rgb}{0.150000,0.150000,0.150000}%
\pgfsetstrokecolor{currentstroke}%
\pgfsetdash{}{0pt}%
\pgfsys@defobject{currentmarker}{\pgfqpoint{-0.066667in}{0.000000in}}{\pgfqpoint{0.000000in}{0.000000in}}{%
\pgfpathmoveto{\pgfqpoint{0.000000in}{0.000000in}}%
\pgfpathlineto{\pgfqpoint{-0.066667in}{0.000000in}}%
\pgfusepath{stroke,fill}%
}%
\begin{pgfscope}%
\pgfsys@transformshift{2.170064in}{4.509594in}%
\pgfsys@useobject{currentmarker}{}%
\end{pgfscope}%
\end{pgfscope}%
\begin{pgfscope}%
\pgfsetbuttcap%
\pgfsetroundjoin%
\definecolor{currentfill}{rgb}{0.150000,0.150000,0.150000}%
\pgfsetfillcolor{currentfill}%
\pgfsetlinewidth{1.003750pt}%
\definecolor{currentstroke}{rgb}{0.150000,0.150000,0.150000}%
\pgfsetstrokecolor{currentstroke}%
\pgfsetdash{}{0pt}%
\pgfsys@defobject{currentmarker}{\pgfqpoint{-0.066667in}{0.000000in}}{\pgfqpoint{0.000000in}{0.000000in}}{%
\pgfpathmoveto{\pgfqpoint{0.000000in}{0.000000in}}%
\pgfpathlineto{\pgfqpoint{-0.066667in}{0.000000in}}%
\pgfusepath{stroke,fill}%
}%
\begin{pgfscope}%
\pgfsys@transformshift{2.170064in}{4.691069in}%
\pgfsys@useobject{currentmarker}{}%
\end{pgfscope}%
\end{pgfscope}%
\begin{pgfscope}%
\pgfpathrectangle{\pgfqpoint{2.170064in}{4.083121in}}{\pgfqpoint{1.223103in}{0.607948in}}%
\pgfusepath{clip}%
\pgfsetroundcap%
\pgfsetroundjoin%
\pgfsetlinewidth{1.204500pt}%
\definecolor{currentstroke}{rgb}{0.000000,0.501961,0.000000}%
\pgfsetstrokecolor{currentstroke}%
\pgfsetdash{}{0pt}%
\pgfpathmoveto{\pgfqpoint{2.170064in}{4.492698in}}%
\pgfpathlineto{\pgfqpoint{2.410774in}{4.495565in}}%
\pgfpathlineto{\pgfqpoint{2.522305in}{4.498346in}}%
\pgfpathlineto{\pgfqpoint{2.595701in}{4.501025in}}%
\pgfpathlineto{\pgfqpoint{2.650497in}{4.503586in}}%
\pgfpathlineto{\pgfqpoint{2.694239in}{4.506010in}}%
\pgfpathlineto{\pgfqpoint{2.730647in}{4.508282in}}%
\pgfpathlineto{\pgfqpoint{2.761831in}{4.510381in}}%
\pgfpathlineto{\pgfqpoint{2.789104in}{4.512290in}}%
\pgfpathlineto{\pgfqpoint{2.813338in}{4.513988in}}%
\pgfpathlineto{\pgfqpoint{2.835143in}{4.515455in}}%
\pgfpathlineto{\pgfqpoint{2.854962in}{4.516671in}}%
\pgfpathlineto{\pgfqpoint{2.873127in}{4.517613in}}%
\pgfpathlineto{\pgfqpoint{2.889892in}{4.518258in}}%
\pgfpathlineto{\pgfqpoint{2.905459in}{4.518582in}}%
\pgfpathlineto{\pgfqpoint{2.919986in}{4.518562in}}%
\pgfpathlineto{\pgfqpoint{2.933605in}{4.518171in}}%
\pgfpathlineto{\pgfqpoint{2.946422in}{4.517383in}}%
\pgfpathlineto{\pgfqpoint{2.958526in}{4.516169in}}%
\pgfpathlineto{\pgfqpoint{2.969993in}{4.514502in}}%
\pgfpathlineto{\pgfqpoint{2.980886in}{4.512350in}}%
\pgfpathlineto{\pgfqpoint{2.991260in}{4.509682in}}%
\pgfpathlineto{\pgfqpoint{3.001162in}{4.506466in}}%
\pgfpathlineto{\pgfqpoint{3.010633in}{4.502669in}}%
\pgfpathlineto{\pgfqpoint{3.019710in}{4.498253in}}%
\pgfpathlineto{\pgfqpoint{3.028423in}{4.493184in}}%
\pgfpathlineto{\pgfqpoint{3.036801in}{4.487422in}}%
\pgfpathlineto{\pgfqpoint{3.044869in}{4.480928in}}%
\pgfpathlineto{\pgfqpoint{3.052648in}{4.473661in}}%
\pgfpathlineto{\pgfqpoint{3.060159in}{4.465578in}}%
\pgfpathlineto{\pgfqpoint{3.067420in}{4.456634in}}%
\pgfpathlineto{\pgfqpoint{3.074446in}{4.446784in}}%
\pgfpathlineto{\pgfqpoint{3.081253in}{4.435979in}}%
\pgfpathlineto{\pgfqpoint{3.087853in}{4.424170in}}%
\pgfpathlineto{\pgfqpoint{3.094259in}{4.411304in}}%
\pgfpathlineto{\pgfqpoint{3.100482in}{4.397330in}}%
\pgfpathlineto{\pgfqpoint{3.106532in}{4.382191in}}%
\pgfpathlineto{\pgfqpoint{3.112419in}{4.365830in}}%
\pgfpathlineto{\pgfqpoint{3.118150in}{4.348188in}}%
\pgfpathlineto{\pgfqpoint{3.123735in}{4.329204in}}%
\pgfpathlineto{\pgfqpoint{3.129180in}{4.308814in}}%
\pgfpathlineto{\pgfqpoint{3.134492in}{4.286953in}}%
\pgfpathlineto{\pgfqpoint{3.139677in}{4.263553in}}%
\pgfpathlineto{\pgfqpoint{3.144742in}{4.238543in}}%
\pgfpathlineto{\pgfqpoint{3.149692in}{4.211852in}}%
\pgfpathlineto{\pgfqpoint{3.154532in}{4.183404in}}%
\pgfpathlineto{\pgfqpoint{3.159267in}{4.153124in}}%
\pgfpathlineto{\pgfqpoint{3.163901in}{4.120931in}}%
\pgfpathlineto{\pgfqpoint{3.168438in}{4.086743in}}%
\pgfpathlineto{\pgfqpoint{3.169290in}{4.079788in}}%
\pgfusepath{stroke}%
\end{pgfscope}%
\begin{pgfscope}%
\pgfsetrectcap%
\pgfsetmiterjoin%
\pgfsetlinewidth{1.003750pt}%
\definecolor{currentstroke}{rgb}{0.150000,0.150000,0.150000}%
\pgfsetstrokecolor{currentstroke}%
\pgfsetdash{}{0pt}%
\pgfpathmoveto{\pgfqpoint{2.170064in}{4.083121in}}%
\pgfpathlineto{\pgfqpoint{2.170064in}{4.691069in}}%
\pgfusepath{stroke}%
\end{pgfscope}%
\begin{pgfscope}%
\pgfsetrectcap%
\pgfsetmiterjoin%
\pgfsetlinewidth{1.003750pt}%
\definecolor{currentstroke}{rgb}{0.150000,0.150000,0.150000}%
\pgfsetstrokecolor{currentstroke}%
\pgfsetdash{}{0pt}%
\pgfpathmoveto{\pgfqpoint{2.170064in}{4.083121in}}%
\pgfpathlineto{\pgfqpoint{3.393168in}{4.083121in}}%
\pgfusepath{stroke}%
\end{pgfscope}%
\begin{pgfscope}%
\pgfpathrectangle{\pgfqpoint{2.170064in}{4.083121in}}{\pgfqpoint{1.223103in}{0.607948in}}%
\pgfusepath{clip}%
\pgfsetbuttcap%
\pgfsetroundjoin%
\definecolor{currentfill}{rgb}{0.000000,0.000000,0.000000}%
\pgfsetfillcolor{currentfill}%
\pgfsetlinewidth{1.003750pt}%
\definecolor{currentstroke}{rgb}{0.000000,0.000000,0.000000}%
\pgfsetstrokecolor{currentstroke}%
\pgfsetdash{}{0pt}%
\pgfsys@defobject{currentmarker}{\pgfqpoint{-0.013889in}{-0.013889in}}{\pgfqpoint{0.013889in}{0.013889in}}{%
\pgfpathmoveto{\pgfqpoint{0.000000in}{-0.013889in}}%
\pgfpathcurveto{\pgfqpoint{0.003683in}{-0.013889in}}{\pgfqpoint{0.007216in}{-0.012425in}}{\pgfqpoint{0.009821in}{-0.009821in}}%
\pgfpathcurveto{\pgfqpoint{0.012425in}{-0.007216in}}{\pgfqpoint{0.013889in}{-0.003683in}}{\pgfqpoint{0.013889in}{0.000000in}}%
\pgfpathcurveto{\pgfqpoint{0.013889in}{0.003683in}}{\pgfqpoint{0.012425in}{0.007216in}}{\pgfqpoint{0.009821in}{0.009821in}}%
\pgfpathcurveto{\pgfqpoint{0.007216in}{0.012425in}}{\pgfqpoint{0.003683in}{0.013889in}}{\pgfqpoint{0.000000in}{0.013889in}}%
\pgfpathcurveto{\pgfqpoint{-0.003683in}{0.013889in}}{\pgfqpoint{-0.007216in}{0.012425in}}{\pgfqpoint{-0.009821in}{0.009821in}}%
\pgfpathcurveto{\pgfqpoint{-0.012425in}{0.007216in}}{\pgfqpoint{-0.013889in}{0.003683in}}{\pgfqpoint{-0.013889in}{0.000000in}}%
\pgfpathcurveto{\pgfqpoint{-0.013889in}{-0.003683in}}{\pgfqpoint{-0.012425in}{-0.007216in}}{\pgfqpoint{-0.009821in}{-0.009821in}}%
\pgfpathcurveto{\pgfqpoint{-0.007216in}{-0.012425in}}{\pgfqpoint{-0.003683in}{-0.013889in}}{\pgfqpoint{0.000000in}{-0.013889in}}%
\pgfpathclose%
\pgfusepath{stroke,fill}%
}%
\begin{pgfscope}%
\pgfsys@transformshift{3.172883in}{4.048574in}%
\pgfsys@useobject{currentmarker}{}%
\end{pgfscope}%
\begin{pgfscope}%
\pgfsys@transformshift{3.084589in}{4.437178in}%
\pgfsys@useobject{currentmarker}{}%
\end{pgfscope}%
\begin{pgfscope}%
\pgfsys@transformshift{3.021943in}{4.495755in}%
\pgfsys@useobject{currentmarker}{}%
\end{pgfscope}%
\begin{pgfscope}%
\pgfsys@transformshift{2.973352in}{4.510577in}%
\pgfsys@useobject{currentmarker}{}%
\end{pgfscope}%
\begin{pgfscope}%
\pgfsys@transformshift{2.933650in}{4.515124in}%
\pgfsys@useobject{currentmarker}{}%
\end{pgfscope}%
\begin{pgfscope}%
\pgfsys@transformshift{2.822413in}{4.516293in}%
\pgfsys@useobject{currentmarker}{}%
\end{pgfscope}%
\begin{pgfscope}%
\pgfsys@transformshift{2.826812in}{4.516362in}%
\pgfsys@useobject{currentmarker}{}%
\end{pgfscope}%
\begin{pgfscope}%
\pgfsys@transformshift{2.831302in}{4.516430in}%
\pgfsys@useobject{currentmarker}{}%
\end{pgfscope}%
\begin{pgfscope}%
\pgfsys@transformshift{2.835887in}{4.516494in}%
\pgfsys@useobject{currentmarker}{}%
\end{pgfscope}%
\begin{pgfscope}%
\pgfsys@transformshift{2.900082in}{4.516509in}%
\pgfsys@useobject{currentmarker}{}%
\end{pgfscope}%
\begin{pgfscope}%
\pgfsys@transformshift{2.840570in}{4.516555in}%
\pgfsys@useobject{currentmarker}{}%
\end{pgfscope}%
\begin{pgfscope}%
\pgfsys@transformshift{2.845356in}{4.516611in}%
\pgfsys@useobject{currentmarker}{}%
\end{pgfscope}%
\begin{pgfscope}%
\pgfsys@transformshift{2.850250in}{4.516661in}%
\pgfsys@useobject{currentmarker}{}%
\end{pgfscope}%
\begin{pgfscope}%
\pgfsys@transformshift{2.855256in}{4.516704in}%
\pgfsys@useobject{currentmarker}{}%
\end{pgfscope}%
\begin{pgfscope}%
\pgfsys@transformshift{2.860380in}{4.516738in}%
\pgfsys@useobject{currentmarker}{}%
\end{pgfscope}%
\begin{pgfscope}%
\pgfsys@transformshift{2.865627in}{4.516762in}%
\pgfsys@useobject{currentmarker}{}%
\end{pgfscope}%
\begin{pgfscope}%
\pgfsys@transformshift{2.871004in}{4.516772in}%
\pgfsys@useobject{currentmarker}{}%
\end{pgfscope}%
\end{pgfscope}%
\begin{pgfscope}%
\pgfsetbuttcap%
\pgfsetmiterjoin%
\definecolor{currentfill}{rgb}{1.000000,1.000000,1.000000}%
\pgfsetfillcolor{currentfill}%
\pgfsetlinewidth{0.000000pt}%
\definecolor{currentstroke}{rgb}{0.000000,0.000000,0.000000}%
\pgfsetstrokecolor{currentstroke}%
\pgfsetstrokeopacity{0.000000}%
\pgfsetdash{}{0pt}%
\pgfpathmoveto{\pgfqpoint{3.637789in}{4.083121in}}%
\pgfpathlineto{\pgfqpoint{4.860892in}{4.083121in}}%
\pgfpathlineto{\pgfqpoint{4.860892in}{4.691069in}}%
\pgfpathlineto{\pgfqpoint{3.637789in}{4.691069in}}%
\pgfpathclose%
\pgfusepath{fill}%
\end{pgfscope}%
\begin{pgfscope}%
\pgfpathrectangle{\pgfqpoint{3.637789in}{4.083121in}}{\pgfqpoint{1.223103in}{0.607948in}}%
\pgfusepath{clip}%
\pgfsetbuttcap%
\pgfsetmiterjoin%
\definecolor{currentfill}{rgb}{0.000000,0.000000,1.000000}%
\pgfsetfillcolor{currentfill}%
\pgfsetfillopacity{0.100000}%
\pgfsetlinewidth{0.803000pt}%
\definecolor{currentstroke}{rgb}{0.000000,0.000000,1.000000}%
\pgfsetstrokecolor{currentstroke}%
\pgfsetstrokeopacity{0.100000}%
\pgfsetdash{}{0pt}%
\pgfpathmoveto{\pgfqpoint{3.637789in}{4.494856in}}%
\pgfpathlineto{\pgfqpoint{3.637789in}{4.518836in}}%
\pgfpathlineto{\pgfqpoint{4.860892in}{4.518836in}}%
\pgfpathlineto{\pgfqpoint{4.860892in}{4.494856in}}%
\pgfpathclose%
\pgfusepath{stroke,fill}%
\end{pgfscope}%
\begin{pgfscope}%
\pgfpathrectangle{\pgfqpoint{3.637789in}{4.083121in}}{\pgfqpoint{1.223103in}{0.607948in}}%
\pgfusepath{clip}%
\pgfsetbuttcap%
\pgfsetroundjoin%
\definecolor{currentfill}{rgb}{0.000000,0.501961,0.000000}%
\pgfsetfillcolor{currentfill}%
\pgfsetfillopacity{0.500000}%
\pgfsetlinewidth{0.803000pt}%
\definecolor{currentstroke}{rgb}{0.000000,0.501961,0.000000}%
\pgfsetstrokecolor{currentstroke}%
\pgfsetstrokeopacity{0.500000}%
\pgfsetdash{}{0pt}%
\pgfpathmoveto{\pgfqpoint{3.637789in}{4.518773in}}%
\pgfpathlineto{\pgfqpoint{3.637789in}{4.496520in}}%
\pgfpathlineto{\pgfqpoint{3.878498in}{4.499759in}}%
\pgfpathlineto{\pgfqpoint{3.990029in}{4.502841in}}%
\pgfpathlineto{\pgfqpoint{4.063425in}{4.505775in}}%
\pgfpathlineto{\pgfqpoint{4.118221in}{4.508568in}}%
\pgfpathlineto{\pgfqpoint{4.161963in}{4.511227in}}%
\pgfpathlineto{\pgfqpoint{4.198371in}{4.513758in}}%
\pgfpathlineto{\pgfqpoint{4.229555in}{4.516166in}}%
\pgfpathlineto{\pgfqpoint{4.256828in}{4.518456in}}%
\pgfpathlineto{\pgfqpoint{4.281062in}{4.520633in}}%
\pgfpathlineto{\pgfqpoint{4.302867in}{4.522698in}}%
\pgfpathlineto{\pgfqpoint{4.322686in}{4.524520in}}%
\pgfpathlineto{\pgfqpoint{4.340851in}{4.525532in}}%
\pgfpathlineto{\pgfqpoint{4.357617in}{4.526577in}}%
\pgfpathlineto{\pgfqpoint{4.373183in}{4.527665in}}%
\pgfpathlineto{\pgfqpoint{4.387711in}{4.528787in}}%
\pgfpathlineto{\pgfqpoint{4.401329in}{4.529939in}}%
\pgfpathlineto{\pgfqpoint{4.414146in}{4.531112in}}%
\pgfpathlineto{\pgfqpoint{4.426250in}{4.532301in}}%
\pgfpathlineto{\pgfqpoint{4.437717in}{4.533496in}}%
\pgfpathlineto{\pgfqpoint{4.448610in}{4.534692in}}%
\pgfpathlineto{\pgfqpoint{4.458984in}{4.535877in}}%
\pgfpathlineto{\pgfqpoint{4.468886in}{4.537044in}}%
\pgfpathlineto{\pgfqpoint{4.478357in}{4.538179in}}%
\pgfpathlineto{\pgfqpoint{4.487434in}{4.539267in}}%
\pgfpathlineto{\pgfqpoint{4.496147in}{4.540279in}}%
\pgfpathlineto{\pgfqpoint{4.504525in}{4.541114in}}%
\pgfpathlineto{\pgfqpoint{4.512593in}{4.541559in}}%
\pgfpathlineto{\pgfqpoint{4.520372in}{4.541704in}}%
\pgfpathlineto{\pgfqpoint{4.527883in}{4.541668in}}%
\pgfpathlineto{\pgfqpoint{4.535144in}{4.541474in}}%
\pgfpathlineto{\pgfqpoint{4.542170in}{4.541124in}}%
\pgfpathlineto{\pgfqpoint{4.548977in}{4.540616in}}%
\pgfpathlineto{\pgfqpoint{4.555577in}{4.539943in}}%
\pgfpathlineto{\pgfqpoint{4.561983in}{4.539101in}}%
\pgfpathlineto{\pgfqpoint{4.568206in}{4.538084in}}%
\pgfpathlineto{\pgfqpoint{4.574256in}{4.536884in}}%
\pgfpathlineto{\pgfqpoint{4.580143in}{4.535496in}}%
\pgfpathlineto{\pgfqpoint{4.585874in}{4.533913in}}%
\pgfpathlineto{\pgfqpoint{4.591459in}{4.532128in}}%
\pgfpathlineto{\pgfqpoint{4.596904in}{4.530135in}}%
\pgfpathlineto{\pgfqpoint{4.602216in}{4.527926in}}%
\pgfpathlineto{\pgfqpoint{4.607401in}{4.525495in}}%
\pgfpathlineto{\pgfqpoint{4.612466in}{4.522835in}}%
\pgfpathlineto{\pgfqpoint{4.617416in}{4.519940in}}%
\pgfpathlineto{\pgfqpoint{4.622256in}{4.516801in}}%
\pgfpathlineto{\pgfqpoint{4.626991in}{4.513411in}}%
\pgfpathlineto{\pgfqpoint{4.631625in}{4.509759in}}%
\pgfpathlineto{\pgfqpoint{4.636162in}{4.505781in}}%
\pgfpathlineto{\pgfqpoint{4.640607in}{4.499859in}}%
\pgfpathlineto{\pgfqpoint{4.640607in}{4.501809in}}%
\pgfpathlineto{\pgfqpoint{4.640607in}{4.501809in}}%
\pgfpathlineto{\pgfqpoint{4.636162in}{4.507074in}}%
\pgfpathlineto{\pgfqpoint{4.631625in}{4.513239in}}%
\pgfpathlineto{\pgfqpoint{4.626991in}{4.518748in}}%
\pgfpathlineto{\pgfqpoint{4.622256in}{4.523597in}}%
\pgfpathlineto{\pgfqpoint{4.617416in}{4.527831in}}%
\pgfpathlineto{\pgfqpoint{4.612466in}{4.531496in}}%
\pgfpathlineto{\pgfqpoint{4.607401in}{4.534636in}}%
\pgfpathlineto{\pgfqpoint{4.602216in}{4.537293in}}%
\pgfpathlineto{\pgfqpoint{4.596904in}{4.539507in}}%
\pgfpathlineto{\pgfqpoint{4.591459in}{4.541316in}}%
\pgfpathlineto{\pgfqpoint{4.585874in}{4.542755in}}%
\pgfpathlineto{\pgfqpoint{4.580143in}{4.543859in}}%
\pgfpathlineto{\pgfqpoint{4.574256in}{4.544657in}}%
\pgfpathlineto{\pgfqpoint{4.568206in}{4.545180in}}%
\pgfpathlineto{\pgfqpoint{4.561983in}{4.545455in}}%
\pgfpathlineto{\pgfqpoint{4.555577in}{4.545507in}}%
\pgfpathlineto{\pgfqpoint{4.548977in}{4.545362in}}%
\pgfpathlineto{\pgfqpoint{4.542170in}{4.545041in}}%
\pgfpathlineto{\pgfqpoint{4.535144in}{4.544567in}}%
\pgfpathlineto{\pgfqpoint{4.527883in}{4.543962in}}%
\pgfpathlineto{\pgfqpoint{4.520372in}{4.543250in}}%
\pgfpathlineto{\pgfqpoint{4.512593in}{4.542477in}}%
\pgfpathlineto{\pgfqpoint{4.504525in}{4.541784in}}%
\pgfpathlineto{\pgfqpoint{4.496147in}{4.541276in}}%
\pgfpathlineto{\pgfqpoint{4.487434in}{4.540761in}}%
\pgfpathlineto{\pgfqpoint{4.478357in}{4.540154in}}%
\pgfpathlineto{\pgfqpoint{4.468886in}{4.539440in}}%
\pgfpathlineto{\pgfqpoint{4.458984in}{4.538616in}}%
\pgfpathlineto{\pgfqpoint{4.448610in}{4.537685in}}%
\pgfpathlineto{\pgfqpoint{4.437717in}{4.536647in}}%
\pgfpathlineto{\pgfqpoint{4.426250in}{4.535505in}}%
\pgfpathlineto{\pgfqpoint{4.414146in}{4.534260in}}%
\pgfpathlineto{\pgfqpoint{4.401329in}{4.532913in}}%
\pgfpathlineto{\pgfqpoint{4.387711in}{4.531465in}}%
\pgfpathlineto{\pgfqpoint{4.373183in}{4.529917in}}%
\pgfpathlineto{\pgfqpoint{4.357617in}{4.528267in}}%
\pgfpathlineto{\pgfqpoint{4.340851in}{4.526517in}}%
\pgfpathlineto{\pgfqpoint{4.322686in}{4.524677in}}%
\pgfpathlineto{\pgfqpoint{4.302867in}{4.523599in}}%
\pgfpathlineto{\pgfqpoint{4.281062in}{4.522718in}}%
\pgfpathlineto{\pgfqpoint{4.256828in}{4.521905in}}%
\pgfpathlineto{\pgfqpoint{4.229555in}{4.521168in}}%
\pgfpathlineto{\pgfqpoint{4.198371in}{4.520513in}}%
\pgfpathlineto{\pgfqpoint{4.161963in}{4.519948in}}%
\pgfpathlineto{\pgfqpoint{4.118221in}{4.519481in}}%
\pgfpathlineto{\pgfqpoint{4.063425in}{4.519120in}}%
\pgfpathlineto{\pgfqpoint{3.990029in}{4.518875in}}%
\pgfpathlineto{\pgfqpoint{3.878498in}{4.518756in}}%
\pgfpathlineto{\pgfqpoint{3.637789in}{4.518773in}}%
\pgfpathclose%
\pgfusepath{stroke,fill}%
\end{pgfscope}%
\begin{pgfscope}%
\pgfpathrectangle{\pgfqpoint{3.637789in}{4.083121in}}{\pgfqpoint{1.223103in}{0.607948in}}%
\pgfusepath{clip}%
\pgfsetroundcap%
\pgfsetroundjoin%
\pgfsetlinewidth{0.501875pt}%
\definecolor{currentstroke}{rgb}{0.000000,0.000000,1.000000}%
\pgfsetstrokecolor{currentstroke}%
\pgfsetstrokeopacity{0.800000}%
\pgfsetdash{}{0pt}%
\pgfpathmoveto{\pgfqpoint{3.637789in}{4.506846in}}%
\pgfpathlineto{\pgfqpoint{4.860892in}{4.506846in}}%
\pgfusepath{stroke}%
\end{pgfscope}%
\begin{pgfscope}%
\pgfpathrectangle{\pgfqpoint{3.637789in}{4.083121in}}{\pgfqpoint{1.223103in}{0.607948in}}%
\pgfusepath{clip}%
\pgfsetbuttcap%
\pgfsetroundjoin%
\pgfsetlinewidth{1.003750pt}%
\definecolor{currentstroke}{rgb}{0.000000,0.000000,0.000000}%
\pgfsetstrokecolor{currentstroke}%
\pgfsetdash{{3.700000pt}{1.600000pt}}{0.000000pt}%
\pgfpathmoveto{\pgfqpoint{3.637789in}{4.509594in}}%
\pgfpathlineto{\pgfqpoint{4.860892in}{4.509594in}}%
\pgfusepath{stroke}%
\end{pgfscope}%
\begin{pgfscope}%
\pgfsetroundcap%
\pgfsetroundjoin%
\pgfsetlinewidth{0.501875pt}%
\definecolor{currentstroke}{rgb}{0.000000,0.000000,1.000000}%
\pgfsetstrokecolor{currentstroke}%
\pgfsetstrokeopacity{0.800000}%
\pgfsetdash{}{0pt}%
\pgfpathmoveto{\pgfqpoint{4.451637in}{4.624183in}}%
\pgfpathquadraticcurveto{\pgfqpoint{4.382119in}{4.573675in}}{\pgfqpoint{4.312601in}{4.523167in}}%
\pgfusepath{stroke}%
\end{pgfscope}%
\begin{pgfscope}%
\pgfsetfillopacity{0.800000}%
\pgfsetstrokeopacity{0.800000}%
\definecolor{textcolor}{rgb}{0.000000,0.000000,1.000000}%
\pgfsetstrokecolor{textcolor}%
\pgfsetfillcolor{textcolor}%
\pgftext[x=4.378430in,y=4.689230in,left,base]{\color{textcolor}\sffamily\fontsize{5.647059}{6.776471}\selectfont 11.697(20)}%
\end{pgfscope}%
\begin{pgfscope}%
\pgfsetbuttcap%
\pgfsetroundjoin%
\definecolor{currentfill}{rgb}{0.150000,0.150000,0.150000}%
\pgfsetfillcolor{currentfill}%
\pgfsetlinewidth{1.003750pt}%
\definecolor{currentstroke}{rgb}{0.150000,0.150000,0.150000}%
\pgfsetstrokecolor{currentstroke}%
\pgfsetdash{}{0pt}%
\pgfsys@defobject{currentmarker}{\pgfqpoint{0.000000in}{-0.066667in}}{\pgfqpoint{0.000000in}{0.000000in}}{%
\pgfpathmoveto{\pgfqpoint{0.000000in}{0.000000in}}%
\pgfpathlineto{\pgfqpoint{0.000000in}{-0.066667in}}%
\pgfusepath{stroke,fill}%
}%
\begin{pgfscope}%
\pgfsys@transformshift{3.637789in}{4.083121in}%
\pgfsys@useobject{currentmarker}{}%
\end{pgfscope}%
\end{pgfscope}%
\begin{pgfscope}%
\pgfsetbuttcap%
\pgfsetroundjoin%
\definecolor{currentfill}{rgb}{0.150000,0.150000,0.150000}%
\pgfsetfillcolor{currentfill}%
\pgfsetlinewidth{1.003750pt}%
\definecolor{currentstroke}{rgb}{0.150000,0.150000,0.150000}%
\pgfsetstrokecolor{currentstroke}%
\pgfsetdash{}{0pt}%
\pgfsys@defobject{currentmarker}{\pgfqpoint{0.000000in}{-0.066667in}}{\pgfqpoint{0.000000in}{0.000000in}}{%
\pgfpathmoveto{\pgfqpoint{0.000000in}{0.000000in}}%
\pgfpathlineto{\pgfqpoint{0.000000in}{-0.066667in}}%
\pgfusepath{stroke,fill}%
}%
\begin{pgfscope}%
\pgfsys@transformshift{4.139198in}{4.083121in}%
\pgfsys@useobject{currentmarker}{}%
\end{pgfscope}%
\end{pgfscope}%
\begin{pgfscope}%
\pgfsetbuttcap%
\pgfsetroundjoin%
\definecolor{currentfill}{rgb}{0.150000,0.150000,0.150000}%
\pgfsetfillcolor{currentfill}%
\pgfsetlinewidth{1.003750pt}%
\definecolor{currentstroke}{rgb}{0.150000,0.150000,0.150000}%
\pgfsetstrokecolor{currentstroke}%
\pgfsetdash{}{0pt}%
\pgfsys@defobject{currentmarker}{\pgfqpoint{0.000000in}{-0.066667in}}{\pgfqpoint{0.000000in}{0.000000in}}{%
\pgfpathmoveto{\pgfqpoint{0.000000in}{0.000000in}}%
\pgfpathlineto{\pgfqpoint{0.000000in}{-0.066667in}}%
\pgfusepath{stroke,fill}%
}%
\begin{pgfscope}%
\pgfsys@transformshift{4.640607in}{4.083121in}%
\pgfsys@useobject{currentmarker}{}%
\end{pgfscope}%
\end{pgfscope}%
\begin{pgfscope}%
\pgfsetbuttcap%
\pgfsetroundjoin%
\definecolor{currentfill}{rgb}{0.150000,0.150000,0.150000}%
\pgfsetfillcolor{currentfill}%
\pgfsetlinewidth{0.803000pt}%
\definecolor{currentstroke}{rgb}{0.150000,0.150000,0.150000}%
\pgfsetstrokecolor{currentstroke}%
\pgfsetdash{}{0pt}%
\pgfsys@defobject{currentmarker}{\pgfqpoint{0.000000in}{-0.044444in}}{\pgfqpoint{0.000000in}{0.000000in}}{%
\pgfpathmoveto{\pgfqpoint{0.000000in}{0.000000in}}%
\pgfpathlineto{\pgfqpoint{0.000000in}{-0.044444in}}%
\pgfusepath{stroke,fill}%
}%
\begin{pgfscope}%
\pgfsys@transformshift{3.788728in}{4.083121in}%
\pgfsys@useobject{currentmarker}{}%
\end{pgfscope}%
\end{pgfscope}%
\begin{pgfscope}%
\pgfsetbuttcap%
\pgfsetroundjoin%
\definecolor{currentfill}{rgb}{0.150000,0.150000,0.150000}%
\pgfsetfillcolor{currentfill}%
\pgfsetlinewidth{0.803000pt}%
\definecolor{currentstroke}{rgb}{0.150000,0.150000,0.150000}%
\pgfsetstrokecolor{currentstroke}%
\pgfsetdash{}{0pt}%
\pgfsys@defobject{currentmarker}{\pgfqpoint{0.000000in}{-0.044444in}}{\pgfqpoint{0.000000in}{0.000000in}}{%
\pgfpathmoveto{\pgfqpoint{0.000000in}{0.000000in}}%
\pgfpathlineto{\pgfqpoint{0.000000in}{-0.044444in}}%
\pgfusepath{stroke,fill}%
}%
\begin{pgfscope}%
\pgfsys@transformshift{3.877021in}{4.083121in}%
\pgfsys@useobject{currentmarker}{}%
\end{pgfscope}%
\end{pgfscope}%
\begin{pgfscope}%
\pgfsetbuttcap%
\pgfsetroundjoin%
\definecolor{currentfill}{rgb}{0.150000,0.150000,0.150000}%
\pgfsetfillcolor{currentfill}%
\pgfsetlinewidth{0.803000pt}%
\definecolor{currentstroke}{rgb}{0.150000,0.150000,0.150000}%
\pgfsetstrokecolor{currentstroke}%
\pgfsetdash{}{0pt}%
\pgfsys@defobject{currentmarker}{\pgfqpoint{0.000000in}{-0.044444in}}{\pgfqpoint{0.000000in}{0.000000in}}{%
\pgfpathmoveto{\pgfqpoint{0.000000in}{0.000000in}}%
\pgfpathlineto{\pgfqpoint{0.000000in}{-0.044444in}}%
\pgfusepath{stroke,fill}%
}%
\begin{pgfscope}%
\pgfsys@transformshift{3.939667in}{4.083121in}%
\pgfsys@useobject{currentmarker}{}%
\end{pgfscope}%
\end{pgfscope}%
\begin{pgfscope}%
\pgfsetbuttcap%
\pgfsetroundjoin%
\definecolor{currentfill}{rgb}{0.150000,0.150000,0.150000}%
\pgfsetfillcolor{currentfill}%
\pgfsetlinewidth{0.803000pt}%
\definecolor{currentstroke}{rgb}{0.150000,0.150000,0.150000}%
\pgfsetstrokecolor{currentstroke}%
\pgfsetdash{}{0pt}%
\pgfsys@defobject{currentmarker}{\pgfqpoint{0.000000in}{-0.044444in}}{\pgfqpoint{0.000000in}{0.000000in}}{%
\pgfpathmoveto{\pgfqpoint{0.000000in}{0.000000in}}%
\pgfpathlineto{\pgfqpoint{0.000000in}{-0.044444in}}%
\pgfusepath{stroke,fill}%
}%
\begin{pgfscope}%
\pgfsys@transformshift{3.988258in}{4.083121in}%
\pgfsys@useobject{currentmarker}{}%
\end{pgfscope}%
\end{pgfscope}%
\begin{pgfscope}%
\pgfsetbuttcap%
\pgfsetroundjoin%
\definecolor{currentfill}{rgb}{0.150000,0.150000,0.150000}%
\pgfsetfillcolor{currentfill}%
\pgfsetlinewidth{0.803000pt}%
\definecolor{currentstroke}{rgb}{0.150000,0.150000,0.150000}%
\pgfsetstrokecolor{currentstroke}%
\pgfsetdash{}{0pt}%
\pgfsys@defobject{currentmarker}{\pgfqpoint{0.000000in}{-0.044444in}}{\pgfqpoint{0.000000in}{0.000000in}}{%
\pgfpathmoveto{\pgfqpoint{0.000000in}{0.000000in}}%
\pgfpathlineto{\pgfqpoint{0.000000in}{-0.044444in}}%
\pgfusepath{stroke,fill}%
}%
\begin{pgfscope}%
\pgfsys@transformshift{4.027961in}{4.083121in}%
\pgfsys@useobject{currentmarker}{}%
\end{pgfscope}%
\end{pgfscope}%
\begin{pgfscope}%
\pgfsetbuttcap%
\pgfsetroundjoin%
\definecolor{currentfill}{rgb}{0.150000,0.150000,0.150000}%
\pgfsetfillcolor{currentfill}%
\pgfsetlinewidth{0.803000pt}%
\definecolor{currentstroke}{rgb}{0.150000,0.150000,0.150000}%
\pgfsetstrokecolor{currentstroke}%
\pgfsetdash{}{0pt}%
\pgfsys@defobject{currentmarker}{\pgfqpoint{0.000000in}{-0.044444in}}{\pgfqpoint{0.000000in}{0.000000in}}{%
\pgfpathmoveto{\pgfqpoint{0.000000in}{0.000000in}}%
\pgfpathlineto{\pgfqpoint{0.000000in}{-0.044444in}}%
\pgfusepath{stroke,fill}%
}%
\begin{pgfscope}%
\pgfsys@transformshift{4.061528in}{4.083121in}%
\pgfsys@useobject{currentmarker}{}%
\end{pgfscope}%
\end{pgfscope}%
\begin{pgfscope}%
\pgfsetbuttcap%
\pgfsetroundjoin%
\definecolor{currentfill}{rgb}{0.150000,0.150000,0.150000}%
\pgfsetfillcolor{currentfill}%
\pgfsetlinewidth{0.803000pt}%
\definecolor{currentstroke}{rgb}{0.150000,0.150000,0.150000}%
\pgfsetstrokecolor{currentstroke}%
\pgfsetdash{}{0pt}%
\pgfsys@defobject{currentmarker}{\pgfqpoint{0.000000in}{-0.044444in}}{\pgfqpoint{0.000000in}{0.000000in}}{%
\pgfpathmoveto{\pgfqpoint{0.000000in}{0.000000in}}%
\pgfpathlineto{\pgfqpoint{0.000000in}{-0.044444in}}%
\pgfusepath{stroke,fill}%
}%
\begin{pgfscope}%
\pgfsys@transformshift{4.090606in}{4.083121in}%
\pgfsys@useobject{currentmarker}{}%
\end{pgfscope}%
\end{pgfscope}%
\begin{pgfscope}%
\pgfsetbuttcap%
\pgfsetroundjoin%
\definecolor{currentfill}{rgb}{0.150000,0.150000,0.150000}%
\pgfsetfillcolor{currentfill}%
\pgfsetlinewidth{0.803000pt}%
\definecolor{currentstroke}{rgb}{0.150000,0.150000,0.150000}%
\pgfsetstrokecolor{currentstroke}%
\pgfsetdash{}{0pt}%
\pgfsys@defobject{currentmarker}{\pgfqpoint{0.000000in}{-0.044444in}}{\pgfqpoint{0.000000in}{0.000000in}}{%
\pgfpathmoveto{\pgfqpoint{0.000000in}{0.000000in}}%
\pgfpathlineto{\pgfqpoint{0.000000in}{-0.044444in}}%
\pgfusepath{stroke,fill}%
}%
\begin{pgfscope}%
\pgfsys@transformshift{4.116254in}{4.083121in}%
\pgfsys@useobject{currentmarker}{}%
\end{pgfscope}%
\end{pgfscope}%
\begin{pgfscope}%
\pgfsetbuttcap%
\pgfsetroundjoin%
\definecolor{currentfill}{rgb}{0.150000,0.150000,0.150000}%
\pgfsetfillcolor{currentfill}%
\pgfsetlinewidth{0.803000pt}%
\definecolor{currentstroke}{rgb}{0.150000,0.150000,0.150000}%
\pgfsetstrokecolor{currentstroke}%
\pgfsetdash{}{0pt}%
\pgfsys@defobject{currentmarker}{\pgfqpoint{0.000000in}{-0.044444in}}{\pgfqpoint{0.000000in}{0.000000in}}{%
\pgfpathmoveto{\pgfqpoint{0.000000in}{0.000000in}}%
\pgfpathlineto{\pgfqpoint{0.000000in}{-0.044444in}}%
\pgfusepath{stroke,fill}%
}%
\begin{pgfscope}%
\pgfsys@transformshift{4.290137in}{4.083121in}%
\pgfsys@useobject{currentmarker}{}%
\end{pgfscope}%
\end{pgfscope}%
\begin{pgfscope}%
\pgfsetbuttcap%
\pgfsetroundjoin%
\definecolor{currentfill}{rgb}{0.150000,0.150000,0.150000}%
\pgfsetfillcolor{currentfill}%
\pgfsetlinewidth{0.803000pt}%
\definecolor{currentstroke}{rgb}{0.150000,0.150000,0.150000}%
\pgfsetstrokecolor{currentstroke}%
\pgfsetdash{}{0pt}%
\pgfsys@defobject{currentmarker}{\pgfqpoint{0.000000in}{-0.044444in}}{\pgfqpoint{0.000000in}{0.000000in}}{%
\pgfpathmoveto{\pgfqpoint{0.000000in}{0.000000in}}%
\pgfpathlineto{\pgfqpoint{0.000000in}{-0.044444in}}%
\pgfusepath{stroke,fill}%
}%
\begin{pgfscope}%
\pgfsys@transformshift{4.378430in}{4.083121in}%
\pgfsys@useobject{currentmarker}{}%
\end{pgfscope}%
\end{pgfscope}%
\begin{pgfscope}%
\pgfsetbuttcap%
\pgfsetroundjoin%
\definecolor{currentfill}{rgb}{0.150000,0.150000,0.150000}%
\pgfsetfillcolor{currentfill}%
\pgfsetlinewidth{0.803000pt}%
\definecolor{currentstroke}{rgb}{0.150000,0.150000,0.150000}%
\pgfsetstrokecolor{currentstroke}%
\pgfsetdash{}{0pt}%
\pgfsys@defobject{currentmarker}{\pgfqpoint{0.000000in}{-0.044444in}}{\pgfqpoint{0.000000in}{0.000000in}}{%
\pgfpathmoveto{\pgfqpoint{0.000000in}{0.000000in}}%
\pgfpathlineto{\pgfqpoint{0.000000in}{-0.044444in}}%
\pgfusepath{stroke,fill}%
}%
\begin{pgfscope}%
\pgfsys@transformshift{4.441076in}{4.083121in}%
\pgfsys@useobject{currentmarker}{}%
\end{pgfscope}%
\end{pgfscope}%
\begin{pgfscope}%
\pgfsetbuttcap%
\pgfsetroundjoin%
\definecolor{currentfill}{rgb}{0.150000,0.150000,0.150000}%
\pgfsetfillcolor{currentfill}%
\pgfsetlinewidth{0.803000pt}%
\definecolor{currentstroke}{rgb}{0.150000,0.150000,0.150000}%
\pgfsetstrokecolor{currentstroke}%
\pgfsetdash{}{0pt}%
\pgfsys@defobject{currentmarker}{\pgfqpoint{0.000000in}{-0.044444in}}{\pgfqpoint{0.000000in}{0.000000in}}{%
\pgfpathmoveto{\pgfqpoint{0.000000in}{0.000000in}}%
\pgfpathlineto{\pgfqpoint{0.000000in}{-0.044444in}}%
\pgfusepath{stroke,fill}%
}%
\begin{pgfscope}%
\pgfsys@transformshift{4.489667in}{4.083121in}%
\pgfsys@useobject{currentmarker}{}%
\end{pgfscope}%
\end{pgfscope}%
\begin{pgfscope}%
\pgfsetbuttcap%
\pgfsetroundjoin%
\definecolor{currentfill}{rgb}{0.150000,0.150000,0.150000}%
\pgfsetfillcolor{currentfill}%
\pgfsetlinewidth{0.803000pt}%
\definecolor{currentstroke}{rgb}{0.150000,0.150000,0.150000}%
\pgfsetstrokecolor{currentstroke}%
\pgfsetdash{}{0pt}%
\pgfsys@defobject{currentmarker}{\pgfqpoint{0.000000in}{-0.044444in}}{\pgfqpoint{0.000000in}{0.000000in}}{%
\pgfpathmoveto{\pgfqpoint{0.000000in}{0.000000in}}%
\pgfpathlineto{\pgfqpoint{0.000000in}{-0.044444in}}%
\pgfusepath{stroke,fill}%
}%
\begin{pgfscope}%
\pgfsys@transformshift{4.529370in}{4.083121in}%
\pgfsys@useobject{currentmarker}{}%
\end{pgfscope}%
\end{pgfscope}%
\begin{pgfscope}%
\pgfsetbuttcap%
\pgfsetroundjoin%
\definecolor{currentfill}{rgb}{0.150000,0.150000,0.150000}%
\pgfsetfillcolor{currentfill}%
\pgfsetlinewidth{0.803000pt}%
\definecolor{currentstroke}{rgb}{0.150000,0.150000,0.150000}%
\pgfsetstrokecolor{currentstroke}%
\pgfsetdash{}{0pt}%
\pgfsys@defobject{currentmarker}{\pgfqpoint{0.000000in}{-0.044444in}}{\pgfqpoint{0.000000in}{0.000000in}}{%
\pgfpathmoveto{\pgfqpoint{0.000000in}{0.000000in}}%
\pgfpathlineto{\pgfqpoint{0.000000in}{-0.044444in}}%
\pgfusepath{stroke,fill}%
}%
\begin{pgfscope}%
\pgfsys@transformshift{4.562937in}{4.083121in}%
\pgfsys@useobject{currentmarker}{}%
\end{pgfscope}%
\end{pgfscope}%
\begin{pgfscope}%
\pgfsetbuttcap%
\pgfsetroundjoin%
\definecolor{currentfill}{rgb}{0.150000,0.150000,0.150000}%
\pgfsetfillcolor{currentfill}%
\pgfsetlinewidth{0.803000pt}%
\definecolor{currentstroke}{rgb}{0.150000,0.150000,0.150000}%
\pgfsetstrokecolor{currentstroke}%
\pgfsetdash{}{0pt}%
\pgfsys@defobject{currentmarker}{\pgfqpoint{0.000000in}{-0.044444in}}{\pgfqpoint{0.000000in}{0.000000in}}{%
\pgfpathmoveto{\pgfqpoint{0.000000in}{0.000000in}}%
\pgfpathlineto{\pgfqpoint{0.000000in}{-0.044444in}}%
\pgfusepath{stroke,fill}%
}%
\begin{pgfscope}%
\pgfsys@transformshift{4.592015in}{4.083121in}%
\pgfsys@useobject{currentmarker}{}%
\end{pgfscope}%
\end{pgfscope}%
\begin{pgfscope}%
\pgfsetbuttcap%
\pgfsetroundjoin%
\definecolor{currentfill}{rgb}{0.150000,0.150000,0.150000}%
\pgfsetfillcolor{currentfill}%
\pgfsetlinewidth{0.803000pt}%
\definecolor{currentstroke}{rgb}{0.150000,0.150000,0.150000}%
\pgfsetstrokecolor{currentstroke}%
\pgfsetdash{}{0pt}%
\pgfsys@defobject{currentmarker}{\pgfqpoint{0.000000in}{-0.044444in}}{\pgfqpoint{0.000000in}{0.000000in}}{%
\pgfpathmoveto{\pgfqpoint{0.000000in}{0.000000in}}%
\pgfpathlineto{\pgfqpoint{0.000000in}{-0.044444in}}%
\pgfusepath{stroke,fill}%
}%
\begin{pgfscope}%
\pgfsys@transformshift{4.617663in}{4.083121in}%
\pgfsys@useobject{currentmarker}{}%
\end{pgfscope}%
\end{pgfscope}%
\begin{pgfscope}%
\pgfsetbuttcap%
\pgfsetroundjoin%
\definecolor{currentfill}{rgb}{0.150000,0.150000,0.150000}%
\pgfsetfillcolor{currentfill}%
\pgfsetlinewidth{0.803000pt}%
\definecolor{currentstroke}{rgb}{0.150000,0.150000,0.150000}%
\pgfsetstrokecolor{currentstroke}%
\pgfsetdash{}{0pt}%
\pgfsys@defobject{currentmarker}{\pgfqpoint{0.000000in}{-0.044444in}}{\pgfqpoint{0.000000in}{0.000000in}}{%
\pgfpathmoveto{\pgfqpoint{0.000000in}{0.000000in}}%
\pgfpathlineto{\pgfqpoint{0.000000in}{-0.044444in}}%
\pgfusepath{stroke,fill}%
}%
\begin{pgfscope}%
\pgfsys@transformshift{4.791546in}{4.083121in}%
\pgfsys@useobject{currentmarker}{}%
\end{pgfscope}%
\end{pgfscope}%
\begin{pgfscope}%
\pgfsetbuttcap%
\pgfsetroundjoin%
\definecolor{currentfill}{rgb}{0.150000,0.150000,0.150000}%
\pgfsetfillcolor{currentfill}%
\pgfsetlinewidth{1.003750pt}%
\definecolor{currentstroke}{rgb}{0.150000,0.150000,0.150000}%
\pgfsetstrokecolor{currentstroke}%
\pgfsetdash{}{0pt}%
\pgfsys@defobject{currentmarker}{\pgfqpoint{-0.066667in}{0.000000in}}{\pgfqpoint{0.000000in}{0.000000in}}{%
\pgfpathmoveto{\pgfqpoint{0.000000in}{0.000000in}}%
\pgfpathlineto{\pgfqpoint{-0.066667in}{0.000000in}}%
\pgfusepath{stroke,fill}%
}%
\begin{pgfscope}%
\pgfsys@transformshift{3.637789in}{4.083121in}%
\pgfsys@useobject{currentmarker}{}%
\end{pgfscope}%
\end{pgfscope}%
\begin{pgfscope}%
\pgfsetbuttcap%
\pgfsetroundjoin%
\definecolor{currentfill}{rgb}{0.150000,0.150000,0.150000}%
\pgfsetfillcolor{currentfill}%
\pgfsetlinewidth{1.003750pt}%
\definecolor{currentstroke}{rgb}{0.150000,0.150000,0.150000}%
\pgfsetstrokecolor{currentstroke}%
\pgfsetdash{}{0pt}%
\pgfsys@defobject{currentmarker}{\pgfqpoint{-0.066667in}{0.000000in}}{\pgfqpoint{0.000000in}{0.000000in}}{%
\pgfpathmoveto{\pgfqpoint{0.000000in}{0.000000in}}%
\pgfpathlineto{\pgfqpoint{-0.066667in}{0.000000in}}%
\pgfusepath{stroke,fill}%
}%
\begin{pgfscope}%
\pgfsys@transformshift{3.637789in}{4.509594in}%
\pgfsys@useobject{currentmarker}{}%
\end{pgfscope}%
\end{pgfscope}%
\begin{pgfscope}%
\pgfsetbuttcap%
\pgfsetroundjoin%
\definecolor{currentfill}{rgb}{0.150000,0.150000,0.150000}%
\pgfsetfillcolor{currentfill}%
\pgfsetlinewidth{1.003750pt}%
\definecolor{currentstroke}{rgb}{0.150000,0.150000,0.150000}%
\pgfsetstrokecolor{currentstroke}%
\pgfsetdash{}{0pt}%
\pgfsys@defobject{currentmarker}{\pgfqpoint{-0.066667in}{0.000000in}}{\pgfqpoint{0.000000in}{0.000000in}}{%
\pgfpathmoveto{\pgfqpoint{0.000000in}{0.000000in}}%
\pgfpathlineto{\pgfqpoint{-0.066667in}{0.000000in}}%
\pgfusepath{stroke,fill}%
}%
\begin{pgfscope}%
\pgfsys@transformshift{3.637789in}{4.691069in}%
\pgfsys@useobject{currentmarker}{}%
\end{pgfscope}%
\end{pgfscope}%
\begin{pgfscope}%
\pgfpathrectangle{\pgfqpoint{3.637789in}{4.083121in}}{\pgfqpoint{1.223103in}{0.607948in}}%
\pgfusepath{clip}%
\pgfsetroundcap%
\pgfsetroundjoin%
\pgfsetlinewidth{1.204500pt}%
\definecolor{currentstroke}{rgb}{0.000000,0.501961,0.000000}%
\pgfsetstrokecolor{currentstroke}%
\pgfsetdash{}{0pt}%
\pgfpathmoveto{\pgfqpoint{3.637789in}{4.507646in}}%
\pgfpathlineto{\pgfqpoint{3.878498in}{4.509257in}}%
\pgfpathlineto{\pgfqpoint{3.990029in}{4.510858in}}%
\pgfpathlineto{\pgfqpoint{4.063425in}{4.512448in}}%
\pgfpathlineto{\pgfqpoint{4.118221in}{4.514025in}}%
\pgfpathlineto{\pgfqpoint{4.161963in}{4.515588in}}%
\pgfpathlineto{\pgfqpoint{4.198371in}{4.517136in}}%
\pgfpathlineto{\pgfqpoint{4.229555in}{4.518667in}}%
\pgfpathlineto{\pgfqpoint{4.256828in}{4.520181in}}%
\pgfpathlineto{\pgfqpoint{4.281062in}{4.521675in}}%
\pgfpathlineto{\pgfqpoint{4.302867in}{4.523148in}}%
\pgfpathlineto{\pgfqpoint{4.322686in}{4.524599in}}%
\pgfpathlineto{\pgfqpoint{4.340851in}{4.526024in}}%
\pgfpathlineto{\pgfqpoint{4.357617in}{4.527422in}}%
\pgfpathlineto{\pgfqpoint{4.373183in}{4.528791in}}%
\pgfpathlineto{\pgfqpoint{4.387711in}{4.530126in}}%
\pgfpathlineto{\pgfqpoint{4.401329in}{4.531426in}}%
\pgfpathlineto{\pgfqpoint{4.414146in}{4.532686in}}%
\pgfpathlineto{\pgfqpoint{4.426250in}{4.533903in}}%
\pgfpathlineto{\pgfqpoint{4.437717in}{4.535072in}}%
\pgfpathlineto{\pgfqpoint{4.448610in}{4.536188in}}%
\pgfpathlineto{\pgfqpoint{4.458984in}{4.537247in}}%
\pgfpathlineto{\pgfqpoint{4.468886in}{4.538242in}}%
\pgfpathlineto{\pgfqpoint{4.478357in}{4.539166in}}%
\pgfpathlineto{\pgfqpoint{4.487434in}{4.540014in}}%
\pgfpathlineto{\pgfqpoint{4.496147in}{4.540778in}}%
\pgfpathlineto{\pgfqpoint{4.504525in}{4.541449in}}%
\pgfpathlineto{\pgfqpoint{4.512593in}{4.542018in}}%
\pgfpathlineto{\pgfqpoint{4.520372in}{4.542477in}}%
\pgfpathlineto{\pgfqpoint{4.527883in}{4.542815in}}%
\pgfpathlineto{\pgfqpoint{4.535144in}{4.543021in}}%
\pgfpathlineto{\pgfqpoint{4.542170in}{4.543083in}}%
\pgfpathlineto{\pgfqpoint{4.548977in}{4.542989in}}%
\pgfpathlineto{\pgfqpoint{4.555577in}{4.542725in}}%
\pgfpathlineto{\pgfqpoint{4.561983in}{4.542278in}}%
\pgfpathlineto{\pgfqpoint{4.568206in}{4.541632in}}%
\pgfpathlineto{\pgfqpoint{4.574256in}{4.540770in}}%
\pgfpathlineto{\pgfqpoint{4.580143in}{4.539677in}}%
\pgfpathlineto{\pgfqpoint{4.585874in}{4.538334in}}%
\pgfpathlineto{\pgfqpoint{4.591459in}{4.536722in}}%
\pgfpathlineto{\pgfqpoint{4.596904in}{4.534821in}}%
\pgfpathlineto{\pgfqpoint{4.602216in}{4.532609in}}%
\pgfpathlineto{\pgfqpoint{4.607401in}{4.530065in}}%
\pgfpathlineto{\pgfqpoint{4.612466in}{4.527166in}}%
\pgfpathlineto{\pgfqpoint{4.617416in}{4.523885in}}%
\pgfpathlineto{\pgfqpoint{4.622256in}{4.520199in}}%
\pgfpathlineto{\pgfqpoint{4.626991in}{4.516080in}}%
\pgfpathlineto{\pgfqpoint{4.631625in}{4.511499in}}%
\pgfpathlineto{\pgfqpoint{4.636162in}{4.506427in}}%
\pgfpathlineto{\pgfqpoint{4.640607in}{4.500834in}}%
\pgfusepath{stroke}%
\end{pgfscope}%
\begin{pgfscope}%
\pgfsetrectcap%
\pgfsetmiterjoin%
\pgfsetlinewidth{1.003750pt}%
\definecolor{currentstroke}{rgb}{0.150000,0.150000,0.150000}%
\pgfsetstrokecolor{currentstroke}%
\pgfsetdash{}{0pt}%
\pgfpathmoveto{\pgfqpoint{3.637789in}{4.083121in}}%
\pgfpathlineto{\pgfqpoint{3.637789in}{4.691069in}}%
\pgfusepath{stroke}%
\end{pgfscope}%
\begin{pgfscope}%
\pgfsetrectcap%
\pgfsetmiterjoin%
\pgfsetlinewidth{1.003750pt}%
\definecolor{currentstroke}{rgb}{0.150000,0.150000,0.150000}%
\pgfsetstrokecolor{currentstroke}%
\pgfsetdash{}{0pt}%
\pgfpathmoveto{\pgfqpoint{3.637789in}{4.083121in}}%
\pgfpathlineto{\pgfqpoint{4.860892in}{4.083121in}}%
\pgfusepath{stroke}%
\end{pgfscope}%
\begin{pgfscope}%
\pgfpathrectangle{\pgfqpoint{3.637789in}{4.083121in}}{\pgfqpoint{1.223103in}{0.607948in}}%
\pgfusepath{clip}%
\pgfsetbuttcap%
\pgfsetroundjoin%
\definecolor{currentfill}{rgb}{0.000000,0.000000,0.000000}%
\pgfsetfillcolor{currentfill}%
\pgfsetlinewidth{1.003750pt}%
\definecolor{currentstroke}{rgb}{0.000000,0.000000,0.000000}%
\pgfsetstrokecolor{currentstroke}%
\pgfsetdash{}{0pt}%
\pgfsys@defobject{currentmarker}{\pgfqpoint{-0.013889in}{-0.013889in}}{\pgfqpoint{0.013889in}{0.013889in}}{%
\pgfpathmoveto{\pgfqpoint{0.000000in}{-0.013889in}}%
\pgfpathcurveto{\pgfqpoint{0.003683in}{-0.013889in}}{\pgfqpoint{0.007216in}{-0.012425in}}{\pgfqpoint{0.009821in}{-0.009821in}}%
\pgfpathcurveto{\pgfqpoint{0.012425in}{-0.007216in}}{\pgfqpoint{0.013889in}{-0.003683in}}{\pgfqpoint{0.013889in}{0.000000in}}%
\pgfpathcurveto{\pgfqpoint{0.013889in}{0.003683in}}{\pgfqpoint{0.012425in}{0.007216in}}{\pgfqpoint{0.009821in}{0.009821in}}%
\pgfpathcurveto{\pgfqpoint{0.007216in}{0.012425in}}{\pgfqpoint{0.003683in}{0.013889in}}{\pgfqpoint{0.000000in}{0.013889in}}%
\pgfpathcurveto{\pgfqpoint{-0.003683in}{0.013889in}}{\pgfqpoint{-0.007216in}{0.012425in}}{\pgfqpoint{-0.009821in}{0.009821in}}%
\pgfpathcurveto{\pgfqpoint{-0.012425in}{0.007216in}}{\pgfqpoint{-0.013889in}{0.003683in}}{\pgfqpoint{-0.013889in}{0.000000in}}%
\pgfpathcurveto{\pgfqpoint{-0.013889in}{-0.003683in}}{\pgfqpoint{-0.012425in}{-0.007216in}}{\pgfqpoint{-0.009821in}{-0.009821in}}%
\pgfpathcurveto{\pgfqpoint{-0.007216in}{-0.012425in}}{\pgfqpoint{-0.003683in}{-0.013889in}}{\pgfqpoint{0.000000in}{-0.013889in}}%
\pgfpathclose%
\pgfusepath{stroke,fill}%
}%
\begin{pgfscope}%
\pgfsys@transformshift{4.640607in}{4.500244in}%
\pgfsys@useobject{currentmarker}{}%
\end{pgfscope}%
\begin{pgfscope}%
\pgfsys@transformshift{4.290137in}{4.522601in}%
\pgfsys@useobject{currentmarker}{}%
\end{pgfscope}%
\begin{pgfscope}%
\pgfsys@transformshift{4.294536in}{4.522857in}%
\pgfsys@useobject{currentmarker}{}%
\end{pgfscope}%
\begin{pgfscope}%
\pgfsys@transformshift{4.299026in}{4.523124in}%
\pgfsys@useobject{currentmarker}{}%
\end{pgfscope}%
\begin{pgfscope}%
\pgfsys@transformshift{4.303611in}{4.523401in}%
\pgfsys@useobject{currentmarker}{}%
\end{pgfscope}%
\begin{pgfscope}%
\pgfsys@transformshift{4.308294in}{4.523689in}%
\pgfsys@useobject{currentmarker}{}%
\end{pgfscope}%
\begin{pgfscope}%
\pgfsys@transformshift{4.313080in}{4.523990in}%
\pgfsys@useobject{currentmarker}{}%
\end{pgfscope}%
\begin{pgfscope}%
\pgfsys@transformshift{4.317974in}{4.524304in}%
\pgfsys@useobject{currentmarker}{}%
\end{pgfscope}%
\begin{pgfscope}%
\pgfsys@transformshift{4.322980in}{4.524631in}%
\pgfsys@useobject{currentmarker}{}%
\end{pgfscope}%
\begin{pgfscope}%
\pgfsys@transformshift{4.328104in}{4.524973in}%
\pgfsys@useobject{currentmarker}{}%
\end{pgfscope}%
\begin{pgfscope}%
\pgfsys@transformshift{4.333351in}{4.525332in}%
\pgfsys@useobject{currentmarker}{}%
\end{pgfscope}%
\begin{pgfscope}%
\pgfsys@transformshift{4.338728in}{4.525706in}%
\pgfsys@useobject{currentmarker}{}%
\end{pgfscope}%
\begin{pgfscope}%
\pgfsys@transformshift{4.367806in}{4.527879in}%
\pgfsys@useobject{currentmarker}{}%
\end{pgfscope}%
\begin{pgfscope}%
\pgfsys@transformshift{4.401374in}{4.530707in}%
\pgfsys@useobject{currentmarker}{}%
\end{pgfscope}%
\begin{pgfscope}%
\pgfsys@transformshift{4.441076in}{4.534492in}%
\pgfsys@useobject{currentmarker}{}%
\end{pgfscope}%
\begin{pgfscope}%
\pgfsys@transformshift{4.489667in}{4.539575in}%
\pgfsys@useobject{currentmarker}{}%
\end{pgfscope}%
\begin{pgfscope}%
\pgfsys@transformshift{4.552313in}{4.544541in}%
\pgfsys@useobject{currentmarker}{}%
\end{pgfscope}%
\end{pgfscope}%
\begin{pgfscope}%
\pgfsetbuttcap%
\pgfsetmiterjoin%
\definecolor{currentfill}{rgb}{1.000000,1.000000,1.000000}%
\pgfsetfillcolor{currentfill}%
\pgfsetlinewidth{0.000000pt}%
\definecolor{currentstroke}{rgb}{0.000000,0.000000,0.000000}%
\pgfsetstrokecolor{currentstroke}%
\pgfsetstrokeopacity{0.000000}%
\pgfsetdash{}{0pt}%
\pgfpathmoveto{\pgfqpoint{5.105513in}{4.083121in}}%
\pgfpathlineto{\pgfqpoint{6.328616in}{4.083121in}}%
\pgfpathlineto{\pgfqpoint{6.328616in}{4.691069in}}%
\pgfpathlineto{\pgfqpoint{5.105513in}{4.691069in}}%
\pgfpathclose%
\pgfusepath{fill}%
\end{pgfscope}%
\begin{pgfscope}%
\pgfpathrectangle{\pgfqpoint{5.105513in}{4.083121in}}{\pgfqpoint{1.223103in}{0.607948in}}%
\pgfusepath{clip}%
\pgfsetbuttcap%
\pgfsetmiterjoin%
\definecolor{currentfill}{rgb}{0.000000,0.000000,1.000000}%
\pgfsetfillcolor{currentfill}%
\pgfsetfillopacity{0.100000}%
\pgfsetlinewidth{0.803000pt}%
\definecolor{currentstroke}{rgb}{0.000000,0.000000,1.000000}%
\pgfsetstrokecolor{currentstroke}%
\pgfsetstrokeopacity{0.100000}%
\pgfsetdash{}{0pt}%
\pgfpathmoveto{\pgfqpoint{5.105513in}{4.508392in}}%
\pgfpathlineto{\pgfqpoint{5.105513in}{4.510650in}}%
\pgfpathlineto{\pgfqpoint{6.328616in}{4.510650in}}%
\pgfpathlineto{\pgfqpoint{6.328616in}{4.508392in}}%
\pgfpathclose%
\pgfusepath{stroke,fill}%
\end{pgfscope}%
\begin{pgfscope}%
\pgfpathrectangle{\pgfqpoint{5.105513in}{4.083121in}}{\pgfqpoint{1.223103in}{0.607948in}}%
\pgfusepath{clip}%
\pgfsetbuttcap%
\pgfsetroundjoin%
\definecolor{currentfill}{rgb}{0.000000,0.501961,0.000000}%
\pgfsetfillcolor{currentfill}%
\pgfsetfillopacity{0.500000}%
\pgfsetlinewidth{0.803000pt}%
\definecolor{currentstroke}{rgb}{0.000000,0.501961,0.000000}%
\pgfsetstrokecolor{currentstroke}%
\pgfsetstrokeopacity{0.500000}%
\pgfsetdash{}{0pt}%
\pgfpathmoveto{\pgfqpoint{5.105513in}{4.511359in}}%
\pgfpathlineto{\pgfqpoint{5.105513in}{4.509345in}}%
\pgfpathlineto{\pgfqpoint{5.346222in}{4.511219in}}%
\pgfpathlineto{\pgfqpoint{5.457753in}{4.513032in}}%
\pgfpathlineto{\pgfqpoint{5.531149in}{4.514791in}}%
\pgfpathlineto{\pgfqpoint{5.585945in}{4.516505in}}%
\pgfpathlineto{\pgfqpoint{5.629687in}{4.518180in}}%
\pgfpathlineto{\pgfqpoint{5.666095in}{4.519822in}}%
\pgfpathlineto{\pgfqpoint{5.697279in}{4.521437in}}%
\pgfpathlineto{\pgfqpoint{5.724552in}{4.523029in}}%
\pgfpathlineto{\pgfqpoint{5.748786in}{4.524603in}}%
\pgfpathlineto{\pgfqpoint{5.770591in}{4.526160in}}%
\pgfpathlineto{\pgfqpoint{5.790410in}{4.527698in}}%
\pgfpathlineto{\pgfqpoint{5.808575in}{4.529193in}}%
\pgfpathlineto{\pgfqpoint{5.825341in}{4.530682in}}%
\pgfpathlineto{\pgfqpoint{5.840907in}{4.532168in}}%
\pgfpathlineto{\pgfqpoint{5.855435in}{4.533651in}}%
\pgfpathlineto{\pgfqpoint{5.869053in}{4.535130in}}%
\pgfpathlineto{\pgfqpoint{5.881870in}{4.536607in}}%
\pgfpathlineto{\pgfqpoint{5.893974in}{4.538080in}}%
\pgfpathlineto{\pgfqpoint{5.905441in}{4.539549in}}%
\pgfpathlineto{\pgfqpoint{5.916334in}{4.541015in}}%
\pgfpathlineto{\pgfqpoint{5.926708in}{4.542476in}}%
\pgfpathlineto{\pgfqpoint{5.936610in}{4.543934in}}%
\pgfpathlineto{\pgfqpoint{5.946082in}{4.545387in}}%
\pgfpathlineto{\pgfqpoint{5.955158in}{4.546835in}}%
\pgfpathlineto{\pgfqpoint{5.963871in}{4.548279in}}%
\pgfpathlineto{\pgfqpoint{5.972249in}{4.549718in}}%
\pgfpathlineto{\pgfqpoint{5.980317in}{4.551151in}}%
\pgfpathlineto{\pgfqpoint{5.988096in}{4.552575in}}%
\pgfpathlineto{\pgfqpoint{5.995607in}{4.553987in}}%
\pgfpathlineto{\pgfqpoint{6.002868in}{4.555385in}}%
\pgfpathlineto{\pgfqpoint{6.009894in}{4.556770in}}%
\pgfpathlineto{\pgfqpoint{6.016701in}{4.558145in}}%
\pgfpathlineto{\pgfqpoint{6.023301in}{4.559510in}}%
\pgfpathlineto{\pgfqpoint{6.029707in}{4.560870in}}%
\pgfpathlineto{\pgfqpoint{6.035930in}{4.562224in}}%
\pgfpathlineto{\pgfqpoint{6.041980in}{4.563576in}}%
\pgfpathlineto{\pgfqpoint{6.047867in}{4.564929in}}%
\pgfpathlineto{\pgfqpoint{6.053598in}{4.566285in}}%
\pgfpathlineto{\pgfqpoint{6.059183in}{4.567647in}}%
\pgfpathlineto{\pgfqpoint{6.064628in}{4.569021in}}%
\pgfpathlineto{\pgfqpoint{6.069940in}{4.570410in}}%
\pgfpathlineto{\pgfqpoint{6.075126in}{4.571821in}}%
\pgfpathlineto{\pgfqpoint{6.080191in}{4.573259in}}%
\pgfpathlineto{\pgfqpoint{6.085140in}{4.574732in}}%
\pgfpathlineto{\pgfqpoint{6.089980in}{4.576247in}}%
\pgfpathlineto{\pgfqpoint{6.094715in}{4.577813in}}%
\pgfpathlineto{\pgfqpoint{6.099349in}{4.579438in}}%
\pgfpathlineto{\pgfqpoint{6.103886in}{4.581133in}}%
\pgfpathlineto{\pgfqpoint{6.108331in}{4.582862in}}%
\pgfpathlineto{\pgfqpoint{6.108331in}{4.582922in}}%
\pgfpathlineto{\pgfqpoint{6.108331in}{4.582922in}}%
\pgfpathlineto{\pgfqpoint{6.103886in}{4.581430in}}%
\pgfpathlineto{\pgfqpoint{6.099349in}{4.579982in}}%
\pgfpathlineto{\pgfqpoint{6.094715in}{4.578533in}}%
\pgfpathlineto{\pgfqpoint{6.089980in}{4.577082in}}%
\pgfpathlineto{\pgfqpoint{6.085140in}{4.575631in}}%
\pgfpathlineto{\pgfqpoint{6.080191in}{4.574181in}}%
\pgfpathlineto{\pgfqpoint{6.075126in}{4.572732in}}%
\pgfpathlineto{\pgfqpoint{6.069940in}{4.571285in}}%
\pgfpathlineto{\pgfqpoint{6.064628in}{4.569840in}}%
\pgfpathlineto{\pgfqpoint{6.059183in}{4.568397in}}%
\pgfpathlineto{\pgfqpoint{6.053598in}{4.566957in}}%
\pgfpathlineto{\pgfqpoint{6.047867in}{4.565519in}}%
\pgfpathlineto{\pgfqpoint{6.041980in}{4.564083in}}%
\pgfpathlineto{\pgfqpoint{6.035930in}{4.562650in}}%
\pgfpathlineto{\pgfqpoint{6.029707in}{4.561218in}}%
\pgfpathlineto{\pgfqpoint{6.023301in}{4.559788in}}%
\pgfpathlineto{\pgfqpoint{6.016701in}{4.558359in}}%
\pgfpathlineto{\pgfqpoint{6.009894in}{4.556932in}}%
\pgfpathlineto{\pgfqpoint{6.002868in}{4.555507in}}%
\pgfpathlineto{\pgfqpoint{5.995607in}{4.554087in}}%
\pgfpathlineto{\pgfqpoint{5.988096in}{4.552670in}}%
\pgfpathlineto{\pgfqpoint{5.980317in}{4.551254in}}%
\pgfpathlineto{\pgfqpoint{5.972249in}{4.549835in}}%
\pgfpathlineto{\pgfqpoint{5.963871in}{4.548411in}}%
\pgfpathlineto{\pgfqpoint{5.955158in}{4.546980in}}%
\pgfpathlineto{\pgfqpoint{5.946082in}{4.545543in}}%
\pgfpathlineto{\pgfqpoint{5.936610in}{4.544098in}}%
\pgfpathlineto{\pgfqpoint{5.926708in}{4.542646in}}%
\pgfpathlineto{\pgfqpoint{5.916334in}{4.541186in}}%
\pgfpathlineto{\pgfqpoint{5.905441in}{4.539720in}}%
\pgfpathlineto{\pgfqpoint{5.893974in}{4.538246in}}%
\pgfpathlineto{\pgfqpoint{5.881870in}{4.536765in}}%
\pgfpathlineto{\pgfqpoint{5.869053in}{4.535276in}}%
\pgfpathlineto{\pgfqpoint{5.855435in}{4.533779in}}%
\pgfpathlineto{\pgfqpoint{5.840907in}{4.532274in}}%
\pgfpathlineto{\pgfqpoint{5.825341in}{4.530761in}}%
\pgfpathlineto{\pgfqpoint{5.808575in}{4.529239in}}%
\pgfpathlineto{\pgfqpoint{5.790410in}{4.527709in}}%
\pgfpathlineto{\pgfqpoint{5.770591in}{4.526207in}}%
\pgfpathlineto{\pgfqpoint{5.748786in}{4.524709in}}%
\pgfpathlineto{\pgfqpoint{5.724552in}{4.523210in}}%
\pgfpathlineto{\pgfqpoint{5.697279in}{4.521711in}}%
\pgfpathlineto{\pgfqpoint{5.666095in}{4.520212in}}%
\pgfpathlineto{\pgfqpoint{5.629687in}{4.518716in}}%
\pgfpathlineto{\pgfqpoint{5.585945in}{4.517224in}}%
\pgfpathlineto{\pgfqpoint{5.531149in}{4.515740in}}%
\pgfpathlineto{\pgfqpoint{5.457753in}{4.514266in}}%
\pgfpathlineto{\pgfqpoint{5.346222in}{4.512804in}}%
\pgfpathlineto{\pgfqpoint{5.105513in}{4.511359in}}%
\pgfpathclose%
\pgfusepath{stroke,fill}%
\end{pgfscope}%
\begin{pgfscope}%
\pgfpathrectangle{\pgfqpoint{5.105513in}{4.083121in}}{\pgfqpoint{1.223103in}{0.607948in}}%
\pgfusepath{clip}%
\pgfsetroundcap%
\pgfsetroundjoin%
\pgfsetlinewidth{0.501875pt}%
\definecolor{currentstroke}{rgb}{0.000000,0.000000,1.000000}%
\pgfsetstrokecolor{currentstroke}%
\pgfsetstrokeopacity{0.800000}%
\pgfsetdash{}{0pt}%
\pgfpathmoveto{\pgfqpoint{5.105513in}{4.509521in}}%
\pgfpathlineto{\pgfqpoint{6.328616in}{4.509521in}}%
\pgfusepath{stroke}%
\end{pgfscope}%
\begin{pgfscope}%
\pgfpathrectangle{\pgfqpoint{5.105513in}{4.083121in}}{\pgfqpoint{1.223103in}{0.607948in}}%
\pgfusepath{clip}%
\pgfsetbuttcap%
\pgfsetroundjoin%
\pgfsetlinewidth{1.003750pt}%
\definecolor{currentstroke}{rgb}{0.000000,0.000000,0.000000}%
\pgfsetstrokecolor{currentstroke}%
\pgfsetdash{{3.700000pt}{1.600000pt}}{0.000000pt}%
\pgfpathmoveto{\pgfqpoint{5.105513in}{4.509594in}}%
\pgfpathlineto{\pgfqpoint{6.328616in}{4.509594in}}%
\pgfusepath{stroke}%
\end{pgfscope}%
\begin{pgfscope}%
\pgfsetroundcap%
\pgfsetroundjoin%
\pgfsetlinewidth{0.501875pt}%
\definecolor{currentstroke}{rgb}{0.000000,0.000000,1.000000}%
\pgfsetstrokecolor{currentstroke}%
\pgfsetstrokeopacity{0.800000}%
\pgfsetdash{}{0pt}%
\pgfpathmoveto{\pgfqpoint{5.932592in}{4.627624in}}%
\pgfpathquadraticcurveto{\pgfqpoint{5.856728in}{4.576347in}}{\pgfqpoint{5.780865in}{4.525070in}}%
\pgfusepath{stroke}%
\end{pgfscope}%
\begin{pgfscope}%
\pgfsetfillopacity{0.800000}%
\pgfsetstrokeopacity{0.800000}%
\definecolor{textcolor}{rgb}{0.000000,0.000000,1.000000}%
\pgfsetstrokecolor{textcolor}%
\pgfsetfillcolor{textcolor}%
\pgftext[x=5.846155in,y=4.691905in,left,base]{\color{textcolor}\sffamily\fontsize{5.647059}{6.776471}\selectfont 11.7014(19)}%
\end{pgfscope}%
\begin{pgfscope}%
\pgfsetbuttcap%
\pgfsetroundjoin%
\definecolor{currentfill}{rgb}{0.150000,0.150000,0.150000}%
\pgfsetfillcolor{currentfill}%
\pgfsetlinewidth{1.003750pt}%
\definecolor{currentstroke}{rgb}{0.150000,0.150000,0.150000}%
\pgfsetstrokecolor{currentstroke}%
\pgfsetdash{}{0pt}%
\pgfsys@defobject{currentmarker}{\pgfqpoint{0.000000in}{-0.066667in}}{\pgfqpoint{0.000000in}{0.000000in}}{%
\pgfpathmoveto{\pgfqpoint{0.000000in}{0.000000in}}%
\pgfpathlineto{\pgfqpoint{0.000000in}{-0.066667in}}%
\pgfusepath{stroke,fill}%
}%
\begin{pgfscope}%
\pgfsys@transformshift{5.105513in}{4.083121in}%
\pgfsys@useobject{currentmarker}{}%
\end{pgfscope}%
\end{pgfscope}%
\begin{pgfscope}%
\pgfsetbuttcap%
\pgfsetroundjoin%
\definecolor{currentfill}{rgb}{0.150000,0.150000,0.150000}%
\pgfsetfillcolor{currentfill}%
\pgfsetlinewidth{1.003750pt}%
\definecolor{currentstroke}{rgb}{0.150000,0.150000,0.150000}%
\pgfsetstrokecolor{currentstroke}%
\pgfsetdash{}{0pt}%
\pgfsys@defobject{currentmarker}{\pgfqpoint{0.000000in}{-0.066667in}}{\pgfqpoint{0.000000in}{0.000000in}}{%
\pgfpathmoveto{\pgfqpoint{0.000000in}{0.000000in}}%
\pgfpathlineto{\pgfqpoint{0.000000in}{-0.066667in}}%
\pgfusepath{stroke,fill}%
}%
\begin{pgfscope}%
\pgfsys@transformshift{5.606922in}{4.083121in}%
\pgfsys@useobject{currentmarker}{}%
\end{pgfscope}%
\end{pgfscope}%
\begin{pgfscope}%
\pgfsetbuttcap%
\pgfsetroundjoin%
\definecolor{currentfill}{rgb}{0.150000,0.150000,0.150000}%
\pgfsetfillcolor{currentfill}%
\pgfsetlinewidth{1.003750pt}%
\definecolor{currentstroke}{rgb}{0.150000,0.150000,0.150000}%
\pgfsetstrokecolor{currentstroke}%
\pgfsetdash{}{0pt}%
\pgfsys@defobject{currentmarker}{\pgfqpoint{0.000000in}{-0.066667in}}{\pgfqpoint{0.000000in}{0.000000in}}{%
\pgfpathmoveto{\pgfqpoint{0.000000in}{0.000000in}}%
\pgfpathlineto{\pgfqpoint{0.000000in}{-0.066667in}}%
\pgfusepath{stroke,fill}%
}%
\begin{pgfscope}%
\pgfsys@transformshift{6.108331in}{4.083121in}%
\pgfsys@useobject{currentmarker}{}%
\end{pgfscope}%
\end{pgfscope}%
\begin{pgfscope}%
\pgfsetbuttcap%
\pgfsetroundjoin%
\definecolor{currentfill}{rgb}{0.150000,0.150000,0.150000}%
\pgfsetfillcolor{currentfill}%
\pgfsetlinewidth{0.803000pt}%
\definecolor{currentstroke}{rgb}{0.150000,0.150000,0.150000}%
\pgfsetstrokecolor{currentstroke}%
\pgfsetdash{}{0pt}%
\pgfsys@defobject{currentmarker}{\pgfqpoint{0.000000in}{-0.044444in}}{\pgfqpoint{0.000000in}{0.000000in}}{%
\pgfpathmoveto{\pgfqpoint{0.000000in}{0.000000in}}%
\pgfpathlineto{\pgfqpoint{0.000000in}{-0.044444in}}%
\pgfusepath{stroke,fill}%
}%
\begin{pgfscope}%
\pgfsys@transformshift{5.256452in}{4.083121in}%
\pgfsys@useobject{currentmarker}{}%
\end{pgfscope}%
\end{pgfscope}%
\begin{pgfscope}%
\pgfsetbuttcap%
\pgfsetroundjoin%
\definecolor{currentfill}{rgb}{0.150000,0.150000,0.150000}%
\pgfsetfillcolor{currentfill}%
\pgfsetlinewidth{0.803000pt}%
\definecolor{currentstroke}{rgb}{0.150000,0.150000,0.150000}%
\pgfsetstrokecolor{currentstroke}%
\pgfsetdash{}{0pt}%
\pgfsys@defobject{currentmarker}{\pgfqpoint{0.000000in}{-0.044444in}}{\pgfqpoint{0.000000in}{0.000000in}}{%
\pgfpathmoveto{\pgfqpoint{0.000000in}{0.000000in}}%
\pgfpathlineto{\pgfqpoint{0.000000in}{-0.044444in}}%
\pgfusepath{stroke,fill}%
}%
\begin{pgfscope}%
\pgfsys@transformshift{5.344746in}{4.083121in}%
\pgfsys@useobject{currentmarker}{}%
\end{pgfscope}%
\end{pgfscope}%
\begin{pgfscope}%
\pgfsetbuttcap%
\pgfsetroundjoin%
\definecolor{currentfill}{rgb}{0.150000,0.150000,0.150000}%
\pgfsetfillcolor{currentfill}%
\pgfsetlinewidth{0.803000pt}%
\definecolor{currentstroke}{rgb}{0.150000,0.150000,0.150000}%
\pgfsetstrokecolor{currentstroke}%
\pgfsetdash{}{0pt}%
\pgfsys@defobject{currentmarker}{\pgfqpoint{0.000000in}{-0.044444in}}{\pgfqpoint{0.000000in}{0.000000in}}{%
\pgfpathmoveto{\pgfqpoint{0.000000in}{0.000000in}}%
\pgfpathlineto{\pgfqpoint{0.000000in}{-0.044444in}}%
\pgfusepath{stroke,fill}%
}%
\begin{pgfscope}%
\pgfsys@transformshift{5.407391in}{4.083121in}%
\pgfsys@useobject{currentmarker}{}%
\end{pgfscope}%
\end{pgfscope}%
\begin{pgfscope}%
\pgfsetbuttcap%
\pgfsetroundjoin%
\definecolor{currentfill}{rgb}{0.150000,0.150000,0.150000}%
\pgfsetfillcolor{currentfill}%
\pgfsetlinewidth{0.803000pt}%
\definecolor{currentstroke}{rgb}{0.150000,0.150000,0.150000}%
\pgfsetstrokecolor{currentstroke}%
\pgfsetdash{}{0pt}%
\pgfsys@defobject{currentmarker}{\pgfqpoint{0.000000in}{-0.044444in}}{\pgfqpoint{0.000000in}{0.000000in}}{%
\pgfpathmoveto{\pgfqpoint{0.000000in}{0.000000in}}%
\pgfpathlineto{\pgfqpoint{0.000000in}{-0.044444in}}%
\pgfusepath{stroke,fill}%
}%
\begin{pgfscope}%
\pgfsys@transformshift{5.455982in}{4.083121in}%
\pgfsys@useobject{currentmarker}{}%
\end{pgfscope}%
\end{pgfscope}%
\begin{pgfscope}%
\pgfsetbuttcap%
\pgfsetroundjoin%
\definecolor{currentfill}{rgb}{0.150000,0.150000,0.150000}%
\pgfsetfillcolor{currentfill}%
\pgfsetlinewidth{0.803000pt}%
\definecolor{currentstroke}{rgb}{0.150000,0.150000,0.150000}%
\pgfsetstrokecolor{currentstroke}%
\pgfsetdash{}{0pt}%
\pgfsys@defobject{currentmarker}{\pgfqpoint{0.000000in}{-0.044444in}}{\pgfqpoint{0.000000in}{0.000000in}}{%
\pgfpathmoveto{\pgfqpoint{0.000000in}{0.000000in}}%
\pgfpathlineto{\pgfqpoint{0.000000in}{-0.044444in}}%
\pgfusepath{stroke,fill}%
}%
\begin{pgfscope}%
\pgfsys@transformshift{5.495685in}{4.083121in}%
\pgfsys@useobject{currentmarker}{}%
\end{pgfscope}%
\end{pgfscope}%
\begin{pgfscope}%
\pgfsetbuttcap%
\pgfsetroundjoin%
\definecolor{currentfill}{rgb}{0.150000,0.150000,0.150000}%
\pgfsetfillcolor{currentfill}%
\pgfsetlinewidth{0.803000pt}%
\definecolor{currentstroke}{rgb}{0.150000,0.150000,0.150000}%
\pgfsetstrokecolor{currentstroke}%
\pgfsetdash{}{0pt}%
\pgfsys@defobject{currentmarker}{\pgfqpoint{0.000000in}{-0.044444in}}{\pgfqpoint{0.000000in}{0.000000in}}{%
\pgfpathmoveto{\pgfqpoint{0.000000in}{0.000000in}}%
\pgfpathlineto{\pgfqpoint{0.000000in}{-0.044444in}}%
\pgfusepath{stroke,fill}%
}%
\begin{pgfscope}%
\pgfsys@transformshift{5.529252in}{4.083121in}%
\pgfsys@useobject{currentmarker}{}%
\end{pgfscope}%
\end{pgfscope}%
\begin{pgfscope}%
\pgfsetbuttcap%
\pgfsetroundjoin%
\definecolor{currentfill}{rgb}{0.150000,0.150000,0.150000}%
\pgfsetfillcolor{currentfill}%
\pgfsetlinewidth{0.803000pt}%
\definecolor{currentstroke}{rgb}{0.150000,0.150000,0.150000}%
\pgfsetstrokecolor{currentstroke}%
\pgfsetdash{}{0pt}%
\pgfsys@defobject{currentmarker}{\pgfqpoint{0.000000in}{-0.044444in}}{\pgfqpoint{0.000000in}{0.000000in}}{%
\pgfpathmoveto{\pgfqpoint{0.000000in}{0.000000in}}%
\pgfpathlineto{\pgfqpoint{0.000000in}{-0.044444in}}%
\pgfusepath{stroke,fill}%
}%
\begin{pgfscope}%
\pgfsys@transformshift{5.558330in}{4.083121in}%
\pgfsys@useobject{currentmarker}{}%
\end{pgfscope}%
\end{pgfscope}%
\begin{pgfscope}%
\pgfsetbuttcap%
\pgfsetroundjoin%
\definecolor{currentfill}{rgb}{0.150000,0.150000,0.150000}%
\pgfsetfillcolor{currentfill}%
\pgfsetlinewidth{0.803000pt}%
\definecolor{currentstroke}{rgb}{0.150000,0.150000,0.150000}%
\pgfsetstrokecolor{currentstroke}%
\pgfsetdash{}{0pt}%
\pgfsys@defobject{currentmarker}{\pgfqpoint{0.000000in}{-0.044444in}}{\pgfqpoint{0.000000in}{0.000000in}}{%
\pgfpathmoveto{\pgfqpoint{0.000000in}{0.000000in}}%
\pgfpathlineto{\pgfqpoint{0.000000in}{-0.044444in}}%
\pgfusepath{stroke,fill}%
}%
\begin{pgfscope}%
\pgfsys@transformshift{5.583978in}{4.083121in}%
\pgfsys@useobject{currentmarker}{}%
\end{pgfscope}%
\end{pgfscope}%
\begin{pgfscope}%
\pgfsetbuttcap%
\pgfsetroundjoin%
\definecolor{currentfill}{rgb}{0.150000,0.150000,0.150000}%
\pgfsetfillcolor{currentfill}%
\pgfsetlinewidth{0.803000pt}%
\definecolor{currentstroke}{rgb}{0.150000,0.150000,0.150000}%
\pgfsetstrokecolor{currentstroke}%
\pgfsetdash{}{0pt}%
\pgfsys@defobject{currentmarker}{\pgfqpoint{0.000000in}{-0.044444in}}{\pgfqpoint{0.000000in}{0.000000in}}{%
\pgfpathmoveto{\pgfqpoint{0.000000in}{0.000000in}}%
\pgfpathlineto{\pgfqpoint{0.000000in}{-0.044444in}}%
\pgfusepath{stroke,fill}%
}%
\begin{pgfscope}%
\pgfsys@transformshift{5.757861in}{4.083121in}%
\pgfsys@useobject{currentmarker}{}%
\end{pgfscope}%
\end{pgfscope}%
\begin{pgfscope}%
\pgfsetbuttcap%
\pgfsetroundjoin%
\definecolor{currentfill}{rgb}{0.150000,0.150000,0.150000}%
\pgfsetfillcolor{currentfill}%
\pgfsetlinewidth{0.803000pt}%
\definecolor{currentstroke}{rgb}{0.150000,0.150000,0.150000}%
\pgfsetstrokecolor{currentstroke}%
\pgfsetdash{}{0pt}%
\pgfsys@defobject{currentmarker}{\pgfqpoint{0.000000in}{-0.044444in}}{\pgfqpoint{0.000000in}{0.000000in}}{%
\pgfpathmoveto{\pgfqpoint{0.000000in}{0.000000in}}%
\pgfpathlineto{\pgfqpoint{0.000000in}{-0.044444in}}%
\pgfusepath{stroke,fill}%
}%
\begin{pgfscope}%
\pgfsys@transformshift{5.846155in}{4.083121in}%
\pgfsys@useobject{currentmarker}{}%
\end{pgfscope}%
\end{pgfscope}%
\begin{pgfscope}%
\pgfsetbuttcap%
\pgfsetroundjoin%
\definecolor{currentfill}{rgb}{0.150000,0.150000,0.150000}%
\pgfsetfillcolor{currentfill}%
\pgfsetlinewidth{0.803000pt}%
\definecolor{currentstroke}{rgb}{0.150000,0.150000,0.150000}%
\pgfsetstrokecolor{currentstroke}%
\pgfsetdash{}{0pt}%
\pgfsys@defobject{currentmarker}{\pgfqpoint{0.000000in}{-0.044444in}}{\pgfqpoint{0.000000in}{0.000000in}}{%
\pgfpathmoveto{\pgfqpoint{0.000000in}{0.000000in}}%
\pgfpathlineto{\pgfqpoint{0.000000in}{-0.044444in}}%
\pgfusepath{stroke,fill}%
}%
\begin{pgfscope}%
\pgfsys@transformshift{5.908800in}{4.083121in}%
\pgfsys@useobject{currentmarker}{}%
\end{pgfscope}%
\end{pgfscope}%
\begin{pgfscope}%
\pgfsetbuttcap%
\pgfsetroundjoin%
\definecolor{currentfill}{rgb}{0.150000,0.150000,0.150000}%
\pgfsetfillcolor{currentfill}%
\pgfsetlinewidth{0.803000pt}%
\definecolor{currentstroke}{rgb}{0.150000,0.150000,0.150000}%
\pgfsetstrokecolor{currentstroke}%
\pgfsetdash{}{0pt}%
\pgfsys@defobject{currentmarker}{\pgfqpoint{0.000000in}{-0.044444in}}{\pgfqpoint{0.000000in}{0.000000in}}{%
\pgfpathmoveto{\pgfqpoint{0.000000in}{0.000000in}}%
\pgfpathlineto{\pgfqpoint{0.000000in}{-0.044444in}}%
\pgfusepath{stroke,fill}%
}%
\begin{pgfscope}%
\pgfsys@transformshift{5.957391in}{4.083121in}%
\pgfsys@useobject{currentmarker}{}%
\end{pgfscope}%
\end{pgfscope}%
\begin{pgfscope}%
\pgfsetbuttcap%
\pgfsetroundjoin%
\definecolor{currentfill}{rgb}{0.150000,0.150000,0.150000}%
\pgfsetfillcolor{currentfill}%
\pgfsetlinewidth{0.803000pt}%
\definecolor{currentstroke}{rgb}{0.150000,0.150000,0.150000}%
\pgfsetstrokecolor{currentstroke}%
\pgfsetdash{}{0pt}%
\pgfsys@defobject{currentmarker}{\pgfqpoint{0.000000in}{-0.044444in}}{\pgfqpoint{0.000000in}{0.000000in}}{%
\pgfpathmoveto{\pgfqpoint{0.000000in}{0.000000in}}%
\pgfpathlineto{\pgfqpoint{0.000000in}{-0.044444in}}%
\pgfusepath{stroke,fill}%
}%
\begin{pgfscope}%
\pgfsys@transformshift{5.997094in}{4.083121in}%
\pgfsys@useobject{currentmarker}{}%
\end{pgfscope}%
\end{pgfscope}%
\begin{pgfscope}%
\pgfsetbuttcap%
\pgfsetroundjoin%
\definecolor{currentfill}{rgb}{0.150000,0.150000,0.150000}%
\pgfsetfillcolor{currentfill}%
\pgfsetlinewidth{0.803000pt}%
\definecolor{currentstroke}{rgb}{0.150000,0.150000,0.150000}%
\pgfsetstrokecolor{currentstroke}%
\pgfsetdash{}{0pt}%
\pgfsys@defobject{currentmarker}{\pgfqpoint{0.000000in}{-0.044444in}}{\pgfqpoint{0.000000in}{0.000000in}}{%
\pgfpathmoveto{\pgfqpoint{0.000000in}{0.000000in}}%
\pgfpathlineto{\pgfqpoint{0.000000in}{-0.044444in}}%
\pgfusepath{stroke,fill}%
}%
\begin{pgfscope}%
\pgfsys@transformshift{6.030661in}{4.083121in}%
\pgfsys@useobject{currentmarker}{}%
\end{pgfscope}%
\end{pgfscope}%
\begin{pgfscope}%
\pgfsetbuttcap%
\pgfsetroundjoin%
\definecolor{currentfill}{rgb}{0.150000,0.150000,0.150000}%
\pgfsetfillcolor{currentfill}%
\pgfsetlinewidth{0.803000pt}%
\definecolor{currentstroke}{rgb}{0.150000,0.150000,0.150000}%
\pgfsetstrokecolor{currentstroke}%
\pgfsetdash{}{0pt}%
\pgfsys@defobject{currentmarker}{\pgfqpoint{0.000000in}{-0.044444in}}{\pgfqpoint{0.000000in}{0.000000in}}{%
\pgfpathmoveto{\pgfqpoint{0.000000in}{0.000000in}}%
\pgfpathlineto{\pgfqpoint{0.000000in}{-0.044444in}}%
\pgfusepath{stroke,fill}%
}%
\begin{pgfscope}%
\pgfsys@transformshift{6.059739in}{4.083121in}%
\pgfsys@useobject{currentmarker}{}%
\end{pgfscope}%
\end{pgfscope}%
\begin{pgfscope}%
\pgfsetbuttcap%
\pgfsetroundjoin%
\definecolor{currentfill}{rgb}{0.150000,0.150000,0.150000}%
\pgfsetfillcolor{currentfill}%
\pgfsetlinewidth{0.803000pt}%
\definecolor{currentstroke}{rgb}{0.150000,0.150000,0.150000}%
\pgfsetstrokecolor{currentstroke}%
\pgfsetdash{}{0pt}%
\pgfsys@defobject{currentmarker}{\pgfqpoint{0.000000in}{-0.044444in}}{\pgfqpoint{0.000000in}{0.000000in}}{%
\pgfpathmoveto{\pgfqpoint{0.000000in}{0.000000in}}%
\pgfpathlineto{\pgfqpoint{0.000000in}{-0.044444in}}%
\pgfusepath{stroke,fill}%
}%
\begin{pgfscope}%
\pgfsys@transformshift{6.085387in}{4.083121in}%
\pgfsys@useobject{currentmarker}{}%
\end{pgfscope}%
\end{pgfscope}%
\begin{pgfscope}%
\pgfsetbuttcap%
\pgfsetroundjoin%
\definecolor{currentfill}{rgb}{0.150000,0.150000,0.150000}%
\pgfsetfillcolor{currentfill}%
\pgfsetlinewidth{0.803000pt}%
\definecolor{currentstroke}{rgb}{0.150000,0.150000,0.150000}%
\pgfsetstrokecolor{currentstroke}%
\pgfsetdash{}{0pt}%
\pgfsys@defobject{currentmarker}{\pgfqpoint{0.000000in}{-0.044444in}}{\pgfqpoint{0.000000in}{0.000000in}}{%
\pgfpathmoveto{\pgfqpoint{0.000000in}{0.000000in}}%
\pgfpathlineto{\pgfqpoint{0.000000in}{-0.044444in}}%
\pgfusepath{stroke,fill}%
}%
\begin{pgfscope}%
\pgfsys@transformshift{6.259270in}{4.083121in}%
\pgfsys@useobject{currentmarker}{}%
\end{pgfscope}%
\end{pgfscope}%
\begin{pgfscope}%
\pgfsetbuttcap%
\pgfsetroundjoin%
\definecolor{currentfill}{rgb}{0.150000,0.150000,0.150000}%
\pgfsetfillcolor{currentfill}%
\pgfsetlinewidth{1.003750pt}%
\definecolor{currentstroke}{rgb}{0.150000,0.150000,0.150000}%
\pgfsetstrokecolor{currentstroke}%
\pgfsetdash{}{0pt}%
\pgfsys@defobject{currentmarker}{\pgfqpoint{-0.066667in}{0.000000in}}{\pgfqpoint{0.000000in}{0.000000in}}{%
\pgfpathmoveto{\pgfqpoint{0.000000in}{0.000000in}}%
\pgfpathlineto{\pgfqpoint{-0.066667in}{0.000000in}}%
\pgfusepath{stroke,fill}%
}%
\begin{pgfscope}%
\pgfsys@transformshift{5.105513in}{4.083121in}%
\pgfsys@useobject{currentmarker}{}%
\end{pgfscope}%
\end{pgfscope}%
\begin{pgfscope}%
\pgfsetbuttcap%
\pgfsetroundjoin%
\definecolor{currentfill}{rgb}{0.150000,0.150000,0.150000}%
\pgfsetfillcolor{currentfill}%
\pgfsetlinewidth{1.003750pt}%
\definecolor{currentstroke}{rgb}{0.150000,0.150000,0.150000}%
\pgfsetstrokecolor{currentstroke}%
\pgfsetdash{}{0pt}%
\pgfsys@defobject{currentmarker}{\pgfqpoint{-0.066667in}{0.000000in}}{\pgfqpoint{0.000000in}{0.000000in}}{%
\pgfpathmoveto{\pgfqpoint{0.000000in}{0.000000in}}%
\pgfpathlineto{\pgfqpoint{-0.066667in}{0.000000in}}%
\pgfusepath{stroke,fill}%
}%
\begin{pgfscope}%
\pgfsys@transformshift{5.105513in}{4.509594in}%
\pgfsys@useobject{currentmarker}{}%
\end{pgfscope}%
\end{pgfscope}%
\begin{pgfscope}%
\pgfsetbuttcap%
\pgfsetroundjoin%
\definecolor{currentfill}{rgb}{0.150000,0.150000,0.150000}%
\pgfsetfillcolor{currentfill}%
\pgfsetlinewidth{1.003750pt}%
\definecolor{currentstroke}{rgb}{0.150000,0.150000,0.150000}%
\pgfsetstrokecolor{currentstroke}%
\pgfsetdash{}{0pt}%
\pgfsys@defobject{currentmarker}{\pgfqpoint{-0.066667in}{0.000000in}}{\pgfqpoint{0.000000in}{0.000000in}}{%
\pgfpathmoveto{\pgfqpoint{0.000000in}{0.000000in}}%
\pgfpathlineto{\pgfqpoint{-0.066667in}{0.000000in}}%
\pgfusepath{stroke,fill}%
}%
\begin{pgfscope}%
\pgfsys@transformshift{5.105513in}{4.691069in}%
\pgfsys@useobject{currentmarker}{}%
\end{pgfscope}%
\end{pgfscope}%
\begin{pgfscope}%
\pgfpathrectangle{\pgfqpoint{5.105513in}{4.083121in}}{\pgfqpoint{1.223103in}{0.607948in}}%
\pgfusepath{clip}%
\pgfsetroundcap%
\pgfsetroundjoin%
\pgfsetlinewidth{1.204500pt}%
\definecolor{currentstroke}{rgb}{0.000000,0.501961,0.000000}%
\pgfsetstrokecolor{currentstroke}%
\pgfsetdash{}{0pt}%
\pgfpathmoveto{\pgfqpoint{5.105513in}{4.510352in}}%
\pgfpathlineto{\pgfqpoint{5.346222in}{4.512012in}}%
\pgfpathlineto{\pgfqpoint{5.457753in}{4.513649in}}%
\pgfpathlineto{\pgfqpoint{5.531149in}{4.515265in}}%
\pgfpathlineto{\pgfqpoint{5.585945in}{4.516864in}}%
\pgfpathlineto{\pgfqpoint{5.629687in}{4.518448in}}%
\pgfpathlineto{\pgfqpoint{5.666095in}{4.520017in}}%
\pgfpathlineto{\pgfqpoint{5.697279in}{4.521574in}}%
\pgfpathlineto{\pgfqpoint{5.724552in}{4.523120in}}%
\pgfpathlineto{\pgfqpoint{5.748786in}{4.524656in}}%
\pgfpathlineto{\pgfqpoint{5.770591in}{4.526184in}}%
\pgfpathlineto{\pgfqpoint{5.790410in}{4.527703in}}%
\pgfpathlineto{\pgfqpoint{5.808575in}{4.529216in}}%
\pgfpathlineto{\pgfqpoint{5.825341in}{4.530722in}}%
\pgfpathlineto{\pgfqpoint{5.840907in}{4.532221in}}%
\pgfpathlineto{\pgfqpoint{5.855435in}{4.533715in}}%
\pgfpathlineto{\pgfqpoint{5.869053in}{4.535203in}}%
\pgfpathlineto{\pgfqpoint{5.881870in}{4.536686in}}%
\pgfpathlineto{\pgfqpoint{5.893974in}{4.538163in}}%
\pgfpathlineto{\pgfqpoint{5.905441in}{4.539634in}}%
\pgfpathlineto{\pgfqpoint{5.916334in}{4.541100in}}%
\pgfpathlineto{\pgfqpoint{5.926708in}{4.542561in}}%
\pgfpathlineto{\pgfqpoint{5.936610in}{4.544016in}}%
\pgfpathlineto{\pgfqpoint{5.946082in}{4.545465in}}%
\pgfpathlineto{\pgfqpoint{5.955158in}{4.546908in}}%
\pgfpathlineto{\pgfqpoint{5.963871in}{4.548345in}}%
\pgfpathlineto{\pgfqpoint{5.972249in}{4.549777in}}%
\pgfpathlineto{\pgfqpoint{5.980317in}{4.551202in}}%
\pgfpathlineto{\pgfqpoint{5.988096in}{4.552622in}}%
\pgfpathlineto{\pgfqpoint{5.995607in}{4.554037in}}%
\pgfpathlineto{\pgfqpoint{6.002868in}{4.555446in}}%
\pgfpathlineto{\pgfqpoint{6.009894in}{4.556851in}}%
\pgfpathlineto{\pgfqpoint{6.016701in}{4.558252in}}%
\pgfpathlineto{\pgfqpoint{6.023301in}{4.559649in}}%
\pgfpathlineto{\pgfqpoint{6.029707in}{4.561044in}}%
\pgfpathlineto{\pgfqpoint{6.035930in}{4.562437in}}%
\pgfpathlineto{\pgfqpoint{6.041980in}{4.563830in}}%
\pgfpathlineto{\pgfqpoint{6.047867in}{4.565224in}}%
\pgfpathlineto{\pgfqpoint{6.053598in}{4.566621in}}%
\pgfpathlineto{\pgfqpoint{6.059183in}{4.568022in}}%
\pgfpathlineto{\pgfqpoint{6.064628in}{4.569430in}}%
\pgfpathlineto{\pgfqpoint{6.069940in}{4.570848in}}%
\pgfpathlineto{\pgfqpoint{6.075126in}{4.572277in}}%
\pgfpathlineto{\pgfqpoint{6.080191in}{4.573720in}}%
\pgfpathlineto{\pgfqpoint{6.085140in}{4.575182in}}%
\pgfpathlineto{\pgfqpoint{6.089980in}{4.576665in}}%
\pgfpathlineto{\pgfqpoint{6.094715in}{4.578173in}}%
\pgfpathlineto{\pgfqpoint{6.099349in}{4.579710in}}%
\pgfpathlineto{\pgfqpoint{6.103886in}{4.581282in}}%
\pgfpathlineto{\pgfqpoint{6.108331in}{4.582892in}}%
\pgfusepath{stroke}%
\end{pgfscope}%
\begin{pgfscope}%
\pgfsetrectcap%
\pgfsetmiterjoin%
\pgfsetlinewidth{1.003750pt}%
\definecolor{currentstroke}{rgb}{0.150000,0.150000,0.150000}%
\pgfsetstrokecolor{currentstroke}%
\pgfsetdash{}{0pt}%
\pgfpathmoveto{\pgfqpoint{5.105513in}{4.083121in}}%
\pgfpathlineto{\pgfqpoint{5.105513in}{4.691069in}}%
\pgfusepath{stroke}%
\end{pgfscope}%
\begin{pgfscope}%
\pgfsetrectcap%
\pgfsetmiterjoin%
\pgfsetlinewidth{1.003750pt}%
\definecolor{currentstroke}{rgb}{0.150000,0.150000,0.150000}%
\pgfsetstrokecolor{currentstroke}%
\pgfsetdash{}{0pt}%
\pgfpathmoveto{\pgfqpoint{5.105513in}{4.083121in}}%
\pgfpathlineto{\pgfqpoint{6.328616in}{4.083121in}}%
\pgfusepath{stroke}%
\end{pgfscope}%
\begin{pgfscope}%
\pgfpathrectangle{\pgfqpoint{5.105513in}{4.083121in}}{\pgfqpoint{1.223103in}{0.607948in}}%
\pgfusepath{clip}%
\pgfsetbuttcap%
\pgfsetroundjoin%
\definecolor{currentfill}{rgb}{0.000000,0.000000,0.000000}%
\pgfsetfillcolor{currentfill}%
\pgfsetlinewidth{1.003750pt}%
\definecolor{currentstroke}{rgb}{0.000000,0.000000,0.000000}%
\pgfsetstrokecolor{currentstroke}%
\pgfsetdash{}{0pt}%
\pgfsys@defobject{currentmarker}{\pgfqpoint{-0.013889in}{-0.013889in}}{\pgfqpoint{0.013889in}{0.013889in}}{%
\pgfpathmoveto{\pgfqpoint{0.000000in}{-0.013889in}}%
\pgfpathcurveto{\pgfqpoint{0.003683in}{-0.013889in}}{\pgfqpoint{0.007216in}{-0.012425in}}{\pgfqpoint{0.009821in}{-0.009821in}}%
\pgfpathcurveto{\pgfqpoint{0.012425in}{-0.007216in}}{\pgfqpoint{0.013889in}{-0.003683in}}{\pgfqpoint{0.013889in}{0.000000in}}%
\pgfpathcurveto{\pgfqpoint{0.013889in}{0.003683in}}{\pgfqpoint{0.012425in}{0.007216in}}{\pgfqpoint{0.009821in}{0.009821in}}%
\pgfpathcurveto{\pgfqpoint{0.007216in}{0.012425in}}{\pgfqpoint{0.003683in}{0.013889in}}{\pgfqpoint{0.000000in}{0.013889in}}%
\pgfpathcurveto{\pgfqpoint{-0.003683in}{0.013889in}}{\pgfqpoint{-0.007216in}{0.012425in}}{\pgfqpoint{-0.009821in}{0.009821in}}%
\pgfpathcurveto{\pgfqpoint{-0.012425in}{0.007216in}}{\pgfqpoint{-0.013889in}{0.003683in}}{\pgfqpoint{-0.013889in}{0.000000in}}%
\pgfpathcurveto{\pgfqpoint{-0.013889in}{-0.003683in}}{\pgfqpoint{-0.012425in}{-0.007216in}}{\pgfqpoint{-0.009821in}{-0.009821in}}%
\pgfpathcurveto{\pgfqpoint{-0.007216in}{-0.012425in}}{\pgfqpoint{-0.003683in}{-0.013889in}}{\pgfqpoint{0.000000in}{-0.013889in}}%
\pgfpathclose%
\pgfusepath{stroke,fill}%
}%
\begin{pgfscope}%
\pgfsys@transformshift{5.757861in}{4.525262in}%
\pgfsys@useobject{currentmarker}{}%
\end{pgfscope}%
\begin{pgfscope}%
\pgfsys@transformshift{5.762260in}{4.525563in}%
\pgfsys@useobject{currentmarker}{}%
\end{pgfscope}%
\begin{pgfscope}%
\pgfsys@transformshift{5.766750in}{4.525896in}%
\pgfsys@useobject{currentmarker}{}%
\end{pgfscope}%
\begin{pgfscope}%
\pgfsys@transformshift{5.771335in}{4.526222in}%
\pgfsys@useobject{currentmarker}{}%
\end{pgfscope}%
\begin{pgfscope}%
\pgfsys@transformshift{5.776018in}{4.526584in}%
\pgfsys@useobject{currentmarker}{}%
\end{pgfscope}%
\begin{pgfscope}%
\pgfsys@transformshift{5.780804in}{4.526938in}%
\pgfsys@useobject{currentmarker}{}%
\end{pgfscope}%
\begin{pgfscope}%
\pgfsys@transformshift{5.785698in}{4.527332in}%
\pgfsys@useobject{currentmarker}{}%
\end{pgfscope}%
\begin{pgfscope}%
\pgfsys@transformshift{5.790704in}{4.527718in}%
\pgfsys@useobject{currentmarker}{}%
\end{pgfscope}%
\begin{pgfscope}%
\pgfsys@transformshift{5.795828in}{4.528150in}%
\pgfsys@useobject{currentmarker}{}%
\end{pgfscope}%
\begin{pgfscope}%
\pgfsys@transformshift{5.801075in}{4.528572in}%
\pgfsys@useobject{currentmarker}{}%
\end{pgfscope}%
\begin{pgfscope}%
\pgfsys@transformshift{5.806452in}{4.529047in}%
\pgfsys@useobject{currentmarker}{}%
\end{pgfscope}%
\begin{pgfscope}%
\pgfsys@transformshift{5.835530in}{4.531696in}%
\pgfsys@useobject{currentmarker}{}%
\end{pgfscope}%
\begin{pgfscope}%
\pgfsys@transformshift{5.869098in}{4.535255in}%
\pgfsys@useobject{currentmarker}{}%
\end{pgfscope}%
\begin{pgfscope}%
\pgfsys@transformshift{5.908800in}{4.540073in}%
\pgfsys@useobject{currentmarker}{}%
\end{pgfscope}%
\begin{pgfscope}%
\pgfsys@transformshift{5.957391in}{4.547363in}%
\pgfsys@useobject{currentmarker}{}%
\end{pgfscope}%
\begin{pgfscope}%
\pgfsys@transformshift{6.020037in}{4.558867in}%
\pgfsys@useobject{currentmarker}{}%
\end{pgfscope}%
\begin{pgfscope}%
\pgfsys@transformshift{6.108331in}{4.582911in}%
\pgfsys@useobject{currentmarker}{}%
\end{pgfscope}%
\end{pgfscope}%
\begin{pgfscope}%
\pgfsetbuttcap%
\pgfsetmiterjoin%
\definecolor{currentfill}{rgb}{1.000000,1.000000,1.000000}%
\pgfsetfillcolor{currentfill}%
\pgfsetlinewidth{0.803000pt}%
\definecolor{currentstroke}{rgb}{1.000000,1.000000,1.000000}%
\pgfsetstrokecolor{currentstroke}%
\pgfsetdash{}{0pt}%
\pgfpathmoveto{\pgfqpoint{6.297392in}{4.642193in}}%
\pgfpathlineto{\pgfqpoint{6.297392in}{4.131997in}}%
\pgfpathlineto{\pgfqpoint{6.478411in}{4.131997in}}%
\pgfpathlineto{\pgfqpoint{6.478411in}{4.642193in}}%
\pgfpathclose%
\pgfusepath{stroke,fill}%
\end{pgfscope}%
\begin{pgfscope}%
\definecolor{textcolor}{rgb}{0.150000,0.150000,0.150000}%
\pgfsetstrokecolor{textcolor}%
\pgfsetfillcolor{textcolor}%
\pgftext[x=6.368294in,y=4.586507in,left,base,rotate=270.000000]{\color{textcolor}\sffamily\fontsize{5.647059}{6.776471}\selectfont nlevel = 12}%
\end{pgfscope}%
\begin{pgfscope}%
\pgfsetbuttcap%
\pgfsetmiterjoin%
\definecolor{currentfill}{rgb}{1.000000,1.000000,1.000000}%
\pgfsetfillcolor{currentfill}%
\pgfsetlinewidth{0.803000pt}%
\definecolor{currentstroke}{rgb}{1.000000,1.000000,1.000000}%
\pgfsetstrokecolor{currentstroke}%
\pgfsetdash{}{0pt}%
\pgfpathmoveto{\pgfqpoint{6.297392in}{4.642193in}}%
\pgfpathlineto{\pgfqpoint{6.297392in}{4.131997in}}%
\pgfpathlineto{\pgfqpoint{6.478411in}{4.131997in}}%
\pgfpathlineto{\pgfqpoint{6.478411in}{4.642193in}}%
\pgfpathclose%
\pgfusepath{stroke,fill}%
\end{pgfscope}%
\begin{pgfscope}%
\definecolor{textcolor}{rgb}{0.150000,0.150000,0.150000}%
\pgfsetstrokecolor{textcolor}%
\pgfsetfillcolor{textcolor}%
\pgftext[x=6.368294in,y=4.586507in,left,base,rotate=270.000000]{\color{textcolor}\sffamily\fontsize{5.647059}{6.776471}\selectfont nlevel = 12}%
\end{pgfscope}%
\begin{pgfscope}%
\pgfsetbuttcap%
\pgfsetmiterjoin%
\definecolor{currentfill}{rgb}{1.000000,1.000000,1.000000}%
\pgfsetfillcolor{currentfill}%
\pgfsetlinewidth{0.000000pt}%
\definecolor{currentstroke}{rgb}{0.000000,0.000000,0.000000}%
\pgfsetstrokecolor{currentstroke}%
\pgfsetstrokeopacity{0.000000}%
\pgfsetdash{}{0pt}%
\pgfpathmoveto{\pgfqpoint{0.702340in}{3.353583in}}%
\pgfpathlineto{\pgfqpoint{1.925444in}{3.353583in}}%
\pgfpathlineto{\pgfqpoint{1.925444in}{3.961532in}}%
\pgfpathlineto{\pgfqpoint{0.702340in}{3.961532in}}%
\pgfpathclose%
\pgfusepath{fill}%
\end{pgfscope}%
\begin{pgfscope}%
\pgfpathrectangle{\pgfqpoint{0.702340in}{3.353583in}}{\pgfqpoint{1.223103in}{0.607948in}}%
\pgfusepath{clip}%
\pgfsetbuttcap%
\pgfsetmiterjoin%
\definecolor{currentfill}{rgb}{0.000000,0.000000,1.000000}%
\pgfsetfillcolor{currentfill}%
\pgfsetfillopacity{0.100000}%
\pgfsetlinewidth{0.803000pt}%
\definecolor{currentstroke}{rgb}{0.000000,0.000000,1.000000}%
\pgfsetstrokecolor{currentstroke}%
\pgfsetstrokeopacity{0.100000}%
\pgfsetdash{}{0pt}%
\pgfpathmoveto{\pgfqpoint{0.702340in}{3.675282in}}%
\pgfpathlineto{\pgfqpoint{0.702340in}{3.694622in}}%
\pgfpathlineto{\pgfqpoint{1.925444in}{3.694622in}}%
\pgfpathlineto{\pgfqpoint{1.925444in}{3.675282in}}%
\pgfpathclose%
\pgfusepath{stroke,fill}%
\end{pgfscope}%
\begin{pgfscope}%
\pgfpathrectangle{\pgfqpoint{0.702340in}{3.353583in}}{\pgfqpoint{1.223103in}{0.607948in}}%
\pgfusepath{clip}%
\pgfsetbuttcap%
\pgfsetroundjoin%
\definecolor{currentfill}{rgb}{0.000000,0.501961,0.000000}%
\pgfsetfillcolor{currentfill}%
\pgfsetfillopacity{0.500000}%
\pgfsetlinewidth{0.803000pt}%
\definecolor{currentstroke}{rgb}{0.000000,0.501961,0.000000}%
\pgfsetstrokecolor{currentstroke}%
\pgfsetstrokeopacity{0.500000}%
\pgfsetdash{}{0pt}%
\pgfpathmoveto{\pgfqpoint{0.702340in}{3.692802in}}%
\pgfpathlineto{\pgfqpoint{0.702340in}{3.675073in}}%
\pgfpathlineto{\pgfqpoint{0.943050in}{3.673800in}}%
\pgfpathlineto{\pgfqpoint{1.054581in}{3.671411in}}%
\pgfpathlineto{\pgfqpoint{1.127977in}{3.667930in}}%
\pgfpathlineto{\pgfqpoint{1.182772in}{3.663381in}}%
\pgfpathlineto{\pgfqpoint{1.226515in}{3.657788in}}%
\pgfpathlineto{\pgfqpoint{1.262923in}{3.651172in}}%
\pgfpathlineto{\pgfqpoint{1.294107in}{3.643557in}}%
\pgfpathlineto{\pgfqpoint{1.321380in}{3.634964in}}%
\pgfpathlineto{\pgfqpoint{1.345614in}{3.625414in}}%
\pgfpathlineto{\pgfqpoint{1.367419in}{3.614926in}}%
\pgfpathlineto{\pgfqpoint{1.387238in}{3.603303in}}%
\pgfpathlineto{\pgfqpoint{1.405403in}{3.590570in}}%
\pgfpathlineto{\pgfqpoint{1.422168in}{3.577080in}}%
\pgfpathlineto{\pgfqpoint{1.437735in}{3.562837in}}%
\pgfpathlineto{\pgfqpoint{1.452262in}{3.547842in}}%
\pgfpathlineto{\pgfqpoint{1.465881in}{3.532098in}}%
\pgfpathlineto{\pgfqpoint{1.478698in}{3.515607in}}%
\pgfpathlineto{\pgfqpoint{1.490802in}{3.498372in}}%
\pgfpathlineto{\pgfqpoint{1.502269in}{3.480394in}}%
\pgfpathlineto{\pgfqpoint{1.513162in}{3.461674in}}%
\pgfpathlineto{\pgfqpoint{1.523536in}{3.442212in}}%
\pgfpathlineto{\pgfqpoint{1.533438in}{3.422004in}}%
\pgfpathlineto{\pgfqpoint{1.542909in}{3.401041in}}%
\pgfpathlineto{\pgfqpoint{1.551986in}{3.379313in}}%
\pgfpathlineto{\pgfqpoint{1.560699in}{3.356824in}}%
\pgfpathlineto{\pgfqpoint{1.569077in}{3.333593in}}%
\pgfpathlineto{\pgfqpoint{1.577145in}{3.309640in}}%
\pgfpathlineto{\pgfqpoint{1.584924in}{3.284980in}}%
\pgfpathlineto{\pgfqpoint{1.592435in}{3.259619in}}%
\pgfpathlineto{\pgfqpoint{1.599696in}{3.233564in}}%
\pgfpathlineto{\pgfqpoint{1.606722in}{3.206817in}}%
\pgfpathlineto{\pgfqpoint{1.613528in}{3.179381in}}%
\pgfpathlineto{\pgfqpoint{1.620129in}{3.151256in}}%
\pgfpathlineto{\pgfqpoint{1.626535in}{3.122441in}}%
\pgfpathlineto{\pgfqpoint{1.632758in}{3.092937in}}%
\pgfpathlineto{\pgfqpoint{1.638808in}{3.062742in}}%
\pgfpathlineto{\pgfqpoint{1.644694in}{3.031856in}}%
\pgfpathlineto{\pgfqpoint{1.650426in}{3.000280in}}%
\pgfpathlineto{\pgfqpoint{1.656011in}{2.968013in}}%
\pgfpathlineto{\pgfqpoint{1.661456in}{2.935056in}}%
\pgfpathlineto{\pgfqpoint{1.666768in}{2.901411in}}%
\pgfpathlineto{\pgfqpoint{1.671953in}{2.867080in}}%
\pgfpathlineto{\pgfqpoint{1.677018in}{2.832066in}}%
\pgfpathlineto{\pgfqpoint{1.681968in}{2.796369in}}%
\pgfpathlineto{\pgfqpoint{1.686808in}{2.759993in}}%
\pgfpathlineto{\pgfqpoint{1.691543in}{2.722939in}}%
\pgfpathlineto{\pgfqpoint{1.696176in}{2.685206in}}%
\pgfpathlineto{\pgfqpoint{1.700714in}{2.646777in}}%
\pgfpathlineto{\pgfqpoint{1.705158in}{2.607498in}}%
\pgfpathlineto{\pgfqpoint{1.705158in}{2.608036in}}%
\pgfpathlineto{\pgfqpoint{1.705158in}{2.608036in}}%
\pgfpathlineto{\pgfqpoint{1.700714in}{2.647598in}}%
\pgfpathlineto{\pgfqpoint{1.696176in}{2.686510in}}%
\pgfpathlineto{\pgfqpoint{1.691543in}{2.724635in}}%
\pgfpathlineto{\pgfqpoint{1.686808in}{2.761971in}}%
\pgfpathlineto{\pgfqpoint{1.681968in}{2.798532in}}%
\pgfpathlineto{\pgfqpoint{1.677018in}{2.834334in}}%
\pgfpathlineto{\pgfqpoint{1.671953in}{2.869389in}}%
\pgfpathlineto{\pgfqpoint{1.666768in}{2.903711in}}%
\pgfpathlineto{\pgfqpoint{1.661456in}{2.937310in}}%
\pgfpathlineto{\pgfqpoint{1.656011in}{2.970197in}}%
\pgfpathlineto{\pgfqpoint{1.650426in}{3.002379in}}%
\pgfpathlineto{\pgfqpoint{1.644694in}{3.033862in}}%
\pgfpathlineto{\pgfqpoint{1.638808in}{3.064650in}}%
\pgfpathlineto{\pgfqpoint{1.632758in}{3.094746in}}%
\pgfpathlineto{\pgfqpoint{1.626535in}{3.124152in}}%
\pgfpathlineto{\pgfqpoint{1.620129in}{3.152865in}}%
\pgfpathlineto{\pgfqpoint{1.613528in}{3.180885in}}%
\pgfpathlineto{\pgfqpoint{1.606722in}{3.208209in}}%
\pgfpathlineto{\pgfqpoint{1.599696in}{3.234835in}}%
\pgfpathlineto{\pgfqpoint{1.592435in}{3.260759in}}%
\pgfpathlineto{\pgfqpoint{1.584924in}{3.285980in}}%
\pgfpathlineto{\pgfqpoint{1.577145in}{3.310497in}}%
\pgfpathlineto{\pgfqpoint{1.569077in}{3.334309in}}%
\pgfpathlineto{\pgfqpoint{1.560699in}{3.357422in}}%
\pgfpathlineto{\pgfqpoint{1.551986in}{3.379847in}}%
\pgfpathlineto{\pgfqpoint{1.542909in}{3.401591in}}%
\pgfpathlineto{\pgfqpoint{1.533438in}{3.422648in}}%
\pgfpathlineto{\pgfqpoint{1.523536in}{3.442994in}}%
\pgfpathlineto{\pgfqpoint{1.513162in}{3.462608in}}%
\pgfpathlineto{\pgfqpoint{1.502269in}{3.481473in}}%
\pgfpathlineto{\pgfqpoint{1.490802in}{3.499571in}}%
\pgfpathlineto{\pgfqpoint{1.478698in}{3.516888in}}%
\pgfpathlineto{\pgfqpoint{1.465881in}{3.533410in}}%
\pgfpathlineto{\pgfqpoint{1.452262in}{3.549121in}}%
\pgfpathlineto{\pgfqpoint{1.437735in}{3.564005in}}%
\pgfpathlineto{\pgfqpoint{1.422168in}{3.578046in}}%
\pgfpathlineto{\pgfqpoint{1.405403in}{3.591225in}}%
\pgfpathlineto{\pgfqpoint{1.387238in}{3.603527in}}%
\pgfpathlineto{\pgfqpoint{1.367419in}{3.615283in}}%
\pgfpathlineto{\pgfqpoint{1.345614in}{3.626497in}}%
\pgfpathlineto{\pgfqpoint{1.321380in}{3.636948in}}%
\pgfpathlineto{\pgfqpoint{1.294107in}{3.646634in}}%
\pgfpathlineto{\pgfqpoint{1.262923in}{3.655552in}}%
\pgfpathlineto{\pgfqpoint{1.226515in}{3.663699in}}%
\pgfpathlineto{\pgfqpoint{1.182772in}{3.671074in}}%
\pgfpathlineto{\pgfqpoint{1.127977in}{3.677673in}}%
\pgfpathlineto{\pgfqpoint{1.054581in}{3.683496in}}%
\pgfpathlineto{\pgfqpoint{0.943050in}{3.688539in}}%
\pgfpathlineto{\pgfqpoint{0.702340in}{3.692802in}}%
\pgfpathclose%
\pgfusepath{stroke,fill}%
\end{pgfscope}%
\begin{pgfscope}%
\pgfpathrectangle{\pgfqpoint{0.702340in}{3.353583in}}{\pgfqpoint{1.223103in}{0.607948in}}%
\pgfusepath{clip}%
\pgfsetroundcap%
\pgfsetroundjoin%
\pgfsetlinewidth{0.501875pt}%
\definecolor{currentstroke}{rgb}{0.000000,0.000000,1.000000}%
\pgfsetstrokecolor{currentstroke}%
\pgfsetstrokeopacity{0.800000}%
\pgfsetdash{}{0pt}%
\pgfpathmoveto{\pgfqpoint{0.702340in}{3.684952in}}%
\pgfpathlineto{\pgfqpoint{1.925444in}{3.684952in}}%
\pgfusepath{stroke}%
\end{pgfscope}%
\begin{pgfscope}%
\pgfpathrectangle{\pgfqpoint{0.702340in}{3.353583in}}{\pgfqpoint{1.223103in}{0.607948in}}%
\pgfusepath{clip}%
\pgfsetbuttcap%
\pgfsetroundjoin%
\pgfsetlinewidth{1.003750pt}%
\definecolor{currentstroke}{rgb}{0.000000,0.000000,0.000000}%
\pgfsetstrokecolor{currentstroke}%
\pgfsetdash{{3.700000pt}{1.600000pt}}{0.000000pt}%
\pgfpathmoveto{\pgfqpoint{0.702340in}{3.678551in}}%
\pgfpathlineto{\pgfqpoint{1.925444in}{3.678551in}}%
\pgfusepath{stroke}%
\end{pgfscope}%
\begin{pgfscope}%
\pgfsetroundcap%
\pgfsetroundjoin%
\pgfsetlinewidth{0.501875pt}%
\definecolor{currentstroke}{rgb}{0.000000,0.000000,1.000000}%
\pgfsetstrokecolor{currentstroke}%
\pgfsetstrokeopacity{0.800000}%
\pgfsetdash{}{0pt}%
\pgfpathmoveto{\pgfqpoint{1.503033in}{3.801457in}}%
\pgfpathquadraticcurveto{\pgfqpoint{1.439785in}{3.751784in}}{\pgfqpoint{1.376536in}{3.702111in}}%
\pgfusepath{stroke}%
\end{pgfscope}%
\begin{pgfscope}%
\pgfsetfillopacity{0.800000}%
\pgfsetstrokeopacity{0.800000}%
\definecolor{textcolor}{rgb}{0.000000,0.000000,1.000000}%
\pgfsetstrokecolor{textcolor}%
\pgfsetfillcolor{textcolor}%
\pgftext[x=1.442982in,y=3.867336in,left,base]{\color{textcolor}\sffamily\fontsize{5.647059}{6.776471}\selectfont 9.545(16)}%
\end{pgfscope}%
\begin{pgfscope}%
\pgfsetbuttcap%
\pgfsetroundjoin%
\definecolor{currentfill}{rgb}{0.150000,0.150000,0.150000}%
\pgfsetfillcolor{currentfill}%
\pgfsetlinewidth{1.003750pt}%
\definecolor{currentstroke}{rgb}{0.150000,0.150000,0.150000}%
\pgfsetstrokecolor{currentstroke}%
\pgfsetdash{}{0pt}%
\pgfsys@defobject{currentmarker}{\pgfqpoint{0.000000in}{-0.066667in}}{\pgfqpoint{0.000000in}{0.000000in}}{%
\pgfpathmoveto{\pgfqpoint{0.000000in}{0.000000in}}%
\pgfpathlineto{\pgfqpoint{0.000000in}{-0.066667in}}%
\pgfusepath{stroke,fill}%
}%
\begin{pgfscope}%
\pgfsys@transformshift{0.702340in}{3.353583in}%
\pgfsys@useobject{currentmarker}{}%
\end{pgfscope}%
\end{pgfscope}%
\begin{pgfscope}%
\pgfsetbuttcap%
\pgfsetroundjoin%
\definecolor{currentfill}{rgb}{0.150000,0.150000,0.150000}%
\pgfsetfillcolor{currentfill}%
\pgfsetlinewidth{1.003750pt}%
\definecolor{currentstroke}{rgb}{0.150000,0.150000,0.150000}%
\pgfsetstrokecolor{currentstroke}%
\pgfsetdash{}{0pt}%
\pgfsys@defobject{currentmarker}{\pgfqpoint{0.000000in}{-0.066667in}}{\pgfqpoint{0.000000in}{0.000000in}}{%
\pgfpathmoveto{\pgfqpoint{0.000000in}{0.000000in}}%
\pgfpathlineto{\pgfqpoint{0.000000in}{-0.066667in}}%
\pgfusepath{stroke,fill}%
}%
\begin{pgfscope}%
\pgfsys@transformshift{1.203749in}{3.353583in}%
\pgfsys@useobject{currentmarker}{}%
\end{pgfscope}%
\end{pgfscope}%
\begin{pgfscope}%
\pgfsetbuttcap%
\pgfsetroundjoin%
\definecolor{currentfill}{rgb}{0.150000,0.150000,0.150000}%
\pgfsetfillcolor{currentfill}%
\pgfsetlinewidth{1.003750pt}%
\definecolor{currentstroke}{rgb}{0.150000,0.150000,0.150000}%
\pgfsetstrokecolor{currentstroke}%
\pgfsetdash{}{0pt}%
\pgfsys@defobject{currentmarker}{\pgfqpoint{0.000000in}{-0.066667in}}{\pgfqpoint{0.000000in}{0.000000in}}{%
\pgfpathmoveto{\pgfqpoint{0.000000in}{0.000000in}}%
\pgfpathlineto{\pgfqpoint{0.000000in}{-0.066667in}}%
\pgfusepath{stroke,fill}%
}%
\begin{pgfscope}%
\pgfsys@transformshift{1.705158in}{3.353583in}%
\pgfsys@useobject{currentmarker}{}%
\end{pgfscope}%
\end{pgfscope}%
\begin{pgfscope}%
\pgfsetbuttcap%
\pgfsetroundjoin%
\definecolor{currentfill}{rgb}{0.150000,0.150000,0.150000}%
\pgfsetfillcolor{currentfill}%
\pgfsetlinewidth{0.803000pt}%
\definecolor{currentstroke}{rgb}{0.150000,0.150000,0.150000}%
\pgfsetstrokecolor{currentstroke}%
\pgfsetdash{}{0pt}%
\pgfsys@defobject{currentmarker}{\pgfqpoint{0.000000in}{-0.044444in}}{\pgfqpoint{0.000000in}{0.000000in}}{%
\pgfpathmoveto{\pgfqpoint{0.000000in}{0.000000in}}%
\pgfpathlineto{\pgfqpoint{0.000000in}{-0.044444in}}%
\pgfusepath{stroke,fill}%
}%
\begin{pgfscope}%
\pgfsys@transformshift{0.853280in}{3.353583in}%
\pgfsys@useobject{currentmarker}{}%
\end{pgfscope}%
\end{pgfscope}%
\begin{pgfscope}%
\pgfsetbuttcap%
\pgfsetroundjoin%
\definecolor{currentfill}{rgb}{0.150000,0.150000,0.150000}%
\pgfsetfillcolor{currentfill}%
\pgfsetlinewidth{0.803000pt}%
\definecolor{currentstroke}{rgb}{0.150000,0.150000,0.150000}%
\pgfsetstrokecolor{currentstroke}%
\pgfsetdash{}{0pt}%
\pgfsys@defobject{currentmarker}{\pgfqpoint{0.000000in}{-0.044444in}}{\pgfqpoint{0.000000in}{0.000000in}}{%
\pgfpathmoveto{\pgfqpoint{0.000000in}{0.000000in}}%
\pgfpathlineto{\pgfqpoint{0.000000in}{-0.044444in}}%
\pgfusepath{stroke,fill}%
}%
\begin{pgfscope}%
\pgfsys@transformshift{0.941573in}{3.353583in}%
\pgfsys@useobject{currentmarker}{}%
\end{pgfscope}%
\end{pgfscope}%
\begin{pgfscope}%
\pgfsetbuttcap%
\pgfsetroundjoin%
\definecolor{currentfill}{rgb}{0.150000,0.150000,0.150000}%
\pgfsetfillcolor{currentfill}%
\pgfsetlinewidth{0.803000pt}%
\definecolor{currentstroke}{rgb}{0.150000,0.150000,0.150000}%
\pgfsetstrokecolor{currentstroke}%
\pgfsetdash{}{0pt}%
\pgfsys@defobject{currentmarker}{\pgfqpoint{0.000000in}{-0.044444in}}{\pgfqpoint{0.000000in}{0.000000in}}{%
\pgfpathmoveto{\pgfqpoint{0.000000in}{0.000000in}}%
\pgfpathlineto{\pgfqpoint{0.000000in}{-0.044444in}}%
\pgfusepath{stroke,fill}%
}%
\begin{pgfscope}%
\pgfsys@transformshift{1.004219in}{3.353583in}%
\pgfsys@useobject{currentmarker}{}%
\end{pgfscope}%
\end{pgfscope}%
\begin{pgfscope}%
\pgfsetbuttcap%
\pgfsetroundjoin%
\definecolor{currentfill}{rgb}{0.150000,0.150000,0.150000}%
\pgfsetfillcolor{currentfill}%
\pgfsetlinewidth{0.803000pt}%
\definecolor{currentstroke}{rgb}{0.150000,0.150000,0.150000}%
\pgfsetstrokecolor{currentstroke}%
\pgfsetdash{}{0pt}%
\pgfsys@defobject{currentmarker}{\pgfqpoint{0.000000in}{-0.044444in}}{\pgfqpoint{0.000000in}{0.000000in}}{%
\pgfpathmoveto{\pgfqpoint{0.000000in}{0.000000in}}%
\pgfpathlineto{\pgfqpoint{0.000000in}{-0.044444in}}%
\pgfusepath{stroke,fill}%
}%
\begin{pgfscope}%
\pgfsys@transformshift{1.052810in}{3.353583in}%
\pgfsys@useobject{currentmarker}{}%
\end{pgfscope}%
\end{pgfscope}%
\begin{pgfscope}%
\pgfsetbuttcap%
\pgfsetroundjoin%
\definecolor{currentfill}{rgb}{0.150000,0.150000,0.150000}%
\pgfsetfillcolor{currentfill}%
\pgfsetlinewidth{0.803000pt}%
\definecolor{currentstroke}{rgb}{0.150000,0.150000,0.150000}%
\pgfsetstrokecolor{currentstroke}%
\pgfsetdash{}{0pt}%
\pgfsys@defobject{currentmarker}{\pgfqpoint{0.000000in}{-0.044444in}}{\pgfqpoint{0.000000in}{0.000000in}}{%
\pgfpathmoveto{\pgfqpoint{0.000000in}{0.000000in}}%
\pgfpathlineto{\pgfqpoint{0.000000in}{-0.044444in}}%
\pgfusepath{stroke,fill}%
}%
\begin{pgfscope}%
\pgfsys@transformshift{1.092512in}{3.353583in}%
\pgfsys@useobject{currentmarker}{}%
\end{pgfscope}%
\end{pgfscope}%
\begin{pgfscope}%
\pgfsetbuttcap%
\pgfsetroundjoin%
\definecolor{currentfill}{rgb}{0.150000,0.150000,0.150000}%
\pgfsetfillcolor{currentfill}%
\pgfsetlinewidth{0.803000pt}%
\definecolor{currentstroke}{rgb}{0.150000,0.150000,0.150000}%
\pgfsetstrokecolor{currentstroke}%
\pgfsetdash{}{0pt}%
\pgfsys@defobject{currentmarker}{\pgfqpoint{0.000000in}{-0.044444in}}{\pgfqpoint{0.000000in}{0.000000in}}{%
\pgfpathmoveto{\pgfqpoint{0.000000in}{0.000000in}}%
\pgfpathlineto{\pgfqpoint{0.000000in}{-0.044444in}}%
\pgfusepath{stroke,fill}%
}%
\begin{pgfscope}%
\pgfsys@transformshift{1.126080in}{3.353583in}%
\pgfsys@useobject{currentmarker}{}%
\end{pgfscope}%
\end{pgfscope}%
\begin{pgfscope}%
\pgfsetbuttcap%
\pgfsetroundjoin%
\definecolor{currentfill}{rgb}{0.150000,0.150000,0.150000}%
\pgfsetfillcolor{currentfill}%
\pgfsetlinewidth{0.803000pt}%
\definecolor{currentstroke}{rgb}{0.150000,0.150000,0.150000}%
\pgfsetstrokecolor{currentstroke}%
\pgfsetdash{}{0pt}%
\pgfsys@defobject{currentmarker}{\pgfqpoint{0.000000in}{-0.044444in}}{\pgfqpoint{0.000000in}{0.000000in}}{%
\pgfpathmoveto{\pgfqpoint{0.000000in}{0.000000in}}%
\pgfpathlineto{\pgfqpoint{0.000000in}{-0.044444in}}%
\pgfusepath{stroke,fill}%
}%
\begin{pgfscope}%
\pgfsys@transformshift{1.155158in}{3.353583in}%
\pgfsys@useobject{currentmarker}{}%
\end{pgfscope}%
\end{pgfscope}%
\begin{pgfscope}%
\pgfsetbuttcap%
\pgfsetroundjoin%
\definecolor{currentfill}{rgb}{0.150000,0.150000,0.150000}%
\pgfsetfillcolor{currentfill}%
\pgfsetlinewidth{0.803000pt}%
\definecolor{currentstroke}{rgb}{0.150000,0.150000,0.150000}%
\pgfsetstrokecolor{currentstroke}%
\pgfsetdash{}{0pt}%
\pgfsys@defobject{currentmarker}{\pgfqpoint{0.000000in}{-0.044444in}}{\pgfqpoint{0.000000in}{0.000000in}}{%
\pgfpathmoveto{\pgfqpoint{0.000000in}{0.000000in}}%
\pgfpathlineto{\pgfqpoint{0.000000in}{-0.044444in}}%
\pgfusepath{stroke,fill}%
}%
\begin{pgfscope}%
\pgfsys@transformshift{1.180806in}{3.353583in}%
\pgfsys@useobject{currentmarker}{}%
\end{pgfscope}%
\end{pgfscope}%
\begin{pgfscope}%
\pgfsetbuttcap%
\pgfsetroundjoin%
\definecolor{currentfill}{rgb}{0.150000,0.150000,0.150000}%
\pgfsetfillcolor{currentfill}%
\pgfsetlinewidth{0.803000pt}%
\definecolor{currentstroke}{rgb}{0.150000,0.150000,0.150000}%
\pgfsetstrokecolor{currentstroke}%
\pgfsetdash{}{0pt}%
\pgfsys@defobject{currentmarker}{\pgfqpoint{0.000000in}{-0.044444in}}{\pgfqpoint{0.000000in}{0.000000in}}{%
\pgfpathmoveto{\pgfqpoint{0.000000in}{0.000000in}}%
\pgfpathlineto{\pgfqpoint{0.000000in}{-0.044444in}}%
\pgfusepath{stroke,fill}%
}%
\begin{pgfscope}%
\pgfsys@transformshift{1.354689in}{3.353583in}%
\pgfsys@useobject{currentmarker}{}%
\end{pgfscope}%
\end{pgfscope}%
\begin{pgfscope}%
\pgfsetbuttcap%
\pgfsetroundjoin%
\definecolor{currentfill}{rgb}{0.150000,0.150000,0.150000}%
\pgfsetfillcolor{currentfill}%
\pgfsetlinewidth{0.803000pt}%
\definecolor{currentstroke}{rgb}{0.150000,0.150000,0.150000}%
\pgfsetstrokecolor{currentstroke}%
\pgfsetdash{}{0pt}%
\pgfsys@defobject{currentmarker}{\pgfqpoint{0.000000in}{-0.044444in}}{\pgfqpoint{0.000000in}{0.000000in}}{%
\pgfpathmoveto{\pgfqpoint{0.000000in}{0.000000in}}%
\pgfpathlineto{\pgfqpoint{0.000000in}{-0.044444in}}%
\pgfusepath{stroke,fill}%
}%
\begin{pgfscope}%
\pgfsys@transformshift{1.442982in}{3.353583in}%
\pgfsys@useobject{currentmarker}{}%
\end{pgfscope}%
\end{pgfscope}%
\begin{pgfscope}%
\pgfsetbuttcap%
\pgfsetroundjoin%
\definecolor{currentfill}{rgb}{0.150000,0.150000,0.150000}%
\pgfsetfillcolor{currentfill}%
\pgfsetlinewidth{0.803000pt}%
\definecolor{currentstroke}{rgb}{0.150000,0.150000,0.150000}%
\pgfsetstrokecolor{currentstroke}%
\pgfsetdash{}{0pt}%
\pgfsys@defobject{currentmarker}{\pgfqpoint{0.000000in}{-0.044444in}}{\pgfqpoint{0.000000in}{0.000000in}}{%
\pgfpathmoveto{\pgfqpoint{0.000000in}{0.000000in}}%
\pgfpathlineto{\pgfqpoint{0.000000in}{-0.044444in}}%
\pgfusepath{stroke,fill}%
}%
\begin{pgfscope}%
\pgfsys@transformshift{1.505628in}{3.353583in}%
\pgfsys@useobject{currentmarker}{}%
\end{pgfscope}%
\end{pgfscope}%
\begin{pgfscope}%
\pgfsetbuttcap%
\pgfsetroundjoin%
\definecolor{currentfill}{rgb}{0.150000,0.150000,0.150000}%
\pgfsetfillcolor{currentfill}%
\pgfsetlinewidth{0.803000pt}%
\definecolor{currentstroke}{rgb}{0.150000,0.150000,0.150000}%
\pgfsetstrokecolor{currentstroke}%
\pgfsetdash{}{0pt}%
\pgfsys@defobject{currentmarker}{\pgfqpoint{0.000000in}{-0.044444in}}{\pgfqpoint{0.000000in}{0.000000in}}{%
\pgfpathmoveto{\pgfqpoint{0.000000in}{0.000000in}}%
\pgfpathlineto{\pgfqpoint{0.000000in}{-0.044444in}}%
\pgfusepath{stroke,fill}%
}%
\begin{pgfscope}%
\pgfsys@transformshift{1.554219in}{3.353583in}%
\pgfsys@useobject{currentmarker}{}%
\end{pgfscope}%
\end{pgfscope}%
\begin{pgfscope}%
\pgfsetbuttcap%
\pgfsetroundjoin%
\definecolor{currentfill}{rgb}{0.150000,0.150000,0.150000}%
\pgfsetfillcolor{currentfill}%
\pgfsetlinewidth{0.803000pt}%
\definecolor{currentstroke}{rgb}{0.150000,0.150000,0.150000}%
\pgfsetstrokecolor{currentstroke}%
\pgfsetdash{}{0pt}%
\pgfsys@defobject{currentmarker}{\pgfqpoint{0.000000in}{-0.044444in}}{\pgfqpoint{0.000000in}{0.000000in}}{%
\pgfpathmoveto{\pgfqpoint{0.000000in}{0.000000in}}%
\pgfpathlineto{\pgfqpoint{0.000000in}{-0.044444in}}%
\pgfusepath{stroke,fill}%
}%
\begin{pgfscope}%
\pgfsys@transformshift{1.593921in}{3.353583in}%
\pgfsys@useobject{currentmarker}{}%
\end{pgfscope}%
\end{pgfscope}%
\begin{pgfscope}%
\pgfsetbuttcap%
\pgfsetroundjoin%
\definecolor{currentfill}{rgb}{0.150000,0.150000,0.150000}%
\pgfsetfillcolor{currentfill}%
\pgfsetlinewidth{0.803000pt}%
\definecolor{currentstroke}{rgb}{0.150000,0.150000,0.150000}%
\pgfsetstrokecolor{currentstroke}%
\pgfsetdash{}{0pt}%
\pgfsys@defobject{currentmarker}{\pgfqpoint{0.000000in}{-0.044444in}}{\pgfqpoint{0.000000in}{0.000000in}}{%
\pgfpathmoveto{\pgfqpoint{0.000000in}{0.000000in}}%
\pgfpathlineto{\pgfqpoint{0.000000in}{-0.044444in}}%
\pgfusepath{stroke,fill}%
}%
\begin{pgfscope}%
\pgfsys@transformshift{1.627489in}{3.353583in}%
\pgfsys@useobject{currentmarker}{}%
\end{pgfscope}%
\end{pgfscope}%
\begin{pgfscope}%
\pgfsetbuttcap%
\pgfsetroundjoin%
\definecolor{currentfill}{rgb}{0.150000,0.150000,0.150000}%
\pgfsetfillcolor{currentfill}%
\pgfsetlinewidth{0.803000pt}%
\definecolor{currentstroke}{rgb}{0.150000,0.150000,0.150000}%
\pgfsetstrokecolor{currentstroke}%
\pgfsetdash{}{0pt}%
\pgfsys@defobject{currentmarker}{\pgfqpoint{0.000000in}{-0.044444in}}{\pgfqpoint{0.000000in}{0.000000in}}{%
\pgfpathmoveto{\pgfqpoint{0.000000in}{0.000000in}}%
\pgfpathlineto{\pgfqpoint{0.000000in}{-0.044444in}}%
\pgfusepath{stroke,fill}%
}%
\begin{pgfscope}%
\pgfsys@transformshift{1.656567in}{3.353583in}%
\pgfsys@useobject{currentmarker}{}%
\end{pgfscope}%
\end{pgfscope}%
\begin{pgfscope}%
\pgfsetbuttcap%
\pgfsetroundjoin%
\definecolor{currentfill}{rgb}{0.150000,0.150000,0.150000}%
\pgfsetfillcolor{currentfill}%
\pgfsetlinewidth{0.803000pt}%
\definecolor{currentstroke}{rgb}{0.150000,0.150000,0.150000}%
\pgfsetstrokecolor{currentstroke}%
\pgfsetdash{}{0pt}%
\pgfsys@defobject{currentmarker}{\pgfqpoint{0.000000in}{-0.044444in}}{\pgfqpoint{0.000000in}{0.000000in}}{%
\pgfpathmoveto{\pgfqpoint{0.000000in}{0.000000in}}%
\pgfpathlineto{\pgfqpoint{0.000000in}{-0.044444in}}%
\pgfusepath{stroke,fill}%
}%
\begin{pgfscope}%
\pgfsys@transformshift{1.682215in}{3.353583in}%
\pgfsys@useobject{currentmarker}{}%
\end{pgfscope}%
\end{pgfscope}%
\begin{pgfscope}%
\pgfsetbuttcap%
\pgfsetroundjoin%
\definecolor{currentfill}{rgb}{0.150000,0.150000,0.150000}%
\pgfsetfillcolor{currentfill}%
\pgfsetlinewidth{0.803000pt}%
\definecolor{currentstroke}{rgb}{0.150000,0.150000,0.150000}%
\pgfsetstrokecolor{currentstroke}%
\pgfsetdash{}{0pt}%
\pgfsys@defobject{currentmarker}{\pgfqpoint{0.000000in}{-0.044444in}}{\pgfqpoint{0.000000in}{0.000000in}}{%
\pgfpathmoveto{\pgfqpoint{0.000000in}{0.000000in}}%
\pgfpathlineto{\pgfqpoint{0.000000in}{-0.044444in}}%
\pgfusepath{stroke,fill}%
}%
\begin{pgfscope}%
\pgfsys@transformshift{1.856098in}{3.353583in}%
\pgfsys@useobject{currentmarker}{}%
\end{pgfscope}%
\end{pgfscope}%
\begin{pgfscope}%
\pgfsetbuttcap%
\pgfsetroundjoin%
\definecolor{currentfill}{rgb}{0.150000,0.150000,0.150000}%
\pgfsetfillcolor{currentfill}%
\pgfsetlinewidth{1.003750pt}%
\definecolor{currentstroke}{rgb}{0.150000,0.150000,0.150000}%
\pgfsetstrokecolor{currentstroke}%
\pgfsetdash{}{0pt}%
\pgfsys@defobject{currentmarker}{\pgfqpoint{-0.066667in}{0.000000in}}{\pgfqpoint{0.000000in}{0.000000in}}{%
\pgfpathmoveto{\pgfqpoint{0.000000in}{0.000000in}}%
\pgfpathlineto{\pgfqpoint{-0.066667in}{0.000000in}}%
\pgfusepath{stroke,fill}%
}%
\begin{pgfscope}%
\pgfsys@transformshift{0.702340in}{3.353583in}%
\pgfsys@useobject{currentmarker}{}%
\end{pgfscope}%
\end{pgfscope}%
\begin{pgfscope}%
\definecolor{textcolor}{rgb}{0.150000,0.150000,0.150000}%
\pgfsetstrokecolor{textcolor}%
\pgfsetfillcolor{textcolor}%
\pgftext[x=0.413148in,y=3.328636in,left,base]{\color{textcolor}\sffamily\fontsize{5.176471}{6.211765}\selectfont 9.000}%
\end{pgfscope}%
\begin{pgfscope}%
\pgfsetbuttcap%
\pgfsetroundjoin%
\definecolor{currentfill}{rgb}{0.150000,0.150000,0.150000}%
\pgfsetfillcolor{currentfill}%
\pgfsetlinewidth{1.003750pt}%
\definecolor{currentstroke}{rgb}{0.150000,0.150000,0.150000}%
\pgfsetstrokecolor{currentstroke}%
\pgfsetdash{}{0pt}%
\pgfsys@defobject{currentmarker}{\pgfqpoint{-0.066667in}{0.000000in}}{\pgfqpoint{0.000000in}{0.000000in}}{%
\pgfpathmoveto{\pgfqpoint{0.000000in}{0.000000in}}%
\pgfpathlineto{\pgfqpoint{-0.066667in}{0.000000in}}%
\pgfusepath{stroke,fill}%
}%
\begin{pgfscope}%
\pgfsys@transformshift{0.702340in}{3.678551in}%
\pgfsys@useobject{currentmarker}{}%
\end{pgfscope}%
\end{pgfscope}%
\begin{pgfscope}%
\definecolor{textcolor}{rgb}{0.150000,0.150000,0.150000}%
\pgfsetstrokecolor{textcolor}%
\pgfsetfillcolor{textcolor}%
\pgftext[x=0.413148in,y=3.653603in,left,base]{\color{textcolor}\sffamily\fontsize{5.176471}{6.211765}\selectfont 9.535}%
\end{pgfscope}%
\begin{pgfscope}%
\pgfsetbuttcap%
\pgfsetroundjoin%
\definecolor{currentfill}{rgb}{0.150000,0.150000,0.150000}%
\pgfsetfillcolor{currentfill}%
\pgfsetlinewidth{1.003750pt}%
\definecolor{currentstroke}{rgb}{0.150000,0.150000,0.150000}%
\pgfsetstrokecolor{currentstroke}%
\pgfsetdash{}{0pt}%
\pgfsys@defobject{currentmarker}{\pgfqpoint{-0.066667in}{0.000000in}}{\pgfqpoint{0.000000in}{0.000000in}}{%
\pgfpathmoveto{\pgfqpoint{0.000000in}{0.000000in}}%
\pgfpathlineto{\pgfqpoint{-0.066667in}{0.000000in}}%
\pgfusepath{stroke,fill}%
}%
\begin{pgfscope}%
\pgfsys@transformshift{0.702340in}{3.961532in}%
\pgfsys@useobject{currentmarker}{}%
\end{pgfscope}%
\end{pgfscope}%
\begin{pgfscope}%
\definecolor{textcolor}{rgb}{0.150000,0.150000,0.150000}%
\pgfsetstrokecolor{textcolor}%
\pgfsetfillcolor{textcolor}%
\pgftext[x=0.374971in,y=3.936584in,left,base]{\color{textcolor}\sffamily\fontsize{5.176471}{6.211765}\selectfont 10.000}%
\end{pgfscope}%
\begin{pgfscope}%
\definecolor{textcolor}{rgb}{0.150000,0.150000,0.150000}%
\pgfsetstrokecolor{textcolor}%
\pgfsetfillcolor{textcolor}%
\pgftext[x=0.319416in,y=3.657557in,,bottom,rotate=90.000000]{\color{textcolor}\sffamily\fontsize{5.647059}{6.776471}\selectfont \(\displaystyle x = \frac{2 \mu E L^2}{4 \pi^2}\)}%
\end{pgfscope}%
\begin{pgfscope}%
\pgfpathrectangle{\pgfqpoint{0.702340in}{3.353583in}}{\pgfqpoint{1.223103in}{0.607948in}}%
\pgfusepath{clip}%
\pgfsetroundcap%
\pgfsetroundjoin%
\pgfsetlinewidth{1.204500pt}%
\definecolor{currentstroke}{rgb}{0.000000,0.501961,0.000000}%
\pgfsetstrokecolor{currentstroke}%
\pgfsetdash{}{0pt}%
\pgfpathmoveto{\pgfqpoint{0.702340in}{3.683937in}}%
\pgfpathlineto{\pgfqpoint{0.943050in}{3.681170in}}%
\pgfpathlineto{\pgfqpoint{1.054581in}{3.677454in}}%
\pgfpathlineto{\pgfqpoint{1.127977in}{3.672802in}}%
\pgfpathlineto{\pgfqpoint{1.182772in}{3.667227in}}%
\pgfpathlineto{\pgfqpoint{1.226515in}{3.660743in}}%
\pgfpathlineto{\pgfqpoint{1.262923in}{3.653362in}}%
\pgfpathlineto{\pgfqpoint{1.294107in}{3.645096in}}%
\pgfpathlineto{\pgfqpoint{1.321380in}{3.635956in}}%
\pgfpathlineto{\pgfqpoint{1.345614in}{3.625955in}}%
\pgfpathlineto{\pgfqpoint{1.367419in}{3.615105in}}%
\pgfpathlineto{\pgfqpoint{1.387238in}{3.603415in}}%
\pgfpathlineto{\pgfqpoint{1.405403in}{3.590898in}}%
\pgfpathlineto{\pgfqpoint{1.422168in}{3.577563in}}%
\pgfpathlineto{\pgfqpoint{1.437735in}{3.563421in}}%
\pgfpathlineto{\pgfqpoint{1.452262in}{3.548481in}}%
\pgfpathlineto{\pgfqpoint{1.465881in}{3.532754in}}%
\pgfpathlineto{\pgfqpoint{1.478698in}{3.516248in}}%
\pgfpathlineto{\pgfqpoint{1.490802in}{3.498971in}}%
\pgfpathlineto{\pgfqpoint{1.502269in}{3.480933in}}%
\pgfpathlineto{\pgfqpoint{1.513162in}{3.462141in}}%
\pgfpathlineto{\pgfqpoint{1.523536in}{3.442603in}}%
\pgfpathlineto{\pgfqpoint{1.533438in}{3.422326in}}%
\pgfpathlineto{\pgfqpoint{1.542909in}{3.401316in}}%
\pgfpathlineto{\pgfqpoint{1.551986in}{3.379580in}}%
\pgfpathlineto{\pgfqpoint{1.560699in}{3.357123in}}%
\pgfpathlineto{\pgfqpoint{1.563184in}{3.350250in}}%
\pgfusepath{stroke}%
\end{pgfscope}%
\begin{pgfscope}%
\pgfsetrectcap%
\pgfsetmiterjoin%
\pgfsetlinewidth{1.003750pt}%
\definecolor{currentstroke}{rgb}{0.150000,0.150000,0.150000}%
\pgfsetstrokecolor{currentstroke}%
\pgfsetdash{}{0pt}%
\pgfpathmoveto{\pgfqpoint{0.702340in}{3.353583in}}%
\pgfpathlineto{\pgfqpoint{0.702340in}{3.961532in}}%
\pgfusepath{stroke}%
\end{pgfscope}%
\begin{pgfscope}%
\pgfsetrectcap%
\pgfsetmiterjoin%
\pgfsetlinewidth{1.003750pt}%
\definecolor{currentstroke}{rgb}{0.150000,0.150000,0.150000}%
\pgfsetstrokecolor{currentstroke}%
\pgfsetdash{}{0pt}%
\pgfpathmoveto{\pgfqpoint{0.702340in}{3.353583in}}%
\pgfpathlineto{\pgfqpoint{1.925444in}{3.353583in}}%
\pgfusepath{stroke}%
\end{pgfscope}%
\begin{pgfscope}%
\pgfpathrectangle{\pgfqpoint{0.702340in}{3.353583in}}{\pgfqpoint{1.223103in}{0.607948in}}%
\pgfusepath{clip}%
\pgfsetbuttcap%
\pgfsetroundjoin%
\definecolor{currentfill}{rgb}{0.000000,0.000000,0.000000}%
\pgfsetfillcolor{currentfill}%
\pgfsetlinewidth{1.003750pt}%
\definecolor{currentstroke}{rgb}{0.000000,0.000000,0.000000}%
\pgfsetstrokecolor{currentstroke}%
\pgfsetdash{}{0pt}%
\pgfsys@defobject{currentmarker}{\pgfqpoint{-0.013889in}{-0.013889in}}{\pgfqpoint{0.013889in}{0.013889in}}{%
\pgfpathmoveto{\pgfqpoint{0.000000in}{-0.013889in}}%
\pgfpathcurveto{\pgfqpoint{0.003683in}{-0.013889in}}{\pgfqpoint{0.007216in}{-0.012425in}}{\pgfqpoint{0.009821in}{-0.009821in}}%
\pgfpathcurveto{\pgfqpoint{0.012425in}{-0.007216in}}{\pgfqpoint{0.013889in}{-0.003683in}}{\pgfqpoint{0.013889in}{0.000000in}}%
\pgfpathcurveto{\pgfqpoint{0.013889in}{0.003683in}}{\pgfqpoint{0.012425in}{0.007216in}}{\pgfqpoint{0.009821in}{0.009821in}}%
\pgfpathcurveto{\pgfqpoint{0.007216in}{0.012425in}}{\pgfqpoint{0.003683in}{0.013889in}}{\pgfqpoint{0.000000in}{0.013889in}}%
\pgfpathcurveto{\pgfqpoint{-0.003683in}{0.013889in}}{\pgfqpoint{-0.007216in}{0.012425in}}{\pgfqpoint{-0.009821in}{0.009821in}}%
\pgfpathcurveto{\pgfqpoint{-0.012425in}{0.007216in}}{\pgfqpoint{-0.013889in}{0.003683in}}{\pgfqpoint{-0.013889in}{0.000000in}}%
\pgfpathcurveto{\pgfqpoint{-0.013889in}{-0.003683in}}{\pgfqpoint{-0.012425in}{-0.007216in}}{\pgfqpoint{-0.009821in}{-0.009821in}}%
\pgfpathcurveto{\pgfqpoint{-0.007216in}{-0.012425in}}{\pgfqpoint{-0.003683in}{-0.013889in}}{\pgfqpoint{0.000000in}{-0.013889in}}%
\pgfpathclose%
\pgfusepath{stroke,fill}%
}%
\begin{pgfscope}%
\pgfsys@transformshift{1.705158in}{2.607908in}%
\pgfsys@useobject{currentmarker}{}%
\end{pgfscope}%
\begin{pgfscope}%
\pgfsys@transformshift{1.616865in}{3.165798in}%
\pgfsys@useobject{currentmarker}{}%
\end{pgfscope}%
\begin{pgfscope}%
\pgfsys@transformshift{1.554219in}{3.373729in}%
\pgfsys@useobject{currentmarker}{}%
\end{pgfscope}%
\begin{pgfscope}%
\pgfsys@transformshift{1.505628in}{3.475546in}%
\pgfsys@useobject{currentmarker}{}%
\end{pgfscope}%
\begin{pgfscope}%
\pgfsys@transformshift{1.465926in}{3.533149in}%
\pgfsys@useobject{currentmarker}{}%
\end{pgfscope}%
\begin{pgfscope}%
\pgfsys@transformshift{1.432358in}{3.568934in}%
\pgfsys@useobject{currentmarker}{}%
\end{pgfscope}%
\begin{pgfscope}%
\pgfsys@transformshift{1.403280in}{3.592703in}%
\pgfsys@useobject{currentmarker}{}%
\end{pgfscope}%
\begin{pgfscope}%
\pgfsys@transformshift{1.397903in}{3.596491in}%
\pgfsys@useobject{currentmarker}{}%
\end{pgfscope}%
\begin{pgfscope}%
\pgfsys@transformshift{1.392656in}{3.600025in}%
\pgfsys@useobject{currentmarker}{}%
\end{pgfscope}%
\begin{pgfscope}%
\pgfsys@transformshift{1.387532in}{3.603326in}%
\pgfsys@useobject{currentmarker}{}%
\end{pgfscope}%
\begin{pgfscope}%
\pgfsys@transformshift{1.382525in}{3.606415in}%
\pgfsys@useobject{currentmarker}{}%
\end{pgfscope}%
\begin{pgfscope}%
\pgfsys@transformshift{1.377632in}{3.609310in}%
\pgfsys@useobject{currentmarker}{}%
\end{pgfscope}%
\begin{pgfscope}%
\pgfsys@transformshift{1.372846in}{3.612027in}%
\pgfsys@useobject{currentmarker}{}%
\end{pgfscope}%
\begin{pgfscope}%
\pgfsys@transformshift{1.368163in}{3.614581in}%
\pgfsys@useobject{currentmarker}{}%
\end{pgfscope}%
\begin{pgfscope}%
\pgfsys@transformshift{1.363578in}{3.616983in}%
\pgfsys@useobject{currentmarker}{}%
\end{pgfscope}%
\begin{pgfscope}%
\pgfsys@transformshift{1.359088in}{3.619247in}%
\pgfsys@useobject{currentmarker}{}%
\end{pgfscope}%
\begin{pgfscope}%
\pgfsys@transformshift{1.354689in}{3.621382in}%
\pgfsys@useobject{currentmarker}{}%
\end{pgfscope}%
\end{pgfscope}%
\begin{pgfscope}%
\pgfsetbuttcap%
\pgfsetmiterjoin%
\definecolor{currentfill}{rgb}{1.000000,1.000000,1.000000}%
\pgfsetfillcolor{currentfill}%
\pgfsetlinewidth{0.000000pt}%
\definecolor{currentstroke}{rgb}{0.000000,0.000000,0.000000}%
\pgfsetstrokecolor{currentstroke}%
\pgfsetstrokeopacity{0.000000}%
\pgfsetdash{}{0pt}%
\pgfpathmoveto{\pgfqpoint{2.170064in}{3.353583in}}%
\pgfpathlineto{\pgfqpoint{3.393168in}{3.353583in}}%
\pgfpathlineto{\pgfqpoint{3.393168in}{3.961532in}}%
\pgfpathlineto{\pgfqpoint{2.170064in}{3.961532in}}%
\pgfpathclose%
\pgfusepath{fill}%
\end{pgfscope}%
\begin{pgfscope}%
\pgfpathrectangle{\pgfqpoint{2.170064in}{3.353583in}}{\pgfqpoint{1.223103in}{0.607948in}}%
\pgfusepath{clip}%
\pgfsetbuttcap%
\pgfsetmiterjoin%
\definecolor{currentfill}{rgb}{0.000000,0.000000,1.000000}%
\pgfsetfillcolor{currentfill}%
\pgfsetfillopacity{0.100000}%
\pgfsetlinewidth{0.803000pt}%
\definecolor{currentstroke}{rgb}{0.000000,0.000000,1.000000}%
\pgfsetstrokecolor{currentstroke}%
\pgfsetstrokeopacity{0.100000}%
\pgfsetdash{}{0pt}%
\pgfpathmoveto{\pgfqpoint{2.170064in}{3.637797in}}%
\pgfpathlineto{\pgfqpoint{2.170064in}{3.692191in}}%
\pgfpathlineto{\pgfqpoint{3.393168in}{3.692191in}}%
\pgfpathlineto{\pgfqpoint{3.393168in}{3.637797in}}%
\pgfpathclose%
\pgfusepath{stroke,fill}%
\end{pgfscope}%
\begin{pgfscope}%
\pgfpathrectangle{\pgfqpoint{2.170064in}{3.353583in}}{\pgfqpoint{1.223103in}{0.607948in}}%
\pgfusepath{clip}%
\pgfsetbuttcap%
\pgfsetroundjoin%
\definecolor{currentfill}{rgb}{0.000000,0.501961,0.000000}%
\pgfsetfillcolor{currentfill}%
\pgfsetfillopacity{0.500000}%
\pgfsetlinewidth{0.803000pt}%
\definecolor{currentstroke}{rgb}{0.000000,0.501961,0.000000}%
\pgfsetstrokecolor{currentstroke}%
\pgfsetstrokeopacity{0.500000}%
\pgfsetdash{}{0pt}%
\pgfpathmoveto{\pgfqpoint{2.170064in}{3.691289in}}%
\pgfpathlineto{\pgfqpoint{2.170064in}{3.640826in}}%
\pgfpathlineto{\pgfqpoint{2.410774in}{3.646659in}}%
\pgfpathlineto{\pgfqpoint{2.522305in}{3.652102in}}%
\pgfpathlineto{\pgfqpoint{2.595701in}{3.657150in}}%
\pgfpathlineto{\pgfqpoint{2.650497in}{3.661798in}}%
\pgfpathlineto{\pgfqpoint{2.694239in}{3.666039in}}%
\pgfpathlineto{\pgfqpoint{2.730647in}{3.669867in}}%
\pgfpathlineto{\pgfqpoint{2.761831in}{3.673273in}}%
\pgfpathlineto{\pgfqpoint{2.789104in}{3.676251in}}%
\pgfpathlineto{\pgfqpoint{2.813338in}{3.678792in}}%
\pgfpathlineto{\pgfqpoint{2.835143in}{3.680888in}}%
\pgfpathlineto{\pgfqpoint{2.854962in}{3.682374in}}%
\pgfpathlineto{\pgfqpoint{2.873127in}{3.681855in}}%
\pgfpathlineto{\pgfqpoint{2.889892in}{3.681240in}}%
\pgfpathlineto{\pgfqpoint{2.905459in}{3.680499in}}%
\pgfpathlineto{\pgfqpoint{2.919986in}{3.679597in}}%
\pgfpathlineto{\pgfqpoint{2.933605in}{3.678500in}}%
\pgfpathlineto{\pgfqpoint{2.946422in}{3.677173in}}%
\pgfpathlineto{\pgfqpoint{2.958526in}{3.675580in}}%
\pgfpathlineto{\pgfqpoint{2.969993in}{3.673688in}}%
\pgfpathlineto{\pgfqpoint{2.980886in}{3.671461in}}%
\pgfpathlineto{\pgfqpoint{2.991260in}{3.668863in}}%
\pgfpathlineto{\pgfqpoint{3.001162in}{3.665860in}}%
\pgfpathlineto{\pgfqpoint{3.010633in}{3.662412in}}%
\pgfpathlineto{\pgfqpoint{3.019710in}{3.658366in}}%
\pgfpathlineto{\pgfqpoint{3.028423in}{3.652858in}}%
\pgfpathlineto{\pgfqpoint{3.036801in}{3.646623in}}%
\pgfpathlineto{\pgfqpoint{3.044869in}{3.639782in}}%
\pgfpathlineto{\pgfqpoint{3.052648in}{3.632331in}}%
\pgfpathlineto{\pgfqpoint{3.060159in}{3.624263in}}%
\pgfpathlineto{\pgfqpoint{3.067420in}{3.615569in}}%
\pgfpathlineto{\pgfqpoint{3.074446in}{3.606241in}}%
\pgfpathlineto{\pgfqpoint{3.081253in}{3.596271in}}%
\pgfpathlineto{\pgfqpoint{3.087853in}{3.585650in}}%
\pgfpathlineto{\pgfqpoint{3.094259in}{3.574372in}}%
\pgfpathlineto{\pgfqpoint{3.100482in}{3.562427in}}%
\pgfpathlineto{\pgfqpoint{3.106532in}{3.549808in}}%
\pgfpathlineto{\pgfqpoint{3.112419in}{3.536509in}}%
\pgfpathlineto{\pgfqpoint{3.118150in}{3.522524in}}%
\pgfpathlineto{\pgfqpoint{3.123735in}{3.507844in}}%
\pgfpathlineto{\pgfqpoint{3.129180in}{3.492466in}}%
\pgfpathlineto{\pgfqpoint{3.134492in}{3.476382in}}%
\pgfpathlineto{\pgfqpoint{3.139677in}{3.459590in}}%
\pgfpathlineto{\pgfqpoint{3.144742in}{3.442085in}}%
\pgfpathlineto{\pgfqpoint{3.149692in}{3.423864in}}%
\pgfpathlineto{\pgfqpoint{3.154532in}{3.404924in}}%
\pgfpathlineto{\pgfqpoint{3.159267in}{3.385262in}}%
\pgfpathlineto{\pgfqpoint{3.163901in}{3.364868in}}%
\pgfpathlineto{\pgfqpoint{3.168438in}{3.343546in}}%
\pgfpathlineto{\pgfqpoint{3.172883in}{3.319047in}}%
\pgfpathlineto{\pgfqpoint{3.172883in}{3.322074in}}%
\pgfpathlineto{\pgfqpoint{3.172883in}{3.322074in}}%
\pgfpathlineto{\pgfqpoint{3.168438in}{3.344346in}}%
\pgfpathlineto{\pgfqpoint{3.163901in}{3.367986in}}%
\pgfpathlineto{\pgfqpoint{3.159267in}{3.390757in}}%
\pgfpathlineto{\pgfqpoint{3.154532in}{3.412474in}}%
\pgfpathlineto{\pgfqpoint{3.149692in}{3.433144in}}%
\pgfpathlineto{\pgfqpoint{3.144742in}{3.452780in}}%
\pgfpathlineto{\pgfqpoint{3.139677in}{3.471401in}}%
\pgfpathlineto{\pgfqpoint{3.134492in}{3.489024in}}%
\pgfpathlineto{\pgfqpoint{3.129180in}{3.505668in}}%
\pgfpathlineto{\pgfqpoint{3.123735in}{3.521353in}}%
\pgfpathlineto{\pgfqpoint{3.118150in}{3.536102in}}%
\pgfpathlineto{\pgfqpoint{3.112419in}{3.549936in}}%
\pgfpathlineto{\pgfqpoint{3.106532in}{3.562879in}}%
\pgfpathlineto{\pgfqpoint{3.100482in}{3.574955in}}%
\pgfpathlineto{\pgfqpoint{3.094259in}{3.586190in}}%
\pgfpathlineto{\pgfqpoint{3.087853in}{3.596611in}}%
\pgfpathlineto{\pgfqpoint{3.081253in}{3.606244in}}%
\pgfpathlineto{\pgfqpoint{3.074446in}{3.615118in}}%
\pgfpathlineto{\pgfqpoint{3.067420in}{3.623263in}}%
\pgfpathlineto{\pgfqpoint{3.060159in}{3.630708in}}%
\pgfpathlineto{\pgfqpoint{3.052648in}{3.637484in}}%
\pgfpathlineto{\pgfqpoint{3.044869in}{3.643623in}}%
\pgfpathlineto{\pgfqpoint{3.036801in}{3.649158in}}%
\pgfpathlineto{\pgfqpoint{3.028423in}{3.654126in}}%
\pgfpathlineto{\pgfqpoint{3.019710in}{3.658691in}}%
\pgfpathlineto{\pgfqpoint{3.010633in}{3.663631in}}%
\pgfpathlineto{\pgfqpoint{3.001162in}{3.668123in}}%
\pgfpathlineto{\pgfqpoint{2.991260in}{3.672059in}}%
\pgfpathlineto{\pgfqpoint{2.980886in}{3.675443in}}%
\pgfpathlineto{\pgfqpoint{2.969993in}{3.678284in}}%
\pgfpathlineto{\pgfqpoint{2.958526in}{3.680590in}}%
\pgfpathlineto{\pgfqpoint{2.946422in}{3.682371in}}%
\pgfpathlineto{\pgfqpoint{2.933605in}{3.683635in}}%
\pgfpathlineto{\pgfqpoint{2.919986in}{3.684392in}}%
\pgfpathlineto{\pgfqpoint{2.905459in}{3.684651in}}%
\pgfpathlineto{\pgfqpoint{2.889892in}{3.684421in}}%
\pgfpathlineto{\pgfqpoint{2.873127in}{3.683711in}}%
\pgfpathlineto{\pgfqpoint{2.854962in}{3.682533in}}%
\pgfpathlineto{\pgfqpoint{2.835143in}{3.682839in}}%
\pgfpathlineto{\pgfqpoint{2.813338in}{3.683276in}}%
\pgfpathlineto{\pgfqpoint{2.789104in}{3.683720in}}%
\pgfpathlineto{\pgfqpoint{2.761831in}{3.684203in}}%
\pgfpathlineto{\pgfqpoint{2.730647in}{3.684757in}}%
\pgfpathlineto{\pgfqpoint{2.694239in}{3.685414in}}%
\pgfpathlineto{\pgfqpoint{2.650497in}{3.686204in}}%
\pgfpathlineto{\pgfqpoint{2.595701in}{3.687157in}}%
\pgfpathlineto{\pgfqpoint{2.522305in}{3.688303in}}%
\pgfpathlineto{\pgfqpoint{2.410774in}{3.689671in}}%
\pgfpathlineto{\pgfqpoint{2.170064in}{3.691289in}}%
\pgfpathclose%
\pgfusepath{stroke,fill}%
\end{pgfscope}%
\begin{pgfscope}%
\pgfpathrectangle{\pgfqpoint{2.170064in}{3.353583in}}{\pgfqpoint{1.223103in}{0.607948in}}%
\pgfusepath{clip}%
\pgfsetroundcap%
\pgfsetroundjoin%
\pgfsetlinewidth{0.501875pt}%
\definecolor{currentstroke}{rgb}{0.000000,0.000000,1.000000}%
\pgfsetstrokecolor{currentstroke}%
\pgfsetstrokeopacity{0.800000}%
\pgfsetdash{}{0pt}%
\pgfpathmoveto{\pgfqpoint{2.170064in}{3.664994in}}%
\pgfpathlineto{\pgfqpoint{3.393168in}{3.664994in}}%
\pgfusepath{stroke}%
\end{pgfscope}%
\begin{pgfscope}%
\pgfpathrectangle{\pgfqpoint{2.170064in}{3.353583in}}{\pgfqpoint{1.223103in}{0.607948in}}%
\pgfusepath{clip}%
\pgfsetbuttcap%
\pgfsetroundjoin%
\pgfsetlinewidth{1.003750pt}%
\definecolor{currentstroke}{rgb}{0.000000,0.000000,0.000000}%
\pgfsetstrokecolor{currentstroke}%
\pgfsetdash{{3.700000pt}{1.600000pt}}{0.000000pt}%
\pgfpathmoveto{\pgfqpoint{2.170064in}{3.678551in}}%
\pgfpathlineto{\pgfqpoint{3.393168in}{3.678551in}}%
\pgfusepath{stroke}%
\end{pgfscope}%
\begin{pgfscope}%
\pgfsetroundcap%
\pgfsetroundjoin%
\pgfsetlinewidth{0.501875pt}%
\definecolor{currentstroke}{rgb}{0.000000,0.000000,1.000000}%
\pgfsetstrokecolor{currentstroke}%
\pgfsetstrokeopacity{0.800000}%
\pgfsetdash{}{0pt}%
\pgfpathmoveto{\pgfqpoint{2.970757in}{3.781499in}}%
\pgfpathquadraticcurveto{\pgfqpoint{2.907509in}{3.731826in}}{\pgfqpoint{2.844261in}{3.682153in}}%
\pgfusepath{stroke}%
\end{pgfscope}%
\begin{pgfscope}%
\pgfsetfillopacity{0.800000}%
\pgfsetstrokeopacity{0.800000}%
\definecolor{textcolor}{rgb}{0.000000,0.000000,1.000000}%
\pgfsetstrokecolor{textcolor}%
\pgfsetfillcolor{textcolor}%
\pgftext[x=2.910706in,y=3.847378in,left,base]{\color{textcolor}\sffamily\fontsize{5.647059}{6.776471}\selectfont 9.512(45)}%
\end{pgfscope}%
\begin{pgfscope}%
\pgfsetbuttcap%
\pgfsetroundjoin%
\definecolor{currentfill}{rgb}{0.150000,0.150000,0.150000}%
\pgfsetfillcolor{currentfill}%
\pgfsetlinewidth{1.003750pt}%
\definecolor{currentstroke}{rgb}{0.150000,0.150000,0.150000}%
\pgfsetstrokecolor{currentstroke}%
\pgfsetdash{}{0pt}%
\pgfsys@defobject{currentmarker}{\pgfqpoint{0.000000in}{-0.066667in}}{\pgfqpoint{0.000000in}{0.000000in}}{%
\pgfpathmoveto{\pgfqpoint{0.000000in}{0.000000in}}%
\pgfpathlineto{\pgfqpoint{0.000000in}{-0.066667in}}%
\pgfusepath{stroke,fill}%
}%
\begin{pgfscope}%
\pgfsys@transformshift{2.170064in}{3.353583in}%
\pgfsys@useobject{currentmarker}{}%
\end{pgfscope}%
\end{pgfscope}%
\begin{pgfscope}%
\pgfsetbuttcap%
\pgfsetroundjoin%
\definecolor{currentfill}{rgb}{0.150000,0.150000,0.150000}%
\pgfsetfillcolor{currentfill}%
\pgfsetlinewidth{1.003750pt}%
\definecolor{currentstroke}{rgb}{0.150000,0.150000,0.150000}%
\pgfsetstrokecolor{currentstroke}%
\pgfsetdash{}{0pt}%
\pgfsys@defobject{currentmarker}{\pgfqpoint{0.000000in}{-0.066667in}}{\pgfqpoint{0.000000in}{0.000000in}}{%
\pgfpathmoveto{\pgfqpoint{0.000000in}{0.000000in}}%
\pgfpathlineto{\pgfqpoint{0.000000in}{-0.066667in}}%
\pgfusepath{stroke,fill}%
}%
\begin{pgfscope}%
\pgfsys@transformshift{2.671473in}{3.353583in}%
\pgfsys@useobject{currentmarker}{}%
\end{pgfscope}%
\end{pgfscope}%
\begin{pgfscope}%
\pgfsetbuttcap%
\pgfsetroundjoin%
\definecolor{currentfill}{rgb}{0.150000,0.150000,0.150000}%
\pgfsetfillcolor{currentfill}%
\pgfsetlinewidth{1.003750pt}%
\definecolor{currentstroke}{rgb}{0.150000,0.150000,0.150000}%
\pgfsetstrokecolor{currentstroke}%
\pgfsetdash{}{0pt}%
\pgfsys@defobject{currentmarker}{\pgfqpoint{0.000000in}{-0.066667in}}{\pgfqpoint{0.000000in}{0.000000in}}{%
\pgfpathmoveto{\pgfqpoint{0.000000in}{0.000000in}}%
\pgfpathlineto{\pgfqpoint{0.000000in}{-0.066667in}}%
\pgfusepath{stroke,fill}%
}%
\begin{pgfscope}%
\pgfsys@transformshift{3.172883in}{3.353583in}%
\pgfsys@useobject{currentmarker}{}%
\end{pgfscope}%
\end{pgfscope}%
\begin{pgfscope}%
\pgfsetbuttcap%
\pgfsetroundjoin%
\definecolor{currentfill}{rgb}{0.150000,0.150000,0.150000}%
\pgfsetfillcolor{currentfill}%
\pgfsetlinewidth{0.803000pt}%
\definecolor{currentstroke}{rgb}{0.150000,0.150000,0.150000}%
\pgfsetstrokecolor{currentstroke}%
\pgfsetdash{}{0pt}%
\pgfsys@defobject{currentmarker}{\pgfqpoint{0.000000in}{-0.044444in}}{\pgfqpoint{0.000000in}{0.000000in}}{%
\pgfpathmoveto{\pgfqpoint{0.000000in}{0.000000in}}%
\pgfpathlineto{\pgfqpoint{0.000000in}{-0.044444in}}%
\pgfusepath{stroke,fill}%
}%
\begin{pgfscope}%
\pgfsys@transformshift{2.321004in}{3.353583in}%
\pgfsys@useobject{currentmarker}{}%
\end{pgfscope}%
\end{pgfscope}%
\begin{pgfscope}%
\pgfsetbuttcap%
\pgfsetroundjoin%
\definecolor{currentfill}{rgb}{0.150000,0.150000,0.150000}%
\pgfsetfillcolor{currentfill}%
\pgfsetlinewidth{0.803000pt}%
\definecolor{currentstroke}{rgb}{0.150000,0.150000,0.150000}%
\pgfsetstrokecolor{currentstroke}%
\pgfsetdash{}{0pt}%
\pgfsys@defobject{currentmarker}{\pgfqpoint{0.000000in}{-0.044444in}}{\pgfqpoint{0.000000in}{0.000000in}}{%
\pgfpathmoveto{\pgfqpoint{0.000000in}{0.000000in}}%
\pgfpathlineto{\pgfqpoint{0.000000in}{-0.044444in}}%
\pgfusepath{stroke,fill}%
}%
\begin{pgfscope}%
\pgfsys@transformshift{2.409297in}{3.353583in}%
\pgfsys@useobject{currentmarker}{}%
\end{pgfscope}%
\end{pgfscope}%
\begin{pgfscope}%
\pgfsetbuttcap%
\pgfsetroundjoin%
\definecolor{currentfill}{rgb}{0.150000,0.150000,0.150000}%
\pgfsetfillcolor{currentfill}%
\pgfsetlinewidth{0.803000pt}%
\definecolor{currentstroke}{rgb}{0.150000,0.150000,0.150000}%
\pgfsetstrokecolor{currentstroke}%
\pgfsetdash{}{0pt}%
\pgfsys@defobject{currentmarker}{\pgfqpoint{0.000000in}{-0.044444in}}{\pgfqpoint{0.000000in}{0.000000in}}{%
\pgfpathmoveto{\pgfqpoint{0.000000in}{0.000000in}}%
\pgfpathlineto{\pgfqpoint{0.000000in}{-0.044444in}}%
\pgfusepath{stroke,fill}%
}%
\begin{pgfscope}%
\pgfsys@transformshift{2.471943in}{3.353583in}%
\pgfsys@useobject{currentmarker}{}%
\end{pgfscope}%
\end{pgfscope}%
\begin{pgfscope}%
\pgfsetbuttcap%
\pgfsetroundjoin%
\definecolor{currentfill}{rgb}{0.150000,0.150000,0.150000}%
\pgfsetfillcolor{currentfill}%
\pgfsetlinewidth{0.803000pt}%
\definecolor{currentstroke}{rgb}{0.150000,0.150000,0.150000}%
\pgfsetstrokecolor{currentstroke}%
\pgfsetdash{}{0pt}%
\pgfsys@defobject{currentmarker}{\pgfqpoint{0.000000in}{-0.044444in}}{\pgfqpoint{0.000000in}{0.000000in}}{%
\pgfpathmoveto{\pgfqpoint{0.000000in}{0.000000in}}%
\pgfpathlineto{\pgfqpoint{0.000000in}{-0.044444in}}%
\pgfusepath{stroke,fill}%
}%
\begin{pgfscope}%
\pgfsys@transformshift{2.520534in}{3.353583in}%
\pgfsys@useobject{currentmarker}{}%
\end{pgfscope}%
\end{pgfscope}%
\begin{pgfscope}%
\pgfsetbuttcap%
\pgfsetroundjoin%
\definecolor{currentfill}{rgb}{0.150000,0.150000,0.150000}%
\pgfsetfillcolor{currentfill}%
\pgfsetlinewidth{0.803000pt}%
\definecolor{currentstroke}{rgb}{0.150000,0.150000,0.150000}%
\pgfsetstrokecolor{currentstroke}%
\pgfsetdash{}{0pt}%
\pgfsys@defobject{currentmarker}{\pgfqpoint{0.000000in}{-0.044444in}}{\pgfqpoint{0.000000in}{0.000000in}}{%
\pgfpathmoveto{\pgfqpoint{0.000000in}{0.000000in}}%
\pgfpathlineto{\pgfqpoint{0.000000in}{-0.044444in}}%
\pgfusepath{stroke,fill}%
}%
\begin{pgfscope}%
\pgfsys@transformshift{2.560237in}{3.353583in}%
\pgfsys@useobject{currentmarker}{}%
\end{pgfscope}%
\end{pgfscope}%
\begin{pgfscope}%
\pgfsetbuttcap%
\pgfsetroundjoin%
\definecolor{currentfill}{rgb}{0.150000,0.150000,0.150000}%
\pgfsetfillcolor{currentfill}%
\pgfsetlinewidth{0.803000pt}%
\definecolor{currentstroke}{rgb}{0.150000,0.150000,0.150000}%
\pgfsetstrokecolor{currentstroke}%
\pgfsetdash{}{0pt}%
\pgfsys@defobject{currentmarker}{\pgfqpoint{0.000000in}{-0.044444in}}{\pgfqpoint{0.000000in}{0.000000in}}{%
\pgfpathmoveto{\pgfqpoint{0.000000in}{0.000000in}}%
\pgfpathlineto{\pgfqpoint{0.000000in}{-0.044444in}}%
\pgfusepath{stroke,fill}%
}%
\begin{pgfscope}%
\pgfsys@transformshift{2.593804in}{3.353583in}%
\pgfsys@useobject{currentmarker}{}%
\end{pgfscope}%
\end{pgfscope}%
\begin{pgfscope}%
\pgfsetbuttcap%
\pgfsetroundjoin%
\definecolor{currentfill}{rgb}{0.150000,0.150000,0.150000}%
\pgfsetfillcolor{currentfill}%
\pgfsetlinewidth{0.803000pt}%
\definecolor{currentstroke}{rgb}{0.150000,0.150000,0.150000}%
\pgfsetstrokecolor{currentstroke}%
\pgfsetdash{}{0pt}%
\pgfsys@defobject{currentmarker}{\pgfqpoint{0.000000in}{-0.044444in}}{\pgfqpoint{0.000000in}{0.000000in}}{%
\pgfpathmoveto{\pgfqpoint{0.000000in}{0.000000in}}%
\pgfpathlineto{\pgfqpoint{0.000000in}{-0.044444in}}%
\pgfusepath{stroke,fill}%
}%
\begin{pgfscope}%
\pgfsys@transformshift{2.622882in}{3.353583in}%
\pgfsys@useobject{currentmarker}{}%
\end{pgfscope}%
\end{pgfscope}%
\begin{pgfscope}%
\pgfsetbuttcap%
\pgfsetroundjoin%
\definecolor{currentfill}{rgb}{0.150000,0.150000,0.150000}%
\pgfsetfillcolor{currentfill}%
\pgfsetlinewidth{0.803000pt}%
\definecolor{currentstroke}{rgb}{0.150000,0.150000,0.150000}%
\pgfsetstrokecolor{currentstroke}%
\pgfsetdash{}{0pt}%
\pgfsys@defobject{currentmarker}{\pgfqpoint{0.000000in}{-0.044444in}}{\pgfqpoint{0.000000in}{0.000000in}}{%
\pgfpathmoveto{\pgfqpoint{0.000000in}{0.000000in}}%
\pgfpathlineto{\pgfqpoint{0.000000in}{-0.044444in}}%
\pgfusepath{stroke,fill}%
}%
\begin{pgfscope}%
\pgfsys@transformshift{2.648530in}{3.353583in}%
\pgfsys@useobject{currentmarker}{}%
\end{pgfscope}%
\end{pgfscope}%
\begin{pgfscope}%
\pgfsetbuttcap%
\pgfsetroundjoin%
\definecolor{currentfill}{rgb}{0.150000,0.150000,0.150000}%
\pgfsetfillcolor{currentfill}%
\pgfsetlinewidth{0.803000pt}%
\definecolor{currentstroke}{rgb}{0.150000,0.150000,0.150000}%
\pgfsetstrokecolor{currentstroke}%
\pgfsetdash{}{0pt}%
\pgfsys@defobject{currentmarker}{\pgfqpoint{0.000000in}{-0.044444in}}{\pgfqpoint{0.000000in}{0.000000in}}{%
\pgfpathmoveto{\pgfqpoint{0.000000in}{0.000000in}}%
\pgfpathlineto{\pgfqpoint{0.000000in}{-0.044444in}}%
\pgfusepath{stroke,fill}%
}%
\begin{pgfscope}%
\pgfsys@transformshift{2.822413in}{3.353583in}%
\pgfsys@useobject{currentmarker}{}%
\end{pgfscope}%
\end{pgfscope}%
\begin{pgfscope}%
\pgfsetbuttcap%
\pgfsetroundjoin%
\definecolor{currentfill}{rgb}{0.150000,0.150000,0.150000}%
\pgfsetfillcolor{currentfill}%
\pgfsetlinewidth{0.803000pt}%
\definecolor{currentstroke}{rgb}{0.150000,0.150000,0.150000}%
\pgfsetstrokecolor{currentstroke}%
\pgfsetdash{}{0pt}%
\pgfsys@defobject{currentmarker}{\pgfqpoint{0.000000in}{-0.044444in}}{\pgfqpoint{0.000000in}{0.000000in}}{%
\pgfpathmoveto{\pgfqpoint{0.000000in}{0.000000in}}%
\pgfpathlineto{\pgfqpoint{0.000000in}{-0.044444in}}%
\pgfusepath{stroke,fill}%
}%
\begin{pgfscope}%
\pgfsys@transformshift{2.910706in}{3.353583in}%
\pgfsys@useobject{currentmarker}{}%
\end{pgfscope}%
\end{pgfscope}%
\begin{pgfscope}%
\pgfsetbuttcap%
\pgfsetroundjoin%
\definecolor{currentfill}{rgb}{0.150000,0.150000,0.150000}%
\pgfsetfillcolor{currentfill}%
\pgfsetlinewidth{0.803000pt}%
\definecolor{currentstroke}{rgb}{0.150000,0.150000,0.150000}%
\pgfsetstrokecolor{currentstroke}%
\pgfsetdash{}{0pt}%
\pgfsys@defobject{currentmarker}{\pgfqpoint{0.000000in}{-0.044444in}}{\pgfqpoint{0.000000in}{0.000000in}}{%
\pgfpathmoveto{\pgfqpoint{0.000000in}{0.000000in}}%
\pgfpathlineto{\pgfqpoint{0.000000in}{-0.044444in}}%
\pgfusepath{stroke,fill}%
}%
\begin{pgfscope}%
\pgfsys@transformshift{2.973352in}{3.353583in}%
\pgfsys@useobject{currentmarker}{}%
\end{pgfscope}%
\end{pgfscope}%
\begin{pgfscope}%
\pgfsetbuttcap%
\pgfsetroundjoin%
\definecolor{currentfill}{rgb}{0.150000,0.150000,0.150000}%
\pgfsetfillcolor{currentfill}%
\pgfsetlinewidth{0.803000pt}%
\definecolor{currentstroke}{rgb}{0.150000,0.150000,0.150000}%
\pgfsetstrokecolor{currentstroke}%
\pgfsetdash{}{0pt}%
\pgfsys@defobject{currentmarker}{\pgfqpoint{0.000000in}{-0.044444in}}{\pgfqpoint{0.000000in}{0.000000in}}{%
\pgfpathmoveto{\pgfqpoint{0.000000in}{0.000000in}}%
\pgfpathlineto{\pgfqpoint{0.000000in}{-0.044444in}}%
\pgfusepath{stroke,fill}%
}%
\begin{pgfscope}%
\pgfsys@transformshift{3.021943in}{3.353583in}%
\pgfsys@useobject{currentmarker}{}%
\end{pgfscope}%
\end{pgfscope}%
\begin{pgfscope}%
\pgfsetbuttcap%
\pgfsetroundjoin%
\definecolor{currentfill}{rgb}{0.150000,0.150000,0.150000}%
\pgfsetfillcolor{currentfill}%
\pgfsetlinewidth{0.803000pt}%
\definecolor{currentstroke}{rgb}{0.150000,0.150000,0.150000}%
\pgfsetstrokecolor{currentstroke}%
\pgfsetdash{}{0pt}%
\pgfsys@defobject{currentmarker}{\pgfqpoint{0.000000in}{-0.044444in}}{\pgfqpoint{0.000000in}{0.000000in}}{%
\pgfpathmoveto{\pgfqpoint{0.000000in}{0.000000in}}%
\pgfpathlineto{\pgfqpoint{0.000000in}{-0.044444in}}%
\pgfusepath{stroke,fill}%
}%
\begin{pgfscope}%
\pgfsys@transformshift{3.061646in}{3.353583in}%
\pgfsys@useobject{currentmarker}{}%
\end{pgfscope}%
\end{pgfscope}%
\begin{pgfscope}%
\pgfsetbuttcap%
\pgfsetroundjoin%
\definecolor{currentfill}{rgb}{0.150000,0.150000,0.150000}%
\pgfsetfillcolor{currentfill}%
\pgfsetlinewidth{0.803000pt}%
\definecolor{currentstroke}{rgb}{0.150000,0.150000,0.150000}%
\pgfsetstrokecolor{currentstroke}%
\pgfsetdash{}{0pt}%
\pgfsys@defobject{currentmarker}{\pgfqpoint{0.000000in}{-0.044444in}}{\pgfqpoint{0.000000in}{0.000000in}}{%
\pgfpathmoveto{\pgfqpoint{0.000000in}{0.000000in}}%
\pgfpathlineto{\pgfqpoint{0.000000in}{-0.044444in}}%
\pgfusepath{stroke,fill}%
}%
\begin{pgfscope}%
\pgfsys@transformshift{3.095213in}{3.353583in}%
\pgfsys@useobject{currentmarker}{}%
\end{pgfscope}%
\end{pgfscope}%
\begin{pgfscope}%
\pgfsetbuttcap%
\pgfsetroundjoin%
\definecolor{currentfill}{rgb}{0.150000,0.150000,0.150000}%
\pgfsetfillcolor{currentfill}%
\pgfsetlinewidth{0.803000pt}%
\definecolor{currentstroke}{rgb}{0.150000,0.150000,0.150000}%
\pgfsetstrokecolor{currentstroke}%
\pgfsetdash{}{0pt}%
\pgfsys@defobject{currentmarker}{\pgfqpoint{0.000000in}{-0.044444in}}{\pgfqpoint{0.000000in}{0.000000in}}{%
\pgfpathmoveto{\pgfqpoint{0.000000in}{0.000000in}}%
\pgfpathlineto{\pgfqpoint{0.000000in}{-0.044444in}}%
\pgfusepath{stroke,fill}%
}%
\begin{pgfscope}%
\pgfsys@transformshift{3.124291in}{3.353583in}%
\pgfsys@useobject{currentmarker}{}%
\end{pgfscope}%
\end{pgfscope}%
\begin{pgfscope}%
\pgfsetbuttcap%
\pgfsetroundjoin%
\definecolor{currentfill}{rgb}{0.150000,0.150000,0.150000}%
\pgfsetfillcolor{currentfill}%
\pgfsetlinewidth{0.803000pt}%
\definecolor{currentstroke}{rgb}{0.150000,0.150000,0.150000}%
\pgfsetstrokecolor{currentstroke}%
\pgfsetdash{}{0pt}%
\pgfsys@defobject{currentmarker}{\pgfqpoint{0.000000in}{-0.044444in}}{\pgfqpoint{0.000000in}{0.000000in}}{%
\pgfpathmoveto{\pgfqpoint{0.000000in}{0.000000in}}%
\pgfpathlineto{\pgfqpoint{0.000000in}{-0.044444in}}%
\pgfusepath{stroke,fill}%
}%
\begin{pgfscope}%
\pgfsys@transformshift{3.149939in}{3.353583in}%
\pgfsys@useobject{currentmarker}{}%
\end{pgfscope}%
\end{pgfscope}%
\begin{pgfscope}%
\pgfsetbuttcap%
\pgfsetroundjoin%
\definecolor{currentfill}{rgb}{0.150000,0.150000,0.150000}%
\pgfsetfillcolor{currentfill}%
\pgfsetlinewidth{0.803000pt}%
\definecolor{currentstroke}{rgb}{0.150000,0.150000,0.150000}%
\pgfsetstrokecolor{currentstroke}%
\pgfsetdash{}{0pt}%
\pgfsys@defobject{currentmarker}{\pgfqpoint{0.000000in}{-0.044444in}}{\pgfqpoint{0.000000in}{0.000000in}}{%
\pgfpathmoveto{\pgfqpoint{0.000000in}{0.000000in}}%
\pgfpathlineto{\pgfqpoint{0.000000in}{-0.044444in}}%
\pgfusepath{stroke,fill}%
}%
\begin{pgfscope}%
\pgfsys@transformshift{3.323822in}{3.353583in}%
\pgfsys@useobject{currentmarker}{}%
\end{pgfscope}%
\end{pgfscope}%
\begin{pgfscope}%
\pgfsetbuttcap%
\pgfsetroundjoin%
\definecolor{currentfill}{rgb}{0.150000,0.150000,0.150000}%
\pgfsetfillcolor{currentfill}%
\pgfsetlinewidth{1.003750pt}%
\definecolor{currentstroke}{rgb}{0.150000,0.150000,0.150000}%
\pgfsetstrokecolor{currentstroke}%
\pgfsetdash{}{0pt}%
\pgfsys@defobject{currentmarker}{\pgfqpoint{-0.066667in}{0.000000in}}{\pgfqpoint{0.000000in}{0.000000in}}{%
\pgfpathmoveto{\pgfqpoint{0.000000in}{0.000000in}}%
\pgfpathlineto{\pgfqpoint{-0.066667in}{0.000000in}}%
\pgfusepath{stroke,fill}%
}%
\begin{pgfscope}%
\pgfsys@transformshift{2.170064in}{3.353583in}%
\pgfsys@useobject{currentmarker}{}%
\end{pgfscope}%
\end{pgfscope}%
\begin{pgfscope}%
\pgfsetbuttcap%
\pgfsetroundjoin%
\definecolor{currentfill}{rgb}{0.150000,0.150000,0.150000}%
\pgfsetfillcolor{currentfill}%
\pgfsetlinewidth{1.003750pt}%
\definecolor{currentstroke}{rgb}{0.150000,0.150000,0.150000}%
\pgfsetstrokecolor{currentstroke}%
\pgfsetdash{}{0pt}%
\pgfsys@defobject{currentmarker}{\pgfqpoint{-0.066667in}{0.000000in}}{\pgfqpoint{0.000000in}{0.000000in}}{%
\pgfpathmoveto{\pgfqpoint{0.000000in}{0.000000in}}%
\pgfpathlineto{\pgfqpoint{-0.066667in}{0.000000in}}%
\pgfusepath{stroke,fill}%
}%
\begin{pgfscope}%
\pgfsys@transformshift{2.170064in}{3.678551in}%
\pgfsys@useobject{currentmarker}{}%
\end{pgfscope}%
\end{pgfscope}%
\begin{pgfscope}%
\pgfsetbuttcap%
\pgfsetroundjoin%
\definecolor{currentfill}{rgb}{0.150000,0.150000,0.150000}%
\pgfsetfillcolor{currentfill}%
\pgfsetlinewidth{1.003750pt}%
\definecolor{currentstroke}{rgb}{0.150000,0.150000,0.150000}%
\pgfsetstrokecolor{currentstroke}%
\pgfsetdash{}{0pt}%
\pgfsys@defobject{currentmarker}{\pgfqpoint{-0.066667in}{0.000000in}}{\pgfqpoint{0.000000in}{0.000000in}}{%
\pgfpathmoveto{\pgfqpoint{0.000000in}{0.000000in}}%
\pgfpathlineto{\pgfqpoint{-0.066667in}{0.000000in}}%
\pgfusepath{stroke,fill}%
}%
\begin{pgfscope}%
\pgfsys@transformshift{2.170064in}{3.961532in}%
\pgfsys@useobject{currentmarker}{}%
\end{pgfscope}%
\end{pgfscope}%
\begin{pgfscope}%
\pgfpathrectangle{\pgfqpoint{2.170064in}{3.353583in}}{\pgfqpoint{1.223103in}{0.607948in}}%
\pgfusepath{clip}%
\pgfsetroundcap%
\pgfsetroundjoin%
\pgfsetlinewidth{1.204500pt}%
\definecolor{currentstroke}{rgb}{0.000000,0.501961,0.000000}%
\pgfsetstrokecolor{currentstroke}%
\pgfsetdash{}{0pt}%
\pgfpathmoveto{\pgfqpoint{2.170064in}{3.666058in}}%
\pgfpathlineto{\pgfqpoint{2.410774in}{3.668165in}}%
\pgfpathlineto{\pgfqpoint{2.522305in}{3.670203in}}%
\pgfpathlineto{\pgfqpoint{2.595701in}{3.672154in}}%
\pgfpathlineto{\pgfqpoint{2.650497in}{3.674001in}}%
\pgfpathlineto{\pgfqpoint{2.694239in}{3.675727in}}%
\pgfpathlineto{\pgfqpoint{2.730647in}{3.677312in}}%
\pgfpathlineto{\pgfqpoint{2.761831in}{3.678738in}}%
\pgfpathlineto{\pgfqpoint{2.789104in}{3.679985in}}%
\pgfpathlineto{\pgfqpoint{2.813338in}{3.681034in}}%
\pgfpathlineto{\pgfqpoint{2.835143in}{3.681863in}}%
\pgfpathlineto{\pgfqpoint{2.854962in}{3.682454in}}%
\pgfpathlineto{\pgfqpoint{2.873127in}{3.682783in}}%
\pgfpathlineto{\pgfqpoint{2.889892in}{3.682831in}}%
\pgfpathlineto{\pgfqpoint{2.905459in}{3.682575in}}%
\pgfpathlineto{\pgfqpoint{2.919986in}{3.681995in}}%
\pgfpathlineto{\pgfqpoint{2.933605in}{3.681068in}}%
\pgfpathlineto{\pgfqpoint{2.946422in}{3.679772in}}%
\pgfpathlineto{\pgfqpoint{2.958526in}{3.678085in}}%
\pgfpathlineto{\pgfqpoint{2.969993in}{3.675986in}}%
\pgfpathlineto{\pgfqpoint{2.980886in}{3.673452in}}%
\pgfpathlineto{\pgfqpoint{2.991260in}{3.670461in}}%
\pgfpathlineto{\pgfqpoint{3.001162in}{3.666991in}}%
\pgfpathlineto{\pgfqpoint{3.010633in}{3.663021in}}%
\pgfpathlineto{\pgfqpoint{3.019710in}{3.658528in}}%
\pgfpathlineto{\pgfqpoint{3.028423in}{3.653492in}}%
\pgfpathlineto{\pgfqpoint{3.036801in}{3.647890in}}%
\pgfpathlineto{\pgfqpoint{3.044869in}{3.641702in}}%
\pgfpathlineto{\pgfqpoint{3.052648in}{3.634907in}}%
\pgfpathlineto{\pgfqpoint{3.060159in}{3.627485in}}%
\pgfpathlineto{\pgfqpoint{3.067420in}{3.619416in}}%
\pgfpathlineto{\pgfqpoint{3.074446in}{3.610680in}}%
\pgfpathlineto{\pgfqpoint{3.081253in}{3.601257in}}%
\pgfpathlineto{\pgfqpoint{3.087853in}{3.591131in}}%
\pgfpathlineto{\pgfqpoint{3.094259in}{3.580281in}}%
\pgfpathlineto{\pgfqpoint{3.100482in}{3.568691in}}%
\pgfpathlineto{\pgfqpoint{3.106532in}{3.556343in}}%
\pgfpathlineto{\pgfqpoint{3.112419in}{3.543223in}}%
\pgfpathlineto{\pgfqpoint{3.118150in}{3.529313in}}%
\pgfpathlineto{\pgfqpoint{3.123735in}{3.514599in}}%
\pgfpathlineto{\pgfqpoint{3.129180in}{3.499067in}}%
\pgfpathlineto{\pgfqpoint{3.134492in}{3.482703in}}%
\pgfpathlineto{\pgfqpoint{3.139677in}{3.465496in}}%
\pgfpathlineto{\pgfqpoint{3.144742in}{3.447433in}}%
\pgfpathlineto{\pgfqpoint{3.149692in}{3.428504in}}%
\pgfpathlineto{\pgfqpoint{3.154532in}{3.408699in}}%
\pgfpathlineto{\pgfqpoint{3.159267in}{3.388009in}}%
\pgfpathlineto{\pgfqpoint{3.163901in}{3.366427in}}%
\pgfpathlineto{\pgfqpoint{3.167165in}{3.350250in}}%
\pgfusepath{stroke}%
\end{pgfscope}%
\begin{pgfscope}%
\pgfsetrectcap%
\pgfsetmiterjoin%
\pgfsetlinewidth{1.003750pt}%
\definecolor{currentstroke}{rgb}{0.150000,0.150000,0.150000}%
\pgfsetstrokecolor{currentstroke}%
\pgfsetdash{}{0pt}%
\pgfpathmoveto{\pgfqpoint{2.170064in}{3.353583in}}%
\pgfpathlineto{\pgfqpoint{2.170064in}{3.961532in}}%
\pgfusepath{stroke}%
\end{pgfscope}%
\begin{pgfscope}%
\pgfsetrectcap%
\pgfsetmiterjoin%
\pgfsetlinewidth{1.003750pt}%
\definecolor{currentstroke}{rgb}{0.150000,0.150000,0.150000}%
\pgfsetstrokecolor{currentstroke}%
\pgfsetdash{}{0pt}%
\pgfpathmoveto{\pgfqpoint{2.170064in}{3.353583in}}%
\pgfpathlineto{\pgfqpoint{3.393168in}{3.353583in}}%
\pgfusepath{stroke}%
\end{pgfscope}%
\begin{pgfscope}%
\pgfpathrectangle{\pgfqpoint{2.170064in}{3.353583in}}{\pgfqpoint{1.223103in}{0.607948in}}%
\pgfusepath{clip}%
\pgfsetbuttcap%
\pgfsetroundjoin%
\definecolor{currentfill}{rgb}{0.000000,0.000000,0.000000}%
\pgfsetfillcolor{currentfill}%
\pgfsetlinewidth{1.003750pt}%
\definecolor{currentstroke}{rgb}{0.000000,0.000000,0.000000}%
\pgfsetstrokecolor{currentstroke}%
\pgfsetdash{}{0pt}%
\pgfsys@defobject{currentmarker}{\pgfqpoint{-0.013889in}{-0.013889in}}{\pgfqpoint{0.013889in}{0.013889in}}{%
\pgfpathmoveto{\pgfqpoint{0.000000in}{-0.013889in}}%
\pgfpathcurveto{\pgfqpoint{0.003683in}{-0.013889in}}{\pgfqpoint{0.007216in}{-0.012425in}}{\pgfqpoint{0.009821in}{-0.009821in}}%
\pgfpathcurveto{\pgfqpoint{0.012425in}{-0.007216in}}{\pgfqpoint{0.013889in}{-0.003683in}}{\pgfqpoint{0.013889in}{0.000000in}}%
\pgfpathcurveto{\pgfqpoint{0.013889in}{0.003683in}}{\pgfqpoint{0.012425in}{0.007216in}}{\pgfqpoint{0.009821in}{0.009821in}}%
\pgfpathcurveto{\pgfqpoint{0.007216in}{0.012425in}}{\pgfqpoint{0.003683in}{0.013889in}}{\pgfqpoint{0.000000in}{0.013889in}}%
\pgfpathcurveto{\pgfqpoint{-0.003683in}{0.013889in}}{\pgfqpoint{-0.007216in}{0.012425in}}{\pgfqpoint{-0.009821in}{0.009821in}}%
\pgfpathcurveto{\pgfqpoint{-0.012425in}{0.007216in}}{\pgfqpoint{-0.013889in}{0.003683in}}{\pgfqpoint{-0.013889in}{0.000000in}}%
\pgfpathcurveto{\pgfqpoint{-0.013889in}{-0.003683in}}{\pgfqpoint{-0.012425in}{-0.007216in}}{\pgfqpoint{-0.009821in}{-0.009821in}}%
\pgfpathcurveto{\pgfqpoint{-0.007216in}{-0.012425in}}{\pgfqpoint{-0.003683in}{-0.013889in}}{\pgfqpoint{0.000000in}{-0.013889in}}%
\pgfpathclose%
\pgfusepath{stroke,fill}%
}%
\begin{pgfscope}%
\pgfsys@transformshift{3.172883in}{3.319695in}%
\pgfsys@useobject{currentmarker}{}%
\end{pgfscope}%
\begin{pgfscope}%
\pgfsys@transformshift{3.084589in}{3.599208in}%
\pgfsys@useobject{currentmarker}{}%
\end{pgfscope}%
\begin{pgfscope}%
\pgfsys@transformshift{3.021943in}{3.657576in}%
\pgfsys@useobject{currentmarker}{}%
\end{pgfscope}%
\begin{pgfscope}%
\pgfsys@transformshift{2.973352in}{3.673991in}%
\pgfsys@useobject{currentmarker}{}%
\end{pgfscope}%
\begin{pgfscope}%
\pgfsys@transformshift{2.933650in}{3.679516in}%
\pgfsys@useobject{currentmarker}{}%
\end{pgfscope}%
\begin{pgfscope}%
\pgfsys@transformshift{2.900082in}{3.681535in}%
\pgfsys@useobject{currentmarker}{}%
\end{pgfscope}%
\begin{pgfscope}%
\pgfsys@transformshift{2.871004in}{3.682253in}%
\pgfsys@useobject{currentmarker}{}%
\end{pgfscope}%
\begin{pgfscope}%
\pgfsys@transformshift{2.865627in}{3.682319in}%
\pgfsys@useobject{currentmarker}{}%
\end{pgfscope}%
\begin{pgfscope}%
\pgfsys@transformshift{2.860380in}{3.682369in}%
\pgfsys@useobject{currentmarker}{}%
\end{pgfscope}%
\begin{pgfscope}%
\pgfsys@transformshift{2.855256in}{3.682404in}%
\pgfsys@useobject{currentmarker}{}%
\end{pgfscope}%
\begin{pgfscope}%
\pgfsys@transformshift{2.822413in}{3.682405in}%
\pgfsys@useobject{currentmarker}{}%
\end{pgfscope}%
\begin{pgfscope}%
\pgfsys@transformshift{2.826812in}{3.682423in}%
\pgfsys@useobject{currentmarker}{}%
\end{pgfscope}%
\begin{pgfscope}%
\pgfsys@transformshift{2.850250in}{3.682428in}%
\pgfsys@useobject{currentmarker}{}%
\end{pgfscope}%
\begin{pgfscope}%
\pgfsys@transformshift{2.831302in}{3.682436in}%
\pgfsys@useobject{currentmarker}{}%
\end{pgfscope}%
\begin{pgfscope}%
\pgfsys@transformshift{2.845356in}{3.682442in}%
\pgfsys@useobject{currentmarker}{}%
\end{pgfscope}%
\begin{pgfscope}%
\pgfsys@transformshift{2.835887in}{3.682445in}%
\pgfsys@useobject{currentmarker}{}%
\end{pgfscope}%
\begin{pgfscope}%
\pgfsys@transformshift{2.840570in}{3.682447in}%
\pgfsys@useobject{currentmarker}{}%
\end{pgfscope}%
\end{pgfscope}%
\begin{pgfscope}%
\pgfsetbuttcap%
\pgfsetmiterjoin%
\definecolor{currentfill}{rgb}{1.000000,1.000000,1.000000}%
\pgfsetfillcolor{currentfill}%
\pgfsetlinewidth{0.000000pt}%
\definecolor{currentstroke}{rgb}{0.000000,0.000000,0.000000}%
\pgfsetstrokecolor{currentstroke}%
\pgfsetstrokeopacity{0.000000}%
\pgfsetdash{}{0pt}%
\pgfpathmoveto{\pgfqpoint{3.637789in}{3.353583in}}%
\pgfpathlineto{\pgfqpoint{4.860892in}{3.353583in}}%
\pgfpathlineto{\pgfqpoint{4.860892in}{3.961532in}}%
\pgfpathlineto{\pgfqpoint{3.637789in}{3.961532in}}%
\pgfpathclose%
\pgfusepath{fill}%
\end{pgfscope}%
\begin{pgfscope}%
\pgfpathrectangle{\pgfqpoint{3.637789in}{3.353583in}}{\pgfqpoint{1.223103in}{0.607948in}}%
\pgfusepath{clip}%
\pgfsetbuttcap%
\pgfsetmiterjoin%
\definecolor{currentfill}{rgb}{0.000000,0.000000,1.000000}%
\pgfsetfillcolor{currentfill}%
\pgfsetfillopacity{0.100000}%
\pgfsetlinewidth{0.803000pt}%
\definecolor{currentstroke}{rgb}{0.000000,0.000000,1.000000}%
\pgfsetstrokecolor{currentstroke}%
\pgfsetstrokeopacity{0.100000}%
\pgfsetdash{}{0pt}%
\pgfpathmoveto{\pgfqpoint{3.637789in}{3.659477in}}%
\pgfpathlineto{\pgfqpoint{3.637789in}{3.690686in}}%
\pgfpathlineto{\pgfqpoint{4.860892in}{3.690686in}}%
\pgfpathlineto{\pgfqpoint{4.860892in}{3.659477in}}%
\pgfpathclose%
\pgfusepath{stroke,fill}%
\end{pgfscope}%
\begin{pgfscope}%
\pgfpathrectangle{\pgfqpoint{3.637789in}{3.353583in}}{\pgfqpoint{1.223103in}{0.607948in}}%
\pgfusepath{clip}%
\pgfsetbuttcap%
\pgfsetroundjoin%
\definecolor{currentfill}{rgb}{0.000000,0.501961,0.000000}%
\pgfsetfillcolor{currentfill}%
\pgfsetfillopacity{0.500000}%
\pgfsetlinewidth{0.803000pt}%
\definecolor{currentstroke}{rgb}{0.000000,0.501961,0.000000}%
\pgfsetstrokecolor{currentstroke}%
\pgfsetstrokeopacity{0.500000}%
\pgfsetdash{}{0pt}%
\pgfpathmoveto{\pgfqpoint{3.637789in}{3.690131in}}%
\pgfpathlineto{\pgfqpoint{3.637789in}{3.661209in}}%
\pgfpathlineto{\pgfqpoint{3.878498in}{3.664544in}}%
\pgfpathlineto{\pgfqpoint{3.990029in}{3.667673in}}%
\pgfpathlineto{\pgfqpoint{4.063425in}{3.670606in}}%
\pgfpathlineto{\pgfqpoint{4.118221in}{3.673356in}}%
\pgfpathlineto{\pgfqpoint{4.161963in}{3.675932in}}%
\pgfpathlineto{\pgfqpoint{4.198371in}{3.678345in}}%
\pgfpathlineto{\pgfqpoint{4.229555in}{3.680602in}}%
\pgfpathlineto{\pgfqpoint{4.256828in}{3.682711in}}%
\pgfpathlineto{\pgfqpoint{4.281062in}{3.684678in}}%
\pgfpathlineto{\pgfqpoint{4.302867in}{3.686508in}}%
\pgfpathlineto{\pgfqpoint{4.322686in}{3.687869in}}%
\pgfpathlineto{\pgfqpoint{4.340851in}{3.688376in}}%
\pgfpathlineto{\pgfqpoint{4.357617in}{3.688955in}}%
\pgfpathlineto{\pgfqpoint{4.373183in}{3.689606in}}%
\pgfpathlineto{\pgfqpoint{4.387711in}{3.690318in}}%
\pgfpathlineto{\pgfqpoint{4.401329in}{3.691085in}}%
\pgfpathlineto{\pgfqpoint{4.414146in}{3.691896in}}%
\pgfpathlineto{\pgfqpoint{4.426250in}{3.692742in}}%
\pgfpathlineto{\pgfqpoint{4.437717in}{3.693612in}}%
\pgfpathlineto{\pgfqpoint{4.448610in}{3.694496in}}%
\pgfpathlineto{\pgfqpoint{4.458984in}{3.695380in}}%
\pgfpathlineto{\pgfqpoint{4.468886in}{3.696250in}}%
\pgfpathlineto{\pgfqpoint{4.478357in}{3.697085in}}%
\pgfpathlineto{\pgfqpoint{4.487434in}{3.697844in}}%
\pgfpathlineto{\pgfqpoint{4.496147in}{3.698360in}}%
\pgfpathlineto{\pgfqpoint{4.504525in}{3.698326in}}%
\pgfpathlineto{\pgfqpoint{4.512593in}{3.697951in}}%
\pgfpathlineto{\pgfqpoint{4.520372in}{3.697369in}}%
\pgfpathlineto{\pgfqpoint{4.527883in}{3.696600in}}%
\pgfpathlineto{\pgfqpoint{4.535144in}{3.695642in}}%
\pgfpathlineto{\pgfqpoint{4.542170in}{3.694491in}}%
\pgfpathlineto{\pgfqpoint{4.548977in}{3.693140in}}%
\pgfpathlineto{\pgfqpoint{4.555577in}{3.691580in}}%
\pgfpathlineto{\pgfqpoint{4.561983in}{3.689804in}}%
\pgfpathlineto{\pgfqpoint{4.568206in}{3.687803in}}%
\pgfpathlineto{\pgfqpoint{4.574256in}{3.685568in}}%
\pgfpathlineto{\pgfqpoint{4.580143in}{3.683090in}}%
\pgfpathlineto{\pgfqpoint{4.585874in}{3.680361in}}%
\pgfpathlineto{\pgfqpoint{4.591459in}{3.677370in}}%
\pgfpathlineto{\pgfqpoint{4.596904in}{3.674109in}}%
\pgfpathlineto{\pgfqpoint{4.602216in}{3.670569in}}%
\pgfpathlineto{\pgfqpoint{4.607401in}{3.666740in}}%
\pgfpathlineto{\pgfqpoint{4.612466in}{3.662612in}}%
\pgfpathlineto{\pgfqpoint{4.617416in}{3.658176in}}%
\pgfpathlineto{\pgfqpoint{4.622256in}{3.653424in}}%
\pgfpathlineto{\pgfqpoint{4.626991in}{3.648344in}}%
\pgfpathlineto{\pgfqpoint{4.631625in}{3.642925in}}%
\pgfpathlineto{\pgfqpoint{4.636162in}{3.637134in}}%
\pgfpathlineto{\pgfqpoint{4.640607in}{3.629176in}}%
\pgfpathlineto{\pgfqpoint{4.640607in}{3.631163in}}%
\pgfpathlineto{\pgfqpoint{4.640607in}{3.631163in}}%
\pgfpathlineto{\pgfqpoint{4.636162in}{3.639451in}}%
\pgfpathlineto{\pgfqpoint{4.631625in}{3.648499in}}%
\pgfpathlineto{\pgfqpoint{4.626991in}{3.656595in}}%
\pgfpathlineto{\pgfqpoint{4.622256in}{3.663787in}}%
\pgfpathlineto{\pgfqpoint{4.617416in}{3.670138in}}%
\pgfpathlineto{\pgfqpoint{4.612466in}{3.675713in}}%
\pgfpathlineto{\pgfqpoint{4.607401in}{3.680572in}}%
\pgfpathlineto{\pgfqpoint{4.602216in}{3.684773in}}%
\pgfpathlineto{\pgfqpoint{4.596904in}{3.688371in}}%
\pgfpathlineto{\pgfqpoint{4.591459in}{3.691417in}}%
\pgfpathlineto{\pgfqpoint{4.585874in}{3.693960in}}%
\pgfpathlineto{\pgfqpoint{4.580143in}{3.696047in}}%
\pgfpathlineto{\pgfqpoint{4.574256in}{3.697720in}}%
\pgfpathlineto{\pgfqpoint{4.568206in}{3.699021in}}%
\pgfpathlineto{\pgfqpoint{4.561983in}{3.699987in}}%
\pgfpathlineto{\pgfqpoint{4.555577in}{3.700654in}}%
\pgfpathlineto{\pgfqpoint{4.548977in}{3.701057in}}%
\pgfpathlineto{\pgfqpoint{4.542170in}{3.701226in}}%
\pgfpathlineto{\pgfqpoint{4.535144in}{3.701190in}}%
\pgfpathlineto{\pgfqpoint{4.527883in}{3.700979in}}%
\pgfpathlineto{\pgfqpoint{4.520372in}{3.700620in}}%
\pgfpathlineto{\pgfqpoint{4.512593in}{3.700142in}}%
\pgfpathlineto{\pgfqpoint{4.504525in}{3.699593in}}%
\pgfpathlineto{\pgfqpoint{4.496147in}{3.699135in}}%
\pgfpathlineto{\pgfqpoint{4.487434in}{3.699002in}}%
\pgfpathlineto{\pgfqpoint{4.478357in}{3.698909in}}%
\pgfpathlineto{\pgfqpoint{4.468886in}{3.698711in}}%
\pgfpathlineto{\pgfqpoint{4.458984in}{3.698385in}}%
\pgfpathlineto{\pgfqpoint{4.448610in}{3.697929in}}%
\pgfpathlineto{\pgfqpoint{4.437717in}{3.697344in}}%
\pgfpathlineto{\pgfqpoint{4.426250in}{3.696633in}}%
\pgfpathlineto{\pgfqpoint{4.414146in}{3.695797in}}%
\pgfpathlineto{\pgfqpoint{4.401329in}{3.694839in}}%
\pgfpathlineto{\pgfqpoint{4.387711in}{3.693758in}}%
\pgfpathlineto{\pgfqpoint{4.373183in}{3.692557in}}%
\pgfpathlineto{\pgfqpoint{4.357617in}{3.691233in}}%
\pgfpathlineto{\pgfqpoint{4.340851in}{3.689787in}}%
\pgfpathlineto{\pgfqpoint{4.322686in}{3.688222in}}%
\pgfpathlineto{\pgfqpoint{4.302867in}{3.687472in}}%
\pgfpathlineto{\pgfqpoint{4.281062in}{3.687154in}}%
\pgfpathlineto{\pgfqpoint{4.256828in}{3.686942in}}%
\pgfpathlineto{\pgfqpoint{4.229555in}{3.686844in}}%
\pgfpathlineto{\pgfqpoint{4.198371in}{3.686867in}}%
\pgfpathlineto{\pgfqpoint{4.161963in}{3.687022in}}%
\pgfpathlineto{\pgfqpoint{4.118221in}{3.687318in}}%
\pgfpathlineto{\pgfqpoint{4.063425in}{3.687765in}}%
\pgfpathlineto{\pgfqpoint{3.990029in}{3.688375in}}%
\pgfpathlineto{\pgfqpoint{3.878498in}{3.689160in}}%
\pgfpathlineto{\pgfqpoint{3.637789in}{3.690131in}}%
\pgfpathclose%
\pgfusepath{stroke,fill}%
\end{pgfscope}%
\begin{pgfscope}%
\pgfpathrectangle{\pgfqpoint{3.637789in}{3.353583in}}{\pgfqpoint{1.223103in}{0.607948in}}%
\pgfusepath{clip}%
\pgfsetroundcap%
\pgfsetroundjoin%
\pgfsetlinewidth{0.501875pt}%
\definecolor{currentstroke}{rgb}{0.000000,0.000000,1.000000}%
\pgfsetstrokecolor{currentstroke}%
\pgfsetstrokeopacity{0.800000}%
\pgfsetdash{}{0pt}%
\pgfpathmoveto{\pgfqpoint{3.637789in}{3.675081in}}%
\pgfpathlineto{\pgfqpoint{4.860892in}{3.675081in}}%
\pgfusepath{stroke}%
\end{pgfscope}%
\begin{pgfscope}%
\pgfpathrectangle{\pgfqpoint{3.637789in}{3.353583in}}{\pgfqpoint{1.223103in}{0.607948in}}%
\pgfusepath{clip}%
\pgfsetbuttcap%
\pgfsetroundjoin%
\pgfsetlinewidth{1.003750pt}%
\definecolor{currentstroke}{rgb}{0.000000,0.000000,0.000000}%
\pgfsetstrokecolor{currentstroke}%
\pgfsetdash{{3.700000pt}{1.600000pt}}{0.000000pt}%
\pgfpathmoveto{\pgfqpoint{3.637789in}{3.678551in}}%
\pgfpathlineto{\pgfqpoint{4.860892in}{3.678551in}}%
\pgfusepath{stroke}%
\end{pgfscope}%
\begin{pgfscope}%
\pgfsetroundcap%
\pgfsetroundjoin%
\pgfsetlinewidth{0.501875pt}%
\definecolor{currentstroke}{rgb}{0.000000,0.000000,1.000000}%
\pgfsetstrokecolor{currentstroke}%
\pgfsetstrokeopacity{0.800000}%
\pgfsetdash{}{0pt}%
\pgfpathmoveto{\pgfqpoint{4.438481in}{3.791586in}}%
\pgfpathquadraticcurveto{\pgfqpoint{4.375233in}{3.741913in}}{\pgfqpoint{4.311985in}{3.692240in}}%
\pgfusepath{stroke}%
\end{pgfscope}%
\begin{pgfscope}%
\pgfsetfillopacity{0.800000}%
\pgfsetstrokeopacity{0.800000}%
\definecolor{textcolor}{rgb}{0.000000,0.000000,1.000000}%
\pgfsetstrokecolor{textcolor}%
\pgfsetfillcolor{textcolor}%
\pgftext[x=4.378430in,y=3.857466in,left,base]{\color{textcolor}\sffamily\fontsize{5.647059}{6.776471}\selectfont 9.529(26)}%
\end{pgfscope}%
\begin{pgfscope}%
\pgfsetbuttcap%
\pgfsetroundjoin%
\definecolor{currentfill}{rgb}{0.150000,0.150000,0.150000}%
\pgfsetfillcolor{currentfill}%
\pgfsetlinewidth{1.003750pt}%
\definecolor{currentstroke}{rgb}{0.150000,0.150000,0.150000}%
\pgfsetstrokecolor{currentstroke}%
\pgfsetdash{}{0pt}%
\pgfsys@defobject{currentmarker}{\pgfqpoint{0.000000in}{-0.066667in}}{\pgfqpoint{0.000000in}{0.000000in}}{%
\pgfpathmoveto{\pgfqpoint{0.000000in}{0.000000in}}%
\pgfpathlineto{\pgfqpoint{0.000000in}{-0.066667in}}%
\pgfusepath{stroke,fill}%
}%
\begin{pgfscope}%
\pgfsys@transformshift{3.637789in}{3.353583in}%
\pgfsys@useobject{currentmarker}{}%
\end{pgfscope}%
\end{pgfscope}%
\begin{pgfscope}%
\pgfsetbuttcap%
\pgfsetroundjoin%
\definecolor{currentfill}{rgb}{0.150000,0.150000,0.150000}%
\pgfsetfillcolor{currentfill}%
\pgfsetlinewidth{1.003750pt}%
\definecolor{currentstroke}{rgb}{0.150000,0.150000,0.150000}%
\pgfsetstrokecolor{currentstroke}%
\pgfsetdash{}{0pt}%
\pgfsys@defobject{currentmarker}{\pgfqpoint{0.000000in}{-0.066667in}}{\pgfqpoint{0.000000in}{0.000000in}}{%
\pgfpathmoveto{\pgfqpoint{0.000000in}{0.000000in}}%
\pgfpathlineto{\pgfqpoint{0.000000in}{-0.066667in}}%
\pgfusepath{stroke,fill}%
}%
\begin{pgfscope}%
\pgfsys@transformshift{4.139198in}{3.353583in}%
\pgfsys@useobject{currentmarker}{}%
\end{pgfscope}%
\end{pgfscope}%
\begin{pgfscope}%
\pgfsetbuttcap%
\pgfsetroundjoin%
\definecolor{currentfill}{rgb}{0.150000,0.150000,0.150000}%
\pgfsetfillcolor{currentfill}%
\pgfsetlinewidth{1.003750pt}%
\definecolor{currentstroke}{rgb}{0.150000,0.150000,0.150000}%
\pgfsetstrokecolor{currentstroke}%
\pgfsetdash{}{0pt}%
\pgfsys@defobject{currentmarker}{\pgfqpoint{0.000000in}{-0.066667in}}{\pgfqpoint{0.000000in}{0.000000in}}{%
\pgfpathmoveto{\pgfqpoint{0.000000in}{0.000000in}}%
\pgfpathlineto{\pgfqpoint{0.000000in}{-0.066667in}}%
\pgfusepath{stroke,fill}%
}%
\begin{pgfscope}%
\pgfsys@transformshift{4.640607in}{3.353583in}%
\pgfsys@useobject{currentmarker}{}%
\end{pgfscope}%
\end{pgfscope}%
\begin{pgfscope}%
\pgfsetbuttcap%
\pgfsetroundjoin%
\definecolor{currentfill}{rgb}{0.150000,0.150000,0.150000}%
\pgfsetfillcolor{currentfill}%
\pgfsetlinewidth{0.803000pt}%
\definecolor{currentstroke}{rgb}{0.150000,0.150000,0.150000}%
\pgfsetstrokecolor{currentstroke}%
\pgfsetdash{}{0pt}%
\pgfsys@defobject{currentmarker}{\pgfqpoint{0.000000in}{-0.044444in}}{\pgfqpoint{0.000000in}{0.000000in}}{%
\pgfpathmoveto{\pgfqpoint{0.000000in}{0.000000in}}%
\pgfpathlineto{\pgfqpoint{0.000000in}{-0.044444in}}%
\pgfusepath{stroke,fill}%
}%
\begin{pgfscope}%
\pgfsys@transformshift{3.788728in}{3.353583in}%
\pgfsys@useobject{currentmarker}{}%
\end{pgfscope}%
\end{pgfscope}%
\begin{pgfscope}%
\pgfsetbuttcap%
\pgfsetroundjoin%
\definecolor{currentfill}{rgb}{0.150000,0.150000,0.150000}%
\pgfsetfillcolor{currentfill}%
\pgfsetlinewidth{0.803000pt}%
\definecolor{currentstroke}{rgb}{0.150000,0.150000,0.150000}%
\pgfsetstrokecolor{currentstroke}%
\pgfsetdash{}{0pt}%
\pgfsys@defobject{currentmarker}{\pgfqpoint{0.000000in}{-0.044444in}}{\pgfqpoint{0.000000in}{0.000000in}}{%
\pgfpathmoveto{\pgfqpoint{0.000000in}{0.000000in}}%
\pgfpathlineto{\pgfqpoint{0.000000in}{-0.044444in}}%
\pgfusepath{stroke,fill}%
}%
\begin{pgfscope}%
\pgfsys@transformshift{3.877021in}{3.353583in}%
\pgfsys@useobject{currentmarker}{}%
\end{pgfscope}%
\end{pgfscope}%
\begin{pgfscope}%
\pgfsetbuttcap%
\pgfsetroundjoin%
\definecolor{currentfill}{rgb}{0.150000,0.150000,0.150000}%
\pgfsetfillcolor{currentfill}%
\pgfsetlinewidth{0.803000pt}%
\definecolor{currentstroke}{rgb}{0.150000,0.150000,0.150000}%
\pgfsetstrokecolor{currentstroke}%
\pgfsetdash{}{0pt}%
\pgfsys@defobject{currentmarker}{\pgfqpoint{0.000000in}{-0.044444in}}{\pgfqpoint{0.000000in}{0.000000in}}{%
\pgfpathmoveto{\pgfqpoint{0.000000in}{0.000000in}}%
\pgfpathlineto{\pgfqpoint{0.000000in}{-0.044444in}}%
\pgfusepath{stroke,fill}%
}%
\begin{pgfscope}%
\pgfsys@transformshift{3.939667in}{3.353583in}%
\pgfsys@useobject{currentmarker}{}%
\end{pgfscope}%
\end{pgfscope}%
\begin{pgfscope}%
\pgfsetbuttcap%
\pgfsetroundjoin%
\definecolor{currentfill}{rgb}{0.150000,0.150000,0.150000}%
\pgfsetfillcolor{currentfill}%
\pgfsetlinewidth{0.803000pt}%
\definecolor{currentstroke}{rgb}{0.150000,0.150000,0.150000}%
\pgfsetstrokecolor{currentstroke}%
\pgfsetdash{}{0pt}%
\pgfsys@defobject{currentmarker}{\pgfqpoint{0.000000in}{-0.044444in}}{\pgfqpoint{0.000000in}{0.000000in}}{%
\pgfpathmoveto{\pgfqpoint{0.000000in}{0.000000in}}%
\pgfpathlineto{\pgfqpoint{0.000000in}{-0.044444in}}%
\pgfusepath{stroke,fill}%
}%
\begin{pgfscope}%
\pgfsys@transformshift{3.988258in}{3.353583in}%
\pgfsys@useobject{currentmarker}{}%
\end{pgfscope}%
\end{pgfscope}%
\begin{pgfscope}%
\pgfsetbuttcap%
\pgfsetroundjoin%
\definecolor{currentfill}{rgb}{0.150000,0.150000,0.150000}%
\pgfsetfillcolor{currentfill}%
\pgfsetlinewidth{0.803000pt}%
\definecolor{currentstroke}{rgb}{0.150000,0.150000,0.150000}%
\pgfsetstrokecolor{currentstroke}%
\pgfsetdash{}{0pt}%
\pgfsys@defobject{currentmarker}{\pgfqpoint{0.000000in}{-0.044444in}}{\pgfqpoint{0.000000in}{0.000000in}}{%
\pgfpathmoveto{\pgfqpoint{0.000000in}{0.000000in}}%
\pgfpathlineto{\pgfqpoint{0.000000in}{-0.044444in}}%
\pgfusepath{stroke,fill}%
}%
\begin{pgfscope}%
\pgfsys@transformshift{4.027961in}{3.353583in}%
\pgfsys@useobject{currentmarker}{}%
\end{pgfscope}%
\end{pgfscope}%
\begin{pgfscope}%
\pgfsetbuttcap%
\pgfsetroundjoin%
\definecolor{currentfill}{rgb}{0.150000,0.150000,0.150000}%
\pgfsetfillcolor{currentfill}%
\pgfsetlinewidth{0.803000pt}%
\definecolor{currentstroke}{rgb}{0.150000,0.150000,0.150000}%
\pgfsetstrokecolor{currentstroke}%
\pgfsetdash{}{0pt}%
\pgfsys@defobject{currentmarker}{\pgfqpoint{0.000000in}{-0.044444in}}{\pgfqpoint{0.000000in}{0.000000in}}{%
\pgfpathmoveto{\pgfqpoint{0.000000in}{0.000000in}}%
\pgfpathlineto{\pgfqpoint{0.000000in}{-0.044444in}}%
\pgfusepath{stroke,fill}%
}%
\begin{pgfscope}%
\pgfsys@transformshift{4.061528in}{3.353583in}%
\pgfsys@useobject{currentmarker}{}%
\end{pgfscope}%
\end{pgfscope}%
\begin{pgfscope}%
\pgfsetbuttcap%
\pgfsetroundjoin%
\definecolor{currentfill}{rgb}{0.150000,0.150000,0.150000}%
\pgfsetfillcolor{currentfill}%
\pgfsetlinewidth{0.803000pt}%
\definecolor{currentstroke}{rgb}{0.150000,0.150000,0.150000}%
\pgfsetstrokecolor{currentstroke}%
\pgfsetdash{}{0pt}%
\pgfsys@defobject{currentmarker}{\pgfqpoint{0.000000in}{-0.044444in}}{\pgfqpoint{0.000000in}{0.000000in}}{%
\pgfpathmoveto{\pgfqpoint{0.000000in}{0.000000in}}%
\pgfpathlineto{\pgfqpoint{0.000000in}{-0.044444in}}%
\pgfusepath{stroke,fill}%
}%
\begin{pgfscope}%
\pgfsys@transformshift{4.090606in}{3.353583in}%
\pgfsys@useobject{currentmarker}{}%
\end{pgfscope}%
\end{pgfscope}%
\begin{pgfscope}%
\pgfsetbuttcap%
\pgfsetroundjoin%
\definecolor{currentfill}{rgb}{0.150000,0.150000,0.150000}%
\pgfsetfillcolor{currentfill}%
\pgfsetlinewidth{0.803000pt}%
\definecolor{currentstroke}{rgb}{0.150000,0.150000,0.150000}%
\pgfsetstrokecolor{currentstroke}%
\pgfsetdash{}{0pt}%
\pgfsys@defobject{currentmarker}{\pgfqpoint{0.000000in}{-0.044444in}}{\pgfqpoint{0.000000in}{0.000000in}}{%
\pgfpathmoveto{\pgfqpoint{0.000000in}{0.000000in}}%
\pgfpathlineto{\pgfqpoint{0.000000in}{-0.044444in}}%
\pgfusepath{stroke,fill}%
}%
\begin{pgfscope}%
\pgfsys@transformshift{4.116254in}{3.353583in}%
\pgfsys@useobject{currentmarker}{}%
\end{pgfscope}%
\end{pgfscope}%
\begin{pgfscope}%
\pgfsetbuttcap%
\pgfsetroundjoin%
\definecolor{currentfill}{rgb}{0.150000,0.150000,0.150000}%
\pgfsetfillcolor{currentfill}%
\pgfsetlinewidth{0.803000pt}%
\definecolor{currentstroke}{rgb}{0.150000,0.150000,0.150000}%
\pgfsetstrokecolor{currentstroke}%
\pgfsetdash{}{0pt}%
\pgfsys@defobject{currentmarker}{\pgfqpoint{0.000000in}{-0.044444in}}{\pgfqpoint{0.000000in}{0.000000in}}{%
\pgfpathmoveto{\pgfqpoint{0.000000in}{0.000000in}}%
\pgfpathlineto{\pgfqpoint{0.000000in}{-0.044444in}}%
\pgfusepath{stroke,fill}%
}%
\begin{pgfscope}%
\pgfsys@transformshift{4.290137in}{3.353583in}%
\pgfsys@useobject{currentmarker}{}%
\end{pgfscope}%
\end{pgfscope}%
\begin{pgfscope}%
\pgfsetbuttcap%
\pgfsetroundjoin%
\definecolor{currentfill}{rgb}{0.150000,0.150000,0.150000}%
\pgfsetfillcolor{currentfill}%
\pgfsetlinewidth{0.803000pt}%
\definecolor{currentstroke}{rgb}{0.150000,0.150000,0.150000}%
\pgfsetstrokecolor{currentstroke}%
\pgfsetdash{}{0pt}%
\pgfsys@defobject{currentmarker}{\pgfqpoint{0.000000in}{-0.044444in}}{\pgfqpoint{0.000000in}{0.000000in}}{%
\pgfpathmoveto{\pgfqpoint{0.000000in}{0.000000in}}%
\pgfpathlineto{\pgfqpoint{0.000000in}{-0.044444in}}%
\pgfusepath{stroke,fill}%
}%
\begin{pgfscope}%
\pgfsys@transformshift{4.378430in}{3.353583in}%
\pgfsys@useobject{currentmarker}{}%
\end{pgfscope}%
\end{pgfscope}%
\begin{pgfscope}%
\pgfsetbuttcap%
\pgfsetroundjoin%
\definecolor{currentfill}{rgb}{0.150000,0.150000,0.150000}%
\pgfsetfillcolor{currentfill}%
\pgfsetlinewidth{0.803000pt}%
\definecolor{currentstroke}{rgb}{0.150000,0.150000,0.150000}%
\pgfsetstrokecolor{currentstroke}%
\pgfsetdash{}{0pt}%
\pgfsys@defobject{currentmarker}{\pgfqpoint{0.000000in}{-0.044444in}}{\pgfqpoint{0.000000in}{0.000000in}}{%
\pgfpathmoveto{\pgfqpoint{0.000000in}{0.000000in}}%
\pgfpathlineto{\pgfqpoint{0.000000in}{-0.044444in}}%
\pgfusepath{stroke,fill}%
}%
\begin{pgfscope}%
\pgfsys@transformshift{4.441076in}{3.353583in}%
\pgfsys@useobject{currentmarker}{}%
\end{pgfscope}%
\end{pgfscope}%
\begin{pgfscope}%
\pgfsetbuttcap%
\pgfsetroundjoin%
\definecolor{currentfill}{rgb}{0.150000,0.150000,0.150000}%
\pgfsetfillcolor{currentfill}%
\pgfsetlinewidth{0.803000pt}%
\definecolor{currentstroke}{rgb}{0.150000,0.150000,0.150000}%
\pgfsetstrokecolor{currentstroke}%
\pgfsetdash{}{0pt}%
\pgfsys@defobject{currentmarker}{\pgfqpoint{0.000000in}{-0.044444in}}{\pgfqpoint{0.000000in}{0.000000in}}{%
\pgfpathmoveto{\pgfqpoint{0.000000in}{0.000000in}}%
\pgfpathlineto{\pgfqpoint{0.000000in}{-0.044444in}}%
\pgfusepath{stroke,fill}%
}%
\begin{pgfscope}%
\pgfsys@transformshift{4.489667in}{3.353583in}%
\pgfsys@useobject{currentmarker}{}%
\end{pgfscope}%
\end{pgfscope}%
\begin{pgfscope}%
\pgfsetbuttcap%
\pgfsetroundjoin%
\definecolor{currentfill}{rgb}{0.150000,0.150000,0.150000}%
\pgfsetfillcolor{currentfill}%
\pgfsetlinewidth{0.803000pt}%
\definecolor{currentstroke}{rgb}{0.150000,0.150000,0.150000}%
\pgfsetstrokecolor{currentstroke}%
\pgfsetdash{}{0pt}%
\pgfsys@defobject{currentmarker}{\pgfqpoint{0.000000in}{-0.044444in}}{\pgfqpoint{0.000000in}{0.000000in}}{%
\pgfpathmoveto{\pgfqpoint{0.000000in}{0.000000in}}%
\pgfpathlineto{\pgfqpoint{0.000000in}{-0.044444in}}%
\pgfusepath{stroke,fill}%
}%
\begin{pgfscope}%
\pgfsys@transformshift{4.529370in}{3.353583in}%
\pgfsys@useobject{currentmarker}{}%
\end{pgfscope}%
\end{pgfscope}%
\begin{pgfscope}%
\pgfsetbuttcap%
\pgfsetroundjoin%
\definecolor{currentfill}{rgb}{0.150000,0.150000,0.150000}%
\pgfsetfillcolor{currentfill}%
\pgfsetlinewidth{0.803000pt}%
\definecolor{currentstroke}{rgb}{0.150000,0.150000,0.150000}%
\pgfsetstrokecolor{currentstroke}%
\pgfsetdash{}{0pt}%
\pgfsys@defobject{currentmarker}{\pgfqpoint{0.000000in}{-0.044444in}}{\pgfqpoint{0.000000in}{0.000000in}}{%
\pgfpathmoveto{\pgfqpoint{0.000000in}{0.000000in}}%
\pgfpathlineto{\pgfqpoint{0.000000in}{-0.044444in}}%
\pgfusepath{stroke,fill}%
}%
\begin{pgfscope}%
\pgfsys@transformshift{4.562937in}{3.353583in}%
\pgfsys@useobject{currentmarker}{}%
\end{pgfscope}%
\end{pgfscope}%
\begin{pgfscope}%
\pgfsetbuttcap%
\pgfsetroundjoin%
\definecolor{currentfill}{rgb}{0.150000,0.150000,0.150000}%
\pgfsetfillcolor{currentfill}%
\pgfsetlinewidth{0.803000pt}%
\definecolor{currentstroke}{rgb}{0.150000,0.150000,0.150000}%
\pgfsetstrokecolor{currentstroke}%
\pgfsetdash{}{0pt}%
\pgfsys@defobject{currentmarker}{\pgfqpoint{0.000000in}{-0.044444in}}{\pgfqpoint{0.000000in}{0.000000in}}{%
\pgfpathmoveto{\pgfqpoint{0.000000in}{0.000000in}}%
\pgfpathlineto{\pgfqpoint{0.000000in}{-0.044444in}}%
\pgfusepath{stroke,fill}%
}%
\begin{pgfscope}%
\pgfsys@transformshift{4.592015in}{3.353583in}%
\pgfsys@useobject{currentmarker}{}%
\end{pgfscope}%
\end{pgfscope}%
\begin{pgfscope}%
\pgfsetbuttcap%
\pgfsetroundjoin%
\definecolor{currentfill}{rgb}{0.150000,0.150000,0.150000}%
\pgfsetfillcolor{currentfill}%
\pgfsetlinewidth{0.803000pt}%
\definecolor{currentstroke}{rgb}{0.150000,0.150000,0.150000}%
\pgfsetstrokecolor{currentstroke}%
\pgfsetdash{}{0pt}%
\pgfsys@defobject{currentmarker}{\pgfqpoint{0.000000in}{-0.044444in}}{\pgfqpoint{0.000000in}{0.000000in}}{%
\pgfpathmoveto{\pgfqpoint{0.000000in}{0.000000in}}%
\pgfpathlineto{\pgfqpoint{0.000000in}{-0.044444in}}%
\pgfusepath{stroke,fill}%
}%
\begin{pgfscope}%
\pgfsys@transformshift{4.617663in}{3.353583in}%
\pgfsys@useobject{currentmarker}{}%
\end{pgfscope}%
\end{pgfscope}%
\begin{pgfscope}%
\pgfsetbuttcap%
\pgfsetroundjoin%
\definecolor{currentfill}{rgb}{0.150000,0.150000,0.150000}%
\pgfsetfillcolor{currentfill}%
\pgfsetlinewidth{0.803000pt}%
\definecolor{currentstroke}{rgb}{0.150000,0.150000,0.150000}%
\pgfsetstrokecolor{currentstroke}%
\pgfsetdash{}{0pt}%
\pgfsys@defobject{currentmarker}{\pgfqpoint{0.000000in}{-0.044444in}}{\pgfqpoint{0.000000in}{0.000000in}}{%
\pgfpathmoveto{\pgfqpoint{0.000000in}{0.000000in}}%
\pgfpathlineto{\pgfqpoint{0.000000in}{-0.044444in}}%
\pgfusepath{stroke,fill}%
}%
\begin{pgfscope}%
\pgfsys@transformshift{4.791546in}{3.353583in}%
\pgfsys@useobject{currentmarker}{}%
\end{pgfscope}%
\end{pgfscope}%
\begin{pgfscope}%
\pgfsetbuttcap%
\pgfsetroundjoin%
\definecolor{currentfill}{rgb}{0.150000,0.150000,0.150000}%
\pgfsetfillcolor{currentfill}%
\pgfsetlinewidth{1.003750pt}%
\definecolor{currentstroke}{rgb}{0.150000,0.150000,0.150000}%
\pgfsetstrokecolor{currentstroke}%
\pgfsetdash{}{0pt}%
\pgfsys@defobject{currentmarker}{\pgfqpoint{-0.066667in}{0.000000in}}{\pgfqpoint{0.000000in}{0.000000in}}{%
\pgfpathmoveto{\pgfqpoint{0.000000in}{0.000000in}}%
\pgfpathlineto{\pgfqpoint{-0.066667in}{0.000000in}}%
\pgfusepath{stroke,fill}%
}%
\begin{pgfscope}%
\pgfsys@transformshift{3.637789in}{3.353583in}%
\pgfsys@useobject{currentmarker}{}%
\end{pgfscope}%
\end{pgfscope}%
\begin{pgfscope}%
\pgfsetbuttcap%
\pgfsetroundjoin%
\definecolor{currentfill}{rgb}{0.150000,0.150000,0.150000}%
\pgfsetfillcolor{currentfill}%
\pgfsetlinewidth{1.003750pt}%
\definecolor{currentstroke}{rgb}{0.150000,0.150000,0.150000}%
\pgfsetstrokecolor{currentstroke}%
\pgfsetdash{}{0pt}%
\pgfsys@defobject{currentmarker}{\pgfqpoint{-0.066667in}{0.000000in}}{\pgfqpoint{0.000000in}{0.000000in}}{%
\pgfpathmoveto{\pgfqpoint{0.000000in}{0.000000in}}%
\pgfpathlineto{\pgfqpoint{-0.066667in}{0.000000in}}%
\pgfusepath{stroke,fill}%
}%
\begin{pgfscope}%
\pgfsys@transformshift{3.637789in}{3.678551in}%
\pgfsys@useobject{currentmarker}{}%
\end{pgfscope}%
\end{pgfscope}%
\begin{pgfscope}%
\pgfsetbuttcap%
\pgfsetroundjoin%
\definecolor{currentfill}{rgb}{0.150000,0.150000,0.150000}%
\pgfsetfillcolor{currentfill}%
\pgfsetlinewidth{1.003750pt}%
\definecolor{currentstroke}{rgb}{0.150000,0.150000,0.150000}%
\pgfsetstrokecolor{currentstroke}%
\pgfsetdash{}{0pt}%
\pgfsys@defobject{currentmarker}{\pgfqpoint{-0.066667in}{0.000000in}}{\pgfqpoint{0.000000in}{0.000000in}}{%
\pgfpathmoveto{\pgfqpoint{0.000000in}{0.000000in}}%
\pgfpathlineto{\pgfqpoint{-0.066667in}{0.000000in}}%
\pgfusepath{stroke,fill}%
}%
\begin{pgfscope}%
\pgfsys@transformshift{3.637789in}{3.961532in}%
\pgfsys@useobject{currentmarker}{}%
\end{pgfscope}%
\end{pgfscope}%
\begin{pgfscope}%
\pgfpathrectangle{\pgfqpoint{3.637789in}{3.353583in}}{\pgfqpoint{1.223103in}{0.607948in}}%
\pgfusepath{clip}%
\pgfsetroundcap%
\pgfsetroundjoin%
\pgfsetlinewidth{1.204500pt}%
\definecolor{currentstroke}{rgb}{0.000000,0.501961,0.000000}%
\pgfsetstrokecolor{currentstroke}%
\pgfsetdash{}{0pt}%
\pgfpathmoveto{\pgfqpoint{3.637789in}{3.675670in}}%
\pgfpathlineto{\pgfqpoint{3.878498in}{3.676852in}}%
\pgfpathlineto{\pgfqpoint{3.990029in}{3.678024in}}%
\pgfpathlineto{\pgfqpoint{4.063425in}{3.679186in}}%
\pgfpathlineto{\pgfqpoint{4.118221in}{3.680337in}}%
\pgfpathlineto{\pgfqpoint{4.161963in}{3.681477in}}%
\pgfpathlineto{\pgfqpoint{4.198371in}{3.682606in}}%
\pgfpathlineto{\pgfqpoint{4.229555in}{3.683723in}}%
\pgfpathlineto{\pgfqpoint{4.256828in}{3.684827in}}%
\pgfpathlineto{\pgfqpoint{4.281062in}{3.685916in}}%
\pgfpathlineto{\pgfqpoint{4.302867in}{3.686990in}}%
\pgfpathlineto{\pgfqpoint{4.322686in}{3.688046in}}%
\pgfpathlineto{\pgfqpoint{4.340851in}{3.689081in}}%
\pgfpathlineto{\pgfqpoint{4.357617in}{3.690094in}}%
\pgfpathlineto{\pgfqpoint{4.373183in}{3.691081in}}%
\pgfpathlineto{\pgfqpoint{4.387711in}{3.692038in}}%
\pgfpathlineto{\pgfqpoint{4.401329in}{3.692962in}}%
\pgfpathlineto{\pgfqpoint{4.414146in}{3.693846in}}%
\pgfpathlineto{\pgfqpoint{4.426250in}{3.694687in}}%
\pgfpathlineto{\pgfqpoint{4.437717in}{3.695478in}}%
\pgfpathlineto{\pgfqpoint{4.448610in}{3.696212in}}%
\pgfpathlineto{\pgfqpoint{4.458984in}{3.696882in}}%
\pgfpathlineto{\pgfqpoint{4.468886in}{3.697480in}}%
\pgfpathlineto{\pgfqpoint{4.478357in}{3.697997in}}%
\pgfpathlineto{\pgfqpoint{4.487434in}{3.698423in}}%
\pgfpathlineto{\pgfqpoint{4.496147in}{3.698748in}}%
\pgfpathlineto{\pgfqpoint{4.504525in}{3.698960in}}%
\pgfpathlineto{\pgfqpoint{4.512593in}{3.699046in}}%
\pgfpathlineto{\pgfqpoint{4.520372in}{3.698994in}}%
\pgfpathlineto{\pgfqpoint{4.527883in}{3.698789in}}%
\pgfpathlineto{\pgfqpoint{4.535144in}{3.698416in}}%
\pgfpathlineto{\pgfqpoint{4.542170in}{3.697858in}}%
\pgfpathlineto{\pgfqpoint{4.548977in}{3.697098in}}%
\pgfpathlineto{\pgfqpoint{4.555577in}{3.696117in}}%
\pgfpathlineto{\pgfqpoint{4.561983in}{3.694896in}}%
\pgfpathlineto{\pgfqpoint{4.568206in}{3.693412in}}%
\pgfpathlineto{\pgfqpoint{4.574256in}{3.691644in}}%
\pgfpathlineto{\pgfqpoint{4.580143in}{3.689569in}}%
\pgfpathlineto{\pgfqpoint{4.585874in}{3.687161in}}%
\pgfpathlineto{\pgfqpoint{4.591459in}{3.684394in}}%
\pgfpathlineto{\pgfqpoint{4.596904in}{3.681240in}}%
\pgfpathlineto{\pgfqpoint{4.602216in}{3.677671in}}%
\pgfpathlineto{\pgfqpoint{4.607401in}{3.673656in}}%
\pgfpathlineto{\pgfqpoint{4.612466in}{3.669162in}}%
\pgfpathlineto{\pgfqpoint{4.617416in}{3.664157in}}%
\pgfpathlineto{\pgfqpoint{4.622256in}{3.658605in}}%
\pgfpathlineto{\pgfqpoint{4.626991in}{3.652470in}}%
\pgfpathlineto{\pgfqpoint{4.631625in}{3.645712in}}%
\pgfpathlineto{\pgfqpoint{4.636162in}{3.638293in}}%
\pgfpathlineto{\pgfqpoint{4.640607in}{3.630169in}}%
\pgfusepath{stroke}%
\end{pgfscope}%
\begin{pgfscope}%
\pgfsetrectcap%
\pgfsetmiterjoin%
\pgfsetlinewidth{1.003750pt}%
\definecolor{currentstroke}{rgb}{0.150000,0.150000,0.150000}%
\pgfsetstrokecolor{currentstroke}%
\pgfsetdash{}{0pt}%
\pgfpathmoveto{\pgfqpoint{3.637789in}{3.353583in}}%
\pgfpathlineto{\pgfqpoint{3.637789in}{3.961532in}}%
\pgfusepath{stroke}%
\end{pgfscope}%
\begin{pgfscope}%
\pgfsetrectcap%
\pgfsetmiterjoin%
\pgfsetlinewidth{1.003750pt}%
\definecolor{currentstroke}{rgb}{0.150000,0.150000,0.150000}%
\pgfsetstrokecolor{currentstroke}%
\pgfsetdash{}{0pt}%
\pgfpathmoveto{\pgfqpoint{3.637789in}{3.353583in}}%
\pgfpathlineto{\pgfqpoint{4.860892in}{3.353583in}}%
\pgfusepath{stroke}%
\end{pgfscope}%
\begin{pgfscope}%
\pgfpathrectangle{\pgfqpoint{3.637789in}{3.353583in}}{\pgfqpoint{1.223103in}{0.607948in}}%
\pgfusepath{clip}%
\pgfsetbuttcap%
\pgfsetroundjoin%
\definecolor{currentfill}{rgb}{0.000000,0.000000,0.000000}%
\pgfsetfillcolor{currentfill}%
\pgfsetlinewidth{1.003750pt}%
\definecolor{currentstroke}{rgb}{0.000000,0.000000,0.000000}%
\pgfsetstrokecolor{currentstroke}%
\pgfsetdash{}{0pt}%
\pgfsys@defobject{currentmarker}{\pgfqpoint{-0.013889in}{-0.013889in}}{\pgfqpoint{0.013889in}{0.013889in}}{%
\pgfpathmoveto{\pgfqpoint{0.000000in}{-0.013889in}}%
\pgfpathcurveto{\pgfqpoint{0.003683in}{-0.013889in}}{\pgfqpoint{0.007216in}{-0.012425in}}{\pgfqpoint{0.009821in}{-0.009821in}}%
\pgfpathcurveto{\pgfqpoint{0.012425in}{-0.007216in}}{\pgfqpoint{0.013889in}{-0.003683in}}{\pgfqpoint{0.013889in}{0.000000in}}%
\pgfpathcurveto{\pgfqpoint{0.013889in}{0.003683in}}{\pgfqpoint{0.012425in}{0.007216in}}{\pgfqpoint{0.009821in}{0.009821in}}%
\pgfpathcurveto{\pgfqpoint{0.007216in}{0.012425in}}{\pgfqpoint{0.003683in}{0.013889in}}{\pgfqpoint{0.000000in}{0.013889in}}%
\pgfpathcurveto{\pgfqpoint{-0.003683in}{0.013889in}}{\pgfqpoint{-0.007216in}{0.012425in}}{\pgfqpoint{-0.009821in}{0.009821in}}%
\pgfpathcurveto{\pgfqpoint{-0.012425in}{0.007216in}}{\pgfqpoint{-0.013889in}{0.003683in}}{\pgfqpoint{-0.013889in}{0.000000in}}%
\pgfpathcurveto{\pgfqpoint{-0.013889in}{-0.003683in}}{\pgfqpoint{-0.012425in}{-0.007216in}}{\pgfqpoint{-0.009821in}{-0.009821in}}%
\pgfpathcurveto{\pgfqpoint{-0.007216in}{-0.012425in}}{\pgfqpoint{-0.003683in}{-0.013889in}}{\pgfqpoint{0.000000in}{-0.013889in}}%
\pgfpathclose%
\pgfusepath{stroke,fill}%
}%
\begin{pgfscope}%
\pgfsys@transformshift{4.640607in}{3.629560in}%
\pgfsys@useobject{currentmarker}{}%
\end{pgfscope}%
\begin{pgfscope}%
\pgfsys@transformshift{4.290137in}{3.686721in}%
\pgfsys@useobject{currentmarker}{}%
\end{pgfscope}%
\begin{pgfscope}%
\pgfsys@transformshift{4.294536in}{3.686886in}%
\pgfsys@useobject{currentmarker}{}%
\end{pgfscope}%
\begin{pgfscope}%
\pgfsys@transformshift{4.299026in}{3.687059in}%
\pgfsys@useobject{currentmarker}{}%
\end{pgfscope}%
\begin{pgfscope}%
\pgfsys@transformshift{4.303611in}{3.687238in}%
\pgfsys@useobject{currentmarker}{}%
\end{pgfscope}%
\begin{pgfscope}%
\pgfsys@transformshift{4.308294in}{3.687426in}%
\pgfsys@useobject{currentmarker}{}%
\end{pgfscope}%
\begin{pgfscope}%
\pgfsys@transformshift{4.313080in}{3.687621in}%
\pgfsys@useobject{currentmarker}{}%
\end{pgfscope}%
\begin{pgfscope}%
\pgfsys@transformshift{4.317974in}{3.687826in}%
\pgfsys@useobject{currentmarker}{}%
\end{pgfscope}%
\begin{pgfscope}%
\pgfsys@transformshift{4.322980in}{3.688040in}%
\pgfsys@useobject{currentmarker}{}%
\end{pgfscope}%
\begin{pgfscope}%
\pgfsys@transformshift{4.328104in}{3.688263in}%
\pgfsys@useobject{currentmarker}{}%
\end{pgfscope}%
\begin{pgfscope}%
\pgfsys@transformshift{4.333351in}{3.688498in}%
\pgfsys@useobject{currentmarker}{}%
\end{pgfscope}%
\begin{pgfscope}%
\pgfsys@transformshift{4.338728in}{3.688744in}%
\pgfsys@useobject{currentmarker}{}%
\end{pgfscope}%
\begin{pgfscope}%
\pgfsys@transformshift{4.367806in}{3.690178in}%
\pgfsys@useobject{currentmarker}{}%
\end{pgfscope}%
\begin{pgfscope}%
\pgfsys@transformshift{4.401374in}{3.692065in}%
\pgfsys@useobject{currentmarker}{}%
\end{pgfscope}%
\begin{pgfscope}%
\pgfsys@transformshift{4.441076in}{3.694621in}%
\pgfsys@useobject{currentmarker}{}%
\end{pgfscope}%
\begin{pgfscope}%
\pgfsys@transformshift{4.489667in}{3.697989in}%
\pgfsys@useobject{currentmarker}{}%
\end{pgfscope}%
\begin{pgfscope}%
\pgfsys@transformshift{4.552313in}{3.699470in}%
\pgfsys@useobject{currentmarker}{}%
\end{pgfscope}%
\end{pgfscope}%
\begin{pgfscope}%
\pgfsetbuttcap%
\pgfsetmiterjoin%
\definecolor{currentfill}{rgb}{1.000000,1.000000,1.000000}%
\pgfsetfillcolor{currentfill}%
\pgfsetlinewidth{0.000000pt}%
\definecolor{currentstroke}{rgb}{0.000000,0.000000,0.000000}%
\pgfsetstrokecolor{currentstroke}%
\pgfsetstrokeopacity{0.000000}%
\pgfsetdash{}{0pt}%
\pgfpathmoveto{\pgfqpoint{5.105513in}{3.353583in}}%
\pgfpathlineto{\pgfqpoint{6.328616in}{3.353583in}}%
\pgfpathlineto{\pgfqpoint{6.328616in}{3.961532in}}%
\pgfpathlineto{\pgfqpoint{5.105513in}{3.961532in}}%
\pgfpathclose%
\pgfusepath{fill}%
\end{pgfscope}%
\begin{pgfscope}%
\pgfpathrectangle{\pgfqpoint{5.105513in}{3.353583in}}{\pgfqpoint{1.223103in}{0.607948in}}%
\pgfusepath{clip}%
\pgfsetbuttcap%
\pgfsetmiterjoin%
\definecolor{currentfill}{rgb}{0.000000,0.000000,1.000000}%
\pgfsetfillcolor{currentfill}%
\pgfsetfillopacity{0.100000}%
\pgfsetlinewidth{0.803000pt}%
\definecolor{currentstroke}{rgb}{0.000000,0.000000,1.000000}%
\pgfsetstrokecolor{currentstroke}%
\pgfsetstrokeopacity{0.100000}%
\pgfsetdash{}{0pt}%
\pgfpathmoveto{\pgfqpoint{5.105513in}{3.677520in}}%
\pgfpathlineto{\pgfqpoint{5.105513in}{3.679407in}}%
\pgfpathlineto{\pgfqpoint{6.328616in}{3.679407in}}%
\pgfpathlineto{\pgfqpoint{6.328616in}{3.677520in}}%
\pgfpathclose%
\pgfusepath{stroke,fill}%
\end{pgfscope}%
\begin{pgfscope}%
\pgfpathrectangle{\pgfqpoint{5.105513in}{3.353583in}}{\pgfqpoint{1.223103in}{0.607948in}}%
\pgfusepath{clip}%
\pgfsetbuttcap%
\pgfsetroundjoin%
\definecolor{currentfill}{rgb}{0.000000,0.501961,0.000000}%
\pgfsetfillcolor{currentfill}%
\pgfsetfillopacity{0.500000}%
\pgfsetlinewidth{0.803000pt}%
\definecolor{currentstroke}{rgb}{0.000000,0.501961,0.000000}%
\pgfsetstrokecolor{currentstroke}%
\pgfsetstrokeopacity{0.500000}%
\pgfsetdash{}{0pt}%
\pgfpathmoveto{\pgfqpoint{5.105513in}{3.679822in}}%
\pgfpathlineto{\pgfqpoint{5.105513in}{3.678140in}}%
\pgfpathlineto{\pgfqpoint{5.346222in}{3.679355in}}%
\pgfpathlineto{\pgfqpoint{5.457753in}{3.680527in}}%
\pgfpathlineto{\pgfqpoint{5.531149in}{3.681661in}}%
\pgfpathlineto{\pgfqpoint{5.585945in}{3.682764in}}%
\pgfpathlineto{\pgfqpoint{5.629687in}{3.683843in}}%
\pgfpathlineto{\pgfqpoint{5.666095in}{3.684902in}}%
\pgfpathlineto{\pgfqpoint{5.697279in}{3.685946in}}%
\pgfpathlineto{\pgfqpoint{5.724552in}{3.686979in}}%
\pgfpathlineto{\pgfqpoint{5.748786in}{3.688004in}}%
\pgfpathlineto{\pgfqpoint{5.770591in}{3.689023in}}%
\pgfpathlineto{\pgfqpoint{5.790410in}{3.690034in}}%
\pgfpathlineto{\pgfqpoint{5.808575in}{3.691020in}}%
\pgfpathlineto{\pgfqpoint{5.825341in}{3.692010in}}%
\pgfpathlineto{\pgfqpoint{5.840907in}{3.693006in}}%
\pgfpathlineto{\pgfqpoint{5.855435in}{3.694007in}}%
\pgfpathlineto{\pgfqpoint{5.869053in}{3.695015in}}%
\pgfpathlineto{\pgfqpoint{5.881870in}{3.696028in}}%
\pgfpathlineto{\pgfqpoint{5.893974in}{3.697047in}}%
\pgfpathlineto{\pgfqpoint{5.905441in}{3.698072in}}%
\pgfpathlineto{\pgfqpoint{5.916334in}{3.699102in}}%
\pgfpathlineto{\pgfqpoint{5.926708in}{3.700138in}}%
\pgfpathlineto{\pgfqpoint{5.936610in}{3.701178in}}%
\pgfpathlineto{\pgfqpoint{5.946082in}{3.702223in}}%
\pgfpathlineto{\pgfqpoint{5.955158in}{3.703272in}}%
\pgfpathlineto{\pgfqpoint{5.963871in}{3.704324in}}%
\pgfpathlineto{\pgfqpoint{5.972249in}{3.705378in}}%
\pgfpathlineto{\pgfqpoint{5.980317in}{3.706432in}}%
\pgfpathlineto{\pgfqpoint{5.988096in}{3.707486in}}%
\pgfpathlineto{\pgfqpoint{5.995607in}{3.708539in}}%
\pgfpathlineto{\pgfqpoint{6.002868in}{3.709592in}}%
\pgfpathlineto{\pgfqpoint{6.009894in}{3.710647in}}%
\pgfpathlineto{\pgfqpoint{6.016701in}{3.711704in}}%
\pgfpathlineto{\pgfqpoint{6.023301in}{3.712764in}}%
\pgfpathlineto{\pgfqpoint{6.029707in}{3.713829in}}%
\pgfpathlineto{\pgfqpoint{6.035930in}{3.714898in}}%
\pgfpathlineto{\pgfqpoint{6.041980in}{3.715975in}}%
\pgfpathlineto{\pgfqpoint{6.047867in}{3.717062in}}%
\pgfpathlineto{\pgfqpoint{6.053598in}{3.718160in}}%
\pgfpathlineto{\pgfqpoint{6.059183in}{3.719273in}}%
\pgfpathlineto{\pgfqpoint{6.064628in}{3.720403in}}%
\pgfpathlineto{\pgfqpoint{6.069940in}{3.721556in}}%
\pgfpathlineto{\pgfqpoint{6.075126in}{3.722736in}}%
\pgfpathlineto{\pgfqpoint{6.080191in}{3.723947in}}%
\pgfpathlineto{\pgfqpoint{6.085140in}{3.725195in}}%
\pgfpathlineto{\pgfqpoint{6.089980in}{3.726487in}}%
\pgfpathlineto{\pgfqpoint{6.094715in}{3.727829in}}%
\pgfpathlineto{\pgfqpoint{6.099349in}{3.729230in}}%
\pgfpathlineto{\pgfqpoint{6.103886in}{3.730697in}}%
\pgfpathlineto{\pgfqpoint{6.108331in}{3.732209in}}%
\pgfpathlineto{\pgfqpoint{6.108331in}{3.732247in}}%
\pgfpathlineto{\pgfqpoint{6.108331in}{3.732247in}}%
\pgfpathlineto{\pgfqpoint{6.103886in}{3.730952in}}%
\pgfpathlineto{\pgfqpoint{6.099349in}{3.729693in}}%
\pgfpathlineto{\pgfqpoint{6.094715in}{3.728441in}}%
\pgfpathlineto{\pgfqpoint{6.089980in}{3.727197in}}%
\pgfpathlineto{\pgfqpoint{6.085140in}{3.725962in}}%
\pgfpathlineto{\pgfqpoint{6.080191in}{3.724736in}}%
\pgfpathlineto{\pgfqpoint{6.075126in}{3.723521in}}%
\pgfpathlineto{\pgfqpoint{6.069940in}{3.722315in}}%
\pgfpathlineto{\pgfqpoint{6.064628in}{3.721120in}}%
\pgfpathlineto{\pgfqpoint{6.059183in}{3.719936in}}%
\pgfpathlineto{\pgfqpoint{6.053598in}{3.718762in}}%
\pgfpathlineto{\pgfqpoint{6.047867in}{3.717599in}}%
\pgfpathlineto{\pgfqpoint{6.041980in}{3.716447in}}%
\pgfpathlineto{\pgfqpoint{6.035930in}{3.715305in}}%
\pgfpathlineto{\pgfqpoint{6.029707in}{3.714173in}}%
\pgfpathlineto{\pgfqpoint{6.023301in}{3.713051in}}%
\pgfpathlineto{\pgfqpoint{6.016701in}{3.711939in}}%
\pgfpathlineto{\pgfqpoint{6.009894in}{3.710835in}}%
\pgfpathlineto{\pgfqpoint{6.002868in}{3.709740in}}%
\pgfpathlineto{\pgfqpoint{5.995607in}{3.708655in}}%
\pgfpathlineto{\pgfqpoint{5.988096in}{3.707577in}}%
\pgfpathlineto{\pgfqpoint{5.980317in}{3.706509in}}%
\pgfpathlineto{\pgfqpoint{5.972249in}{3.705450in}}%
\pgfpathlineto{\pgfqpoint{5.963871in}{3.704399in}}%
\pgfpathlineto{\pgfqpoint{5.955158in}{3.703354in}}%
\pgfpathlineto{\pgfqpoint{5.946082in}{3.702314in}}%
\pgfpathlineto{\pgfqpoint{5.936610in}{3.701277in}}%
\pgfpathlineto{\pgfqpoint{5.926708in}{3.700244in}}%
\pgfpathlineto{\pgfqpoint{5.916334in}{3.699214in}}%
\pgfpathlineto{\pgfqpoint{5.905441in}{3.698187in}}%
\pgfpathlineto{\pgfqpoint{5.893974in}{3.697162in}}%
\pgfpathlineto{\pgfqpoint{5.881870in}{3.696140in}}%
\pgfpathlineto{\pgfqpoint{5.869053in}{3.695120in}}%
\pgfpathlineto{\pgfqpoint{5.855435in}{3.694102in}}%
\pgfpathlineto{\pgfqpoint{5.840907in}{3.693086in}}%
\pgfpathlineto{\pgfqpoint{5.825341in}{3.692070in}}%
\pgfpathlineto{\pgfqpoint{5.808575in}{3.691056in}}%
\pgfpathlineto{\pgfqpoint{5.790410in}{3.690042in}}%
\pgfpathlineto{\pgfqpoint{5.770591in}{3.689059in}}%
\pgfpathlineto{\pgfqpoint{5.748786in}{3.688087in}}%
\pgfpathlineto{\pgfqpoint{5.724552in}{3.687123in}}%
\pgfpathlineto{\pgfqpoint{5.697279in}{3.686166in}}%
\pgfpathlineto{\pgfqpoint{5.666095in}{3.685219in}}%
\pgfpathlineto{\pgfqpoint{5.629687in}{3.684282in}}%
\pgfpathlineto{\pgfqpoint{5.585945in}{3.683357in}}%
\pgfpathlineto{\pgfqpoint{5.531149in}{3.682446in}}%
\pgfpathlineto{\pgfqpoint{5.457753in}{3.681551in}}%
\pgfpathlineto{\pgfqpoint{5.346222in}{3.680676in}}%
\pgfpathlineto{\pgfqpoint{5.105513in}{3.679822in}}%
\pgfpathclose%
\pgfusepath{stroke,fill}%
\end{pgfscope}%
\begin{pgfscope}%
\pgfpathrectangle{\pgfqpoint{5.105513in}{3.353583in}}{\pgfqpoint{1.223103in}{0.607948in}}%
\pgfusepath{clip}%
\pgfsetroundcap%
\pgfsetroundjoin%
\pgfsetlinewidth{0.501875pt}%
\definecolor{currentstroke}{rgb}{0.000000,0.000000,1.000000}%
\pgfsetstrokecolor{currentstroke}%
\pgfsetstrokeopacity{0.800000}%
\pgfsetdash{}{0pt}%
\pgfpathmoveto{\pgfqpoint{5.105513in}{3.678464in}}%
\pgfpathlineto{\pgfqpoint{6.328616in}{3.678464in}}%
\pgfusepath{stroke}%
\end{pgfscope}%
\begin{pgfscope}%
\pgfpathrectangle{\pgfqpoint{5.105513in}{3.353583in}}{\pgfqpoint{1.223103in}{0.607948in}}%
\pgfusepath{clip}%
\pgfsetbuttcap%
\pgfsetroundjoin%
\pgfsetlinewidth{1.003750pt}%
\definecolor{currentstroke}{rgb}{0.000000,0.000000,0.000000}%
\pgfsetstrokecolor{currentstroke}%
\pgfsetdash{{3.700000pt}{1.600000pt}}{0.000000pt}%
\pgfpathmoveto{\pgfqpoint{5.105513in}{3.678551in}}%
\pgfpathlineto{\pgfqpoint{6.328616in}{3.678551in}}%
\pgfusepath{stroke}%
\end{pgfscope}%
\begin{pgfscope}%
\pgfsetroundcap%
\pgfsetroundjoin%
\pgfsetlinewidth{0.501875pt}%
\definecolor{currentstroke}{rgb}{0.000000,0.000000,1.000000}%
\pgfsetstrokecolor{currentstroke}%
\pgfsetstrokeopacity{0.800000}%
\pgfsetdash{}{0pt}%
\pgfpathmoveto{\pgfqpoint{5.919361in}{3.795800in}}%
\pgfpathquadraticcurveto{\pgfqpoint{5.849843in}{3.745293in}}{\pgfqpoint{5.780325in}{3.694785in}}%
\pgfusepath{stroke}%
\end{pgfscope}%
\begin{pgfscope}%
\pgfsetfillopacity{0.800000}%
\pgfsetstrokeopacity{0.800000}%
\definecolor{textcolor}{rgb}{0.000000,0.000000,1.000000}%
\pgfsetstrokecolor{textcolor}%
\pgfsetfillcolor{textcolor}%
\pgftext[x=5.846155in,y=3.860848in,left,base]{\color{textcolor}\sffamily\fontsize{5.647059}{6.776471}\selectfont 9.5344(16)}%
\end{pgfscope}%
\begin{pgfscope}%
\pgfsetbuttcap%
\pgfsetroundjoin%
\definecolor{currentfill}{rgb}{0.150000,0.150000,0.150000}%
\pgfsetfillcolor{currentfill}%
\pgfsetlinewidth{1.003750pt}%
\definecolor{currentstroke}{rgb}{0.150000,0.150000,0.150000}%
\pgfsetstrokecolor{currentstroke}%
\pgfsetdash{}{0pt}%
\pgfsys@defobject{currentmarker}{\pgfqpoint{0.000000in}{-0.066667in}}{\pgfqpoint{0.000000in}{0.000000in}}{%
\pgfpathmoveto{\pgfqpoint{0.000000in}{0.000000in}}%
\pgfpathlineto{\pgfqpoint{0.000000in}{-0.066667in}}%
\pgfusepath{stroke,fill}%
}%
\begin{pgfscope}%
\pgfsys@transformshift{5.105513in}{3.353583in}%
\pgfsys@useobject{currentmarker}{}%
\end{pgfscope}%
\end{pgfscope}%
\begin{pgfscope}%
\pgfsetbuttcap%
\pgfsetroundjoin%
\definecolor{currentfill}{rgb}{0.150000,0.150000,0.150000}%
\pgfsetfillcolor{currentfill}%
\pgfsetlinewidth{1.003750pt}%
\definecolor{currentstroke}{rgb}{0.150000,0.150000,0.150000}%
\pgfsetstrokecolor{currentstroke}%
\pgfsetdash{}{0pt}%
\pgfsys@defobject{currentmarker}{\pgfqpoint{0.000000in}{-0.066667in}}{\pgfqpoint{0.000000in}{0.000000in}}{%
\pgfpathmoveto{\pgfqpoint{0.000000in}{0.000000in}}%
\pgfpathlineto{\pgfqpoint{0.000000in}{-0.066667in}}%
\pgfusepath{stroke,fill}%
}%
\begin{pgfscope}%
\pgfsys@transformshift{5.606922in}{3.353583in}%
\pgfsys@useobject{currentmarker}{}%
\end{pgfscope}%
\end{pgfscope}%
\begin{pgfscope}%
\pgfsetbuttcap%
\pgfsetroundjoin%
\definecolor{currentfill}{rgb}{0.150000,0.150000,0.150000}%
\pgfsetfillcolor{currentfill}%
\pgfsetlinewidth{1.003750pt}%
\definecolor{currentstroke}{rgb}{0.150000,0.150000,0.150000}%
\pgfsetstrokecolor{currentstroke}%
\pgfsetdash{}{0pt}%
\pgfsys@defobject{currentmarker}{\pgfqpoint{0.000000in}{-0.066667in}}{\pgfqpoint{0.000000in}{0.000000in}}{%
\pgfpathmoveto{\pgfqpoint{0.000000in}{0.000000in}}%
\pgfpathlineto{\pgfqpoint{0.000000in}{-0.066667in}}%
\pgfusepath{stroke,fill}%
}%
\begin{pgfscope}%
\pgfsys@transformshift{6.108331in}{3.353583in}%
\pgfsys@useobject{currentmarker}{}%
\end{pgfscope}%
\end{pgfscope}%
\begin{pgfscope}%
\pgfsetbuttcap%
\pgfsetroundjoin%
\definecolor{currentfill}{rgb}{0.150000,0.150000,0.150000}%
\pgfsetfillcolor{currentfill}%
\pgfsetlinewidth{0.803000pt}%
\definecolor{currentstroke}{rgb}{0.150000,0.150000,0.150000}%
\pgfsetstrokecolor{currentstroke}%
\pgfsetdash{}{0pt}%
\pgfsys@defobject{currentmarker}{\pgfqpoint{0.000000in}{-0.044444in}}{\pgfqpoint{0.000000in}{0.000000in}}{%
\pgfpathmoveto{\pgfqpoint{0.000000in}{0.000000in}}%
\pgfpathlineto{\pgfqpoint{0.000000in}{-0.044444in}}%
\pgfusepath{stroke,fill}%
}%
\begin{pgfscope}%
\pgfsys@transformshift{5.256452in}{3.353583in}%
\pgfsys@useobject{currentmarker}{}%
\end{pgfscope}%
\end{pgfscope}%
\begin{pgfscope}%
\pgfsetbuttcap%
\pgfsetroundjoin%
\definecolor{currentfill}{rgb}{0.150000,0.150000,0.150000}%
\pgfsetfillcolor{currentfill}%
\pgfsetlinewidth{0.803000pt}%
\definecolor{currentstroke}{rgb}{0.150000,0.150000,0.150000}%
\pgfsetstrokecolor{currentstroke}%
\pgfsetdash{}{0pt}%
\pgfsys@defobject{currentmarker}{\pgfqpoint{0.000000in}{-0.044444in}}{\pgfqpoint{0.000000in}{0.000000in}}{%
\pgfpathmoveto{\pgfqpoint{0.000000in}{0.000000in}}%
\pgfpathlineto{\pgfqpoint{0.000000in}{-0.044444in}}%
\pgfusepath{stroke,fill}%
}%
\begin{pgfscope}%
\pgfsys@transformshift{5.344746in}{3.353583in}%
\pgfsys@useobject{currentmarker}{}%
\end{pgfscope}%
\end{pgfscope}%
\begin{pgfscope}%
\pgfsetbuttcap%
\pgfsetroundjoin%
\definecolor{currentfill}{rgb}{0.150000,0.150000,0.150000}%
\pgfsetfillcolor{currentfill}%
\pgfsetlinewidth{0.803000pt}%
\definecolor{currentstroke}{rgb}{0.150000,0.150000,0.150000}%
\pgfsetstrokecolor{currentstroke}%
\pgfsetdash{}{0pt}%
\pgfsys@defobject{currentmarker}{\pgfqpoint{0.000000in}{-0.044444in}}{\pgfqpoint{0.000000in}{0.000000in}}{%
\pgfpathmoveto{\pgfqpoint{0.000000in}{0.000000in}}%
\pgfpathlineto{\pgfqpoint{0.000000in}{-0.044444in}}%
\pgfusepath{stroke,fill}%
}%
\begin{pgfscope}%
\pgfsys@transformshift{5.407391in}{3.353583in}%
\pgfsys@useobject{currentmarker}{}%
\end{pgfscope}%
\end{pgfscope}%
\begin{pgfscope}%
\pgfsetbuttcap%
\pgfsetroundjoin%
\definecolor{currentfill}{rgb}{0.150000,0.150000,0.150000}%
\pgfsetfillcolor{currentfill}%
\pgfsetlinewidth{0.803000pt}%
\definecolor{currentstroke}{rgb}{0.150000,0.150000,0.150000}%
\pgfsetstrokecolor{currentstroke}%
\pgfsetdash{}{0pt}%
\pgfsys@defobject{currentmarker}{\pgfqpoint{0.000000in}{-0.044444in}}{\pgfqpoint{0.000000in}{0.000000in}}{%
\pgfpathmoveto{\pgfqpoint{0.000000in}{0.000000in}}%
\pgfpathlineto{\pgfqpoint{0.000000in}{-0.044444in}}%
\pgfusepath{stroke,fill}%
}%
\begin{pgfscope}%
\pgfsys@transformshift{5.455982in}{3.353583in}%
\pgfsys@useobject{currentmarker}{}%
\end{pgfscope}%
\end{pgfscope}%
\begin{pgfscope}%
\pgfsetbuttcap%
\pgfsetroundjoin%
\definecolor{currentfill}{rgb}{0.150000,0.150000,0.150000}%
\pgfsetfillcolor{currentfill}%
\pgfsetlinewidth{0.803000pt}%
\definecolor{currentstroke}{rgb}{0.150000,0.150000,0.150000}%
\pgfsetstrokecolor{currentstroke}%
\pgfsetdash{}{0pt}%
\pgfsys@defobject{currentmarker}{\pgfqpoint{0.000000in}{-0.044444in}}{\pgfqpoint{0.000000in}{0.000000in}}{%
\pgfpathmoveto{\pgfqpoint{0.000000in}{0.000000in}}%
\pgfpathlineto{\pgfqpoint{0.000000in}{-0.044444in}}%
\pgfusepath{stroke,fill}%
}%
\begin{pgfscope}%
\pgfsys@transformshift{5.495685in}{3.353583in}%
\pgfsys@useobject{currentmarker}{}%
\end{pgfscope}%
\end{pgfscope}%
\begin{pgfscope}%
\pgfsetbuttcap%
\pgfsetroundjoin%
\definecolor{currentfill}{rgb}{0.150000,0.150000,0.150000}%
\pgfsetfillcolor{currentfill}%
\pgfsetlinewidth{0.803000pt}%
\definecolor{currentstroke}{rgb}{0.150000,0.150000,0.150000}%
\pgfsetstrokecolor{currentstroke}%
\pgfsetdash{}{0pt}%
\pgfsys@defobject{currentmarker}{\pgfqpoint{0.000000in}{-0.044444in}}{\pgfqpoint{0.000000in}{0.000000in}}{%
\pgfpathmoveto{\pgfqpoint{0.000000in}{0.000000in}}%
\pgfpathlineto{\pgfqpoint{0.000000in}{-0.044444in}}%
\pgfusepath{stroke,fill}%
}%
\begin{pgfscope}%
\pgfsys@transformshift{5.529252in}{3.353583in}%
\pgfsys@useobject{currentmarker}{}%
\end{pgfscope}%
\end{pgfscope}%
\begin{pgfscope}%
\pgfsetbuttcap%
\pgfsetroundjoin%
\definecolor{currentfill}{rgb}{0.150000,0.150000,0.150000}%
\pgfsetfillcolor{currentfill}%
\pgfsetlinewidth{0.803000pt}%
\definecolor{currentstroke}{rgb}{0.150000,0.150000,0.150000}%
\pgfsetstrokecolor{currentstroke}%
\pgfsetdash{}{0pt}%
\pgfsys@defobject{currentmarker}{\pgfqpoint{0.000000in}{-0.044444in}}{\pgfqpoint{0.000000in}{0.000000in}}{%
\pgfpathmoveto{\pgfqpoint{0.000000in}{0.000000in}}%
\pgfpathlineto{\pgfqpoint{0.000000in}{-0.044444in}}%
\pgfusepath{stroke,fill}%
}%
\begin{pgfscope}%
\pgfsys@transformshift{5.558330in}{3.353583in}%
\pgfsys@useobject{currentmarker}{}%
\end{pgfscope}%
\end{pgfscope}%
\begin{pgfscope}%
\pgfsetbuttcap%
\pgfsetroundjoin%
\definecolor{currentfill}{rgb}{0.150000,0.150000,0.150000}%
\pgfsetfillcolor{currentfill}%
\pgfsetlinewidth{0.803000pt}%
\definecolor{currentstroke}{rgb}{0.150000,0.150000,0.150000}%
\pgfsetstrokecolor{currentstroke}%
\pgfsetdash{}{0pt}%
\pgfsys@defobject{currentmarker}{\pgfqpoint{0.000000in}{-0.044444in}}{\pgfqpoint{0.000000in}{0.000000in}}{%
\pgfpathmoveto{\pgfqpoint{0.000000in}{0.000000in}}%
\pgfpathlineto{\pgfqpoint{0.000000in}{-0.044444in}}%
\pgfusepath{stroke,fill}%
}%
\begin{pgfscope}%
\pgfsys@transformshift{5.583978in}{3.353583in}%
\pgfsys@useobject{currentmarker}{}%
\end{pgfscope}%
\end{pgfscope}%
\begin{pgfscope}%
\pgfsetbuttcap%
\pgfsetroundjoin%
\definecolor{currentfill}{rgb}{0.150000,0.150000,0.150000}%
\pgfsetfillcolor{currentfill}%
\pgfsetlinewidth{0.803000pt}%
\definecolor{currentstroke}{rgb}{0.150000,0.150000,0.150000}%
\pgfsetstrokecolor{currentstroke}%
\pgfsetdash{}{0pt}%
\pgfsys@defobject{currentmarker}{\pgfqpoint{0.000000in}{-0.044444in}}{\pgfqpoint{0.000000in}{0.000000in}}{%
\pgfpathmoveto{\pgfqpoint{0.000000in}{0.000000in}}%
\pgfpathlineto{\pgfqpoint{0.000000in}{-0.044444in}}%
\pgfusepath{stroke,fill}%
}%
\begin{pgfscope}%
\pgfsys@transformshift{5.757861in}{3.353583in}%
\pgfsys@useobject{currentmarker}{}%
\end{pgfscope}%
\end{pgfscope}%
\begin{pgfscope}%
\pgfsetbuttcap%
\pgfsetroundjoin%
\definecolor{currentfill}{rgb}{0.150000,0.150000,0.150000}%
\pgfsetfillcolor{currentfill}%
\pgfsetlinewidth{0.803000pt}%
\definecolor{currentstroke}{rgb}{0.150000,0.150000,0.150000}%
\pgfsetstrokecolor{currentstroke}%
\pgfsetdash{}{0pt}%
\pgfsys@defobject{currentmarker}{\pgfqpoint{0.000000in}{-0.044444in}}{\pgfqpoint{0.000000in}{0.000000in}}{%
\pgfpathmoveto{\pgfqpoint{0.000000in}{0.000000in}}%
\pgfpathlineto{\pgfqpoint{0.000000in}{-0.044444in}}%
\pgfusepath{stroke,fill}%
}%
\begin{pgfscope}%
\pgfsys@transformshift{5.846155in}{3.353583in}%
\pgfsys@useobject{currentmarker}{}%
\end{pgfscope}%
\end{pgfscope}%
\begin{pgfscope}%
\pgfsetbuttcap%
\pgfsetroundjoin%
\definecolor{currentfill}{rgb}{0.150000,0.150000,0.150000}%
\pgfsetfillcolor{currentfill}%
\pgfsetlinewidth{0.803000pt}%
\definecolor{currentstroke}{rgb}{0.150000,0.150000,0.150000}%
\pgfsetstrokecolor{currentstroke}%
\pgfsetdash{}{0pt}%
\pgfsys@defobject{currentmarker}{\pgfqpoint{0.000000in}{-0.044444in}}{\pgfqpoint{0.000000in}{0.000000in}}{%
\pgfpathmoveto{\pgfqpoint{0.000000in}{0.000000in}}%
\pgfpathlineto{\pgfqpoint{0.000000in}{-0.044444in}}%
\pgfusepath{stroke,fill}%
}%
\begin{pgfscope}%
\pgfsys@transformshift{5.908800in}{3.353583in}%
\pgfsys@useobject{currentmarker}{}%
\end{pgfscope}%
\end{pgfscope}%
\begin{pgfscope}%
\pgfsetbuttcap%
\pgfsetroundjoin%
\definecolor{currentfill}{rgb}{0.150000,0.150000,0.150000}%
\pgfsetfillcolor{currentfill}%
\pgfsetlinewidth{0.803000pt}%
\definecolor{currentstroke}{rgb}{0.150000,0.150000,0.150000}%
\pgfsetstrokecolor{currentstroke}%
\pgfsetdash{}{0pt}%
\pgfsys@defobject{currentmarker}{\pgfqpoint{0.000000in}{-0.044444in}}{\pgfqpoint{0.000000in}{0.000000in}}{%
\pgfpathmoveto{\pgfqpoint{0.000000in}{0.000000in}}%
\pgfpathlineto{\pgfqpoint{0.000000in}{-0.044444in}}%
\pgfusepath{stroke,fill}%
}%
\begin{pgfscope}%
\pgfsys@transformshift{5.957391in}{3.353583in}%
\pgfsys@useobject{currentmarker}{}%
\end{pgfscope}%
\end{pgfscope}%
\begin{pgfscope}%
\pgfsetbuttcap%
\pgfsetroundjoin%
\definecolor{currentfill}{rgb}{0.150000,0.150000,0.150000}%
\pgfsetfillcolor{currentfill}%
\pgfsetlinewidth{0.803000pt}%
\definecolor{currentstroke}{rgb}{0.150000,0.150000,0.150000}%
\pgfsetstrokecolor{currentstroke}%
\pgfsetdash{}{0pt}%
\pgfsys@defobject{currentmarker}{\pgfqpoint{0.000000in}{-0.044444in}}{\pgfqpoint{0.000000in}{0.000000in}}{%
\pgfpathmoveto{\pgfqpoint{0.000000in}{0.000000in}}%
\pgfpathlineto{\pgfqpoint{0.000000in}{-0.044444in}}%
\pgfusepath{stroke,fill}%
}%
\begin{pgfscope}%
\pgfsys@transformshift{5.997094in}{3.353583in}%
\pgfsys@useobject{currentmarker}{}%
\end{pgfscope}%
\end{pgfscope}%
\begin{pgfscope}%
\pgfsetbuttcap%
\pgfsetroundjoin%
\definecolor{currentfill}{rgb}{0.150000,0.150000,0.150000}%
\pgfsetfillcolor{currentfill}%
\pgfsetlinewidth{0.803000pt}%
\definecolor{currentstroke}{rgb}{0.150000,0.150000,0.150000}%
\pgfsetstrokecolor{currentstroke}%
\pgfsetdash{}{0pt}%
\pgfsys@defobject{currentmarker}{\pgfqpoint{0.000000in}{-0.044444in}}{\pgfqpoint{0.000000in}{0.000000in}}{%
\pgfpathmoveto{\pgfqpoint{0.000000in}{0.000000in}}%
\pgfpathlineto{\pgfqpoint{0.000000in}{-0.044444in}}%
\pgfusepath{stroke,fill}%
}%
\begin{pgfscope}%
\pgfsys@transformshift{6.030661in}{3.353583in}%
\pgfsys@useobject{currentmarker}{}%
\end{pgfscope}%
\end{pgfscope}%
\begin{pgfscope}%
\pgfsetbuttcap%
\pgfsetroundjoin%
\definecolor{currentfill}{rgb}{0.150000,0.150000,0.150000}%
\pgfsetfillcolor{currentfill}%
\pgfsetlinewidth{0.803000pt}%
\definecolor{currentstroke}{rgb}{0.150000,0.150000,0.150000}%
\pgfsetstrokecolor{currentstroke}%
\pgfsetdash{}{0pt}%
\pgfsys@defobject{currentmarker}{\pgfqpoint{0.000000in}{-0.044444in}}{\pgfqpoint{0.000000in}{0.000000in}}{%
\pgfpathmoveto{\pgfqpoint{0.000000in}{0.000000in}}%
\pgfpathlineto{\pgfqpoint{0.000000in}{-0.044444in}}%
\pgfusepath{stroke,fill}%
}%
\begin{pgfscope}%
\pgfsys@transformshift{6.059739in}{3.353583in}%
\pgfsys@useobject{currentmarker}{}%
\end{pgfscope}%
\end{pgfscope}%
\begin{pgfscope}%
\pgfsetbuttcap%
\pgfsetroundjoin%
\definecolor{currentfill}{rgb}{0.150000,0.150000,0.150000}%
\pgfsetfillcolor{currentfill}%
\pgfsetlinewidth{0.803000pt}%
\definecolor{currentstroke}{rgb}{0.150000,0.150000,0.150000}%
\pgfsetstrokecolor{currentstroke}%
\pgfsetdash{}{0pt}%
\pgfsys@defobject{currentmarker}{\pgfqpoint{0.000000in}{-0.044444in}}{\pgfqpoint{0.000000in}{0.000000in}}{%
\pgfpathmoveto{\pgfqpoint{0.000000in}{0.000000in}}%
\pgfpathlineto{\pgfqpoint{0.000000in}{-0.044444in}}%
\pgfusepath{stroke,fill}%
}%
\begin{pgfscope}%
\pgfsys@transformshift{6.085387in}{3.353583in}%
\pgfsys@useobject{currentmarker}{}%
\end{pgfscope}%
\end{pgfscope}%
\begin{pgfscope}%
\pgfsetbuttcap%
\pgfsetroundjoin%
\definecolor{currentfill}{rgb}{0.150000,0.150000,0.150000}%
\pgfsetfillcolor{currentfill}%
\pgfsetlinewidth{0.803000pt}%
\definecolor{currentstroke}{rgb}{0.150000,0.150000,0.150000}%
\pgfsetstrokecolor{currentstroke}%
\pgfsetdash{}{0pt}%
\pgfsys@defobject{currentmarker}{\pgfqpoint{0.000000in}{-0.044444in}}{\pgfqpoint{0.000000in}{0.000000in}}{%
\pgfpathmoveto{\pgfqpoint{0.000000in}{0.000000in}}%
\pgfpathlineto{\pgfqpoint{0.000000in}{-0.044444in}}%
\pgfusepath{stroke,fill}%
}%
\begin{pgfscope}%
\pgfsys@transformshift{6.259270in}{3.353583in}%
\pgfsys@useobject{currentmarker}{}%
\end{pgfscope}%
\end{pgfscope}%
\begin{pgfscope}%
\pgfsetbuttcap%
\pgfsetroundjoin%
\definecolor{currentfill}{rgb}{0.150000,0.150000,0.150000}%
\pgfsetfillcolor{currentfill}%
\pgfsetlinewidth{1.003750pt}%
\definecolor{currentstroke}{rgb}{0.150000,0.150000,0.150000}%
\pgfsetstrokecolor{currentstroke}%
\pgfsetdash{}{0pt}%
\pgfsys@defobject{currentmarker}{\pgfqpoint{-0.066667in}{0.000000in}}{\pgfqpoint{0.000000in}{0.000000in}}{%
\pgfpathmoveto{\pgfqpoint{0.000000in}{0.000000in}}%
\pgfpathlineto{\pgfqpoint{-0.066667in}{0.000000in}}%
\pgfusepath{stroke,fill}%
}%
\begin{pgfscope}%
\pgfsys@transformshift{5.105513in}{3.353583in}%
\pgfsys@useobject{currentmarker}{}%
\end{pgfscope}%
\end{pgfscope}%
\begin{pgfscope}%
\pgfsetbuttcap%
\pgfsetroundjoin%
\definecolor{currentfill}{rgb}{0.150000,0.150000,0.150000}%
\pgfsetfillcolor{currentfill}%
\pgfsetlinewidth{1.003750pt}%
\definecolor{currentstroke}{rgb}{0.150000,0.150000,0.150000}%
\pgfsetstrokecolor{currentstroke}%
\pgfsetdash{}{0pt}%
\pgfsys@defobject{currentmarker}{\pgfqpoint{-0.066667in}{0.000000in}}{\pgfqpoint{0.000000in}{0.000000in}}{%
\pgfpathmoveto{\pgfqpoint{0.000000in}{0.000000in}}%
\pgfpathlineto{\pgfqpoint{-0.066667in}{0.000000in}}%
\pgfusepath{stroke,fill}%
}%
\begin{pgfscope}%
\pgfsys@transformshift{5.105513in}{3.678551in}%
\pgfsys@useobject{currentmarker}{}%
\end{pgfscope}%
\end{pgfscope}%
\begin{pgfscope}%
\pgfsetbuttcap%
\pgfsetroundjoin%
\definecolor{currentfill}{rgb}{0.150000,0.150000,0.150000}%
\pgfsetfillcolor{currentfill}%
\pgfsetlinewidth{1.003750pt}%
\definecolor{currentstroke}{rgb}{0.150000,0.150000,0.150000}%
\pgfsetstrokecolor{currentstroke}%
\pgfsetdash{}{0pt}%
\pgfsys@defobject{currentmarker}{\pgfqpoint{-0.066667in}{0.000000in}}{\pgfqpoint{0.000000in}{0.000000in}}{%
\pgfpathmoveto{\pgfqpoint{0.000000in}{0.000000in}}%
\pgfpathlineto{\pgfqpoint{-0.066667in}{0.000000in}}%
\pgfusepath{stroke,fill}%
}%
\begin{pgfscope}%
\pgfsys@transformshift{5.105513in}{3.961532in}%
\pgfsys@useobject{currentmarker}{}%
\end{pgfscope}%
\end{pgfscope}%
\begin{pgfscope}%
\pgfpathrectangle{\pgfqpoint{5.105513in}{3.353583in}}{\pgfqpoint{1.223103in}{0.607948in}}%
\pgfusepath{clip}%
\pgfsetroundcap%
\pgfsetroundjoin%
\pgfsetlinewidth{1.204500pt}%
\definecolor{currentstroke}{rgb}{0.000000,0.501961,0.000000}%
\pgfsetstrokecolor{currentstroke}%
\pgfsetdash{}{0pt}%
\pgfpathmoveto{\pgfqpoint{5.105513in}{3.678981in}}%
\pgfpathlineto{\pgfqpoint{5.346222in}{3.680016in}}%
\pgfpathlineto{\pgfqpoint{5.457753in}{3.681039in}}%
\pgfpathlineto{\pgfqpoint{5.531149in}{3.682053in}}%
\pgfpathlineto{\pgfqpoint{5.585945in}{3.683060in}}%
\pgfpathlineto{\pgfqpoint{5.629687in}{3.684062in}}%
\pgfpathlineto{\pgfqpoint{5.666095in}{3.685060in}}%
\pgfpathlineto{\pgfqpoint{5.697279in}{3.686056in}}%
\pgfpathlineto{\pgfqpoint{5.724552in}{3.687051in}}%
\pgfpathlineto{\pgfqpoint{5.748786in}{3.688046in}}%
\pgfpathlineto{\pgfqpoint{5.770591in}{3.689041in}}%
\pgfpathlineto{\pgfqpoint{5.790410in}{3.690038in}}%
\pgfpathlineto{\pgfqpoint{5.808575in}{3.691038in}}%
\pgfpathlineto{\pgfqpoint{5.825341in}{3.692040in}}%
\pgfpathlineto{\pgfqpoint{5.840907in}{3.693046in}}%
\pgfpathlineto{\pgfqpoint{5.855435in}{3.694055in}}%
\pgfpathlineto{\pgfqpoint{5.869053in}{3.695067in}}%
\pgfpathlineto{\pgfqpoint{5.881870in}{3.696084in}}%
\pgfpathlineto{\pgfqpoint{5.893974in}{3.697105in}}%
\pgfpathlineto{\pgfqpoint{5.905441in}{3.698129in}}%
\pgfpathlineto{\pgfqpoint{5.916334in}{3.699158in}}%
\pgfpathlineto{\pgfqpoint{5.926708in}{3.700191in}}%
\pgfpathlineto{\pgfqpoint{5.936610in}{3.701228in}}%
\pgfpathlineto{\pgfqpoint{5.946082in}{3.702268in}}%
\pgfpathlineto{\pgfqpoint{5.955158in}{3.703313in}}%
\pgfpathlineto{\pgfqpoint{5.963871in}{3.704362in}}%
\pgfpathlineto{\pgfqpoint{5.972249in}{3.705414in}}%
\pgfpathlineto{\pgfqpoint{5.980317in}{3.706471in}}%
\pgfpathlineto{\pgfqpoint{5.988096in}{3.707532in}}%
\pgfpathlineto{\pgfqpoint{5.995607in}{3.708597in}}%
\pgfpathlineto{\pgfqpoint{6.002868in}{3.709666in}}%
\pgfpathlineto{\pgfqpoint{6.009894in}{3.710741in}}%
\pgfpathlineto{\pgfqpoint{6.016701in}{3.711821in}}%
\pgfpathlineto{\pgfqpoint{6.023301in}{3.712908in}}%
\pgfpathlineto{\pgfqpoint{6.029707in}{3.714001in}}%
\pgfpathlineto{\pgfqpoint{6.035930in}{3.715102in}}%
\pgfpathlineto{\pgfqpoint{6.041980in}{3.716211in}}%
\pgfpathlineto{\pgfqpoint{6.047867in}{3.717330in}}%
\pgfpathlineto{\pgfqpoint{6.053598in}{3.718461in}}%
\pgfpathlineto{\pgfqpoint{6.059183in}{3.719604in}}%
\pgfpathlineto{\pgfqpoint{6.064628in}{3.720762in}}%
\pgfpathlineto{\pgfqpoint{6.069940in}{3.721936in}}%
\pgfpathlineto{\pgfqpoint{6.075126in}{3.723128in}}%
\pgfpathlineto{\pgfqpoint{6.080191in}{3.724342in}}%
\pgfpathlineto{\pgfqpoint{6.085140in}{3.725579in}}%
\pgfpathlineto{\pgfqpoint{6.089980in}{3.726842in}}%
\pgfpathlineto{\pgfqpoint{6.094715in}{3.728135in}}%
\pgfpathlineto{\pgfqpoint{6.099349in}{3.729461in}}%
\pgfpathlineto{\pgfqpoint{6.103886in}{3.730824in}}%
\pgfpathlineto{\pgfqpoint{6.108331in}{3.732228in}}%
\pgfusepath{stroke}%
\end{pgfscope}%
\begin{pgfscope}%
\pgfsetrectcap%
\pgfsetmiterjoin%
\pgfsetlinewidth{1.003750pt}%
\definecolor{currentstroke}{rgb}{0.150000,0.150000,0.150000}%
\pgfsetstrokecolor{currentstroke}%
\pgfsetdash{}{0pt}%
\pgfpathmoveto{\pgfqpoint{5.105513in}{3.353583in}}%
\pgfpathlineto{\pgfqpoint{5.105513in}{3.961532in}}%
\pgfusepath{stroke}%
\end{pgfscope}%
\begin{pgfscope}%
\pgfsetrectcap%
\pgfsetmiterjoin%
\pgfsetlinewidth{1.003750pt}%
\definecolor{currentstroke}{rgb}{0.150000,0.150000,0.150000}%
\pgfsetstrokecolor{currentstroke}%
\pgfsetdash{}{0pt}%
\pgfpathmoveto{\pgfqpoint{5.105513in}{3.353583in}}%
\pgfpathlineto{\pgfqpoint{6.328616in}{3.353583in}}%
\pgfusepath{stroke}%
\end{pgfscope}%
\begin{pgfscope}%
\pgfpathrectangle{\pgfqpoint{5.105513in}{3.353583in}}{\pgfqpoint{1.223103in}{0.607948in}}%
\pgfusepath{clip}%
\pgfsetbuttcap%
\pgfsetroundjoin%
\definecolor{currentfill}{rgb}{0.000000,0.000000,0.000000}%
\pgfsetfillcolor{currentfill}%
\pgfsetlinewidth{1.003750pt}%
\definecolor{currentstroke}{rgb}{0.000000,0.000000,0.000000}%
\pgfsetstrokecolor{currentstroke}%
\pgfsetdash{}{0pt}%
\pgfsys@defobject{currentmarker}{\pgfqpoint{-0.013889in}{-0.013889in}}{\pgfqpoint{0.013889in}{0.013889in}}{%
\pgfpathmoveto{\pgfqpoint{0.000000in}{-0.013889in}}%
\pgfpathcurveto{\pgfqpoint{0.003683in}{-0.013889in}}{\pgfqpoint{0.007216in}{-0.012425in}}{\pgfqpoint{0.009821in}{-0.009821in}}%
\pgfpathcurveto{\pgfqpoint{0.012425in}{-0.007216in}}{\pgfqpoint{0.013889in}{-0.003683in}}{\pgfqpoint{0.013889in}{0.000000in}}%
\pgfpathcurveto{\pgfqpoint{0.013889in}{0.003683in}}{\pgfqpoint{0.012425in}{0.007216in}}{\pgfqpoint{0.009821in}{0.009821in}}%
\pgfpathcurveto{\pgfqpoint{0.007216in}{0.012425in}}{\pgfqpoint{0.003683in}{0.013889in}}{\pgfqpoint{0.000000in}{0.013889in}}%
\pgfpathcurveto{\pgfqpoint{-0.003683in}{0.013889in}}{\pgfqpoint{-0.007216in}{0.012425in}}{\pgfqpoint{-0.009821in}{0.009821in}}%
\pgfpathcurveto{\pgfqpoint{-0.012425in}{0.007216in}}{\pgfqpoint{-0.013889in}{0.003683in}}{\pgfqpoint{-0.013889in}{0.000000in}}%
\pgfpathcurveto{\pgfqpoint{-0.013889in}{-0.003683in}}{\pgfqpoint{-0.012425in}{-0.007216in}}{\pgfqpoint{-0.009821in}{-0.009821in}}%
\pgfpathcurveto{\pgfqpoint{-0.007216in}{-0.012425in}}{\pgfqpoint{-0.003683in}{-0.013889in}}{\pgfqpoint{0.000000in}{-0.013889in}}%
\pgfpathclose%
\pgfusepath{stroke,fill}%
}%
\begin{pgfscope}%
\pgfsys@transformshift{5.757861in}{3.688438in}%
\pgfsys@useobject{currentmarker}{}%
\end{pgfscope}%
\begin{pgfscope}%
\pgfsys@transformshift{5.762260in}{3.688635in}%
\pgfsys@useobject{currentmarker}{}%
\end{pgfscope}%
\begin{pgfscope}%
\pgfsys@transformshift{5.766750in}{3.688852in}%
\pgfsys@useobject{currentmarker}{}%
\end{pgfscope}%
\begin{pgfscope}%
\pgfsys@transformshift{5.771335in}{3.689065in}%
\pgfsys@useobject{currentmarker}{}%
\end{pgfscope}%
\begin{pgfscope}%
\pgfsys@transformshift{5.776018in}{3.689302in}%
\pgfsys@useobject{currentmarker}{}%
\end{pgfscope}%
\begin{pgfscope}%
\pgfsys@transformshift{5.780804in}{3.689535in}%
\pgfsys@useobject{currentmarker}{}%
\end{pgfscope}%
\begin{pgfscope}%
\pgfsys@transformshift{5.785698in}{3.689794in}%
\pgfsys@useobject{currentmarker}{}%
\end{pgfscope}%
\begin{pgfscope}%
\pgfsys@transformshift{5.790704in}{3.690048in}%
\pgfsys@useobject{currentmarker}{}%
\end{pgfscope}%
\begin{pgfscope}%
\pgfsys@transformshift{5.795828in}{3.690333in}%
\pgfsys@useobject{currentmarker}{}%
\end{pgfscope}%
\begin{pgfscope}%
\pgfsys@transformshift{5.801075in}{3.690612in}%
\pgfsys@useobject{currentmarker}{}%
\end{pgfscope}%
\begin{pgfscope}%
\pgfsys@transformshift{5.806452in}{3.690927in}%
\pgfsys@useobject{currentmarker}{}%
\end{pgfscope}%
\begin{pgfscope}%
\pgfsys@transformshift{5.835530in}{3.692696in}%
\pgfsys@useobject{currentmarker}{}%
\end{pgfscope}%
\begin{pgfscope}%
\pgfsys@transformshift{5.869098in}{3.695106in}%
\pgfsys@useobject{currentmarker}{}%
\end{pgfscope}%
\begin{pgfscope}%
\pgfsys@transformshift{5.908800in}{3.698438in}%
\pgfsys@useobject{currentmarker}{}%
\end{pgfscope}%
\begin{pgfscope}%
\pgfsys@transformshift{5.957391in}{3.703636in}%
\pgfsys@useobject{currentmarker}{}%
\end{pgfscope}%
\begin{pgfscope}%
\pgfsys@transformshift{6.020037in}{3.712272in}%
\pgfsys@useobject{currentmarker}{}%
\end{pgfscope}%
\begin{pgfscope}%
\pgfsys@transformshift{6.108331in}{3.732240in}%
\pgfsys@useobject{currentmarker}{}%
\end{pgfscope}%
\end{pgfscope}%
\begin{pgfscope}%
\pgfsetbuttcap%
\pgfsetmiterjoin%
\definecolor{currentfill}{rgb}{1.000000,1.000000,1.000000}%
\pgfsetfillcolor{currentfill}%
\pgfsetlinewidth{0.803000pt}%
\definecolor{currentstroke}{rgb}{1.000000,1.000000,1.000000}%
\pgfsetstrokecolor{currentstroke}%
\pgfsetdash{}{0pt}%
\pgfpathmoveto{\pgfqpoint{6.297392in}{3.912656in}}%
\pgfpathlineto{\pgfqpoint{6.297392in}{3.402459in}}%
\pgfpathlineto{\pgfqpoint{6.478411in}{3.402459in}}%
\pgfpathlineto{\pgfqpoint{6.478411in}{3.912656in}}%
\pgfpathclose%
\pgfusepath{stroke,fill}%
\end{pgfscope}%
\begin{pgfscope}%
\definecolor{textcolor}{rgb}{0.150000,0.150000,0.150000}%
\pgfsetstrokecolor{textcolor}%
\pgfsetfillcolor{textcolor}%
\pgftext[x=6.368294in,y=3.856969in,left,base,rotate=270.000000]{\color{textcolor}\sffamily\fontsize{5.647059}{6.776471}\selectfont nlevel = 10}%
\end{pgfscope}%
\begin{pgfscope}%
\pgfsetbuttcap%
\pgfsetmiterjoin%
\definecolor{currentfill}{rgb}{1.000000,1.000000,1.000000}%
\pgfsetfillcolor{currentfill}%
\pgfsetlinewidth{0.803000pt}%
\definecolor{currentstroke}{rgb}{1.000000,1.000000,1.000000}%
\pgfsetstrokecolor{currentstroke}%
\pgfsetdash{}{0pt}%
\pgfpathmoveto{\pgfqpoint{6.297392in}{3.912656in}}%
\pgfpathlineto{\pgfqpoint{6.297392in}{3.402459in}}%
\pgfpathlineto{\pgfqpoint{6.478411in}{3.402459in}}%
\pgfpathlineto{\pgfqpoint{6.478411in}{3.912656in}}%
\pgfpathclose%
\pgfusepath{stroke,fill}%
\end{pgfscope}%
\begin{pgfscope}%
\definecolor{textcolor}{rgb}{0.150000,0.150000,0.150000}%
\pgfsetstrokecolor{textcolor}%
\pgfsetfillcolor{textcolor}%
\pgftext[x=6.368294in,y=3.856969in,left,base,rotate=270.000000]{\color{textcolor}\sffamily\fontsize{5.647059}{6.776471}\selectfont nlevel = 10}%
\end{pgfscope}%
\begin{pgfscope}%
\pgfsetbuttcap%
\pgfsetmiterjoin%
\definecolor{currentfill}{rgb}{1.000000,1.000000,1.000000}%
\pgfsetfillcolor{currentfill}%
\pgfsetlinewidth{0.000000pt}%
\definecolor{currentstroke}{rgb}{0.000000,0.000000,0.000000}%
\pgfsetstrokecolor{currentstroke}%
\pgfsetstrokeopacity{0.000000}%
\pgfsetdash{}{0pt}%
\pgfpathmoveto{\pgfqpoint{0.702340in}{2.624046in}}%
\pgfpathlineto{\pgfqpoint{1.925444in}{2.624046in}}%
\pgfpathlineto{\pgfqpoint{1.925444in}{3.231994in}}%
\pgfpathlineto{\pgfqpoint{0.702340in}{3.231994in}}%
\pgfpathclose%
\pgfusepath{fill}%
\end{pgfscope}%
\begin{pgfscope}%
\pgfpathrectangle{\pgfqpoint{0.702340in}{2.624046in}}{\pgfqpoint{1.223103in}{0.607948in}}%
\pgfusepath{clip}%
\pgfsetbuttcap%
\pgfsetmiterjoin%
\definecolor{currentfill}{rgb}{0.000000,0.000000,1.000000}%
\pgfsetfillcolor{currentfill}%
\pgfsetfillopacity{0.100000}%
\pgfsetlinewidth{0.803000pt}%
\definecolor{currentstroke}{rgb}{0.000000,0.000000,1.000000}%
\pgfsetstrokecolor{currentstroke}%
\pgfsetstrokeopacity{0.100000}%
\pgfsetdash{}{0pt}%
\pgfpathmoveto{\pgfqpoint{0.702340in}{2.750987in}}%
\pgfpathlineto{\pgfqpoint{0.702340in}{2.828714in}}%
\pgfpathlineto{\pgfqpoint{1.925444in}{2.828714in}}%
\pgfpathlineto{\pgfqpoint{1.925444in}{2.750987in}}%
\pgfpathclose%
\pgfusepath{stroke,fill}%
\end{pgfscope}%
\begin{pgfscope}%
\pgfpathrectangle{\pgfqpoint{0.702340in}{2.624046in}}{\pgfqpoint{1.223103in}{0.607948in}}%
\pgfusepath{clip}%
\pgfsetbuttcap%
\pgfsetroundjoin%
\definecolor{currentfill}{rgb}{0.000000,0.501961,0.000000}%
\pgfsetfillcolor{currentfill}%
\pgfsetfillopacity{0.500000}%
\pgfsetlinewidth{0.803000pt}%
\definecolor{currentstroke}{rgb}{0.000000,0.501961,0.000000}%
\pgfsetstrokecolor{currentstroke}%
\pgfsetstrokeopacity{0.500000}%
\pgfsetdash{}{0pt}%
\pgfpathmoveto{\pgfqpoint{0.702340in}{2.825915in}}%
\pgfpathlineto{\pgfqpoint{0.702340in}{2.753724in}}%
\pgfpathlineto{\pgfqpoint{0.943050in}{2.758436in}}%
\pgfpathlineto{\pgfqpoint{1.054581in}{2.762079in}}%
\pgfpathlineto{\pgfqpoint{1.127977in}{2.764682in}}%
\pgfpathlineto{\pgfqpoint{1.182772in}{2.766272in}}%
\pgfpathlineto{\pgfqpoint{1.226515in}{2.766873in}}%
\pgfpathlineto{\pgfqpoint{1.262923in}{2.766508in}}%
\pgfpathlineto{\pgfqpoint{1.294107in}{2.765196in}}%
\pgfpathlineto{\pgfqpoint{1.321380in}{2.762955in}}%
\pgfpathlineto{\pgfqpoint{1.345614in}{2.759799in}}%
\pgfpathlineto{\pgfqpoint{1.367419in}{2.755737in}}%
\pgfpathlineto{\pgfqpoint{1.387238in}{2.750413in}}%
\pgfpathlineto{\pgfqpoint{1.405403in}{2.741832in}}%
\pgfpathlineto{\pgfqpoint{1.422168in}{2.732789in}}%
\pgfpathlineto{\pgfqpoint{1.437735in}{2.723338in}}%
\pgfpathlineto{\pgfqpoint{1.452262in}{2.713462in}}%
\pgfpathlineto{\pgfqpoint{1.465881in}{2.703143in}}%
\pgfpathlineto{\pgfqpoint{1.478698in}{2.692359in}}%
\pgfpathlineto{\pgfqpoint{1.490802in}{2.681087in}}%
\pgfpathlineto{\pgfqpoint{1.502269in}{2.669305in}}%
\pgfpathlineto{\pgfqpoint{1.513162in}{2.656986in}}%
\pgfpathlineto{\pgfqpoint{1.523536in}{2.644102in}}%
\pgfpathlineto{\pgfqpoint{1.533438in}{2.630622in}}%
\pgfpathlineto{\pgfqpoint{1.542909in}{2.616507in}}%
\pgfpathlineto{\pgfqpoint{1.551986in}{2.601707in}}%
\pgfpathlineto{\pgfqpoint{1.560699in}{2.586134in}}%
\pgfpathlineto{\pgfqpoint{1.569077in}{2.569524in}}%
\pgfpathlineto{\pgfqpoint{1.577145in}{2.551231in}}%
\pgfpathlineto{\pgfqpoint{1.584924in}{2.531312in}}%
\pgfpathlineto{\pgfqpoint{1.592435in}{2.510209in}}%
\pgfpathlineto{\pgfqpoint{1.599696in}{2.488039in}}%
\pgfpathlineto{\pgfqpoint{1.606722in}{2.464818in}}%
\pgfpathlineto{\pgfqpoint{1.613528in}{2.440540in}}%
\pgfpathlineto{\pgfqpoint{1.620129in}{2.415190in}}%
\pgfpathlineto{\pgfqpoint{1.626535in}{2.388751in}}%
\pgfpathlineto{\pgfqpoint{1.632758in}{2.361203in}}%
\pgfpathlineto{\pgfqpoint{1.638808in}{2.332528in}}%
\pgfpathlineto{\pgfqpoint{1.644694in}{2.302705in}}%
\pgfpathlineto{\pgfqpoint{1.650426in}{2.271713in}}%
\pgfpathlineto{\pgfqpoint{1.656011in}{2.239530in}}%
\pgfpathlineto{\pgfqpoint{1.661456in}{2.206136in}}%
\pgfpathlineto{\pgfqpoint{1.666768in}{2.171508in}}%
\pgfpathlineto{\pgfqpoint{1.671953in}{2.135625in}}%
\pgfpathlineto{\pgfqpoint{1.677018in}{2.098466in}}%
\pgfpathlineto{\pgfqpoint{1.681968in}{2.060008in}}%
\pgfpathlineto{\pgfqpoint{1.686808in}{2.020230in}}%
\pgfpathlineto{\pgfqpoint{1.691543in}{1.979109in}}%
\pgfpathlineto{\pgfqpoint{1.696176in}{1.936619in}}%
\pgfpathlineto{\pgfqpoint{1.700714in}{1.892705in}}%
\pgfpathlineto{\pgfqpoint{1.705158in}{1.843872in}}%
\pgfpathlineto{\pgfqpoint{1.705158in}{1.847692in}}%
\pgfpathlineto{\pgfqpoint{1.705158in}{1.847692in}}%
\pgfpathlineto{\pgfqpoint{1.700714in}{1.898200in}}%
\pgfpathlineto{\pgfqpoint{1.696176in}{1.949241in}}%
\pgfpathlineto{\pgfqpoint{1.691543in}{1.997513in}}%
\pgfpathlineto{\pgfqpoint{1.686808in}{2.043143in}}%
\pgfpathlineto{\pgfqpoint{1.681968in}{2.086278in}}%
\pgfpathlineto{\pgfqpoint{1.677018in}{2.127066in}}%
\pgfpathlineto{\pgfqpoint{1.671953in}{2.165645in}}%
\pgfpathlineto{\pgfqpoint{1.666768in}{2.202146in}}%
\pgfpathlineto{\pgfqpoint{1.661456in}{2.236695in}}%
\pgfpathlineto{\pgfqpoint{1.656011in}{2.269409in}}%
\pgfpathlineto{\pgfqpoint{1.650426in}{2.300401in}}%
\pgfpathlineto{\pgfqpoint{1.644694in}{2.329775in}}%
\pgfpathlineto{\pgfqpoint{1.638808in}{2.357632in}}%
\pgfpathlineto{\pgfqpoint{1.632758in}{2.384065in}}%
\pgfpathlineto{\pgfqpoint{1.626535in}{2.409162in}}%
\pgfpathlineto{\pgfqpoint{1.620129in}{2.433005in}}%
\pgfpathlineto{\pgfqpoint{1.613528in}{2.455672in}}%
\pgfpathlineto{\pgfqpoint{1.606722in}{2.477239in}}%
\pgfpathlineto{\pgfqpoint{1.599696in}{2.497776in}}%
\pgfpathlineto{\pgfqpoint{1.592435in}{2.517358in}}%
\pgfpathlineto{\pgfqpoint{1.584924in}{2.536078in}}%
\pgfpathlineto{\pgfqpoint{1.577145in}{2.554121in}}%
\pgfpathlineto{\pgfqpoint{1.569077in}{2.571998in}}%
\pgfpathlineto{\pgfqpoint{1.560699in}{2.589826in}}%
\pgfpathlineto{\pgfqpoint{1.551986in}{2.607017in}}%
\pgfpathlineto{\pgfqpoint{1.542909in}{2.623361in}}%
\pgfpathlineto{\pgfqpoint{1.533438in}{2.638817in}}%
\pgfpathlineto{\pgfqpoint{1.523536in}{2.653380in}}%
\pgfpathlineto{\pgfqpoint{1.513162in}{2.667054in}}%
\pgfpathlineto{\pgfqpoint{1.502269in}{2.679845in}}%
\pgfpathlineto{\pgfqpoint{1.490802in}{2.691758in}}%
\pgfpathlineto{\pgfqpoint{1.478698in}{2.702798in}}%
\pgfpathlineto{\pgfqpoint{1.465881in}{2.712969in}}%
\pgfpathlineto{\pgfqpoint{1.452262in}{2.722273in}}%
\pgfpathlineto{\pgfqpoint{1.437735in}{2.730712in}}%
\pgfpathlineto{\pgfqpoint{1.422168in}{2.738284in}}%
\pgfpathlineto{\pgfqpoint{1.405403in}{2.744990in}}%
\pgfpathlineto{\pgfqpoint{1.387238in}{2.750897in}}%
\pgfpathlineto{\pgfqpoint{1.367419in}{2.758813in}}%
\pgfpathlineto{\pgfqpoint{1.345614in}{2.766753in}}%
\pgfpathlineto{\pgfqpoint{1.321380in}{2.774369in}}%
\pgfpathlineto{\pgfqpoint{1.294107in}{2.781675in}}%
\pgfpathlineto{\pgfqpoint{1.262923in}{2.788690in}}%
\pgfpathlineto{\pgfqpoint{1.226515in}{2.795436in}}%
\pgfpathlineto{\pgfqpoint{1.182772in}{2.801933in}}%
\pgfpathlineto{\pgfqpoint{1.127977in}{2.808204in}}%
\pgfpathlineto{\pgfqpoint{1.054581in}{2.814274in}}%
\pgfpathlineto{\pgfqpoint{0.943050in}{2.820169in}}%
\pgfpathlineto{\pgfqpoint{0.702340in}{2.825915in}}%
\pgfpathclose%
\pgfusepath{stroke,fill}%
\end{pgfscope}%
\begin{pgfscope}%
\pgfpathrectangle{\pgfqpoint{0.702340in}{2.624046in}}{\pgfqpoint{1.223103in}{0.607948in}}%
\pgfusepath{clip}%
\pgfsetroundcap%
\pgfsetroundjoin%
\pgfsetlinewidth{0.501875pt}%
\definecolor{currentstroke}{rgb}{0.000000,0.000000,1.000000}%
\pgfsetstrokecolor{currentstroke}%
\pgfsetstrokeopacity{0.800000}%
\pgfsetdash{}{0pt}%
\pgfpathmoveto{\pgfqpoint{0.702340in}{2.789850in}}%
\pgfpathlineto{\pgfqpoint{1.925444in}{2.789850in}}%
\pgfusepath{stroke}%
\end{pgfscope}%
\begin{pgfscope}%
\pgfpathrectangle{\pgfqpoint{0.702340in}{2.624046in}}{\pgfqpoint{1.223103in}{0.607948in}}%
\pgfusepath{clip}%
\pgfsetbuttcap%
\pgfsetroundjoin%
\pgfsetlinewidth{1.003750pt}%
\definecolor{currentstroke}{rgb}{0.000000,0.000000,0.000000}%
\pgfsetstrokecolor{currentstroke}%
\pgfsetdash{{3.700000pt}{1.600000pt}}{0.000000pt}%
\pgfpathmoveto{\pgfqpoint{0.702340in}{2.799107in}}%
\pgfpathlineto{\pgfqpoint{1.925444in}{2.799107in}}%
\pgfusepath{stroke}%
\end{pgfscope}%
\begin{pgfscope}%
\pgfsetroundcap%
\pgfsetroundjoin%
\pgfsetlinewidth{0.501875pt}%
\definecolor{currentstroke}{rgb}{0.000000,0.000000,1.000000}%
\pgfsetstrokecolor{currentstroke}%
\pgfsetstrokeopacity{0.800000}%
\pgfsetdash{}{0pt}%
\pgfpathmoveto{\pgfqpoint{1.503033in}{2.906355in}}%
\pgfpathquadraticcurveto{\pgfqpoint{1.439785in}{2.856682in}}{\pgfqpoint{1.376536in}{2.807009in}}%
\pgfusepath{stroke}%
\end{pgfscope}%
\begin{pgfscope}%
\pgfsetfillopacity{0.800000}%
\pgfsetstrokeopacity{0.800000}%
\definecolor{textcolor}{rgb}{0.000000,0.000000,1.000000}%
\pgfsetstrokecolor{textcolor}%
\pgfsetfillcolor{textcolor}%
\pgftext[x=1.442982in,y=2.972235in,left,base]{\color{textcolor}\sffamily\fontsize{5.647059}{6.776471}\selectfont 8.273(64)}%
\end{pgfscope}%
\begin{pgfscope}%
\pgfsetbuttcap%
\pgfsetroundjoin%
\definecolor{currentfill}{rgb}{0.150000,0.150000,0.150000}%
\pgfsetfillcolor{currentfill}%
\pgfsetlinewidth{1.003750pt}%
\definecolor{currentstroke}{rgb}{0.150000,0.150000,0.150000}%
\pgfsetstrokecolor{currentstroke}%
\pgfsetdash{}{0pt}%
\pgfsys@defobject{currentmarker}{\pgfqpoint{0.000000in}{-0.066667in}}{\pgfqpoint{0.000000in}{0.000000in}}{%
\pgfpathmoveto{\pgfqpoint{0.000000in}{0.000000in}}%
\pgfpathlineto{\pgfqpoint{0.000000in}{-0.066667in}}%
\pgfusepath{stroke,fill}%
}%
\begin{pgfscope}%
\pgfsys@transformshift{0.702340in}{2.624046in}%
\pgfsys@useobject{currentmarker}{}%
\end{pgfscope}%
\end{pgfscope}%
\begin{pgfscope}%
\pgfsetbuttcap%
\pgfsetroundjoin%
\definecolor{currentfill}{rgb}{0.150000,0.150000,0.150000}%
\pgfsetfillcolor{currentfill}%
\pgfsetlinewidth{1.003750pt}%
\definecolor{currentstroke}{rgb}{0.150000,0.150000,0.150000}%
\pgfsetstrokecolor{currentstroke}%
\pgfsetdash{}{0pt}%
\pgfsys@defobject{currentmarker}{\pgfqpoint{0.000000in}{-0.066667in}}{\pgfqpoint{0.000000in}{0.000000in}}{%
\pgfpathmoveto{\pgfqpoint{0.000000in}{0.000000in}}%
\pgfpathlineto{\pgfqpoint{0.000000in}{-0.066667in}}%
\pgfusepath{stroke,fill}%
}%
\begin{pgfscope}%
\pgfsys@transformshift{1.203749in}{2.624046in}%
\pgfsys@useobject{currentmarker}{}%
\end{pgfscope}%
\end{pgfscope}%
\begin{pgfscope}%
\pgfsetbuttcap%
\pgfsetroundjoin%
\definecolor{currentfill}{rgb}{0.150000,0.150000,0.150000}%
\pgfsetfillcolor{currentfill}%
\pgfsetlinewidth{1.003750pt}%
\definecolor{currentstroke}{rgb}{0.150000,0.150000,0.150000}%
\pgfsetstrokecolor{currentstroke}%
\pgfsetdash{}{0pt}%
\pgfsys@defobject{currentmarker}{\pgfqpoint{0.000000in}{-0.066667in}}{\pgfqpoint{0.000000in}{0.000000in}}{%
\pgfpathmoveto{\pgfqpoint{0.000000in}{0.000000in}}%
\pgfpathlineto{\pgfqpoint{0.000000in}{-0.066667in}}%
\pgfusepath{stroke,fill}%
}%
\begin{pgfscope}%
\pgfsys@transformshift{1.705158in}{2.624046in}%
\pgfsys@useobject{currentmarker}{}%
\end{pgfscope}%
\end{pgfscope}%
\begin{pgfscope}%
\pgfsetbuttcap%
\pgfsetroundjoin%
\definecolor{currentfill}{rgb}{0.150000,0.150000,0.150000}%
\pgfsetfillcolor{currentfill}%
\pgfsetlinewidth{0.803000pt}%
\definecolor{currentstroke}{rgb}{0.150000,0.150000,0.150000}%
\pgfsetstrokecolor{currentstroke}%
\pgfsetdash{}{0pt}%
\pgfsys@defobject{currentmarker}{\pgfqpoint{0.000000in}{-0.044444in}}{\pgfqpoint{0.000000in}{0.000000in}}{%
\pgfpathmoveto{\pgfqpoint{0.000000in}{0.000000in}}%
\pgfpathlineto{\pgfqpoint{0.000000in}{-0.044444in}}%
\pgfusepath{stroke,fill}%
}%
\begin{pgfscope}%
\pgfsys@transformshift{0.853280in}{2.624046in}%
\pgfsys@useobject{currentmarker}{}%
\end{pgfscope}%
\end{pgfscope}%
\begin{pgfscope}%
\pgfsetbuttcap%
\pgfsetroundjoin%
\definecolor{currentfill}{rgb}{0.150000,0.150000,0.150000}%
\pgfsetfillcolor{currentfill}%
\pgfsetlinewidth{0.803000pt}%
\definecolor{currentstroke}{rgb}{0.150000,0.150000,0.150000}%
\pgfsetstrokecolor{currentstroke}%
\pgfsetdash{}{0pt}%
\pgfsys@defobject{currentmarker}{\pgfqpoint{0.000000in}{-0.044444in}}{\pgfqpoint{0.000000in}{0.000000in}}{%
\pgfpathmoveto{\pgfqpoint{0.000000in}{0.000000in}}%
\pgfpathlineto{\pgfqpoint{0.000000in}{-0.044444in}}%
\pgfusepath{stroke,fill}%
}%
\begin{pgfscope}%
\pgfsys@transformshift{0.941573in}{2.624046in}%
\pgfsys@useobject{currentmarker}{}%
\end{pgfscope}%
\end{pgfscope}%
\begin{pgfscope}%
\pgfsetbuttcap%
\pgfsetroundjoin%
\definecolor{currentfill}{rgb}{0.150000,0.150000,0.150000}%
\pgfsetfillcolor{currentfill}%
\pgfsetlinewidth{0.803000pt}%
\definecolor{currentstroke}{rgb}{0.150000,0.150000,0.150000}%
\pgfsetstrokecolor{currentstroke}%
\pgfsetdash{}{0pt}%
\pgfsys@defobject{currentmarker}{\pgfqpoint{0.000000in}{-0.044444in}}{\pgfqpoint{0.000000in}{0.000000in}}{%
\pgfpathmoveto{\pgfqpoint{0.000000in}{0.000000in}}%
\pgfpathlineto{\pgfqpoint{0.000000in}{-0.044444in}}%
\pgfusepath{stroke,fill}%
}%
\begin{pgfscope}%
\pgfsys@transformshift{1.004219in}{2.624046in}%
\pgfsys@useobject{currentmarker}{}%
\end{pgfscope}%
\end{pgfscope}%
\begin{pgfscope}%
\pgfsetbuttcap%
\pgfsetroundjoin%
\definecolor{currentfill}{rgb}{0.150000,0.150000,0.150000}%
\pgfsetfillcolor{currentfill}%
\pgfsetlinewidth{0.803000pt}%
\definecolor{currentstroke}{rgb}{0.150000,0.150000,0.150000}%
\pgfsetstrokecolor{currentstroke}%
\pgfsetdash{}{0pt}%
\pgfsys@defobject{currentmarker}{\pgfqpoint{0.000000in}{-0.044444in}}{\pgfqpoint{0.000000in}{0.000000in}}{%
\pgfpathmoveto{\pgfqpoint{0.000000in}{0.000000in}}%
\pgfpathlineto{\pgfqpoint{0.000000in}{-0.044444in}}%
\pgfusepath{stroke,fill}%
}%
\begin{pgfscope}%
\pgfsys@transformshift{1.052810in}{2.624046in}%
\pgfsys@useobject{currentmarker}{}%
\end{pgfscope}%
\end{pgfscope}%
\begin{pgfscope}%
\pgfsetbuttcap%
\pgfsetroundjoin%
\definecolor{currentfill}{rgb}{0.150000,0.150000,0.150000}%
\pgfsetfillcolor{currentfill}%
\pgfsetlinewidth{0.803000pt}%
\definecolor{currentstroke}{rgb}{0.150000,0.150000,0.150000}%
\pgfsetstrokecolor{currentstroke}%
\pgfsetdash{}{0pt}%
\pgfsys@defobject{currentmarker}{\pgfqpoint{0.000000in}{-0.044444in}}{\pgfqpoint{0.000000in}{0.000000in}}{%
\pgfpathmoveto{\pgfqpoint{0.000000in}{0.000000in}}%
\pgfpathlineto{\pgfqpoint{0.000000in}{-0.044444in}}%
\pgfusepath{stroke,fill}%
}%
\begin{pgfscope}%
\pgfsys@transformshift{1.092512in}{2.624046in}%
\pgfsys@useobject{currentmarker}{}%
\end{pgfscope}%
\end{pgfscope}%
\begin{pgfscope}%
\pgfsetbuttcap%
\pgfsetroundjoin%
\definecolor{currentfill}{rgb}{0.150000,0.150000,0.150000}%
\pgfsetfillcolor{currentfill}%
\pgfsetlinewidth{0.803000pt}%
\definecolor{currentstroke}{rgb}{0.150000,0.150000,0.150000}%
\pgfsetstrokecolor{currentstroke}%
\pgfsetdash{}{0pt}%
\pgfsys@defobject{currentmarker}{\pgfqpoint{0.000000in}{-0.044444in}}{\pgfqpoint{0.000000in}{0.000000in}}{%
\pgfpathmoveto{\pgfqpoint{0.000000in}{0.000000in}}%
\pgfpathlineto{\pgfqpoint{0.000000in}{-0.044444in}}%
\pgfusepath{stroke,fill}%
}%
\begin{pgfscope}%
\pgfsys@transformshift{1.126080in}{2.624046in}%
\pgfsys@useobject{currentmarker}{}%
\end{pgfscope}%
\end{pgfscope}%
\begin{pgfscope}%
\pgfsetbuttcap%
\pgfsetroundjoin%
\definecolor{currentfill}{rgb}{0.150000,0.150000,0.150000}%
\pgfsetfillcolor{currentfill}%
\pgfsetlinewidth{0.803000pt}%
\definecolor{currentstroke}{rgb}{0.150000,0.150000,0.150000}%
\pgfsetstrokecolor{currentstroke}%
\pgfsetdash{}{0pt}%
\pgfsys@defobject{currentmarker}{\pgfqpoint{0.000000in}{-0.044444in}}{\pgfqpoint{0.000000in}{0.000000in}}{%
\pgfpathmoveto{\pgfqpoint{0.000000in}{0.000000in}}%
\pgfpathlineto{\pgfqpoint{0.000000in}{-0.044444in}}%
\pgfusepath{stroke,fill}%
}%
\begin{pgfscope}%
\pgfsys@transformshift{1.155158in}{2.624046in}%
\pgfsys@useobject{currentmarker}{}%
\end{pgfscope}%
\end{pgfscope}%
\begin{pgfscope}%
\pgfsetbuttcap%
\pgfsetroundjoin%
\definecolor{currentfill}{rgb}{0.150000,0.150000,0.150000}%
\pgfsetfillcolor{currentfill}%
\pgfsetlinewidth{0.803000pt}%
\definecolor{currentstroke}{rgb}{0.150000,0.150000,0.150000}%
\pgfsetstrokecolor{currentstroke}%
\pgfsetdash{}{0pt}%
\pgfsys@defobject{currentmarker}{\pgfqpoint{0.000000in}{-0.044444in}}{\pgfqpoint{0.000000in}{0.000000in}}{%
\pgfpathmoveto{\pgfqpoint{0.000000in}{0.000000in}}%
\pgfpathlineto{\pgfqpoint{0.000000in}{-0.044444in}}%
\pgfusepath{stroke,fill}%
}%
\begin{pgfscope}%
\pgfsys@transformshift{1.180806in}{2.624046in}%
\pgfsys@useobject{currentmarker}{}%
\end{pgfscope}%
\end{pgfscope}%
\begin{pgfscope}%
\pgfsetbuttcap%
\pgfsetroundjoin%
\definecolor{currentfill}{rgb}{0.150000,0.150000,0.150000}%
\pgfsetfillcolor{currentfill}%
\pgfsetlinewidth{0.803000pt}%
\definecolor{currentstroke}{rgb}{0.150000,0.150000,0.150000}%
\pgfsetstrokecolor{currentstroke}%
\pgfsetdash{}{0pt}%
\pgfsys@defobject{currentmarker}{\pgfqpoint{0.000000in}{-0.044444in}}{\pgfqpoint{0.000000in}{0.000000in}}{%
\pgfpathmoveto{\pgfqpoint{0.000000in}{0.000000in}}%
\pgfpathlineto{\pgfqpoint{0.000000in}{-0.044444in}}%
\pgfusepath{stroke,fill}%
}%
\begin{pgfscope}%
\pgfsys@transformshift{1.354689in}{2.624046in}%
\pgfsys@useobject{currentmarker}{}%
\end{pgfscope}%
\end{pgfscope}%
\begin{pgfscope}%
\pgfsetbuttcap%
\pgfsetroundjoin%
\definecolor{currentfill}{rgb}{0.150000,0.150000,0.150000}%
\pgfsetfillcolor{currentfill}%
\pgfsetlinewidth{0.803000pt}%
\definecolor{currentstroke}{rgb}{0.150000,0.150000,0.150000}%
\pgfsetstrokecolor{currentstroke}%
\pgfsetdash{}{0pt}%
\pgfsys@defobject{currentmarker}{\pgfqpoint{0.000000in}{-0.044444in}}{\pgfqpoint{0.000000in}{0.000000in}}{%
\pgfpathmoveto{\pgfqpoint{0.000000in}{0.000000in}}%
\pgfpathlineto{\pgfqpoint{0.000000in}{-0.044444in}}%
\pgfusepath{stroke,fill}%
}%
\begin{pgfscope}%
\pgfsys@transformshift{1.442982in}{2.624046in}%
\pgfsys@useobject{currentmarker}{}%
\end{pgfscope}%
\end{pgfscope}%
\begin{pgfscope}%
\pgfsetbuttcap%
\pgfsetroundjoin%
\definecolor{currentfill}{rgb}{0.150000,0.150000,0.150000}%
\pgfsetfillcolor{currentfill}%
\pgfsetlinewidth{0.803000pt}%
\definecolor{currentstroke}{rgb}{0.150000,0.150000,0.150000}%
\pgfsetstrokecolor{currentstroke}%
\pgfsetdash{}{0pt}%
\pgfsys@defobject{currentmarker}{\pgfqpoint{0.000000in}{-0.044444in}}{\pgfqpoint{0.000000in}{0.000000in}}{%
\pgfpathmoveto{\pgfqpoint{0.000000in}{0.000000in}}%
\pgfpathlineto{\pgfqpoint{0.000000in}{-0.044444in}}%
\pgfusepath{stroke,fill}%
}%
\begin{pgfscope}%
\pgfsys@transformshift{1.505628in}{2.624046in}%
\pgfsys@useobject{currentmarker}{}%
\end{pgfscope}%
\end{pgfscope}%
\begin{pgfscope}%
\pgfsetbuttcap%
\pgfsetroundjoin%
\definecolor{currentfill}{rgb}{0.150000,0.150000,0.150000}%
\pgfsetfillcolor{currentfill}%
\pgfsetlinewidth{0.803000pt}%
\definecolor{currentstroke}{rgb}{0.150000,0.150000,0.150000}%
\pgfsetstrokecolor{currentstroke}%
\pgfsetdash{}{0pt}%
\pgfsys@defobject{currentmarker}{\pgfqpoint{0.000000in}{-0.044444in}}{\pgfqpoint{0.000000in}{0.000000in}}{%
\pgfpathmoveto{\pgfqpoint{0.000000in}{0.000000in}}%
\pgfpathlineto{\pgfqpoint{0.000000in}{-0.044444in}}%
\pgfusepath{stroke,fill}%
}%
\begin{pgfscope}%
\pgfsys@transformshift{1.554219in}{2.624046in}%
\pgfsys@useobject{currentmarker}{}%
\end{pgfscope}%
\end{pgfscope}%
\begin{pgfscope}%
\pgfsetbuttcap%
\pgfsetroundjoin%
\definecolor{currentfill}{rgb}{0.150000,0.150000,0.150000}%
\pgfsetfillcolor{currentfill}%
\pgfsetlinewidth{0.803000pt}%
\definecolor{currentstroke}{rgb}{0.150000,0.150000,0.150000}%
\pgfsetstrokecolor{currentstroke}%
\pgfsetdash{}{0pt}%
\pgfsys@defobject{currentmarker}{\pgfqpoint{0.000000in}{-0.044444in}}{\pgfqpoint{0.000000in}{0.000000in}}{%
\pgfpathmoveto{\pgfqpoint{0.000000in}{0.000000in}}%
\pgfpathlineto{\pgfqpoint{0.000000in}{-0.044444in}}%
\pgfusepath{stroke,fill}%
}%
\begin{pgfscope}%
\pgfsys@transformshift{1.593921in}{2.624046in}%
\pgfsys@useobject{currentmarker}{}%
\end{pgfscope}%
\end{pgfscope}%
\begin{pgfscope}%
\pgfsetbuttcap%
\pgfsetroundjoin%
\definecolor{currentfill}{rgb}{0.150000,0.150000,0.150000}%
\pgfsetfillcolor{currentfill}%
\pgfsetlinewidth{0.803000pt}%
\definecolor{currentstroke}{rgb}{0.150000,0.150000,0.150000}%
\pgfsetstrokecolor{currentstroke}%
\pgfsetdash{}{0pt}%
\pgfsys@defobject{currentmarker}{\pgfqpoint{0.000000in}{-0.044444in}}{\pgfqpoint{0.000000in}{0.000000in}}{%
\pgfpathmoveto{\pgfqpoint{0.000000in}{0.000000in}}%
\pgfpathlineto{\pgfqpoint{0.000000in}{-0.044444in}}%
\pgfusepath{stroke,fill}%
}%
\begin{pgfscope}%
\pgfsys@transformshift{1.627489in}{2.624046in}%
\pgfsys@useobject{currentmarker}{}%
\end{pgfscope}%
\end{pgfscope}%
\begin{pgfscope}%
\pgfsetbuttcap%
\pgfsetroundjoin%
\definecolor{currentfill}{rgb}{0.150000,0.150000,0.150000}%
\pgfsetfillcolor{currentfill}%
\pgfsetlinewidth{0.803000pt}%
\definecolor{currentstroke}{rgb}{0.150000,0.150000,0.150000}%
\pgfsetstrokecolor{currentstroke}%
\pgfsetdash{}{0pt}%
\pgfsys@defobject{currentmarker}{\pgfqpoint{0.000000in}{-0.044444in}}{\pgfqpoint{0.000000in}{0.000000in}}{%
\pgfpathmoveto{\pgfqpoint{0.000000in}{0.000000in}}%
\pgfpathlineto{\pgfqpoint{0.000000in}{-0.044444in}}%
\pgfusepath{stroke,fill}%
}%
\begin{pgfscope}%
\pgfsys@transformshift{1.656567in}{2.624046in}%
\pgfsys@useobject{currentmarker}{}%
\end{pgfscope}%
\end{pgfscope}%
\begin{pgfscope}%
\pgfsetbuttcap%
\pgfsetroundjoin%
\definecolor{currentfill}{rgb}{0.150000,0.150000,0.150000}%
\pgfsetfillcolor{currentfill}%
\pgfsetlinewidth{0.803000pt}%
\definecolor{currentstroke}{rgb}{0.150000,0.150000,0.150000}%
\pgfsetstrokecolor{currentstroke}%
\pgfsetdash{}{0pt}%
\pgfsys@defobject{currentmarker}{\pgfqpoint{0.000000in}{-0.044444in}}{\pgfqpoint{0.000000in}{0.000000in}}{%
\pgfpathmoveto{\pgfqpoint{0.000000in}{0.000000in}}%
\pgfpathlineto{\pgfqpoint{0.000000in}{-0.044444in}}%
\pgfusepath{stroke,fill}%
}%
\begin{pgfscope}%
\pgfsys@transformshift{1.682215in}{2.624046in}%
\pgfsys@useobject{currentmarker}{}%
\end{pgfscope}%
\end{pgfscope}%
\begin{pgfscope}%
\pgfsetbuttcap%
\pgfsetroundjoin%
\definecolor{currentfill}{rgb}{0.150000,0.150000,0.150000}%
\pgfsetfillcolor{currentfill}%
\pgfsetlinewidth{0.803000pt}%
\definecolor{currentstroke}{rgb}{0.150000,0.150000,0.150000}%
\pgfsetstrokecolor{currentstroke}%
\pgfsetdash{}{0pt}%
\pgfsys@defobject{currentmarker}{\pgfqpoint{0.000000in}{-0.044444in}}{\pgfqpoint{0.000000in}{0.000000in}}{%
\pgfpathmoveto{\pgfqpoint{0.000000in}{0.000000in}}%
\pgfpathlineto{\pgfqpoint{0.000000in}{-0.044444in}}%
\pgfusepath{stroke,fill}%
}%
\begin{pgfscope}%
\pgfsys@transformshift{1.856098in}{2.624046in}%
\pgfsys@useobject{currentmarker}{}%
\end{pgfscope}%
\end{pgfscope}%
\begin{pgfscope}%
\pgfsetbuttcap%
\pgfsetroundjoin%
\definecolor{currentfill}{rgb}{0.150000,0.150000,0.150000}%
\pgfsetfillcolor{currentfill}%
\pgfsetlinewidth{1.003750pt}%
\definecolor{currentstroke}{rgb}{0.150000,0.150000,0.150000}%
\pgfsetstrokecolor{currentstroke}%
\pgfsetdash{}{0pt}%
\pgfsys@defobject{currentmarker}{\pgfqpoint{-0.066667in}{0.000000in}}{\pgfqpoint{0.000000in}{0.000000in}}{%
\pgfpathmoveto{\pgfqpoint{0.000000in}{0.000000in}}%
\pgfpathlineto{\pgfqpoint{-0.066667in}{0.000000in}}%
\pgfusepath{stroke,fill}%
}%
\begin{pgfscope}%
\pgfsys@transformshift{0.702340in}{2.624046in}%
\pgfsys@useobject{currentmarker}{}%
\end{pgfscope}%
\end{pgfscope}%
\begin{pgfscope}%
\definecolor{textcolor}{rgb}{0.150000,0.150000,0.150000}%
\pgfsetstrokecolor{textcolor}%
\pgfsetfillcolor{textcolor}%
\pgftext[x=0.413148in,y=2.599098in,left,base]{\color{textcolor}\sffamily\fontsize{5.176471}{6.211765}\selectfont 8.000}%
\end{pgfscope}%
\begin{pgfscope}%
\pgfsetbuttcap%
\pgfsetroundjoin%
\definecolor{currentfill}{rgb}{0.150000,0.150000,0.150000}%
\pgfsetfillcolor{currentfill}%
\pgfsetlinewidth{1.003750pt}%
\definecolor{currentstroke}{rgb}{0.150000,0.150000,0.150000}%
\pgfsetstrokecolor{currentstroke}%
\pgfsetdash{}{0pt}%
\pgfsys@defobject{currentmarker}{\pgfqpoint{-0.066667in}{0.000000in}}{\pgfqpoint{0.000000in}{0.000000in}}{%
\pgfpathmoveto{\pgfqpoint{0.000000in}{0.000000in}}%
\pgfpathlineto{\pgfqpoint{-0.066667in}{0.000000in}}%
\pgfusepath{stroke,fill}%
}%
\begin{pgfscope}%
\pgfsys@transformshift{0.702340in}{2.799107in}%
\pgfsys@useobject{currentmarker}{}%
\end{pgfscope}%
\end{pgfscope}%
\begin{pgfscope}%
\definecolor{textcolor}{rgb}{0.150000,0.150000,0.150000}%
\pgfsetstrokecolor{textcolor}%
\pgfsetfillcolor{textcolor}%
\pgftext[x=0.413148in,y=2.774159in,left,base]{\color{textcolor}\sffamily\fontsize{5.176471}{6.211765}\selectfont 8.288}%
\end{pgfscope}%
\begin{pgfscope}%
\pgfsetbuttcap%
\pgfsetroundjoin%
\definecolor{currentfill}{rgb}{0.150000,0.150000,0.150000}%
\pgfsetfillcolor{currentfill}%
\pgfsetlinewidth{1.003750pt}%
\definecolor{currentstroke}{rgb}{0.150000,0.150000,0.150000}%
\pgfsetstrokecolor{currentstroke}%
\pgfsetdash{}{0pt}%
\pgfsys@defobject{currentmarker}{\pgfqpoint{-0.066667in}{0.000000in}}{\pgfqpoint{0.000000in}{0.000000in}}{%
\pgfpathmoveto{\pgfqpoint{0.000000in}{0.000000in}}%
\pgfpathlineto{\pgfqpoint{-0.066667in}{0.000000in}}%
\pgfusepath{stroke,fill}%
}%
\begin{pgfscope}%
\pgfsys@transformshift{0.702340in}{3.231994in}%
\pgfsys@useobject{currentmarker}{}%
\end{pgfscope}%
\end{pgfscope}%
\begin{pgfscope}%
\definecolor{textcolor}{rgb}{0.150000,0.150000,0.150000}%
\pgfsetstrokecolor{textcolor}%
\pgfsetfillcolor{textcolor}%
\pgftext[x=0.413148in,y=3.207046in,left,base]{\color{textcolor}\sffamily\fontsize{5.176471}{6.211765}\selectfont 9.000}%
\end{pgfscope}%
\begin{pgfscope}%
\definecolor{textcolor}{rgb}{0.150000,0.150000,0.150000}%
\pgfsetstrokecolor{textcolor}%
\pgfsetfillcolor{textcolor}%
\pgftext[x=0.357592in,y=2.928020in,,bottom,rotate=90.000000]{\color{textcolor}\sffamily\fontsize{5.647059}{6.776471}\selectfont \(\displaystyle x = \frac{2 \mu E L^2}{4 \pi^2}\)}%
\end{pgfscope}%
\begin{pgfscope}%
\pgfpathrectangle{\pgfqpoint{0.702340in}{2.624046in}}{\pgfqpoint{1.223103in}{0.607948in}}%
\pgfusepath{clip}%
\pgfsetroundcap%
\pgfsetroundjoin%
\pgfsetlinewidth{1.204500pt}%
\definecolor{currentstroke}{rgb}{0.000000,0.501961,0.000000}%
\pgfsetstrokecolor{currentstroke}%
\pgfsetdash{}{0pt}%
\pgfpathmoveto{\pgfqpoint{0.702340in}{2.789820in}}%
\pgfpathlineto{\pgfqpoint{0.943050in}{2.789302in}}%
\pgfpathlineto{\pgfqpoint{1.054581in}{2.788177in}}%
\pgfpathlineto{\pgfqpoint{1.127977in}{2.786443in}}%
\pgfpathlineto{\pgfqpoint{1.182772in}{2.784102in}}%
\pgfpathlineto{\pgfqpoint{1.226515in}{2.781154in}}%
\pgfpathlineto{\pgfqpoint{1.262923in}{2.777599in}}%
\pgfpathlineto{\pgfqpoint{1.294107in}{2.773435in}}%
\pgfpathlineto{\pgfqpoint{1.321380in}{2.768662in}}%
\pgfpathlineto{\pgfqpoint{1.345614in}{2.763276in}}%
\pgfpathlineto{\pgfqpoint{1.367419in}{2.757275in}}%
\pgfpathlineto{\pgfqpoint{1.387238in}{2.750655in}}%
\pgfpathlineto{\pgfqpoint{1.405403in}{2.743411in}}%
\pgfpathlineto{\pgfqpoint{1.422168in}{2.735536in}}%
\pgfpathlineto{\pgfqpoint{1.437735in}{2.727025in}}%
\pgfpathlineto{\pgfqpoint{1.452262in}{2.717868in}}%
\pgfpathlineto{\pgfqpoint{1.465881in}{2.708056in}}%
\pgfpathlineto{\pgfqpoint{1.478698in}{2.697578in}}%
\pgfpathlineto{\pgfqpoint{1.490802in}{2.686423in}}%
\pgfpathlineto{\pgfqpoint{1.502269in}{2.674575in}}%
\pgfpathlineto{\pgfqpoint{1.513162in}{2.662020in}}%
\pgfpathlineto{\pgfqpoint{1.523536in}{2.648741in}}%
\pgfpathlineto{\pgfqpoint{1.533438in}{2.634719in}}%
\pgfpathlineto{\pgfqpoint{1.542411in}{2.620712in}}%
\pgfusepath{stroke}%
\end{pgfscope}%
\begin{pgfscope}%
\pgfsetrectcap%
\pgfsetmiterjoin%
\pgfsetlinewidth{1.003750pt}%
\definecolor{currentstroke}{rgb}{0.150000,0.150000,0.150000}%
\pgfsetstrokecolor{currentstroke}%
\pgfsetdash{}{0pt}%
\pgfpathmoveto{\pgfqpoint{0.702340in}{2.624046in}}%
\pgfpathlineto{\pgfqpoint{0.702340in}{3.231994in}}%
\pgfusepath{stroke}%
\end{pgfscope}%
\begin{pgfscope}%
\pgfsetrectcap%
\pgfsetmiterjoin%
\pgfsetlinewidth{1.003750pt}%
\definecolor{currentstroke}{rgb}{0.150000,0.150000,0.150000}%
\pgfsetstrokecolor{currentstroke}%
\pgfsetdash{}{0pt}%
\pgfpathmoveto{\pgfqpoint{0.702340in}{2.624046in}}%
\pgfpathlineto{\pgfqpoint{1.925444in}{2.624046in}}%
\pgfusepath{stroke}%
\end{pgfscope}%
\begin{pgfscope}%
\pgfpathrectangle{\pgfqpoint{0.702340in}{2.624046in}}{\pgfqpoint{1.223103in}{0.607948in}}%
\pgfusepath{clip}%
\pgfsetbuttcap%
\pgfsetroundjoin%
\definecolor{currentfill}{rgb}{0.000000,0.000000,0.000000}%
\pgfsetfillcolor{currentfill}%
\pgfsetlinewidth{1.003750pt}%
\definecolor{currentstroke}{rgb}{0.000000,0.000000,0.000000}%
\pgfsetstrokecolor{currentstroke}%
\pgfsetdash{}{0pt}%
\pgfsys@defobject{currentmarker}{\pgfqpoint{-0.013889in}{-0.013889in}}{\pgfqpoint{0.013889in}{0.013889in}}{%
\pgfpathmoveto{\pgfqpoint{0.000000in}{-0.013889in}}%
\pgfpathcurveto{\pgfqpoint{0.003683in}{-0.013889in}}{\pgfqpoint{0.007216in}{-0.012425in}}{\pgfqpoint{0.009821in}{-0.009821in}}%
\pgfpathcurveto{\pgfqpoint{0.012425in}{-0.007216in}}{\pgfqpoint{0.013889in}{-0.003683in}}{\pgfqpoint{0.013889in}{0.000000in}}%
\pgfpathcurveto{\pgfqpoint{0.013889in}{0.003683in}}{\pgfqpoint{0.012425in}{0.007216in}}{\pgfqpoint{0.009821in}{0.009821in}}%
\pgfpathcurveto{\pgfqpoint{0.007216in}{0.012425in}}{\pgfqpoint{0.003683in}{0.013889in}}{\pgfqpoint{0.000000in}{0.013889in}}%
\pgfpathcurveto{\pgfqpoint{-0.003683in}{0.013889in}}{\pgfqpoint{-0.007216in}{0.012425in}}{\pgfqpoint{-0.009821in}{0.009821in}}%
\pgfpathcurveto{\pgfqpoint{-0.012425in}{0.007216in}}{\pgfqpoint{-0.013889in}{0.003683in}}{\pgfqpoint{-0.013889in}{0.000000in}}%
\pgfpathcurveto{\pgfqpoint{-0.013889in}{-0.003683in}}{\pgfqpoint{-0.012425in}{-0.007216in}}{\pgfqpoint{-0.009821in}{-0.009821in}}%
\pgfpathcurveto{\pgfqpoint{-0.007216in}{-0.012425in}}{\pgfqpoint{-0.003683in}{-0.013889in}}{\pgfqpoint{0.000000in}{-0.013889in}}%
\pgfpathclose%
\pgfusepath{stroke,fill}%
}%
\begin{pgfscope}%
\pgfsys@transformshift{1.705158in}{1.844638in}%
\pgfsys@useobject{currentmarker}{}%
\end{pgfscope}%
\begin{pgfscope}%
\pgfsys@transformshift{1.616865in}{2.441387in}%
\pgfsys@useobject{currentmarker}{}%
\end{pgfscope}%
\begin{pgfscope}%
\pgfsys@transformshift{1.554219in}{2.598008in}%
\pgfsys@useobject{currentmarker}{}%
\end{pgfscope}%
\begin{pgfscope}%
\pgfsys@transformshift{1.505628in}{2.667625in}%
\pgfsys@useobject{currentmarker}{}%
\end{pgfscope}%
\begin{pgfscope}%
\pgfsys@transformshift{1.465926in}{2.705564in}%
\pgfsys@useobject{currentmarker}{}%
\end{pgfscope}%
\begin{pgfscope}%
\pgfsys@transformshift{1.432358in}{2.728681in}%
\pgfsys@useobject{currentmarker}{}%
\end{pgfscope}%
\begin{pgfscope}%
\pgfsys@transformshift{1.403280in}{2.743871in}%
\pgfsys@useobject{currentmarker}{}%
\end{pgfscope}%
\begin{pgfscope}%
\pgfsys@transformshift{1.397903in}{2.746281in}%
\pgfsys@useobject{currentmarker}{}%
\end{pgfscope}%
\begin{pgfscope}%
\pgfsys@transformshift{1.392656in}{2.748528in}%
\pgfsys@useobject{currentmarker}{}%
\end{pgfscope}%
\begin{pgfscope}%
\pgfsys@transformshift{1.387532in}{2.750624in}%
\pgfsys@useobject{currentmarker}{}%
\end{pgfscope}%
\begin{pgfscope}%
\pgfsys@transformshift{1.382525in}{2.752585in}%
\pgfsys@useobject{currentmarker}{}%
\end{pgfscope}%
\begin{pgfscope}%
\pgfsys@transformshift{1.377632in}{2.754421in}%
\pgfsys@useobject{currentmarker}{}%
\end{pgfscope}%
\begin{pgfscope}%
\pgfsys@transformshift{1.372846in}{2.756143in}%
\pgfsys@useobject{currentmarker}{}%
\end{pgfscope}%
\begin{pgfscope}%
\pgfsys@transformshift{1.368163in}{2.757760in}%
\pgfsys@useobject{currentmarker}{}%
\end{pgfscope}%
\begin{pgfscope}%
\pgfsys@transformshift{1.363578in}{2.759282in}%
\pgfsys@useobject{currentmarker}{}%
\end{pgfscope}%
\begin{pgfscope}%
\pgfsys@transformshift{1.359088in}{2.760715in}%
\pgfsys@useobject{currentmarker}{}%
\end{pgfscope}%
\begin{pgfscope}%
\pgfsys@transformshift{1.354689in}{2.762066in}%
\pgfsys@useobject{currentmarker}{}%
\end{pgfscope}%
\end{pgfscope}%
\begin{pgfscope}%
\pgfsetbuttcap%
\pgfsetmiterjoin%
\definecolor{currentfill}{rgb}{1.000000,1.000000,1.000000}%
\pgfsetfillcolor{currentfill}%
\pgfsetlinewidth{0.000000pt}%
\definecolor{currentstroke}{rgb}{0.000000,0.000000,0.000000}%
\pgfsetstrokecolor{currentstroke}%
\pgfsetstrokeopacity{0.000000}%
\pgfsetdash{}{0pt}%
\pgfpathmoveto{\pgfqpoint{2.170064in}{2.624046in}}%
\pgfpathlineto{\pgfqpoint{3.393168in}{2.624046in}}%
\pgfpathlineto{\pgfqpoint{3.393168in}{3.231994in}}%
\pgfpathlineto{\pgfqpoint{2.170064in}{3.231994in}}%
\pgfpathclose%
\pgfusepath{fill}%
\end{pgfscope}%
\begin{pgfscope}%
\pgfpathrectangle{\pgfqpoint{2.170064in}{2.624046in}}{\pgfqpoint{1.223103in}{0.607948in}}%
\pgfusepath{clip}%
\pgfsetbuttcap%
\pgfsetmiterjoin%
\definecolor{currentfill}{rgb}{0.000000,0.000000,1.000000}%
\pgfsetfillcolor{currentfill}%
\pgfsetfillopacity{0.100000}%
\pgfsetlinewidth{0.803000pt}%
\definecolor{currentstroke}{rgb}{0.000000,0.000000,1.000000}%
\pgfsetstrokecolor{currentstroke}%
\pgfsetstrokeopacity{0.100000}%
\pgfsetdash{}{0pt}%
\pgfpathmoveto{\pgfqpoint{2.170064in}{2.767933in}}%
\pgfpathlineto{\pgfqpoint{2.170064in}{2.814085in}}%
\pgfpathlineto{\pgfqpoint{3.393168in}{2.814085in}}%
\pgfpathlineto{\pgfqpoint{3.393168in}{2.767933in}}%
\pgfpathclose%
\pgfusepath{stroke,fill}%
\end{pgfscope}%
\begin{pgfscope}%
\pgfpathrectangle{\pgfqpoint{2.170064in}{2.624046in}}{\pgfqpoint{1.223103in}{0.607948in}}%
\pgfusepath{clip}%
\pgfsetbuttcap%
\pgfsetroundjoin%
\definecolor{currentfill}{rgb}{0.000000,0.501961,0.000000}%
\pgfsetfillcolor{currentfill}%
\pgfsetfillopacity{0.500000}%
\pgfsetlinewidth{0.803000pt}%
\definecolor{currentstroke}{rgb}{0.000000,0.501961,0.000000}%
\pgfsetstrokecolor{currentstroke}%
\pgfsetstrokeopacity{0.500000}%
\pgfsetdash{}{0pt}%
\pgfpathmoveto{\pgfqpoint{2.170064in}{2.813127in}}%
\pgfpathlineto{\pgfqpoint{2.170064in}{2.770234in}}%
\pgfpathlineto{\pgfqpoint{2.410774in}{2.774673in}}%
\pgfpathlineto{\pgfqpoint{2.522305in}{2.778831in}}%
\pgfpathlineto{\pgfqpoint{2.595701in}{2.782711in}}%
\pgfpathlineto{\pgfqpoint{2.650497in}{2.786315in}}%
\pgfpathlineto{\pgfqpoint{2.694239in}{2.789642in}}%
\pgfpathlineto{\pgfqpoint{2.730647in}{2.792693in}}%
\pgfpathlineto{\pgfqpoint{2.761831in}{2.795469in}}%
\pgfpathlineto{\pgfqpoint{2.789104in}{2.797968in}}%
\pgfpathlineto{\pgfqpoint{2.813338in}{2.800189in}}%
\pgfpathlineto{\pgfqpoint{2.835143in}{2.802131in}}%
\pgfpathlineto{\pgfqpoint{2.854962in}{2.803503in}}%
\pgfpathlineto{\pgfqpoint{2.873127in}{2.803272in}}%
\pgfpathlineto{\pgfqpoint{2.889892in}{2.803056in}}%
\pgfpathlineto{\pgfqpoint{2.905459in}{2.802847in}}%
\pgfpathlineto{\pgfqpoint{2.919986in}{2.802626in}}%
\pgfpathlineto{\pgfqpoint{2.933605in}{2.802374in}}%
\pgfpathlineto{\pgfqpoint{2.946422in}{2.802071in}}%
\pgfpathlineto{\pgfqpoint{2.958526in}{2.801696in}}%
\pgfpathlineto{\pgfqpoint{2.969993in}{2.801230in}}%
\pgfpathlineto{\pgfqpoint{2.980886in}{2.800650in}}%
\pgfpathlineto{\pgfqpoint{2.991260in}{2.799933in}}%
\pgfpathlineto{\pgfqpoint{3.001162in}{2.799055in}}%
\pgfpathlineto{\pgfqpoint{3.010633in}{2.797989in}}%
\pgfpathlineto{\pgfqpoint{3.019710in}{2.796691in}}%
\pgfpathlineto{\pgfqpoint{3.028423in}{2.795018in}}%
\pgfpathlineto{\pgfqpoint{3.036801in}{2.792419in}}%
\pgfpathlineto{\pgfqpoint{3.044869in}{2.789096in}}%
\pgfpathlineto{\pgfqpoint{3.052648in}{2.785345in}}%
\pgfpathlineto{\pgfqpoint{3.060159in}{2.781199in}}%
\pgfpathlineto{\pgfqpoint{3.067420in}{2.776656in}}%
\pgfpathlineto{\pgfqpoint{3.074446in}{2.771709in}}%
\pgfpathlineto{\pgfqpoint{3.081253in}{2.766349in}}%
\pgfpathlineto{\pgfqpoint{3.087853in}{2.760567in}}%
\pgfpathlineto{\pgfqpoint{3.094259in}{2.754351in}}%
\pgfpathlineto{\pgfqpoint{3.100482in}{2.747692in}}%
\pgfpathlineto{\pgfqpoint{3.106532in}{2.740578in}}%
\pgfpathlineto{\pgfqpoint{3.112419in}{2.732998in}}%
\pgfpathlineto{\pgfqpoint{3.118150in}{2.724940in}}%
\pgfpathlineto{\pgfqpoint{3.123735in}{2.716393in}}%
\pgfpathlineto{\pgfqpoint{3.129180in}{2.707344in}}%
\pgfpathlineto{\pgfqpoint{3.134492in}{2.697781in}}%
\pgfpathlineto{\pgfqpoint{3.139677in}{2.687692in}}%
\pgfpathlineto{\pgfqpoint{3.144742in}{2.677065in}}%
\pgfpathlineto{\pgfqpoint{3.149692in}{2.665886in}}%
\pgfpathlineto{\pgfqpoint{3.154532in}{2.654144in}}%
\pgfpathlineto{\pgfqpoint{3.159267in}{2.641822in}}%
\pgfpathlineto{\pgfqpoint{3.163901in}{2.628899in}}%
\pgfpathlineto{\pgfqpoint{3.168438in}{2.615233in}}%
\pgfpathlineto{\pgfqpoint{3.172883in}{2.597610in}}%
\pgfpathlineto{\pgfqpoint{3.172883in}{2.601423in}}%
\pgfpathlineto{\pgfqpoint{3.172883in}{2.601423in}}%
\pgfpathlineto{\pgfqpoint{3.168438in}{2.616759in}}%
\pgfpathlineto{\pgfqpoint{3.163901in}{2.634065in}}%
\pgfpathlineto{\pgfqpoint{3.159267in}{2.650215in}}%
\pgfpathlineto{\pgfqpoint{3.154532in}{2.665153in}}%
\pgfpathlineto{\pgfqpoint{3.149692in}{2.678939in}}%
\pgfpathlineto{\pgfqpoint{3.144742in}{2.691639in}}%
\pgfpathlineto{\pgfqpoint{3.139677in}{2.703316in}}%
\pgfpathlineto{\pgfqpoint{3.134492in}{2.714032in}}%
\pgfpathlineto{\pgfqpoint{3.129180in}{2.723844in}}%
\pgfpathlineto{\pgfqpoint{3.123735in}{2.732810in}}%
\pgfpathlineto{\pgfqpoint{3.118150in}{2.740982in}}%
\pgfpathlineto{\pgfqpoint{3.112419in}{2.748412in}}%
\pgfpathlineto{\pgfqpoint{3.106532in}{2.755149in}}%
\pgfpathlineto{\pgfqpoint{3.100482in}{2.761238in}}%
\pgfpathlineto{\pgfqpoint{3.094259in}{2.766725in}}%
\pgfpathlineto{\pgfqpoint{3.087853in}{2.771652in}}%
\pgfpathlineto{\pgfqpoint{3.081253in}{2.776058in}}%
\pgfpathlineto{\pgfqpoint{3.074446in}{2.779984in}}%
\pgfpathlineto{\pgfqpoint{3.067420in}{2.783465in}}%
\pgfpathlineto{\pgfqpoint{3.060159in}{2.786539in}}%
\pgfpathlineto{\pgfqpoint{3.052648in}{2.789241in}}%
\pgfpathlineto{\pgfqpoint{3.044869in}{2.791611in}}%
\pgfpathlineto{\pgfqpoint{3.036801in}{2.793721in}}%
\pgfpathlineto{\pgfqpoint{3.028423in}{2.795905in}}%
\pgfpathlineto{\pgfqpoint{3.019710in}{2.798401in}}%
\pgfpathlineto{\pgfqpoint{3.010633in}{2.800692in}}%
\pgfpathlineto{\pgfqpoint{3.001162in}{2.802668in}}%
\pgfpathlineto{\pgfqpoint{2.991260in}{2.804318in}}%
\pgfpathlineto{\pgfqpoint{2.980886in}{2.805645in}}%
\pgfpathlineto{\pgfqpoint{2.969993in}{2.806652in}}%
\pgfpathlineto{\pgfqpoint{2.958526in}{2.807346in}}%
\pgfpathlineto{\pgfqpoint{2.946422in}{2.807730in}}%
\pgfpathlineto{\pgfqpoint{2.933605in}{2.807811in}}%
\pgfpathlineto{\pgfqpoint{2.919986in}{2.807592in}}%
\pgfpathlineto{\pgfqpoint{2.905459in}{2.807078in}}%
\pgfpathlineto{\pgfqpoint{2.889892in}{2.806272in}}%
\pgfpathlineto{\pgfqpoint{2.873127in}{2.805178in}}%
\pgfpathlineto{\pgfqpoint{2.854962in}{2.803809in}}%
\pgfpathlineto{\pgfqpoint{2.835143in}{2.803803in}}%
\pgfpathlineto{\pgfqpoint{2.813338in}{2.804150in}}%
\pgfpathlineto{\pgfqpoint{2.789104in}{2.804578in}}%
\pgfpathlineto{\pgfqpoint{2.761831in}{2.805104in}}%
\pgfpathlineto{\pgfqpoint{2.730647in}{2.805748in}}%
\pgfpathlineto{\pgfqpoint{2.694239in}{2.806527in}}%
\pgfpathlineto{\pgfqpoint{2.650497in}{2.807460in}}%
\pgfpathlineto{\pgfqpoint{2.595701in}{2.808567in}}%
\pgfpathlineto{\pgfqpoint{2.522305in}{2.809867in}}%
\pgfpathlineto{\pgfqpoint{2.410774in}{2.811380in}}%
\pgfpathlineto{\pgfqpoint{2.170064in}{2.813127in}}%
\pgfpathclose%
\pgfusepath{stroke,fill}%
\end{pgfscope}%
\begin{pgfscope}%
\pgfpathrectangle{\pgfqpoint{2.170064in}{2.624046in}}{\pgfqpoint{1.223103in}{0.607948in}}%
\pgfusepath{clip}%
\pgfsetroundcap%
\pgfsetroundjoin%
\pgfsetlinewidth{0.501875pt}%
\definecolor{currentstroke}{rgb}{0.000000,0.000000,1.000000}%
\pgfsetstrokecolor{currentstroke}%
\pgfsetstrokeopacity{0.800000}%
\pgfsetdash{}{0pt}%
\pgfpathmoveto{\pgfqpoint{2.170064in}{2.791009in}}%
\pgfpathlineto{\pgfqpoint{3.393168in}{2.791009in}}%
\pgfusepath{stroke}%
\end{pgfscope}%
\begin{pgfscope}%
\pgfpathrectangle{\pgfqpoint{2.170064in}{2.624046in}}{\pgfqpoint{1.223103in}{0.607948in}}%
\pgfusepath{clip}%
\pgfsetbuttcap%
\pgfsetroundjoin%
\pgfsetlinewidth{1.003750pt}%
\definecolor{currentstroke}{rgb}{0.000000,0.000000,0.000000}%
\pgfsetstrokecolor{currentstroke}%
\pgfsetdash{{3.700000pt}{1.600000pt}}{0.000000pt}%
\pgfpathmoveto{\pgfqpoint{2.170064in}{2.799107in}}%
\pgfpathlineto{\pgfqpoint{3.393168in}{2.799107in}}%
\pgfusepath{stroke}%
\end{pgfscope}%
\begin{pgfscope}%
\pgfsetroundcap%
\pgfsetroundjoin%
\pgfsetlinewidth{0.501875pt}%
\definecolor{currentstroke}{rgb}{0.000000,0.000000,1.000000}%
\pgfsetstrokecolor{currentstroke}%
\pgfsetstrokeopacity{0.800000}%
\pgfsetdash{}{0pt}%
\pgfpathmoveto{\pgfqpoint{2.970757in}{2.907514in}}%
\pgfpathquadraticcurveto{\pgfqpoint{2.907509in}{2.857841in}}{\pgfqpoint{2.844261in}{2.808168in}}%
\pgfusepath{stroke}%
\end{pgfscope}%
\begin{pgfscope}%
\pgfsetfillopacity{0.800000}%
\pgfsetstrokeopacity{0.800000}%
\definecolor{textcolor}{rgb}{0.000000,0.000000,1.000000}%
\pgfsetstrokecolor{textcolor}%
\pgfsetfillcolor{textcolor}%
\pgftext[x=2.910706in,y=2.973394in,left,base]{\color{textcolor}\sffamily\fontsize{5.647059}{6.776471}\selectfont 8.275(38)}%
\end{pgfscope}%
\begin{pgfscope}%
\pgfsetbuttcap%
\pgfsetroundjoin%
\definecolor{currentfill}{rgb}{0.150000,0.150000,0.150000}%
\pgfsetfillcolor{currentfill}%
\pgfsetlinewidth{1.003750pt}%
\definecolor{currentstroke}{rgb}{0.150000,0.150000,0.150000}%
\pgfsetstrokecolor{currentstroke}%
\pgfsetdash{}{0pt}%
\pgfsys@defobject{currentmarker}{\pgfqpoint{0.000000in}{-0.066667in}}{\pgfqpoint{0.000000in}{0.000000in}}{%
\pgfpathmoveto{\pgfqpoint{0.000000in}{0.000000in}}%
\pgfpathlineto{\pgfqpoint{0.000000in}{-0.066667in}}%
\pgfusepath{stroke,fill}%
}%
\begin{pgfscope}%
\pgfsys@transformshift{2.170064in}{2.624046in}%
\pgfsys@useobject{currentmarker}{}%
\end{pgfscope}%
\end{pgfscope}%
\begin{pgfscope}%
\pgfsetbuttcap%
\pgfsetroundjoin%
\definecolor{currentfill}{rgb}{0.150000,0.150000,0.150000}%
\pgfsetfillcolor{currentfill}%
\pgfsetlinewidth{1.003750pt}%
\definecolor{currentstroke}{rgb}{0.150000,0.150000,0.150000}%
\pgfsetstrokecolor{currentstroke}%
\pgfsetdash{}{0pt}%
\pgfsys@defobject{currentmarker}{\pgfqpoint{0.000000in}{-0.066667in}}{\pgfqpoint{0.000000in}{0.000000in}}{%
\pgfpathmoveto{\pgfqpoint{0.000000in}{0.000000in}}%
\pgfpathlineto{\pgfqpoint{0.000000in}{-0.066667in}}%
\pgfusepath{stroke,fill}%
}%
\begin{pgfscope}%
\pgfsys@transformshift{2.671473in}{2.624046in}%
\pgfsys@useobject{currentmarker}{}%
\end{pgfscope}%
\end{pgfscope}%
\begin{pgfscope}%
\pgfsetbuttcap%
\pgfsetroundjoin%
\definecolor{currentfill}{rgb}{0.150000,0.150000,0.150000}%
\pgfsetfillcolor{currentfill}%
\pgfsetlinewidth{1.003750pt}%
\definecolor{currentstroke}{rgb}{0.150000,0.150000,0.150000}%
\pgfsetstrokecolor{currentstroke}%
\pgfsetdash{}{0pt}%
\pgfsys@defobject{currentmarker}{\pgfqpoint{0.000000in}{-0.066667in}}{\pgfqpoint{0.000000in}{0.000000in}}{%
\pgfpathmoveto{\pgfqpoint{0.000000in}{0.000000in}}%
\pgfpathlineto{\pgfqpoint{0.000000in}{-0.066667in}}%
\pgfusepath{stroke,fill}%
}%
\begin{pgfscope}%
\pgfsys@transformshift{3.172883in}{2.624046in}%
\pgfsys@useobject{currentmarker}{}%
\end{pgfscope}%
\end{pgfscope}%
\begin{pgfscope}%
\pgfsetbuttcap%
\pgfsetroundjoin%
\definecolor{currentfill}{rgb}{0.150000,0.150000,0.150000}%
\pgfsetfillcolor{currentfill}%
\pgfsetlinewidth{0.803000pt}%
\definecolor{currentstroke}{rgb}{0.150000,0.150000,0.150000}%
\pgfsetstrokecolor{currentstroke}%
\pgfsetdash{}{0pt}%
\pgfsys@defobject{currentmarker}{\pgfqpoint{0.000000in}{-0.044444in}}{\pgfqpoint{0.000000in}{0.000000in}}{%
\pgfpathmoveto{\pgfqpoint{0.000000in}{0.000000in}}%
\pgfpathlineto{\pgfqpoint{0.000000in}{-0.044444in}}%
\pgfusepath{stroke,fill}%
}%
\begin{pgfscope}%
\pgfsys@transformshift{2.321004in}{2.624046in}%
\pgfsys@useobject{currentmarker}{}%
\end{pgfscope}%
\end{pgfscope}%
\begin{pgfscope}%
\pgfsetbuttcap%
\pgfsetroundjoin%
\definecolor{currentfill}{rgb}{0.150000,0.150000,0.150000}%
\pgfsetfillcolor{currentfill}%
\pgfsetlinewidth{0.803000pt}%
\definecolor{currentstroke}{rgb}{0.150000,0.150000,0.150000}%
\pgfsetstrokecolor{currentstroke}%
\pgfsetdash{}{0pt}%
\pgfsys@defobject{currentmarker}{\pgfqpoint{0.000000in}{-0.044444in}}{\pgfqpoint{0.000000in}{0.000000in}}{%
\pgfpathmoveto{\pgfqpoint{0.000000in}{0.000000in}}%
\pgfpathlineto{\pgfqpoint{0.000000in}{-0.044444in}}%
\pgfusepath{stroke,fill}%
}%
\begin{pgfscope}%
\pgfsys@transformshift{2.409297in}{2.624046in}%
\pgfsys@useobject{currentmarker}{}%
\end{pgfscope}%
\end{pgfscope}%
\begin{pgfscope}%
\pgfsetbuttcap%
\pgfsetroundjoin%
\definecolor{currentfill}{rgb}{0.150000,0.150000,0.150000}%
\pgfsetfillcolor{currentfill}%
\pgfsetlinewidth{0.803000pt}%
\definecolor{currentstroke}{rgb}{0.150000,0.150000,0.150000}%
\pgfsetstrokecolor{currentstroke}%
\pgfsetdash{}{0pt}%
\pgfsys@defobject{currentmarker}{\pgfqpoint{0.000000in}{-0.044444in}}{\pgfqpoint{0.000000in}{0.000000in}}{%
\pgfpathmoveto{\pgfqpoint{0.000000in}{0.000000in}}%
\pgfpathlineto{\pgfqpoint{0.000000in}{-0.044444in}}%
\pgfusepath{stroke,fill}%
}%
\begin{pgfscope}%
\pgfsys@transformshift{2.471943in}{2.624046in}%
\pgfsys@useobject{currentmarker}{}%
\end{pgfscope}%
\end{pgfscope}%
\begin{pgfscope}%
\pgfsetbuttcap%
\pgfsetroundjoin%
\definecolor{currentfill}{rgb}{0.150000,0.150000,0.150000}%
\pgfsetfillcolor{currentfill}%
\pgfsetlinewidth{0.803000pt}%
\definecolor{currentstroke}{rgb}{0.150000,0.150000,0.150000}%
\pgfsetstrokecolor{currentstroke}%
\pgfsetdash{}{0pt}%
\pgfsys@defobject{currentmarker}{\pgfqpoint{0.000000in}{-0.044444in}}{\pgfqpoint{0.000000in}{0.000000in}}{%
\pgfpathmoveto{\pgfqpoint{0.000000in}{0.000000in}}%
\pgfpathlineto{\pgfqpoint{0.000000in}{-0.044444in}}%
\pgfusepath{stroke,fill}%
}%
\begin{pgfscope}%
\pgfsys@transformshift{2.520534in}{2.624046in}%
\pgfsys@useobject{currentmarker}{}%
\end{pgfscope}%
\end{pgfscope}%
\begin{pgfscope}%
\pgfsetbuttcap%
\pgfsetroundjoin%
\definecolor{currentfill}{rgb}{0.150000,0.150000,0.150000}%
\pgfsetfillcolor{currentfill}%
\pgfsetlinewidth{0.803000pt}%
\definecolor{currentstroke}{rgb}{0.150000,0.150000,0.150000}%
\pgfsetstrokecolor{currentstroke}%
\pgfsetdash{}{0pt}%
\pgfsys@defobject{currentmarker}{\pgfqpoint{0.000000in}{-0.044444in}}{\pgfqpoint{0.000000in}{0.000000in}}{%
\pgfpathmoveto{\pgfqpoint{0.000000in}{0.000000in}}%
\pgfpathlineto{\pgfqpoint{0.000000in}{-0.044444in}}%
\pgfusepath{stroke,fill}%
}%
\begin{pgfscope}%
\pgfsys@transformshift{2.560237in}{2.624046in}%
\pgfsys@useobject{currentmarker}{}%
\end{pgfscope}%
\end{pgfscope}%
\begin{pgfscope}%
\pgfsetbuttcap%
\pgfsetroundjoin%
\definecolor{currentfill}{rgb}{0.150000,0.150000,0.150000}%
\pgfsetfillcolor{currentfill}%
\pgfsetlinewidth{0.803000pt}%
\definecolor{currentstroke}{rgb}{0.150000,0.150000,0.150000}%
\pgfsetstrokecolor{currentstroke}%
\pgfsetdash{}{0pt}%
\pgfsys@defobject{currentmarker}{\pgfqpoint{0.000000in}{-0.044444in}}{\pgfqpoint{0.000000in}{0.000000in}}{%
\pgfpathmoveto{\pgfqpoint{0.000000in}{0.000000in}}%
\pgfpathlineto{\pgfqpoint{0.000000in}{-0.044444in}}%
\pgfusepath{stroke,fill}%
}%
\begin{pgfscope}%
\pgfsys@transformshift{2.593804in}{2.624046in}%
\pgfsys@useobject{currentmarker}{}%
\end{pgfscope}%
\end{pgfscope}%
\begin{pgfscope}%
\pgfsetbuttcap%
\pgfsetroundjoin%
\definecolor{currentfill}{rgb}{0.150000,0.150000,0.150000}%
\pgfsetfillcolor{currentfill}%
\pgfsetlinewidth{0.803000pt}%
\definecolor{currentstroke}{rgb}{0.150000,0.150000,0.150000}%
\pgfsetstrokecolor{currentstroke}%
\pgfsetdash{}{0pt}%
\pgfsys@defobject{currentmarker}{\pgfqpoint{0.000000in}{-0.044444in}}{\pgfqpoint{0.000000in}{0.000000in}}{%
\pgfpathmoveto{\pgfqpoint{0.000000in}{0.000000in}}%
\pgfpathlineto{\pgfqpoint{0.000000in}{-0.044444in}}%
\pgfusepath{stroke,fill}%
}%
\begin{pgfscope}%
\pgfsys@transformshift{2.622882in}{2.624046in}%
\pgfsys@useobject{currentmarker}{}%
\end{pgfscope}%
\end{pgfscope}%
\begin{pgfscope}%
\pgfsetbuttcap%
\pgfsetroundjoin%
\definecolor{currentfill}{rgb}{0.150000,0.150000,0.150000}%
\pgfsetfillcolor{currentfill}%
\pgfsetlinewidth{0.803000pt}%
\definecolor{currentstroke}{rgb}{0.150000,0.150000,0.150000}%
\pgfsetstrokecolor{currentstroke}%
\pgfsetdash{}{0pt}%
\pgfsys@defobject{currentmarker}{\pgfqpoint{0.000000in}{-0.044444in}}{\pgfqpoint{0.000000in}{0.000000in}}{%
\pgfpathmoveto{\pgfqpoint{0.000000in}{0.000000in}}%
\pgfpathlineto{\pgfqpoint{0.000000in}{-0.044444in}}%
\pgfusepath{stroke,fill}%
}%
\begin{pgfscope}%
\pgfsys@transformshift{2.648530in}{2.624046in}%
\pgfsys@useobject{currentmarker}{}%
\end{pgfscope}%
\end{pgfscope}%
\begin{pgfscope}%
\pgfsetbuttcap%
\pgfsetroundjoin%
\definecolor{currentfill}{rgb}{0.150000,0.150000,0.150000}%
\pgfsetfillcolor{currentfill}%
\pgfsetlinewidth{0.803000pt}%
\definecolor{currentstroke}{rgb}{0.150000,0.150000,0.150000}%
\pgfsetstrokecolor{currentstroke}%
\pgfsetdash{}{0pt}%
\pgfsys@defobject{currentmarker}{\pgfqpoint{0.000000in}{-0.044444in}}{\pgfqpoint{0.000000in}{0.000000in}}{%
\pgfpathmoveto{\pgfqpoint{0.000000in}{0.000000in}}%
\pgfpathlineto{\pgfqpoint{0.000000in}{-0.044444in}}%
\pgfusepath{stroke,fill}%
}%
\begin{pgfscope}%
\pgfsys@transformshift{2.822413in}{2.624046in}%
\pgfsys@useobject{currentmarker}{}%
\end{pgfscope}%
\end{pgfscope}%
\begin{pgfscope}%
\pgfsetbuttcap%
\pgfsetroundjoin%
\definecolor{currentfill}{rgb}{0.150000,0.150000,0.150000}%
\pgfsetfillcolor{currentfill}%
\pgfsetlinewidth{0.803000pt}%
\definecolor{currentstroke}{rgb}{0.150000,0.150000,0.150000}%
\pgfsetstrokecolor{currentstroke}%
\pgfsetdash{}{0pt}%
\pgfsys@defobject{currentmarker}{\pgfqpoint{0.000000in}{-0.044444in}}{\pgfqpoint{0.000000in}{0.000000in}}{%
\pgfpathmoveto{\pgfqpoint{0.000000in}{0.000000in}}%
\pgfpathlineto{\pgfqpoint{0.000000in}{-0.044444in}}%
\pgfusepath{stroke,fill}%
}%
\begin{pgfscope}%
\pgfsys@transformshift{2.910706in}{2.624046in}%
\pgfsys@useobject{currentmarker}{}%
\end{pgfscope}%
\end{pgfscope}%
\begin{pgfscope}%
\pgfsetbuttcap%
\pgfsetroundjoin%
\definecolor{currentfill}{rgb}{0.150000,0.150000,0.150000}%
\pgfsetfillcolor{currentfill}%
\pgfsetlinewidth{0.803000pt}%
\definecolor{currentstroke}{rgb}{0.150000,0.150000,0.150000}%
\pgfsetstrokecolor{currentstroke}%
\pgfsetdash{}{0pt}%
\pgfsys@defobject{currentmarker}{\pgfqpoint{0.000000in}{-0.044444in}}{\pgfqpoint{0.000000in}{0.000000in}}{%
\pgfpathmoveto{\pgfqpoint{0.000000in}{0.000000in}}%
\pgfpathlineto{\pgfqpoint{0.000000in}{-0.044444in}}%
\pgfusepath{stroke,fill}%
}%
\begin{pgfscope}%
\pgfsys@transformshift{2.973352in}{2.624046in}%
\pgfsys@useobject{currentmarker}{}%
\end{pgfscope}%
\end{pgfscope}%
\begin{pgfscope}%
\pgfsetbuttcap%
\pgfsetroundjoin%
\definecolor{currentfill}{rgb}{0.150000,0.150000,0.150000}%
\pgfsetfillcolor{currentfill}%
\pgfsetlinewidth{0.803000pt}%
\definecolor{currentstroke}{rgb}{0.150000,0.150000,0.150000}%
\pgfsetstrokecolor{currentstroke}%
\pgfsetdash{}{0pt}%
\pgfsys@defobject{currentmarker}{\pgfqpoint{0.000000in}{-0.044444in}}{\pgfqpoint{0.000000in}{0.000000in}}{%
\pgfpathmoveto{\pgfqpoint{0.000000in}{0.000000in}}%
\pgfpathlineto{\pgfqpoint{0.000000in}{-0.044444in}}%
\pgfusepath{stroke,fill}%
}%
\begin{pgfscope}%
\pgfsys@transformshift{3.021943in}{2.624046in}%
\pgfsys@useobject{currentmarker}{}%
\end{pgfscope}%
\end{pgfscope}%
\begin{pgfscope}%
\pgfsetbuttcap%
\pgfsetroundjoin%
\definecolor{currentfill}{rgb}{0.150000,0.150000,0.150000}%
\pgfsetfillcolor{currentfill}%
\pgfsetlinewidth{0.803000pt}%
\definecolor{currentstroke}{rgb}{0.150000,0.150000,0.150000}%
\pgfsetstrokecolor{currentstroke}%
\pgfsetdash{}{0pt}%
\pgfsys@defobject{currentmarker}{\pgfqpoint{0.000000in}{-0.044444in}}{\pgfqpoint{0.000000in}{0.000000in}}{%
\pgfpathmoveto{\pgfqpoint{0.000000in}{0.000000in}}%
\pgfpathlineto{\pgfqpoint{0.000000in}{-0.044444in}}%
\pgfusepath{stroke,fill}%
}%
\begin{pgfscope}%
\pgfsys@transformshift{3.061646in}{2.624046in}%
\pgfsys@useobject{currentmarker}{}%
\end{pgfscope}%
\end{pgfscope}%
\begin{pgfscope}%
\pgfsetbuttcap%
\pgfsetroundjoin%
\definecolor{currentfill}{rgb}{0.150000,0.150000,0.150000}%
\pgfsetfillcolor{currentfill}%
\pgfsetlinewidth{0.803000pt}%
\definecolor{currentstroke}{rgb}{0.150000,0.150000,0.150000}%
\pgfsetstrokecolor{currentstroke}%
\pgfsetdash{}{0pt}%
\pgfsys@defobject{currentmarker}{\pgfqpoint{0.000000in}{-0.044444in}}{\pgfqpoint{0.000000in}{0.000000in}}{%
\pgfpathmoveto{\pgfqpoint{0.000000in}{0.000000in}}%
\pgfpathlineto{\pgfqpoint{0.000000in}{-0.044444in}}%
\pgfusepath{stroke,fill}%
}%
\begin{pgfscope}%
\pgfsys@transformshift{3.095213in}{2.624046in}%
\pgfsys@useobject{currentmarker}{}%
\end{pgfscope}%
\end{pgfscope}%
\begin{pgfscope}%
\pgfsetbuttcap%
\pgfsetroundjoin%
\definecolor{currentfill}{rgb}{0.150000,0.150000,0.150000}%
\pgfsetfillcolor{currentfill}%
\pgfsetlinewidth{0.803000pt}%
\definecolor{currentstroke}{rgb}{0.150000,0.150000,0.150000}%
\pgfsetstrokecolor{currentstroke}%
\pgfsetdash{}{0pt}%
\pgfsys@defobject{currentmarker}{\pgfqpoint{0.000000in}{-0.044444in}}{\pgfqpoint{0.000000in}{0.000000in}}{%
\pgfpathmoveto{\pgfqpoint{0.000000in}{0.000000in}}%
\pgfpathlineto{\pgfqpoint{0.000000in}{-0.044444in}}%
\pgfusepath{stroke,fill}%
}%
\begin{pgfscope}%
\pgfsys@transformshift{3.124291in}{2.624046in}%
\pgfsys@useobject{currentmarker}{}%
\end{pgfscope}%
\end{pgfscope}%
\begin{pgfscope}%
\pgfsetbuttcap%
\pgfsetroundjoin%
\definecolor{currentfill}{rgb}{0.150000,0.150000,0.150000}%
\pgfsetfillcolor{currentfill}%
\pgfsetlinewidth{0.803000pt}%
\definecolor{currentstroke}{rgb}{0.150000,0.150000,0.150000}%
\pgfsetstrokecolor{currentstroke}%
\pgfsetdash{}{0pt}%
\pgfsys@defobject{currentmarker}{\pgfqpoint{0.000000in}{-0.044444in}}{\pgfqpoint{0.000000in}{0.000000in}}{%
\pgfpathmoveto{\pgfqpoint{0.000000in}{0.000000in}}%
\pgfpathlineto{\pgfqpoint{0.000000in}{-0.044444in}}%
\pgfusepath{stroke,fill}%
}%
\begin{pgfscope}%
\pgfsys@transformshift{3.149939in}{2.624046in}%
\pgfsys@useobject{currentmarker}{}%
\end{pgfscope}%
\end{pgfscope}%
\begin{pgfscope}%
\pgfsetbuttcap%
\pgfsetroundjoin%
\definecolor{currentfill}{rgb}{0.150000,0.150000,0.150000}%
\pgfsetfillcolor{currentfill}%
\pgfsetlinewidth{0.803000pt}%
\definecolor{currentstroke}{rgb}{0.150000,0.150000,0.150000}%
\pgfsetstrokecolor{currentstroke}%
\pgfsetdash{}{0pt}%
\pgfsys@defobject{currentmarker}{\pgfqpoint{0.000000in}{-0.044444in}}{\pgfqpoint{0.000000in}{0.000000in}}{%
\pgfpathmoveto{\pgfqpoint{0.000000in}{0.000000in}}%
\pgfpathlineto{\pgfqpoint{0.000000in}{-0.044444in}}%
\pgfusepath{stroke,fill}%
}%
\begin{pgfscope}%
\pgfsys@transformshift{3.323822in}{2.624046in}%
\pgfsys@useobject{currentmarker}{}%
\end{pgfscope}%
\end{pgfscope}%
\begin{pgfscope}%
\pgfsetbuttcap%
\pgfsetroundjoin%
\definecolor{currentfill}{rgb}{0.150000,0.150000,0.150000}%
\pgfsetfillcolor{currentfill}%
\pgfsetlinewidth{1.003750pt}%
\definecolor{currentstroke}{rgb}{0.150000,0.150000,0.150000}%
\pgfsetstrokecolor{currentstroke}%
\pgfsetdash{}{0pt}%
\pgfsys@defobject{currentmarker}{\pgfqpoint{-0.066667in}{0.000000in}}{\pgfqpoint{0.000000in}{0.000000in}}{%
\pgfpathmoveto{\pgfqpoint{0.000000in}{0.000000in}}%
\pgfpathlineto{\pgfqpoint{-0.066667in}{0.000000in}}%
\pgfusepath{stroke,fill}%
}%
\begin{pgfscope}%
\pgfsys@transformshift{2.170064in}{2.624046in}%
\pgfsys@useobject{currentmarker}{}%
\end{pgfscope}%
\end{pgfscope}%
\begin{pgfscope}%
\pgfsetbuttcap%
\pgfsetroundjoin%
\definecolor{currentfill}{rgb}{0.150000,0.150000,0.150000}%
\pgfsetfillcolor{currentfill}%
\pgfsetlinewidth{1.003750pt}%
\definecolor{currentstroke}{rgb}{0.150000,0.150000,0.150000}%
\pgfsetstrokecolor{currentstroke}%
\pgfsetdash{}{0pt}%
\pgfsys@defobject{currentmarker}{\pgfqpoint{-0.066667in}{0.000000in}}{\pgfqpoint{0.000000in}{0.000000in}}{%
\pgfpathmoveto{\pgfqpoint{0.000000in}{0.000000in}}%
\pgfpathlineto{\pgfqpoint{-0.066667in}{0.000000in}}%
\pgfusepath{stroke,fill}%
}%
\begin{pgfscope}%
\pgfsys@transformshift{2.170064in}{2.799107in}%
\pgfsys@useobject{currentmarker}{}%
\end{pgfscope}%
\end{pgfscope}%
\begin{pgfscope}%
\pgfsetbuttcap%
\pgfsetroundjoin%
\definecolor{currentfill}{rgb}{0.150000,0.150000,0.150000}%
\pgfsetfillcolor{currentfill}%
\pgfsetlinewidth{1.003750pt}%
\definecolor{currentstroke}{rgb}{0.150000,0.150000,0.150000}%
\pgfsetstrokecolor{currentstroke}%
\pgfsetdash{}{0pt}%
\pgfsys@defobject{currentmarker}{\pgfqpoint{-0.066667in}{0.000000in}}{\pgfqpoint{0.000000in}{0.000000in}}{%
\pgfpathmoveto{\pgfqpoint{0.000000in}{0.000000in}}%
\pgfpathlineto{\pgfqpoint{-0.066667in}{0.000000in}}%
\pgfusepath{stroke,fill}%
}%
\begin{pgfscope}%
\pgfsys@transformshift{2.170064in}{3.231994in}%
\pgfsys@useobject{currentmarker}{}%
\end{pgfscope}%
\end{pgfscope}%
\begin{pgfscope}%
\pgfpathrectangle{\pgfqpoint{2.170064in}{2.624046in}}{\pgfqpoint{1.223103in}{0.607948in}}%
\pgfusepath{clip}%
\pgfsetroundcap%
\pgfsetroundjoin%
\pgfsetlinewidth{1.204500pt}%
\definecolor{currentstroke}{rgb}{0.000000,0.501961,0.000000}%
\pgfsetstrokecolor{currentstroke}%
\pgfsetdash{}{0pt}%
\pgfpathmoveto{\pgfqpoint{2.170064in}{2.791681in}}%
\pgfpathlineto{\pgfqpoint{2.410774in}{2.793026in}}%
\pgfpathlineto{\pgfqpoint{2.522305in}{2.794349in}}%
\pgfpathlineto{\pgfqpoint{2.595701in}{2.795639in}}%
\pgfpathlineto{\pgfqpoint{2.650497in}{2.796887in}}%
\pgfpathlineto{\pgfqpoint{2.694239in}{2.798084in}}%
\pgfpathlineto{\pgfqpoint{2.730647in}{2.799220in}}%
\pgfpathlineto{\pgfqpoint{2.761831in}{2.800286in}}%
\pgfpathlineto{\pgfqpoint{2.789104in}{2.801273in}}%
\pgfpathlineto{\pgfqpoint{2.813338in}{2.802170in}}%
\pgfpathlineto{\pgfqpoint{2.835143in}{2.802967in}}%
\pgfpathlineto{\pgfqpoint{2.854962in}{2.803656in}}%
\pgfpathlineto{\pgfqpoint{2.873127in}{2.804225in}}%
\pgfpathlineto{\pgfqpoint{2.889892in}{2.804664in}}%
\pgfpathlineto{\pgfqpoint{2.905459in}{2.804963in}}%
\pgfpathlineto{\pgfqpoint{2.919986in}{2.805109in}}%
\pgfpathlineto{\pgfqpoint{2.933605in}{2.805092in}}%
\pgfpathlineto{\pgfqpoint{2.946422in}{2.804901in}}%
\pgfpathlineto{\pgfqpoint{2.958526in}{2.804521in}}%
\pgfpathlineto{\pgfqpoint{2.969993in}{2.803941in}}%
\pgfpathlineto{\pgfqpoint{2.980886in}{2.803147in}}%
\pgfpathlineto{\pgfqpoint{2.991260in}{2.802126in}}%
\pgfpathlineto{\pgfqpoint{3.001162in}{2.800862in}}%
\pgfpathlineto{\pgfqpoint{3.010633in}{2.799340in}}%
\pgfpathlineto{\pgfqpoint{3.019710in}{2.797546in}}%
\pgfpathlineto{\pgfqpoint{3.028423in}{2.795461in}}%
\pgfpathlineto{\pgfqpoint{3.036801in}{2.793070in}}%
\pgfpathlineto{\pgfqpoint{3.044869in}{2.790354in}}%
\pgfpathlineto{\pgfqpoint{3.052648in}{2.787293in}}%
\pgfpathlineto{\pgfqpoint{3.060159in}{2.783869in}}%
\pgfpathlineto{\pgfqpoint{3.067420in}{2.780061in}}%
\pgfpathlineto{\pgfqpoint{3.074446in}{2.775846in}}%
\pgfpathlineto{\pgfqpoint{3.081253in}{2.771204in}}%
\pgfpathlineto{\pgfqpoint{3.087853in}{2.766109in}}%
\pgfpathlineto{\pgfqpoint{3.094259in}{2.760538in}}%
\pgfpathlineto{\pgfqpoint{3.100482in}{2.754465in}}%
\pgfpathlineto{\pgfqpoint{3.106532in}{2.747864in}}%
\pgfpathlineto{\pgfqpoint{3.112419in}{2.740705in}}%
\pgfpathlineto{\pgfqpoint{3.118150in}{2.732961in}}%
\pgfpathlineto{\pgfqpoint{3.123735in}{2.724601in}}%
\pgfpathlineto{\pgfqpoint{3.129180in}{2.715594in}}%
\pgfpathlineto{\pgfqpoint{3.134492in}{2.705906in}}%
\pgfpathlineto{\pgfqpoint{3.139677in}{2.695504in}}%
\pgfpathlineto{\pgfqpoint{3.144742in}{2.684352in}}%
\pgfpathlineto{\pgfqpoint{3.149692in}{2.672413in}}%
\pgfpathlineto{\pgfqpoint{3.154532in}{2.659648in}}%
\pgfpathlineto{\pgfqpoint{3.159267in}{2.646018in}}%
\pgfpathlineto{\pgfqpoint{3.163901in}{2.631482in}}%
\pgfpathlineto{\pgfqpoint{3.167056in}{2.620712in}}%
\pgfusepath{stroke}%
\end{pgfscope}%
\begin{pgfscope}%
\pgfsetrectcap%
\pgfsetmiterjoin%
\pgfsetlinewidth{1.003750pt}%
\definecolor{currentstroke}{rgb}{0.150000,0.150000,0.150000}%
\pgfsetstrokecolor{currentstroke}%
\pgfsetdash{}{0pt}%
\pgfpathmoveto{\pgfqpoint{2.170064in}{2.624046in}}%
\pgfpathlineto{\pgfqpoint{2.170064in}{3.231994in}}%
\pgfusepath{stroke}%
\end{pgfscope}%
\begin{pgfscope}%
\pgfsetrectcap%
\pgfsetmiterjoin%
\pgfsetlinewidth{1.003750pt}%
\definecolor{currentstroke}{rgb}{0.150000,0.150000,0.150000}%
\pgfsetstrokecolor{currentstroke}%
\pgfsetdash{}{0pt}%
\pgfpathmoveto{\pgfqpoint{2.170064in}{2.624046in}}%
\pgfpathlineto{\pgfqpoint{3.393168in}{2.624046in}}%
\pgfusepath{stroke}%
\end{pgfscope}%
\begin{pgfscope}%
\pgfpathrectangle{\pgfqpoint{2.170064in}{2.624046in}}{\pgfqpoint{1.223103in}{0.607948in}}%
\pgfusepath{clip}%
\pgfsetbuttcap%
\pgfsetroundjoin%
\definecolor{currentfill}{rgb}{0.000000,0.000000,0.000000}%
\pgfsetfillcolor{currentfill}%
\pgfsetlinewidth{1.003750pt}%
\definecolor{currentstroke}{rgb}{0.000000,0.000000,0.000000}%
\pgfsetstrokecolor{currentstroke}%
\pgfsetdash{}{0pt}%
\pgfsys@defobject{currentmarker}{\pgfqpoint{-0.013889in}{-0.013889in}}{\pgfqpoint{0.013889in}{0.013889in}}{%
\pgfpathmoveto{\pgfqpoint{0.000000in}{-0.013889in}}%
\pgfpathcurveto{\pgfqpoint{0.003683in}{-0.013889in}}{\pgfqpoint{0.007216in}{-0.012425in}}{\pgfqpoint{0.009821in}{-0.009821in}}%
\pgfpathcurveto{\pgfqpoint{0.012425in}{-0.007216in}}{\pgfqpoint{0.013889in}{-0.003683in}}{\pgfqpoint{0.013889in}{0.000000in}}%
\pgfpathcurveto{\pgfqpoint{0.013889in}{0.003683in}}{\pgfqpoint{0.012425in}{0.007216in}}{\pgfqpoint{0.009821in}{0.009821in}}%
\pgfpathcurveto{\pgfqpoint{0.007216in}{0.012425in}}{\pgfqpoint{0.003683in}{0.013889in}}{\pgfqpoint{0.000000in}{0.013889in}}%
\pgfpathcurveto{\pgfqpoint{-0.003683in}{0.013889in}}{\pgfqpoint{-0.007216in}{0.012425in}}{\pgfqpoint{-0.009821in}{0.009821in}}%
\pgfpathcurveto{\pgfqpoint{-0.012425in}{0.007216in}}{\pgfqpoint{-0.013889in}{0.003683in}}{\pgfqpoint{-0.013889in}{0.000000in}}%
\pgfpathcurveto{\pgfqpoint{-0.013889in}{-0.003683in}}{\pgfqpoint{-0.012425in}{-0.007216in}}{\pgfqpoint{-0.009821in}{-0.009821in}}%
\pgfpathcurveto{\pgfqpoint{-0.007216in}{-0.012425in}}{\pgfqpoint{-0.003683in}{-0.013889in}}{\pgfqpoint{0.000000in}{-0.013889in}}%
\pgfpathclose%
\pgfusepath{stroke,fill}%
}%
\begin{pgfscope}%
\pgfsys@transformshift{3.172883in}{2.598383in}%
\pgfsys@useobject{currentmarker}{}%
\end{pgfscope}%
\begin{pgfscope}%
\pgfsys@transformshift{3.084589in}{2.771930in}%
\pgfsys@useobject{currentmarker}{}%
\end{pgfscope}%
\begin{pgfscope}%
\pgfsys@transformshift{3.021943in}{2.796322in}%
\pgfsys@useobject{currentmarker}{}%
\end{pgfscope}%
\begin{pgfscope}%
\pgfsys@transformshift{2.973352in}{2.802139in}%
\pgfsys@useobject{currentmarker}{}%
\end{pgfscope}%
\begin{pgfscope}%
\pgfsys@transformshift{2.822413in}{2.803234in}%
\pgfsys@useobject{currentmarker}{}%
\end{pgfscope}%
\begin{pgfscope}%
\pgfsys@transformshift{2.826812in}{2.803292in}%
\pgfsys@useobject{currentmarker}{}%
\end{pgfscope}%
\begin{pgfscope}%
\pgfsys@transformshift{2.831302in}{2.803350in}%
\pgfsys@useobject{currentmarker}{}%
\end{pgfscope}%
\begin{pgfscope}%
\pgfsys@transformshift{2.835887in}{2.803408in}%
\pgfsys@useobject{currentmarker}{}%
\end{pgfscope}%
\begin{pgfscope}%
\pgfsys@transformshift{2.840570in}{2.803466in}%
\pgfsys@useobject{currentmarker}{}%
\end{pgfscope}%
\begin{pgfscope}%
\pgfsys@transformshift{2.845356in}{2.803524in}%
\pgfsys@useobject{currentmarker}{}%
\end{pgfscope}%
\begin{pgfscope}%
\pgfsys@transformshift{2.850250in}{2.803581in}%
\pgfsys@useobject{currentmarker}{}%
\end{pgfscope}%
\begin{pgfscope}%
\pgfsys@transformshift{2.855256in}{2.803636in}%
\pgfsys@useobject{currentmarker}{}%
\end{pgfscope}%
\begin{pgfscope}%
\pgfsys@transformshift{2.933650in}{2.803659in}%
\pgfsys@useobject{currentmarker}{}%
\end{pgfscope}%
\begin{pgfscope}%
\pgfsys@transformshift{2.860380in}{2.803689in}%
\pgfsys@useobject{currentmarker}{}%
\end{pgfscope}%
\begin{pgfscope}%
\pgfsys@transformshift{2.865627in}{2.803740in}%
\pgfsys@useobject{currentmarker}{}%
\end{pgfscope}%
\begin{pgfscope}%
\pgfsys@transformshift{2.871004in}{2.803787in}%
\pgfsys@useobject{currentmarker}{}%
\end{pgfscope}%
\begin{pgfscope}%
\pgfsys@transformshift{2.900082in}{2.803925in}%
\pgfsys@useobject{currentmarker}{}%
\end{pgfscope}%
\end{pgfscope}%
\begin{pgfscope}%
\pgfsetbuttcap%
\pgfsetmiterjoin%
\definecolor{currentfill}{rgb}{1.000000,1.000000,1.000000}%
\pgfsetfillcolor{currentfill}%
\pgfsetlinewidth{0.000000pt}%
\definecolor{currentstroke}{rgb}{0.000000,0.000000,0.000000}%
\pgfsetstrokecolor{currentstroke}%
\pgfsetstrokeopacity{0.000000}%
\pgfsetdash{}{0pt}%
\pgfpathmoveto{\pgfqpoint{3.637789in}{2.624046in}}%
\pgfpathlineto{\pgfqpoint{4.860892in}{2.624046in}}%
\pgfpathlineto{\pgfqpoint{4.860892in}{3.231994in}}%
\pgfpathlineto{\pgfqpoint{3.637789in}{3.231994in}}%
\pgfpathclose%
\pgfusepath{fill}%
\end{pgfscope}%
\begin{pgfscope}%
\pgfpathrectangle{\pgfqpoint{3.637789in}{2.624046in}}{\pgfqpoint{1.223103in}{0.607948in}}%
\pgfusepath{clip}%
\pgfsetbuttcap%
\pgfsetmiterjoin%
\definecolor{currentfill}{rgb}{0.000000,0.000000,1.000000}%
\pgfsetfillcolor{currentfill}%
\pgfsetfillopacity{0.100000}%
\pgfsetlinewidth{0.803000pt}%
\definecolor{currentstroke}{rgb}{0.000000,0.000000,1.000000}%
\pgfsetstrokecolor{currentstroke}%
\pgfsetstrokeopacity{0.100000}%
\pgfsetdash{}{0pt}%
\pgfpathmoveto{\pgfqpoint{3.637789in}{2.792475in}}%
\pgfpathlineto{\pgfqpoint{3.637789in}{2.803098in}}%
\pgfpathlineto{\pgfqpoint{4.860892in}{2.803098in}}%
\pgfpathlineto{\pgfqpoint{4.860892in}{2.792475in}}%
\pgfpathclose%
\pgfusepath{stroke,fill}%
\end{pgfscope}%
\begin{pgfscope}%
\pgfpathrectangle{\pgfqpoint{3.637789in}{2.624046in}}{\pgfqpoint{1.223103in}{0.607948in}}%
\pgfusepath{clip}%
\pgfsetbuttcap%
\pgfsetroundjoin%
\definecolor{currentfill}{rgb}{0.000000,0.501961,0.000000}%
\pgfsetfillcolor{currentfill}%
\pgfsetfillopacity{0.500000}%
\pgfsetlinewidth{0.803000pt}%
\definecolor{currentstroke}{rgb}{0.000000,0.501961,0.000000}%
\pgfsetstrokecolor{currentstroke}%
\pgfsetstrokeopacity{0.500000}%
\pgfsetdash{}{0pt}%
\pgfpathmoveto{\pgfqpoint{3.637789in}{2.803163in}}%
\pgfpathlineto{\pgfqpoint{3.637789in}{2.793302in}}%
\pgfpathlineto{\pgfqpoint{3.878498in}{2.794923in}}%
\pgfpathlineto{\pgfqpoint{3.990029in}{2.796482in}}%
\pgfpathlineto{\pgfqpoint{4.063425in}{2.797981in}}%
\pgfpathlineto{\pgfqpoint{4.118221in}{2.799424in}}%
\pgfpathlineto{\pgfqpoint{4.161963in}{2.800815in}}%
\pgfpathlineto{\pgfqpoint{4.198371in}{2.802157in}}%
\pgfpathlineto{\pgfqpoint{4.229555in}{2.803451in}}%
\pgfpathlineto{\pgfqpoint{4.256828in}{2.804700in}}%
\pgfpathlineto{\pgfqpoint{4.281062in}{2.805906in}}%
\pgfpathlineto{\pgfqpoint{4.302867in}{2.807070in}}%
\pgfpathlineto{\pgfqpoint{4.322686in}{2.808160in}}%
\pgfpathlineto{\pgfqpoint{4.340851in}{2.808874in}}%
\pgfpathlineto{\pgfqpoint{4.357617in}{2.809606in}}%
\pgfpathlineto{\pgfqpoint{4.373183in}{2.810364in}}%
\pgfpathlineto{\pgfqpoint{4.387711in}{2.811144in}}%
\pgfpathlineto{\pgfqpoint{4.401329in}{2.811944in}}%
\pgfpathlineto{\pgfqpoint{4.414146in}{2.812762in}}%
\pgfpathlineto{\pgfqpoint{4.426250in}{2.813594in}}%
\pgfpathlineto{\pgfqpoint{4.437717in}{2.814437in}}%
\pgfpathlineto{\pgfqpoint{4.448610in}{2.815288in}}%
\pgfpathlineto{\pgfqpoint{4.458984in}{2.816142in}}%
\pgfpathlineto{\pgfqpoint{4.468886in}{2.816995in}}%
\pgfpathlineto{\pgfqpoint{4.478357in}{2.817842in}}%
\pgfpathlineto{\pgfqpoint{4.487434in}{2.818674in}}%
\pgfpathlineto{\pgfqpoint{4.496147in}{2.819474in}}%
\pgfpathlineto{\pgfqpoint{4.504525in}{2.820190in}}%
\pgfpathlineto{\pgfqpoint{4.512593in}{2.820753in}}%
\pgfpathlineto{\pgfqpoint{4.520372in}{2.821200in}}%
\pgfpathlineto{\pgfqpoint{4.527883in}{2.821573in}}%
\pgfpathlineto{\pgfqpoint{4.535144in}{2.821882in}}%
\pgfpathlineto{\pgfqpoint{4.542170in}{2.822128in}}%
\pgfpathlineto{\pgfqpoint{4.548977in}{2.822310in}}%
\pgfpathlineto{\pgfqpoint{4.555577in}{2.822426in}}%
\pgfpathlineto{\pgfqpoint{4.561983in}{2.822473in}}%
\pgfpathlineto{\pgfqpoint{4.568206in}{2.822449in}}%
\pgfpathlineto{\pgfqpoint{4.574256in}{2.822351in}}%
\pgfpathlineto{\pgfqpoint{4.580143in}{2.822177in}}%
\pgfpathlineto{\pgfqpoint{4.585874in}{2.821922in}}%
\pgfpathlineto{\pgfqpoint{4.591459in}{2.821585in}}%
\pgfpathlineto{\pgfqpoint{4.596904in}{2.821163in}}%
\pgfpathlineto{\pgfqpoint{4.602216in}{2.820651in}}%
\pgfpathlineto{\pgfqpoint{4.607401in}{2.820049in}}%
\pgfpathlineto{\pgfqpoint{4.612466in}{2.819351in}}%
\pgfpathlineto{\pgfqpoint{4.617416in}{2.818556in}}%
\pgfpathlineto{\pgfqpoint{4.622256in}{2.817661in}}%
\pgfpathlineto{\pgfqpoint{4.626991in}{2.816663in}}%
\pgfpathlineto{\pgfqpoint{4.631625in}{2.815558in}}%
\pgfpathlineto{\pgfqpoint{4.636162in}{2.814340in}}%
\pgfpathlineto{\pgfqpoint{4.640607in}{2.812588in}}%
\pgfpathlineto{\pgfqpoint{4.640607in}{2.813048in}}%
\pgfpathlineto{\pgfqpoint{4.640607in}{2.813048in}}%
\pgfpathlineto{\pgfqpoint{4.636162in}{2.815224in}}%
\pgfpathlineto{\pgfqpoint{4.631625in}{2.817479in}}%
\pgfpathlineto{\pgfqpoint{4.626991in}{2.819418in}}%
\pgfpathlineto{\pgfqpoint{4.622256in}{2.821063in}}%
\pgfpathlineto{\pgfqpoint{4.617416in}{2.822437in}}%
\pgfpathlineto{\pgfqpoint{4.612466in}{2.823560in}}%
\pgfpathlineto{\pgfqpoint{4.607401in}{2.824454in}}%
\pgfpathlineto{\pgfqpoint{4.602216in}{2.825138in}}%
\pgfpathlineto{\pgfqpoint{4.596904in}{2.825631in}}%
\pgfpathlineto{\pgfqpoint{4.591459in}{2.825949in}}%
\pgfpathlineto{\pgfqpoint{4.585874in}{2.826110in}}%
\pgfpathlineto{\pgfqpoint{4.580143in}{2.826128in}}%
\pgfpathlineto{\pgfqpoint{4.574256in}{2.826019in}}%
\pgfpathlineto{\pgfqpoint{4.568206in}{2.825794in}}%
\pgfpathlineto{\pgfqpoint{4.561983in}{2.825467in}}%
\pgfpathlineto{\pgfqpoint{4.555577in}{2.825050in}}%
\pgfpathlineto{\pgfqpoint{4.548977in}{2.824553in}}%
\pgfpathlineto{\pgfqpoint{4.542170in}{2.823988in}}%
\pgfpathlineto{\pgfqpoint{4.535144in}{2.823363in}}%
\pgfpathlineto{\pgfqpoint{4.527883in}{2.822689in}}%
\pgfpathlineto{\pgfqpoint{4.520372in}{2.821979in}}%
\pgfpathlineto{\pgfqpoint{4.512593in}{2.821251in}}%
\pgfpathlineto{\pgfqpoint{4.504525in}{2.820556in}}%
\pgfpathlineto{\pgfqpoint{4.496147in}{2.819940in}}%
\pgfpathlineto{\pgfqpoint{4.487434in}{2.819341in}}%
\pgfpathlineto{\pgfqpoint{4.478357in}{2.818716in}}%
\pgfpathlineto{\pgfqpoint{4.468886in}{2.818053in}}%
\pgfpathlineto{\pgfqpoint{4.458984in}{2.817350in}}%
\pgfpathlineto{\pgfqpoint{4.448610in}{2.816606in}}%
\pgfpathlineto{\pgfqpoint{4.437717in}{2.815824in}}%
\pgfpathlineto{\pgfqpoint{4.426250in}{2.815002in}}%
\pgfpathlineto{\pgfqpoint{4.414146in}{2.814143in}}%
\pgfpathlineto{\pgfqpoint{4.401329in}{2.813245in}}%
\pgfpathlineto{\pgfqpoint{4.387711in}{2.812310in}}%
\pgfpathlineto{\pgfqpoint{4.373183in}{2.811338in}}%
\pgfpathlineto{\pgfqpoint{4.357617in}{2.810329in}}%
\pgfpathlineto{\pgfqpoint{4.340851in}{2.809282in}}%
\pgfpathlineto{\pgfqpoint{4.322686in}{2.808208in}}%
\pgfpathlineto{\pgfqpoint{4.302867in}{2.807501in}}%
\pgfpathlineto{\pgfqpoint{4.281062in}{2.806862in}}%
\pgfpathlineto{\pgfqpoint{4.256828in}{2.806260in}}%
\pgfpathlineto{\pgfqpoint{4.229555in}{2.805698in}}%
\pgfpathlineto{\pgfqpoint{4.198371in}{2.805178in}}%
\pgfpathlineto{\pgfqpoint{4.161963in}{2.804705in}}%
\pgfpathlineto{\pgfqpoint{4.118221in}{2.804281in}}%
\pgfpathlineto{\pgfqpoint{4.063425in}{2.803911in}}%
\pgfpathlineto{\pgfqpoint{3.990029in}{2.803598in}}%
\pgfpathlineto{\pgfqpoint{3.878498in}{2.803347in}}%
\pgfpathlineto{\pgfqpoint{3.637789in}{2.803163in}}%
\pgfpathclose%
\pgfusepath{stroke,fill}%
\end{pgfscope}%
\begin{pgfscope}%
\pgfpathrectangle{\pgfqpoint{3.637789in}{2.624046in}}{\pgfqpoint{1.223103in}{0.607948in}}%
\pgfusepath{clip}%
\pgfsetroundcap%
\pgfsetroundjoin%
\pgfsetlinewidth{0.501875pt}%
\definecolor{currentstroke}{rgb}{0.000000,0.000000,1.000000}%
\pgfsetstrokecolor{currentstroke}%
\pgfsetstrokeopacity{0.800000}%
\pgfsetdash{}{0pt}%
\pgfpathmoveto{\pgfqpoint{3.637789in}{2.797787in}}%
\pgfpathlineto{\pgfqpoint{4.860892in}{2.797787in}}%
\pgfusepath{stroke}%
\end{pgfscope}%
\begin{pgfscope}%
\pgfpathrectangle{\pgfqpoint{3.637789in}{2.624046in}}{\pgfqpoint{1.223103in}{0.607948in}}%
\pgfusepath{clip}%
\pgfsetbuttcap%
\pgfsetroundjoin%
\pgfsetlinewidth{1.003750pt}%
\definecolor{currentstroke}{rgb}{0.000000,0.000000,0.000000}%
\pgfsetstrokecolor{currentstroke}%
\pgfsetdash{{3.700000pt}{1.600000pt}}{0.000000pt}%
\pgfpathmoveto{\pgfqpoint{3.637789in}{2.799107in}}%
\pgfpathlineto{\pgfqpoint{4.860892in}{2.799107in}}%
\pgfusepath{stroke}%
\end{pgfscope}%
\begin{pgfscope}%
\pgfsetroundcap%
\pgfsetroundjoin%
\pgfsetlinewidth{0.501875pt}%
\definecolor{currentstroke}{rgb}{0.000000,0.000000,1.000000}%
\pgfsetstrokecolor{currentstroke}%
\pgfsetstrokeopacity{0.800000}%
\pgfsetdash{}{0pt}%
\pgfpathmoveto{\pgfqpoint{4.451637in}{2.915123in}}%
\pgfpathquadraticcurveto{\pgfqpoint{4.382119in}{2.864616in}}{\pgfqpoint{4.312601in}{2.814108in}}%
\pgfusepath{stroke}%
\end{pgfscope}%
\begin{pgfscope}%
\pgfsetfillopacity{0.800000}%
\pgfsetstrokeopacity{0.800000}%
\definecolor{textcolor}{rgb}{0.000000,0.000000,1.000000}%
\pgfsetstrokecolor{textcolor}%
\pgfsetfillcolor{textcolor}%
\pgftext[x=4.378430in,y=2.980171in,left,base]{\color{textcolor}\sffamily\fontsize{5.647059}{6.776471}\selectfont 8.2858(87)}%
\end{pgfscope}%
\begin{pgfscope}%
\pgfsetbuttcap%
\pgfsetroundjoin%
\definecolor{currentfill}{rgb}{0.150000,0.150000,0.150000}%
\pgfsetfillcolor{currentfill}%
\pgfsetlinewidth{1.003750pt}%
\definecolor{currentstroke}{rgb}{0.150000,0.150000,0.150000}%
\pgfsetstrokecolor{currentstroke}%
\pgfsetdash{}{0pt}%
\pgfsys@defobject{currentmarker}{\pgfqpoint{0.000000in}{-0.066667in}}{\pgfqpoint{0.000000in}{0.000000in}}{%
\pgfpathmoveto{\pgfqpoint{0.000000in}{0.000000in}}%
\pgfpathlineto{\pgfqpoint{0.000000in}{-0.066667in}}%
\pgfusepath{stroke,fill}%
}%
\begin{pgfscope}%
\pgfsys@transformshift{3.637789in}{2.624046in}%
\pgfsys@useobject{currentmarker}{}%
\end{pgfscope}%
\end{pgfscope}%
\begin{pgfscope}%
\pgfsetbuttcap%
\pgfsetroundjoin%
\definecolor{currentfill}{rgb}{0.150000,0.150000,0.150000}%
\pgfsetfillcolor{currentfill}%
\pgfsetlinewidth{1.003750pt}%
\definecolor{currentstroke}{rgb}{0.150000,0.150000,0.150000}%
\pgfsetstrokecolor{currentstroke}%
\pgfsetdash{}{0pt}%
\pgfsys@defobject{currentmarker}{\pgfqpoint{0.000000in}{-0.066667in}}{\pgfqpoint{0.000000in}{0.000000in}}{%
\pgfpathmoveto{\pgfqpoint{0.000000in}{0.000000in}}%
\pgfpathlineto{\pgfqpoint{0.000000in}{-0.066667in}}%
\pgfusepath{stroke,fill}%
}%
\begin{pgfscope}%
\pgfsys@transformshift{4.139198in}{2.624046in}%
\pgfsys@useobject{currentmarker}{}%
\end{pgfscope}%
\end{pgfscope}%
\begin{pgfscope}%
\pgfsetbuttcap%
\pgfsetroundjoin%
\definecolor{currentfill}{rgb}{0.150000,0.150000,0.150000}%
\pgfsetfillcolor{currentfill}%
\pgfsetlinewidth{1.003750pt}%
\definecolor{currentstroke}{rgb}{0.150000,0.150000,0.150000}%
\pgfsetstrokecolor{currentstroke}%
\pgfsetdash{}{0pt}%
\pgfsys@defobject{currentmarker}{\pgfqpoint{0.000000in}{-0.066667in}}{\pgfqpoint{0.000000in}{0.000000in}}{%
\pgfpathmoveto{\pgfqpoint{0.000000in}{0.000000in}}%
\pgfpathlineto{\pgfqpoint{0.000000in}{-0.066667in}}%
\pgfusepath{stroke,fill}%
}%
\begin{pgfscope}%
\pgfsys@transformshift{4.640607in}{2.624046in}%
\pgfsys@useobject{currentmarker}{}%
\end{pgfscope}%
\end{pgfscope}%
\begin{pgfscope}%
\pgfsetbuttcap%
\pgfsetroundjoin%
\definecolor{currentfill}{rgb}{0.150000,0.150000,0.150000}%
\pgfsetfillcolor{currentfill}%
\pgfsetlinewidth{0.803000pt}%
\definecolor{currentstroke}{rgb}{0.150000,0.150000,0.150000}%
\pgfsetstrokecolor{currentstroke}%
\pgfsetdash{}{0pt}%
\pgfsys@defobject{currentmarker}{\pgfqpoint{0.000000in}{-0.044444in}}{\pgfqpoint{0.000000in}{0.000000in}}{%
\pgfpathmoveto{\pgfqpoint{0.000000in}{0.000000in}}%
\pgfpathlineto{\pgfqpoint{0.000000in}{-0.044444in}}%
\pgfusepath{stroke,fill}%
}%
\begin{pgfscope}%
\pgfsys@transformshift{3.788728in}{2.624046in}%
\pgfsys@useobject{currentmarker}{}%
\end{pgfscope}%
\end{pgfscope}%
\begin{pgfscope}%
\pgfsetbuttcap%
\pgfsetroundjoin%
\definecolor{currentfill}{rgb}{0.150000,0.150000,0.150000}%
\pgfsetfillcolor{currentfill}%
\pgfsetlinewidth{0.803000pt}%
\definecolor{currentstroke}{rgb}{0.150000,0.150000,0.150000}%
\pgfsetstrokecolor{currentstroke}%
\pgfsetdash{}{0pt}%
\pgfsys@defobject{currentmarker}{\pgfqpoint{0.000000in}{-0.044444in}}{\pgfqpoint{0.000000in}{0.000000in}}{%
\pgfpathmoveto{\pgfqpoint{0.000000in}{0.000000in}}%
\pgfpathlineto{\pgfqpoint{0.000000in}{-0.044444in}}%
\pgfusepath{stroke,fill}%
}%
\begin{pgfscope}%
\pgfsys@transformshift{3.877021in}{2.624046in}%
\pgfsys@useobject{currentmarker}{}%
\end{pgfscope}%
\end{pgfscope}%
\begin{pgfscope}%
\pgfsetbuttcap%
\pgfsetroundjoin%
\definecolor{currentfill}{rgb}{0.150000,0.150000,0.150000}%
\pgfsetfillcolor{currentfill}%
\pgfsetlinewidth{0.803000pt}%
\definecolor{currentstroke}{rgb}{0.150000,0.150000,0.150000}%
\pgfsetstrokecolor{currentstroke}%
\pgfsetdash{}{0pt}%
\pgfsys@defobject{currentmarker}{\pgfqpoint{0.000000in}{-0.044444in}}{\pgfqpoint{0.000000in}{0.000000in}}{%
\pgfpathmoveto{\pgfqpoint{0.000000in}{0.000000in}}%
\pgfpathlineto{\pgfqpoint{0.000000in}{-0.044444in}}%
\pgfusepath{stroke,fill}%
}%
\begin{pgfscope}%
\pgfsys@transformshift{3.939667in}{2.624046in}%
\pgfsys@useobject{currentmarker}{}%
\end{pgfscope}%
\end{pgfscope}%
\begin{pgfscope}%
\pgfsetbuttcap%
\pgfsetroundjoin%
\definecolor{currentfill}{rgb}{0.150000,0.150000,0.150000}%
\pgfsetfillcolor{currentfill}%
\pgfsetlinewidth{0.803000pt}%
\definecolor{currentstroke}{rgb}{0.150000,0.150000,0.150000}%
\pgfsetstrokecolor{currentstroke}%
\pgfsetdash{}{0pt}%
\pgfsys@defobject{currentmarker}{\pgfqpoint{0.000000in}{-0.044444in}}{\pgfqpoint{0.000000in}{0.000000in}}{%
\pgfpathmoveto{\pgfqpoint{0.000000in}{0.000000in}}%
\pgfpathlineto{\pgfqpoint{0.000000in}{-0.044444in}}%
\pgfusepath{stroke,fill}%
}%
\begin{pgfscope}%
\pgfsys@transformshift{3.988258in}{2.624046in}%
\pgfsys@useobject{currentmarker}{}%
\end{pgfscope}%
\end{pgfscope}%
\begin{pgfscope}%
\pgfsetbuttcap%
\pgfsetroundjoin%
\definecolor{currentfill}{rgb}{0.150000,0.150000,0.150000}%
\pgfsetfillcolor{currentfill}%
\pgfsetlinewidth{0.803000pt}%
\definecolor{currentstroke}{rgb}{0.150000,0.150000,0.150000}%
\pgfsetstrokecolor{currentstroke}%
\pgfsetdash{}{0pt}%
\pgfsys@defobject{currentmarker}{\pgfqpoint{0.000000in}{-0.044444in}}{\pgfqpoint{0.000000in}{0.000000in}}{%
\pgfpathmoveto{\pgfqpoint{0.000000in}{0.000000in}}%
\pgfpathlineto{\pgfqpoint{0.000000in}{-0.044444in}}%
\pgfusepath{stroke,fill}%
}%
\begin{pgfscope}%
\pgfsys@transformshift{4.027961in}{2.624046in}%
\pgfsys@useobject{currentmarker}{}%
\end{pgfscope}%
\end{pgfscope}%
\begin{pgfscope}%
\pgfsetbuttcap%
\pgfsetroundjoin%
\definecolor{currentfill}{rgb}{0.150000,0.150000,0.150000}%
\pgfsetfillcolor{currentfill}%
\pgfsetlinewidth{0.803000pt}%
\definecolor{currentstroke}{rgb}{0.150000,0.150000,0.150000}%
\pgfsetstrokecolor{currentstroke}%
\pgfsetdash{}{0pt}%
\pgfsys@defobject{currentmarker}{\pgfqpoint{0.000000in}{-0.044444in}}{\pgfqpoint{0.000000in}{0.000000in}}{%
\pgfpathmoveto{\pgfqpoint{0.000000in}{0.000000in}}%
\pgfpathlineto{\pgfqpoint{0.000000in}{-0.044444in}}%
\pgfusepath{stroke,fill}%
}%
\begin{pgfscope}%
\pgfsys@transformshift{4.061528in}{2.624046in}%
\pgfsys@useobject{currentmarker}{}%
\end{pgfscope}%
\end{pgfscope}%
\begin{pgfscope}%
\pgfsetbuttcap%
\pgfsetroundjoin%
\definecolor{currentfill}{rgb}{0.150000,0.150000,0.150000}%
\pgfsetfillcolor{currentfill}%
\pgfsetlinewidth{0.803000pt}%
\definecolor{currentstroke}{rgb}{0.150000,0.150000,0.150000}%
\pgfsetstrokecolor{currentstroke}%
\pgfsetdash{}{0pt}%
\pgfsys@defobject{currentmarker}{\pgfqpoint{0.000000in}{-0.044444in}}{\pgfqpoint{0.000000in}{0.000000in}}{%
\pgfpathmoveto{\pgfqpoint{0.000000in}{0.000000in}}%
\pgfpathlineto{\pgfqpoint{0.000000in}{-0.044444in}}%
\pgfusepath{stroke,fill}%
}%
\begin{pgfscope}%
\pgfsys@transformshift{4.090606in}{2.624046in}%
\pgfsys@useobject{currentmarker}{}%
\end{pgfscope}%
\end{pgfscope}%
\begin{pgfscope}%
\pgfsetbuttcap%
\pgfsetroundjoin%
\definecolor{currentfill}{rgb}{0.150000,0.150000,0.150000}%
\pgfsetfillcolor{currentfill}%
\pgfsetlinewidth{0.803000pt}%
\definecolor{currentstroke}{rgb}{0.150000,0.150000,0.150000}%
\pgfsetstrokecolor{currentstroke}%
\pgfsetdash{}{0pt}%
\pgfsys@defobject{currentmarker}{\pgfqpoint{0.000000in}{-0.044444in}}{\pgfqpoint{0.000000in}{0.000000in}}{%
\pgfpathmoveto{\pgfqpoint{0.000000in}{0.000000in}}%
\pgfpathlineto{\pgfqpoint{0.000000in}{-0.044444in}}%
\pgfusepath{stroke,fill}%
}%
\begin{pgfscope}%
\pgfsys@transformshift{4.116254in}{2.624046in}%
\pgfsys@useobject{currentmarker}{}%
\end{pgfscope}%
\end{pgfscope}%
\begin{pgfscope}%
\pgfsetbuttcap%
\pgfsetroundjoin%
\definecolor{currentfill}{rgb}{0.150000,0.150000,0.150000}%
\pgfsetfillcolor{currentfill}%
\pgfsetlinewidth{0.803000pt}%
\definecolor{currentstroke}{rgb}{0.150000,0.150000,0.150000}%
\pgfsetstrokecolor{currentstroke}%
\pgfsetdash{}{0pt}%
\pgfsys@defobject{currentmarker}{\pgfqpoint{0.000000in}{-0.044444in}}{\pgfqpoint{0.000000in}{0.000000in}}{%
\pgfpathmoveto{\pgfqpoint{0.000000in}{0.000000in}}%
\pgfpathlineto{\pgfqpoint{0.000000in}{-0.044444in}}%
\pgfusepath{stroke,fill}%
}%
\begin{pgfscope}%
\pgfsys@transformshift{4.290137in}{2.624046in}%
\pgfsys@useobject{currentmarker}{}%
\end{pgfscope}%
\end{pgfscope}%
\begin{pgfscope}%
\pgfsetbuttcap%
\pgfsetroundjoin%
\definecolor{currentfill}{rgb}{0.150000,0.150000,0.150000}%
\pgfsetfillcolor{currentfill}%
\pgfsetlinewidth{0.803000pt}%
\definecolor{currentstroke}{rgb}{0.150000,0.150000,0.150000}%
\pgfsetstrokecolor{currentstroke}%
\pgfsetdash{}{0pt}%
\pgfsys@defobject{currentmarker}{\pgfqpoint{0.000000in}{-0.044444in}}{\pgfqpoint{0.000000in}{0.000000in}}{%
\pgfpathmoveto{\pgfqpoint{0.000000in}{0.000000in}}%
\pgfpathlineto{\pgfqpoint{0.000000in}{-0.044444in}}%
\pgfusepath{stroke,fill}%
}%
\begin{pgfscope}%
\pgfsys@transformshift{4.378430in}{2.624046in}%
\pgfsys@useobject{currentmarker}{}%
\end{pgfscope}%
\end{pgfscope}%
\begin{pgfscope}%
\pgfsetbuttcap%
\pgfsetroundjoin%
\definecolor{currentfill}{rgb}{0.150000,0.150000,0.150000}%
\pgfsetfillcolor{currentfill}%
\pgfsetlinewidth{0.803000pt}%
\definecolor{currentstroke}{rgb}{0.150000,0.150000,0.150000}%
\pgfsetstrokecolor{currentstroke}%
\pgfsetdash{}{0pt}%
\pgfsys@defobject{currentmarker}{\pgfqpoint{0.000000in}{-0.044444in}}{\pgfqpoint{0.000000in}{0.000000in}}{%
\pgfpathmoveto{\pgfqpoint{0.000000in}{0.000000in}}%
\pgfpathlineto{\pgfqpoint{0.000000in}{-0.044444in}}%
\pgfusepath{stroke,fill}%
}%
\begin{pgfscope}%
\pgfsys@transformshift{4.441076in}{2.624046in}%
\pgfsys@useobject{currentmarker}{}%
\end{pgfscope}%
\end{pgfscope}%
\begin{pgfscope}%
\pgfsetbuttcap%
\pgfsetroundjoin%
\definecolor{currentfill}{rgb}{0.150000,0.150000,0.150000}%
\pgfsetfillcolor{currentfill}%
\pgfsetlinewidth{0.803000pt}%
\definecolor{currentstroke}{rgb}{0.150000,0.150000,0.150000}%
\pgfsetstrokecolor{currentstroke}%
\pgfsetdash{}{0pt}%
\pgfsys@defobject{currentmarker}{\pgfqpoint{0.000000in}{-0.044444in}}{\pgfqpoint{0.000000in}{0.000000in}}{%
\pgfpathmoveto{\pgfqpoint{0.000000in}{0.000000in}}%
\pgfpathlineto{\pgfqpoint{0.000000in}{-0.044444in}}%
\pgfusepath{stroke,fill}%
}%
\begin{pgfscope}%
\pgfsys@transformshift{4.489667in}{2.624046in}%
\pgfsys@useobject{currentmarker}{}%
\end{pgfscope}%
\end{pgfscope}%
\begin{pgfscope}%
\pgfsetbuttcap%
\pgfsetroundjoin%
\definecolor{currentfill}{rgb}{0.150000,0.150000,0.150000}%
\pgfsetfillcolor{currentfill}%
\pgfsetlinewidth{0.803000pt}%
\definecolor{currentstroke}{rgb}{0.150000,0.150000,0.150000}%
\pgfsetstrokecolor{currentstroke}%
\pgfsetdash{}{0pt}%
\pgfsys@defobject{currentmarker}{\pgfqpoint{0.000000in}{-0.044444in}}{\pgfqpoint{0.000000in}{0.000000in}}{%
\pgfpathmoveto{\pgfqpoint{0.000000in}{0.000000in}}%
\pgfpathlineto{\pgfqpoint{0.000000in}{-0.044444in}}%
\pgfusepath{stroke,fill}%
}%
\begin{pgfscope}%
\pgfsys@transformshift{4.529370in}{2.624046in}%
\pgfsys@useobject{currentmarker}{}%
\end{pgfscope}%
\end{pgfscope}%
\begin{pgfscope}%
\pgfsetbuttcap%
\pgfsetroundjoin%
\definecolor{currentfill}{rgb}{0.150000,0.150000,0.150000}%
\pgfsetfillcolor{currentfill}%
\pgfsetlinewidth{0.803000pt}%
\definecolor{currentstroke}{rgb}{0.150000,0.150000,0.150000}%
\pgfsetstrokecolor{currentstroke}%
\pgfsetdash{}{0pt}%
\pgfsys@defobject{currentmarker}{\pgfqpoint{0.000000in}{-0.044444in}}{\pgfqpoint{0.000000in}{0.000000in}}{%
\pgfpathmoveto{\pgfqpoint{0.000000in}{0.000000in}}%
\pgfpathlineto{\pgfqpoint{0.000000in}{-0.044444in}}%
\pgfusepath{stroke,fill}%
}%
\begin{pgfscope}%
\pgfsys@transformshift{4.562937in}{2.624046in}%
\pgfsys@useobject{currentmarker}{}%
\end{pgfscope}%
\end{pgfscope}%
\begin{pgfscope}%
\pgfsetbuttcap%
\pgfsetroundjoin%
\definecolor{currentfill}{rgb}{0.150000,0.150000,0.150000}%
\pgfsetfillcolor{currentfill}%
\pgfsetlinewidth{0.803000pt}%
\definecolor{currentstroke}{rgb}{0.150000,0.150000,0.150000}%
\pgfsetstrokecolor{currentstroke}%
\pgfsetdash{}{0pt}%
\pgfsys@defobject{currentmarker}{\pgfqpoint{0.000000in}{-0.044444in}}{\pgfqpoint{0.000000in}{0.000000in}}{%
\pgfpathmoveto{\pgfqpoint{0.000000in}{0.000000in}}%
\pgfpathlineto{\pgfqpoint{0.000000in}{-0.044444in}}%
\pgfusepath{stroke,fill}%
}%
\begin{pgfscope}%
\pgfsys@transformshift{4.592015in}{2.624046in}%
\pgfsys@useobject{currentmarker}{}%
\end{pgfscope}%
\end{pgfscope}%
\begin{pgfscope}%
\pgfsetbuttcap%
\pgfsetroundjoin%
\definecolor{currentfill}{rgb}{0.150000,0.150000,0.150000}%
\pgfsetfillcolor{currentfill}%
\pgfsetlinewidth{0.803000pt}%
\definecolor{currentstroke}{rgb}{0.150000,0.150000,0.150000}%
\pgfsetstrokecolor{currentstroke}%
\pgfsetdash{}{0pt}%
\pgfsys@defobject{currentmarker}{\pgfqpoint{0.000000in}{-0.044444in}}{\pgfqpoint{0.000000in}{0.000000in}}{%
\pgfpathmoveto{\pgfqpoint{0.000000in}{0.000000in}}%
\pgfpathlineto{\pgfqpoint{0.000000in}{-0.044444in}}%
\pgfusepath{stroke,fill}%
}%
\begin{pgfscope}%
\pgfsys@transformshift{4.617663in}{2.624046in}%
\pgfsys@useobject{currentmarker}{}%
\end{pgfscope}%
\end{pgfscope}%
\begin{pgfscope}%
\pgfsetbuttcap%
\pgfsetroundjoin%
\definecolor{currentfill}{rgb}{0.150000,0.150000,0.150000}%
\pgfsetfillcolor{currentfill}%
\pgfsetlinewidth{0.803000pt}%
\definecolor{currentstroke}{rgb}{0.150000,0.150000,0.150000}%
\pgfsetstrokecolor{currentstroke}%
\pgfsetdash{}{0pt}%
\pgfsys@defobject{currentmarker}{\pgfqpoint{0.000000in}{-0.044444in}}{\pgfqpoint{0.000000in}{0.000000in}}{%
\pgfpathmoveto{\pgfqpoint{0.000000in}{0.000000in}}%
\pgfpathlineto{\pgfqpoint{0.000000in}{-0.044444in}}%
\pgfusepath{stroke,fill}%
}%
\begin{pgfscope}%
\pgfsys@transformshift{4.791546in}{2.624046in}%
\pgfsys@useobject{currentmarker}{}%
\end{pgfscope}%
\end{pgfscope}%
\begin{pgfscope}%
\pgfsetbuttcap%
\pgfsetroundjoin%
\definecolor{currentfill}{rgb}{0.150000,0.150000,0.150000}%
\pgfsetfillcolor{currentfill}%
\pgfsetlinewidth{1.003750pt}%
\definecolor{currentstroke}{rgb}{0.150000,0.150000,0.150000}%
\pgfsetstrokecolor{currentstroke}%
\pgfsetdash{}{0pt}%
\pgfsys@defobject{currentmarker}{\pgfqpoint{-0.066667in}{0.000000in}}{\pgfqpoint{0.000000in}{0.000000in}}{%
\pgfpathmoveto{\pgfqpoint{0.000000in}{0.000000in}}%
\pgfpathlineto{\pgfqpoint{-0.066667in}{0.000000in}}%
\pgfusepath{stroke,fill}%
}%
\begin{pgfscope}%
\pgfsys@transformshift{3.637789in}{2.624046in}%
\pgfsys@useobject{currentmarker}{}%
\end{pgfscope}%
\end{pgfscope}%
\begin{pgfscope}%
\pgfsetbuttcap%
\pgfsetroundjoin%
\definecolor{currentfill}{rgb}{0.150000,0.150000,0.150000}%
\pgfsetfillcolor{currentfill}%
\pgfsetlinewidth{1.003750pt}%
\definecolor{currentstroke}{rgb}{0.150000,0.150000,0.150000}%
\pgfsetstrokecolor{currentstroke}%
\pgfsetdash{}{0pt}%
\pgfsys@defobject{currentmarker}{\pgfqpoint{-0.066667in}{0.000000in}}{\pgfqpoint{0.000000in}{0.000000in}}{%
\pgfpathmoveto{\pgfqpoint{0.000000in}{0.000000in}}%
\pgfpathlineto{\pgfqpoint{-0.066667in}{0.000000in}}%
\pgfusepath{stroke,fill}%
}%
\begin{pgfscope}%
\pgfsys@transformshift{3.637789in}{2.799107in}%
\pgfsys@useobject{currentmarker}{}%
\end{pgfscope}%
\end{pgfscope}%
\begin{pgfscope}%
\pgfsetbuttcap%
\pgfsetroundjoin%
\definecolor{currentfill}{rgb}{0.150000,0.150000,0.150000}%
\pgfsetfillcolor{currentfill}%
\pgfsetlinewidth{1.003750pt}%
\definecolor{currentstroke}{rgb}{0.150000,0.150000,0.150000}%
\pgfsetstrokecolor{currentstroke}%
\pgfsetdash{}{0pt}%
\pgfsys@defobject{currentmarker}{\pgfqpoint{-0.066667in}{0.000000in}}{\pgfqpoint{0.000000in}{0.000000in}}{%
\pgfpathmoveto{\pgfqpoint{0.000000in}{0.000000in}}%
\pgfpathlineto{\pgfqpoint{-0.066667in}{0.000000in}}%
\pgfusepath{stroke,fill}%
}%
\begin{pgfscope}%
\pgfsys@transformshift{3.637789in}{3.231994in}%
\pgfsys@useobject{currentmarker}{}%
\end{pgfscope}%
\end{pgfscope}%
\begin{pgfscope}%
\pgfpathrectangle{\pgfqpoint{3.637789in}{2.624046in}}{\pgfqpoint{1.223103in}{0.607948in}}%
\pgfusepath{clip}%
\pgfsetroundcap%
\pgfsetroundjoin%
\pgfsetlinewidth{1.204500pt}%
\definecolor{currentstroke}{rgb}{0.000000,0.501961,0.000000}%
\pgfsetstrokecolor{currentstroke}%
\pgfsetdash{}{0pt}%
\pgfpathmoveto{\pgfqpoint{3.637789in}{2.798233in}}%
\pgfpathlineto{\pgfqpoint{3.878498in}{2.799135in}}%
\pgfpathlineto{\pgfqpoint{3.990029in}{2.800040in}}%
\pgfpathlineto{\pgfqpoint{4.063425in}{2.800946in}}%
\pgfpathlineto{\pgfqpoint{4.118221in}{2.801853in}}%
\pgfpathlineto{\pgfqpoint{4.161963in}{2.802760in}}%
\pgfpathlineto{\pgfqpoint{4.198371in}{2.803668in}}%
\pgfpathlineto{\pgfqpoint{4.229555in}{2.804574in}}%
\pgfpathlineto{\pgfqpoint{4.256828in}{2.805480in}}%
\pgfpathlineto{\pgfqpoint{4.281062in}{2.806384in}}%
\pgfpathlineto{\pgfqpoint{4.302867in}{2.807285in}}%
\pgfpathlineto{\pgfqpoint{4.322686in}{2.808184in}}%
\pgfpathlineto{\pgfqpoint{4.340851in}{2.809078in}}%
\pgfpathlineto{\pgfqpoint{4.357617in}{2.809968in}}%
\pgfpathlineto{\pgfqpoint{4.373183in}{2.810851in}}%
\pgfpathlineto{\pgfqpoint{4.387711in}{2.811727in}}%
\pgfpathlineto{\pgfqpoint{4.401329in}{2.812595in}}%
\pgfpathlineto{\pgfqpoint{4.414146in}{2.813452in}}%
\pgfpathlineto{\pgfqpoint{4.426250in}{2.814298in}}%
\pgfpathlineto{\pgfqpoint{4.437717in}{2.815130in}}%
\pgfpathlineto{\pgfqpoint{4.448610in}{2.815947in}}%
\pgfpathlineto{\pgfqpoint{4.458984in}{2.816746in}}%
\pgfpathlineto{\pgfqpoint{4.468886in}{2.817524in}}%
\pgfpathlineto{\pgfqpoint{4.478357in}{2.818279in}}%
\pgfpathlineto{\pgfqpoint{4.487434in}{2.819008in}}%
\pgfpathlineto{\pgfqpoint{4.496147in}{2.819707in}}%
\pgfpathlineto{\pgfqpoint{4.504525in}{2.820373in}}%
\pgfpathlineto{\pgfqpoint{4.512593in}{2.821002in}}%
\pgfpathlineto{\pgfqpoint{4.520372in}{2.821589in}}%
\pgfpathlineto{\pgfqpoint{4.527883in}{2.822131in}}%
\pgfpathlineto{\pgfqpoint{4.535144in}{2.822623in}}%
\pgfpathlineto{\pgfqpoint{4.542170in}{2.823058in}}%
\pgfpathlineto{\pgfqpoint{4.548977in}{2.823432in}}%
\pgfpathlineto{\pgfqpoint{4.555577in}{2.823738in}}%
\pgfpathlineto{\pgfqpoint{4.561983in}{2.823970in}}%
\pgfpathlineto{\pgfqpoint{4.568206in}{2.824122in}}%
\pgfpathlineto{\pgfqpoint{4.574256in}{2.824185in}}%
\pgfpathlineto{\pgfqpoint{4.580143in}{2.824153in}}%
\pgfpathlineto{\pgfqpoint{4.585874in}{2.824016in}}%
\pgfpathlineto{\pgfqpoint{4.591459in}{2.823767in}}%
\pgfpathlineto{\pgfqpoint{4.596904in}{2.823397in}}%
\pgfpathlineto{\pgfqpoint{4.602216in}{2.822895in}}%
\pgfpathlineto{\pgfqpoint{4.607401in}{2.822251in}}%
\pgfpathlineto{\pgfqpoint{4.612466in}{2.821456in}}%
\pgfpathlineto{\pgfqpoint{4.617416in}{2.820497in}}%
\pgfpathlineto{\pgfqpoint{4.622256in}{2.819362in}}%
\pgfpathlineto{\pgfqpoint{4.626991in}{2.818041in}}%
\pgfpathlineto{\pgfqpoint{4.631625in}{2.816518in}}%
\pgfpathlineto{\pgfqpoint{4.636162in}{2.814782in}}%
\pgfpathlineto{\pgfqpoint{4.640607in}{2.812818in}}%
\pgfusepath{stroke}%
\end{pgfscope}%
\begin{pgfscope}%
\pgfsetrectcap%
\pgfsetmiterjoin%
\pgfsetlinewidth{1.003750pt}%
\definecolor{currentstroke}{rgb}{0.150000,0.150000,0.150000}%
\pgfsetstrokecolor{currentstroke}%
\pgfsetdash{}{0pt}%
\pgfpathmoveto{\pgfqpoint{3.637789in}{2.624046in}}%
\pgfpathlineto{\pgfqpoint{3.637789in}{3.231994in}}%
\pgfusepath{stroke}%
\end{pgfscope}%
\begin{pgfscope}%
\pgfsetrectcap%
\pgfsetmiterjoin%
\pgfsetlinewidth{1.003750pt}%
\definecolor{currentstroke}{rgb}{0.150000,0.150000,0.150000}%
\pgfsetstrokecolor{currentstroke}%
\pgfsetdash{}{0pt}%
\pgfpathmoveto{\pgfqpoint{3.637789in}{2.624046in}}%
\pgfpathlineto{\pgfqpoint{4.860892in}{2.624046in}}%
\pgfusepath{stroke}%
\end{pgfscope}%
\begin{pgfscope}%
\pgfpathrectangle{\pgfqpoint{3.637789in}{2.624046in}}{\pgfqpoint{1.223103in}{0.607948in}}%
\pgfusepath{clip}%
\pgfsetbuttcap%
\pgfsetroundjoin%
\definecolor{currentfill}{rgb}{0.000000,0.000000,0.000000}%
\pgfsetfillcolor{currentfill}%
\pgfsetlinewidth{1.003750pt}%
\definecolor{currentstroke}{rgb}{0.000000,0.000000,0.000000}%
\pgfsetstrokecolor{currentstroke}%
\pgfsetdash{}{0pt}%
\pgfsys@defobject{currentmarker}{\pgfqpoint{-0.013889in}{-0.013889in}}{\pgfqpoint{0.013889in}{0.013889in}}{%
\pgfpathmoveto{\pgfqpoint{0.000000in}{-0.013889in}}%
\pgfpathcurveto{\pgfqpoint{0.003683in}{-0.013889in}}{\pgfqpoint{0.007216in}{-0.012425in}}{\pgfqpoint{0.009821in}{-0.009821in}}%
\pgfpathcurveto{\pgfqpoint{0.012425in}{-0.007216in}}{\pgfqpoint{0.013889in}{-0.003683in}}{\pgfqpoint{0.013889in}{0.000000in}}%
\pgfpathcurveto{\pgfqpoint{0.013889in}{0.003683in}}{\pgfqpoint{0.012425in}{0.007216in}}{\pgfqpoint{0.009821in}{0.009821in}}%
\pgfpathcurveto{\pgfqpoint{0.007216in}{0.012425in}}{\pgfqpoint{0.003683in}{0.013889in}}{\pgfqpoint{0.000000in}{0.013889in}}%
\pgfpathcurveto{\pgfqpoint{-0.003683in}{0.013889in}}{\pgfqpoint{-0.007216in}{0.012425in}}{\pgfqpoint{-0.009821in}{0.009821in}}%
\pgfpathcurveto{\pgfqpoint{-0.012425in}{0.007216in}}{\pgfqpoint{-0.013889in}{0.003683in}}{\pgfqpoint{-0.013889in}{0.000000in}}%
\pgfpathcurveto{\pgfqpoint{-0.013889in}{-0.003683in}}{\pgfqpoint{-0.012425in}{-0.007216in}}{\pgfqpoint{-0.009821in}{-0.009821in}}%
\pgfpathcurveto{\pgfqpoint{-0.007216in}{-0.012425in}}{\pgfqpoint{-0.003683in}{-0.013889in}}{\pgfqpoint{0.000000in}{-0.013889in}}%
\pgfpathclose%
\pgfusepath{stroke,fill}%
}%
\begin{pgfscope}%
\pgfsys@transformshift{4.290137in}{2.806908in}%
\pgfsys@useobject{currentmarker}{}%
\end{pgfscope}%
\begin{pgfscope}%
\pgfsys@transformshift{4.294536in}{2.807070in}%
\pgfsys@useobject{currentmarker}{}%
\end{pgfscope}%
\begin{pgfscope}%
\pgfsys@transformshift{4.299026in}{2.807239in}%
\pgfsys@useobject{currentmarker}{}%
\end{pgfscope}%
\begin{pgfscope}%
\pgfsys@transformshift{4.303611in}{2.807416in}%
\pgfsys@useobject{currentmarker}{}%
\end{pgfscope}%
\begin{pgfscope}%
\pgfsys@transformshift{4.308294in}{2.807601in}%
\pgfsys@useobject{currentmarker}{}%
\end{pgfscope}%
\begin{pgfscope}%
\pgfsys@transformshift{4.313080in}{2.807793in}%
\pgfsys@useobject{currentmarker}{}%
\end{pgfscope}%
\begin{pgfscope}%
\pgfsys@transformshift{4.317974in}{2.807995in}%
\pgfsys@useobject{currentmarker}{}%
\end{pgfscope}%
\begin{pgfscope}%
\pgfsys@transformshift{4.322980in}{2.808206in}%
\pgfsys@useobject{currentmarker}{}%
\end{pgfscope}%
\begin{pgfscope}%
\pgfsys@transformshift{4.328104in}{2.808428in}%
\pgfsys@useobject{currentmarker}{}%
\end{pgfscope}%
\begin{pgfscope}%
\pgfsys@transformshift{4.333351in}{2.808660in}%
\pgfsys@useobject{currentmarker}{}%
\end{pgfscope}%
\begin{pgfscope}%
\pgfsys@transformshift{4.338728in}{2.808904in}%
\pgfsys@useobject{currentmarker}{}%
\end{pgfscope}%
\begin{pgfscope}%
\pgfsys@transformshift{4.367806in}{2.810339in}%
\pgfsys@useobject{currentmarker}{}%
\end{pgfscope}%
\begin{pgfscope}%
\pgfsys@transformshift{4.401374in}{2.812257in}%
\pgfsys@useobject{currentmarker}{}%
\end{pgfscope}%
\begin{pgfscope}%
\pgfsys@transformshift{4.640607in}{2.812679in}%
\pgfsys@useobject{currentmarker}{}%
\end{pgfscope}%
\begin{pgfscope}%
\pgfsys@transformshift{4.441076in}{2.814942in}%
\pgfsys@useobject{currentmarker}{}%
\end{pgfscope}%
\begin{pgfscope}%
\pgfsys@transformshift{4.489667in}{2.818878in}%
\pgfsys@useobject{currentmarker}{}%
\end{pgfscope}%
\begin{pgfscope}%
\pgfsys@transformshift{4.552313in}{2.824373in}%
\pgfsys@useobject{currentmarker}{}%
\end{pgfscope}%
\end{pgfscope}%
\begin{pgfscope}%
\pgfsetbuttcap%
\pgfsetmiterjoin%
\definecolor{currentfill}{rgb}{1.000000,1.000000,1.000000}%
\pgfsetfillcolor{currentfill}%
\pgfsetlinewidth{0.000000pt}%
\definecolor{currentstroke}{rgb}{0.000000,0.000000,0.000000}%
\pgfsetstrokecolor{currentstroke}%
\pgfsetstrokeopacity{0.000000}%
\pgfsetdash{}{0pt}%
\pgfpathmoveto{\pgfqpoint{5.105513in}{2.624046in}}%
\pgfpathlineto{\pgfqpoint{6.328616in}{2.624046in}}%
\pgfpathlineto{\pgfqpoint{6.328616in}{3.231994in}}%
\pgfpathlineto{\pgfqpoint{5.105513in}{3.231994in}}%
\pgfpathclose%
\pgfusepath{fill}%
\end{pgfscope}%
\begin{pgfscope}%
\pgfpathrectangle{\pgfqpoint{5.105513in}{2.624046in}}{\pgfqpoint{1.223103in}{0.607948in}}%
\pgfusepath{clip}%
\pgfsetbuttcap%
\pgfsetmiterjoin%
\definecolor{currentfill}{rgb}{0.000000,0.000000,1.000000}%
\pgfsetfillcolor{currentfill}%
\pgfsetfillopacity{0.100000}%
\pgfsetlinewidth{0.803000pt}%
\definecolor{currentstroke}{rgb}{0.000000,0.000000,1.000000}%
\pgfsetstrokecolor{currentstroke}%
\pgfsetstrokeopacity{0.100000}%
\pgfsetdash{}{0pt}%
\pgfpathmoveto{\pgfqpoint{5.105513in}{2.797925in}}%
\pgfpathlineto{\pgfqpoint{5.105513in}{2.800072in}}%
\pgfpathlineto{\pgfqpoint{6.328616in}{2.800072in}}%
\pgfpathlineto{\pgfqpoint{6.328616in}{2.797925in}}%
\pgfpathclose%
\pgfusepath{stroke,fill}%
\end{pgfscope}%
\begin{pgfscope}%
\pgfpathrectangle{\pgfqpoint{5.105513in}{2.624046in}}{\pgfqpoint{1.223103in}{0.607948in}}%
\pgfusepath{clip}%
\pgfsetbuttcap%
\pgfsetroundjoin%
\definecolor{currentfill}{rgb}{0.000000,0.501961,0.000000}%
\pgfsetfillcolor{currentfill}%
\pgfsetfillopacity{0.500000}%
\pgfsetlinewidth{0.803000pt}%
\definecolor{currentstroke}{rgb}{0.000000,0.501961,0.000000}%
\pgfsetstrokecolor{currentstroke}%
\pgfsetstrokeopacity{0.500000}%
\pgfsetdash{}{0pt}%
\pgfpathmoveto{\pgfqpoint{5.105513in}{2.800445in}}%
\pgfpathlineto{\pgfqpoint{5.105513in}{2.798530in}}%
\pgfpathlineto{\pgfqpoint{5.346222in}{2.799713in}}%
\pgfpathlineto{\pgfqpoint{5.457753in}{2.800851in}}%
\pgfpathlineto{\pgfqpoint{5.531149in}{2.801953in}}%
\pgfpathlineto{\pgfqpoint{5.585945in}{2.803026in}}%
\pgfpathlineto{\pgfqpoint{5.629687in}{2.804076in}}%
\pgfpathlineto{\pgfqpoint{5.666095in}{2.805110in}}%
\pgfpathlineto{\pgfqpoint{5.697279in}{2.806132in}}%
\pgfpathlineto{\pgfqpoint{5.724552in}{2.807147in}}%
\pgfpathlineto{\pgfqpoint{5.748786in}{2.808158in}}%
\pgfpathlineto{\pgfqpoint{5.770591in}{2.809169in}}%
\pgfpathlineto{\pgfqpoint{5.790410in}{2.810175in}}%
\pgfpathlineto{\pgfqpoint{5.808575in}{2.811156in}}%
\pgfpathlineto{\pgfqpoint{5.825341in}{2.812147in}}%
\pgfpathlineto{\pgfqpoint{5.840907in}{2.813151in}}%
\pgfpathlineto{\pgfqpoint{5.855435in}{2.814168in}}%
\pgfpathlineto{\pgfqpoint{5.869053in}{2.815197in}}%
\pgfpathlineto{\pgfqpoint{5.881870in}{2.816239in}}%
\pgfpathlineto{\pgfqpoint{5.893974in}{2.817294in}}%
\pgfpathlineto{\pgfqpoint{5.905441in}{2.818361in}}%
\pgfpathlineto{\pgfqpoint{5.916334in}{2.819441in}}%
\pgfpathlineto{\pgfqpoint{5.926708in}{2.820533in}}%
\pgfpathlineto{\pgfqpoint{5.936610in}{2.821636in}}%
\pgfpathlineto{\pgfqpoint{5.946082in}{2.822751in}}%
\pgfpathlineto{\pgfqpoint{5.955158in}{2.823876in}}%
\pgfpathlineto{\pgfqpoint{5.963871in}{2.825010in}}%
\pgfpathlineto{\pgfqpoint{5.972249in}{2.826152in}}%
\pgfpathlineto{\pgfqpoint{5.980317in}{2.827301in}}%
\pgfpathlineto{\pgfqpoint{5.988096in}{2.828456in}}%
\pgfpathlineto{\pgfqpoint{5.995607in}{2.829617in}}%
\pgfpathlineto{\pgfqpoint{6.002868in}{2.830787in}}%
\pgfpathlineto{\pgfqpoint{6.009894in}{2.831965in}}%
\pgfpathlineto{\pgfqpoint{6.016701in}{2.833153in}}%
\pgfpathlineto{\pgfqpoint{6.023301in}{2.834351in}}%
\pgfpathlineto{\pgfqpoint{6.029707in}{2.835561in}}%
\pgfpathlineto{\pgfqpoint{6.035930in}{2.836784in}}%
\pgfpathlineto{\pgfqpoint{6.041980in}{2.838022in}}%
\pgfpathlineto{\pgfqpoint{6.047867in}{2.839278in}}%
\pgfpathlineto{\pgfqpoint{6.053598in}{2.840554in}}%
\pgfpathlineto{\pgfqpoint{6.059183in}{2.841854in}}%
\pgfpathlineto{\pgfqpoint{6.064628in}{2.843181in}}%
\pgfpathlineto{\pgfqpoint{6.069940in}{2.844540in}}%
\pgfpathlineto{\pgfqpoint{6.075126in}{2.845936in}}%
\pgfpathlineto{\pgfqpoint{6.080191in}{2.847375in}}%
\pgfpathlineto{\pgfqpoint{6.085140in}{2.848863in}}%
\pgfpathlineto{\pgfqpoint{6.089980in}{2.850406in}}%
\pgfpathlineto{\pgfqpoint{6.094715in}{2.852014in}}%
\pgfpathlineto{\pgfqpoint{6.099349in}{2.853694in}}%
\pgfpathlineto{\pgfqpoint{6.103886in}{2.855454in}}%
\pgfpathlineto{\pgfqpoint{6.108331in}{2.857233in}}%
\pgfpathlineto{\pgfqpoint{6.108331in}{2.857329in}}%
\pgfpathlineto{\pgfqpoint{6.108331in}{2.857329in}}%
\pgfpathlineto{\pgfqpoint{6.103886in}{2.855717in}}%
\pgfpathlineto{\pgfqpoint{6.099349in}{2.854192in}}%
\pgfpathlineto{\pgfqpoint{6.094715in}{2.852683in}}%
\pgfpathlineto{\pgfqpoint{6.089980in}{2.851190in}}%
\pgfpathlineto{\pgfqpoint{6.085140in}{2.849714in}}%
\pgfpathlineto{\pgfqpoint{6.080191in}{2.848255in}}%
\pgfpathlineto{\pgfqpoint{6.075126in}{2.846815in}}%
\pgfpathlineto{\pgfqpoint{6.069940in}{2.845393in}}%
\pgfpathlineto{\pgfqpoint{6.064628in}{2.843990in}}%
\pgfpathlineto{\pgfqpoint{6.059183in}{2.842605in}}%
\pgfpathlineto{\pgfqpoint{6.053598in}{2.841240in}}%
\pgfpathlineto{\pgfqpoint{6.047867in}{2.839893in}}%
\pgfpathlineto{\pgfqpoint{6.041980in}{2.838566in}}%
\pgfpathlineto{\pgfqpoint{6.035930in}{2.837256in}}%
\pgfpathlineto{\pgfqpoint{6.029707in}{2.835965in}}%
\pgfpathlineto{\pgfqpoint{6.023301in}{2.834692in}}%
\pgfpathlineto{\pgfqpoint{6.016701in}{2.833435in}}%
\pgfpathlineto{\pgfqpoint{6.009894in}{2.832196in}}%
\pgfpathlineto{\pgfqpoint{6.002868in}{2.830973in}}%
\pgfpathlineto{\pgfqpoint{5.995607in}{2.829766in}}%
\pgfpathlineto{\pgfqpoint{5.988096in}{2.828575in}}%
\pgfpathlineto{\pgfqpoint{5.980317in}{2.827400in}}%
\pgfpathlineto{\pgfqpoint{5.972249in}{2.826242in}}%
\pgfpathlineto{\pgfqpoint{5.963871in}{2.825099in}}%
\pgfpathlineto{\pgfqpoint{5.955158in}{2.823970in}}%
\pgfpathlineto{\pgfqpoint{5.946082in}{2.822854in}}%
\pgfpathlineto{\pgfqpoint{5.936610in}{2.821749in}}%
\pgfpathlineto{\pgfqpoint{5.926708in}{2.820655in}}%
\pgfpathlineto{\pgfqpoint{5.916334in}{2.819570in}}%
\pgfpathlineto{\pgfqpoint{5.905441in}{2.818495in}}%
\pgfpathlineto{\pgfqpoint{5.893974in}{2.817429in}}%
\pgfpathlineto{\pgfqpoint{5.881870in}{2.816371in}}%
\pgfpathlineto{\pgfqpoint{5.869053in}{2.815322in}}%
\pgfpathlineto{\pgfqpoint{5.855435in}{2.814280in}}%
\pgfpathlineto{\pgfqpoint{5.840907in}{2.813246in}}%
\pgfpathlineto{\pgfqpoint{5.825341in}{2.812219in}}%
\pgfpathlineto{\pgfqpoint{5.808575in}{2.811198in}}%
\pgfpathlineto{\pgfqpoint{5.790410in}{2.810185in}}%
\pgfpathlineto{\pgfqpoint{5.770591in}{2.809213in}}%
\pgfpathlineto{\pgfqpoint{5.748786in}{2.808260in}}%
\pgfpathlineto{\pgfqpoint{5.724552in}{2.807320in}}%
\pgfpathlineto{\pgfqpoint{5.697279in}{2.806395in}}%
\pgfpathlineto{\pgfqpoint{5.666095in}{2.805485in}}%
\pgfpathlineto{\pgfqpoint{5.629687in}{2.804591in}}%
\pgfpathlineto{\pgfqpoint{5.585945in}{2.803716in}}%
\pgfpathlineto{\pgfqpoint{5.531149in}{2.802862in}}%
\pgfpathlineto{\pgfqpoint{5.457753in}{2.802030in}}%
\pgfpathlineto{\pgfqpoint{5.346222in}{2.801224in}}%
\pgfpathlineto{\pgfqpoint{5.105513in}{2.800445in}}%
\pgfpathclose%
\pgfusepath{stroke,fill}%
\end{pgfscope}%
\begin{pgfscope}%
\pgfpathrectangle{\pgfqpoint{5.105513in}{2.624046in}}{\pgfqpoint{1.223103in}{0.607948in}}%
\pgfusepath{clip}%
\pgfsetroundcap%
\pgfsetroundjoin%
\pgfsetlinewidth{0.501875pt}%
\definecolor{currentstroke}{rgb}{0.000000,0.000000,1.000000}%
\pgfsetstrokecolor{currentstroke}%
\pgfsetstrokeopacity{0.800000}%
\pgfsetdash{}{0pt}%
\pgfpathmoveto{\pgfqpoint{5.105513in}{2.798999in}}%
\pgfpathlineto{\pgfqpoint{6.328616in}{2.798999in}}%
\pgfusepath{stroke}%
\end{pgfscope}%
\begin{pgfscope}%
\pgfpathrectangle{\pgfqpoint{5.105513in}{2.624046in}}{\pgfqpoint{1.223103in}{0.607948in}}%
\pgfusepath{clip}%
\pgfsetbuttcap%
\pgfsetroundjoin%
\pgfsetlinewidth{1.003750pt}%
\definecolor{currentstroke}{rgb}{0.000000,0.000000,0.000000}%
\pgfsetstrokecolor{currentstroke}%
\pgfsetdash{{3.700000pt}{1.600000pt}}{0.000000pt}%
\pgfpathmoveto{\pgfqpoint{5.105513in}{2.799107in}}%
\pgfpathlineto{\pgfqpoint{6.328616in}{2.799107in}}%
\pgfusepath{stroke}%
\end{pgfscope}%
\begin{pgfscope}%
\pgfsetroundcap%
\pgfsetroundjoin%
\pgfsetlinewidth{0.501875pt}%
\definecolor{currentstroke}{rgb}{0.000000,0.000000,1.000000}%
\pgfsetstrokecolor{currentstroke}%
\pgfsetstrokeopacity{0.800000}%
\pgfsetdash{}{0pt}%
\pgfpathmoveto{\pgfqpoint{5.919361in}{2.916335in}}%
\pgfpathquadraticcurveto{\pgfqpoint{5.849843in}{2.865828in}}{\pgfqpoint{5.780325in}{2.815320in}}%
\pgfusepath{stroke}%
\end{pgfscope}%
\begin{pgfscope}%
\pgfsetfillopacity{0.800000}%
\pgfsetstrokeopacity{0.800000}%
\definecolor{textcolor}{rgb}{0.000000,0.000000,1.000000}%
\pgfsetstrokecolor{textcolor}%
\pgfsetfillcolor{textcolor}%
\pgftext[x=5.846155in,y=2.981383in,left,base]{\color{textcolor}\sffamily\fontsize{5.647059}{6.776471}\selectfont 8.2878(18)}%
\end{pgfscope}%
\begin{pgfscope}%
\pgfsetbuttcap%
\pgfsetroundjoin%
\definecolor{currentfill}{rgb}{0.150000,0.150000,0.150000}%
\pgfsetfillcolor{currentfill}%
\pgfsetlinewidth{1.003750pt}%
\definecolor{currentstroke}{rgb}{0.150000,0.150000,0.150000}%
\pgfsetstrokecolor{currentstroke}%
\pgfsetdash{}{0pt}%
\pgfsys@defobject{currentmarker}{\pgfqpoint{0.000000in}{-0.066667in}}{\pgfqpoint{0.000000in}{0.000000in}}{%
\pgfpathmoveto{\pgfqpoint{0.000000in}{0.000000in}}%
\pgfpathlineto{\pgfqpoint{0.000000in}{-0.066667in}}%
\pgfusepath{stroke,fill}%
}%
\begin{pgfscope}%
\pgfsys@transformshift{5.105513in}{2.624046in}%
\pgfsys@useobject{currentmarker}{}%
\end{pgfscope}%
\end{pgfscope}%
\begin{pgfscope}%
\pgfsetbuttcap%
\pgfsetroundjoin%
\definecolor{currentfill}{rgb}{0.150000,0.150000,0.150000}%
\pgfsetfillcolor{currentfill}%
\pgfsetlinewidth{1.003750pt}%
\definecolor{currentstroke}{rgb}{0.150000,0.150000,0.150000}%
\pgfsetstrokecolor{currentstroke}%
\pgfsetdash{}{0pt}%
\pgfsys@defobject{currentmarker}{\pgfqpoint{0.000000in}{-0.066667in}}{\pgfqpoint{0.000000in}{0.000000in}}{%
\pgfpathmoveto{\pgfqpoint{0.000000in}{0.000000in}}%
\pgfpathlineto{\pgfqpoint{0.000000in}{-0.066667in}}%
\pgfusepath{stroke,fill}%
}%
\begin{pgfscope}%
\pgfsys@transformshift{5.606922in}{2.624046in}%
\pgfsys@useobject{currentmarker}{}%
\end{pgfscope}%
\end{pgfscope}%
\begin{pgfscope}%
\pgfsetbuttcap%
\pgfsetroundjoin%
\definecolor{currentfill}{rgb}{0.150000,0.150000,0.150000}%
\pgfsetfillcolor{currentfill}%
\pgfsetlinewidth{1.003750pt}%
\definecolor{currentstroke}{rgb}{0.150000,0.150000,0.150000}%
\pgfsetstrokecolor{currentstroke}%
\pgfsetdash{}{0pt}%
\pgfsys@defobject{currentmarker}{\pgfqpoint{0.000000in}{-0.066667in}}{\pgfqpoint{0.000000in}{0.000000in}}{%
\pgfpathmoveto{\pgfqpoint{0.000000in}{0.000000in}}%
\pgfpathlineto{\pgfqpoint{0.000000in}{-0.066667in}}%
\pgfusepath{stroke,fill}%
}%
\begin{pgfscope}%
\pgfsys@transformshift{6.108331in}{2.624046in}%
\pgfsys@useobject{currentmarker}{}%
\end{pgfscope}%
\end{pgfscope}%
\begin{pgfscope}%
\pgfsetbuttcap%
\pgfsetroundjoin%
\definecolor{currentfill}{rgb}{0.150000,0.150000,0.150000}%
\pgfsetfillcolor{currentfill}%
\pgfsetlinewidth{0.803000pt}%
\definecolor{currentstroke}{rgb}{0.150000,0.150000,0.150000}%
\pgfsetstrokecolor{currentstroke}%
\pgfsetdash{}{0pt}%
\pgfsys@defobject{currentmarker}{\pgfqpoint{0.000000in}{-0.044444in}}{\pgfqpoint{0.000000in}{0.000000in}}{%
\pgfpathmoveto{\pgfqpoint{0.000000in}{0.000000in}}%
\pgfpathlineto{\pgfqpoint{0.000000in}{-0.044444in}}%
\pgfusepath{stroke,fill}%
}%
\begin{pgfscope}%
\pgfsys@transformshift{5.256452in}{2.624046in}%
\pgfsys@useobject{currentmarker}{}%
\end{pgfscope}%
\end{pgfscope}%
\begin{pgfscope}%
\pgfsetbuttcap%
\pgfsetroundjoin%
\definecolor{currentfill}{rgb}{0.150000,0.150000,0.150000}%
\pgfsetfillcolor{currentfill}%
\pgfsetlinewidth{0.803000pt}%
\definecolor{currentstroke}{rgb}{0.150000,0.150000,0.150000}%
\pgfsetstrokecolor{currentstroke}%
\pgfsetdash{}{0pt}%
\pgfsys@defobject{currentmarker}{\pgfqpoint{0.000000in}{-0.044444in}}{\pgfqpoint{0.000000in}{0.000000in}}{%
\pgfpathmoveto{\pgfqpoint{0.000000in}{0.000000in}}%
\pgfpathlineto{\pgfqpoint{0.000000in}{-0.044444in}}%
\pgfusepath{stroke,fill}%
}%
\begin{pgfscope}%
\pgfsys@transformshift{5.344746in}{2.624046in}%
\pgfsys@useobject{currentmarker}{}%
\end{pgfscope}%
\end{pgfscope}%
\begin{pgfscope}%
\pgfsetbuttcap%
\pgfsetroundjoin%
\definecolor{currentfill}{rgb}{0.150000,0.150000,0.150000}%
\pgfsetfillcolor{currentfill}%
\pgfsetlinewidth{0.803000pt}%
\definecolor{currentstroke}{rgb}{0.150000,0.150000,0.150000}%
\pgfsetstrokecolor{currentstroke}%
\pgfsetdash{}{0pt}%
\pgfsys@defobject{currentmarker}{\pgfqpoint{0.000000in}{-0.044444in}}{\pgfqpoint{0.000000in}{0.000000in}}{%
\pgfpathmoveto{\pgfqpoint{0.000000in}{0.000000in}}%
\pgfpathlineto{\pgfqpoint{0.000000in}{-0.044444in}}%
\pgfusepath{stroke,fill}%
}%
\begin{pgfscope}%
\pgfsys@transformshift{5.407391in}{2.624046in}%
\pgfsys@useobject{currentmarker}{}%
\end{pgfscope}%
\end{pgfscope}%
\begin{pgfscope}%
\pgfsetbuttcap%
\pgfsetroundjoin%
\definecolor{currentfill}{rgb}{0.150000,0.150000,0.150000}%
\pgfsetfillcolor{currentfill}%
\pgfsetlinewidth{0.803000pt}%
\definecolor{currentstroke}{rgb}{0.150000,0.150000,0.150000}%
\pgfsetstrokecolor{currentstroke}%
\pgfsetdash{}{0pt}%
\pgfsys@defobject{currentmarker}{\pgfqpoint{0.000000in}{-0.044444in}}{\pgfqpoint{0.000000in}{0.000000in}}{%
\pgfpathmoveto{\pgfqpoint{0.000000in}{0.000000in}}%
\pgfpathlineto{\pgfqpoint{0.000000in}{-0.044444in}}%
\pgfusepath{stroke,fill}%
}%
\begin{pgfscope}%
\pgfsys@transformshift{5.455982in}{2.624046in}%
\pgfsys@useobject{currentmarker}{}%
\end{pgfscope}%
\end{pgfscope}%
\begin{pgfscope}%
\pgfsetbuttcap%
\pgfsetroundjoin%
\definecolor{currentfill}{rgb}{0.150000,0.150000,0.150000}%
\pgfsetfillcolor{currentfill}%
\pgfsetlinewidth{0.803000pt}%
\definecolor{currentstroke}{rgb}{0.150000,0.150000,0.150000}%
\pgfsetstrokecolor{currentstroke}%
\pgfsetdash{}{0pt}%
\pgfsys@defobject{currentmarker}{\pgfqpoint{0.000000in}{-0.044444in}}{\pgfqpoint{0.000000in}{0.000000in}}{%
\pgfpathmoveto{\pgfqpoint{0.000000in}{0.000000in}}%
\pgfpathlineto{\pgfqpoint{0.000000in}{-0.044444in}}%
\pgfusepath{stroke,fill}%
}%
\begin{pgfscope}%
\pgfsys@transformshift{5.495685in}{2.624046in}%
\pgfsys@useobject{currentmarker}{}%
\end{pgfscope}%
\end{pgfscope}%
\begin{pgfscope}%
\pgfsetbuttcap%
\pgfsetroundjoin%
\definecolor{currentfill}{rgb}{0.150000,0.150000,0.150000}%
\pgfsetfillcolor{currentfill}%
\pgfsetlinewidth{0.803000pt}%
\definecolor{currentstroke}{rgb}{0.150000,0.150000,0.150000}%
\pgfsetstrokecolor{currentstroke}%
\pgfsetdash{}{0pt}%
\pgfsys@defobject{currentmarker}{\pgfqpoint{0.000000in}{-0.044444in}}{\pgfqpoint{0.000000in}{0.000000in}}{%
\pgfpathmoveto{\pgfqpoint{0.000000in}{0.000000in}}%
\pgfpathlineto{\pgfqpoint{0.000000in}{-0.044444in}}%
\pgfusepath{stroke,fill}%
}%
\begin{pgfscope}%
\pgfsys@transformshift{5.529252in}{2.624046in}%
\pgfsys@useobject{currentmarker}{}%
\end{pgfscope}%
\end{pgfscope}%
\begin{pgfscope}%
\pgfsetbuttcap%
\pgfsetroundjoin%
\definecolor{currentfill}{rgb}{0.150000,0.150000,0.150000}%
\pgfsetfillcolor{currentfill}%
\pgfsetlinewidth{0.803000pt}%
\definecolor{currentstroke}{rgb}{0.150000,0.150000,0.150000}%
\pgfsetstrokecolor{currentstroke}%
\pgfsetdash{}{0pt}%
\pgfsys@defobject{currentmarker}{\pgfqpoint{0.000000in}{-0.044444in}}{\pgfqpoint{0.000000in}{0.000000in}}{%
\pgfpathmoveto{\pgfqpoint{0.000000in}{0.000000in}}%
\pgfpathlineto{\pgfqpoint{0.000000in}{-0.044444in}}%
\pgfusepath{stroke,fill}%
}%
\begin{pgfscope}%
\pgfsys@transformshift{5.558330in}{2.624046in}%
\pgfsys@useobject{currentmarker}{}%
\end{pgfscope}%
\end{pgfscope}%
\begin{pgfscope}%
\pgfsetbuttcap%
\pgfsetroundjoin%
\definecolor{currentfill}{rgb}{0.150000,0.150000,0.150000}%
\pgfsetfillcolor{currentfill}%
\pgfsetlinewidth{0.803000pt}%
\definecolor{currentstroke}{rgb}{0.150000,0.150000,0.150000}%
\pgfsetstrokecolor{currentstroke}%
\pgfsetdash{}{0pt}%
\pgfsys@defobject{currentmarker}{\pgfqpoint{0.000000in}{-0.044444in}}{\pgfqpoint{0.000000in}{0.000000in}}{%
\pgfpathmoveto{\pgfqpoint{0.000000in}{0.000000in}}%
\pgfpathlineto{\pgfqpoint{0.000000in}{-0.044444in}}%
\pgfusepath{stroke,fill}%
}%
\begin{pgfscope}%
\pgfsys@transformshift{5.583978in}{2.624046in}%
\pgfsys@useobject{currentmarker}{}%
\end{pgfscope}%
\end{pgfscope}%
\begin{pgfscope}%
\pgfsetbuttcap%
\pgfsetroundjoin%
\definecolor{currentfill}{rgb}{0.150000,0.150000,0.150000}%
\pgfsetfillcolor{currentfill}%
\pgfsetlinewidth{0.803000pt}%
\definecolor{currentstroke}{rgb}{0.150000,0.150000,0.150000}%
\pgfsetstrokecolor{currentstroke}%
\pgfsetdash{}{0pt}%
\pgfsys@defobject{currentmarker}{\pgfqpoint{0.000000in}{-0.044444in}}{\pgfqpoint{0.000000in}{0.000000in}}{%
\pgfpathmoveto{\pgfqpoint{0.000000in}{0.000000in}}%
\pgfpathlineto{\pgfqpoint{0.000000in}{-0.044444in}}%
\pgfusepath{stroke,fill}%
}%
\begin{pgfscope}%
\pgfsys@transformshift{5.757861in}{2.624046in}%
\pgfsys@useobject{currentmarker}{}%
\end{pgfscope}%
\end{pgfscope}%
\begin{pgfscope}%
\pgfsetbuttcap%
\pgfsetroundjoin%
\definecolor{currentfill}{rgb}{0.150000,0.150000,0.150000}%
\pgfsetfillcolor{currentfill}%
\pgfsetlinewidth{0.803000pt}%
\definecolor{currentstroke}{rgb}{0.150000,0.150000,0.150000}%
\pgfsetstrokecolor{currentstroke}%
\pgfsetdash{}{0pt}%
\pgfsys@defobject{currentmarker}{\pgfqpoint{0.000000in}{-0.044444in}}{\pgfqpoint{0.000000in}{0.000000in}}{%
\pgfpathmoveto{\pgfqpoint{0.000000in}{0.000000in}}%
\pgfpathlineto{\pgfqpoint{0.000000in}{-0.044444in}}%
\pgfusepath{stroke,fill}%
}%
\begin{pgfscope}%
\pgfsys@transformshift{5.846155in}{2.624046in}%
\pgfsys@useobject{currentmarker}{}%
\end{pgfscope}%
\end{pgfscope}%
\begin{pgfscope}%
\pgfsetbuttcap%
\pgfsetroundjoin%
\definecolor{currentfill}{rgb}{0.150000,0.150000,0.150000}%
\pgfsetfillcolor{currentfill}%
\pgfsetlinewidth{0.803000pt}%
\definecolor{currentstroke}{rgb}{0.150000,0.150000,0.150000}%
\pgfsetstrokecolor{currentstroke}%
\pgfsetdash{}{0pt}%
\pgfsys@defobject{currentmarker}{\pgfqpoint{0.000000in}{-0.044444in}}{\pgfqpoint{0.000000in}{0.000000in}}{%
\pgfpathmoveto{\pgfqpoint{0.000000in}{0.000000in}}%
\pgfpathlineto{\pgfqpoint{0.000000in}{-0.044444in}}%
\pgfusepath{stroke,fill}%
}%
\begin{pgfscope}%
\pgfsys@transformshift{5.908800in}{2.624046in}%
\pgfsys@useobject{currentmarker}{}%
\end{pgfscope}%
\end{pgfscope}%
\begin{pgfscope}%
\pgfsetbuttcap%
\pgfsetroundjoin%
\definecolor{currentfill}{rgb}{0.150000,0.150000,0.150000}%
\pgfsetfillcolor{currentfill}%
\pgfsetlinewidth{0.803000pt}%
\definecolor{currentstroke}{rgb}{0.150000,0.150000,0.150000}%
\pgfsetstrokecolor{currentstroke}%
\pgfsetdash{}{0pt}%
\pgfsys@defobject{currentmarker}{\pgfqpoint{0.000000in}{-0.044444in}}{\pgfqpoint{0.000000in}{0.000000in}}{%
\pgfpathmoveto{\pgfqpoint{0.000000in}{0.000000in}}%
\pgfpathlineto{\pgfqpoint{0.000000in}{-0.044444in}}%
\pgfusepath{stroke,fill}%
}%
\begin{pgfscope}%
\pgfsys@transformshift{5.957391in}{2.624046in}%
\pgfsys@useobject{currentmarker}{}%
\end{pgfscope}%
\end{pgfscope}%
\begin{pgfscope}%
\pgfsetbuttcap%
\pgfsetroundjoin%
\definecolor{currentfill}{rgb}{0.150000,0.150000,0.150000}%
\pgfsetfillcolor{currentfill}%
\pgfsetlinewidth{0.803000pt}%
\definecolor{currentstroke}{rgb}{0.150000,0.150000,0.150000}%
\pgfsetstrokecolor{currentstroke}%
\pgfsetdash{}{0pt}%
\pgfsys@defobject{currentmarker}{\pgfqpoint{0.000000in}{-0.044444in}}{\pgfqpoint{0.000000in}{0.000000in}}{%
\pgfpathmoveto{\pgfqpoint{0.000000in}{0.000000in}}%
\pgfpathlineto{\pgfqpoint{0.000000in}{-0.044444in}}%
\pgfusepath{stroke,fill}%
}%
\begin{pgfscope}%
\pgfsys@transformshift{5.997094in}{2.624046in}%
\pgfsys@useobject{currentmarker}{}%
\end{pgfscope}%
\end{pgfscope}%
\begin{pgfscope}%
\pgfsetbuttcap%
\pgfsetroundjoin%
\definecolor{currentfill}{rgb}{0.150000,0.150000,0.150000}%
\pgfsetfillcolor{currentfill}%
\pgfsetlinewidth{0.803000pt}%
\definecolor{currentstroke}{rgb}{0.150000,0.150000,0.150000}%
\pgfsetstrokecolor{currentstroke}%
\pgfsetdash{}{0pt}%
\pgfsys@defobject{currentmarker}{\pgfqpoint{0.000000in}{-0.044444in}}{\pgfqpoint{0.000000in}{0.000000in}}{%
\pgfpathmoveto{\pgfqpoint{0.000000in}{0.000000in}}%
\pgfpathlineto{\pgfqpoint{0.000000in}{-0.044444in}}%
\pgfusepath{stroke,fill}%
}%
\begin{pgfscope}%
\pgfsys@transformshift{6.030661in}{2.624046in}%
\pgfsys@useobject{currentmarker}{}%
\end{pgfscope}%
\end{pgfscope}%
\begin{pgfscope}%
\pgfsetbuttcap%
\pgfsetroundjoin%
\definecolor{currentfill}{rgb}{0.150000,0.150000,0.150000}%
\pgfsetfillcolor{currentfill}%
\pgfsetlinewidth{0.803000pt}%
\definecolor{currentstroke}{rgb}{0.150000,0.150000,0.150000}%
\pgfsetstrokecolor{currentstroke}%
\pgfsetdash{}{0pt}%
\pgfsys@defobject{currentmarker}{\pgfqpoint{0.000000in}{-0.044444in}}{\pgfqpoint{0.000000in}{0.000000in}}{%
\pgfpathmoveto{\pgfqpoint{0.000000in}{0.000000in}}%
\pgfpathlineto{\pgfqpoint{0.000000in}{-0.044444in}}%
\pgfusepath{stroke,fill}%
}%
\begin{pgfscope}%
\pgfsys@transformshift{6.059739in}{2.624046in}%
\pgfsys@useobject{currentmarker}{}%
\end{pgfscope}%
\end{pgfscope}%
\begin{pgfscope}%
\pgfsetbuttcap%
\pgfsetroundjoin%
\definecolor{currentfill}{rgb}{0.150000,0.150000,0.150000}%
\pgfsetfillcolor{currentfill}%
\pgfsetlinewidth{0.803000pt}%
\definecolor{currentstroke}{rgb}{0.150000,0.150000,0.150000}%
\pgfsetstrokecolor{currentstroke}%
\pgfsetdash{}{0pt}%
\pgfsys@defobject{currentmarker}{\pgfqpoint{0.000000in}{-0.044444in}}{\pgfqpoint{0.000000in}{0.000000in}}{%
\pgfpathmoveto{\pgfqpoint{0.000000in}{0.000000in}}%
\pgfpathlineto{\pgfqpoint{0.000000in}{-0.044444in}}%
\pgfusepath{stroke,fill}%
}%
\begin{pgfscope}%
\pgfsys@transformshift{6.085387in}{2.624046in}%
\pgfsys@useobject{currentmarker}{}%
\end{pgfscope}%
\end{pgfscope}%
\begin{pgfscope}%
\pgfsetbuttcap%
\pgfsetroundjoin%
\definecolor{currentfill}{rgb}{0.150000,0.150000,0.150000}%
\pgfsetfillcolor{currentfill}%
\pgfsetlinewidth{0.803000pt}%
\definecolor{currentstroke}{rgb}{0.150000,0.150000,0.150000}%
\pgfsetstrokecolor{currentstroke}%
\pgfsetdash{}{0pt}%
\pgfsys@defobject{currentmarker}{\pgfqpoint{0.000000in}{-0.044444in}}{\pgfqpoint{0.000000in}{0.000000in}}{%
\pgfpathmoveto{\pgfqpoint{0.000000in}{0.000000in}}%
\pgfpathlineto{\pgfqpoint{0.000000in}{-0.044444in}}%
\pgfusepath{stroke,fill}%
}%
\begin{pgfscope}%
\pgfsys@transformshift{6.259270in}{2.624046in}%
\pgfsys@useobject{currentmarker}{}%
\end{pgfscope}%
\end{pgfscope}%
\begin{pgfscope}%
\pgfsetbuttcap%
\pgfsetroundjoin%
\definecolor{currentfill}{rgb}{0.150000,0.150000,0.150000}%
\pgfsetfillcolor{currentfill}%
\pgfsetlinewidth{1.003750pt}%
\definecolor{currentstroke}{rgb}{0.150000,0.150000,0.150000}%
\pgfsetstrokecolor{currentstroke}%
\pgfsetdash{}{0pt}%
\pgfsys@defobject{currentmarker}{\pgfqpoint{-0.066667in}{0.000000in}}{\pgfqpoint{0.000000in}{0.000000in}}{%
\pgfpathmoveto{\pgfqpoint{0.000000in}{0.000000in}}%
\pgfpathlineto{\pgfqpoint{-0.066667in}{0.000000in}}%
\pgfusepath{stroke,fill}%
}%
\begin{pgfscope}%
\pgfsys@transformshift{5.105513in}{2.624046in}%
\pgfsys@useobject{currentmarker}{}%
\end{pgfscope}%
\end{pgfscope}%
\begin{pgfscope}%
\pgfsetbuttcap%
\pgfsetroundjoin%
\definecolor{currentfill}{rgb}{0.150000,0.150000,0.150000}%
\pgfsetfillcolor{currentfill}%
\pgfsetlinewidth{1.003750pt}%
\definecolor{currentstroke}{rgb}{0.150000,0.150000,0.150000}%
\pgfsetstrokecolor{currentstroke}%
\pgfsetdash{}{0pt}%
\pgfsys@defobject{currentmarker}{\pgfqpoint{-0.066667in}{0.000000in}}{\pgfqpoint{0.000000in}{0.000000in}}{%
\pgfpathmoveto{\pgfqpoint{0.000000in}{0.000000in}}%
\pgfpathlineto{\pgfqpoint{-0.066667in}{0.000000in}}%
\pgfusepath{stroke,fill}%
}%
\begin{pgfscope}%
\pgfsys@transformshift{5.105513in}{2.799107in}%
\pgfsys@useobject{currentmarker}{}%
\end{pgfscope}%
\end{pgfscope}%
\begin{pgfscope}%
\pgfsetbuttcap%
\pgfsetroundjoin%
\definecolor{currentfill}{rgb}{0.150000,0.150000,0.150000}%
\pgfsetfillcolor{currentfill}%
\pgfsetlinewidth{1.003750pt}%
\definecolor{currentstroke}{rgb}{0.150000,0.150000,0.150000}%
\pgfsetstrokecolor{currentstroke}%
\pgfsetdash{}{0pt}%
\pgfsys@defobject{currentmarker}{\pgfqpoint{-0.066667in}{0.000000in}}{\pgfqpoint{0.000000in}{0.000000in}}{%
\pgfpathmoveto{\pgfqpoint{0.000000in}{0.000000in}}%
\pgfpathlineto{\pgfqpoint{-0.066667in}{0.000000in}}%
\pgfusepath{stroke,fill}%
}%
\begin{pgfscope}%
\pgfsys@transformshift{5.105513in}{3.231994in}%
\pgfsys@useobject{currentmarker}{}%
\end{pgfscope}%
\end{pgfscope}%
\begin{pgfscope}%
\pgfpathrectangle{\pgfqpoint{5.105513in}{2.624046in}}{\pgfqpoint{1.223103in}{0.607948in}}%
\pgfusepath{clip}%
\pgfsetroundcap%
\pgfsetroundjoin%
\pgfsetlinewidth{1.204500pt}%
\definecolor{currentstroke}{rgb}{0.000000,0.501961,0.000000}%
\pgfsetstrokecolor{currentstroke}%
\pgfsetdash{}{0pt}%
\pgfpathmoveto{\pgfqpoint{5.105513in}{2.799488in}}%
\pgfpathlineto{\pgfqpoint{5.346222in}{2.800468in}}%
\pgfpathlineto{\pgfqpoint{5.457753in}{2.801441in}}%
\pgfpathlineto{\pgfqpoint{5.531149in}{2.802407in}}%
\pgfpathlineto{\pgfqpoint{5.585945in}{2.803371in}}%
\pgfpathlineto{\pgfqpoint{5.629687in}{2.804334in}}%
\pgfpathlineto{\pgfqpoint{5.666095in}{2.805297in}}%
\pgfpathlineto{\pgfqpoint{5.697279in}{2.806263in}}%
\pgfpathlineto{\pgfqpoint{5.724552in}{2.807234in}}%
\pgfpathlineto{\pgfqpoint{5.748786in}{2.808209in}}%
\pgfpathlineto{\pgfqpoint{5.770591in}{2.809191in}}%
\pgfpathlineto{\pgfqpoint{5.790410in}{2.810180in}}%
\pgfpathlineto{\pgfqpoint{5.808575in}{2.811177in}}%
\pgfpathlineto{\pgfqpoint{5.825341in}{2.812183in}}%
\pgfpathlineto{\pgfqpoint{5.840907in}{2.813199in}}%
\pgfpathlineto{\pgfqpoint{5.855435in}{2.814224in}}%
\pgfpathlineto{\pgfqpoint{5.869053in}{2.815259in}}%
\pgfpathlineto{\pgfqpoint{5.881870in}{2.816305in}}%
\pgfpathlineto{\pgfqpoint{5.893974in}{2.817361in}}%
\pgfpathlineto{\pgfqpoint{5.905441in}{2.818428in}}%
\pgfpathlineto{\pgfqpoint{5.916334in}{2.819505in}}%
\pgfpathlineto{\pgfqpoint{5.926708in}{2.820594in}}%
\pgfpathlineto{\pgfqpoint{5.936610in}{2.821693in}}%
\pgfpathlineto{\pgfqpoint{5.946082in}{2.822802in}}%
\pgfpathlineto{\pgfqpoint{5.955158in}{2.823923in}}%
\pgfpathlineto{\pgfqpoint{5.963871in}{2.825055in}}%
\pgfpathlineto{\pgfqpoint{5.972249in}{2.826197in}}%
\pgfpathlineto{\pgfqpoint{5.980317in}{2.827350in}}%
\pgfpathlineto{\pgfqpoint{5.988096in}{2.828515in}}%
\pgfpathlineto{\pgfqpoint{5.995607in}{2.829692in}}%
\pgfpathlineto{\pgfqpoint{6.002868in}{2.830880in}}%
\pgfpathlineto{\pgfqpoint{6.009894in}{2.832081in}}%
\pgfpathlineto{\pgfqpoint{6.016701in}{2.833294in}}%
\pgfpathlineto{\pgfqpoint{6.023301in}{2.834521in}}%
\pgfpathlineto{\pgfqpoint{6.029707in}{2.835763in}}%
\pgfpathlineto{\pgfqpoint{6.035930in}{2.837020in}}%
\pgfpathlineto{\pgfqpoint{6.041980in}{2.838294in}}%
\pgfpathlineto{\pgfqpoint{6.047867in}{2.839586in}}%
\pgfpathlineto{\pgfqpoint{6.053598in}{2.840897in}}%
\pgfpathlineto{\pgfqpoint{6.059183in}{2.842230in}}%
\pgfpathlineto{\pgfqpoint{6.064628in}{2.843585in}}%
\pgfpathlineto{\pgfqpoint{6.069940in}{2.844967in}}%
\pgfpathlineto{\pgfqpoint{6.075126in}{2.846376in}}%
\pgfpathlineto{\pgfqpoint{6.080191in}{2.847815in}}%
\pgfpathlineto{\pgfqpoint{6.085140in}{2.849288in}}%
\pgfpathlineto{\pgfqpoint{6.089980in}{2.850798in}}%
\pgfpathlineto{\pgfqpoint{6.094715in}{2.852348in}}%
\pgfpathlineto{\pgfqpoint{6.099349in}{2.853943in}}%
\pgfpathlineto{\pgfqpoint{6.103886in}{2.855585in}}%
\pgfpathlineto{\pgfqpoint{6.108331in}{2.857281in}}%
\pgfusepath{stroke}%
\end{pgfscope}%
\begin{pgfscope}%
\pgfsetrectcap%
\pgfsetmiterjoin%
\pgfsetlinewidth{1.003750pt}%
\definecolor{currentstroke}{rgb}{0.150000,0.150000,0.150000}%
\pgfsetstrokecolor{currentstroke}%
\pgfsetdash{}{0pt}%
\pgfpathmoveto{\pgfqpoint{5.105513in}{2.624046in}}%
\pgfpathlineto{\pgfqpoint{5.105513in}{3.231994in}}%
\pgfusepath{stroke}%
\end{pgfscope}%
\begin{pgfscope}%
\pgfsetrectcap%
\pgfsetmiterjoin%
\pgfsetlinewidth{1.003750pt}%
\definecolor{currentstroke}{rgb}{0.150000,0.150000,0.150000}%
\pgfsetstrokecolor{currentstroke}%
\pgfsetdash{}{0pt}%
\pgfpathmoveto{\pgfqpoint{5.105513in}{2.624046in}}%
\pgfpathlineto{\pgfqpoint{6.328616in}{2.624046in}}%
\pgfusepath{stroke}%
\end{pgfscope}%
\begin{pgfscope}%
\pgfpathrectangle{\pgfqpoint{5.105513in}{2.624046in}}{\pgfqpoint{1.223103in}{0.607948in}}%
\pgfusepath{clip}%
\pgfsetbuttcap%
\pgfsetroundjoin%
\definecolor{currentfill}{rgb}{0.000000,0.000000,0.000000}%
\pgfsetfillcolor{currentfill}%
\pgfsetlinewidth{1.003750pt}%
\definecolor{currentstroke}{rgb}{0.000000,0.000000,0.000000}%
\pgfsetstrokecolor{currentstroke}%
\pgfsetdash{}{0pt}%
\pgfsys@defobject{currentmarker}{\pgfqpoint{-0.013889in}{-0.013889in}}{\pgfqpoint{0.013889in}{0.013889in}}{%
\pgfpathmoveto{\pgfqpoint{0.000000in}{-0.013889in}}%
\pgfpathcurveto{\pgfqpoint{0.003683in}{-0.013889in}}{\pgfqpoint{0.007216in}{-0.012425in}}{\pgfqpoint{0.009821in}{-0.009821in}}%
\pgfpathcurveto{\pgfqpoint{0.012425in}{-0.007216in}}{\pgfqpoint{0.013889in}{-0.003683in}}{\pgfqpoint{0.013889in}{0.000000in}}%
\pgfpathcurveto{\pgfqpoint{0.013889in}{0.003683in}}{\pgfqpoint{0.012425in}{0.007216in}}{\pgfqpoint{0.009821in}{0.009821in}}%
\pgfpathcurveto{\pgfqpoint{0.007216in}{0.012425in}}{\pgfqpoint{0.003683in}{0.013889in}}{\pgfqpoint{0.000000in}{0.013889in}}%
\pgfpathcurveto{\pgfqpoint{-0.003683in}{0.013889in}}{\pgfqpoint{-0.007216in}{0.012425in}}{\pgfqpoint{-0.009821in}{0.009821in}}%
\pgfpathcurveto{\pgfqpoint{-0.012425in}{0.007216in}}{\pgfqpoint{-0.013889in}{0.003683in}}{\pgfqpoint{-0.013889in}{0.000000in}}%
\pgfpathcurveto{\pgfqpoint{-0.013889in}{-0.003683in}}{\pgfqpoint{-0.012425in}{-0.007216in}}{\pgfqpoint{-0.009821in}{-0.009821in}}%
\pgfpathcurveto{\pgfqpoint{-0.007216in}{-0.012425in}}{\pgfqpoint{-0.003683in}{-0.013889in}}{\pgfqpoint{0.000000in}{-0.013889in}}%
\pgfpathclose%
\pgfusepath{stroke,fill}%
}%
\begin{pgfscope}%
\pgfsys@transformshift{5.757861in}{2.808591in}%
\pgfsys@useobject{currentmarker}{}%
\end{pgfscope}%
\begin{pgfscope}%
\pgfsys@transformshift{5.762260in}{2.808785in}%
\pgfsys@useobject{currentmarker}{}%
\end{pgfscope}%
\begin{pgfscope}%
\pgfsys@transformshift{5.766750in}{2.809001in}%
\pgfsys@useobject{currentmarker}{}%
\end{pgfscope}%
\begin{pgfscope}%
\pgfsys@transformshift{5.771335in}{2.809212in}%
\pgfsys@useobject{currentmarker}{}%
\end{pgfscope}%
\begin{pgfscope}%
\pgfsys@transformshift{5.776018in}{2.809447in}%
\pgfsys@useobject{currentmarker}{}%
\end{pgfscope}%
\begin{pgfscope}%
\pgfsys@transformshift{5.780804in}{2.809678in}%
\pgfsys@useobject{currentmarker}{}%
\end{pgfscope}%
\begin{pgfscope}%
\pgfsys@transformshift{5.785698in}{2.809936in}%
\pgfsys@useobject{currentmarker}{}%
\end{pgfscope}%
\begin{pgfscope}%
\pgfsys@transformshift{5.790704in}{2.810190in}%
\pgfsys@useobject{currentmarker}{}%
\end{pgfscope}%
\begin{pgfscope}%
\pgfsys@transformshift{5.795828in}{2.810474in}%
\pgfsys@useobject{currentmarker}{}%
\end{pgfscope}%
\begin{pgfscope}%
\pgfsys@transformshift{5.801075in}{2.810753in}%
\pgfsys@useobject{currentmarker}{}%
\end{pgfscope}%
\begin{pgfscope}%
\pgfsys@transformshift{5.806452in}{2.811068in}%
\pgfsys@useobject{currentmarker}{}%
\end{pgfscope}%
\begin{pgfscope}%
\pgfsys@transformshift{5.835530in}{2.812851in}%
\pgfsys@useobject{currentmarker}{}%
\end{pgfscope}%
\begin{pgfscope}%
\pgfsys@transformshift{5.869098in}{2.815309in}%
\pgfsys@useobject{currentmarker}{}%
\end{pgfscope}%
\begin{pgfscope}%
\pgfsys@transformshift{5.908800in}{2.818764in}%
\pgfsys@useobject{currentmarker}{}%
\end{pgfscope}%
\begin{pgfscope}%
\pgfsys@transformshift{5.957391in}{2.824284in}%
\pgfsys@useobject{currentmarker}{}%
\end{pgfscope}%
\begin{pgfscope}%
\pgfsys@transformshift{6.020037in}{2.833801in}%
\pgfsys@useobject{currentmarker}{}%
\end{pgfscope}%
\begin{pgfscope}%
\pgfsys@transformshift{6.108331in}{2.857311in}%
\pgfsys@useobject{currentmarker}{}%
\end{pgfscope}%
\end{pgfscope}%
\begin{pgfscope}%
\pgfsetbuttcap%
\pgfsetmiterjoin%
\definecolor{currentfill}{rgb}{1.000000,1.000000,1.000000}%
\pgfsetfillcolor{currentfill}%
\pgfsetlinewidth{0.803000pt}%
\definecolor{currentstroke}{rgb}{1.000000,1.000000,1.000000}%
\pgfsetstrokecolor{currentstroke}%
\pgfsetdash{}{0pt}%
\pgfpathmoveto{\pgfqpoint{6.297392in}{3.162294in}}%
\pgfpathlineto{\pgfqpoint{6.297392in}{2.693745in}}%
\pgfpathlineto{\pgfqpoint{6.478411in}{2.693745in}}%
\pgfpathlineto{\pgfqpoint{6.478411in}{3.162294in}}%
\pgfpathclose%
\pgfusepath{stroke,fill}%
\end{pgfscope}%
\begin{pgfscope}%
\definecolor{textcolor}{rgb}{0.150000,0.150000,0.150000}%
\pgfsetstrokecolor{textcolor}%
\pgfsetfillcolor{textcolor}%
\pgftext[x=6.368294in,y=3.106608in,left,base,rotate=270.000000]{\color{textcolor}\sffamily\fontsize{5.647059}{6.776471}\selectfont nlevel = 8}%
\end{pgfscope}%
\begin{pgfscope}%
\pgfsetbuttcap%
\pgfsetmiterjoin%
\definecolor{currentfill}{rgb}{1.000000,1.000000,1.000000}%
\pgfsetfillcolor{currentfill}%
\pgfsetlinewidth{0.803000pt}%
\definecolor{currentstroke}{rgb}{1.000000,1.000000,1.000000}%
\pgfsetstrokecolor{currentstroke}%
\pgfsetdash{}{0pt}%
\pgfpathmoveto{\pgfqpoint{6.297392in}{3.162294in}}%
\pgfpathlineto{\pgfqpoint{6.297392in}{2.693745in}}%
\pgfpathlineto{\pgfqpoint{6.478411in}{2.693745in}}%
\pgfpathlineto{\pgfqpoint{6.478411in}{3.162294in}}%
\pgfpathclose%
\pgfusepath{stroke,fill}%
\end{pgfscope}%
\begin{pgfscope}%
\definecolor{textcolor}{rgb}{0.150000,0.150000,0.150000}%
\pgfsetstrokecolor{textcolor}%
\pgfsetfillcolor{textcolor}%
\pgftext[x=6.368294in,y=3.106608in,left,base,rotate=270.000000]{\color{textcolor}\sffamily\fontsize{5.647059}{6.776471}\selectfont nlevel = 8}%
\end{pgfscope}%
\begin{pgfscope}%
\pgfsetbuttcap%
\pgfsetmiterjoin%
\definecolor{currentfill}{rgb}{1.000000,1.000000,1.000000}%
\pgfsetfillcolor{currentfill}%
\pgfsetlinewidth{0.000000pt}%
\definecolor{currentstroke}{rgb}{0.000000,0.000000,0.000000}%
\pgfsetstrokecolor{currentstroke}%
\pgfsetstrokeopacity{0.000000}%
\pgfsetdash{}{0pt}%
\pgfpathmoveto{\pgfqpoint{0.702340in}{1.894508in}}%
\pgfpathlineto{\pgfqpoint{1.925444in}{1.894508in}}%
\pgfpathlineto{\pgfqpoint{1.925444in}{2.502456in}}%
\pgfpathlineto{\pgfqpoint{0.702340in}{2.502456in}}%
\pgfpathclose%
\pgfusepath{fill}%
\end{pgfscope}%
\begin{pgfscope}%
\pgfpathrectangle{\pgfqpoint{0.702340in}{1.894508in}}{\pgfqpoint{1.223103in}{0.607948in}}%
\pgfusepath{clip}%
\pgfsetbuttcap%
\pgfsetmiterjoin%
\definecolor{currentfill}{rgb}{0.000000,0.000000,1.000000}%
\pgfsetfillcolor{currentfill}%
\pgfsetfillopacity{0.100000}%
\pgfsetlinewidth{0.803000pt}%
\definecolor{currentstroke}{rgb}{0.000000,0.000000,1.000000}%
\pgfsetstrokecolor{currentstroke}%
\pgfsetstrokeopacity{0.100000}%
\pgfsetdash{}{0pt}%
\pgfpathmoveto{\pgfqpoint{0.702340in}{2.220464in}}%
\pgfpathlineto{\pgfqpoint{0.702340in}{2.223716in}}%
\pgfpathlineto{\pgfqpoint{1.925444in}{2.223716in}}%
\pgfpathlineto{\pgfqpoint{1.925444in}{2.220464in}}%
\pgfpathclose%
\pgfusepath{stroke,fill}%
\end{pgfscope}%
\begin{pgfscope}%
\pgfpathrectangle{\pgfqpoint{0.702340in}{1.894508in}}{\pgfqpoint{1.223103in}{0.607948in}}%
\pgfusepath{clip}%
\pgfsetbuttcap%
\pgfsetroundjoin%
\definecolor{currentfill}{rgb}{0.000000,0.501961,0.000000}%
\pgfsetfillcolor{currentfill}%
\pgfsetfillopacity{0.500000}%
\pgfsetlinewidth{0.803000pt}%
\definecolor{currentstroke}{rgb}{0.000000,0.501961,0.000000}%
\pgfsetstrokecolor{currentstroke}%
\pgfsetstrokeopacity{0.500000}%
\pgfsetdash{}{0pt}%
\pgfpathmoveto{\pgfqpoint{0.702340in}{2.223224in}}%
\pgfpathlineto{\pgfqpoint{0.702340in}{2.220196in}}%
\pgfpathlineto{\pgfqpoint{0.943050in}{2.219433in}}%
\pgfpathlineto{\pgfqpoint{1.054581in}{2.218373in}}%
\pgfpathlineto{\pgfqpoint{1.127977in}{2.217019in}}%
\pgfpathlineto{\pgfqpoint{1.182772in}{2.215370in}}%
\pgfpathlineto{\pgfqpoint{1.226515in}{2.213428in}}%
\pgfpathlineto{\pgfqpoint{1.262923in}{2.211195in}}%
\pgfpathlineto{\pgfqpoint{1.294107in}{2.208671in}}%
\pgfpathlineto{\pgfqpoint{1.321380in}{2.205858in}}%
\pgfpathlineto{\pgfqpoint{1.345614in}{2.202757in}}%
\pgfpathlineto{\pgfqpoint{1.367419in}{2.199368in}}%
\pgfpathlineto{\pgfqpoint{1.387238in}{2.195680in}}%
\pgfpathlineto{\pgfqpoint{1.405403in}{2.191606in}}%
\pgfpathlineto{\pgfqpoint{1.422168in}{2.187270in}}%
\pgfpathlineto{\pgfqpoint{1.437735in}{2.182674in}}%
\pgfpathlineto{\pgfqpoint{1.452262in}{2.177819in}}%
\pgfpathlineto{\pgfqpoint{1.465881in}{2.172704in}}%
\pgfpathlineto{\pgfqpoint{1.478698in}{2.167330in}}%
\pgfpathlineto{\pgfqpoint{1.490802in}{2.161698in}}%
\pgfpathlineto{\pgfqpoint{1.502269in}{2.155808in}}%
\pgfpathlineto{\pgfqpoint{1.513162in}{2.149660in}}%
\pgfpathlineto{\pgfqpoint{1.523536in}{2.143255in}}%
\pgfpathlineto{\pgfqpoint{1.533438in}{2.136593in}}%
\pgfpathlineto{\pgfqpoint{1.542909in}{2.129675in}}%
\pgfpathlineto{\pgfqpoint{1.551986in}{2.122501in}}%
\pgfpathlineto{\pgfqpoint{1.560699in}{2.115071in}}%
\pgfpathlineto{\pgfqpoint{1.569077in}{2.107382in}}%
\pgfpathlineto{\pgfqpoint{1.577145in}{2.099375in}}%
\pgfpathlineto{\pgfqpoint{1.584924in}{2.091098in}}%
\pgfpathlineto{\pgfqpoint{1.592435in}{2.082566in}}%
\pgfpathlineto{\pgfqpoint{1.599696in}{2.073782in}}%
\pgfpathlineto{\pgfqpoint{1.606722in}{2.064748in}}%
\pgfpathlineto{\pgfqpoint{1.613528in}{2.055466in}}%
\pgfpathlineto{\pgfqpoint{1.620129in}{2.045939in}}%
\pgfpathlineto{\pgfqpoint{1.626535in}{2.036168in}}%
\pgfpathlineto{\pgfqpoint{1.632758in}{2.026155in}}%
\pgfpathlineto{\pgfqpoint{1.638808in}{2.015904in}}%
\pgfpathlineto{\pgfqpoint{1.644694in}{2.005416in}}%
\pgfpathlineto{\pgfqpoint{1.650426in}{1.994694in}}%
\pgfpathlineto{\pgfqpoint{1.656011in}{1.983739in}}%
\pgfpathlineto{\pgfqpoint{1.661456in}{1.972555in}}%
\pgfpathlineto{\pgfqpoint{1.666768in}{1.961144in}}%
\pgfpathlineto{\pgfqpoint{1.671953in}{1.949508in}}%
\pgfpathlineto{\pgfqpoint{1.677018in}{1.937649in}}%
\pgfpathlineto{\pgfqpoint{1.681968in}{1.925571in}}%
\pgfpathlineto{\pgfqpoint{1.686808in}{1.913275in}}%
\pgfpathlineto{\pgfqpoint{1.691543in}{1.900764in}}%
\pgfpathlineto{\pgfqpoint{1.696176in}{1.888040in}}%
\pgfpathlineto{\pgfqpoint{1.700714in}{1.875092in}}%
\pgfpathlineto{\pgfqpoint{1.705158in}{1.861723in}}%
\pgfpathlineto{\pgfqpoint{1.705158in}{1.861968in}}%
\pgfpathlineto{\pgfqpoint{1.705158in}{1.861968in}}%
\pgfpathlineto{\pgfqpoint{1.700714in}{1.875117in}}%
\pgfpathlineto{\pgfqpoint{1.696176in}{1.888239in}}%
\pgfpathlineto{\pgfqpoint{1.691543in}{1.901134in}}%
\pgfpathlineto{\pgfqpoint{1.686808in}{1.913787in}}%
\pgfpathlineto{\pgfqpoint{1.681968in}{1.926198in}}%
\pgfpathlineto{\pgfqpoint{1.677018in}{1.938366in}}%
\pgfpathlineto{\pgfqpoint{1.671953in}{1.950289in}}%
\pgfpathlineto{\pgfqpoint{1.666768in}{1.961968in}}%
\pgfpathlineto{\pgfqpoint{1.661456in}{1.973402in}}%
\pgfpathlineto{\pgfqpoint{1.656011in}{1.984590in}}%
\pgfpathlineto{\pgfqpoint{1.650426in}{1.995532in}}%
\pgfpathlineto{\pgfqpoint{1.644694in}{2.006227in}}%
\pgfpathlineto{\pgfqpoint{1.638808in}{2.016674in}}%
\pgfpathlineto{\pgfqpoint{1.632758in}{2.026873in}}%
\pgfpathlineto{\pgfqpoint{1.626535in}{2.036822in}}%
\pgfpathlineto{\pgfqpoint{1.620129in}{2.046523in}}%
\pgfpathlineto{\pgfqpoint{1.613528in}{2.055973in}}%
\pgfpathlineto{\pgfqpoint{1.606722in}{2.065172in}}%
\pgfpathlineto{\pgfqpoint{1.599696in}{2.074121in}}%
\pgfpathlineto{\pgfqpoint{1.592435in}{2.082817in}}%
\pgfpathlineto{\pgfqpoint{1.584924in}{2.091261in}}%
\pgfpathlineto{\pgfqpoint{1.577145in}{2.099453in}}%
\pgfpathlineto{\pgfqpoint{1.569077in}{2.107405in}}%
\pgfpathlineto{\pgfqpoint{1.560699in}{2.115163in}}%
\pgfpathlineto{\pgfqpoint{1.551986in}{2.122665in}}%
\pgfpathlineto{\pgfqpoint{1.542909in}{2.129906in}}%
\pgfpathlineto{\pgfqpoint{1.533438in}{2.136883in}}%
\pgfpathlineto{\pgfqpoint{1.523536in}{2.143594in}}%
\pgfpathlineto{\pgfqpoint{1.513162in}{2.150038in}}%
\pgfpathlineto{\pgfqpoint{1.502269in}{2.156211in}}%
\pgfpathlineto{\pgfqpoint{1.490802in}{2.162114in}}%
\pgfpathlineto{\pgfqpoint{1.478698in}{2.167743in}}%
\pgfpathlineto{\pgfqpoint{1.465881in}{2.173098in}}%
\pgfpathlineto{\pgfqpoint{1.452262in}{2.178176in}}%
\pgfpathlineto{\pgfqpoint{1.437735in}{2.182977in}}%
\pgfpathlineto{\pgfqpoint{1.422168in}{2.187498in}}%
\pgfpathlineto{\pgfqpoint{1.405403in}{2.191738in}}%
\pgfpathlineto{\pgfqpoint{1.387238in}{2.195695in}}%
\pgfpathlineto{\pgfqpoint{1.367419in}{2.199495in}}%
\pgfpathlineto{\pgfqpoint{1.345614in}{2.203047in}}%
\pgfpathlineto{\pgfqpoint{1.321380in}{2.206339in}}%
\pgfpathlineto{\pgfqpoint{1.294107in}{2.209368in}}%
\pgfpathlineto{\pgfqpoint{1.262923in}{2.212136in}}%
\pgfpathlineto{\pgfqpoint{1.226515in}{2.214641in}}%
\pgfpathlineto{\pgfqpoint{1.182772in}{2.216883in}}%
\pgfpathlineto{\pgfqpoint{1.127977in}{2.218863in}}%
\pgfpathlineto{\pgfqpoint{1.054581in}{2.220580in}}%
\pgfpathlineto{\pgfqpoint{0.943050in}{2.222034in}}%
\pgfpathlineto{\pgfqpoint{0.702340in}{2.223224in}}%
\pgfpathclose%
\pgfusepath{stroke,fill}%
\end{pgfscope}%
\begin{pgfscope}%
\pgfpathrectangle{\pgfqpoint{0.702340in}{1.894508in}}{\pgfqpoint{1.223103in}{0.607948in}}%
\pgfusepath{clip}%
\pgfsetroundcap%
\pgfsetroundjoin%
\pgfsetlinewidth{0.501875pt}%
\definecolor{currentstroke}{rgb}{0.000000,0.000000,1.000000}%
\pgfsetstrokecolor{currentstroke}%
\pgfsetstrokeopacity{0.800000}%
\pgfsetdash{}{0pt}%
\pgfpathmoveto{\pgfqpoint{0.702340in}{2.222090in}}%
\pgfpathlineto{\pgfqpoint{1.925444in}{2.222090in}}%
\pgfusepath{stroke}%
\end{pgfscope}%
\begin{pgfscope}%
\pgfpathrectangle{\pgfqpoint{0.702340in}{1.894508in}}{\pgfqpoint{1.223103in}{0.607948in}}%
\pgfusepath{clip}%
\pgfsetbuttcap%
\pgfsetroundjoin%
\pgfsetlinewidth{1.003750pt}%
\definecolor{currentstroke}{rgb}{0.000000,0.000000,0.000000}%
\pgfsetstrokecolor{currentstroke}%
\pgfsetdash{{3.700000pt}{1.600000pt}}{0.000000pt}%
\pgfpathmoveto{\pgfqpoint{0.702340in}{2.221402in}}%
\pgfpathlineto{\pgfqpoint{1.925444in}{2.221402in}}%
\pgfusepath{stroke}%
\end{pgfscope}%
\begin{pgfscope}%
\pgfsetroundcap%
\pgfsetroundjoin%
\pgfsetlinewidth{0.501875pt}%
\definecolor{currentstroke}{rgb}{0.000000,0.000000,1.000000}%
\pgfsetstrokecolor{currentstroke}%
\pgfsetstrokeopacity{0.800000}%
\pgfsetdash{}{0pt}%
\pgfpathmoveto{\pgfqpoint{1.516189in}{2.339427in}}%
\pgfpathquadraticcurveto{\pgfqpoint{1.446671in}{2.288919in}}{\pgfqpoint{1.377153in}{2.238412in}}%
\pgfusepath{stroke}%
\end{pgfscope}%
\begin{pgfscope}%
\pgfsetfillopacity{0.800000}%
\pgfsetstrokeopacity{0.800000}%
\definecolor{textcolor}{rgb}{0.000000,0.000000,1.000000}%
\pgfsetstrokecolor{textcolor}%
\pgfsetfillcolor{textcolor}%
\pgftext[x=1.442982in,y=2.404475in,left,base]{\color{textcolor}\sffamily\fontsize{5.647059}{6.776471}\selectfont 5.5388(27)}%
\end{pgfscope}%
\begin{pgfscope}%
\pgfsetbuttcap%
\pgfsetroundjoin%
\definecolor{currentfill}{rgb}{0.150000,0.150000,0.150000}%
\pgfsetfillcolor{currentfill}%
\pgfsetlinewidth{1.003750pt}%
\definecolor{currentstroke}{rgb}{0.150000,0.150000,0.150000}%
\pgfsetstrokecolor{currentstroke}%
\pgfsetdash{}{0pt}%
\pgfsys@defobject{currentmarker}{\pgfqpoint{0.000000in}{-0.066667in}}{\pgfqpoint{0.000000in}{0.000000in}}{%
\pgfpathmoveto{\pgfqpoint{0.000000in}{0.000000in}}%
\pgfpathlineto{\pgfqpoint{0.000000in}{-0.066667in}}%
\pgfusepath{stroke,fill}%
}%
\begin{pgfscope}%
\pgfsys@transformshift{0.702340in}{1.894508in}%
\pgfsys@useobject{currentmarker}{}%
\end{pgfscope}%
\end{pgfscope}%
\begin{pgfscope}%
\pgfsetbuttcap%
\pgfsetroundjoin%
\definecolor{currentfill}{rgb}{0.150000,0.150000,0.150000}%
\pgfsetfillcolor{currentfill}%
\pgfsetlinewidth{1.003750pt}%
\definecolor{currentstroke}{rgb}{0.150000,0.150000,0.150000}%
\pgfsetstrokecolor{currentstroke}%
\pgfsetdash{}{0pt}%
\pgfsys@defobject{currentmarker}{\pgfqpoint{0.000000in}{-0.066667in}}{\pgfqpoint{0.000000in}{0.000000in}}{%
\pgfpathmoveto{\pgfqpoint{0.000000in}{0.000000in}}%
\pgfpathlineto{\pgfqpoint{0.000000in}{-0.066667in}}%
\pgfusepath{stroke,fill}%
}%
\begin{pgfscope}%
\pgfsys@transformshift{1.203749in}{1.894508in}%
\pgfsys@useobject{currentmarker}{}%
\end{pgfscope}%
\end{pgfscope}%
\begin{pgfscope}%
\pgfsetbuttcap%
\pgfsetroundjoin%
\definecolor{currentfill}{rgb}{0.150000,0.150000,0.150000}%
\pgfsetfillcolor{currentfill}%
\pgfsetlinewidth{1.003750pt}%
\definecolor{currentstroke}{rgb}{0.150000,0.150000,0.150000}%
\pgfsetstrokecolor{currentstroke}%
\pgfsetdash{}{0pt}%
\pgfsys@defobject{currentmarker}{\pgfqpoint{0.000000in}{-0.066667in}}{\pgfqpoint{0.000000in}{0.000000in}}{%
\pgfpathmoveto{\pgfqpoint{0.000000in}{0.000000in}}%
\pgfpathlineto{\pgfqpoint{0.000000in}{-0.066667in}}%
\pgfusepath{stroke,fill}%
}%
\begin{pgfscope}%
\pgfsys@transformshift{1.705158in}{1.894508in}%
\pgfsys@useobject{currentmarker}{}%
\end{pgfscope}%
\end{pgfscope}%
\begin{pgfscope}%
\pgfsetbuttcap%
\pgfsetroundjoin%
\definecolor{currentfill}{rgb}{0.150000,0.150000,0.150000}%
\pgfsetfillcolor{currentfill}%
\pgfsetlinewidth{0.803000pt}%
\definecolor{currentstroke}{rgb}{0.150000,0.150000,0.150000}%
\pgfsetstrokecolor{currentstroke}%
\pgfsetdash{}{0pt}%
\pgfsys@defobject{currentmarker}{\pgfqpoint{0.000000in}{-0.044444in}}{\pgfqpoint{0.000000in}{0.000000in}}{%
\pgfpathmoveto{\pgfqpoint{0.000000in}{0.000000in}}%
\pgfpathlineto{\pgfqpoint{0.000000in}{-0.044444in}}%
\pgfusepath{stroke,fill}%
}%
\begin{pgfscope}%
\pgfsys@transformshift{0.853280in}{1.894508in}%
\pgfsys@useobject{currentmarker}{}%
\end{pgfscope}%
\end{pgfscope}%
\begin{pgfscope}%
\pgfsetbuttcap%
\pgfsetroundjoin%
\definecolor{currentfill}{rgb}{0.150000,0.150000,0.150000}%
\pgfsetfillcolor{currentfill}%
\pgfsetlinewidth{0.803000pt}%
\definecolor{currentstroke}{rgb}{0.150000,0.150000,0.150000}%
\pgfsetstrokecolor{currentstroke}%
\pgfsetdash{}{0pt}%
\pgfsys@defobject{currentmarker}{\pgfqpoint{0.000000in}{-0.044444in}}{\pgfqpoint{0.000000in}{0.000000in}}{%
\pgfpathmoveto{\pgfqpoint{0.000000in}{0.000000in}}%
\pgfpathlineto{\pgfqpoint{0.000000in}{-0.044444in}}%
\pgfusepath{stroke,fill}%
}%
\begin{pgfscope}%
\pgfsys@transformshift{0.941573in}{1.894508in}%
\pgfsys@useobject{currentmarker}{}%
\end{pgfscope}%
\end{pgfscope}%
\begin{pgfscope}%
\pgfsetbuttcap%
\pgfsetroundjoin%
\definecolor{currentfill}{rgb}{0.150000,0.150000,0.150000}%
\pgfsetfillcolor{currentfill}%
\pgfsetlinewidth{0.803000pt}%
\definecolor{currentstroke}{rgb}{0.150000,0.150000,0.150000}%
\pgfsetstrokecolor{currentstroke}%
\pgfsetdash{}{0pt}%
\pgfsys@defobject{currentmarker}{\pgfqpoint{0.000000in}{-0.044444in}}{\pgfqpoint{0.000000in}{0.000000in}}{%
\pgfpathmoveto{\pgfqpoint{0.000000in}{0.000000in}}%
\pgfpathlineto{\pgfqpoint{0.000000in}{-0.044444in}}%
\pgfusepath{stroke,fill}%
}%
\begin{pgfscope}%
\pgfsys@transformshift{1.004219in}{1.894508in}%
\pgfsys@useobject{currentmarker}{}%
\end{pgfscope}%
\end{pgfscope}%
\begin{pgfscope}%
\pgfsetbuttcap%
\pgfsetroundjoin%
\definecolor{currentfill}{rgb}{0.150000,0.150000,0.150000}%
\pgfsetfillcolor{currentfill}%
\pgfsetlinewidth{0.803000pt}%
\definecolor{currentstroke}{rgb}{0.150000,0.150000,0.150000}%
\pgfsetstrokecolor{currentstroke}%
\pgfsetdash{}{0pt}%
\pgfsys@defobject{currentmarker}{\pgfqpoint{0.000000in}{-0.044444in}}{\pgfqpoint{0.000000in}{0.000000in}}{%
\pgfpathmoveto{\pgfqpoint{0.000000in}{0.000000in}}%
\pgfpathlineto{\pgfqpoint{0.000000in}{-0.044444in}}%
\pgfusepath{stroke,fill}%
}%
\begin{pgfscope}%
\pgfsys@transformshift{1.052810in}{1.894508in}%
\pgfsys@useobject{currentmarker}{}%
\end{pgfscope}%
\end{pgfscope}%
\begin{pgfscope}%
\pgfsetbuttcap%
\pgfsetroundjoin%
\definecolor{currentfill}{rgb}{0.150000,0.150000,0.150000}%
\pgfsetfillcolor{currentfill}%
\pgfsetlinewidth{0.803000pt}%
\definecolor{currentstroke}{rgb}{0.150000,0.150000,0.150000}%
\pgfsetstrokecolor{currentstroke}%
\pgfsetdash{}{0pt}%
\pgfsys@defobject{currentmarker}{\pgfqpoint{0.000000in}{-0.044444in}}{\pgfqpoint{0.000000in}{0.000000in}}{%
\pgfpathmoveto{\pgfqpoint{0.000000in}{0.000000in}}%
\pgfpathlineto{\pgfqpoint{0.000000in}{-0.044444in}}%
\pgfusepath{stroke,fill}%
}%
\begin{pgfscope}%
\pgfsys@transformshift{1.092512in}{1.894508in}%
\pgfsys@useobject{currentmarker}{}%
\end{pgfscope}%
\end{pgfscope}%
\begin{pgfscope}%
\pgfsetbuttcap%
\pgfsetroundjoin%
\definecolor{currentfill}{rgb}{0.150000,0.150000,0.150000}%
\pgfsetfillcolor{currentfill}%
\pgfsetlinewidth{0.803000pt}%
\definecolor{currentstroke}{rgb}{0.150000,0.150000,0.150000}%
\pgfsetstrokecolor{currentstroke}%
\pgfsetdash{}{0pt}%
\pgfsys@defobject{currentmarker}{\pgfqpoint{0.000000in}{-0.044444in}}{\pgfqpoint{0.000000in}{0.000000in}}{%
\pgfpathmoveto{\pgfqpoint{0.000000in}{0.000000in}}%
\pgfpathlineto{\pgfqpoint{0.000000in}{-0.044444in}}%
\pgfusepath{stroke,fill}%
}%
\begin{pgfscope}%
\pgfsys@transformshift{1.126080in}{1.894508in}%
\pgfsys@useobject{currentmarker}{}%
\end{pgfscope}%
\end{pgfscope}%
\begin{pgfscope}%
\pgfsetbuttcap%
\pgfsetroundjoin%
\definecolor{currentfill}{rgb}{0.150000,0.150000,0.150000}%
\pgfsetfillcolor{currentfill}%
\pgfsetlinewidth{0.803000pt}%
\definecolor{currentstroke}{rgb}{0.150000,0.150000,0.150000}%
\pgfsetstrokecolor{currentstroke}%
\pgfsetdash{}{0pt}%
\pgfsys@defobject{currentmarker}{\pgfqpoint{0.000000in}{-0.044444in}}{\pgfqpoint{0.000000in}{0.000000in}}{%
\pgfpathmoveto{\pgfqpoint{0.000000in}{0.000000in}}%
\pgfpathlineto{\pgfqpoint{0.000000in}{-0.044444in}}%
\pgfusepath{stroke,fill}%
}%
\begin{pgfscope}%
\pgfsys@transformshift{1.155158in}{1.894508in}%
\pgfsys@useobject{currentmarker}{}%
\end{pgfscope}%
\end{pgfscope}%
\begin{pgfscope}%
\pgfsetbuttcap%
\pgfsetroundjoin%
\definecolor{currentfill}{rgb}{0.150000,0.150000,0.150000}%
\pgfsetfillcolor{currentfill}%
\pgfsetlinewidth{0.803000pt}%
\definecolor{currentstroke}{rgb}{0.150000,0.150000,0.150000}%
\pgfsetstrokecolor{currentstroke}%
\pgfsetdash{}{0pt}%
\pgfsys@defobject{currentmarker}{\pgfqpoint{0.000000in}{-0.044444in}}{\pgfqpoint{0.000000in}{0.000000in}}{%
\pgfpathmoveto{\pgfqpoint{0.000000in}{0.000000in}}%
\pgfpathlineto{\pgfqpoint{0.000000in}{-0.044444in}}%
\pgfusepath{stroke,fill}%
}%
\begin{pgfscope}%
\pgfsys@transformshift{1.180806in}{1.894508in}%
\pgfsys@useobject{currentmarker}{}%
\end{pgfscope}%
\end{pgfscope}%
\begin{pgfscope}%
\pgfsetbuttcap%
\pgfsetroundjoin%
\definecolor{currentfill}{rgb}{0.150000,0.150000,0.150000}%
\pgfsetfillcolor{currentfill}%
\pgfsetlinewidth{0.803000pt}%
\definecolor{currentstroke}{rgb}{0.150000,0.150000,0.150000}%
\pgfsetstrokecolor{currentstroke}%
\pgfsetdash{}{0pt}%
\pgfsys@defobject{currentmarker}{\pgfqpoint{0.000000in}{-0.044444in}}{\pgfqpoint{0.000000in}{0.000000in}}{%
\pgfpathmoveto{\pgfqpoint{0.000000in}{0.000000in}}%
\pgfpathlineto{\pgfqpoint{0.000000in}{-0.044444in}}%
\pgfusepath{stroke,fill}%
}%
\begin{pgfscope}%
\pgfsys@transformshift{1.354689in}{1.894508in}%
\pgfsys@useobject{currentmarker}{}%
\end{pgfscope}%
\end{pgfscope}%
\begin{pgfscope}%
\pgfsetbuttcap%
\pgfsetroundjoin%
\definecolor{currentfill}{rgb}{0.150000,0.150000,0.150000}%
\pgfsetfillcolor{currentfill}%
\pgfsetlinewidth{0.803000pt}%
\definecolor{currentstroke}{rgb}{0.150000,0.150000,0.150000}%
\pgfsetstrokecolor{currentstroke}%
\pgfsetdash{}{0pt}%
\pgfsys@defobject{currentmarker}{\pgfqpoint{0.000000in}{-0.044444in}}{\pgfqpoint{0.000000in}{0.000000in}}{%
\pgfpathmoveto{\pgfqpoint{0.000000in}{0.000000in}}%
\pgfpathlineto{\pgfqpoint{0.000000in}{-0.044444in}}%
\pgfusepath{stroke,fill}%
}%
\begin{pgfscope}%
\pgfsys@transformshift{1.442982in}{1.894508in}%
\pgfsys@useobject{currentmarker}{}%
\end{pgfscope}%
\end{pgfscope}%
\begin{pgfscope}%
\pgfsetbuttcap%
\pgfsetroundjoin%
\definecolor{currentfill}{rgb}{0.150000,0.150000,0.150000}%
\pgfsetfillcolor{currentfill}%
\pgfsetlinewidth{0.803000pt}%
\definecolor{currentstroke}{rgb}{0.150000,0.150000,0.150000}%
\pgfsetstrokecolor{currentstroke}%
\pgfsetdash{}{0pt}%
\pgfsys@defobject{currentmarker}{\pgfqpoint{0.000000in}{-0.044444in}}{\pgfqpoint{0.000000in}{0.000000in}}{%
\pgfpathmoveto{\pgfqpoint{0.000000in}{0.000000in}}%
\pgfpathlineto{\pgfqpoint{0.000000in}{-0.044444in}}%
\pgfusepath{stroke,fill}%
}%
\begin{pgfscope}%
\pgfsys@transformshift{1.505628in}{1.894508in}%
\pgfsys@useobject{currentmarker}{}%
\end{pgfscope}%
\end{pgfscope}%
\begin{pgfscope}%
\pgfsetbuttcap%
\pgfsetroundjoin%
\definecolor{currentfill}{rgb}{0.150000,0.150000,0.150000}%
\pgfsetfillcolor{currentfill}%
\pgfsetlinewidth{0.803000pt}%
\definecolor{currentstroke}{rgb}{0.150000,0.150000,0.150000}%
\pgfsetstrokecolor{currentstroke}%
\pgfsetdash{}{0pt}%
\pgfsys@defobject{currentmarker}{\pgfqpoint{0.000000in}{-0.044444in}}{\pgfqpoint{0.000000in}{0.000000in}}{%
\pgfpathmoveto{\pgfqpoint{0.000000in}{0.000000in}}%
\pgfpathlineto{\pgfqpoint{0.000000in}{-0.044444in}}%
\pgfusepath{stroke,fill}%
}%
\begin{pgfscope}%
\pgfsys@transformshift{1.554219in}{1.894508in}%
\pgfsys@useobject{currentmarker}{}%
\end{pgfscope}%
\end{pgfscope}%
\begin{pgfscope}%
\pgfsetbuttcap%
\pgfsetroundjoin%
\definecolor{currentfill}{rgb}{0.150000,0.150000,0.150000}%
\pgfsetfillcolor{currentfill}%
\pgfsetlinewidth{0.803000pt}%
\definecolor{currentstroke}{rgb}{0.150000,0.150000,0.150000}%
\pgfsetstrokecolor{currentstroke}%
\pgfsetdash{}{0pt}%
\pgfsys@defobject{currentmarker}{\pgfqpoint{0.000000in}{-0.044444in}}{\pgfqpoint{0.000000in}{0.000000in}}{%
\pgfpathmoveto{\pgfqpoint{0.000000in}{0.000000in}}%
\pgfpathlineto{\pgfqpoint{0.000000in}{-0.044444in}}%
\pgfusepath{stroke,fill}%
}%
\begin{pgfscope}%
\pgfsys@transformshift{1.593921in}{1.894508in}%
\pgfsys@useobject{currentmarker}{}%
\end{pgfscope}%
\end{pgfscope}%
\begin{pgfscope}%
\pgfsetbuttcap%
\pgfsetroundjoin%
\definecolor{currentfill}{rgb}{0.150000,0.150000,0.150000}%
\pgfsetfillcolor{currentfill}%
\pgfsetlinewidth{0.803000pt}%
\definecolor{currentstroke}{rgb}{0.150000,0.150000,0.150000}%
\pgfsetstrokecolor{currentstroke}%
\pgfsetdash{}{0pt}%
\pgfsys@defobject{currentmarker}{\pgfqpoint{0.000000in}{-0.044444in}}{\pgfqpoint{0.000000in}{0.000000in}}{%
\pgfpathmoveto{\pgfqpoint{0.000000in}{0.000000in}}%
\pgfpathlineto{\pgfqpoint{0.000000in}{-0.044444in}}%
\pgfusepath{stroke,fill}%
}%
\begin{pgfscope}%
\pgfsys@transformshift{1.627489in}{1.894508in}%
\pgfsys@useobject{currentmarker}{}%
\end{pgfscope}%
\end{pgfscope}%
\begin{pgfscope}%
\pgfsetbuttcap%
\pgfsetroundjoin%
\definecolor{currentfill}{rgb}{0.150000,0.150000,0.150000}%
\pgfsetfillcolor{currentfill}%
\pgfsetlinewidth{0.803000pt}%
\definecolor{currentstroke}{rgb}{0.150000,0.150000,0.150000}%
\pgfsetstrokecolor{currentstroke}%
\pgfsetdash{}{0pt}%
\pgfsys@defobject{currentmarker}{\pgfqpoint{0.000000in}{-0.044444in}}{\pgfqpoint{0.000000in}{0.000000in}}{%
\pgfpathmoveto{\pgfqpoint{0.000000in}{0.000000in}}%
\pgfpathlineto{\pgfqpoint{0.000000in}{-0.044444in}}%
\pgfusepath{stroke,fill}%
}%
\begin{pgfscope}%
\pgfsys@transformshift{1.656567in}{1.894508in}%
\pgfsys@useobject{currentmarker}{}%
\end{pgfscope}%
\end{pgfscope}%
\begin{pgfscope}%
\pgfsetbuttcap%
\pgfsetroundjoin%
\definecolor{currentfill}{rgb}{0.150000,0.150000,0.150000}%
\pgfsetfillcolor{currentfill}%
\pgfsetlinewidth{0.803000pt}%
\definecolor{currentstroke}{rgb}{0.150000,0.150000,0.150000}%
\pgfsetstrokecolor{currentstroke}%
\pgfsetdash{}{0pt}%
\pgfsys@defobject{currentmarker}{\pgfqpoint{0.000000in}{-0.044444in}}{\pgfqpoint{0.000000in}{0.000000in}}{%
\pgfpathmoveto{\pgfqpoint{0.000000in}{0.000000in}}%
\pgfpathlineto{\pgfqpoint{0.000000in}{-0.044444in}}%
\pgfusepath{stroke,fill}%
}%
\begin{pgfscope}%
\pgfsys@transformshift{1.682215in}{1.894508in}%
\pgfsys@useobject{currentmarker}{}%
\end{pgfscope}%
\end{pgfscope}%
\begin{pgfscope}%
\pgfsetbuttcap%
\pgfsetroundjoin%
\definecolor{currentfill}{rgb}{0.150000,0.150000,0.150000}%
\pgfsetfillcolor{currentfill}%
\pgfsetlinewidth{0.803000pt}%
\definecolor{currentstroke}{rgb}{0.150000,0.150000,0.150000}%
\pgfsetstrokecolor{currentstroke}%
\pgfsetdash{}{0pt}%
\pgfsys@defobject{currentmarker}{\pgfqpoint{0.000000in}{-0.044444in}}{\pgfqpoint{0.000000in}{0.000000in}}{%
\pgfpathmoveto{\pgfqpoint{0.000000in}{0.000000in}}%
\pgfpathlineto{\pgfqpoint{0.000000in}{-0.044444in}}%
\pgfusepath{stroke,fill}%
}%
\begin{pgfscope}%
\pgfsys@transformshift{1.856098in}{1.894508in}%
\pgfsys@useobject{currentmarker}{}%
\end{pgfscope}%
\end{pgfscope}%
\begin{pgfscope}%
\pgfsetbuttcap%
\pgfsetroundjoin%
\definecolor{currentfill}{rgb}{0.150000,0.150000,0.150000}%
\pgfsetfillcolor{currentfill}%
\pgfsetlinewidth{1.003750pt}%
\definecolor{currentstroke}{rgb}{0.150000,0.150000,0.150000}%
\pgfsetstrokecolor{currentstroke}%
\pgfsetdash{}{0pt}%
\pgfsys@defobject{currentmarker}{\pgfqpoint{-0.066667in}{0.000000in}}{\pgfqpoint{0.000000in}{0.000000in}}{%
\pgfpathmoveto{\pgfqpoint{0.000000in}{0.000000in}}%
\pgfpathlineto{\pgfqpoint{-0.066667in}{0.000000in}}%
\pgfusepath{stroke,fill}%
}%
\begin{pgfscope}%
\pgfsys@transformshift{0.702340in}{1.894508in}%
\pgfsys@useobject{currentmarker}{}%
\end{pgfscope}%
\end{pgfscope}%
\begin{pgfscope}%
\definecolor{textcolor}{rgb}{0.150000,0.150000,0.150000}%
\pgfsetstrokecolor{textcolor}%
\pgfsetfillcolor{textcolor}%
\pgftext[x=0.413148in,y=1.869560in,left,base]{\color{textcolor}\sffamily\fontsize{5.176471}{6.211765}\selectfont 5.000}%
\end{pgfscope}%
\begin{pgfscope}%
\pgfsetbuttcap%
\pgfsetroundjoin%
\definecolor{currentfill}{rgb}{0.150000,0.150000,0.150000}%
\pgfsetfillcolor{currentfill}%
\pgfsetlinewidth{1.003750pt}%
\definecolor{currentstroke}{rgb}{0.150000,0.150000,0.150000}%
\pgfsetstrokecolor{currentstroke}%
\pgfsetdash{}{0pt}%
\pgfsys@defobject{currentmarker}{\pgfqpoint{-0.066667in}{0.000000in}}{\pgfqpoint{0.000000in}{0.000000in}}{%
\pgfpathmoveto{\pgfqpoint{0.000000in}{0.000000in}}%
\pgfpathlineto{\pgfqpoint{-0.066667in}{0.000000in}}%
\pgfusepath{stroke,fill}%
}%
\begin{pgfscope}%
\pgfsys@transformshift{0.702340in}{2.221402in}%
\pgfsys@useobject{currentmarker}{}%
\end{pgfscope}%
\end{pgfscope}%
\begin{pgfscope}%
\definecolor{textcolor}{rgb}{0.150000,0.150000,0.150000}%
\pgfsetstrokecolor{textcolor}%
\pgfsetfillcolor{textcolor}%
\pgftext[x=0.413148in,y=2.196454in,left,base]{\color{textcolor}\sffamily\fontsize{5.176471}{6.211765}\selectfont 5.538}%
\end{pgfscope}%
\begin{pgfscope}%
\pgfsetbuttcap%
\pgfsetroundjoin%
\definecolor{currentfill}{rgb}{0.150000,0.150000,0.150000}%
\pgfsetfillcolor{currentfill}%
\pgfsetlinewidth{1.003750pt}%
\definecolor{currentstroke}{rgb}{0.150000,0.150000,0.150000}%
\pgfsetstrokecolor{currentstroke}%
\pgfsetdash{}{0pt}%
\pgfsys@defobject{currentmarker}{\pgfqpoint{-0.066667in}{0.000000in}}{\pgfqpoint{0.000000in}{0.000000in}}{%
\pgfpathmoveto{\pgfqpoint{0.000000in}{0.000000in}}%
\pgfpathlineto{\pgfqpoint{-0.066667in}{0.000000in}}%
\pgfusepath{stroke,fill}%
}%
\begin{pgfscope}%
\pgfsys@transformshift{0.702340in}{2.502456in}%
\pgfsys@useobject{currentmarker}{}%
\end{pgfscope}%
\end{pgfscope}%
\begin{pgfscope}%
\definecolor{textcolor}{rgb}{0.150000,0.150000,0.150000}%
\pgfsetstrokecolor{textcolor}%
\pgfsetfillcolor{textcolor}%
\pgftext[x=0.413148in,y=2.477508in,left,base]{\color{textcolor}\sffamily\fontsize{5.176471}{6.211765}\selectfont 6.000}%
\end{pgfscope}%
\begin{pgfscope}%
\definecolor{textcolor}{rgb}{0.150000,0.150000,0.150000}%
\pgfsetstrokecolor{textcolor}%
\pgfsetfillcolor{textcolor}%
\pgftext[x=0.357592in,y=2.198482in,,bottom,rotate=90.000000]{\color{textcolor}\sffamily\fontsize{5.647059}{6.776471}\selectfont \(\displaystyle x = \frac{2 \mu E L^2}{4 \pi^2}\)}%
\end{pgfscope}%
\begin{pgfscope}%
\pgfpathrectangle{\pgfqpoint{0.702340in}{1.894508in}}{\pgfqpoint{1.223103in}{0.607948in}}%
\pgfusepath{clip}%
\pgfsetroundcap%
\pgfsetroundjoin%
\pgfsetlinewidth{1.204500pt}%
\definecolor{currentstroke}{rgb}{0.000000,0.501961,0.000000}%
\pgfsetstrokecolor{currentstroke}%
\pgfsetdash{}{0pt}%
\pgfpathmoveto{\pgfqpoint{0.702340in}{2.221710in}}%
\pgfpathlineto{\pgfqpoint{0.943050in}{2.220733in}}%
\pgfpathlineto{\pgfqpoint{1.054581in}{2.219477in}}%
\pgfpathlineto{\pgfqpoint{1.127977in}{2.217941in}}%
\pgfpathlineto{\pgfqpoint{1.182772in}{2.216127in}}%
\pgfpathlineto{\pgfqpoint{1.226515in}{2.214035in}}%
\pgfpathlineto{\pgfqpoint{1.262923in}{2.211665in}}%
\pgfpathlineto{\pgfqpoint{1.294107in}{2.209020in}}%
\pgfpathlineto{\pgfqpoint{1.321380in}{2.206098in}}%
\pgfpathlineto{\pgfqpoint{1.345614in}{2.202902in}}%
\pgfpathlineto{\pgfqpoint{1.367419in}{2.199432in}}%
\pgfpathlineto{\pgfqpoint{1.387238in}{2.195688in}}%
\pgfpathlineto{\pgfqpoint{1.405403in}{2.191672in}}%
\pgfpathlineto{\pgfqpoint{1.422168in}{2.187384in}}%
\pgfpathlineto{\pgfqpoint{1.437735in}{2.182826in}}%
\pgfpathlineto{\pgfqpoint{1.452262in}{2.177998in}}%
\pgfpathlineto{\pgfqpoint{1.465881in}{2.172901in}}%
\pgfpathlineto{\pgfqpoint{1.478698in}{2.167537in}}%
\pgfpathlineto{\pgfqpoint{1.490802in}{2.161906in}}%
\pgfpathlineto{\pgfqpoint{1.502269in}{2.156010in}}%
\pgfpathlineto{\pgfqpoint{1.513162in}{2.149849in}}%
\pgfpathlineto{\pgfqpoint{1.523536in}{2.143424in}}%
\pgfpathlineto{\pgfqpoint{1.533438in}{2.136738in}}%
\pgfpathlineto{\pgfqpoint{1.542909in}{2.129790in}}%
\pgfpathlineto{\pgfqpoint{1.551986in}{2.122583in}}%
\pgfpathlineto{\pgfqpoint{1.560699in}{2.115117in}}%
\pgfpathlineto{\pgfqpoint{1.569077in}{2.107394in}}%
\pgfpathlineto{\pgfqpoint{1.577145in}{2.099414in}}%
\pgfpathlineto{\pgfqpoint{1.584924in}{2.091180in}}%
\pgfpathlineto{\pgfqpoint{1.592435in}{2.082692in}}%
\pgfpathlineto{\pgfqpoint{1.599696in}{2.073951in}}%
\pgfpathlineto{\pgfqpoint{1.606722in}{2.064960in}}%
\pgfpathlineto{\pgfqpoint{1.613528in}{2.055720in}}%
\pgfpathlineto{\pgfqpoint{1.620129in}{2.046231in}}%
\pgfpathlineto{\pgfqpoint{1.626535in}{2.036495in}}%
\pgfpathlineto{\pgfqpoint{1.632758in}{2.026514in}}%
\pgfpathlineto{\pgfqpoint{1.638808in}{2.016289in}}%
\pgfpathlineto{\pgfqpoint{1.644694in}{2.005821in}}%
\pgfpathlineto{\pgfqpoint{1.650426in}{1.995113in}}%
\pgfpathlineto{\pgfqpoint{1.656011in}{1.984165in}}%
\pgfpathlineto{\pgfqpoint{1.661456in}{1.972979in}}%
\pgfpathlineto{\pgfqpoint{1.666768in}{1.961556in}}%
\pgfpathlineto{\pgfqpoint{1.671953in}{1.949899in}}%
\pgfpathlineto{\pgfqpoint{1.677018in}{1.938007in}}%
\pgfpathlineto{\pgfqpoint{1.681968in}{1.925884in}}%
\pgfpathlineto{\pgfqpoint{1.686808in}{1.913531in}}%
\pgfpathlineto{\pgfqpoint{1.691543in}{1.900949in}}%
\pgfpathlineto{\pgfqpoint{1.695079in}{1.891175in}}%
\pgfusepath{stroke}%
\end{pgfscope}%
\begin{pgfscope}%
\pgfsetrectcap%
\pgfsetmiterjoin%
\pgfsetlinewidth{1.003750pt}%
\definecolor{currentstroke}{rgb}{0.150000,0.150000,0.150000}%
\pgfsetstrokecolor{currentstroke}%
\pgfsetdash{}{0pt}%
\pgfpathmoveto{\pgfqpoint{0.702340in}{1.894508in}}%
\pgfpathlineto{\pgfqpoint{0.702340in}{2.502456in}}%
\pgfusepath{stroke}%
\end{pgfscope}%
\begin{pgfscope}%
\pgfsetrectcap%
\pgfsetmiterjoin%
\pgfsetlinewidth{1.003750pt}%
\definecolor{currentstroke}{rgb}{0.150000,0.150000,0.150000}%
\pgfsetstrokecolor{currentstroke}%
\pgfsetdash{}{0pt}%
\pgfpathmoveto{\pgfqpoint{0.702340in}{1.894508in}}%
\pgfpathlineto{\pgfqpoint{1.925444in}{1.894508in}}%
\pgfusepath{stroke}%
\end{pgfscope}%
\begin{pgfscope}%
\pgfpathrectangle{\pgfqpoint{0.702340in}{1.894508in}}{\pgfqpoint{1.223103in}{0.607948in}}%
\pgfusepath{clip}%
\pgfsetbuttcap%
\pgfsetroundjoin%
\definecolor{currentfill}{rgb}{0.000000,0.000000,0.000000}%
\pgfsetfillcolor{currentfill}%
\pgfsetlinewidth{1.003750pt}%
\definecolor{currentstroke}{rgb}{0.000000,0.000000,0.000000}%
\pgfsetstrokecolor{currentstroke}%
\pgfsetdash{}{0pt}%
\pgfsys@defobject{currentmarker}{\pgfqpoint{-0.013889in}{-0.013889in}}{\pgfqpoint{0.013889in}{0.013889in}}{%
\pgfpathmoveto{\pgfqpoint{0.000000in}{-0.013889in}}%
\pgfpathcurveto{\pgfqpoint{0.003683in}{-0.013889in}}{\pgfqpoint{0.007216in}{-0.012425in}}{\pgfqpoint{0.009821in}{-0.009821in}}%
\pgfpathcurveto{\pgfqpoint{0.012425in}{-0.007216in}}{\pgfqpoint{0.013889in}{-0.003683in}}{\pgfqpoint{0.013889in}{0.000000in}}%
\pgfpathcurveto{\pgfqpoint{0.013889in}{0.003683in}}{\pgfqpoint{0.012425in}{0.007216in}}{\pgfqpoint{0.009821in}{0.009821in}}%
\pgfpathcurveto{\pgfqpoint{0.007216in}{0.012425in}}{\pgfqpoint{0.003683in}{0.013889in}}{\pgfqpoint{0.000000in}{0.013889in}}%
\pgfpathcurveto{\pgfqpoint{-0.003683in}{0.013889in}}{\pgfqpoint{-0.007216in}{0.012425in}}{\pgfqpoint{-0.009821in}{0.009821in}}%
\pgfpathcurveto{\pgfqpoint{-0.012425in}{0.007216in}}{\pgfqpoint{-0.013889in}{0.003683in}}{\pgfqpoint{-0.013889in}{0.000000in}}%
\pgfpathcurveto{\pgfqpoint{-0.013889in}{-0.003683in}}{\pgfqpoint{-0.012425in}{-0.007216in}}{\pgfqpoint{-0.009821in}{-0.009821in}}%
\pgfpathcurveto{\pgfqpoint{-0.007216in}{-0.012425in}}{\pgfqpoint{-0.003683in}{-0.013889in}}{\pgfqpoint{0.000000in}{-0.013889in}}%
\pgfpathclose%
\pgfusepath{stroke,fill}%
}%
\begin{pgfscope}%
\pgfsys@transformshift{1.705158in}{1.861917in}%
\pgfsys@useobject{currentmarker}{}%
\end{pgfscope}%
\begin{pgfscope}%
\pgfsys@transformshift{1.616865in}{2.050828in}%
\pgfsys@useobject{currentmarker}{}%
\end{pgfscope}%
\begin{pgfscope}%
\pgfsys@transformshift{1.554219in}{2.120772in}%
\pgfsys@useobject{currentmarker}{}%
\end{pgfscope}%
\begin{pgfscope}%
\pgfsys@transformshift{1.505628in}{2.154289in}%
\pgfsys@useobject{currentmarker}{}%
\end{pgfscope}%
\begin{pgfscope}%
\pgfsys@transformshift{1.465926in}{2.172994in}%
\pgfsys@useobject{currentmarker}{}%
\end{pgfscope}%
\begin{pgfscope}%
\pgfsys@transformshift{1.432358in}{2.184543in}%
\pgfsys@useobject{currentmarker}{}%
\end{pgfscope}%
\begin{pgfscope}%
\pgfsys@transformshift{1.403280in}{2.192203in}%
\pgfsys@useobject{currentmarker}{}%
\end{pgfscope}%
\begin{pgfscope}%
\pgfsys@transformshift{1.397903in}{2.193425in}%
\pgfsys@useobject{currentmarker}{}%
\end{pgfscope}%
\begin{pgfscope}%
\pgfsys@transformshift{1.392656in}{2.194564in}%
\pgfsys@useobject{currentmarker}{}%
\end{pgfscope}%
\begin{pgfscope}%
\pgfsys@transformshift{1.387532in}{2.195630in}%
\pgfsys@useobject{currentmarker}{}%
\end{pgfscope}%
\begin{pgfscope}%
\pgfsys@transformshift{1.382525in}{2.196628in}%
\pgfsys@useobject{currentmarker}{}%
\end{pgfscope}%
\begin{pgfscope}%
\pgfsys@transformshift{1.377632in}{2.197563in}%
\pgfsys@useobject{currentmarker}{}%
\end{pgfscope}%
\begin{pgfscope}%
\pgfsys@transformshift{1.372846in}{2.198442in}%
\pgfsys@useobject{currentmarker}{}%
\end{pgfscope}%
\begin{pgfscope}%
\pgfsys@transformshift{1.368163in}{2.199269in}%
\pgfsys@useobject{currentmarker}{}%
\end{pgfscope}%
\begin{pgfscope}%
\pgfsys@transformshift{1.363578in}{2.200047in}%
\pgfsys@useobject{currentmarker}{}%
\end{pgfscope}%
\begin{pgfscope}%
\pgfsys@transformshift{1.359088in}{2.200781in}%
\pgfsys@useobject{currentmarker}{}%
\end{pgfscope}%
\begin{pgfscope}%
\pgfsys@transformshift{1.354689in}{2.201474in}%
\pgfsys@useobject{currentmarker}{}%
\end{pgfscope}%
\end{pgfscope}%
\begin{pgfscope}%
\pgfsetbuttcap%
\pgfsetmiterjoin%
\definecolor{currentfill}{rgb}{1.000000,1.000000,1.000000}%
\pgfsetfillcolor{currentfill}%
\pgfsetlinewidth{0.000000pt}%
\definecolor{currentstroke}{rgb}{0.000000,0.000000,0.000000}%
\pgfsetstrokecolor{currentstroke}%
\pgfsetstrokeopacity{0.000000}%
\pgfsetdash{}{0pt}%
\pgfpathmoveto{\pgfqpoint{2.170064in}{1.894508in}}%
\pgfpathlineto{\pgfqpoint{3.393168in}{1.894508in}}%
\pgfpathlineto{\pgfqpoint{3.393168in}{2.502456in}}%
\pgfpathlineto{\pgfqpoint{2.170064in}{2.502456in}}%
\pgfpathclose%
\pgfusepath{fill}%
\end{pgfscope}%
\begin{pgfscope}%
\pgfpathrectangle{\pgfqpoint{2.170064in}{1.894508in}}{\pgfqpoint{1.223103in}{0.607948in}}%
\pgfusepath{clip}%
\pgfsetbuttcap%
\pgfsetmiterjoin%
\definecolor{currentfill}{rgb}{0.000000,0.000000,1.000000}%
\pgfsetfillcolor{currentfill}%
\pgfsetfillopacity{0.100000}%
\pgfsetlinewidth{0.803000pt}%
\definecolor{currentstroke}{rgb}{0.000000,0.000000,1.000000}%
\pgfsetstrokecolor{currentstroke}%
\pgfsetstrokeopacity{0.100000}%
\pgfsetdash{}{0pt}%
\pgfpathmoveto{\pgfqpoint{2.170064in}{2.213162in}}%
\pgfpathlineto{\pgfqpoint{2.170064in}{2.224548in}}%
\pgfpathlineto{\pgfqpoint{3.393168in}{2.224548in}}%
\pgfpathlineto{\pgfqpoint{3.393168in}{2.213162in}}%
\pgfpathclose%
\pgfusepath{stroke,fill}%
\end{pgfscope}%
\begin{pgfscope}%
\pgfpathrectangle{\pgfqpoint{2.170064in}{1.894508in}}{\pgfqpoint{1.223103in}{0.607948in}}%
\pgfusepath{clip}%
\pgfsetbuttcap%
\pgfsetroundjoin%
\definecolor{currentfill}{rgb}{0.000000,0.501961,0.000000}%
\pgfsetfillcolor{currentfill}%
\pgfsetfillopacity{0.500000}%
\pgfsetlinewidth{0.803000pt}%
\definecolor{currentstroke}{rgb}{0.000000,0.501961,0.000000}%
\pgfsetstrokecolor{currentstroke}%
\pgfsetstrokeopacity{0.500000}%
\pgfsetdash{}{0pt}%
\pgfpathmoveto{\pgfqpoint{2.170064in}{2.224472in}}%
\pgfpathlineto{\pgfqpoint{2.170064in}{2.213874in}}%
\pgfpathlineto{\pgfqpoint{2.410774in}{2.215259in}}%
\pgfpathlineto{\pgfqpoint{2.522305in}{2.216571in}}%
\pgfpathlineto{\pgfqpoint{2.595701in}{2.217809in}}%
\pgfpathlineto{\pgfqpoint{2.650497in}{2.218973in}}%
\pgfpathlineto{\pgfqpoint{2.694239in}{2.220062in}}%
\pgfpathlineto{\pgfqpoint{2.730647in}{2.221074in}}%
\pgfpathlineto{\pgfqpoint{2.761831in}{2.222009in}}%
\pgfpathlineto{\pgfqpoint{2.789104in}{2.222866in}}%
\pgfpathlineto{\pgfqpoint{2.813338in}{2.223643in}}%
\pgfpathlineto{\pgfqpoint{2.835143in}{2.224340in}}%
\pgfpathlineto{\pgfqpoint{2.854962in}{2.224931in}}%
\pgfpathlineto{\pgfqpoint{2.873127in}{2.225062in}}%
\pgfpathlineto{\pgfqpoint{2.889892in}{2.225185in}}%
\pgfpathlineto{\pgfqpoint{2.905459in}{2.225295in}}%
\pgfpathlineto{\pgfqpoint{2.919986in}{2.225386in}}%
\pgfpathlineto{\pgfqpoint{2.933605in}{2.225453in}}%
\pgfpathlineto{\pgfqpoint{2.946422in}{2.225490in}}%
\pgfpathlineto{\pgfqpoint{2.958526in}{2.225491in}}%
\pgfpathlineto{\pgfqpoint{2.969993in}{2.225450in}}%
\pgfpathlineto{\pgfqpoint{2.980886in}{2.225361in}}%
\pgfpathlineto{\pgfqpoint{2.991260in}{2.225216in}}%
\pgfpathlineto{\pgfqpoint{3.001162in}{2.225011in}}%
\pgfpathlineto{\pgfqpoint{3.010633in}{2.224738in}}%
\pgfpathlineto{\pgfqpoint{3.019710in}{2.224390in}}%
\pgfpathlineto{\pgfqpoint{3.028423in}{2.223958in}}%
\pgfpathlineto{\pgfqpoint{3.036801in}{2.223318in}}%
\pgfpathlineto{\pgfqpoint{3.044869in}{2.222416in}}%
\pgfpathlineto{\pgfqpoint{3.052648in}{2.221400in}}%
\pgfpathlineto{\pgfqpoint{3.060159in}{2.220272in}}%
\pgfpathlineto{\pgfqpoint{3.067420in}{2.219031in}}%
\pgfpathlineto{\pgfqpoint{3.074446in}{2.217673in}}%
\pgfpathlineto{\pgfqpoint{3.081253in}{2.216198in}}%
\pgfpathlineto{\pgfqpoint{3.087853in}{2.214603in}}%
\pgfpathlineto{\pgfqpoint{3.094259in}{2.212887in}}%
\pgfpathlineto{\pgfqpoint{3.100482in}{2.211047in}}%
\pgfpathlineto{\pgfqpoint{3.106532in}{2.209082in}}%
\pgfpathlineto{\pgfqpoint{3.112419in}{2.206989in}}%
\pgfpathlineto{\pgfqpoint{3.118150in}{2.204766in}}%
\pgfpathlineto{\pgfqpoint{3.123735in}{2.202412in}}%
\pgfpathlineto{\pgfqpoint{3.129180in}{2.199924in}}%
\pgfpathlineto{\pgfqpoint{3.134492in}{2.197300in}}%
\pgfpathlineto{\pgfqpoint{3.139677in}{2.194538in}}%
\pgfpathlineto{\pgfqpoint{3.144742in}{2.191636in}}%
\pgfpathlineto{\pgfqpoint{3.149692in}{2.188592in}}%
\pgfpathlineto{\pgfqpoint{3.154532in}{2.185404in}}%
\pgfpathlineto{\pgfqpoint{3.159267in}{2.182070in}}%
\pgfpathlineto{\pgfqpoint{3.163901in}{2.178587in}}%
\pgfpathlineto{\pgfqpoint{3.168438in}{2.174953in}}%
\pgfpathlineto{\pgfqpoint{3.172883in}{2.170613in}}%
\pgfpathlineto{\pgfqpoint{3.172883in}{2.171170in}}%
\pgfpathlineto{\pgfqpoint{3.172883in}{2.171170in}}%
\pgfpathlineto{\pgfqpoint{3.168438in}{2.175190in}}%
\pgfpathlineto{\pgfqpoint{3.163901in}{2.179501in}}%
\pgfpathlineto{\pgfqpoint{3.159267in}{2.183559in}}%
\pgfpathlineto{\pgfqpoint{3.154532in}{2.187371in}}%
\pgfpathlineto{\pgfqpoint{3.149692in}{2.190944in}}%
\pgfpathlineto{\pgfqpoint{3.144742in}{2.194288in}}%
\pgfpathlineto{\pgfqpoint{3.139677in}{2.197409in}}%
\pgfpathlineto{\pgfqpoint{3.134492in}{2.200316in}}%
\pgfpathlineto{\pgfqpoint{3.129180in}{2.203016in}}%
\pgfpathlineto{\pgfqpoint{3.123735in}{2.205518in}}%
\pgfpathlineto{\pgfqpoint{3.118150in}{2.207829in}}%
\pgfpathlineto{\pgfqpoint{3.112419in}{2.209957in}}%
\pgfpathlineto{\pgfqpoint{3.106532in}{2.211910in}}%
\pgfpathlineto{\pgfqpoint{3.100482in}{2.213695in}}%
\pgfpathlineto{\pgfqpoint{3.094259in}{2.215320in}}%
\pgfpathlineto{\pgfqpoint{3.087853in}{2.216792in}}%
\pgfpathlineto{\pgfqpoint{3.081253in}{2.218120in}}%
\pgfpathlineto{\pgfqpoint{3.074446in}{2.219309in}}%
\pgfpathlineto{\pgfqpoint{3.067420in}{2.220369in}}%
\pgfpathlineto{\pgfqpoint{3.060159in}{2.221306in}}%
\pgfpathlineto{\pgfqpoint{3.052648in}{2.222127in}}%
\pgfpathlineto{\pgfqpoint{3.044869in}{2.222840in}}%
\pgfpathlineto{\pgfqpoint{3.036801in}{2.223455in}}%
\pgfpathlineto{\pgfqpoint{3.028423in}{2.224128in}}%
\pgfpathlineto{\pgfqpoint{3.019710in}{2.224817in}}%
\pgfpathlineto{\pgfqpoint{3.010633in}{2.225405in}}%
\pgfpathlineto{\pgfqpoint{3.001162in}{2.225890in}}%
\pgfpathlineto{\pgfqpoint{2.991260in}{2.226275in}}%
\pgfpathlineto{\pgfqpoint{2.980886in}{2.226561in}}%
\pgfpathlineto{\pgfqpoint{2.969993in}{2.226749in}}%
\pgfpathlineto{\pgfqpoint{2.958526in}{2.226842in}}%
\pgfpathlineto{\pgfqpoint{2.946422in}{2.226842in}}%
\pgfpathlineto{\pgfqpoint{2.933605in}{2.226749in}}%
\pgfpathlineto{\pgfqpoint{2.919986in}{2.226565in}}%
\pgfpathlineto{\pgfqpoint{2.905459in}{2.226292in}}%
\pgfpathlineto{\pgfqpoint{2.889892in}{2.225932in}}%
\pgfpathlineto{\pgfqpoint{2.873127in}{2.225485in}}%
\pgfpathlineto{\pgfqpoint{2.854962in}{2.224955in}}%
\pgfpathlineto{\pgfqpoint{2.835143in}{2.224800in}}%
\pgfpathlineto{\pgfqpoint{2.813338in}{2.224671in}}%
\pgfpathlineto{\pgfqpoint{2.789104in}{2.224551in}}%
\pgfpathlineto{\pgfqpoint{2.761831in}{2.224444in}}%
\pgfpathlineto{\pgfqpoint{2.730647in}{2.224356in}}%
\pgfpathlineto{\pgfqpoint{2.694239in}{2.224291in}}%
\pgfpathlineto{\pgfqpoint{2.650497in}{2.224254in}}%
\pgfpathlineto{\pgfqpoint{2.595701in}{2.224249in}}%
\pgfpathlineto{\pgfqpoint{2.522305in}{2.224281in}}%
\pgfpathlineto{\pgfqpoint{2.410774in}{2.224354in}}%
\pgfpathlineto{\pgfqpoint{2.170064in}{2.224472in}}%
\pgfpathclose%
\pgfusepath{stroke,fill}%
\end{pgfscope}%
\begin{pgfscope}%
\pgfpathrectangle{\pgfqpoint{2.170064in}{1.894508in}}{\pgfqpoint{1.223103in}{0.607948in}}%
\pgfusepath{clip}%
\pgfsetroundcap%
\pgfsetroundjoin%
\pgfsetlinewidth{0.501875pt}%
\definecolor{currentstroke}{rgb}{0.000000,0.000000,1.000000}%
\pgfsetstrokecolor{currentstroke}%
\pgfsetstrokeopacity{0.800000}%
\pgfsetdash{}{0pt}%
\pgfpathmoveto{\pgfqpoint{2.170064in}{2.218855in}}%
\pgfpathlineto{\pgfqpoint{3.393168in}{2.218855in}}%
\pgfusepath{stroke}%
\end{pgfscope}%
\begin{pgfscope}%
\pgfpathrectangle{\pgfqpoint{2.170064in}{1.894508in}}{\pgfqpoint{1.223103in}{0.607948in}}%
\pgfusepath{clip}%
\pgfsetbuttcap%
\pgfsetroundjoin%
\pgfsetlinewidth{1.003750pt}%
\definecolor{currentstroke}{rgb}{0.000000,0.000000,0.000000}%
\pgfsetstrokecolor{currentstroke}%
\pgfsetdash{{3.700000pt}{1.600000pt}}{0.000000pt}%
\pgfpathmoveto{\pgfqpoint{2.170064in}{2.221402in}}%
\pgfpathlineto{\pgfqpoint{3.393168in}{2.221402in}}%
\pgfusepath{stroke}%
\end{pgfscope}%
\begin{pgfscope}%
\pgfsetroundcap%
\pgfsetroundjoin%
\pgfsetlinewidth{0.501875pt}%
\definecolor{currentstroke}{rgb}{0.000000,0.000000,1.000000}%
\pgfsetstrokecolor{currentstroke}%
\pgfsetstrokeopacity{0.800000}%
\pgfsetdash{}{0pt}%
\pgfpathmoveto{\pgfqpoint{2.983913in}{2.336192in}}%
\pgfpathquadraticcurveto{\pgfqpoint{2.914395in}{2.285684in}}{\pgfqpoint{2.844877in}{2.235176in}}%
\pgfusepath{stroke}%
\end{pgfscope}%
\begin{pgfscope}%
\pgfsetfillopacity{0.800000}%
\pgfsetstrokeopacity{0.800000}%
\definecolor{textcolor}{rgb}{0.000000,0.000000,1.000000}%
\pgfsetstrokecolor{textcolor}%
\pgfsetfillcolor{textcolor}%
\pgftext[x=2.910706in,y=2.401239in,left,base]{\color{textcolor}\sffamily\fontsize{5.647059}{6.776471}\selectfont 5.5335(94)}%
\end{pgfscope}%
\begin{pgfscope}%
\pgfsetbuttcap%
\pgfsetroundjoin%
\definecolor{currentfill}{rgb}{0.150000,0.150000,0.150000}%
\pgfsetfillcolor{currentfill}%
\pgfsetlinewidth{1.003750pt}%
\definecolor{currentstroke}{rgb}{0.150000,0.150000,0.150000}%
\pgfsetstrokecolor{currentstroke}%
\pgfsetdash{}{0pt}%
\pgfsys@defobject{currentmarker}{\pgfqpoint{0.000000in}{-0.066667in}}{\pgfqpoint{0.000000in}{0.000000in}}{%
\pgfpathmoveto{\pgfqpoint{0.000000in}{0.000000in}}%
\pgfpathlineto{\pgfqpoint{0.000000in}{-0.066667in}}%
\pgfusepath{stroke,fill}%
}%
\begin{pgfscope}%
\pgfsys@transformshift{2.170064in}{1.894508in}%
\pgfsys@useobject{currentmarker}{}%
\end{pgfscope}%
\end{pgfscope}%
\begin{pgfscope}%
\pgfsetbuttcap%
\pgfsetroundjoin%
\definecolor{currentfill}{rgb}{0.150000,0.150000,0.150000}%
\pgfsetfillcolor{currentfill}%
\pgfsetlinewidth{1.003750pt}%
\definecolor{currentstroke}{rgb}{0.150000,0.150000,0.150000}%
\pgfsetstrokecolor{currentstroke}%
\pgfsetdash{}{0pt}%
\pgfsys@defobject{currentmarker}{\pgfqpoint{0.000000in}{-0.066667in}}{\pgfqpoint{0.000000in}{0.000000in}}{%
\pgfpathmoveto{\pgfqpoint{0.000000in}{0.000000in}}%
\pgfpathlineto{\pgfqpoint{0.000000in}{-0.066667in}}%
\pgfusepath{stroke,fill}%
}%
\begin{pgfscope}%
\pgfsys@transformshift{2.671473in}{1.894508in}%
\pgfsys@useobject{currentmarker}{}%
\end{pgfscope}%
\end{pgfscope}%
\begin{pgfscope}%
\pgfsetbuttcap%
\pgfsetroundjoin%
\definecolor{currentfill}{rgb}{0.150000,0.150000,0.150000}%
\pgfsetfillcolor{currentfill}%
\pgfsetlinewidth{1.003750pt}%
\definecolor{currentstroke}{rgb}{0.150000,0.150000,0.150000}%
\pgfsetstrokecolor{currentstroke}%
\pgfsetdash{}{0pt}%
\pgfsys@defobject{currentmarker}{\pgfqpoint{0.000000in}{-0.066667in}}{\pgfqpoint{0.000000in}{0.000000in}}{%
\pgfpathmoveto{\pgfqpoint{0.000000in}{0.000000in}}%
\pgfpathlineto{\pgfqpoint{0.000000in}{-0.066667in}}%
\pgfusepath{stroke,fill}%
}%
\begin{pgfscope}%
\pgfsys@transformshift{3.172883in}{1.894508in}%
\pgfsys@useobject{currentmarker}{}%
\end{pgfscope}%
\end{pgfscope}%
\begin{pgfscope}%
\pgfsetbuttcap%
\pgfsetroundjoin%
\definecolor{currentfill}{rgb}{0.150000,0.150000,0.150000}%
\pgfsetfillcolor{currentfill}%
\pgfsetlinewidth{0.803000pt}%
\definecolor{currentstroke}{rgb}{0.150000,0.150000,0.150000}%
\pgfsetstrokecolor{currentstroke}%
\pgfsetdash{}{0pt}%
\pgfsys@defobject{currentmarker}{\pgfqpoint{0.000000in}{-0.044444in}}{\pgfqpoint{0.000000in}{0.000000in}}{%
\pgfpathmoveto{\pgfqpoint{0.000000in}{0.000000in}}%
\pgfpathlineto{\pgfqpoint{0.000000in}{-0.044444in}}%
\pgfusepath{stroke,fill}%
}%
\begin{pgfscope}%
\pgfsys@transformshift{2.321004in}{1.894508in}%
\pgfsys@useobject{currentmarker}{}%
\end{pgfscope}%
\end{pgfscope}%
\begin{pgfscope}%
\pgfsetbuttcap%
\pgfsetroundjoin%
\definecolor{currentfill}{rgb}{0.150000,0.150000,0.150000}%
\pgfsetfillcolor{currentfill}%
\pgfsetlinewidth{0.803000pt}%
\definecolor{currentstroke}{rgb}{0.150000,0.150000,0.150000}%
\pgfsetstrokecolor{currentstroke}%
\pgfsetdash{}{0pt}%
\pgfsys@defobject{currentmarker}{\pgfqpoint{0.000000in}{-0.044444in}}{\pgfqpoint{0.000000in}{0.000000in}}{%
\pgfpathmoveto{\pgfqpoint{0.000000in}{0.000000in}}%
\pgfpathlineto{\pgfqpoint{0.000000in}{-0.044444in}}%
\pgfusepath{stroke,fill}%
}%
\begin{pgfscope}%
\pgfsys@transformshift{2.409297in}{1.894508in}%
\pgfsys@useobject{currentmarker}{}%
\end{pgfscope}%
\end{pgfscope}%
\begin{pgfscope}%
\pgfsetbuttcap%
\pgfsetroundjoin%
\definecolor{currentfill}{rgb}{0.150000,0.150000,0.150000}%
\pgfsetfillcolor{currentfill}%
\pgfsetlinewidth{0.803000pt}%
\definecolor{currentstroke}{rgb}{0.150000,0.150000,0.150000}%
\pgfsetstrokecolor{currentstroke}%
\pgfsetdash{}{0pt}%
\pgfsys@defobject{currentmarker}{\pgfqpoint{0.000000in}{-0.044444in}}{\pgfqpoint{0.000000in}{0.000000in}}{%
\pgfpathmoveto{\pgfqpoint{0.000000in}{0.000000in}}%
\pgfpathlineto{\pgfqpoint{0.000000in}{-0.044444in}}%
\pgfusepath{stroke,fill}%
}%
\begin{pgfscope}%
\pgfsys@transformshift{2.471943in}{1.894508in}%
\pgfsys@useobject{currentmarker}{}%
\end{pgfscope}%
\end{pgfscope}%
\begin{pgfscope}%
\pgfsetbuttcap%
\pgfsetroundjoin%
\definecolor{currentfill}{rgb}{0.150000,0.150000,0.150000}%
\pgfsetfillcolor{currentfill}%
\pgfsetlinewidth{0.803000pt}%
\definecolor{currentstroke}{rgb}{0.150000,0.150000,0.150000}%
\pgfsetstrokecolor{currentstroke}%
\pgfsetdash{}{0pt}%
\pgfsys@defobject{currentmarker}{\pgfqpoint{0.000000in}{-0.044444in}}{\pgfqpoint{0.000000in}{0.000000in}}{%
\pgfpathmoveto{\pgfqpoint{0.000000in}{0.000000in}}%
\pgfpathlineto{\pgfqpoint{0.000000in}{-0.044444in}}%
\pgfusepath{stroke,fill}%
}%
\begin{pgfscope}%
\pgfsys@transformshift{2.520534in}{1.894508in}%
\pgfsys@useobject{currentmarker}{}%
\end{pgfscope}%
\end{pgfscope}%
\begin{pgfscope}%
\pgfsetbuttcap%
\pgfsetroundjoin%
\definecolor{currentfill}{rgb}{0.150000,0.150000,0.150000}%
\pgfsetfillcolor{currentfill}%
\pgfsetlinewidth{0.803000pt}%
\definecolor{currentstroke}{rgb}{0.150000,0.150000,0.150000}%
\pgfsetstrokecolor{currentstroke}%
\pgfsetdash{}{0pt}%
\pgfsys@defobject{currentmarker}{\pgfqpoint{0.000000in}{-0.044444in}}{\pgfqpoint{0.000000in}{0.000000in}}{%
\pgfpathmoveto{\pgfqpoint{0.000000in}{0.000000in}}%
\pgfpathlineto{\pgfqpoint{0.000000in}{-0.044444in}}%
\pgfusepath{stroke,fill}%
}%
\begin{pgfscope}%
\pgfsys@transformshift{2.560237in}{1.894508in}%
\pgfsys@useobject{currentmarker}{}%
\end{pgfscope}%
\end{pgfscope}%
\begin{pgfscope}%
\pgfsetbuttcap%
\pgfsetroundjoin%
\definecolor{currentfill}{rgb}{0.150000,0.150000,0.150000}%
\pgfsetfillcolor{currentfill}%
\pgfsetlinewidth{0.803000pt}%
\definecolor{currentstroke}{rgb}{0.150000,0.150000,0.150000}%
\pgfsetstrokecolor{currentstroke}%
\pgfsetdash{}{0pt}%
\pgfsys@defobject{currentmarker}{\pgfqpoint{0.000000in}{-0.044444in}}{\pgfqpoint{0.000000in}{0.000000in}}{%
\pgfpathmoveto{\pgfqpoint{0.000000in}{0.000000in}}%
\pgfpathlineto{\pgfqpoint{0.000000in}{-0.044444in}}%
\pgfusepath{stroke,fill}%
}%
\begin{pgfscope}%
\pgfsys@transformshift{2.593804in}{1.894508in}%
\pgfsys@useobject{currentmarker}{}%
\end{pgfscope}%
\end{pgfscope}%
\begin{pgfscope}%
\pgfsetbuttcap%
\pgfsetroundjoin%
\definecolor{currentfill}{rgb}{0.150000,0.150000,0.150000}%
\pgfsetfillcolor{currentfill}%
\pgfsetlinewidth{0.803000pt}%
\definecolor{currentstroke}{rgb}{0.150000,0.150000,0.150000}%
\pgfsetstrokecolor{currentstroke}%
\pgfsetdash{}{0pt}%
\pgfsys@defobject{currentmarker}{\pgfqpoint{0.000000in}{-0.044444in}}{\pgfqpoint{0.000000in}{0.000000in}}{%
\pgfpathmoveto{\pgfqpoint{0.000000in}{0.000000in}}%
\pgfpathlineto{\pgfqpoint{0.000000in}{-0.044444in}}%
\pgfusepath{stroke,fill}%
}%
\begin{pgfscope}%
\pgfsys@transformshift{2.622882in}{1.894508in}%
\pgfsys@useobject{currentmarker}{}%
\end{pgfscope}%
\end{pgfscope}%
\begin{pgfscope}%
\pgfsetbuttcap%
\pgfsetroundjoin%
\definecolor{currentfill}{rgb}{0.150000,0.150000,0.150000}%
\pgfsetfillcolor{currentfill}%
\pgfsetlinewidth{0.803000pt}%
\definecolor{currentstroke}{rgb}{0.150000,0.150000,0.150000}%
\pgfsetstrokecolor{currentstroke}%
\pgfsetdash{}{0pt}%
\pgfsys@defobject{currentmarker}{\pgfqpoint{0.000000in}{-0.044444in}}{\pgfqpoint{0.000000in}{0.000000in}}{%
\pgfpathmoveto{\pgfqpoint{0.000000in}{0.000000in}}%
\pgfpathlineto{\pgfqpoint{0.000000in}{-0.044444in}}%
\pgfusepath{stroke,fill}%
}%
\begin{pgfscope}%
\pgfsys@transformshift{2.648530in}{1.894508in}%
\pgfsys@useobject{currentmarker}{}%
\end{pgfscope}%
\end{pgfscope}%
\begin{pgfscope}%
\pgfsetbuttcap%
\pgfsetroundjoin%
\definecolor{currentfill}{rgb}{0.150000,0.150000,0.150000}%
\pgfsetfillcolor{currentfill}%
\pgfsetlinewidth{0.803000pt}%
\definecolor{currentstroke}{rgb}{0.150000,0.150000,0.150000}%
\pgfsetstrokecolor{currentstroke}%
\pgfsetdash{}{0pt}%
\pgfsys@defobject{currentmarker}{\pgfqpoint{0.000000in}{-0.044444in}}{\pgfqpoint{0.000000in}{0.000000in}}{%
\pgfpathmoveto{\pgfqpoint{0.000000in}{0.000000in}}%
\pgfpathlineto{\pgfqpoint{0.000000in}{-0.044444in}}%
\pgfusepath{stroke,fill}%
}%
\begin{pgfscope}%
\pgfsys@transformshift{2.822413in}{1.894508in}%
\pgfsys@useobject{currentmarker}{}%
\end{pgfscope}%
\end{pgfscope}%
\begin{pgfscope}%
\pgfsetbuttcap%
\pgfsetroundjoin%
\definecolor{currentfill}{rgb}{0.150000,0.150000,0.150000}%
\pgfsetfillcolor{currentfill}%
\pgfsetlinewidth{0.803000pt}%
\definecolor{currentstroke}{rgb}{0.150000,0.150000,0.150000}%
\pgfsetstrokecolor{currentstroke}%
\pgfsetdash{}{0pt}%
\pgfsys@defobject{currentmarker}{\pgfqpoint{0.000000in}{-0.044444in}}{\pgfqpoint{0.000000in}{0.000000in}}{%
\pgfpathmoveto{\pgfqpoint{0.000000in}{0.000000in}}%
\pgfpathlineto{\pgfqpoint{0.000000in}{-0.044444in}}%
\pgfusepath{stroke,fill}%
}%
\begin{pgfscope}%
\pgfsys@transformshift{2.910706in}{1.894508in}%
\pgfsys@useobject{currentmarker}{}%
\end{pgfscope}%
\end{pgfscope}%
\begin{pgfscope}%
\pgfsetbuttcap%
\pgfsetroundjoin%
\definecolor{currentfill}{rgb}{0.150000,0.150000,0.150000}%
\pgfsetfillcolor{currentfill}%
\pgfsetlinewidth{0.803000pt}%
\definecolor{currentstroke}{rgb}{0.150000,0.150000,0.150000}%
\pgfsetstrokecolor{currentstroke}%
\pgfsetdash{}{0pt}%
\pgfsys@defobject{currentmarker}{\pgfqpoint{0.000000in}{-0.044444in}}{\pgfqpoint{0.000000in}{0.000000in}}{%
\pgfpathmoveto{\pgfqpoint{0.000000in}{0.000000in}}%
\pgfpathlineto{\pgfqpoint{0.000000in}{-0.044444in}}%
\pgfusepath{stroke,fill}%
}%
\begin{pgfscope}%
\pgfsys@transformshift{2.973352in}{1.894508in}%
\pgfsys@useobject{currentmarker}{}%
\end{pgfscope}%
\end{pgfscope}%
\begin{pgfscope}%
\pgfsetbuttcap%
\pgfsetroundjoin%
\definecolor{currentfill}{rgb}{0.150000,0.150000,0.150000}%
\pgfsetfillcolor{currentfill}%
\pgfsetlinewidth{0.803000pt}%
\definecolor{currentstroke}{rgb}{0.150000,0.150000,0.150000}%
\pgfsetstrokecolor{currentstroke}%
\pgfsetdash{}{0pt}%
\pgfsys@defobject{currentmarker}{\pgfqpoint{0.000000in}{-0.044444in}}{\pgfqpoint{0.000000in}{0.000000in}}{%
\pgfpathmoveto{\pgfqpoint{0.000000in}{0.000000in}}%
\pgfpathlineto{\pgfqpoint{0.000000in}{-0.044444in}}%
\pgfusepath{stroke,fill}%
}%
\begin{pgfscope}%
\pgfsys@transformshift{3.021943in}{1.894508in}%
\pgfsys@useobject{currentmarker}{}%
\end{pgfscope}%
\end{pgfscope}%
\begin{pgfscope}%
\pgfsetbuttcap%
\pgfsetroundjoin%
\definecolor{currentfill}{rgb}{0.150000,0.150000,0.150000}%
\pgfsetfillcolor{currentfill}%
\pgfsetlinewidth{0.803000pt}%
\definecolor{currentstroke}{rgb}{0.150000,0.150000,0.150000}%
\pgfsetstrokecolor{currentstroke}%
\pgfsetdash{}{0pt}%
\pgfsys@defobject{currentmarker}{\pgfqpoint{0.000000in}{-0.044444in}}{\pgfqpoint{0.000000in}{0.000000in}}{%
\pgfpathmoveto{\pgfqpoint{0.000000in}{0.000000in}}%
\pgfpathlineto{\pgfqpoint{0.000000in}{-0.044444in}}%
\pgfusepath{stroke,fill}%
}%
\begin{pgfscope}%
\pgfsys@transformshift{3.061646in}{1.894508in}%
\pgfsys@useobject{currentmarker}{}%
\end{pgfscope}%
\end{pgfscope}%
\begin{pgfscope}%
\pgfsetbuttcap%
\pgfsetroundjoin%
\definecolor{currentfill}{rgb}{0.150000,0.150000,0.150000}%
\pgfsetfillcolor{currentfill}%
\pgfsetlinewidth{0.803000pt}%
\definecolor{currentstroke}{rgb}{0.150000,0.150000,0.150000}%
\pgfsetstrokecolor{currentstroke}%
\pgfsetdash{}{0pt}%
\pgfsys@defobject{currentmarker}{\pgfqpoint{0.000000in}{-0.044444in}}{\pgfqpoint{0.000000in}{0.000000in}}{%
\pgfpathmoveto{\pgfqpoint{0.000000in}{0.000000in}}%
\pgfpathlineto{\pgfqpoint{0.000000in}{-0.044444in}}%
\pgfusepath{stroke,fill}%
}%
\begin{pgfscope}%
\pgfsys@transformshift{3.095213in}{1.894508in}%
\pgfsys@useobject{currentmarker}{}%
\end{pgfscope}%
\end{pgfscope}%
\begin{pgfscope}%
\pgfsetbuttcap%
\pgfsetroundjoin%
\definecolor{currentfill}{rgb}{0.150000,0.150000,0.150000}%
\pgfsetfillcolor{currentfill}%
\pgfsetlinewidth{0.803000pt}%
\definecolor{currentstroke}{rgb}{0.150000,0.150000,0.150000}%
\pgfsetstrokecolor{currentstroke}%
\pgfsetdash{}{0pt}%
\pgfsys@defobject{currentmarker}{\pgfqpoint{0.000000in}{-0.044444in}}{\pgfqpoint{0.000000in}{0.000000in}}{%
\pgfpathmoveto{\pgfqpoint{0.000000in}{0.000000in}}%
\pgfpathlineto{\pgfqpoint{0.000000in}{-0.044444in}}%
\pgfusepath{stroke,fill}%
}%
\begin{pgfscope}%
\pgfsys@transformshift{3.124291in}{1.894508in}%
\pgfsys@useobject{currentmarker}{}%
\end{pgfscope}%
\end{pgfscope}%
\begin{pgfscope}%
\pgfsetbuttcap%
\pgfsetroundjoin%
\definecolor{currentfill}{rgb}{0.150000,0.150000,0.150000}%
\pgfsetfillcolor{currentfill}%
\pgfsetlinewidth{0.803000pt}%
\definecolor{currentstroke}{rgb}{0.150000,0.150000,0.150000}%
\pgfsetstrokecolor{currentstroke}%
\pgfsetdash{}{0pt}%
\pgfsys@defobject{currentmarker}{\pgfqpoint{0.000000in}{-0.044444in}}{\pgfqpoint{0.000000in}{0.000000in}}{%
\pgfpathmoveto{\pgfqpoint{0.000000in}{0.000000in}}%
\pgfpathlineto{\pgfqpoint{0.000000in}{-0.044444in}}%
\pgfusepath{stroke,fill}%
}%
\begin{pgfscope}%
\pgfsys@transformshift{3.149939in}{1.894508in}%
\pgfsys@useobject{currentmarker}{}%
\end{pgfscope}%
\end{pgfscope}%
\begin{pgfscope}%
\pgfsetbuttcap%
\pgfsetroundjoin%
\definecolor{currentfill}{rgb}{0.150000,0.150000,0.150000}%
\pgfsetfillcolor{currentfill}%
\pgfsetlinewidth{0.803000pt}%
\definecolor{currentstroke}{rgb}{0.150000,0.150000,0.150000}%
\pgfsetstrokecolor{currentstroke}%
\pgfsetdash{}{0pt}%
\pgfsys@defobject{currentmarker}{\pgfqpoint{0.000000in}{-0.044444in}}{\pgfqpoint{0.000000in}{0.000000in}}{%
\pgfpathmoveto{\pgfqpoint{0.000000in}{0.000000in}}%
\pgfpathlineto{\pgfqpoint{0.000000in}{-0.044444in}}%
\pgfusepath{stroke,fill}%
}%
\begin{pgfscope}%
\pgfsys@transformshift{3.323822in}{1.894508in}%
\pgfsys@useobject{currentmarker}{}%
\end{pgfscope}%
\end{pgfscope}%
\begin{pgfscope}%
\pgfsetbuttcap%
\pgfsetroundjoin%
\definecolor{currentfill}{rgb}{0.150000,0.150000,0.150000}%
\pgfsetfillcolor{currentfill}%
\pgfsetlinewidth{1.003750pt}%
\definecolor{currentstroke}{rgb}{0.150000,0.150000,0.150000}%
\pgfsetstrokecolor{currentstroke}%
\pgfsetdash{}{0pt}%
\pgfsys@defobject{currentmarker}{\pgfqpoint{-0.066667in}{0.000000in}}{\pgfqpoint{0.000000in}{0.000000in}}{%
\pgfpathmoveto{\pgfqpoint{0.000000in}{0.000000in}}%
\pgfpathlineto{\pgfqpoint{-0.066667in}{0.000000in}}%
\pgfusepath{stroke,fill}%
}%
\begin{pgfscope}%
\pgfsys@transformshift{2.170064in}{1.894508in}%
\pgfsys@useobject{currentmarker}{}%
\end{pgfscope}%
\end{pgfscope}%
\begin{pgfscope}%
\pgfsetbuttcap%
\pgfsetroundjoin%
\definecolor{currentfill}{rgb}{0.150000,0.150000,0.150000}%
\pgfsetfillcolor{currentfill}%
\pgfsetlinewidth{1.003750pt}%
\definecolor{currentstroke}{rgb}{0.150000,0.150000,0.150000}%
\pgfsetstrokecolor{currentstroke}%
\pgfsetdash{}{0pt}%
\pgfsys@defobject{currentmarker}{\pgfqpoint{-0.066667in}{0.000000in}}{\pgfqpoint{0.000000in}{0.000000in}}{%
\pgfpathmoveto{\pgfqpoint{0.000000in}{0.000000in}}%
\pgfpathlineto{\pgfqpoint{-0.066667in}{0.000000in}}%
\pgfusepath{stroke,fill}%
}%
\begin{pgfscope}%
\pgfsys@transformshift{2.170064in}{2.221402in}%
\pgfsys@useobject{currentmarker}{}%
\end{pgfscope}%
\end{pgfscope}%
\begin{pgfscope}%
\pgfsetbuttcap%
\pgfsetroundjoin%
\definecolor{currentfill}{rgb}{0.150000,0.150000,0.150000}%
\pgfsetfillcolor{currentfill}%
\pgfsetlinewidth{1.003750pt}%
\definecolor{currentstroke}{rgb}{0.150000,0.150000,0.150000}%
\pgfsetstrokecolor{currentstroke}%
\pgfsetdash{}{0pt}%
\pgfsys@defobject{currentmarker}{\pgfqpoint{-0.066667in}{0.000000in}}{\pgfqpoint{0.000000in}{0.000000in}}{%
\pgfpathmoveto{\pgfqpoint{0.000000in}{0.000000in}}%
\pgfpathlineto{\pgfqpoint{-0.066667in}{0.000000in}}%
\pgfusepath{stroke,fill}%
}%
\begin{pgfscope}%
\pgfsys@transformshift{2.170064in}{2.502456in}%
\pgfsys@useobject{currentmarker}{}%
\end{pgfscope}%
\end{pgfscope}%
\begin{pgfscope}%
\pgfpathrectangle{\pgfqpoint{2.170064in}{1.894508in}}{\pgfqpoint{1.223103in}{0.607948in}}%
\pgfusepath{clip}%
\pgfsetroundcap%
\pgfsetroundjoin%
\pgfsetlinewidth{1.204500pt}%
\definecolor{currentstroke}{rgb}{0.000000,0.501961,0.000000}%
\pgfsetstrokecolor{currentstroke}%
\pgfsetdash{}{0pt}%
\pgfpathmoveto{\pgfqpoint{2.170064in}{2.219173in}}%
\pgfpathlineto{\pgfqpoint{2.410774in}{2.219806in}}%
\pgfpathlineto{\pgfqpoint{2.522305in}{2.220426in}}%
\pgfpathlineto{\pgfqpoint{2.595701in}{2.221029in}}%
\pgfpathlineto{\pgfqpoint{2.650497in}{2.221614in}}%
\pgfpathlineto{\pgfqpoint{2.694239in}{2.222176in}}%
\pgfpathlineto{\pgfqpoint{2.730647in}{2.222715in}}%
\pgfpathlineto{\pgfqpoint{2.761831in}{2.223227in}}%
\pgfpathlineto{\pgfqpoint{2.789104in}{2.223709in}}%
\pgfpathlineto{\pgfqpoint{2.813338in}{2.224157in}}%
\pgfpathlineto{\pgfqpoint{2.835143in}{2.224570in}}%
\pgfpathlineto{\pgfqpoint{2.854962in}{2.224943in}}%
\pgfpathlineto{\pgfqpoint{2.873127in}{2.225274in}}%
\pgfpathlineto{\pgfqpoint{2.889892in}{2.225558in}}%
\pgfpathlineto{\pgfqpoint{2.905459in}{2.225793in}}%
\pgfpathlineto{\pgfqpoint{2.919986in}{2.225976in}}%
\pgfpathlineto{\pgfqpoint{2.933605in}{2.226101in}}%
\pgfpathlineto{\pgfqpoint{2.946422in}{2.226166in}}%
\pgfpathlineto{\pgfqpoint{2.958526in}{2.226167in}}%
\pgfpathlineto{\pgfqpoint{2.969993in}{2.226100in}}%
\pgfpathlineto{\pgfqpoint{2.980886in}{2.225961in}}%
\pgfpathlineto{\pgfqpoint{2.991260in}{2.225746in}}%
\pgfpathlineto{\pgfqpoint{3.001162in}{2.225450in}}%
\pgfpathlineto{\pgfqpoint{3.010633in}{2.225071in}}%
\pgfpathlineto{\pgfqpoint{3.019710in}{2.224604in}}%
\pgfpathlineto{\pgfqpoint{3.028423in}{2.224043in}}%
\pgfpathlineto{\pgfqpoint{3.036801in}{2.223386in}}%
\pgfpathlineto{\pgfqpoint{3.044869in}{2.222628in}}%
\pgfpathlineto{\pgfqpoint{3.052648in}{2.221763in}}%
\pgfpathlineto{\pgfqpoint{3.060159in}{2.220789in}}%
\pgfpathlineto{\pgfqpoint{3.067420in}{2.219700in}}%
\pgfpathlineto{\pgfqpoint{3.074446in}{2.218491in}}%
\pgfpathlineto{\pgfqpoint{3.081253in}{2.217159in}}%
\pgfpathlineto{\pgfqpoint{3.087853in}{2.215698in}}%
\pgfpathlineto{\pgfqpoint{3.094259in}{2.214104in}}%
\pgfpathlineto{\pgfqpoint{3.100482in}{2.212371in}}%
\pgfpathlineto{\pgfqpoint{3.106532in}{2.210496in}}%
\pgfpathlineto{\pgfqpoint{3.112419in}{2.208473in}}%
\pgfpathlineto{\pgfqpoint{3.118150in}{2.206298in}}%
\pgfpathlineto{\pgfqpoint{3.123735in}{2.203965in}}%
\pgfpathlineto{\pgfqpoint{3.129180in}{2.201470in}}%
\pgfpathlineto{\pgfqpoint{3.134492in}{2.198808in}}%
\pgfpathlineto{\pgfqpoint{3.139677in}{2.195973in}}%
\pgfpathlineto{\pgfqpoint{3.144742in}{2.192962in}}%
\pgfpathlineto{\pgfqpoint{3.149692in}{2.189768in}}%
\pgfpathlineto{\pgfqpoint{3.154532in}{2.186387in}}%
\pgfpathlineto{\pgfqpoint{3.159267in}{2.182814in}}%
\pgfpathlineto{\pgfqpoint{3.163901in}{2.179044in}}%
\pgfpathlineto{\pgfqpoint{3.168438in}{2.175071in}}%
\pgfpathlineto{\pgfqpoint{3.172883in}{2.170891in}}%
\pgfusepath{stroke}%
\end{pgfscope}%
\begin{pgfscope}%
\pgfsetrectcap%
\pgfsetmiterjoin%
\pgfsetlinewidth{1.003750pt}%
\definecolor{currentstroke}{rgb}{0.150000,0.150000,0.150000}%
\pgfsetstrokecolor{currentstroke}%
\pgfsetdash{}{0pt}%
\pgfpathmoveto{\pgfqpoint{2.170064in}{1.894508in}}%
\pgfpathlineto{\pgfqpoint{2.170064in}{2.502456in}}%
\pgfusepath{stroke}%
\end{pgfscope}%
\begin{pgfscope}%
\pgfsetrectcap%
\pgfsetmiterjoin%
\pgfsetlinewidth{1.003750pt}%
\definecolor{currentstroke}{rgb}{0.150000,0.150000,0.150000}%
\pgfsetstrokecolor{currentstroke}%
\pgfsetdash{}{0pt}%
\pgfpathmoveto{\pgfqpoint{2.170064in}{1.894508in}}%
\pgfpathlineto{\pgfqpoint{3.393168in}{1.894508in}}%
\pgfusepath{stroke}%
\end{pgfscope}%
\begin{pgfscope}%
\pgfpathrectangle{\pgfqpoint{2.170064in}{1.894508in}}{\pgfqpoint{1.223103in}{0.607948in}}%
\pgfusepath{clip}%
\pgfsetbuttcap%
\pgfsetroundjoin%
\definecolor{currentfill}{rgb}{0.000000,0.000000,0.000000}%
\pgfsetfillcolor{currentfill}%
\pgfsetlinewidth{1.003750pt}%
\definecolor{currentstroke}{rgb}{0.000000,0.000000,0.000000}%
\pgfsetstrokecolor{currentstroke}%
\pgfsetdash{}{0pt}%
\pgfsys@defobject{currentmarker}{\pgfqpoint{-0.013889in}{-0.013889in}}{\pgfqpoint{0.013889in}{0.013889in}}{%
\pgfpathmoveto{\pgfqpoint{0.000000in}{-0.013889in}}%
\pgfpathcurveto{\pgfqpoint{0.003683in}{-0.013889in}}{\pgfqpoint{0.007216in}{-0.012425in}}{\pgfqpoint{0.009821in}{-0.009821in}}%
\pgfpathcurveto{\pgfqpoint{0.012425in}{-0.007216in}}{\pgfqpoint{0.013889in}{-0.003683in}}{\pgfqpoint{0.013889in}{0.000000in}}%
\pgfpathcurveto{\pgfqpoint{0.013889in}{0.003683in}}{\pgfqpoint{0.012425in}{0.007216in}}{\pgfqpoint{0.009821in}{0.009821in}}%
\pgfpathcurveto{\pgfqpoint{0.007216in}{0.012425in}}{\pgfqpoint{0.003683in}{0.013889in}}{\pgfqpoint{0.000000in}{0.013889in}}%
\pgfpathcurveto{\pgfqpoint{-0.003683in}{0.013889in}}{\pgfqpoint{-0.007216in}{0.012425in}}{\pgfqpoint{-0.009821in}{0.009821in}}%
\pgfpathcurveto{\pgfqpoint{-0.012425in}{0.007216in}}{\pgfqpoint{-0.013889in}{0.003683in}}{\pgfqpoint{-0.013889in}{0.000000in}}%
\pgfpathcurveto{\pgfqpoint{-0.013889in}{-0.003683in}}{\pgfqpoint{-0.012425in}{-0.007216in}}{\pgfqpoint{-0.009821in}{-0.009821in}}%
\pgfpathcurveto{\pgfqpoint{-0.007216in}{-0.012425in}}{\pgfqpoint{-0.003683in}{-0.013889in}}{\pgfqpoint{0.000000in}{-0.013889in}}%
\pgfpathclose%
\pgfusepath{stroke,fill}%
}%
\begin{pgfscope}%
\pgfsys@transformshift{3.172883in}{2.170729in}%
\pgfsys@useobject{currentmarker}{}%
\end{pgfscope}%
\begin{pgfscope}%
\pgfsys@transformshift{3.084589in}{2.217045in}%
\pgfsys@useobject{currentmarker}{}%
\end{pgfscope}%
\begin{pgfscope}%
\pgfsys@transformshift{3.021943in}{2.224346in}%
\pgfsys@useobject{currentmarker}{}%
\end{pgfscope}%
\begin{pgfscope}%
\pgfsys@transformshift{2.822413in}{2.224543in}%
\pgfsys@useobject{currentmarker}{}%
\end{pgfscope}%
\begin{pgfscope}%
\pgfsys@transformshift{2.826812in}{2.224596in}%
\pgfsys@useobject{currentmarker}{}%
\end{pgfscope}%
\begin{pgfscope}%
\pgfsys@transformshift{2.831302in}{2.224651in}%
\pgfsys@useobject{currentmarker}{}%
\end{pgfscope}%
\begin{pgfscope}%
\pgfsys@transformshift{2.835887in}{2.224707in}%
\pgfsys@useobject{currentmarker}{}%
\end{pgfscope}%
\begin{pgfscope}%
\pgfsys@transformshift{2.840570in}{2.224765in}%
\pgfsys@useobject{currentmarker}{}%
\end{pgfscope}%
\begin{pgfscope}%
\pgfsys@transformshift{2.845356in}{2.224824in}%
\pgfsys@useobject{currentmarker}{}%
\end{pgfscope}%
\begin{pgfscope}%
\pgfsys@transformshift{2.850250in}{2.224884in}%
\pgfsys@useobject{currentmarker}{}%
\end{pgfscope}%
\begin{pgfscope}%
\pgfsys@transformshift{2.855256in}{2.224946in}%
\pgfsys@useobject{currentmarker}{}%
\end{pgfscope}%
\begin{pgfscope}%
\pgfsys@transformshift{2.860380in}{2.225008in}%
\pgfsys@useobject{currentmarker}{}%
\end{pgfscope}%
\begin{pgfscope}%
\pgfsys@transformshift{2.865627in}{2.225072in}%
\pgfsys@useobject{currentmarker}{}%
\end{pgfscope}%
\begin{pgfscope}%
\pgfsys@transformshift{2.871004in}{2.225137in}%
\pgfsys@useobject{currentmarker}{}%
\end{pgfscope}%
\begin{pgfscope}%
\pgfsys@transformshift{2.900082in}{2.225464in}%
\pgfsys@useobject{currentmarker}{}%
\end{pgfscope}%
\begin{pgfscope}%
\pgfsys@transformshift{2.973352in}{2.225687in}%
\pgfsys@useobject{currentmarker}{}%
\end{pgfscope}%
\begin{pgfscope}%
\pgfsys@transformshift{2.933650in}{2.225732in}%
\pgfsys@useobject{currentmarker}{}%
\end{pgfscope}%
\end{pgfscope}%
\begin{pgfscope}%
\pgfsetbuttcap%
\pgfsetmiterjoin%
\definecolor{currentfill}{rgb}{1.000000,1.000000,1.000000}%
\pgfsetfillcolor{currentfill}%
\pgfsetlinewidth{0.000000pt}%
\definecolor{currentstroke}{rgb}{0.000000,0.000000,0.000000}%
\pgfsetstrokecolor{currentstroke}%
\pgfsetstrokeopacity{0.000000}%
\pgfsetdash{}{0pt}%
\pgfpathmoveto{\pgfqpoint{3.637789in}{1.894508in}}%
\pgfpathlineto{\pgfqpoint{4.860892in}{1.894508in}}%
\pgfpathlineto{\pgfqpoint{4.860892in}{2.502456in}}%
\pgfpathlineto{\pgfqpoint{3.637789in}{2.502456in}}%
\pgfpathclose%
\pgfusepath{fill}%
\end{pgfscope}%
\begin{pgfscope}%
\pgfpathrectangle{\pgfqpoint{3.637789in}{1.894508in}}{\pgfqpoint{1.223103in}{0.607948in}}%
\pgfusepath{clip}%
\pgfsetbuttcap%
\pgfsetmiterjoin%
\definecolor{currentfill}{rgb}{0.000000,0.000000,1.000000}%
\pgfsetfillcolor{currentfill}%
\pgfsetfillopacity{0.100000}%
\pgfsetlinewidth{0.803000pt}%
\definecolor{currentstroke}{rgb}{0.000000,0.000000,1.000000}%
\pgfsetstrokecolor{currentstroke}%
\pgfsetstrokeopacity{0.100000}%
\pgfsetdash{}{0pt}%
\pgfpathmoveto{\pgfqpoint{3.637789in}{2.220314in}}%
\pgfpathlineto{\pgfqpoint{3.637789in}{2.222036in}}%
\pgfpathlineto{\pgfqpoint{4.860892in}{2.222036in}}%
\pgfpathlineto{\pgfqpoint{4.860892in}{2.220314in}}%
\pgfpathclose%
\pgfusepath{stroke,fill}%
\end{pgfscope}%
\begin{pgfscope}%
\pgfpathrectangle{\pgfqpoint{3.637789in}{1.894508in}}{\pgfqpoint{1.223103in}{0.607948in}}%
\pgfusepath{clip}%
\pgfsetbuttcap%
\pgfsetroundjoin%
\definecolor{currentfill}{rgb}{0.000000,0.501961,0.000000}%
\pgfsetfillcolor{currentfill}%
\pgfsetfillopacity{0.500000}%
\pgfsetlinewidth{0.803000pt}%
\definecolor{currentstroke}{rgb}{0.000000,0.501961,0.000000}%
\pgfsetstrokecolor{currentstroke}%
\pgfsetstrokeopacity{0.500000}%
\pgfsetdash{}{0pt}%
\pgfpathmoveto{\pgfqpoint{3.637789in}{2.222266in}}%
\pgfpathlineto{\pgfqpoint{3.637789in}{2.220666in}}%
\pgfpathlineto{\pgfqpoint{3.878498in}{2.221370in}}%
\pgfpathlineto{\pgfqpoint{3.990029in}{2.222062in}}%
\pgfpathlineto{\pgfqpoint{4.063425in}{2.222744in}}%
\pgfpathlineto{\pgfqpoint{4.118221in}{2.223415in}}%
\pgfpathlineto{\pgfqpoint{4.161963in}{2.224077in}}%
\pgfpathlineto{\pgfqpoint{4.198371in}{2.224730in}}%
\pgfpathlineto{\pgfqpoint{4.229555in}{2.225374in}}%
\pgfpathlineto{\pgfqpoint{4.256828in}{2.226010in}}%
\pgfpathlineto{\pgfqpoint{4.281062in}{2.226638in}}%
\pgfpathlineto{\pgfqpoint{4.302867in}{2.227257in}}%
\pgfpathlineto{\pgfqpoint{4.322686in}{2.227864in}}%
\pgfpathlineto{\pgfqpoint{4.340851in}{2.228408in}}%
\pgfpathlineto{\pgfqpoint{4.357617in}{2.228953in}}%
\pgfpathlineto{\pgfqpoint{4.373183in}{2.229501in}}%
\pgfpathlineto{\pgfqpoint{4.387711in}{2.230051in}}%
\pgfpathlineto{\pgfqpoint{4.401329in}{2.230603in}}%
\pgfpathlineto{\pgfqpoint{4.414146in}{2.231157in}}%
\pgfpathlineto{\pgfqpoint{4.426250in}{2.231711in}}%
\pgfpathlineto{\pgfqpoint{4.437717in}{2.232265in}}%
\pgfpathlineto{\pgfqpoint{4.448610in}{2.232820in}}%
\pgfpathlineto{\pgfqpoint{4.458984in}{2.233373in}}%
\pgfpathlineto{\pgfqpoint{4.468886in}{2.233925in}}%
\pgfpathlineto{\pgfqpoint{4.478357in}{2.234474in}}%
\pgfpathlineto{\pgfqpoint{4.487434in}{2.235019in}}%
\pgfpathlineto{\pgfqpoint{4.496147in}{2.235559in}}%
\pgfpathlineto{\pgfqpoint{4.504525in}{2.236085in}}%
\pgfpathlineto{\pgfqpoint{4.512593in}{2.236586in}}%
\pgfpathlineto{\pgfqpoint{4.520372in}{2.237064in}}%
\pgfpathlineto{\pgfqpoint{4.527883in}{2.237529in}}%
\pgfpathlineto{\pgfqpoint{4.535144in}{2.237981in}}%
\pgfpathlineto{\pgfqpoint{4.542170in}{2.238422in}}%
\pgfpathlineto{\pgfqpoint{4.548977in}{2.238851in}}%
\pgfpathlineto{\pgfqpoint{4.555577in}{2.239268in}}%
\pgfpathlineto{\pgfqpoint{4.561983in}{2.239672in}}%
\pgfpathlineto{\pgfqpoint{4.568206in}{2.240064in}}%
\pgfpathlineto{\pgfqpoint{4.574256in}{2.240442in}}%
\pgfpathlineto{\pgfqpoint{4.580143in}{2.240806in}}%
\pgfpathlineto{\pgfqpoint{4.585874in}{2.241156in}}%
\pgfpathlineto{\pgfqpoint{4.591459in}{2.241492in}}%
\pgfpathlineto{\pgfqpoint{4.596904in}{2.241812in}}%
\pgfpathlineto{\pgfqpoint{4.602216in}{2.242116in}}%
\pgfpathlineto{\pgfqpoint{4.607401in}{2.242404in}}%
\pgfpathlineto{\pgfqpoint{4.612466in}{2.242675in}}%
\pgfpathlineto{\pgfqpoint{4.617416in}{2.242930in}}%
\pgfpathlineto{\pgfqpoint{4.622256in}{2.243166in}}%
\pgfpathlineto{\pgfqpoint{4.626991in}{2.243384in}}%
\pgfpathlineto{\pgfqpoint{4.631625in}{2.243584in}}%
\pgfpathlineto{\pgfqpoint{4.636162in}{2.243760in}}%
\pgfpathlineto{\pgfqpoint{4.640607in}{2.243800in}}%
\pgfpathlineto{\pgfqpoint{4.640607in}{2.243935in}}%
\pgfpathlineto{\pgfqpoint{4.640607in}{2.243935in}}%
\pgfpathlineto{\pgfqpoint{4.636162in}{2.243854in}}%
\pgfpathlineto{\pgfqpoint{4.631625in}{2.243835in}}%
\pgfpathlineto{\pgfqpoint{4.626991in}{2.243768in}}%
\pgfpathlineto{\pgfqpoint{4.622256in}{2.243653in}}%
\pgfpathlineto{\pgfqpoint{4.617416in}{2.243495in}}%
\pgfpathlineto{\pgfqpoint{4.612466in}{2.243295in}}%
\pgfpathlineto{\pgfqpoint{4.607401in}{2.243058in}}%
\pgfpathlineto{\pgfqpoint{4.602216in}{2.242786in}}%
\pgfpathlineto{\pgfqpoint{4.596904in}{2.242482in}}%
\pgfpathlineto{\pgfqpoint{4.591459in}{2.242149in}}%
\pgfpathlineto{\pgfqpoint{4.585874in}{2.241788in}}%
\pgfpathlineto{\pgfqpoint{4.580143in}{2.241404in}}%
\pgfpathlineto{\pgfqpoint{4.574256in}{2.240997in}}%
\pgfpathlineto{\pgfqpoint{4.568206in}{2.240571in}}%
\pgfpathlineto{\pgfqpoint{4.561983in}{2.240126in}}%
\pgfpathlineto{\pgfqpoint{4.555577in}{2.239665in}}%
\pgfpathlineto{\pgfqpoint{4.548977in}{2.239190in}}%
\pgfpathlineto{\pgfqpoint{4.542170in}{2.238701in}}%
\pgfpathlineto{\pgfqpoint{4.535144in}{2.238202in}}%
\pgfpathlineto{\pgfqpoint{4.527883in}{2.237693in}}%
\pgfpathlineto{\pgfqpoint{4.520372in}{2.237176in}}%
\pgfpathlineto{\pgfqpoint{4.512593in}{2.236655in}}%
\pgfpathlineto{\pgfqpoint{4.504525in}{2.236140in}}%
\pgfpathlineto{\pgfqpoint{4.496147in}{2.235636in}}%
\pgfpathlineto{\pgfqpoint{4.487434in}{2.235131in}}%
\pgfpathlineto{\pgfqpoint{4.478357in}{2.234619in}}%
\pgfpathlineto{\pgfqpoint{4.468886in}{2.234099in}}%
\pgfpathlineto{\pgfqpoint{4.458984in}{2.233571in}}%
\pgfpathlineto{\pgfqpoint{4.448610in}{2.233035in}}%
\pgfpathlineto{\pgfqpoint{4.437717in}{2.232492in}}%
\pgfpathlineto{\pgfqpoint{4.426250in}{2.231940in}}%
\pgfpathlineto{\pgfqpoint{4.414146in}{2.231381in}}%
\pgfpathlineto{\pgfqpoint{4.401329in}{2.230815in}}%
\pgfpathlineto{\pgfqpoint{4.387711in}{2.230240in}}%
\pgfpathlineto{\pgfqpoint{4.373183in}{2.229659in}}%
\pgfpathlineto{\pgfqpoint{4.357617in}{2.229070in}}%
\pgfpathlineto{\pgfqpoint{4.340851in}{2.228473in}}%
\pgfpathlineto{\pgfqpoint{4.322686in}{2.227872in}}%
\pgfpathlineto{\pgfqpoint{4.302867in}{2.227329in}}%
\pgfpathlineto{\pgfqpoint{4.281062in}{2.226795in}}%
\pgfpathlineto{\pgfqpoint{4.256828in}{2.226266in}}%
\pgfpathlineto{\pgfqpoint{4.229555in}{2.225742in}}%
\pgfpathlineto{\pgfqpoint{4.198371in}{2.225224in}}%
\pgfpathlineto{\pgfqpoint{4.161963in}{2.224712in}}%
\pgfpathlineto{\pgfqpoint{4.118221in}{2.224207in}}%
\pgfpathlineto{\pgfqpoint{4.063425in}{2.223709in}}%
\pgfpathlineto{\pgfqpoint{3.990029in}{2.223219in}}%
\pgfpathlineto{\pgfqpoint{3.878498in}{2.222738in}}%
\pgfpathlineto{\pgfqpoint{3.637789in}{2.222266in}}%
\pgfpathclose%
\pgfusepath{stroke,fill}%
\end{pgfscope}%
\begin{pgfscope}%
\pgfpathrectangle{\pgfqpoint{3.637789in}{1.894508in}}{\pgfqpoint{1.223103in}{0.607948in}}%
\pgfusepath{clip}%
\pgfsetroundcap%
\pgfsetroundjoin%
\pgfsetlinewidth{0.501875pt}%
\definecolor{currentstroke}{rgb}{0.000000,0.000000,1.000000}%
\pgfsetstrokecolor{currentstroke}%
\pgfsetstrokeopacity{0.800000}%
\pgfsetdash{}{0pt}%
\pgfpathmoveto{\pgfqpoint{3.637789in}{2.221175in}}%
\pgfpathlineto{\pgfqpoint{4.860892in}{2.221175in}}%
\pgfusepath{stroke}%
\end{pgfscope}%
\begin{pgfscope}%
\pgfpathrectangle{\pgfqpoint{3.637789in}{1.894508in}}{\pgfqpoint{1.223103in}{0.607948in}}%
\pgfusepath{clip}%
\pgfsetbuttcap%
\pgfsetroundjoin%
\pgfsetlinewidth{1.003750pt}%
\definecolor{currentstroke}{rgb}{0.000000,0.000000,0.000000}%
\pgfsetstrokecolor{currentstroke}%
\pgfsetdash{{3.700000pt}{1.600000pt}}{0.000000pt}%
\pgfpathmoveto{\pgfqpoint{3.637789in}{2.221402in}}%
\pgfpathlineto{\pgfqpoint{4.860892in}{2.221402in}}%
\pgfusepath{stroke}%
\end{pgfscope}%
\begin{pgfscope}%
\pgfsetroundcap%
\pgfsetroundjoin%
\pgfsetlinewidth{0.501875pt}%
\definecolor{currentstroke}{rgb}{0.000000,0.000000,1.000000}%
\pgfsetstrokecolor{currentstroke}%
\pgfsetstrokeopacity{0.800000}%
\pgfsetdash{}{0pt}%
\pgfpathmoveto{\pgfqpoint{4.451637in}{2.338512in}}%
\pgfpathquadraticcurveto{\pgfqpoint{4.382119in}{2.288004in}}{\pgfqpoint{4.312601in}{2.237497in}}%
\pgfusepath{stroke}%
\end{pgfscope}%
\begin{pgfscope}%
\pgfsetfillopacity{0.800000}%
\pgfsetstrokeopacity{0.800000}%
\definecolor{textcolor}{rgb}{0.000000,0.000000,1.000000}%
\pgfsetstrokecolor{textcolor}%
\pgfsetfillcolor{textcolor}%
\pgftext[x=4.378430in,y=2.403560in,left,base]{\color{textcolor}\sffamily\fontsize{5.647059}{6.776471}\selectfont 5.5373(14)}%
\end{pgfscope}%
\begin{pgfscope}%
\pgfsetbuttcap%
\pgfsetroundjoin%
\definecolor{currentfill}{rgb}{0.150000,0.150000,0.150000}%
\pgfsetfillcolor{currentfill}%
\pgfsetlinewidth{1.003750pt}%
\definecolor{currentstroke}{rgb}{0.150000,0.150000,0.150000}%
\pgfsetstrokecolor{currentstroke}%
\pgfsetdash{}{0pt}%
\pgfsys@defobject{currentmarker}{\pgfqpoint{0.000000in}{-0.066667in}}{\pgfqpoint{0.000000in}{0.000000in}}{%
\pgfpathmoveto{\pgfqpoint{0.000000in}{0.000000in}}%
\pgfpathlineto{\pgfqpoint{0.000000in}{-0.066667in}}%
\pgfusepath{stroke,fill}%
}%
\begin{pgfscope}%
\pgfsys@transformshift{3.637789in}{1.894508in}%
\pgfsys@useobject{currentmarker}{}%
\end{pgfscope}%
\end{pgfscope}%
\begin{pgfscope}%
\pgfsetbuttcap%
\pgfsetroundjoin%
\definecolor{currentfill}{rgb}{0.150000,0.150000,0.150000}%
\pgfsetfillcolor{currentfill}%
\pgfsetlinewidth{1.003750pt}%
\definecolor{currentstroke}{rgb}{0.150000,0.150000,0.150000}%
\pgfsetstrokecolor{currentstroke}%
\pgfsetdash{}{0pt}%
\pgfsys@defobject{currentmarker}{\pgfqpoint{0.000000in}{-0.066667in}}{\pgfqpoint{0.000000in}{0.000000in}}{%
\pgfpathmoveto{\pgfqpoint{0.000000in}{0.000000in}}%
\pgfpathlineto{\pgfqpoint{0.000000in}{-0.066667in}}%
\pgfusepath{stroke,fill}%
}%
\begin{pgfscope}%
\pgfsys@transformshift{4.139198in}{1.894508in}%
\pgfsys@useobject{currentmarker}{}%
\end{pgfscope}%
\end{pgfscope}%
\begin{pgfscope}%
\pgfsetbuttcap%
\pgfsetroundjoin%
\definecolor{currentfill}{rgb}{0.150000,0.150000,0.150000}%
\pgfsetfillcolor{currentfill}%
\pgfsetlinewidth{1.003750pt}%
\definecolor{currentstroke}{rgb}{0.150000,0.150000,0.150000}%
\pgfsetstrokecolor{currentstroke}%
\pgfsetdash{}{0pt}%
\pgfsys@defobject{currentmarker}{\pgfqpoint{0.000000in}{-0.066667in}}{\pgfqpoint{0.000000in}{0.000000in}}{%
\pgfpathmoveto{\pgfqpoint{0.000000in}{0.000000in}}%
\pgfpathlineto{\pgfqpoint{0.000000in}{-0.066667in}}%
\pgfusepath{stroke,fill}%
}%
\begin{pgfscope}%
\pgfsys@transformshift{4.640607in}{1.894508in}%
\pgfsys@useobject{currentmarker}{}%
\end{pgfscope}%
\end{pgfscope}%
\begin{pgfscope}%
\pgfsetbuttcap%
\pgfsetroundjoin%
\definecolor{currentfill}{rgb}{0.150000,0.150000,0.150000}%
\pgfsetfillcolor{currentfill}%
\pgfsetlinewidth{0.803000pt}%
\definecolor{currentstroke}{rgb}{0.150000,0.150000,0.150000}%
\pgfsetstrokecolor{currentstroke}%
\pgfsetdash{}{0pt}%
\pgfsys@defobject{currentmarker}{\pgfqpoint{0.000000in}{-0.044444in}}{\pgfqpoint{0.000000in}{0.000000in}}{%
\pgfpathmoveto{\pgfqpoint{0.000000in}{0.000000in}}%
\pgfpathlineto{\pgfqpoint{0.000000in}{-0.044444in}}%
\pgfusepath{stroke,fill}%
}%
\begin{pgfscope}%
\pgfsys@transformshift{3.788728in}{1.894508in}%
\pgfsys@useobject{currentmarker}{}%
\end{pgfscope}%
\end{pgfscope}%
\begin{pgfscope}%
\pgfsetbuttcap%
\pgfsetroundjoin%
\definecolor{currentfill}{rgb}{0.150000,0.150000,0.150000}%
\pgfsetfillcolor{currentfill}%
\pgfsetlinewidth{0.803000pt}%
\definecolor{currentstroke}{rgb}{0.150000,0.150000,0.150000}%
\pgfsetstrokecolor{currentstroke}%
\pgfsetdash{}{0pt}%
\pgfsys@defobject{currentmarker}{\pgfqpoint{0.000000in}{-0.044444in}}{\pgfqpoint{0.000000in}{0.000000in}}{%
\pgfpathmoveto{\pgfqpoint{0.000000in}{0.000000in}}%
\pgfpathlineto{\pgfqpoint{0.000000in}{-0.044444in}}%
\pgfusepath{stroke,fill}%
}%
\begin{pgfscope}%
\pgfsys@transformshift{3.877021in}{1.894508in}%
\pgfsys@useobject{currentmarker}{}%
\end{pgfscope}%
\end{pgfscope}%
\begin{pgfscope}%
\pgfsetbuttcap%
\pgfsetroundjoin%
\definecolor{currentfill}{rgb}{0.150000,0.150000,0.150000}%
\pgfsetfillcolor{currentfill}%
\pgfsetlinewidth{0.803000pt}%
\definecolor{currentstroke}{rgb}{0.150000,0.150000,0.150000}%
\pgfsetstrokecolor{currentstroke}%
\pgfsetdash{}{0pt}%
\pgfsys@defobject{currentmarker}{\pgfqpoint{0.000000in}{-0.044444in}}{\pgfqpoint{0.000000in}{0.000000in}}{%
\pgfpathmoveto{\pgfqpoint{0.000000in}{0.000000in}}%
\pgfpathlineto{\pgfqpoint{0.000000in}{-0.044444in}}%
\pgfusepath{stroke,fill}%
}%
\begin{pgfscope}%
\pgfsys@transformshift{3.939667in}{1.894508in}%
\pgfsys@useobject{currentmarker}{}%
\end{pgfscope}%
\end{pgfscope}%
\begin{pgfscope}%
\pgfsetbuttcap%
\pgfsetroundjoin%
\definecolor{currentfill}{rgb}{0.150000,0.150000,0.150000}%
\pgfsetfillcolor{currentfill}%
\pgfsetlinewidth{0.803000pt}%
\definecolor{currentstroke}{rgb}{0.150000,0.150000,0.150000}%
\pgfsetstrokecolor{currentstroke}%
\pgfsetdash{}{0pt}%
\pgfsys@defobject{currentmarker}{\pgfqpoint{0.000000in}{-0.044444in}}{\pgfqpoint{0.000000in}{0.000000in}}{%
\pgfpathmoveto{\pgfqpoint{0.000000in}{0.000000in}}%
\pgfpathlineto{\pgfqpoint{0.000000in}{-0.044444in}}%
\pgfusepath{stroke,fill}%
}%
\begin{pgfscope}%
\pgfsys@transformshift{3.988258in}{1.894508in}%
\pgfsys@useobject{currentmarker}{}%
\end{pgfscope}%
\end{pgfscope}%
\begin{pgfscope}%
\pgfsetbuttcap%
\pgfsetroundjoin%
\definecolor{currentfill}{rgb}{0.150000,0.150000,0.150000}%
\pgfsetfillcolor{currentfill}%
\pgfsetlinewidth{0.803000pt}%
\definecolor{currentstroke}{rgb}{0.150000,0.150000,0.150000}%
\pgfsetstrokecolor{currentstroke}%
\pgfsetdash{}{0pt}%
\pgfsys@defobject{currentmarker}{\pgfqpoint{0.000000in}{-0.044444in}}{\pgfqpoint{0.000000in}{0.000000in}}{%
\pgfpathmoveto{\pgfqpoint{0.000000in}{0.000000in}}%
\pgfpathlineto{\pgfqpoint{0.000000in}{-0.044444in}}%
\pgfusepath{stroke,fill}%
}%
\begin{pgfscope}%
\pgfsys@transformshift{4.027961in}{1.894508in}%
\pgfsys@useobject{currentmarker}{}%
\end{pgfscope}%
\end{pgfscope}%
\begin{pgfscope}%
\pgfsetbuttcap%
\pgfsetroundjoin%
\definecolor{currentfill}{rgb}{0.150000,0.150000,0.150000}%
\pgfsetfillcolor{currentfill}%
\pgfsetlinewidth{0.803000pt}%
\definecolor{currentstroke}{rgb}{0.150000,0.150000,0.150000}%
\pgfsetstrokecolor{currentstroke}%
\pgfsetdash{}{0pt}%
\pgfsys@defobject{currentmarker}{\pgfqpoint{0.000000in}{-0.044444in}}{\pgfqpoint{0.000000in}{0.000000in}}{%
\pgfpathmoveto{\pgfqpoint{0.000000in}{0.000000in}}%
\pgfpathlineto{\pgfqpoint{0.000000in}{-0.044444in}}%
\pgfusepath{stroke,fill}%
}%
\begin{pgfscope}%
\pgfsys@transformshift{4.061528in}{1.894508in}%
\pgfsys@useobject{currentmarker}{}%
\end{pgfscope}%
\end{pgfscope}%
\begin{pgfscope}%
\pgfsetbuttcap%
\pgfsetroundjoin%
\definecolor{currentfill}{rgb}{0.150000,0.150000,0.150000}%
\pgfsetfillcolor{currentfill}%
\pgfsetlinewidth{0.803000pt}%
\definecolor{currentstroke}{rgb}{0.150000,0.150000,0.150000}%
\pgfsetstrokecolor{currentstroke}%
\pgfsetdash{}{0pt}%
\pgfsys@defobject{currentmarker}{\pgfqpoint{0.000000in}{-0.044444in}}{\pgfqpoint{0.000000in}{0.000000in}}{%
\pgfpathmoveto{\pgfqpoint{0.000000in}{0.000000in}}%
\pgfpathlineto{\pgfqpoint{0.000000in}{-0.044444in}}%
\pgfusepath{stroke,fill}%
}%
\begin{pgfscope}%
\pgfsys@transformshift{4.090606in}{1.894508in}%
\pgfsys@useobject{currentmarker}{}%
\end{pgfscope}%
\end{pgfscope}%
\begin{pgfscope}%
\pgfsetbuttcap%
\pgfsetroundjoin%
\definecolor{currentfill}{rgb}{0.150000,0.150000,0.150000}%
\pgfsetfillcolor{currentfill}%
\pgfsetlinewidth{0.803000pt}%
\definecolor{currentstroke}{rgb}{0.150000,0.150000,0.150000}%
\pgfsetstrokecolor{currentstroke}%
\pgfsetdash{}{0pt}%
\pgfsys@defobject{currentmarker}{\pgfqpoint{0.000000in}{-0.044444in}}{\pgfqpoint{0.000000in}{0.000000in}}{%
\pgfpathmoveto{\pgfqpoint{0.000000in}{0.000000in}}%
\pgfpathlineto{\pgfqpoint{0.000000in}{-0.044444in}}%
\pgfusepath{stroke,fill}%
}%
\begin{pgfscope}%
\pgfsys@transformshift{4.116254in}{1.894508in}%
\pgfsys@useobject{currentmarker}{}%
\end{pgfscope}%
\end{pgfscope}%
\begin{pgfscope}%
\pgfsetbuttcap%
\pgfsetroundjoin%
\definecolor{currentfill}{rgb}{0.150000,0.150000,0.150000}%
\pgfsetfillcolor{currentfill}%
\pgfsetlinewidth{0.803000pt}%
\definecolor{currentstroke}{rgb}{0.150000,0.150000,0.150000}%
\pgfsetstrokecolor{currentstroke}%
\pgfsetdash{}{0pt}%
\pgfsys@defobject{currentmarker}{\pgfqpoint{0.000000in}{-0.044444in}}{\pgfqpoint{0.000000in}{0.000000in}}{%
\pgfpathmoveto{\pgfqpoint{0.000000in}{0.000000in}}%
\pgfpathlineto{\pgfqpoint{0.000000in}{-0.044444in}}%
\pgfusepath{stroke,fill}%
}%
\begin{pgfscope}%
\pgfsys@transformshift{4.290137in}{1.894508in}%
\pgfsys@useobject{currentmarker}{}%
\end{pgfscope}%
\end{pgfscope}%
\begin{pgfscope}%
\pgfsetbuttcap%
\pgfsetroundjoin%
\definecolor{currentfill}{rgb}{0.150000,0.150000,0.150000}%
\pgfsetfillcolor{currentfill}%
\pgfsetlinewidth{0.803000pt}%
\definecolor{currentstroke}{rgb}{0.150000,0.150000,0.150000}%
\pgfsetstrokecolor{currentstroke}%
\pgfsetdash{}{0pt}%
\pgfsys@defobject{currentmarker}{\pgfqpoint{0.000000in}{-0.044444in}}{\pgfqpoint{0.000000in}{0.000000in}}{%
\pgfpathmoveto{\pgfqpoint{0.000000in}{0.000000in}}%
\pgfpathlineto{\pgfqpoint{0.000000in}{-0.044444in}}%
\pgfusepath{stroke,fill}%
}%
\begin{pgfscope}%
\pgfsys@transformshift{4.378430in}{1.894508in}%
\pgfsys@useobject{currentmarker}{}%
\end{pgfscope}%
\end{pgfscope}%
\begin{pgfscope}%
\pgfsetbuttcap%
\pgfsetroundjoin%
\definecolor{currentfill}{rgb}{0.150000,0.150000,0.150000}%
\pgfsetfillcolor{currentfill}%
\pgfsetlinewidth{0.803000pt}%
\definecolor{currentstroke}{rgb}{0.150000,0.150000,0.150000}%
\pgfsetstrokecolor{currentstroke}%
\pgfsetdash{}{0pt}%
\pgfsys@defobject{currentmarker}{\pgfqpoint{0.000000in}{-0.044444in}}{\pgfqpoint{0.000000in}{0.000000in}}{%
\pgfpathmoveto{\pgfqpoint{0.000000in}{0.000000in}}%
\pgfpathlineto{\pgfqpoint{0.000000in}{-0.044444in}}%
\pgfusepath{stroke,fill}%
}%
\begin{pgfscope}%
\pgfsys@transformshift{4.441076in}{1.894508in}%
\pgfsys@useobject{currentmarker}{}%
\end{pgfscope}%
\end{pgfscope}%
\begin{pgfscope}%
\pgfsetbuttcap%
\pgfsetroundjoin%
\definecolor{currentfill}{rgb}{0.150000,0.150000,0.150000}%
\pgfsetfillcolor{currentfill}%
\pgfsetlinewidth{0.803000pt}%
\definecolor{currentstroke}{rgb}{0.150000,0.150000,0.150000}%
\pgfsetstrokecolor{currentstroke}%
\pgfsetdash{}{0pt}%
\pgfsys@defobject{currentmarker}{\pgfqpoint{0.000000in}{-0.044444in}}{\pgfqpoint{0.000000in}{0.000000in}}{%
\pgfpathmoveto{\pgfqpoint{0.000000in}{0.000000in}}%
\pgfpathlineto{\pgfqpoint{0.000000in}{-0.044444in}}%
\pgfusepath{stroke,fill}%
}%
\begin{pgfscope}%
\pgfsys@transformshift{4.489667in}{1.894508in}%
\pgfsys@useobject{currentmarker}{}%
\end{pgfscope}%
\end{pgfscope}%
\begin{pgfscope}%
\pgfsetbuttcap%
\pgfsetroundjoin%
\definecolor{currentfill}{rgb}{0.150000,0.150000,0.150000}%
\pgfsetfillcolor{currentfill}%
\pgfsetlinewidth{0.803000pt}%
\definecolor{currentstroke}{rgb}{0.150000,0.150000,0.150000}%
\pgfsetstrokecolor{currentstroke}%
\pgfsetdash{}{0pt}%
\pgfsys@defobject{currentmarker}{\pgfqpoint{0.000000in}{-0.044444in}}{\pgfqpoint{0.000000in}{0.000000in}}{%
\pgfpathmoveto{\pgfqpoint{0.000000in}{0.000000in}}%
\pgfpathlineto{\pgfqpoint{0.000000in}{-0.044444in}}%
\pgfusepath{stroke,fill}%
}%
\begin{pgfscope}%
\pgfsys@transformshift{4.529370in}{1.894508in}%
\pgfsys@useobject{currentmarker}{}%
\end{pgfscope}%
\end{pgfscope}%
\begin{pgfscope}%
\pgfsetbuttcap%
\pgfsetroundjoin%
\definecolor{currentfill}{rgb}{0.150000,0.150000,0.150000}%
\pgfsetfillcolor{currentfill}%
\pgfsetlinewidth{0.803000pt}%
\definecolor{currentstroke}{rgb}{0.150000,0.150000,0.150000}%
\pgfsetstrokecolor{currentstroke}%
\pgfsetdash{}{0pt}%
\pgfsys@defobject{currentmarker}{\pgfqpoint{0.000000in}{-0.044444in}}{\pgfqpoint{0.000000in}{0.000000in}}{%
\pgfpathmoveto{\pgfqpoint{0.000000in}{0.000000in}}%
\pgfpathlineto{\pgfqpoint{0.000000in}{-0.044444in}}%
\pgfusepath{stroke,fill}%
}%
\begin{pgfscope}%
\pgfsys@transformshift{4.562937in}{1.894508in}%
\pgfsys@useobject{currentmarker}{}%
\end{pgfscope}%
\end{pgfscope}%
\begin{pgfscope}%
\pgfsetbuttcap%
\pgfsetroundjoin%
\definecolor{currentfill}{rgb}{0.150000,0.150000,0.150000}%
\pgfsetfillcolor{currentfill}%
\pgfsetlinewidth{0.803000pt}%
\definecolor{currentstroke}{rgb}{0.150000,0.150000,0.150000}%
\pgfsetstrokecolor{currentstroke}%
\pgfsetdash{}{0pt}%
\pgfsys@defobject{currentmarker}{\pgfqpoint{0.000000in}{-0.044444in}}{\pgfqpoint{0.000000in}{0.000000in}}{%
\pgfpathmoveto{\pgfqpoint{0.000000in}{0.000000in}}%
\pgfpathlineto{\pgfqpoint{0.000000in}{-0.044444in}}%
\pgfusepath{stroke,fill}%
}%
\begin{pgfscope}%
\pgfsys@transformshift{4.592015in}{1.894508in}%
\pgfsys@useobject{currentmarker}{}%
\end{pgfscope}%
\end{pgfscope}%
\begin{pgfscope}%
\pgfsetbuttcap%
\pgfsetroundjoin%
\definecolor{currentfill}{rgb}{0.150000,0.150000,0.150000}%
\pgfsetfillcolor{currentfill}%
\pgfsetlinewidth{0.803000pt}%
\definecolor{currentstroke}{rgb}{0.150000,0.150000,0.150000}%
\pgfsetstrokecolor{currentstroke}%
\pgfsetdash{}{0pt}%
\pgfsys@defobject{currentmarker}{\pgfqpoint{0.000000in}{-0.044444in}}{\pgfqpoint{0.000000in}{0.000000in}}{%
\pgfpathmoveto{\pgfqpoint{0.000000in}{0.000000in}}%
\pgfpathlineto{\pgfqpoint{0.000000in}{-0.044444in}}%
\pgfusepath{stroke,fill}%
}%
\begin{pgfscope}%
\pgfsys@transformshift{4.617663in}{1.894508in}%
\pgfsys@useobject{currentmarker}{}%
\end{pgfscope}%
\end{pgfscope}%
\begin{pgfscope}%
\pgfsetbuttcap%
\pgfsetroundjoin%
\definecolor{currentfill}{rgb}{0.150000,0.150000,0.150000}%
\pgfsetfillcolor{currentfill}%
\pgfsetlinewidth{0.803000pt}%
\definecolor{currentstroke}{rgb}{0.150000,0.150000,0.150000}%
\pgfsetstrokecolor{currentstroke}%
\pgfsetdash{}{0pt}%
\pgfsys@defobject{currentmarker}{\pgfqpoint{0.000000in}{-0.044444in}}{\pgfqpoint{0.000000in}{0.000000in}}{%
\pgfpathmoveto{\pgfqpoint{0.000000in}{0.000000in}}%
\pgfpathlineto{\pgfqpoint{0.000000in}{-0.044444in}}%
\pgfusepath{stroke,fill}%
}%
\begin{pgfscope}%
\pgfsys@transformshift{4.791546in}{1.894508in}%
\pgfsys@useobject{currentmarker}{}%
\end{pgfscope}%
\end{pgfscope}%
\begin{pgfscope}%
\pgfsetbuttcap%
\pgfsetroundjoin%
\definecolor{currentfill}{rgb}{0.150000,0.150000,0.150000}%
\pgfsetfillcolor{currentfill}%
\pgfsetlinewidth{1.003750pt}%
\definecolor{currentstroke}{rgb}{0.150000,0.150000,0.150000}%
\pgfsetstrokecolor{currentstroke}%
\pgfsetdash{}{0pt}%
\pgfsys@defobject{currentmarker}{\pgfqpoint{-0.066667in}{0.000000in}}{\pgfqpoint{0.000000in}{0.000000in}}{%
\pgfpathmoveto{\pgfqpoint{0.000000in}{0.000000in}}%
\pgfpathlineto{\pgfqpoint{-0.066667in}{0.000000in}}%
\pgfusepath{stroke,fill}%
}%
\begin{pgfscope}%
\pgfsys@transformshift{3.637789in}{1.894508in}%
\pgfsys@useobject{currentmarker}{}%
\end{pgfscope}%
\end{pgfscope}%
\begin{pgfscope}%
\pgfsetbuttcap%
\pgfsetroundjoin%
\definecolor{currentfill}{rgb}{0.150000,0.150000,0.150000}%
\pgfsetfillcolor{currentfill}%
\pgfsetlinewidth{1.003750pt}%
\definecolor{currentstroke}{rgb}{0.150000,0.150000,0.150000}%
\pgfsetstrokecolor{currentstroke}%
\pgfsetdash{}{0pt}%
\pgfsys@defobject{currentmarker}{\pgfqpoint{-0.066667in}{0.000000in}}{\pgfqpoint{0.000000in}{0.000000in}}{%
\pgfpathmoveto{\pgfqpoint{0.000000in}{0.000000in}}%
\pgfpathlineto{\pgfqpoint{-0.066667in}{0.000000in}}%
\pgfusepath{stroke,fill}%
}%
\begin{pgfscope}%
\pgfsys@transformshift{3.637789in}{2.221402in}%
\pgfsys@useobject{currentmarker}{}%
\end{pgfscope}%
\end{pgfscope}%
\begin{pgfscope}%
\pgfsetbuttcap%
\pgfsetroundjoin%
\definecolor{currentfill}{rgb}{0.150000,0.150000,0.150000}%
\pgfsetfillcolor{currentfill}%
\pgfsetlinewidth{1.003750pt}%
\definecolor{currentstroke}{rgb}{0.150000,0.150000,0.150000}%
\pgfsetstrokecolor{currentstroke}%
\pgfsetdash{}{0pt}%
\pgfsys@defobject{currentmarker}{\pgfqpoint{-0.066667in}{0.000000in}}{\pgfqpoint{0.000000in}{0.000000in}}{%
\pgfpathmoveto{\pgfqpoint{0.000000in}{0.000000in}}%
\pgfpathlineto{\pgfqpoint{-0.066667in}{0.000000in}}%
\pgfusepath{stroke,fill}%
}%
\begin{pgfscope}%
\pgfsys@transformshift{3.637789in}{2.502456in}%
\pgfsys@useobject{currentmarker}{}%
\end{pgfscope}%
\end{pgfscope}%
\begin{pgfscope}%
\pgfpathrectangle{\pgfqpoint{3.637789in}{1.894508in}}{\pgfqpoint{1.223103in}{0.607948in}}%
\pgfusepath{clip}%
\pgfsetroundcap%
\pgfsetroundjoin%
\pgfsetlinewidth{1.204500pt}%
\definecolor{currentstroke}{rgb}{0.000000,0.501961,0.000000}%
\pgfsetstrokecolor{currentstroke}%
\pgfsetdash{}{0pt}%
\pgfpathmoveto{\pgfqpoint{3.637789in}{2.221466in}}%
\pgfpathlineto{\pgfqpoint{3.878498in}{2.222054in}}%
\pgfpathlineto{\pgfqpoint{3.990029in}{2.222640in}}%
\pgfpathlineto{\pgfqpoint{4.063425in}{2.223226in}}%
\pgfpathlineto{\pgfqpoint{4.118221in}{2.223811in}}%
\pgfpathlineto{\pgfqpoint{4.161963in}{2.224395in}}%
\pgfpathlineto{\pgfqpoint{4.198371in}{2.224977in}}%
\pgfpathlineto{\pgfqpoint{4.229555in}{2.225558in}}%
\pgfpathlineto{\pgfqpoint{4.256828in}{2.226138in}}%
\pgfpathlineto{\pgfqpoint{4.281062in}{2.226716in}}%
\pgfpathlineto{\pgfqpoint{4.302867in}{2.227293in}}%
\pgfpathlineto{\pgfqpoint{4.322686in}{2.227868in}}%
\pgfpathlineto{\pgfqpoint{4.340851in}{2.228441in}}%
\pgfpathlineto{\pgfqpoint{4.357617in}{2.229012in}}%
\pgfpathlineto{\pgfqpoint{4.373183in}{2.229580in}}%
\pgfpathlineto{\pgfqpoint{4.387711in}{2.230146in}}%
\pgfpathlineto{\pgfqpoint{4.401329in}{2.230709in}}%
\pgfpathlineto{\pgfqpoint{4.414146in}{2.231269in}}%
\pgfpathlineto{\pgfqpoint{4.426250in}{2.231826in}}%
\pgfpathlineto{\pgfqpoint{4.437717in}{2.232379in}}%
\pgfpathlineto{\pgfqpoint{4.448610in}{2.232928in}}%
\pgfpathlineto{\pgfqpoint{4.458984in}{2.233472in}}%
\pgfpathlineto{\pgfqpoint{4.468886in}{2.234012in}}%
\pgfpathlineto{\pgfqpoint{4.478357in}{2.234546in}}%
\pgfpathlineto{\pgfqpoint{4.487434in}{2.235075in}}%
\pgfpathlineto{\pgfqpoint{4.496147in}{2.235597in}}%
\pgfpathlineto{\pgfqpoint{4.504525in}{2.236113in}}%
\pgfpathlineto{\pgfqpoint{4.512593in}{2.236620in}}%
\pgfpathlineto{\pgfqpoint{4.520372in}{2.237120in}}%
\pgfpathlineto{\pgfqpoint{4.527883in}{2.237611in}}%
\pgfpathlineto{\pgfqpoint{4.535144in}{2.238091in}}%
\pgfpathlineto{\pgfqpoint{4.542170in}{2.238562in}}%
\pgfpathlineto{\pgfqpoint{4.548977in}{2.239020in}}%
\pgfpathlineto{\pgfqpoint{4.555577in}{2.239466in}}%
\pgfpathlineto{\pgfqpoint{4.561983in}{2.239899in}}%
\pgfpathlineto{\pgfqpoint{4.568206in}{2.240317in}}%
\pgfpathlineto{\pgfqpoint{4.574256in}{2.240720in}}%
\pgfpathlineto{\pgfqpoint{4.580143in}{2.241105in}}%
\pgfpathlineto{\pgfqpoint{4.585874in}{2.241472in}}%
\pgfpathlineto{\pgfqpoint{4.591459in}{2.241820in}}%
\pgfpathlineto{\pgfqpoint{4.596904in}{2.242147in}}%
\pgfpathlineto{\pgfqpoint{4.602216in}{2.242451in}}%
\pgfpathlineto{\pgfqpoint{4.607401in}{2.242731in}}%
\pgfpathlineto{\pgfqpoint{4.612466in}{2.242985in}}%
\pgfpathlineto{\pgfqpoint{4.617416in}{2.243212in}}%
\pgfpathlineto{\pgfqpoint{4.622256in}{2.243410in}}%
\pgfpathlineto{\pgfqpoint{4.626991in}{2.243576in}}%
\pgfpathlineto{\pgfqpoint{4.631625in}{2.243709in}}%
\pgfpathlineto{\pgfqpoint{4.636162in}{2.243807in}}%
\pgfpathlineto{\pgfqpoint{4.640607in}{2.243867in}}%
\pgfusepath{stroke}%
\end{pgfscope}%
\begin{pgfscope}%
\pgfsetrectcap%
\pgfsetmiterjoin%
\pgfsetlinewidth{1.003750pt}%
\definecolor{currentstroke}{rgb}{0.150000,0.150000,0.150000}%
\pgfsetstrokecolor{currentstroke}%
\pgfsetdash{}{0pt}%
\pgfpathmoveto{\pgfqpoint{3.637789in}{1.894508in}}%
\pgfpathlineto{\pgfqpoint{3.637789in}{2.502456in}}%
\pgfusepath{stroke}%
\end{pgfscope}%
\begin{pgfscope}%
\pgfsetrectcap%
\pgfsetmiterjoin%
\pgfsetlinewidth{1.003750pt}%
\definecolor{currentstroke}{rgb}{0.150000,0.150000,0.150000}%
\pgfsetstrokecolor{currentstroke}%
\pgfsetdash{}{0pt}%
\pgfpathmoveto{\pgfqpoint{3.637789in}{1.894508in}}%
\pgfpathlineto{\pgfqpoint{4.860892in}{1.894508in}}%
\pgfusepath{stroke}%
\end{pgfscope}%
\begin{pgfscope}%
\pgfpathrectangle{\pgfqpoint{3.637789in}{1.894508in}}{\pgfqpoint{1.223103in}{0.607948in}}%
\pgfusepath{clip}%
\pgfsetbuttcap%
\pgfsetroundjoin%
\definecolor{currentfill}{rgb}{0.000000,0.000000,0.000000}%
\pgfsetfillcolor{currentfill}%
\pgfsetlinewidth{1.003750pt}%
\definecolor{currentstroke}{rgb}{0.000000,0.000000,0.000000}%
\pgfsetstrokecolor{currentstroke}%
\pgfsetdash{}{0pt}%
\pgfsys@defobject{currentmarker}{\pgfqpoint{-0.013889in}{-0.013889in}}{\pgfqpoint{0.013889in}{0.013889in}}{%
\pgfpathmoveto{\pgfqpoint{0.000000in}{-0.013889in}}%
\pgfpathcurveto{\pgfqpoint{0.003683in}{-0.013889in}}{\pgfqpoint{0.007216in}{-0.012425in}}{\pgfqpoint{0.009821in}{-0.009821in}}%
\pgfpathcurveto{\pgfqpoint{0.012425in}{-0.007216in}}{\pgfqpoint{0.013889in}{-0.003683in}}{\pgfqpoint{0.013889in}{0.000000in}}%
\pgfpathcurveto{\pgfqpoint{0.013889in}{0.003683in}}{\pgfqpoint{0.012425in}{0.007216in}}{\pgfqpoint{0.009821in}{0.009821in}}%
\pgfpathcurveto{\pgfqpoint{0.007216in}{0.012425in}}{\pgfqpoint{0.003683in}{0.013889in}}{\pgfqpoint{0.000000in}{0.013889in}}%
\pgfpathcurveto{\pgfqpoint{-0.003683in}{0.013889in}}{\pgfqpoint{-0.007216in}{0.012425in}}{\pgfqpoint{-0.009821in}{0.009821in}}%
\pgfpathcurveto{\pgfqpoint{-0.012425in}{0.007216in}}{\pgfqpoint{-0.013889in}{0.003683in}}{\pgfqpoint{-0.013889in}{0.000000in}}%
\pgfpathcurveto{\pgfqpoint{-0.013889in}{-0.003683in}}{\pgfqpoint{-0.012425in}{-0.007216in}}{\pgfqpoint{-0.009821in}{-0.009821in}}%
\pgfpathcurveto{\pgfqpoint{-0.007216in}{-0.012425in}}{\pgfqpoint{-0.003683in}{-0.013889in}}{\pgfqpoint{0.000000in}{-0.013889in}}%
\pgfpathclose%
\pgfusepath{stroke,fill}%
}%
\begin{pgfscope}%
\pgfsys@transformshift{4.290137in}{2.226976in}%
\pgfsys@useobject{currentmarker}{}%
\end{pgfscope}%
\begin{pgfscope}%
\pgfsys@transformshift{4.294536in}{2.227090in}%
\pgfsys@useobject{currentmarker}{}%
\end{pgfscope}%
\begin{pgfscope}%
\pgfsys@transformshift{4.299026in}{2.227208in}%
\pgfsys@useobject{currentmarker}{}%
\end{pgfscope}%
\begin{pgfscope}%
\pgfsys@transformshift{4.303611in}{2.227330in}%
\pgfsys@useobject{currentmarker}{}%
\end{pgfscope}%
\begin{pgfscope}%
\pgfsys@transformshift{4.308294in}{2.227459in}%
\pgfsys@useobject{currentmarker}{}%
\end{pgfscope}%
\begin{pgfscope}%
\pgfsys@transformshift{4.313080in}{2.227592in}%
\pgfsys@useobject{currentmarker}{}%
\end{pgfscope}%
\begin{pgfscope}%
\pgfsys@transformshift{4.317974in}{2.227732in}%
\pgfsys@useobject{currentmarker}{}%
\end{pgfscope}%
\begin{pgfscope}%
\pgfsys@transformshift{4.322980in}{2.227879in}%
\pgfsys@useobject{currentmarker}{}%
\end{pgfscope}%
\begin{pgfscope}%
\pgfsys@transformshift{4.328104in}{2.228032in}%
\pgfsys@useobject{currentmarker}{}%
\end{pgfscope}%
\begin{pgfscope}%
\pgfsys@transformshift{4.333351in}{2.228193in}%
\pgfsys@useobject{currentmarker}{}%
\end{pgfscope}%
\begin{pgfscope}%
\pgfsys@transformshift{4.338728in}{2.228361in}%
\pgfsys@useobject{currentmarker}{}%
\end{pgfscope}%
\begin{pgfscope}%
\pgfsys@transformshift{4.367806in}{2.229347in}%
\pgfsys@useobject{currentmarker}{}%
\end{pgfscope}%
\begin{pgfscope}%
\pgfsys@transformshift{4.401374in}{2.230656in}%
\pgfsys@useobject{currentmarker}{}%
\end{pgfscope}%
\begin{pgfscope}%
\pgfsys@transformshift{4.441076in}{2.232475in}%
\pgfsys@useobject{currentmarker}{}%
\end{pgfscope}%
\begin{pgfscope}%
\pgfsys@transformshift{4.489667in}{2.235158in}%
\pgfsys@useobject{currentmarker}{}%
\end{pgfscope}%
\begin{pgfscope}%
\pgfsys@transformshift{4.552313in}{2.239363in}%
\pgfsys@useobject{currentmarker}{}%
\end{pgfscope}%
\begin{pgfscope}%
\pgfsys@transformshift{4.640607in}{2.243827in}%
\pgfsys@useobject{currentmarker}{}%
\end{pgfscope}%
\end{pgfscope}%
\begin{pgfscope}%
\pgfsetbuttcap%
\pgfsetmiterjoin%
\definecolor{currentfill}{rgb}{1.000000,1.000000,1.000000}%
\pgfsetfillcolor{currentfill}%
\pgfsetlinewidth{0.000000pt}%
\definecolor{currentstroke}{rgb}{0.000000,0.000000,0.000000}%
\pgfsetstrokecolor{currentstroke}%
\pgfsetstrokeopacity{0.000000}%
\pgfsetdash{}{0pt}%
\pgfpathmoveto{\pgfqpoint{5.105513in}{1.894508in}}%
\pgfpathlineto{\pgfqpoint{6.328616in}{1.894508in}}%
\pgfpathlineto{\pgfqpoint{6.328616in}{2.502456in}}%
\pgfpathlineto{\pgfqpoint{5.105513in}{2.502456in}}%
\pgfpathclose%
\pgfusepath{fill}%
\end{pgfscope}%
\begin{pgfscope}%
\pgfpathrectangle{\pgfqpoint{5.105513in}{1.894508in}}{\pgfqpoint{1.223103in}{0.607948in}}%
\pgfusepath{clip}%
\pgfsetbuttcap%
\pgfsetmiterjoin%
\definecolor{currentfill}{rgb}{0.000000,0.000000,1.000000}%
\pgfsetfillcolor{currentfill}%
\pgfsetfillopacity{0.100000}%
\pgfsetlinewidth{0.803000pt}%
\definecolor{currentstroke}{rgb}{0.000000,0.000000,1.000000}%
\pgfsetstrokecolor{currentstroke}%
\pgfsetstrokeopacity{0.100000}%
\pgfsetdash{}{0pt}%
\pgfpathmoveto{\pgfqpoint{5.105513in}{2.220603in}}%
\pgfpathlineto{\pgfqpoint{5.105513in}{2.221904in}}%
\pgfpathlineto{\pgfqpoint{6.328616in}{2.221904in}}%
\pgfpathlineto{\pgfqpoint{6.328616in}{2.220603in}}%
\pgfpathclose%
\pgfusepath{stroke,fill}%
\end{pgfscope}%
\begin{pgfscope}%
\pgfpathrectangle{\pgfqpoint{5.105513in}{1.894508in}}{\pgfqpoint{1.223103in}{0.607948in}}%
\pgfusepath{clip}%
\pgfsetbuttcap%
\pgfsetroundjoin%
\definecolor{currentfill}{rgb}{0.000000,0.501961,0.000000}%
\pgfsetfillcolor{currentfill}%
\pgfsetfillopacity{0.500000}%
\pgfsetlinewidth{0.803000pt}%
\definecolor{currentstroke}{rgb}{0.000000,0.501961,0.000000}%
\pgfsetstrokecolor{currentstroke}%
\pgfsetstrokeopacity{0.500000}%
\pgfsetdash{}{0pt}%
\pgfpathmoveto{\pgfqpoint{5.105513in}{2.222190in}}%
\pgfpathlineto{\pgfqpoint{5.105513in}{2.221041in}}%
\pgfpathlineto{\pgfqpoint{5.346222in}{2.221895in}}%
\pgfpathlineto{\pgfqpoint{5.457753in}{2.222713in}}%
\pgfpathlineto{\pgfqpoint{5.531149in}{2.223501in}}%
\pgfpathlineto{\pgfqpoint{5.585945in}{2.224264in}}%
\pgfpathlineto{\pgfqpoint{5.629687in}{2.225006in}}%
\pgfpathlineto{\pgfqpoint{5.666095in}{2.225731in}}%
\pgfpathlineto{\pgfqpoint{5.697279in}{2.226444in}}%
\pgfpathlineto{\pgfqpoint{5.724552in}{2.227146in}}%
\pgfpathlineto{\pgfqpoint{5.748786in}{2.227841in}}%
\pgfpathlineto{\pgfqpoint{5.770591in}{2.228531in}}%
\pgfpathlineto{\pgfqpoint{5.790410in}{2.229214in}}%
\pgfpathlineto{\pgfqpoint{5.808575in}{2.229884in}}%
\pgfpathlineto{\pgfqpoint{5.825341in}{2.230556in}}%
\pgfpathlineto{\pgfqpoint{5.840907in}{2.231230in}}%
\pgfpathlineto{\pgfqpoint{5.855435in}{2.231907in}}%
\pgfpathlineto{\pgfqpoint{5.869053in}{2.232585in}}%
\pgfpathlineto{\pgfqpoint{5.881870in}{2.233266in}}%
\pgfpathlineto{\pgfqpoint{5.893974in}{2.233950in}}%
\pgfpathlineto{\pgfqpoint{5.905441in}{2.234635in}}%
\pgfpathlineto{\pgfqpoint{5.916334in}{2.235323in}}%
\pgfpathlineto{\pgfqpoint{5.926708in}{2.236013in}}%
\pgfpathlineto{\pgfqpoint{5.936610in}{2.236704in}}%
\pgfpathlineto{\pgfqpoint{5.946082in}{2.237396in}}%
\pgfpathlineto{\pgfqpoint{5.955158in}{2.238090in}}%
\pgfpathlineto{\pgfqpoint{5.963871in}{2.238786in}}%
\pgfpathlineto{\pgfqpoint{5.972249in}{2.239483in}}%
\pgfpathlineto{\pgfqpoint{5.980317in}{2.240181in}}%
\pgfpathlineto{\pgfqpoint{5.988096in}{2.240881in}}%
\pgfpathlineto{\pgfqpoint{5.995607in}{2.241581in}}%
\pgfpathlineto{\pgfqpoint{6.002868in}{2.242279in}}%
\pgfpathlineto{\pgfqpoint{6.009894in}{2.242974in}}%
\pgfpathlineto{\pgfqpoint{6.016701in}{2.243665in}}%
\pgfpathlineto{\pgfqpoint{6.023301in}{2.244354in}}%
\pgfpathlineto{\pgfqpoint{6.029707in}{2.245041in}}%
\pgfpathlineto{\pgfqpoint{6.035930in}{2.245728in}}%
\pgfpathlineto{\pgfqpoint{6.041980in}{2.246417in}}%
\pgfpathlineto{\pgfqpoint{6.047867in}{2.247107in}}%
\pgfpathlineto{\pgfqpoint{6.053598in}{2.247802in}}%
\pgfpathlineto{\pgfqpoint{6.059183in}{2.248503in}}%
\pgfpathlineto{\pgfqpoint{6.064628in}{2.249211in}}%
\pgfpathlineto{\pgfqpoint{6.069940in}{2.249931in}}%
\pgfpathlineto{\pgfqpoint{6.075126in}{2.250664in}}%
\pgfpathlineto{\pgfqpoint{6.080191in}{2.251415in}}%
\pgfpathlineto{\pgfqpoint{6.085140in}{2.252187in}}%
\pgfpathlineto{\pgfqpoint{6.089980in}{2.252984in}}%
\pgfpathlineto{\pgfqpoint{6.094715in}{2.253811in}}%
\pgfpathlineto{\pgfqpoint{6.099349in}{2.254674in}}%
\pgfpathlineto{\pgfqpoint{6.103886in}{2.255578in}}%
\pgfpathlineto{\pgfqpoint{6.108331in}{2.256496in}}%
\pgfpathlineto{\pgfqpoint{6.108331in}{2.256541in}}%
\pgfpathlineto{\pgfqpoint{6.108331in}{2.256541in}}%
\pgfpathlineto{\pgfqpoint{6.103886in}{2.255736in}}%
\pgfpathlineto{\pgfqpoint{6.099349in}{2.254963in}}%
\pgfpathlineto{\pgfqpoint{6.094715in}{2.254191in}}%
\pgfpathlineto{\pgfqpoint{6.089980in}{2.253421in}}%
\pgfpathlineto{\pgfqpoint{6.085140in}{2.252653in}}%
\pgfpathlineto{\pgfqpoint{6.080191in}{2.251887in}}%
\pgfpathlineto{\pgfqpoint{6.075126in}{2.251125in}}%
\pgfpathlineto{\pgfqpoint{6.069940in}{2.250366in}}%
\pgfpathlineto{\pgfqpoint{6.064628in}{2.249612in}}%
\pgfpathlineto{\pgfqpoint{6.059183in}{2.248862in}}%
\pgfpathlineto{\pgfqpoint{6.053598in}{2.248116in}}%
\pgfpathlineto{\pgfqpoint{6.047867in}{2.247375in}}%
\pgfpathlineto{\pgfqpoint{6.041980in}{2.246639in}}%
\pgfpathlineto{\pgfqpoint{6.035930in}{2.245908in}}%
\pgfpathlineto{\pgfqpoint{6.029707in}{2.245181in}}%
\pgfpathlineto{\pgfqpoint{6.023301in}{2.244459in}}%
\pgfpathlineto{\pgfqpoint{6.016701in}{2.243743in}}%
\pgfpathlineto{\pgfqpoint{6.009894in}{2.243031in}}%
\pgfpathlineto{\pgfqpoint{6.002868in}{2.242327in}}%
\pgfpathlineto{\pgfqpoint{5.995607in}{2.241627in}}%
\pgfpathlineto{\pgfqpoint{5.988096in}{2.240932in}}%
\pgfpathlineto{\pgfqpoint{5.980317in}{2.240237in}}%
\pgfpathlineto{\pgfqpoint{5.972249in}{2.239543in}}%
\pgfpathlineto{\pgfqpoint{5.963871in}{2.238850in}}%
\pgfpathlineto{\pgfqpoint{5.955158in}{2.238157in}}%
\pgfpathlineto{\pgfqpoint{5.946082in}{2.237464in}}%
\pgfpathlineto{\pgfqpoint{5.936610in}{2.236772in}}%
\pgfpathlineto{\pgfqpoint{5.926708in}{2.236082in}}%
\pgfpathlineto{\pgfqpoint{5.916334in}{2.235392in}}%
\pgfpathlineto{\pgfqpoint{5.905441in}{2.234704in}}%
\pgfpathlineto{\pgfqpoint{5.893974in}{2.234016in}}%
\pgfpathlineto{\pgfqpoint{5.881870in}{2.233330in}}%
\pgfpathlineto{\pgfqpoint{5.869053in}{2.232644in}}%
\pgfpathlineto{\pgfqpoint{5.855435in}{2.231958in}}%
\pgfpathlineto{\pgfqpoint{5.840907in}{2.231274in}}%
\pgfpathlineto{\pgfqpoint{5.825341in}{2.230589in}}%
\pgfpathlineto{\pgfqpoint{5.808575in}{2.229904in}}%
\pgfpathlineto{\pgfqpoint{5.790410in}{2.229220in}}%
\pgfpathlineto{\pgfqpoint{5.770591in}{2.228548in}}%
\pgfpathlineto{\pgfqpoint{5.748786in}{2.227884in}}%
\pgfpathlineto{\pgfqpoint{5.724552in}{2.227223in}}%
\pgfpathlineto{\pgfqpoint{5.697279in}{2.226566in}}%
\pgfpathlineto{\pgfqpoint{5.666095in}{2.225914in}}%
\pgfpathlineto{\pgfqpoint{5.629687in}{2.225269in}}%
\pgfpathlineto{\pgfqpoint{5.585945in}{2.224631in}}%
\pgfpathlineto{\pgfqpoint{5.531149in}{2.224003in}}%
\pgfpathlineto{\pgfqpoint{5.457753in}{2.223385in}}%
\pgfpathlineto{\pgfqpoint{5.346222in}{2.222781in}}%
\pgfpathlineto{\pgfqpoint{5.105513in}{2.222190in}}%
\pgfpathclose%
\pgfusepath{stroke,fill}%
\end{pgfscope}%
\begin{pgfscope}%
\pgfpathrectangle{\pgfqpoint{5.105513in}{1.894508in}}{\pgfqpoint{1.223103in}{0.607948in}}%
\pgfusepath{clip}%
\pgfsetroundcap%
\pgfsetroundjoin%
\pgfsetlinewidth{0.501875pt}%
\definecolor{currentstroke}{rgb}{0.000000,0.000000,1.000000}%
\pgfsetstrokecolor{currentstroke}%
\pgfsetstrokeopacity{0.800000}%
\pgfsetdash{}{0pt}%
\pgfpathmoveto{\pgfqpoint{5.105513in}{2.221254in}}%
\pgfpathlineto{\pgfqpoint{6.328616in}{2.221254in}}%
\pgfusepath{stroke}%
\end{pgfscope}%
\begin{pgfscope}%
\pgfpathrectangle{\pgfqpoint{5.105513in}{1.894508in}}{\pgfqpoint{1.223103in}{0.607948in}}%
\pgfusepath{clip}%
\pgfsetbuttcap%
\pgfsetroundjoin%
\pgfsetlinewidth{1.003750pt}%
\definecolor{currentstroke}{rgb}{0.000000,0.000000,0.000000}%
\pgfsetstrokecolor{currentstroke}%
\pgfsetdash{{3.700000pt}{1.600000pt}}{0.000000pt}%
\pgfpathmoveto{\pgfqpoint{5.105513in}{2.221402in}}%
\pgfpathlineto{\pgfqpoint{6.328616in}{2.221402in}}%
\pgfusepath{stroke}%
\end{pgfscope}%
\begin{pgfscope}%
\pgfsetroundcap%
\pgfsetroundjoin%
\pgfsetlinewidth{0.501875pt}%
\definecolor{currentstroke}{rgb}{0.000000,0.000000,1.000000}%
\pgfsetstrokecolor{currentstroke}%
\pgfsetstrokeopacity{0.800000}%
\pgfsetdash{}{0pt}%
\pgfpathmoveto{\pgfqpoint{5.919361in}{2.338590in}}%
\pgfpathquadraticcurveto{\pgfqpoint{5.849843in}{2.288083in}}{\pgfqpoint{5.780325in}{2.237575in}}%
\pgfusepath{stroke}%
\end{pgfscope}%
\begin{pgfscope}%
\pgfsetfillopacity{0.800000}%
\pgfsetstrokeopacity{0.800000}%
\definecolor{textcolor}{rgb}{0.000000,0.000000,1.000000}%
\pgfsetstrokecolor{textcolor}%
\pgfsetfillcolor{textcolor}%
\pgftext[x=5.846155in,y=2.403638in,left,base]{\color{textcolor}\sffamily\fontsize{5.647059}{6.776471}\selectfont 5.5375(11)}%
\end{pgfscope}%
\begin{pgfscope}%
\pgfsetbuttcap%
\pgfsetroundjoin%
\definecolor{currentfill}{rgb}{0.150000,0.150000,0.150000}%
\pgfsetfillcolor{currentfill}%
\pgfsetlinewidth{1.003750pt}%
\definecolor{currentstroke}{rgb}{0.150000,0.150000,0.150000}%
\pgfsetstrokecolor{currentstroke}%
\pgfsetdash{}{0pt}%
\pgfsys@defobject{currentmarker}{\pgfqpoint{0.000000in}{-0.066667in}}{\pgfqpoint{0.000000in}{0.000000in}}{%
\pgfpathmoveto{\pgfqpoint{0.000000in}{0.000000in}}%
\pgfpathlineto{\pgfqpoint{0.000000in}{-0.066667in}}%
\pgfusepath{stroke,fill}%
}%
\begin{pgfscope}%
\pgfsys@transformshift{5.105513in}{1.894508in}%
\pgfsys@useobject{currentmarker}{}%
\end{pgfscope}%
\end{pgfscope}%
\begin{pgfscope}%
\pgfsetbuttcap%
\pgfsetroundjoin%
\definecolor{currentfill}{rgb}{0.150000,0.150000,0.150000}%
\pgfsetfillcolor{currentfill}%
\pgfsetlinewidth{1.003750pt}%
\definecolor{currentstroke}{rgb}{0.150000,0.150000,0.150000}%
\pgfsetstrokecolor{currentstroke}%
\pgfsetdash{}{0pt}%
\pgfsys@defobject{currentmarker}{\pgfqpoint{0.000000in}{-0.066667in}}{\pgfqpoint{0.000000in}{0.000000in}}{%
\pgfpathmoveto{\pgfqpoint{0.000000in}{0.000000in}}%
\pgfpathlineto{\pgfqpoint{0.000000in}{-0.066667in}}%
\pgfusepath{stroke,fill}%
}%
\begin{pgfscope}%
\pgfsys@transformshift{5.606922in}{1.894508in}%
\pgfsys@useobject{currentmarker}{}%
\end{pgfscope}%
\end{pgfscope}%
\begin{pgfscope}%
\pgfsetbuttcap%
\pgfsetroundjoin%
\definecolor{currentfill}{rgb}{0.150000,0.150000,0.150000}%
\pgfsetfillcolor{currentfill}%
\pgfsetlinewidth{1.003750pt}%
\definecolor{currentstroke}{rgb}{0.150000,0.150000,0.150000}%
\pgfsetstrokecolor{currentstroke}%
\pgfsetdash{}{0pt}%
\pgfsys@defobject{currentmarker}{\pgfqpoint{0.000000in}{-0.066667in}}{\pgfqpoint{0.000000in}{0.000000in}}{%
\pgfpathmoveto{\pgfqpoint{0.000000in}{0.000000in}}%
\pgfpathlineto{\pgfqpoint{0.000000in}{-0.066667in}}%
\pgfusepath{stroke,fill}%
}%
\begin{pgfscope}%
\pgfsys@transformshift{6.108331in}{1.894508in}%
\pgfsys@useobject{currentmarker}{}%
\end{pgfscope}%
\end{pgfscope}%
\begin{pgfscope}%
\pgfsetbuttcap%
\pgfsetroundjoin%
\definecolor{currentfill}{rgb}{0.150000,0.150000,0.150000}%
\pgfsetfillcolor{currentfill}%
\pgfsetlinewidth{0.803000pt}%
\definecolor{currentstroke}{rgb}{0.150000,0.150000,0.150000}%
\pgfsetstrokecolor{currentstroke}%
\pgfsetdash{}{0pt}%
\pgfsys@defobject{currentmarker}{\pgfqpoint{0.000000in}{-0.044444in}}{\pgfqpoint{0.000000in}{0.000000in}}{%
\pgfpathmoveto{\pgfqpoint{0.000000in}{0.000000in}}%
\pgfpathlineto{\pgfqpoint{0.000000in}{-0.044444in}}%
\pgfusepath{stroke,fill}%
}%
\begin{pgfscope}%
\pgfsys@transformshift{5.256452in}{1.894508in}%
\pgfsys@useobject{currentmarker}{}%
\end{pgfscope}%
\end{pgfscope}%
\begin{pgfscope}%
\pgfsetbuttcap%
\pgfsetroundjoin%
\definecolor{currentfill}{rgb}{0.150000,0.150000,0.150000}%
\pgfsetfillcolor{currentfill}%
\pgfsetlinewidth{0.803000pt}%
\definecolor{currentstroke}{rgb}{0.150000,0.150000,0.150000}%
\pgfsetstrokecolor{currentstroke}%
\pgfsetdash{}{0pt}%
\pgfsys@defobject{currentmarker}{\pgfqpoint{0.000000in}{-0.044444in}}{\pgfqpoint{0.000000in}{0.000000in}}{%
\pgfpathmoveto{\pgfqpoint{0.000000in}{0.000000in}}%
\pgfpathlineto{\pgfqpoint{0.000000in}{-0.044444in}}%
\pgfusepath{stroke,fill}%
}%
\begin{pgfscope}%
\pgfsys@transformshift{5.344746in}{1.894508in}%
\pgfsys@useobject{currentmarker}{}%
\end{pgfscope}%
\end{pgfscope}%
\begin{pgfscope}%
\pgfsetbuttcap%
\pgfsetroundjoin%
\definecolor{currentfill}{rgb}{0.150000,0.150000,0.150000}%
\pgfsetfillcolor{currentfill}%
\pgfsetlinewidth{0.803000pt}%
\definecolor{currentstroke}{rgb}{0.150000,0.150000,0.150000}%
\pgfsetstrokecolor{currentstroke}%
\pgfsetdash{}{0pt}%
\pgfsys@defobject{currentmarker}{\pgfqpoint{0.000000in}{-0.044444in}}{\pgfqpoint{0.000000in}{0.000000in}}{%
\pgfpathmoveto{\pgfqpoint{0.000000in}{0.000000in}}%
\pgfpathlineto{\pgfqpoint{0.000000in}{-0.044444in}}%
\pgfusepath{stroke,fill}%
}%
\begin{pgfscope}%
\pgfsys@transformshift{5.407391in}{1.894508in}%
\pgfsys@useobject{currentmarker}{}%
\end{pgfscope}%
\end{pgfscope}%
\begin{pgfscope}%
\pgfsetbuttcap%
\pgfsetroundjoin%
\definecolor{currentfill}{rgb}{0.150000,0.150000,0.150000}%
\pgfsetfillcolor{currentfill}%
\pgfsetlinewidth{0.803000pt}%
\definecolor{currentstroke}{rgb}{0.150000,0.150000,0.150000}%
\pgfsetstrokecolor{currentstroke}%
\pgfsetdash{}{0pt}%
\pgfsys@defobject{currentmarker}{\pgfqpoint{0.000000in}{-0.044444in}}{\pgfqpoint{0.000000in}{0.000000in}}{%
\pgfpathmoveto{\pgfqpoint{0.000000in}{0.000000in}}%
\pgfpathlineto{\pgfqpoint{0.000000in}{-0.044444in}}%
\pgfusepath{stroke,fill}%
}%
\begin{pgfscope}%
\pgfsys@transformshift{5.455982in}{1.894508in}%
\pgfsys@useobject{currentmarker}{}%
\end{pgfscope}%
\end{pgfscope}%
\begin{pgfscope}%
\pgfsetbuttcap%
\pgfsetroundjoin%
\definecolor{currentfill}{rgb}{0.150000,0.150000,0.150000}%
\pgfsetfillcolor{currentfill}%
\pgfsetlinewidth{0.803000pt}%
\definecolor{currentstroke}{rgb}{0.150000,0.150000,0.150000}%
\pgfsetstrokecolor{currentstroke}%
\pgfsetdash{}{0pt}%
\pgfsys@defobject{currentmarker}{\pgfqpoint{0.000000in}{-0.044444in}}{\pgfqpoint{0.000000in}{0.000000in}}{%
\pgfpathmoveto{\pgfqpoint{0.000000in}{0.000000in}}%
\pgfpathlineto{\pgfqpoint{0.000000in}{-0.044444in}}%
\pgfusepath{stroke,fill}%
}%
\begin{pgfscope}%
\pgfsys@transformshift{5.495685in}{1.894508in}%
\pgfsys@useobject{currentmarker}{}%
\end{pgfscope}%
\end{pgfscope}%
\begin{pgfscope}%
\pgfsetbuttcap%
\pgfsetroundjoin%
\definecolor{currentfill}{rgb}{0.150000,0.150000,0.150000}%
\pgfsetfillcolor{currentfill}%
\pgfsetlinewidth{0.803000pt}%
\definecolor{currentstroke}{rgb}{0.150000,0.150000,0.150000}%
\pgfsetstrokecolor{currentstroke}%
\pgfsetdash{}{0pt}%
\pgfsys@defobject{currentmarker}{\pgfqpoint{0.000000in}{-0.044444in}}{\pgfqpoint{0.000000in}{0.000000in}}{%
\pgfpathmoveto{\pgfqpoint{0.000000in}{0.000000in}}%
\pgfpathlineto{\pgfqpoint{0.000000in}{-0.044444in}}%
\pgfusepath{stroke,fill}%
}%
\begin{pgfscope}%
\pgfsys@transformshift{5.529252in}{1.894508in}%
\pgfsys@useobject{currentmarker}{}%
\end{pgfscope}%
\end{pgfscope}%
\begin{pgfscope}%
\pgfsetbuttcap%
\pgfsetroundjoin%
\definecolor{currentfill}{rgb}{0.150000,0.150000,0.150000}%
\pgfsetfillcolor{currentfill}%
\pgfsetlinewidth{0.803000pt}%
\definecolor{currentstroke}{rgb}{0.150000,0.150000,0.150000}%
\pgfsetstrokecolor{currentstroke}%
\pgfsetdash{}{0pt}%
\pgfsys@defobject{currentmarker}{\pgfqpoint{0.000000in}{-0.044444in}}{\pgfqpoint{0.000000in}{0.000000in}}{%
\pgfpathmoveto{\pgfqpoint{0.000000in}{0.000000in}}%
\pgfpathlineto{\pgfqpoint{0.000000in}{-0.044444in}}%
\pgfusepath{stroke,fill}%
}%
\begin{pgfscope}%
\pgfsys@transformshift{5.558330in}{1.894508in}%
\pgfsys@useobject{currentmarker}{}%
\end{pgfscope}%
\end{pgfscope}%
\begin{pgfscope}%
\pgfsetbuttcap%
\pgfsetroundjoin%
\definecolor{currentfill}{rgb}{0.150000,0.150000,0.150000}%
\pgfsetfillcolor{currentfill}%
\pgfsetlinewidth{0.803000pt}%
\definecolor{currentstroke}{rgb}{0.150000,0.150000,0.150000}%
\pgfsetstrokecolor{currentstroke}%
\pgfsetdash{}{0pt}%
\pgfsys@defobject{currentmarker}{\pgfqpoint{0.000000in}{-0.044444in}}{\pgfqpoint{0.000000in}{0.000000in}}{%
\pgfpathmoveto{\pgfqpoint{0.000000in}{0.000000in}}%
\pgfpathlineto{\pgfqpoint{0.000000in}{-0.044444in}}%
\pgfusepath{stroke,fill}%
}%
\begin{pgfscope}%
\pgfsys@transformshift{5.583978in}{1.894508in}%
\pgfsys@useobject{currentmarker}{}%
\end{pgfscope}%
\end{pgfscope}%
\begin{pgfscope}%
\pgfsetbuttcap%
\pgfsetroundjoin%
\definecolor{currentfill}{rgb}{0.150000,0.150000,0.150000}%
\pgfsetfillcolor{currentfill}%
\pgfsetlinewidth{0.803000pt}%
\definecolor{currentstroke}{rgb}{0.150000,0.150000,0.150000}%
\pgfsetstrokecolor{currentstroke}%
\pgfsetdash{}{0pt}%
\pgfsys@defobject{currentmarker}{\pgfqpoint{0.000000in}{-0.044444in}}{\pgfqpoint{0.000000in}{0.000000in}}{%
\pgfpathmoveto{\pgfqpoint{0.000000in}{0.000000in}}%
\pgfpathlineto{\pgfqpoint{0.000000in}{-0.044444in}}%
\pgfusepath{stroke,fill}%
}%
\begin{pgfscope}%
\pgfsys@transformshift{5.757861in}{1.894508in}%
\pgfsys@useobject{currentmarker}{}%
\end{pgfscope}%
\end{pgfscope}%
\begin{pgfscope}%
\pgfsetbuttcap%
\pgfsetroundjoin%
\definecolor{currentfill}{rgb}{0.150000,0.150000,0.150000}%
\pgfsetfillcolor{currentfill}%
\pgfsetlinewidth{0.803000pt}%
\definecolor{currentstroke}{rgb}{0.150000,0.150000,0.150000}%
\pgfsetstrokecolor{currentstroke}%
\pgfsetdash{}{0pt}%
\pgfsys@defobject{currentmarker}{\pgfqpoint{0.000000in}{-0.044444in}}{\pgfqpoint{0.000000in}{0.000000in}}{%
\pgfpathmoveto{\pgfqpoint{0.000000in}{0.000000in}}%
\pgfpathlineto{\pgfqpoint{0.000000in}{-0.044444in}}%
\pgfusepath{stroke,fill}%
}%
\begin{pgfscope}%
\pgfsys@transformshift{5.846155in}{1.894508in}%
\pgfsys@useobject{currentmarker}{}%
\end{pgfscope}%
\end{pgfscope}%
\begin{pgfscope}%
\pgfsetbuttcap%
\pgfsetroundjoin%
\definecolor{currentfill}{rgb}{0.150000,0.150000,0.150000}%
\pgfsetfillcolor{currentfill}%
\pgfsetlinewidth{0.803000pt}%
\definecolor{currentstroke}{rgb}{0.150000,0.150000,0.150000}%
\pgfsetstrokecolor{currentstroke}%
\pgfsetdash{}{0pt}%
\pgfsys@defobject{currentmarker}{\pgfqpoint{0.000000in}{-0.044444in}}{\pgfqpoint{0.000000in}{0.000000in}}{%
\pgfpathmoveto{\pgfqpoint{0.000000in}{0.000000in}}%
\pgfpathlineto{\pgfqpoint{0.000000in}{-0.044444in}}%
\pgfusepath{stroke,fill}%
}%
\begin{pgfscope}%
\pgfsys@transformshift{5.908800in}{1.894508in}%
\pgfsys@useobject{currentmarker}{}%
\end{pgfscope}%
\end{pgfscope}%
\begin{pgfscope}%
\pgfsetbuttcap%
\pgfsetroundjoin%
\definecolor{currentfill}{rgb}{0.150000,0.150000,0.150000}%
\pgfsetfillcolor{currentfill}%
\pgfsetlinewidth{0.803000pt}%
\definecolor{currentstroke}{rgb}{0.150000,0.150000,0.150000}%
\pgfsetstrokecolor{currentstroke}%
\pgfsetdash{}{0pt}%
\pgfsys@defobject{currentmarker}{\pgfqpoint{0.000000in}{-0.044444in}}{\pgfqpoint{0.000000in}{0.000000in}}{%
\pgfpathmoveto{\pgfqpoint{0.000000in}{0.000000in}}%
\pgfpathlineto{\pgfqpoint{0.000000in}{-0.044444in}}%
\pgfusepath{stroke,fill}%
}%
\begin{pgfscope}%
\pgfsys@transformshift{5.957391in}{1.894508in}%
\pgfsys@useobject{currentmarker}{}%
\end{pgfscope}%
\end{pgfscope}%
\begin{pgfscope}%
\pgfsetbuttcap%
\pgfsetroundjoin%
\definecolor{currentfill}{rgb}{0.150000,0.150000,0.150000}%
\pgfsetfillcolor{currentfill}%
\pgfsetlinewidth{0.803000pt}%
\definecolor{currentstroke}{rgb}{0.150000,0.150000,0.150000}%
\pgfsetstrokecolor{currentstroke}%
\pgfsetdash{}{0pt}%
\pgfsys@defobject{currentmarker}{\pgfqpoint{0.000000in}{-0.044444in}}{\pgfqpoint{0.000000in}{0.000000in}}{%
\pgfpathmoveto{\pgfqpoint{0.000000in}{0.000000in}}%
\pgfpathlineto{\pgfqpoint{0.000000in}{-0.044444in}}%
\pgfusepath{stroke,fill}%
}%
\begin{pgfscope}%
\pgfsys@transformshift{5.997094in}{1.894508in}%
\pgfsys@useobject{currentmarker}{}%
\end{pgfscope}%
\end{pgfscope}%
\begin{pgfscope}%
\pgfsetbuttcap%
\pgfsetroundjoin%
\definecolor{currentfill}{rgb}{0.150000,0.150000,0.150000}%
\pgfsetfillcolor{currentfill}%
\pgfsetlinewidth{0.803000pt}%
\definecolor{currentstroke}{rgb}{0.150000,0.150000,0.150000}%
\pgfsetstrokecolor{currentstroke}%
\pgfsetdash{}{0pt}%
\pgfsys@defobject{currentmarker}{\pgfqpoint{0.000000in}{-0.044444in}}{\pgfqpoint{0.000000in}{0.000000in}}{%
\pgfpathmoveto{\pgfqpoint{0.000000in}{0.000000in}}%
\pgfpathlineto{\pgfqpoint{0.000000in}{-0.044444in}}%
\pgfusepath{stroke,fill}%
}%
\begin{pgfscope}%
\pgfsys@transformshift{6.030661in}{1.894508in}%
\pgfsys@useobject{currentmarker}{}%
\end{pgfscope}%
\end{pgfscope}%
\begin{pgfscope}%
\pgfsetbuttcap%
\pgfsetroundjoin%
\definecolor{currentfill}{rgb}{0.150000,0.150000,0.150000}%
\pgfsetfillcolor{currentfill}%
\pgfsetlinewidth{0.803000pt}%
\definecolor{currentstroke}{rgb}{0.150000,0.150000,0.150000}%
\pgfsetstrokecolor{currentstroke}%
\pgfsetdash{}{0pt}%
\pgfsys@defobject{currentmarker}{\pgfqpoint{0.000000in}{-0.044444in}}{\pgfqpoint{0.000000in}{0.000000in}}{%
\pgfpathmoveto{\pgfqpoint{0.000000in}{0.000000in}}%
\pgfpathlineto{\pgfqpoint{0.000000in}{-0.044444in}}%
\pgfusepath{stroke,fill}%
}%
\begin{pgfscope}%
\pgfsys@transformshift{6.059739in}{1.894508in}%
\pgfsys@useobject{currentmarker}{}%
\end{pgfscope}%
\end{pgfscope}%
\begin{pgfscope}%
\pgfsetbuttcap%
\pgfsetroundjoin%
\definecolor{currentfill}{rgb}{0.150000,0.150000,0.150000}%
\pgfsetfillcolor{currentfill}%
\pgfsetlinewidth{0.803000pt}%
\definecolor{currentstroke}{rgb}{0.150000,0.150000,0.150000}%
\pgfsetstrokecolor{currentstroke}%
\pgfsetdash{}{0pt}%
\pgfsys@defobject{currentmarker}{\pgfqpoint{0.000000in}{-0.044444in}}{\pgfqpoint{0.000000in}{0.000000in}}{%
\pgfpathmoveto{\pgfqpoint{0.000000in}{0.000000in}}%
\pgfpathlineto{\pgfqpoint{0.000000in}{-0.044444in}}%
\pgfusepath{stroke,fill}%
}%
\begin{pgfscope}%
\pgfsys@transformshift{6.085387in}{1.894508in}%
\pgfsys@useobject{currentmarker}{}%
\end{pgfscope}%
\end{pgfscope}%
\begin{pgfscope}%
\pgfsetbuttcap%
\pgfsetroundjoin%
\definecolor{currentfill}{rgb}{0.150000,0.150000,0.150000}%
\pgfsetfillcolor{currentfill}%
\pgfsetlinewidth{0.803000pt}%
\definecolor{currentstroke}{rgb}{0.150000,0.150000,0.150000}%
\pgfsetstrokecolor{currentstroke}%
\pgfsetdash{}{0pt}%
\pgfsys@defobject{currentmarker}{\pgfqpoint{0.000000in}{-0.044444in}}{\pgfqpoint{0.000000in}{0.000000in}}{%
\pgfpathmoveto{\pgfqpoint{0.000000in}{0.000000in}}%
\pgfpathlineto{\pgfqpoint{0.000000in}{-0.044444in}}%
\pgfusepath{stroke,fill}%
}%
\begin{pgfscope}%
\pgfsys@transformshift{6.259270in}{1.894508in}%
\pgfsys@useobject{currentmarker}{}%
\end{pgfscope}%
\end{pgfscope}%
\begin{pgfscope}%
\pgfsetbuttcap%
\pgfsetroundjoin%
\definecolor{currentfill}{rgb}{0.150000,0.150000,0.150000}%
\pgfsetfillcolor{currentfill}%
\pgfsetlinewidth{1.003750pt}%
\definecolor{currentstroke}{rgb}{0.150000,0.150000,0.150000}%
\pgfsetstrokecolor{currentstroke}%
\pgfsetdash{}{0pt}%
\pgfsys@defobject{currentmarker}{\pgfqpoint{-0.066667in}{0.000000in}}{\pgfqpoint{0.000000in}{0.000000in}}{%
\pgfpathmoveto{\pgfqpoint{0.000000in}{0.000000in}}%
\pgfpathlineto{\pgfqpoint{-0.066667in}{0.000000in}}%
\pgfusepath{stroke,fill}%
}%
\begin{pgfscope}%
\pgfsys@transformshift{5.105513in}{1.894508in}%
\pgfsys@useobject{currentmarker}{}%
\end{pgfscope}%
\end{pgfscope}%
\begin{pgfscope}%
\pgfsetbuttcap%
\pgfsetroundjoin%
\definecolor{currentfill}{rgb}{0.150000,0.150000,0.150000}%
\pgfsetfillcolor{currentfill}%
\pgfsetlinewidth{1.003750pt}%
\definecolor{currentstroke}{rgb}{0.150000,0.150000,0.150000}%
\pgfsetstrokecolor{currentstroke}%
\pgfsetdash{}{0pt}%
\pgfsys@defobject{currentmarker}{\pgfqpoint{-0.066667in}{0.000000in}}{\pgfqpoint{0.000000in}{0.000000in}}{%
\pgfpathmoveto{\pgfqpoint{0.000000in}{0.000000in}}%
\pgfpathlineto{\pgfqpoint{-0.066667in}{0.000000in}}%
\pgfusepath{stroke,fill}%
}%
\begin{pgfscope}%
\pgfsys@transformshift{5.105513in}{2.221402in}%
\pgfsys@useobject{currentmarker}{}%
\end{pgfscope}%
\end{pgfscope}%
\begin{pgfscope}%
\pgfsetbuttcap%
\pgfsetroundjoin%
\definecolor{currentfill}{rgb}{0.150000,0.150000,0.150000}%
\pgfsetfillcolor{currentfill}%
\pgfsetlinewidth{1.003750pt}%
\definecolor{currentstroke}{rgb}{0.150000,0.150000,0.150000}%
\pgfsetstrokecolor{currentstroke}%
\pgfsetdash{}{0pt}%
\pgfsys@defobject{currentmarker}{\pgfqpoint{-0.066667in}{0.000000in}}{\pgfqpoint{0.000000in}{0.000000in}}{%
\pgfpathmoveto{\pgfqpoint{0.000000in}{0.000000in}}%
\pgfpathlineto{\pgfqpoint{-0.066667in}{0.000000in}}%
\pgfusepath{stroke,fill}%
}%
\begin{pgfscope}%
\pgfsys@transformshift{5.105513in}{2.502456in}%
\pgfsys@useobject{currentmarker}{}%
\end{pgfscope}%
\end{pgfscope}%
\begin{pgfscope}%
\pgfpathrectangle{\pgfqpoint{5.105513in}{1.894508in}}{\pgfqpoint{1.223103in}{0.607948in}}%
\pgfusepath{clip}%
\pgfsetroundcap%
\pgfsetroundjoin%
\pgfsetlinewidth{1.204500pt}%
\definecolor{currentstroke}{rgb}{0.000000,0.501961,0.000000}%
\pgfsetstrokecolor{currentstroke}%
\pgfsetdash{}{0pt}%
\pgfpathmoveto{\pgfqpoint{5.105513in}{2.221616in}}%
\pgfpathlineto{\pgfqpoint{5.346222in}{2.222338in}}%
\pgfpathlineto{\pgfqpoint{5.457753in}{2.223049in}}%
\pgfpathlineto{\pgfqpoint{5.531149in}{2.223752in}}%
\pgfpathlineto{\pgfqpoint{5.585945in}{2.224447in}}%
\pgfpathlineto{\pgfqpoint{5.629687in}{2.225137in}}%
\pgfpathlineto{\pgfqpoint{5.666095in}{2.225823in}}%
\pgfpathlineto{\pgfqpoint{5.697279in}{2.226505in}}%
\pgfpathlineto{\pgfqpoint{5.724552in}{2.227184in}}%
\pgfpathlineto{\pgfqpoint{5.748786in}{2.227862in}}%
\pgfpathlineto{\pgfqpoint{5.770591in}{2.228540in}}%
\pgfpathlineto{\pgfqpoint{5.790410in}{2.229217in}}%
\pgfpathlineto{\pgfqpoint{5.808575in}{2.229894in}}%
\pgfpathlineto{\pgfqpoint{5.825341in}{2.230573in}}%
\pgfpathlineto{\pgfqpoint{5.840907in}{2.231252in}}%
\pgfpathlineto{\pgfqpoint{5.855435in}{2.231932in}}%
\pgfpathlineto{\pgfqpoint{5.869053in}{2.232614in}}%
\pgfpathlineto{\pgfqpoint{5.881870in}{2.233298in}}%
\pgfpathlineto{\pgfqpoint{5.893974in}{2.233983in}}%
\pgfpathlineto{\pgfqpoint{5.905441in}{2.234670in}}%
\pgfpathlineto{\pgfqpoint{5.916334in}{2.235358in}}%
\pgfpathlineto{\pgfqpoint{5.926708in}{2.236047in}}%
\pgfpathlineto{\pgfqpoint{5.936610in}{2.236738in}}%
\pgfpathlineto{\pgfqpoint{5.946082in}{2.237430in}}%
\pgfpathlineto{\pgfqpoint{5.955158in}{2.238124in}}%
\pgfpathlineto{\pgfqpoint{5.963871in}{2.238818in}}%
\pgfpathlineto{\pgfqpoint{5.972249in}{2.239513in}}%
\pgfpathlineto{\pgfqpoint{5.980317in}{2.240209in}}%
\pgfpathlineto{\pgfqpoint{5.988096in}{2.240906in}}%
\pgfpathlineto{\pgfqpoint{5.995607in}{2.241604in}}%
\pgfpathlineto{\pgfqpoint{6.002868in}{2.242303in}}%
\pgfpathlineto{\pgfqpoint{6.009894in}{2.243003in}}%
\pgfpathlineto{\pgfqpoint{6.016701in}{2.243704in}}%
\pgfpathlineto{\pgfqpoint{6.023301in}{2.244407in}}%
\pgfpathlineto{\pgfqpoint{6.029707in}{2.245111in}}%
\pgfpathlineto{\pgfqpoint{6.035930in}{2.245818in}}%
\pgfpathlineto{\pgfqpoint{6.041980in}{2.246528in}}%
\pgfpathlineto{\pgfqpoint{6.047867in}{2.247241in}}%
\pgfpathlineto{\pgfqpoint{6.053598in}{2.247959in}}%
\pgfpathlineto{\pgfqpoint{6.059183in}{2.248682in}}%
\pgfpathlineto{\pgfqpoint{6.064628in}{2.249412in}}%
\pgfpathlineto{\pgfqpoint{6.069940in}{2.250149in}}%
\pgfpathlineto{\pgfqpoint{6.075126in}{2.250895in}}%
\pgfpathlineto{\pgfqpoint{6.080191in}{2.251651in}}%
\pgfpathlineto{\pgfqpoint{6.085140in}{2.252420in}}%
\pgfpathlineto{\pgfqpoint{6.089980in}{2.253202in}}%
\pgfpathlineto{\pgfqpoint{6.094715in}{2.254001in}}%
\pgfpathlineto{\pgfqpoint{6.099349in}{2.254819in}}%
\pgfpathlineto{\pgfqpoint{6.103886in}{2.255657in}}%
\pgfpathlineto{\pgfqpoint{6.108331in}{2.256519in}}%
\pgfusepath{stroke}%
\end{pgfscope}%
\begin{pgfscope}%
\pgfsetrectcap%
\pgfsetmiterjoin%
\pgfsetlinewidth{1.003750pt}%
\definecolor{currentstroke}{rgb}{0.150000,0.150000,0.150000}%
\pgfsetstrokecolor{currentstroke}%
\pgfsetdash{}{0pt}%
\pgfpathmoveto{\pgfqpoint{5.105513in}{1.894508in}}%
\pgfpathlineto{\pgfqpoint{5.105513in}{2.502456in}}%
\pgfusepath{stroke}%
\end{pgfscope}%
\begin{pgfscope}%
\pgfsetrectcap%
\pgfsetmiterjoin%
\pgfsetlinewidth{1.003750pt}%
\definecolor{currentstroke}{rgb}{0.150000,0.150000,0.150000}%
\pgfsetstrokecolor{currentstroke}%
\pgfsetdash{}{0pt}%
\pgfpathmoveto{\pgfqpoint{5.105513in}{1.894508in}}%
\pgfpathlineto{\pgfqpoint{6.328616in}{1.894508in}}%
\pgfusepath{stroke}%
\end{pgfscope}%
\begin{pgfscope}%
\pgfpathrectangle{\pgfqpoint{5.105513in}{1.894508in}}{\pgfqpoint{1.223103in}{0.607948in}}%
\pgfusepath{clip}%
\pgfsetbuttcap%
\pgfsetroundjoin%
\definecolor{currentfill}{rgb}{0.000000,0.000000,0.000000}%
\pgfsetfillcolor{currentfill}%
\pgfsetlinewidth{1.003750pt}%
\definecolor{currentstroke}{rgb}{0.000000,0.000000,0.000000}%
\pgfsetstrokecolor{currentstroke}%
\pgfsetdash{}{0pt}%
\pgfsys@defobject{currentmarker}{\pgfqpoint{-0.013889in}{-0.013889in}}{\pgfqpoint{0.013889in}{0.013889in}}{%
\pgfpathmoveto{\pgfqpoint{0.000000in}{-0.013889in}}%
\pgfpathcurveto{\pgfqpoint{0.003683in}{-0.013889in}}{\pgfqpoint{0.007216in}{-0.012425in}}{\pgfqpoint{0.009821in}{-0.009821in}}%
\pgfpathcurveto{\pgfqpoint{0.012425in}{-0.007216in}}{\pgfqpoint{0.013889in}{-0.003683in}}{\pgfqpoint{0.013889in}{0.000000in}}%
\pgfpathcurveto{\pgfqpoint{0.013889in}{0.003683in}}{\pgfqpoint{0.012425in}{0.007216in}}{\pgfqpoint{0.009821in}{0.009821in}}%
\pgfpathcurveto{\pgfqpoint{0.007216in}{0.012425in}}{\pgfqpoint{0.003683in}{0.013889in}}{\pgfqpoint{0.000000in}{0.013889in}}%
\pgfpathcurveto{\pgfqpoint{-0.003683in}{0.013889in}}{\pgfqpoint{-0.007216in}{0.012425in}}{\pgfqpoint{-0.009821in}{0.009821in}}%
\pgfpathcurveto{\pgfqpoint{-0.012425in}{0.007216in}}{\pgfqpoint{-0.013889in}{0.003683in}}{\pgfqpoint{-0.013889in}{0.000000in}}%
\pgfpathcurveto{\pgfqpoint{-0.013889in}{-0.003683in}}{\pgfqpoint{-0.012425in}{-0.007216in}}{\pgfqpoint{-0.009821in}{-0.009821in}}%
\pgfpathcurveto{\pgfqpoint{-0.007216in}{-0.012425in}}{\pgfqpoint{-0.003683in}{-0.013889in}}{\pgfqpoint{0.000000in}{-0.013889in}}%
\pgfpathclose%
\pgfusepath{stroke,fill}%
}%
\begin{pgfscope}%
\pgfsys@transformshift{5.757861in}{2.228132in}%
\pgfsys@useobject{currentmarker}{}%
\end{pgfscope}%
\begin{pgfscope}%
\pgfsys@transformshift{5.762260in}{2.228266in}%
\pgfsys@useobject{currentmarker}{}%
\end{pgfscope}%
\begin{pgfscope}%
\pgfsys@transformshift{5.766750in}{2.228413in}%
\pgfsys@useobject{currentmarker}{}%
\end{pgfscope}%
\begin{pgfscope}%
\pgfsys@transformshift{5.771335in}{2.228558in}%
\pgfsys@useobject{currentmarker}{}%
\end{pgfscope}%
\begin{pgfscope}%
\pgfsys@transformshift{5.776018in}{2.228718in}%
\pgfsys@useobject{currentmarker}{}%
\end{pgfscope}%
\begin{pgfscope}%
\pgfsys@transformshift{5.780804in}{2.228876in}%
\pgfsys@useobject{currentmarker}{}%
\end{pgfscope}%
\begin{pgfscope}%
\pgfsys@transformshift{5.785698in}{2.229052in}%
\pgfsys@useobject{currentmarker}{}%
\end{pgfscope}%
\begin{pgfscope}%
\pgfsys@transformshift{5.790704in}{2.229224in}%
\pgfsys@useobject{currentmarker}{}%
\end{pgfscope}%
\begin{pgfscope}%
\pgfsys@transformshift{5.795828in}{2.229417in}%
\pgfsys@useobject{currentmarker}{}%
\end{pgfscope}%
\begin{pgfscope}%
\pgfsys@transformshift{5.801075in}{2.229606in}%
\pgfsys@useobject{currentmarker}{}%
\end{pgfscope}%
\begin{pgfscope}%
\pgfsys@transformshift{5.806452in}{2.229819in}%
\pgfsys@useobject{currentmarker}{}%
\end{pgfscope}%
\begin{pgfscope}%
\pgfsys@transformshift{5.835530in}{2.231015in}%
\pgfsys@useobject{currentmarker}{}%
\end{pgfscope}%
\begin{pgfscope}%
\pgfsys@transformshift{5.869098in}{2.232640in}%
\pgfsys@useobject{currentmarker}{}%
\end{pgfscope}%
\begin{pgfscope}%
\pgfsys@transformshift{5.908800in}{2.234878in}%
\pgfsys@useobject{currentmarker}{}%
\end{pgfscope}%
\begin{pgfscope}%
\pgfsys@transformshift{5.957391in}{2.238344in}%
\pgfsys@useobject{currentmarker}{}%
\end{pgfscope}%
\begin{pgfscope}%
\pgfsys@transformshift{6.020037in}{2.244024in}%
\pgfsys@useobject{currentmarker}{}%
\end{pgfscope}%
\begin{pgfscope}%
\pgfsys@transformshift{6.108331in}{2.256533in}%
\pgfsys@useobject{currentmarker}{}%
\end{pgfscope}%
\end{pgfscope}%
\begin{pgfscope}%
\pgfsetbuttcap%
\pgfsetmiterjoin%
\definecolor{currentfill}{rgb}{1.000000,1.000000,1.000000}%
\pgfsetfillcolor{currentfill}%
\pgfsetlinewidth{0.803000pt}%
\definecolor{currentstroke}{rgb}{1.000000,1.000000,1.000000}%
\pgfsetstrokecolor{currentstroke}%
\pgfsetdash{}{0pt}%
\pgfpathmoveto{\pgfqpoint{6.297392in}{2.432757in}}%
\pgfpathlineto{\pgfqpoint{6.297392in}{1.964207in}}%
\pgfpathlineto{\pgfqpoint{6.478411in}{1.964207in}}%
\pgfpathlineto{\pgfqpoint{6.478411in}{2.432757in}}%
\pgfpathclose%
\pgfusepath{stroke,fill}%
\end{pgfscope}%
\begin{pgfscope}%
\definecolor{textcolor}{rgb}{0.150000,0.150000,0.150000}%
\pgfsetstrokecolor{textcolor}%
\pgfsetfillcolor{textcolor}%
\pgftext[x=6.368294in,y=2.377070in,left,base,rotate=270.000000]{\color{textcolor}\sffamily\fontsize{5.647059}{6.776471}\selectfont nlevel = 6}%
\end{pgfscope}%
\begin{pgfscope}%
\pgfsetbuttcap%
\pgfsetmiterjoin%
\definecolor{currentfill}{rgb}{1.000000,1.000000,1.000000}%
\pgfsetfillcolor{currentfill}%
\pgfsetlinewidth{0.803000pt}%
\definecolor{currentstroke}{rgb}{1.000000,1.000000,1.000000}%
\pgfsetstrokecolor{currentstroke}%
\pgfsetdash{}{0pt}%
\pgfpathmoveto{\pgfqpoint{6.297392in}{2.432757in}}%
\pgfpathlineto{\pgfqpoint{6.297392in}{1.964207in}}%
\pgfpathlineto{\pgfqpoint{6.478411in}{1.964207in}}%
\pgfpathlineto{\pgfqpoint{6.478411in}{2.432757in}}%
\pgfpathclose%
\pgfusepath{stroke,fill}%
\end{pgfscope}%
\begin{pgfscope}%
\definecolor{textcolor}{rgb}{0.150000,0.150000,0.150000}%
\pgfsetstrokecolor{textcolor}%
\pgfsetfillcolor{textcolor}%
\pgftext[x=6.368294in,y=2.377070in,left,base,rotate=270.000000]{\color{textcolor}\sffamily\fontsize{5.647059}{6.776471}\selectfont nlevel = 6}%
\end{pgfscope}%
\begin{pgfscope}%
\pgfsetbuttcap%
\pgfsetmiterjoin%
\definecolor{currentfill}{rgb}{1.000000,1.000000,1.000000}%
\pgfsetfillcolor{currentfill}%
\pgfsetlinewidth{0.000000pt}%
\definecolor{currentstroke}{rgb}{0.000000,0.000000,0.000000}%
\pgfsetstrokecolor{currentstroke}%
\pgfsetstrokeopacity{0.000000}%
\pgfsetdash{}{0pt}%
\pgfpathmoveto{\pgfqpoint{0.702340in}{1.164970in}}%
\pgfpathlineto{\pgfqpoint{1.925444in}{1.164970in}}%
\pgfpathlineto{\pgfqpoint{1.925444in}{1.772918in}}%
\pgfpathlineto{\pgfqpoint{0.702340in}{1.772918in}}%
\pgfpathclose%
\pgfusepath{fill}%
\end{pgfscope}%
\begin{pgfscope}%
\pgfpathrectangle{\pgfqpoint{0.702340in}{1.164970in}}{\pgfqpoint{1.223103in}{0.607948in}}%
\pgfusepath{clip}%
\pgfsetbuttcap%
\pgfsetmiterjoin%
\definecolor{currentfill}{rgb}{0.000000,0.000000,1.000000}%
\pgfsetfillcolor{currentfill}%
\pgfsetfillopacity{0.100000}%
\pgfsetlinewidth{0.803000pt}%
\definecolor{currentstroke}{rgb}{0.000000,0.000000,1.000000}%
\pgfsetstrokecolor{currentstroke}%
\pgfsetstrokeopacity{0.100000}%
\pgfsetdash{}{0pt}%
\pgfpathmoveto{\pgfqpoint{0.702340in}{1.490935in}}%
\pgfpathlineto{\pgfqpoint{0.702340in}{1.492886in}}%
\pgfpathlineto{\pgfqpoint{1.925444in}{1.492886in}}%
\pgfpathlineto{\pgfqpoint{1.925444in}{1.490935in}}%
\pgfpathclose%
\pgfusepath{stroke,fill}%
\end{pgfscope}%
\begin{pgfscope}%
\pgfpathrectangle{\pgfqpoint{0.702340in}{1.164970in}}{\pgfqpoint{1.223103in}{0.607948in}}%
\pgfusepath{clip}%
\pgfsetbuttcap%
\pgfsetroundjoin%
\definecolor{currentfill}{rgb}{0.000000,0.501961,0.000000}%
\pgfsetfillcolor{currentfill}%
\pgfsetfillopacity{0.500000}%
\pgfsetlinewidth{0.803000pt}%
\definecolor{currentstroke}{rgb}{0.000000,0.501961,0.000000}%
\pgfsetstrokecolor{currentstroke}%
\pgfsetstrokeopacity{0.500000}%
\pgfsetdash{}{0pt}%
\pgfpathmoveto{\pgfqpoint{0.702340in}{1.492239in}}%
\pgfpathlineto{\pgfqpoint{0.702340in}{1.490441in}}%
\pgfpathlineto{\pgfqpoint{0.943050in}{1.489331in}}%
\pgfpathlineto{\pgfqpoint{1.054581in}{1.488076in}}%
\pgfpathlineto{\pgfqpoint{1.127977in}{1.486676in}}%
\pgfpathlineto{\pgfqpoint{1.182772in}{1.485134in}}%
\pgfpathlineto{\pgfqpoint{1.226515in}{1.483452in}}%
\pgfpathlineto{\pgfqpoint{1.262923in}{1.481631in}}%
\pgfpathlineto{\pgfqpoint{1.294107in}{1.479674in}}%
\pgfpathlineto{\pgfqpoint{1.321380in}{1.477581in}}%
\pgfpathlineto{\pgfqpoint{1.345614in}{1.475355in}}%
\pgfpathlineto{\pgfqpoint{1.367419in}{1.472998in}}%
\pgfpathlineto{\pgfqpoint{1.387238in}{1.470492in}}%
\pgfpathlineto{\pgfqpoint{1.405403in}{1.467823in}}%
\pgfpathlineto{\pgfqpoint{1.422168in}{1.465040in}}%
\pgfpathlineto{\pgfqpoint{1.437735in}{1.462143in}}%
\pgfpathlineto{\pgfqpoint{1.452262in}{1.459134in}}%
\pgfpathlineto{\pgfqpoint{1.465881in}{1.456011in}}%
\pgfpathlineto{\pgfqpoint{1.478698in}{1.452777in}}%
\pgfpathlineto{\pgfqpoint{1.490802in}{1.449430in}}%
\pgfpathlineto{\pgfqpoint{1.502269in}{1.445971in}}%
\pgfpathlineto{\pgfqpoint{1.513162in}{1.442401in}}%
\pgfpathlineto{\pgfqpoint{1.523536in}{1.438719in}}%
\pgfpathlineto{\pgfqpoint{1.533438in}{1.434926in}}%
\pgfpathlineto{\pgfqpoint{1.542909in}{1.431021in}}%
\pgfpathlineto{\pgfqpoint{1.551986in}{1.426995in}}%
\pgfpathlineto{\pgfqpoint{1.560699in}{1.422841in}}%
\pgfpathlineto{\pgfqpoint{1.569077in}{1.418573in}}%
\pgfpathlineto{\pgfqpoint{1.577145in}{1.414195in}}%
\pgfpathlineto{\pgfqpoint{1.584924in}{1.409708in}}%
\pgfpathlineto{\pgfqpoint{1.592435in}{1.405113in}}%
\pgfpathlineto{\pgfqpoint{1.599696in}{1.400411in}}%
\pgfpathlineto{\pgfqpoint{1.606722in}{1.395604in}}%
\pgfpathlineto{\pgfqpoint{1.613528in}{1.390691in}}%
\pgfpathlineto{\pgfqpoint{1.620129in}{1.385674in}}%
\pgfpathlineto{\pgfqpoint{1.626535in}{1.380553in}}%
\pgfpathlineto{\pgfqpoint{1.632758in}{1.375328in}}%
\pgfpathlineto{\pgfqpoint{1.638808in}{1.370002in}}%
\pgfpathlineto{\pgfqpoint{1.644694in}{1.364573in}}%
\pgfpathlineto{\pgfqpoint{1.650426in}{1.359043in}}%
\pgfpathlineto{\pgfqpoint{1.656011in}{1.353412in}}%
\pgfpathlineto{\pgfqpoint{1.661456in}{1.347681in}}%
\pgfpathlineto{\pgfqpoint{1.666768in}{1.341849in}}%
\pgfpathlineto{\pgfqpoint{1.671953in}{1.335918in}}%
\pgfpathlineto{\pgfqpoint{1.677018in}{1.329887in}}%
\pgfpathlineto{\pgfqpoint{1.681968in}{1.323757in}}%
\pgfpathlineto{\pgfqpoint{1.686808in}{1.317529in}}%
\pgfpathlineto{\pgfqpoint{1.691543in}{1.311202in}}%
\pgfpathlineto{\pgfqpoint{1.696176in}{1.304777in}}%
\pgfpathlineto{\pgfqpoint{1.700714in}{1.298249in}}%
\pgfpathlineto{\pgfqpoint{1.705158in}{1.291572in}}%
\pgfpathlineto{\pgfqpoint{1.705158in}{1.291640in}}%
\pgfpathlineto{\pgfqpoint{1.705158in}{1.291640in}}%
\pgfpathlineto{\pgfqpoint{1.700714in}{1.298288in}}%
\pgfpathlineto{\pgfqpoint{1.696176in}{1.304881in}}%
\pgfpathlineto{\pgfqpoint{1.691543in}{1.311370in}}%
\pgfpathlineto{\pgfqpoint{1.686808in}{1.317751in}}%
\pgfpathlineto{\pgfqpoint{1.681968in}{1.324026in}}%
\pgfpathlineto{\pgfqpoint{1.677018in}{1.330194in}}%
\pgfpathlineto{\pgfqpoint{1.671953in}{1.336255in}}%
\pgfpathlineto{\pgfqpoint{1.666768in}{1.342209in}}%
\pgfpathlineto{\pgfqpoint{1.661456in}{1.348056in}}%
\pgfpathlineto{\pgfqpoint{1.656011in}{1.353797in}}%
\pgfpathlineto{\pgfqpoint{1.650426in}{1.359430in}}%
\pgfpathlineto{\pgfqpoint{1.644694in}{1.364957in}}%
\pgfpathlineto{\pgfqpoint{1.638808in}{1.370377in}}%
\pgfpathlineto{\pgfqpoint{1.632758in}{1.375690in}}%
\pgfpathlineto{\pgfqpoint{1.626535in}{1.380896in}}%
\pgfpathlineto{\pgfqpoint{1.620129in}{1.385995in}}%
\pgfpathlineto{\pgfqpoint{1.613528in}{1.390986in}}%
\pgfpathlineto{\pgfqpoint{1.606722in}{1.395869in}}%
\pgfpathlineto{\pgfqpoint{1.599696in}{1.400644in}}%
\pgfpathlineto{\pgfqpoint{1.592435in}{1.405312in}}%
\pgfpathlineto{\pgfqpoint{1.584924in}{1.409871in}}%
\pgfpathlineto{\pgfqpoint{1.577145in}{1.414321in}}%
\pgfpathlineto{\pgfqpoint{1.569077in}{1.418663in}}%
\pgfpathlineto{\pgfqpoint{1.560699in}{1.422896in}}%
\pgfpathlineto{\pgfqpoint{1.551986in}{1.427024in}}%
\pgfpathlineto{\pgfqpoint{1.542909in}{1.431059in}}%
\pgfpathlineto{\pgfqpoint{1.533438in}{1.434991in}}%
\pgfpathlineto{\pgfqpoint{1.523536in}{1.438812in}}%
\pgfpathlineto{\pgfqpoint{1.513162in}{1.442519in}}%
\pgfpathlineto{\pgfqpoint{1.502269in}{1.446110in}}%
\pgfpathlineto{\pgfqpoint{1.490802in}{1.449584in}}%
\pgfpathlineto{\pgfqpoint{1.478698in}{1.452939in}}%
\pgfpathlineto{\pgfqpoint{1.465881in}{1.456175in}}%
\pgfpathlineto{\pgfqpoint{1.452262in}{1.459290in}}%
\pgfpathlineto{\pgfqpoint{1.437735in}{1.462282in}}%
\pgfpathlineto{\pgfqpoint{1.422168in}{1.465151in}}%
\pgfpathlineto{\pgfqpoint{1.405403in}{1.467894in}}%
\pgfpathlineto{\pgfqpoint{1.387238in}{1.470510in}}%
\pgfpathlineto{\pgfqpoint{1.367419in}{1.473047in}}%
\pgfpathlineto{\pgfqpoint{1.345614in}{1.475487in}}%
\pgfpathlineto{\pgfqpoint{1.321380in}{1.477813in}}%
\pgfpathlineto{\pgfqpoint{1.294107in}{1.480023in}}%
\pgfpathlineto{\pgfqpoint{1.262923in}{1.482118in}}%
\pgfpathlineto{\pgfqpoint{1.226515in}{1.484097in}}%
\pgfpathlineto{\pgfqpoint{1.182772in}{1.485960in}}%
\pgfpathlineto{\pgfqpoint{1.127977in}{1.487705in}}%
\pgfpathlineto{\pgfqpoint{1.054581in}{1.489334in}}%
\pgfpathlineto{\pgfqpoint{0.943050in}{1.490846in}}%
\pgfpathlineto{\pgfqpoint{0.702340in}{1.492239in}}%
\pgfpathclose%
\pgfusepath{stroke,fill}%
\end{pgfscope}%
\begin{pgfscope}%
\pgfpathrectangle{\pgfqpoint{0.702340in}{1.164970in}}{\pgfqpoint{1.223103in}{0.607948in}}%
\pgfusepath{clip}%
\pgfsetroundcap%
\pgfsetroundjoin%
\pgfsetlinewidth{0.501875pt}%
\definecolor{currentstroke}{rgb}{0.000000,0.000000,1.000000}%
\pgfsetstrokecolor{currentstroke}%
\pgfsetstrokeopacity{0.800000}%
\pgfsetdash{}{0pt}%
\pgfpathmoveto{\pgfqpoint{0.702340in}{1.491911in}}%
\pgfpathlineto{\pgfqpoint{1.925444in}{1.491911in}}%
\pgfusepath{stroke}%
\end{pgfscope}%
\begin{pgfscope}%
\pgfpathrectangle{\pgfqpoint{0.702340in}{1.164970in}}{\pgfqpoint{1.223103in}{0.607948in}}%
\pgfusepath{clip}%
\pgfsetbuttcap%
\pgfsetroundjoin%
\pgfsetlinewidth{1.003750pt}%
\definecolor{currentstroke}{rgb}{0.000000,0.000000,0.000000}%
\pgfsetstrokecolor{currentstroke}%
\pgfsetdash{{3.700000pt}{1.600000pt}}{0.000000pt}%
\pgfpathmoveto{\pgfqpoint{0.702340in}{1.491207in}}%
\pgfpathlineto{\pgfqpoint{1.925444in}{1.491207in}}%
\pgfusepath{stroke}%
\end{pgfscope}%
\begin{pgfscope}%
\pgfsetroundcap%
\pgfsetroundjoin%
\pgfsetlinewidth{0.501875pt}%
\definecolor{currentstroke}{rgb}{0.000000,0.000000,1.000000}%
\pgfsetstrokecolor{currentstroke}%
\pgfsetstrokeopacity{0.800000}%
\pgfsetdash{}{0pt}%
\pgfpathmoveto{\pgfqpoint{1.516189in}{1.609247in}}%
\pgfpathquadraticcurveto{\pgfqpoint{1.446671in}{1.558740in}}{\pgfqpoint{1.377153in}{1.508232in}}%
\pgfusepath{stroke}%
\end{pgfscope}%
\begin{pgfscope}%
\pgfsetfillopacity{0.800000}%
\pgfsetstrokeopacity{0.800000}%
\definecolor{textcolor}{rgb}{0.000000,0.000000,1.000000}%
\pgfsetstrokecolor{textcolor}%
\pgfsetfillcolor{textcolor}%
\pgftext[x=1.442982in,y=1.674295in,left,base]{\color{textcolor}\sffamily\fontsize{5.647059}{6.776471}\selectfont 3.5378(16)}%
\end{pgfscope}%
\begin{pgfscope}%
\pgfsetbuttcap%
\pgfsetroundjoin%
\definecolor{currentfill}{rgb}{0.150000,0.150000,0.150000}%
\pgfsetfillcolor{currentfill}%
\pgfsetlinewidth{1.003750pt}%
\definecolor{currentstroke}{rgb}{0.150000,0.150000,0.150000}%
\pgfsetstrokecolor{currentstroke}%
\pgfsetdash{}{0pt}%
\pgfsys@defobject{currentmarker}{\pgfqpoint{0.000000in}{-0.066667in}}{\pgfqpoint{0.000000in}{0.000000in}}{%
\pgfpathmoveto{\pgfqpoint{0.000000in}{0.000000in}}%
\pgfpathlineto{\pgfqpoint{0.000000in}{-0.066667in}}%
\pgfusepath{stroke,fill}%
}%
\begin{pgfscope}%
\pgfsys@transformshift{0.702340in}{1.164970in}%
\pgfsys@useobject{currentmarker}{}%
\end{pgfscope}%
\end{pgfscope}%
\begin{pgfscope}%
\pgfsetbuttcap%
\pgfsetroundjoin%
\definecolor{currentfill}{rgb}{0.150000,0.150000,0.150000}%
\pgfsetfillcolor{currentfill}%
\pgfsetlinewidth{1.003750pt}%
\definecolor{currentstroke}{rgb}{0.150000,0.150000,0.150000}%
\pgfsetstrokecolor{currentstroke}%
\pgfsetdash{}{0pt}%
\pgfsys@defobject{currentmarker}{\pgfqpoint{0.000000in}{-0.066667in}}{\pgfqpoint{0.000000in}{0.000000in}}{%
\pgfpathmoveto{\pgfqpoint{0.000000in}{0.000000in}}%
\pgfpathlineto{\pgfqpoint{0.000000in}{-0.066667in}}%
\pgfusepath{stroke,fill}%
}%
\begin{pgfscope}%
\pgfsys@transformshift{1.203749in}{1.164970in}%
\pgfsys@useobject{currentmarker}{}%
\end{pgfscope}%
\end{pgfscope}%
\begin{pgfscope}%
\pgfsetbuttcap%
\pgfsetroundjoin%
\definecolor{currentfill}{rgb}{0.150000,0.150000,0.150000}%
\pgfsetfillcolor{currentfill}%
\pgfsetlinewidth{1.003750pt}%
\definecolor{currentstroke}{rgb}{0.150000,0.150000,0.150000}%
\pgfsetstrokecolor{currentstroke}%
\pgfsetdash{}{0pt}%
\pgfsys@defobject{currentmarker}{\pgfqpoint{0.000000in}{-0.066667in}}{\pgfqpoint{0.000000in}{0.000000in}}{%
\pgfpathmoveto{\pgfqpoint{0.000000in}{0.000000in}}%
\pgfpathlineto{\pgfqpoint{0.000000in}{-0.066667in}}%
\pgfusepath{stroke,fill}%
}%
\begin{pgfscope}%
\pgfsys@transformshift{1.705158in}{1.164970in}%
\pgfsys@useobject{currentmarker}{}%
\end{pgfscope}%
\end{pgfscope}%
\begin{pgfscope}%
\pgfsetbuttcap%
\pgfsetroundjoin%
\definecolor{currentfill}{rgb}{0.150000,0.150000,0.150000}%
\pgfsetfillcolor{currentfill}%
\pgfsetlinewidth{0.803000pt}%
\definecolor{currentstroke}{rgb}{0.150000,0.150000,0.150000}%
\pgfsetstrokecolor{currentstroke}%
\pgfsetdash{}{0pt}%
\pgfsys@defobject{currentmarker}{\pgfqpoint{0.000000in}{-0.044444in}}{\pgfqpoint{0.000000in}{0.000000in}}{%
\pgfpathmoveto{\pgfqpoint{0.000000in}{0.000000in}}%
\pgfpathlineto{\pgfqpoint{0.000000in}{-0.044444in}}%
\pgfusepath{stroke,fill}%
}%
\begin{pgfscope}%
\pgfsys@transformshift{0.853280in}{1.164970in}%
\pgfsys@useobject{currentmarker}{}%
\end{pgfscope}%
\end{pgfscope}%
\begin{pgfscope}%
\pgfsetbuttcap%
\pgfsetroundjoin%
\definecolor{currentfill}{rgb}{0.150000,0.150000,0.150000}%
\pgfsetfillcolor{currentfill}%
\pgfsetlinewidth{0.803000pt}%
\definecolor{currentstroke}{rgb}{0.150000,0.150000,0.150000}%
\pgfsetstrokecolor{currentstroke}%
\pgfsetdash{}{0pt}%
\pgfsys@defobject{currentmarker}{\pgfqpoint{0.000000in}{-0.044444in}}{\pgfqpoint{0.000000in}{0.000000in}}{%
\pgfpathmoveto{\pgfqpoint{0.000000in}{0.000000in}}%
\pgfpathlineto{\pgfqpoint{0.000000in}{-0.044444in}}%
\pgfusepath{stroke,fill}%
}%
\begin{pgfscope}%
\pgfsys@transformshift{0.941573in}{1.164970in}%
\pgfsys@useobject{currentmarker}{}%
\end{pgfscope}%
\end{pgfscope}%
\begin{pgfscope}%
\pgfsetbuttcap%
\pgfsetroundjoin%
\definecolor{currentfill}{rgb}{0.150000,0.150000,0.150000}%
\pgfsetfillcolor{currentfill}%
\pgfsetlinewidth{0.803000pt}%
\definecolor{currentstroke}{rgb}{0.150000,0.150000,0.150000}%
\pgfsetstrokecolor{currentstroke}%
\pgfsetdash{}{0pt}%
\pgfsys@defobject{currentmarker}{\pgfqpoint{0.000000in}{-0.044444in}}{\pgfqpoint{0.000000in}{0.000000in}}{%
\pgfpathmoveto{\pgfqpoint{0.000000in}{0.000000in}}%
\pgfpathlineto{\pgfqpoint{0.000000in}{-0.044444in}}%
\pgfusepath{stroke,fill}%
}%
\begin{pgfscope}%
\pgfsys@transformshift{1.004219in}{1.164970in}%
\pgfsys@useobject{currentmarker}{}%
\end{pgfscope}%
\end{pgfscope}%
\begin{pgfscope}%
\pgfsetbuttcap%
\pgfsetroundjoin%
\definecolor{currentfill}{rgb}{0.150000,0.150000,0.150000}%
\pgfsetfillcolor{currentfill}%
\pgfsetlinewidth{0.803000pt}%
\definecolor{currentstroke}{rgb}{0.150000,0.150000,0.150000}%
\pgfsetstrokecolor{currentstroke}%
\pgfsetdash{}{0pt}%
\pgfsys@defobject{currentmarker}{\pgfqpoint{0.000000in}{-0.044444in}}{\pgfqpoint{0.000000in}{0.000000in}}{%
\pgfpathmoveto{\pgfqpoint{0.000000in}{0.000000in}}%
\pgfpathlineto{\pgfqpoint{0.000000in}{-0.044444in}}%
\pgfusepath{stroke,fill}%
}%
\begin{pgfscope}%
\pgfsys@transformshift{1.052810in}{1.164970in}%
\pgfsys@useobject{currentmarker}{}%
\end{pgfscope}%
\end{pgfscope}%
\begin{pgfscope}%
\pgfsetbuttcap%
\pgfsetroundjoin%
\definecolor{currentfill}{rgb}{0.150000,0.150000,0.150000}%
\pgfsetfillcolor{currentfill}%
\pgfsetlinewidth{0.803000pt}%
\definecolor{currentstroke}{rgb}{0.150000,0.150000,0.150000}%
\pgfsetstrokecolor{currentstroke}%
\pgfsetdash{}{0pt}%
\pgfsys@defobject{currentmarker}{\pgfqpoint{0.000000in}{-0.044444in}}{\pgfqpoint{0.000000in}{0.000000in}}{%
\pgfpathmoveto{\pgfqpoint{0.000000in}{0.000000in}}%
\pgfpathlineto{\pgfqpoint{0.000000in}{-0.044444in}}%
\pgfusepath{stroke,fill}%
}%
\begin{pgfscope}%
\pgfsys@transformshift{1.092512in}{1.164970in}%
\pgfsys@useobject{currentmarker}{}%
\end{pgfscope}%
\end{pgfscope}%
\begin{pgfscope}%
\pgfsetbuttcap%
\pgfsetroundjoin%
\definecolor{currentfill}{rgb}{0.150000,0.150000,0.150000}%
\pgfsetfillcolor{currentfill}%
\pgfsetlinewidth{0.803000pt}%
\definecolor{currentstroke}{rgb}{0.150000,0.150000,0.150000}%
\pgfsetstrokecolor{currentstroke}%
\pgfsetdash{}{0pt}%
\pgfsys@defobject{currentmarker}{\pgfqpoint{0.000000in}{-0.044444in}}{\pgfqpoint{0.000000in}{0.000000in}}{%
\pgfpathmoveto{\pgfqpoint{0.000000in}{0.000000in}}%
\pgfpathlineto{\pgfqpoint{0.000000in}{-0.044444in}}%
\pgfusepath{stroke,fill}%
}%
\begin{pgfscope}%
\pgfsys@transformshift{1.126080in}{1.164970in}%
\pgfsys@useobject{currentmarker}{}%
\end{pgfscope}%
\end{pgfscope}%
\begin{pgfscope}%
\pgfsetbuttcap%
\pgfsetroundjoin%
\definecolor{currentfill}{rgb}{0.150000,0.150000,0.150000}%
\pgfsetfillcolor{currentfill}%
\pgfsetlinewidth{0.803000pt}%
\definecolor{currentstroke}{rgb}{0.150000,0.150000,0.150000}%
\pgfsetstrokecolor{currentstroke}%
\pgfsetdash{}{0pt}%
\pgfsys@defobject{currentmarker}{\pgfqpoint{0.000000in}{-0.044444in}}{\pgfqpoint{0.000000in}{0.000000in}}{%
\pgfpathmoveto{\pgfqpoint{0.000000in}{0.000000in}}%
\pgfpathlineto{\pgfqpoint{0.000000in}{-0.044444in}}%
\pgfusepath{stroke,fill}%
}%
\begin{pgfscope}%
\pgfsys@transformshift{1.155158in}{1.164970in}%
\pgfsys@useobject{currentmarker}{}%
\end{pgfscope}%
\end{pgfscope}%
\begin{pgfscope}%
\pgfsetbuttcap%
\pgfsetroundjoin%
\definecolor{currentfill}{rgb}{0.150000,0.150000,0.150000}%
\pgfsetfillcolor{currentfill}%
\pgfsetlinewidth{0.803000pt}%
\definecolor{currentstroke}{rgb}{0.150000,0.150000,0.150000}%
\pgfsetstrokecolor{currentstroke}%
\pgfsetdash{}{0pt}%
\pgfsys@defobject{currentmarker}{\pgfqpoint{0.000000in}{-0.044444in}}{\pgfqpoint{0.000000in}{0.000000in}}{%
\pgfpathmoveto{\pgfqpoint{0.000000in}{0.000000in}}%
\pgfpathlineto{\pgfqpoint{0.000000in}{-0.044444in}}%
\pgfusepath{stroke,fill}%
}%
\begin{pgfscope}%
\pgfsys@transformshift{1.180806in}{1.164970in}%
\pgfsys@useobject{currentmarker}{}%
\end{pgfscope}%
\end{pgfscope}%
\begin{pgfscope}%
\pgfsetbuttcap%
\pgfsetroundjoin%
\definecolor{currentfill}{rgb}{0.150000,0.150000,0.150000}%
\pgfsetfillcolor{currentfill}%
\pgfsetlinewidth{0.803000pt}%
\definecolor{currentstroke}{rgb}{0.150000,0.150000,0.150000}%
\pgfsetstrokecolor{currentstroke}%
\pgfsetdash{}{0pt}%
\pgfsys@defobject{currentmarker}{\pgfqpoint{0.000000in}{-0.044444in}}{\pgfqpoint{0.000000in}{0.000000in}}{%
\pgfpathmoveto{\pgfqpoint{0.000000in}{0.000000in}}%
\pgfpathlineto{\pgfqpoint{0.000000in}{-0.044444in}}%
\pgfusepath{stroke,fill}%
}%
\begin{pgfscope}%
\pgfsys@transformshift{1.354689in}{1.164970in}%
\pgfsys@useobject{currentmarker}{}%
\end{pgfscope}%
\end{pgfscope}%
\begin{pgfscope}%
\pgfsetbuttcap%
\pgfsetroundjoin%
\definecolor{currentfill}{rgb}{0.150000,0.150000,0.150000}%
\pgfsetfillcolor{currentfill}%
\pgfsetlinewidth{0.803000pt}%
\definecolor{currentstroke}{rgb}{0.150000,0.150000,0.150000}%
\pgfsetstrokecolor{currentstroke}%
\pgfsetdash{}{0pt}%
\pgfsys@defobject{currentmarker}{\pgfqpoint{0.000000in}{-0.044444in}}{\pgfqpoint{0.000000in}{0.000000in}}{%
\pgfpathmoveto{\pgfqpoint{0.000000in}{0.000000in}}%
\pgfpathlineto{\pgfqpoint{0.000000in}{-0.044444in}}%
\pgfusepath{stroke,fill}%
}%
\begin{pgfscope}%
\pgfsys@transformshift{1.442982in}{1.164970in}%
\pgfsys@useobject{currentmarker}{}%
\end{pgfscope}%
\end{pgfscope}%
\begin{pgfscope}%
\pgfsetbuttcap%
\pgfsetroundjoin%
\definecolor{currentfill}{rgb}{0.150000,0.150000,0.150000}%
\pgfsetfillcolor{currentfill}%
\pgfsetlinewidth{0.803000pt}%
\definecolor{currentstroke}{rgb}{0.150000,0.150000,0.150000}%
\pgfsetstrokecolor{currentstroke}%
\pgfsetdash{}{0pt}%
\pgfsys@defobject{currentmarker}{\pgfqpoint{0.000000in}{-0.044444in}}{\pgfqpoint{0.000000in}{0.000000in}}{%
\pgfpathmoveto{\pgfqpoint{0.000000in}{0.000000in}}%
\pgfpathlineto{\pgfqpoint{0.000000in}{-0.044444in}}%
\pgfusepath{stroke,fill}%
}%
\begin{pgfscope}%
\pgfsys@transformshift{1.505628in}{1.164970in}%
\pgfsys@useobject{currentmarker}{}%
\end{pgfscope}%
\end{pgfscope}%
\begin{pgfscope}%
\pgfsetbuttcap%
\pgfsetroundjoin%
\definecolor{currentfill}{rgb}{0.150000,0.150000,0.150000}%
\pgfsetfillcolor{currentfill}%
\pgfsetlinewidth{0.803000pt}%
\definecolor{currentstroke}{rgb}{0.150000,0.150000,0.150000}%
\pgfsetstrokecolor{currentstroke}%
\pgfsetdash{}{0pt}%
\pgfsys@defobject{currentmarker}{\pgfqpoint{0.000000in}{-0.044444in}}{\pgfqpoint{0.000000in}{0.000000in}}{%
\pgfpathmoveto{\pgfqpoint{0.000000in}{0.000000in}}%
\pgfpathlineto{\pgfqpoint{0.000000in}{-0.044444in}}%
\pgfusepath{stroke,fill}%
}%
\begin{pgfscope}%
\pgfsys@transformshift{1.554219in}{1.164970in}%
\pgfsys@useobject{currentmarker}{}%
\end{pgfscope}%
\end{pgfscope}%
\begin{pgfscope}%
\pgfsetbuttcap%
\pgfsetroundjoin%
\definecolor{currentfill}{rgb}{0.150000,0.150000,0.150000}%
\pgfsetfillcolor{currentfill}%
\pgfsetlinewidth{0.803000pt}%
\definecolor{currentstroke}{rgb}{0.150000,0.150000,0.150000}%
\pgfsetstrokecolor{currentstroke}%
\pgfsetdash{}{0pt}%
\pgfsys@defobject{currentmarker}{\pgfqpoint{0.000000in}{-0.044444in}}{\pgfqpoint{0.000000in}{0.000000in}}{%
\pgfpathmoveto{\pgfqpoint{0.000000in}{0.000000in}}%
\pgfpathlineto{\pgfqpoint{0.000000in}{-0.044444in}}%
\pgfusepath{stroke,fill}%
}%
\begin{pgfscope}%
\pgfsys@transformshift{1.593921in}{1.164970in}%
\pgfsys@useobject{currentmarker}{}%
\end{pgfscope}%
\end{pgfscope}%
\begin{pgfscope}%
\pgfsetbuttcap%
\pgfsetroundjoin%
\definecolor{currentfill}{rgb}{0.150000,0.150000,0.150000}%
\pgfsetfillcolor{currentfill}%
\pgfsetlinewidth{0.803000pt}%
\definecolor{currentstroke}{rgb}{0.150000,0.150000,0.150000}%
\pgfsetstrokecolor{currentstroke}%
\pgfsetdash{}{0pt}%
\pgfsys@defobject{currentmarker}{\pgfqpoint{0.000000in}{-0.044444in}}{\pgfqpoint{0.000000in}{0.000000in}}{%
\pgfpathmoveto{\pgfqpoint{0.000000in}{0.000000in}}%
\pgfpathlineto{\pgfqpoint{0.000000in}{-0.044444in}}%
\pgfusepath{stroke,fill}%
}%
\begin{pgfscope}%
\pgfsys@transformshift{1.627489in}{1.164970in}%
\pgfsys@useobject{currentmarker}{}%
\end{pgfscope}%
\end{pgfscope}%
\begin{pgfscope}%
\pgfsetbuttcap%
\pgfsetroundjoin%
\definecolor{currentfill}{rgb}{0.150000,0.150000,0.150000}%
\pgfsetfillcolor{currentfill}%
\pgfsetlinewidth{0.803000pt}%
\definecolor{currentstroke}{rgb}{0.150000,0.150000,0.150000}%
\pgfsetstrokecolor{currentstroke}%
\pgfsetdash{}{0pt}%
\pgfsys@defobject{currentmarker}{\pgfqpoint{0.000000in}{-0.044444in}}{\pgfqpoint{0.000000in}{0.000000in}}{%
\pgfpathmoveto{\pgfqpoint{0.000000in}{0.000000in}}%
\pgfpathlineto{\pgfqpoint{0.000000in}{-0.044444in}}%
\pgfusepath{stroke,fill}%
}%
\begin{pgfscope}%
\pgfsys@transformshift{1.656567in}{1.164970in}%
\pgfsys@useobject{currentmarker}{}%
\end{pgfscope}%
\end{pgfscope}%
\begin{pgfscope}%
\pgfsetbuttcap%
\pgfsetroundjoin%
\definecolor{currentfill}{rgb}{0.150000,0.150000,0.150000}%
\pgfsetfillcolor{currentfill}%
\pgfsetlinewidth{0.803000pt}%
\definecolor{currentstroke}{rgb}{0.150000,0.150000,0.150000}%
\pgfsetstrokecolor{currentstroke}%
\pgfsetdash{}{0pt}%
\pgfsys@defobject{currentmarker}{\pgfqpoint{0.000000in}{-0.044444in}}{\pgfqpoint{0.000000in}{0.000000in}}{%
\pgfpathmoveto{\pgfqpoint{0.000000in}{0.000000in}}%
\pgfpathlineto{\pgfqpoint{0.000000in}{-0.044444in}}%
\pgfusepath{stroke,fill}%
}%
\begin{pgfscope}%
\pgfsys@transformshift{1.682215in}{1.164970in}%
\pgfsys@useobject{currentmarker}{}%
\end{pgfscope}%
\end{pgfscope}%
\begin{pgfscope}%
\pgfsetbuttcap%
\pgfsetroundjoin%
\definecolor{currentfill}{rgb}{0.150000,0.150000,0.150000}%
\pgfsetfillcolor{currentfill}%
\pgfsetlinewidth{0.803000pt}%
\definecolor{currentstroke}{rgb}{0.150000,0.150000,0.150000}%
\pgfsetstrokecolor{currentstroke}%
\pgfsetdash{}{0pt}%
\pgfsys@defobject{currentmarker}{\pgfqpoint{0.000000in}{-0.044444in}}{\pgfqpoint{0.000000in}{0.000000in}}{%
\pgfpathmoveto{\pgfqpoint{0.000000in}{0.000000in}}%
\pgfpathlineto{\pgfqpoint{0.000000in}{-0.044444in}}%
\pgfusepath{stroke,fill}%
}%
\begin{pgfscope}%
\pgfsys@transformshift{1.856098in}{1.164970in}%
\pgfsys@useobject{currentmarker}{}%
\end{pgfscope}%
\end{pgfscope}%
\begin{pgfscope}%
\pgfsetbuttcap%
\pgfsetroundjoin%
\definecolor{currentfill}{rgb}{0.150000,0.150000,0.150000}%
\pgfsetfillcolor{currentfill}%
\pgfsetlinewidth{1.003750pt}%
\definecolor{currentstroke}{rgb}{0.150000,0.150000,0.150000}%
\pgfsetstrokecolor{currentstroke}%
\pgfsetdash{}{0pt}%
\pgfsys@defobject{currentmarker}{\pgfqpoint{-0.066667in}{0.000000in}}{\pgfqpoint{0.000000in}{0.000000in}}{%
\pgfpathmoveto{\pgfqpoint{0.000000in}{0.000000in}}%
\pgfpathlineto{\pgfqpoint{-0.066667in}{0.000000in}}%
\pgfusepath{stroke,fill}%
}%
\begin{pgfscope}%
\pgfsys@transformshift{0.702340in}{1.164970in}%
\pgfsys@useobject{currentmarker}{}%
\end{pgfscope}%
\end{pgfscope}%
\begin{pgfscope}%
\definecolor{textcolor}{rgb}{0.150000,0.150000,0.150000}%
\pgfsetstrokecolor{textcolor}%
\pgfsetfillcolor{textcolor}%
\pgftext[x=0.413148in,y=1.140023in,left,base]{\color{textcolor}\sffamily\fontsize{5.176471}{6.211765}\selectfont 3.000}%
\end{pgfscope}%
\begin{pgfscope}%
\pgfsetbuttcap%
\pgfsetroundjoin%
\definecolor{currentfill}{rgb}{0.150000,0.150000,0.150000}%
\pgfsetfillcolor{currentfill}%
\pgfsetlinewidth{1.003750pt}%
\definecolor{currentstroke}{rgb}{0.150000,0.150000,0.150000}%
\pgfsetstrokecolor{currentstroke}%
\pgfsetdash{}{0pt}%
\pgfsys@defobject{currentmarker}{\pgfqpoint{-0.066667in}{0.000000in}}{\pgfqpoint{0.000000in}{0.000000in}}{%
\pgfpathmoveto{\pgfqpoint{0.000000in}{0.000000in}}%
\pgfpathlineto{\pgfqpoint{-0.066667in}{0.000000in}}%
\pgfusepath{stroke,fill}%
}%
\begin{pgfscope}%
\pgfsys@transformshift{0.702340in}{1.491207in}%
\pgfsys@useobject{currentmarker}{}%
\end{pgfscope}%
\end{pgfscope}%
\begin{pgfscope}%
\definecolor{textcolor}{rgb}{0.150000,0.150000,0.150000}%
\pgfsetstrokecolor{textcolor}%
\pgfsetfillcolor{textcolor}%
\pgftext[x=0.413148in,y=1.466260in,left,base]{\color{textcolor}\sffamily\fontsize{5.176471}{6.211765}\selectfont 3.537}%
\end{pgfscope}%
\begin{pgfscope}%
\pgfsetbuttcap%
\pgfsetroundjoin%
\definecolor{currentfill}{rgb}{0.150000,0.150000,0.150000}%
\pgfsetfillcolor{currentfill}%
\pgfsetlinewidth{1.003750pt}%
\definecolor{currentstroke}{rgb}{0.150000,0.150000,0.150000}%
\pgfsetstrokecolor{currentstroke}%
\pgfsetdash{}{0pt}%
\pgfsys@defobject{currentmarker}{\pgfqpoint{-0.066667in}{0.000000in}}{\pgfqpoint{0.000000in}{0.000000in}}{%
\pgfpathmoveto{\pgfqpoint{0.000000in}{0.000000in}}%
\pgfpathlineto{\pgfqpoint{-0.066667in}{0.000000in}}%
\pgfusepath{stroke,fill}%
}%
\begin{pgfscope}%
\pgfsys@transformshift{0.702340in}{1.772918in}%
\pgfsys@useobject{currentmarker}{}%
\end{pgfscope}%
\end{pgfscope}%
\begin{pgfscope}%
\definecolor{textcolor}{rgb}{0.150000,0.150000,0.150000}%
\pgfsetstrokecolor{textcolor}%
\pgfsetfillcolor{textcolor}%
\pgftext[x=0.413148in,y=1.747971in,left,base]{\color{textcolor}\sffamily\fontsize{5.176471}{6.211765}\selectfont 4.000}%
\end{pgfscope}%
\begin{pgfscope}%
\definecolor{textcolor}{rgb}{0.150000,0.150000,0.150000}%
\pgfsetstrokecolor{textcolor}%
\pgfsetfillcolor{textcolor}%
\pgftext[x=0.357592in,y=1.468944in,,bottom,rotate=90.000000]{\color{textcolor}\sffamily\fontsize{5.647059}{6.776471}\selectfont \(\displaystyle x = \frac{2 \mu E L^2}{4 \pi^2}\)}%
\end{pgfscope}%
\begin{pgfscope}%
\pgfpathrectangle{\pgfqpoint{0.702340in}{1.164970in}}{\pgfqpoint{1.223103in}{0.607948in}}%
\pgfusepath{clip}%
\pgfsetroundcap%
\pgfsetroundjoin%
\pgfsetlinewidth{1.204500pt}%
\definecolor{currentstroke}{rgb}{0.000000,0.501961,0.000000}%
\pgfsetstrokecolor{currentstroke}%
\pgfsetdash{}{0pt}%
\pgfpathmoveto{\pgfqpoint{0.702340in}{1.491340in}}%
\pgfpathlineto{\pgfqpoint{0.943050in}{1.490088in}}%
\pgfpathlineto{\pgfqpoint{1.054581in}{1.488705in}}%
\pgfpathlineto{\pgfqpoint{1.127977in}{1.487191in}}%
\pgfpathlineto{\pgfqpoint{1.182772in}{1.485547in}}%
\pgfpathlineto{\pgfqpoint{1.226515in}{1.483775in}}%
\pgfpathlineto{\pgfqpoint{1.262923in}{1.481875in}}%
\pgfpathlineto{\pgfqpoint{1.294107in}{1.479849in}}%
\pgfpathlineto{\pgfqpoint{1.321380in}{1.477697in}}%
\pgfpathlineto{\pgfqpoint{1.345614in}{1.475421in}}%
\pgfpathlineto{\pgfqpoint{1.367419in}{1.473022in}}%
\pgfpathlineto{\pgfqpoint{1.387238in}{1.470501in}}%
\pgfpathlineto{\pgfqpoint{1.405403in}{1.467858in}}%
\pgfpathlineto{\pgfqpoint{1.422168in}{1.465095in}}%
\pgfpathlineto{\pgfqpoint{1.437735in}{1.462213in}}%
\pgfpathlineto{\pgfqpoint{1.452262in}{1.459212in}}%
\pgfpathlineto{\pgfqpoint{1.465881in}{1.456093in}}%
\pgfpathlineto{\pgfqpoint{1.478698in}{1.452858in}}%
\pgfpathlineto{\pgfqpoint{1.490802in}{1.449507in}}%
\pgfpathlineto{\pgfqpoint{1.502269in}{1.446040in}}%
\pgfpathlineto{\pgfqpoint{1.513162in}{1.442460in}}%
\pgfpathlineto{\pgfqpoint{1.523536in}{1.438765in}}%
\pgfpathlineto{\pgfqpoint{1.533438in}{1.434959in}}%
\pgfpathlineto{\pgfqpoint{1.542909in}{1.431040in}}%
\pgfpathlineto{\pgfqpoint{1.551986in}{1.427009in}}%
\pgfpathlineto{\pgfqpoint{1.560699in}{1.422869in}}%
\pgfpathlineto{\pgfqpoint{1.569077in}{1.418618in}}%
\pgfpathlineto{\pgfqpoint{1.577145in}{1.414258in}}%
\pgfpathlineto{\pgfqpoint{1.584924in}{1.409789in}}%
\pgfpathlineto{\pgfqpoint{1.592435in}{1.405212in}}%
\pgfpathlineto{\pgfqpoint{1.599696in}{1.400528in}}%
\pgfpathlineto{\pgfqpoint{1.606722in}{1.395736in}}%
\pgfpathlineto{\pgfqpoint{1.613528in}{1.390838in}}%
\pgfpathlineto{\pgfqpoint{1.620129in}{1.385834in}}%
\pgfpathlineto{\pgfqpoint{1.626535in}{1.380724in}}%
\pgfpathlineto{\pgfqpoint{1.632758in}{1.375509in}}%
\pgfpathlineto{\pgfqpoint{1.638808in}{1.370190in}}%
\pgfpathlineto{\pgfqpoint{1.644694in}{1.364765in}}%
\pgfpathlineto{\pgfqpoint{1.650426in}{1.359237in}}%
\pgfpathlineto{\pgfqpoint{1.656011in}{1.353604in}}%
\pgfpathlineto{\pgfqpoint{1.661456in}{1.347868in}}%
\pgfpathlineto{\pgfqpoint{1.666768in}{1.342029in}}%
\pgfpathlineto{\pgfqpoint{1.671953in}{1.336086in}}%
\pgfpathlineto{\pgfqpoint{1.677018in}{1.330040in}}%
\pgfpathlineto{\pgfqpoint{1.681968in}{1.323892in}}%
\pgfpathlineto{\pgfqpoint{1.686808in}{1.317640in}}%
\pgfpathlineto{\pgfqpoint{1.691543in}{1.311286in}}%
\pgfpathlineto{\pgfqpoint{1.696176in}{1.304829in}}%
\pgfpathlineto{\pgfqpoint{1.700714in}{1.298269in}}%
\pgfpathlineto{\pgfqpoint{1.705158in}{1.291606in}}%
\pgfusepath{stroke}%
\end{pgfscope}%
\begin{pgfscope}%
\pgfsetrectcap%
\pgfsetmiterjoin%
\pgfsetlinewidth{1.003750pt}%
\definecolor{currentstroke}{rgb}{0.150000,0.150000,0.150000}%
\pgfsetstrokecolor{currentstroke}%
\pgfsetdash{}{0pt}%
\pgfpathmoveto{\pgfqpoint{0.702340in}{1.164970in}}%
\pgfpathlineto{\pgfqpoint{0.702340in}{1.772918in}}%
\pgfusepath{stroke}%
\end{pgfscope}%
\begin{pgfscope}%
\pgfsetrectcap%
\pgfsetmiterjoin%
\pgfsetlinewidth{1.003750pt}%
\definecolor{currentstroke}{rgb}{0.150000,0.150000,0.150000}%
\pgfsetstrokecolor{currentstroke}%
\pgfsetdash{}{0pt}%
\pgfpathmoveto{\pgfqpoint{0.702340in}{1.164970in}}%
\pgfpathlineto{\pgfqpoint{1.925444in}{1.164970in}}%
\pgfusepath{stroke}%
\end{pgfscope}%
\begin{pgfscope}%
\pgfpathrectangle{\pgfqpoint{0.702340in}{1.164970in}}{\pgfqpoint{1.223103in}{0.607948in}}%
\pgfusepath{clip}%
\pgfsetbuttcap%
\pgfsetroundjoin%
\definecolor{currentfill}{rgb}{0.000000,0.000000,0.000000}%
\pgfsetfillcolor{currentfill}%
\pgfsetlinewidth{1.003750pt}%
\definecolor{currentstroke}{rgb}{0.000000,0.000000,0.000000}%
\pgfsetstrokecolor{currentstroke}%
\pgfsetdash{}{0pt}%
\pgfsys@defobject{currentmarker}{\pgfqpoint{-0.013889in}{-0.013889in}}{\pgfqpoint{0.013889in}{0.013889in}}{%
\pgfpathmoveto{\pgfqpoint{0.000000in}{-0.013889in}}%
\pgfpathcurveto{\pgfqpoint{0.003683in}{-0.013889in}}{\pgfqpoint{0.007216in}{-0.012425in}}{\pgfqpoint{0.009821in}{-0.009821in}}%
\pgfpathcurveto{\pgfqpoint{0.012425in}{-0.007216in}}{\pgfqpoint{0.013889in}{-0.003683in}}{\pgfqpoint{0.013889in}{0.000000in}}%
\pgfpathcurveto{\pgfqpoint{0.013889in}{0.003683in}}{\pgfqpoint{0.012425in}{0.007216in}}{\pgfqpoint{0.009821in}{0.009821in}}%
\pgfpathcurveto{\pgfqpoint{0.007216in}{0.012425in}}{\pgfqpoint{0.003683in}{0.013889in}}{\pgfqpoint{0.000000in}{0.013889in}}%
\pgfpathcurveto{\pgfqpoint{-0.003683in}{0.013889in}}{\pgfqpoint{-0.007216in}{0.012425in}}{\pgfqpoint{-0.009821in}{0.009821in}}%
\pgfpathcurveto{\pgfqpoint{-0.012425in}{0.007216in}}{\pgfqpoint{-0.013889in}{0.003683in}}{\pgfqpoint{-0.013889in}{0.000000in}}%
\pgfpathcurveto{\pgfqpoint{-0.013889in}{-0.003683in}}{\pgfqpoint{-0.012425in}{-0.007216in}}{\pgfqpoint{-0.009821in}{-0.009821in}}%
\pgfpathcurveto{\pgfqpoint{-0.007216in}{-0.012425in}}{\pgfqpoint{-0.003683in}{-0.013889in}}{\pgfqpoint{0.000000in}{-0.013889in}}%
\pgfpathclose%
\pgfusepath{stroke,fill}%
}%
\begin{pgfscope}%
\pgfsys@transformshift{1.705158in}{1.291625in}%
\pgfsys@useobject{currentmarker}{}%
\end{pgfscope}%
\begin{pgfscope}%
\pgfsys@transformshift{1.616865in}{1.388260in}%
\pgfsys@useobject{currentmarker}{}%
\end{pgfscope}%
\begin{pgfscope}%
\pgfsys@transformshift{1.554219in}{1.425957in}%
\pgfsys@useobject{currentmarker}{}%
\end{pgfscope}%
\begin{pgfscope}%
\pgfsys@transformshift{1.505628in}{1.445001in}%
\pgfsys@useobject{currentmarker}{}%
\end{pgfscope}%
\begin{pgfscope}%
\pgfsys@transformshift{1.465926in}{1.456132in}%
\pgfsys@useobject{currentmarker}{}%
\end{pgfscope}%
\begin{pgfscope}%
\pgfsys@transformshift{1.432358in}{1.463288in}%
\pgfsys@useobject{currentmarker}{}%
\end{pgfscope}%
\begin{pgfscope}%
\pgfsys@transformshift{1.403280in}{1.468207in}%
\pgfsys@useobject{currentmarker}{}%
\end{pgfscope}%
\begin{pgfscope}%
\pgfsys@transformshift{1.397903in}{1.469008in}%
\pgfsys@useobject{currentmarker}{}%
\end{pgfscope}%
\begin{pgfscope}%
\pgfsys@transformshift{1.392656in}{1.469761in}%
\pgfsys@useobject{currentmarker}{}%
\end{pgfscope}%
\begin{pgfscope}%
\pgfsys@transformshift{1.387532in}{1.470468in}%
\pgfsys@useobject{currentmarker}{}%
\end{pgfscope}%
\begin{pgfscope}%
\pgfsys@transformshift{1.382525in}{1.471135in}%
\pgfsys@useobject{currentmarker}{}%
\end{pgfscope}%
\begin{pgfscope}%
\pgfsys@transformshift{1.377632in}{1.471763in}%
\pgfsys@useobject{currentmarker}{}%
\end{pgfscope}%
\begin{pgfscope}%
\pgfsys@transformshift{1.372846in}{1.472357in}%
\pgfsys@useobject{currentmarker}{}%
\end{pgfscope}%
\begin{pgfscope}%
\pgfsys@transformshift{1.368163in}{1.472919in}%
\pgfsys@useobject{currentmarker}{}%
\end{pgfscope}%
\begin{pgfscope}%
\pgfsys@transformshift{1.363578in}{1.473451in}%
\pgfsys@useobject{currentmarker}{}%
\end{pgfscope}%
\begin{pgfscope}%
\pgfsys@transformshift{1.359088in}{1.473956in}%
\pgfsys@useobject{currentmarker}{}%
\end{pgfscope}%
\begin{pgfscope}%
\pgfsys@transformshift{1.354689in}{1.474435in}%
\pgfsys@useobject{currentmarker}{}%
\end{pgfscope}%
\end{pgfscope}%
\begin{pgfscope}%
\pgfsetbuttcap%
\pgfsetmiterjoin%
\definecolor{currentfill}{rgb}{1.000000,1.000000,1.000000}%
\pgfsetfillcolor{currentfill}%
\pgfsetlinewidth{0.000000pt}%
\definecolor{currentstroke}{rgb}{0.000000,0.000000,0.000000}%
\pgfsetstrokecolor{currentstroke}%
\pgfsetstrokeopacity{0.000000}%
\pgfsetdash{}{0pt}%
\pgfpathmoveto{\pgfqpoint{2.170064in}{1.164970in}}%
\pgfpathlineto{\pgfqpoint{3.393168in}{1.164970in}}%
\pgfpathlineto{\pgfqpoint{3.393168in}{1.772918in}}%
\pgfpathlineto{\pgfqpoint{2.170064in}{1.772918in}}%
\pgfpathclose%
\pgfusepath{fill}%
\end{pgfscope}%
\begin{pgfscope}%
\pgfpathrectangle{\pgfqpoint{2.170064in}{1.164970in}}{\pgfqpoint{1.223103in}{0.607948in}}%
\pgfusepath{clip}%
\pgfsetbuttcap%
\pgfsetmiterjoin%
\definecolor{currentfill}{rgb}{0.000000,0.000000,1.000000}%
\pgfsetfillcolor{currentfill}%
\pgfsetfillopacity{0.100000}%
\pgfsetlinewidth{0.803000pt}%
\definecolor{currentstroke}{rgb}{0.000000,0.000000,1.000000}%
\pgfsetstrokecolor{currentstroke}%
\pgfsetstrokeopacity{0.100000}%
\pgfsetdash{}{0pt}%
\pgfpathmoveto{\pgfqpoint{2.170064in}{1.486788in}}%
\pgfpathlineto{\pgfqpoint{2.170064in}{1.492891in}}%
\pgfpathlineto{\pgfqpoint{3.393168in}{1.492891in}}%
\pgfpathlineto{\pgfqpoint{3.393168in}{1.486788in}}%
\pgfpathclose%
\pgfusepath{stroke,fill}%
\end{pgfscope}%
\begin{pgfscope}%
\pgfpathrectangle{\pgfqpoint{2.170064in}{1.164970in}}{\pgfqpoint{1.223103in}{0.607948in}}%
\pgfusepath{clip}%
\pgfsetbuttcap%
\pgfsetroundjoin%
\definecolor{currentfill}{rgb}{0.000000,0.501961,0.000000}%
\pgfsetfillcolor{currentfill}%
\pgfsetfillopacity{0.500000}%
\pgfsetlinewidth{0.803000pt}%
\definecolor{currentstroke}{rgb}{0.000000,0.501961,0.000000}%
\pgfsetstrokecolor{currentstroke}%
\pgfsetstrokeopacity{0.500000}%
\pgfsetdash{}{0pt}%
\pgfpathmoveto{\pgfqpoint{2.170064in}{1.493032in}}%
\pgfpathlineto{\pgfqpoint{2.170064in}{1.487358in}}%
\pgfpathlineto{\pgfqpoint{2.410774in}{1.488480in}}%
\pgfpathlineto{\pgfqpoint{2.522305in}{1.489560in}}%
\pgfpathlineto{\pgfqpoint{2.595701in}{1.490597in}}%
\pgfpathlineto{\pgfqpoint{2.650497in}{1.491592in}}%
\pgfpathlineto{\pgfqpoint{2.694239in}{1.492544in}}%
\pgfpathlineto{\pgfqpoint{2.730647in}{1.493452in}}%
\pgfpathlineto{\pgfqpoint{2.761831in}{1.494316in}}%
\pgfpathlineto{\pgfqpoint{2.789104in}{1.495135in}}%
\pgfpathlineto{\pgfqpoint{2.813338in}{1.495908in}}%
\pgfpathlineto{\pgfqpoint{2.835143in}{1.496635in}}%
\pgfpathlineto{\pgfqpoint{2.854962in}{1.497278in}}%
\pgfpathlineto{\pgfqpoint{2.873127in}{1.497703in}}%
\pgfpathlineto{\pgfqpoint{2.889892in}{1.498122in}}%
\pgfpathlineto{\pgfqpoint{2.905459in}{1.498533in}}%
\pgfpathlineto{\pgfqpoint{2.919986in}{1.498930in}}%
\pgfpathlineto{\pgfqpoint{2.933605in}{1.499313in}}%
\pgfpathlineto{\pgfqpoint{2.946422in}{1.499676in}}%
\pgfpathlineto{\pgfqpoint{2.958526in}{1.500018in}}%
\pgfpathlineto{\pgfqpoint{2.969993in}{1.500333in}}%
\pgfpathlineto{\pgfqpoint{2.980886in}{1.500620in}}%
\pgfpathlineto{\pgfqpoint{2.991260in}{1.500874in}}%
\pgfpathlineto{\pgfqpoint{3.001162in}{1.501092in}}%
\pgfpathlineto{\pgfqpoint{3.010633in}{1.501269in}}%
\pgfpathlineto{\pgfqpoint{3.019710in}{1.501403in}}%
\pgfpathlineto{\pgfqpoint{3.028423in}{1.501430in}}%
\pgfpathlineto{\pgfqpoint{3.036801in}{1.501304in}}%
\pgfpathlineto{\pgfqpoint{3.044869in}{1.501115in}}%
\pgfpathlineto{\pgfqpoint{3.052648in}{1.500862in}}%
\pgfpathlineto{\pgfqpoint{3.060159in}{1.500546in}}%
\pgfpathlineto{\pgfqpoint{3.067420in}{1.500165in}}%
\pgfpathlineto{\pgfqpoint{3.074446in}{1.499718in}}%
\pgfpathlineto{\pgfqpoint{3.081253in}{1.499204in}}%
\pgfpathlineto{\pgfqpoint{3.087853in}{1.498622in}}%
\pgfpathlineto{\pgfqpoint{3.094259in}{1.497970in}}%
\pgfpathlineto{\pgfqpoint{3.100482in}{1.497249in}}%
\pgfpathlineto{\pgfqpoint{3.106532in}{1.496456in}}%
\pgfpathlineto{\pgfqpoint{3.112419in}{1.495590in}}%
\pgfpathlineto{\pgfqpoint{3.118150in}{1.494650in}}%
\pgfpathlineto{\pgfqpoint{3.123735in}{1.493636in}}%
\pgfpathlineto{\pgfqpoint{3.129180in}{1.492546in}}%
\pgfpathlineto{\pgfqpoint{3.134492in}{1.491378in}}%
\pgfpathlineto{\pgfqpoint{3.139677in}{1.490132in}}%
\pgfpathlineto{\pgfqpoint{3.144742in}{1.488807in}}%
\pgfpathlineto{\pgfqpoint{3.149692in}{1.487401in}}%
\pgfpathlineto{\pgfqpoint{3.154532in}{1.485913in}}%
\pgfpathlineto{\pgfqpoint{3.159267in}{1.484342in}}%
\pgfpathlineto{\pgfqpoint{3.163901in}{1.482686in}}%
\pgfpathlineto{\pgfqpoint{3.168438in}{1.480945in}}%
\pgfpathlineto{\pgfqpoint{3.172883in}{1.478849in}}%
\pgfpathlineto{\pgfqpoint{3.172883in}{1.479119in}}%
\pgfpathlineto{\pgfqpoint{3.172883in}{1.479119in}}%
\pgfpathlineto{\pgfqpoint{3.168438in}{1.481134in}}%
\pgfpathlineto{\pgfqpoint{3.163901in}{1.483268in}}%
\pgfpathlineto{\pgfqpoint{3.159267in}{1.485258in}}%
\pgfpathlineto{\pgfqpoint{3.154532in}{1.487109in}}%
\pgfpathlineto{\pgfqpoint{3.149692in}{1.488824in}}%
\pgfpathlineto{\pgfqpoint{3.144742in}{1.490408in}}%
\pgfpathlineto{\pgfqpoint{3.139677in}{1.491865in}}%
\pgfpathlineto{\pgfqpoint{3.134492in}{1.493201in}}%
\pgfpathlineto{\pgfqpoint{3.129180in}{1.494419in}}%
\pgfpathlineto{\pgfqpoint{3.123735in}{1.495523in}}%
\pgfpathlineto{\pgfqpoint{3.118150in}{1.496519in}}%
\pgfpathlineto{\pgfqpoint{3.112419in}{1.497410in}}%
\pgfpathlineto{\pgfqpoint{3.106532in}{1.498201in}}%
\pgfpathlineto{\pgfqpoint{3.100482in}{1.498896in}}%
\pgfpathlineto{\pgfqpoint{3.094259in}{1.499500in}}%
\pgfpathlineto{\pgfqpoint{3.087853in}{1.500016in}}%
\pgfpathlineto{\pgfqpoint{3.081253in}{1.500449in}}%
\pgfpathlineto{\pgfqpoint{3.074446in}{1.500802in}}%
\pgfpathlineto{\pgfqpoint{3.067420in}{1.501081in}}%
\pgfpathlineto{\pgfqpoint{3.060159in}{1.501289in}}%
\pgfpathlineto{\pgfqpoint{3.052648in}{1.501430in}}%
\pgfpathlineto{\pgfqpoint{3.044869in}{1.501508in}}%
\pgfpathlineto{\pgfqpoint{3.036801in}{1.501528in}}%
\pgfpathlineto{\pgfqpoint{3.028423in}{1.501494in}}%
\pgfpathlineto{\pgfqpoint{3.019710in}{1.501503in}}%
\pgfpathlineto{\pgfqpoint{3.010633in}{1.501510in}}%
\pgfpathlineto{\pgfqpoint{3.001162in}{1.501459in}}%
\pgfpathlineto{\pgfqpoint{2.991260in}{1.501351in}}%
\pgfpathlineto{\pgfqpoint{2.980886in}{1.501187in}}%
\pgfpathlineto{\pgfqpoint{2.969993in}{1.500967in}}%
\pgfpathlineto{\pgfqpoint{2.958526in}{1.500692in}}%
\pgfpathlineto{\pgfqpoint{2.946422in}{1.500363in}}%
\pgfpathlineto{\pgfqpoint{2.933605in}{1.499982in}}%
\pgfpathlineto{\pgfqpoint{2.919986in}{1.499549in}}%
\pgfpathlineto{\pgfqpoint{2.905459in}{1.499065in}}%
\pgfpathlineto{\pgfqpoint{2.889892in}{1.498531in}}%
\pgfpathlineto{\pgfqpoint{2.873127in}{1.497947in}}%
\pgfpathlineto{\pgfqpoint{2.854962in}{1.497315in}}%
\pgfpathlineto{\pgfqpoint{2.835143in}{1.496850in}}%
\pgfpathlineto{\pgfqpoint{2.813338in}{1.496422in}}%
\pgfpathlineto{\pgfqpoint{2.789104in}{1.495997in}}%
\pgfpathlineto{\pgfqpoint{2.761831in}{1.495577in}}%
\pgfpathlineto{\pgfqpoint{2.730647in}{1.495167in}}%
\pgfpathlineto{\pgfqpoint{2.694239in}{1.494768in}}%
\pgfpathlineto{\pgfqpoint{2.650497in}{1.494382in}}%
\pgfpathlineto{\pgfqpoint{2.595701in}{1.494014in}}%
\pgfpathlineto{\pgfqpoint{2.522305in}{1.493664in}}%
\pgfpathlineto{\pgfqpoint{2.410774in}{1.493336in}}%
\pgfpathlineto{\pgfqpoint{2.170064in}{1.493032in}}%
\pgfpathclose%
\pgfusepath{stroke,fill}%
\end{pgfscope}%
\begin{pgfscope}%
\pgfpathrectangle{\pgfqpoint{2.170064in}{1.164970in}}{\pgfqpoint{1.223103in}{0.607948in}}%
\pgfusepath{clip}%
\pgfsetroundcap%
\pgfsetroundjoin%
\pgfsetlinewidth{0.501875pt}%
\definecolor{currentstroke}{rgb}{0.000000,0.000000,1.000000}%
\pgfsetstrokecolor{currentstroke}%
\pgfsetstrokeopacity{0.800000}%
\pgfsetdash{}{0pt}%
\pgfpathmoveto{\pgfqpoint{2.170064in}{1.489839in}}%
\pgfpathlineto{\pgfqpoint{3.393168in}{1.489839in}}%
\pgfusepath{stroke}%
\end{pgfscope}%
\begin{pgfscope}%
\pgfpathrectangle{\pgfqpoint{2.170064in}{1.164970in}}{\pgfqpoint{1.223103in}{0.607948in}}%
\pgfusepath{clip}%
\pgfsetbuttcap%
\pgfsetroundjoin%
\pgfsetlinewidth{1.003750pt}%
\definecolor{currentstroke}{rgb}{0.000000,0.000000,0.000000}%
\pgfsetstrokecolor{currentstroke}%
\pgfsetdash{{3.700000pt}{1.600000pt}}{0.000000pt}%
\pgfpathmoveto{\pgfqpoint{2.170064in}{1.491207in}}%
\pgfpathlineto{\pgfqpoint{3.393168in}{1.491207in}}%
\pgfusepath{stroke}%
\end{pgfscope}%
\begin{pgfscope}%
\pgfsetroundcap%
\pgfsetroundjoin%
\pgfsetlinewidth{0.501875pt}%
\definecolor{currentstroke}{rgb}{0.000000,0.000000,1.000000}%
\pgfsetstrokecolor{currentstroke}%
\pgfsetstrokeopacity{0.800000}%
\pgfsetdash{}{0pt}%
\pgfpathmoveto{\pgfqpoint{2.983913in}{1.607176in}}%
\pgfpathquadraticcurveto{\pgfqpoint{2.914395in}{1.556668in}}{\pgfqpoint{2.844877in}{1.506161in}}%
\pgfusepath{stroke}%
\end{pgfscope}%
\begin{pgfscope}%
\pgfsetfillopacity{0.800000}%
\pgfsetstrokeopacity{0.800000}%
\definecolor{textcolor}{rgb}{0.000000,0.000000,1.000000}%
\pgfsetstrokecolor{textcolor}%
\pgfsetfillcolor{textcolor}%
\pgftext[x=2.910706in,y=1.672224in,left,base]{\color{textcolor}\sffamily\fontsize{5.647059}{6.776471}\selectfont 3.5344(50)}%
\end{pgfscope}%
\begin{pgfscope}%
\pgfsetbuttcap%
\pgfsetroundjoin%
\definecolor{currentfill}{rgb}{0.150000,0.150000,0.150000}%
\pgfsetfillcolor{currentfill}%
\pgfsetlinewidth{1.003750pt}%
\definecolor{currentstroke}{rgb}{0.150000,0.150000,0.150000}%
\pgfsetstrokecolor{currentstroke}%
\pgfsetdash{}{0pt}%
\pgfsys@defobject{currentmarker}{\pgfqpoint{0.000000in}{-0.066667in}}{\pgfqpoint{0.000000in}{0.000000in}}{%
\pgfpathmoveto{\pgfqpoint{0.000000in}{0.000000in}}%
\pgfpathlineto{\pgfqpoint{0.000000in}{-0.066667in}}%
\pgfusepath{stroke,fill}%
}%
\begin{pgfscope}%
\pgfsys@transformshift{2.170064in}{1.164970in}%
\pgfsys@useobject{currentmarker}{}%
\end{pgfscope}%
\end{pgfscope}%
\begin{pgfscope}%
\pgfsetbuttcap%
\pgfsetroundjoin%
\definecolor{currentfill}{rgb}{0.150000,0.150000,0.150000}%
\pgfsetfillcolor{currentfill}%
\pgfsetlinewidth{1.003750pt}%
\definecolor{currentstroke}{rgb}{0.150000,0.150000,0.150000}%
\pgfsetstrokecolor{currentstroke}%
\pgfsetdash{}{0pt}%
\pgfsys@defobject{currentmarker}{\pgfqpoint{0.000000in}{-0.066667in}}{\pgfqpoint{0.000000in}{0.000000in}}{%
\pgfpathmoveto{\pgfqpoint{0.000000in}{0.000000in}}%
\pgfpathlineto{\pgfqpoint{0.000000in}{-0.066667in}}%
\pgfusepath{stroke,fill}%
}%
\begin{pgfscope}%
\pgfsys@transformshift{2.671473in}{1.164970in}%
\pgfsys@useobject{currentmarker}{}%
\end{pgfscope}%
\end{pgfscope}%
\begin{pgfscope}%
\pgfsetbuttcap%
\pgfsetroundjoin%
\definecolor{currentfill}{rgb}{0.150000,0.150000,0.150000}%
\pgfsetfillcolor{currentfill}%
\pgfsetlinewidth{1.003750pt}%
\definecolor{currentstroke}{rgb}{0.150000,0.150000,0.150000}%
\pgfsetstrokecolor{currentstroke}%
\pgfsetdash{}{0pt}%
\pgfsys@defobject{currentmarker}{\pgfqpoint{0.000000in}{-0.066667in}}{\pgfqpoint{0.000000in}{0.000000in}}{%
\pgfpathmoveto{\pgfqpoint{0.000000in}{0.000000in}}%
\pgfpathlineto{\pgfqpoint{0.000000in}{-0.066667in}}%
\pgfusepath{stroke,fill}%
}%
\begin{pgfscope}%
\pgfsys@transformshift{3.172883in}{1.164970in}%
\pgfsys@useobject{currentmarker}{}%
\end{pgfscope}%
\end{pgfscope}%
\begin{pgfscope}%
\pgfsetbuttcap%
\pgfsetroundjoin%
\definecolor{currentfill}{rgb}{0.150000,0.150000,0.150000}%
\pgfsetfillcolor{currentfill}%
\pgfsetlinewidth{0.803000pt}%
\definecolor{currentstroke}{rgb}{0.150000,0.150000,0.150000}%
\pgfsetstrokecolor{currentstroke}%
\pgfsetdash{}{0pt}%
\pgfsys@defobject{currentmarker}{\pgfqpoint{0.000000in}{-0.044444in}}{\pgfqpoint{0.000000in}{0.000000in}}{%
\pgfpathmoveto{\pgfqpoint{0.000000in}{0.000000in}}%
\pgfpathlineto{\pgfqpoint{0.000000in}{-0.044444in}}%
\pgfusepath{stroke,fill}%
}%
\begin{pgfscope}%
\pgfsys@transformshift{2.321004in}{1.164970in}%
\pgfsys@useobject{currentmarker}{}%
\end{pgfscope}%
\end{pgfscope}%
\begin{pgfscope}%
\pgfsetbuttcap%
\pgfsetroundjoin%
\definecolor{currentfill}{rgb}{0.150000,0.150000,0.150000}%
\pgfsetfillcolor{currentfill}%
\pgfsetlinewidth{0.803000pt}%
\definecolor{currentstroke}{rgb}{0.150000,0.150000,0.150000}%
\pgfsetstrokecolor{currentstroke}%
\pgfsetdash{}{0pt}%
\pgfsys@defobject{currentmarker}{\pgfqpoint{0.000000in}{-0.044444in}}{\pgfqpoint{0.000000in}{0.000000in}}{%
\pgfpathmoveto{\pgfqpoint{0.000000in}{0.000000in}}%
\pgfpathlineto{\pgfqpoint{0.000000in}{-0.044444in}}%
\pgfusepath{stroke,fill}%
}%
\begin{pgfscope}%
\pgfsys@transformshift{2.409297in}{1.164970in}%
\pgfsys@useobject{currentmarker}{}%
\end{pgfscope}%
\end{pgfscope}%
\begin{pgfscope}%
\pgfsetbuttcap%
\pgfsetroundjoin%
\definecolor{currentfill}{rgb}{0.150000,0.150000,0.150000}%
\pgfsetfillcolor{currentfill}%
\pgfsetlinewidth{0.803000pt}%
\definecolor{currentstroke}{rgb}{0.150000,0.150000,0.150000}%
\pgfsetstrokecolor{currentstroke}%
\pgfsetdash{}{0pt}%
\pgfsys@defobject{currentmarker}{\pgfqpoint{0.000000in}{-0.044444in}}{\pgfqpoint{0.000000in}{0.000000in}}{%
\pgfpathmoveto{\pgfqpoint{0.000000in}{0.000000in}}%
\pgfpathlineto{\pgfqpoint{0.000000in}{-0.044444in}}%
\pgfusepath{stroke,fill}%
}%
\begin{pgfscope}%
\pgfsys@transformshift{2.471943in}{1.164970in}%
\pgfsys@useobject{currentmarker}{}%
\end{pgfscope}%
\end{pgfscope}%
\begin{pgfscope}%
\pgfsetbuttcap%
\pgfsetroundjoin%
\definecolor{currentfill}{rgb}{0.150000,0.150000,0.150000}%
\pgfsetfillcolor{currentfill}%
\pgfsetlinewidth{0.803000pt}%
\definecolor{currentstroke}{rgb}{0.150000,0.150000,0.150000}%
\pgfsetstrokecolor{currentstroke}%
\pgfsetdash{}{0pt}%
\pgfsys@defobject{currentmarker}{\pgfqpoint{0.000000in}{-0.044444in}}{\pgfqpoint{0.000000in}{0.000000in}}{%
\pgfpathmoveto{\pgfqpoint{0.000000in}{0.000000in}}%
\pgfpathlineto{\pgfqpoint{0.000000in}{-0.044444in}}%
\pgfusepath{stroke,fill}%
}%
\begin{pgfscope}%
\pgfsys@transformshift{2.520534in}{1.164970in}%
\pgfsys@useobject{currentmarker}{}%
\end{pgfscope}%
\end{pgfscope}%
\begin{pgfscope}%
\pgfsetbuttcap%
\pgfsetroundjoin%
\definecolor{currentfill}{rgb}{0.150000,0.150000,0.150000}%
\pgfsetfillcolor{currentfill}%
\pgfsetlinewidth{0.803000pt}%
\definecolor{currentstroke}{rgb}{0.150000,0.150000,0.150000}%
\pgfsetstrokecolor{currentstroke}%
\pgfsetdash{}{0pt}%
\pgfsys@defobject{currentmarker}{\pgfqpoint{0.000000in}{-0.044444in}}{\pgfqpoint{0.000000in}{0.000000in}}{%
\pgfpathmoveto{\pgfqpoint{0.000000in}{0.000000in}}%
\pgfpathlineto{\pgfqpoint{0.000000in}{-0.044444in}}%
\pgfusepath{stroke,fill}%
}%
\begin{pgfscope}%
\pgfsys@transformshift{2.560237in}{1.164970in}%
\pgfsys@useobject{currentmarker}{}%
\end{pgfscope}%
\end{pgfscope}%
\begin{pgfscope}%
\pgfsetbuttcap%
\pgfsetroundjoin%
\definecolor{currentfill}{rgb}{0.150000,0.150000,0.150000}%
\pgfsetfillcolor{currentfill}%
\pgfsetlinewidth{0.803000pt}%
\definecolor{currentstroke}{rgb}{0.150000,0.150000,0.150000}%
\pgfsetstrokecolor{currentstroke}%
\pgfsetdash{}{0pt}%
\pgfsys@defobject{currentmarker}{\pgfqpoint{0.000000in}{-0.044444in}}{\pgfqpoint{0.000000in}{0.000000in}}{%
\pgfpathmoveto{\pgfqpoint{0.000000in}{0.000000in}}%
\pgfpathlineto{\pgfqpoint{0.000000in}{-0.044444in}}%
\pgfusepath{stroke,fill}%
}%
\begin{pgfscope}%
\pgfsys@transformshift{2.593804in}{1.164970in}%
\pgfsys@useobject{currentmarker}{}%
\end{pgfscope}%
\end{pgfscope}%
\begin{pgfscope}%
\pgfsetbuttcap%
\pgfsetroundjoin%
\definecolor{currentfill}{rgb}{0.150000,0.150000,0.150000}%
\pgfsetfillcolor{currentfill}%
\pgfsetlinewidth{0.803000pt}%
\definecolor{currentstroke}{rgb}{0.150000,0.150000,0.150000}%
\pgfsetstrokecolor{currentstroke}%
\pgfsetdash{}{0pt}%
\pgfsys@defobject{currentmarker}{\pgfqpoint{0.000000in}{-0.044444in}}{\pgfqpoint{0.000000in}{0.000000in}}{%
\pgfpathmoveto{\pgfqpoint{0.000000in}{0.000000in}}%
\pgfpathlineto{\pgfqpoint{0.000000in}{-0.044444in}}%
\pgfusepath{stroke,fill}%
}%
\begin{pgfscope}%
\pgfsys@transformshift{2.622882in}{1.164970in}%
\pgfsys@useobject{currentmarker}{}%
\end{pgfscope}%
\end{pgfscope}%
\begin{pgfscope}%
\pgfsetbuttcap%
\pgfsetroundjoin%
\definecolor{currentfill}{rgb}{0.150000,0.150000,0.150000}%
\pgfsetfillcolor{currentfill}%
\pgfsetlinewidth{0.803000pt}%
\definecolor{currentstroke}{rgb}{0.150000,0.150000,0.150000}%
\pgfsetstrokecolor{currentstroke}%
\pgfsetdash{}{0pt}%
\pgfsys@defobject{currentmarker}{\pgfqpoint{0.000000in}{-0.044444in}}{\pgfqpoint{0.000000in}{0.000000in}}{%
\pgfpathmoveto{\pgfqpoint{0.000000in}{0.000000in}}%
\pgfpathlineto{\pgfqpoint{0.000000in}{-0.044444in}}%
\pgfusepath{stroke,fill}%
}%
\begin{pgfscope}%
\pgfsys@transformshift{2.648530in}{1.164970in}%
\pgfsys@useobject{currentmarker}{}%
\end{pgfscope}%
\end{pgfscope}%
\begin{pgfscope}%
\pgfsetbuttcap%
\pgfsetroundjoin%
\definecolor{currentfill}{rgb}{0.150000,0.150000,0.150000}%
\pgfsetfillcolor{currentfill}%
\pgfsetlinewidth{0.803000pt}%
\definecolor{currentstroke}{rgb}{0.150000,0.150000,0.150000}%
\pgfsetstrokecolor{currentstroke}%
\pgfsetdash{}{0pt}%
\pgfsys@defobject{currentmarker}{\pgfqpoint{0.000000in}{-0.044444in}}{\pgfqpoint{0.000000in}{0.000000in}}{%
\pgfpathmoveto{\pgfqpoint{0.000000in}{0.000000in}}%
\pgfpathlineto{\pgfqpoint{0.000000in}{-0.044444in}}%
\pgfusepath{stroke,fill}%
}%
\begin{pgfscope}%
\pgfsys@transformshift{2.822413in}{1.164970in}%
\pgfsys@useobject{currentmarker}{}%
\end{pgfscope}%
\end{pgfscope}%
\begin{pgfscope}%
\pgfsetbuttcap%
\pgfsetroundjoin%
\definecolor{currentfill}{rgb}{0.150000,0.150000,0.150000}%
\pgfsetfillcolor{currentfill}%
\pgfsetlinewidth{0.803000pt}%
\definecolor{currentstroke}{rgb}{0.150000,0.150000,0.150000}%
\pgfsetstrokecolor{currentstroke}%
\pgfsetdash{}{0pt}%
\pgfsys@defobject{currentmarker}{\pgfqpoint{0.000000in}{-0.044444in}}{\pgfqpoint{0.000000in}{0.000000in}}{%
\pgfpathmoveto{\pgfqpoint{0.000000in}{0.000000in}}%
\pgfpathlineto{\pgfqpoint{0.000000in}{-0.044444in}}%
\pgfusepath{stroke,fill}%
}%
\begin{pgfscope}%
\pgfsys@transformshift{2.910706in}{1.164970in}%
\pgfsys@useobject{currentmarker}{}%
\end{pgfscope}%
\end{pgfscope}%
\begin{pgfscope}%
\pgfsetbuttcap%
\pgfsetroundjoin%
\definecolor{currentfill}{rgb}{0.150000,0.150000,0.150000}%
\pgfsetfillcolor{currentfill}%
\pgfsetlinewidth{0.803000pt}%
\definecolor{currentstroke}{rgb}{0.150000,0.150000,0.150000}%
\pgfsetstrokecolor{currentstroke}%
\pgfsetdash{}{0pt}%
\pgfsys@defobject{currentmarker}{\pgfqpoint{0.000000in}{-0.044444in}}{\pgfqpoint{0.000000in}{0.000000in}}{%
\pgfpathmoveto{\pgfqpoint{0.000000in}{0.000000in}}%
\pgfpathlineto{\pgfqpoint{0.000000in}{-0.044444in}}%
\pgfusepath{stroke,fill}%
}%
\begin{pgfscope}%
\pgfsys@transformshift{2.973352in}{1.164970in}%
\pgfsys@useobject{currentmarker}{}%
\end{pgfscope}%
\end{pgfscope}%
\begin{pgfscope}%
\pgfsetbuttcap%
\pgfsetroundjoin%
\definecolor{currentfill}{rgb}{0.150000,0.150000,0.150000}%
\pgfsetfillcolor{currentfill}%
\pgfsetlinewidth{0.803000pt}%
\definecolor{currentstroke}{rgb}{0.150000,0.150000,0.150000}%
\pgfsetstrokecolor{currentstroke}%
\pgfsetdash{}{0pt}%
\pgfsys@defobject{currentmarker}{\pgfqpoint{0.000000in}{-0.044444in}}{\pgfqpoint{0.000000in}{0.000000in}}{%
\pgfpathmoveto{\pgfqpoint{0.000000in}{0.000000in}}%
\pgfpathlineto{\pgfqpoint{0.000000in}{-0.044444in}}%
\pgfusepath{stroke,fill}%
}%
\begin{pgfscope}%
\pgfsys@transformshift{3.021943in}{1.164970in}%
\pgfsys@useobject{currentmarker}{}%
\end{pgfscope}%
\end{pgfscope}%
\begin{pgfscope}%
\pgfsetbuttcap%
\pgfsetroundjoin%
\definecolor{currentfill}{rgb}{0.150000,0.150000,0.150000}%
\pgfsetfillcolor{currentfill}%
\pgfsetlinewidth{0.803000pt}%
\definecolor{currentstroke}{rgb}{0.150000,0.150000,0.150000}%
\pgfsetstrokecolor{currentstroke}%
\pgfsetdash{}{0pt}%
\pgfsys@defobject{currentmarker}{\pgfqpoint{0.000000in}{-0.044444in}}{\pgfqpoint{0.000000in}{0.000000in}}{%
\pgfpathmoveto{\pgfqpoint{0.000000in}{0.000000in}}%
\pgfpathlineto{\pgfqpoint{0.000000in}{-0.044444in}}%
\pgfusepath{stroke,fill}%
}%
\begin{pgfscope}%
\pgfsys@transformshift{3.061646in}{1.164970in}%
\pgfsys@useobject{currentmarker}{}%
\end{pgfscope}%
\end{pgfscope}%
\begin{pgfscope}%
\pgfsetbuttcap%
\pgfsetroundjoin%
\definecolor{currentfill}{rgb}{0.150000,0.150000,0.150000}%
\pgfsetfillcolor{currentfill}%
\pgfsetlinewidth{0.803000pt}%
\definecolor{currentstroke}{rgb}{0.150000,0.150000,0.150000}%
\pgfsetstrokecolor{currentstroke}%
\pgfsetdash{}{0pt}%
\pgfsys@defobject{currentmarker}{\pgfqpoint{0.000000in}{-0.044444in}}{\pgfqpoint{0.000000in}{0.000000in}}{%
\pgfpathmoveto{\pgfqpoint{0.000000in}{0.000000in}}%
\pgfpathlineto{\pgfqpoint{0.000000in}{-0.044444in}}%
\pgfusepath{stroke,fill}%
}%
\begin{pgfscope}%
\pgfsys@transformshift{3.095213in}{1.164970in}%
\pgfsys@useobject{currentmarker}{}%
\end{pgfscope}%
\end{pgfscope}%
\begin{pgfscope}%
\pgfsetbuttcap%
\pgfsetroundjoin%
\definecolor{currentfill}{rgb}{0.150000,0.150000,0.150000}%
\pgfsetfillcolor{currentfill}%
\pgfsetlinewidth{0.803000pt}%
\definecolor{currentstroke}{rgb}{0.150000,0.150000,0.150000}%
\pgfsetstrokecolor{currentstroke}%
\pgfsetdash{}{0pt}%
\pgfsys@defobject{currentmarker}{\pgfqpoint{0.000000in}{-0.044444in}}{\pgfqpoint{0.000000in}{0.000000in}}{%
\pgfpathmoveto{\pgfqpoint{0.000000in}{0.000000in}}%
\pgfpathlineto{\pgfqpoint{0.000000in}{-0.044444in}}%
\pgfusepath{stroke,fill}%
}%
\begin{pgfscope}%
\pgfsys@transformshift{3.124291in}{1.164970in}%
\pgfsys@useobject{currentmarker}{}%
\end{pgfscope}%
\end{pgfscope}%
\begin{pgfscope}%
\pgfsetbuttcap%
\pgfsetroundjoin%
\definecolor{currentfill}{rgb}{0.150000,0.150000,0.150000}%
\pgfsetfillcolor{currentfill}%
\pgfsetlinewidth{0.803000pt}%
\definecolor{currentstroke}{rgb}{0.150000,0.150000,0.150000}%
\pgfsetstrokecolor{currentstroke}%
\pgfsetdash{}{0pt}%
\pgfsys@defobject{currentmarker}{\pgfqpoint{0.000000in}{-0.044444in}}{\pgfqpoint{0.000000in}{0.000000in}}{%
\pgfpathmoveto{\pgfqpoint{0.000000in}{0.000000in}}%
\pgfpathlineto{\pgfqpoint{0.000000in}{-0.044444in}}%
\pgfusepath{stroke,fill}%
}%
\begin{pgfscope}%
\pgfsys@transformshift{3.149939in}{1.164970in}%
\pgfsys@useobject{currentmarker}{}%
\end{pgfscope}%
\end{pgfscope}%
\begin{pgfscope}%
\pgfsetbuttcap%
\pgfsetroundjoin%
\definecolor{currentfill}{rgb}{0.150000,0.150000,0.150000}%
\pgfsetfillcolor{currentfill}%
\pgfsetlinewidth{0.803000pt}%
\definecolor{currentstroke}{rgb}{0.150000,0.150000,0.150000}%
\pgfsetstrokecolor{currentstroke}%
\pgfsetdash{}{0pt}%
\pgfsys@defobject{currentmarker}{\pgfqpoint{0.000000in}{-0.044444in}}{\pgfqpoint{0.000000in}{0.000000in}}{%
\pgfpathmoveto{\pgfqpoint{0.000000in}{0.000000in}}%
\pgfpathlineto{\pgfqpoint{0.000000in}{-0.044444in}}%
\pgfusepath{stroke,fill}%
}%
\begin{pgfscope}%
\pgfsys@transformshift{3.323822in}{1.164970in}%
\pgfsys@useobject{currentmarker}{}%
\end{pgfscope}%
\end{pgfscope}%
\begin{pgfscope}%
\pgfsetbuttcap%
\pgfsetroundjoin%
\definecolor{currentfill}{rgb}{0.150000,0.150000,0.150000}%
\pgfsetfillcolor{currentfill}%
\pgfsetlinewidth{1.003750pt}%
\definecolor{currentstroke}{rgb}{0.150000,0.150000,0.150000}%
\pgfsetstrokecolor{currentstroke}%
\pgfsetdash{}{0pt}%
\pgfsys@defobject{currentmarker}{\pgfqpoint{-0.066667in}{0.000000in}}{\pgfqpoint{0.000000in}{0.000000in}}{%
\pgfpathmoveto{\pgfqpoint{0.000000in}{0.000000in}}%
\pgfpathlineto{\pgfqpoint{-0.066667in}{0.000000in}}%
\pgfusepath{stroke,fill}%
}%
\begin{pgfscope}%
\pgfsys@transformshift{2.170064in}{1.164970in}%
\pgfsys@useobject{currentmarker}{}%
\end{pgfscope}%
\end{pgfscope}%
\begin{pgfscope}%
\pgfsetbuttcap%
\pgfsetroundjoin%
\definecolor{currentfill}{rgb}{0.150000,0.150000,0.150000}%
\pgfsetfillcolor{currentfill}%
\pgfsetlinewidth{1.003750pt}%
\definecolor{currentstroke}{rgb}{0.150000,0.150000,0.150000}%
\pgfsetstrokecolor{currentstroke}%
\pgfsetdash{}{0pt}%
\pgfsys@defobject{currentmarker}{\pgfqpoint{-0.066667in}{0.000000in}}{\pgfqpoint{0.000000in}{0.000000in}}{%
\pgfpathmoveto{\pgfqpoint{0.000000in}{0.000000in}}%
\pgfpathlineto{\pgfqpoint{-0.066667in}{0.000000in}}%
\pgfusepath{stroke,fill}%
}%
\begin{pgfscope}%
\pgfsys@transformshift{2.170064in}{1.491207in}%
\pgfsys@useobject{currentmarker}{}%
\end{pgfscope}%
\end{pgfscope}%
\begin{pgfscope}%
\pgfsetbuttcap%
\pgfsetroundjoin%
\definecolor{currentfill}{rgb}{0.150000,0.150000,0.150000}%
\pgfsetfillcolor{currentfill}%
\pgfsetlinewidth{1.003750pt}%
\definecolor{currentstroke}{rgb}{0.150000,0.150000,0.150000}%
\pgfsetstrokecolor{currentstroke}%
\pgfsetdash{}{0pt}%
\pgfsys@defobject{currentmarker}{\pgfqpoint{-0.066667in}{0.000000in}}{\pgfqpoint{0.000000in}{0.000000in}}{%
\pgfpathmoveto{\pgfqpoint{0.000000in}{0.000000in}}%
\pgfpathlineto{\pgfqpoint{-0.066667in}{0.000000in}}%
\pgfusepath{stroke,fill}%
}%
\begin{pgfscope}%
\pgfsys@transformshift{2.170064in}{1.772918in}%
\pgfsys@useobject{currentmarker}{}%
\end{pgfscope}%
\end{pgfscope}%
\begin{pgfscope}%
\pgfpathrectangle{\pgfqpoint{2.170064in}{1.164970in}}{\pgfqpoint{1.223103in}{0.607948in}}%
\pgfusepath{clip}%
\pgfsetroundcap%
\pgfsetroundjoin%
\pgfsetlinewidth{1.204500pt}%
\definecolor{currentstroke}{rgb}{0.000000,0.501961,0.000000}%
\pgfsetstrokecolor{currentstroke}%
\pgfsetdash{}{0pt}%
\pgfpathmoveto{\pgfqpoint{2.170064in}{1.490195in}}%
\pgfpathlineto{\pgfqpoint{2.410774in}{1.490908in}}%
\pgfpathlineto{\pgfqpoint{2.522305in}{1.491612in}}%
\pgfpathlineto{\pgfqpoint{2.595701in}{1.492306in}}%
\pgfpathlineto{\pgfqpoint{2.650497in}{1.492987in}}%
\pgfpathlineto{\pgfqpoint{2.694239in}{1.493656in}}%
\pgfpathlineto{\pgfqpoint{2.730647in}{1.494310in}}%
\pgfpathlineto{\pgfqpoint{2.761831in}{1.494947in}}%
\pgfpathlineto{\pgfqpoint{2.789104in}{1.495566in}}%
\pgfpathlineto{\pgfqpoint{2.813338in}{1.496165in}}%
\pgfpathlineto{\pgfqpoint{2.835143in}{1.496742in}}%
\pgfpathlineto{\pgfqpoint{2.854962in}{1.497296in}}%
\pgfpathlineto{\pgfqpoint{2.873127in}{1.497825in}}%
\pgfpathlineto{\pgfqpoint{2.889892in}{1.498327in}}%
\pgfpathlineto{\pgfqpoint{2.905459in}{1.498799in}}%
\pgfpathlineto{\pgfqpoint{2.919986in}{1.499240in}}%
\pgfpathlineto{\pgfqpoint{2.933605in}{1.499648in}}%
\pgfpathlineto{\pgfqpoint{2.946422in}{1.500020in}}%
\pgfpathlineto{\pgfqpoint{2.958526in}{1.500355in}}%
\pgfpathlineto{\pgfqpoint{2.969993in}{1.500650in}}%
\pgfpathlineto{\pgfqpoint{2.980886in}{1.500903in}}%
\pgfpathlineto{\pgfqpoint{2.991260in}{1.501113in}}%
\pgfpathlineto{\pgfqpoint{3.001162in}{1.501276in}}%
\pgfpathlineto{\pgfqpoint{3.010633in}{1.501390in}}%
\pgfpathlineto{\pgfqpoint{3.019710in}{1.501453in}}%
\pgfpathlineto{\pgfqpoint{3.028423in}{1.501462in}}%
\pgfpathlineto{\pgfqpoint{3.036801in}{1.501416in}}%
\pgfpathlineto{\pgfqpoint{3.044869in}{1.501312in}}%
\pgfpathlineto{\pgfqpoint{3.052648in}{1.501146in}}%
\pgfpathlineto{\pgfqpoint{3.060159in}{1.500918in}}%
\pgfpathlineto{\pgfqpoint{3.067420in}{1.500623in}}%
\pgfpathlineto{\pgfqpoint{3.074446in}{1.500260in}}%
\pgfpathlineto{\pgfqpoint{3.081253in}{1.499826in}}%
\pgfpathlineto{\pgfqpoint{3.087853in}{1.499319in}}%
\pgfpathlineto{\pgfqpoint{3.094259in}{1.498735in}}%
\pgfpathlineto{\pgfqpoint{3.100482in}{1.498073in}}%
\pgfpathlineto{\pgfqpoint{3.106532in}{1.497328in}}%
\pgfpathlineto{\pgfqpoint{3.112419in}{1.496500in}}%
\pgfpathlineto{\pgfqpoint{3.118150in}{1.495585in}}%
\pgfpathlineto{\pgfqpoint{3.123735in}{1.494580in}}%
\pgfpathlineto{\pgfqpoint{3.129180in}{1.493482in}}%
\pgfpathlineto{\pgfqpoint{3.134492in}{1.492289in}}%
\pgfpathlineto{\pgfqpoint{3.139677in}{1.490999in}}%
\pgfpathlineto{\pgfqpoint{3.144742in}{1.489607in}}%
\pgfpathlineto{\pgfqpoint{3.149692in}{1.488112in}}%
\pgfpathlineto{\pgfqpoint{3.154532in}{1.486511in}}%
\pgfpathlineto{\pgfqpoint{3.159267in}{1.484800in}}%
\pgfpathlineto{\pgfqpoint{3.163901in}{1.482977in}}%
\pgfpathlineto{\pgfqpoint{3.168438in}{1.481040in}}%
\pgfpathlineto{\pgfqpoint{3.172883in}{1.478984in}}%
\pgfusepath{stroke}%
\end{pgfscope}%
\begin{pgfscope}%
\pgfsetrectcap%
\pgfsetmiterjoin%
\pgfsetlinewidth{1.003750pt}%
\definecolor{currentstroke}{rgb}{0.150000,0.150000,0.150000}%
\pgfsetstrokecolor{currentstroke}%
\pgfsetdash{}{0pt}%
\pgfpathmoveto{\pgfqpoint{2.170064in}{1.164970in}}%
\pgfpathlineto{\pgfqpoint{2.170064in}{1.772918in}}%
\pgfusepath{stroke}%
\end{pgfscope}%
\begin{pgfscope}%
\pgfsetrectcap%
\pgfsetmiterjoin%
\pgfsetlinewidth{1.003750pt}%
\definecolor{currentstroke}{rgb}{0.150000,0.150000,0.150000}%
\pgfsetstrokecolor{currentstroke}%
\pgfsetdash{}{0pt}%
\pgfpathmoveto{\pgfqpoint{2.170064in}{1.164970in}}%
\pgfpathlineto{\pgfqpoint{3.393168in}{1.164970in}}%
\pgfusepath{stroke}%
\end{pgfscope}%
\begin{pgfscope}%
\pgfpathrectangle{\pgfqpoint{2.170064in}{1.164970in}}{\pgfqpoint{1.223103in}{0.607948in}}%
\pgfusepath{clip}%
\pgfsetbuttcap%
\pgfsetroundjoin%
\definecolor{currentfill}{rgb}{0.000000,0.000000,0.000000}%
\pgfsetfillcolor{currentfill}%
\pgfsetlinewidth{1.003750pt}%
\definecolor{currentstroke}{rgb}{0.000000,0.000000,0.000000}%
\pgfsetstrokecolor{currentstroke}%
\pgfsetdash{}{0pt}%
\pgfsys@defobject{currentmarker}{\pgfqpoint{-0.013889in}{-0.013889in}}{\pgfqpoint{0.013889in}{0.013889in}}{%
\pgfpathmoveto{\pgfqpoint{0.000000in}{-0.013889in}}%
\pgfpathcurveto{\pgfqpoint{0.003683in}{-0.013889in}}{\pgfqpoint{0.007216in}{-0.012425in}}{\pgfqpoint{0.009821in}{-0.009821in}}%
\pgfpathcurveto{\pgfqpoint{0.012425in}{-0.007216in}}{\pgfqpoint{0.013889in}{-0.003683in}}{\pgfqpoint{0.013889in}{0.000000in}}%
\pgfpathcurveto{\pgfqpoint{0.013889in}{0.003683in}}{\pgfqpoint{0.012425in}{0.007216in}}{\pgfqpoint{0.009821in}{0.009821in}}%
\pgfpathcurveto{\pgfqpoint{0.007216in}{0.012425in}}{\pgfqpoint{0.003683in}{0.013889in}}{\pgfqpoint{0.000000in}{0.013889in}}%
\pgfpathcurveto{\pgfqpoint{-0.003683in}{0.013889in}}{\pgfqpoint{-0.007216in}{0.012425in}}{\pgfqpoint{-0.009821in}{0.009821in}}%
\pgfpathcurveto{\pgfqpoint{-0.012425in}{0.007216in}}{\pgfqpoint{-0.013889in}{0.003683in}}{\pgfqpoint{-0.013889in}{0.000000in}}%
\pgfpathcurveto{\pgfqpoint{-0.013889in}{-0.003683in}}{\pgfqpoint{-0.012425in}{-0.007216in}}{\pgfqpoint{-0.009821in}{-0.009821in}}%
\pgfpathcurveto{\pgfqpoint{-0.007216in}{-0.012425in}}{\pgfqpoint{-0.003683in}{-0.013889in}}{\pgfqpoint{0.000000in}{-0.013889in}}%
\pgfpathclose%
\pgfusepath{stroke,fill}%
}%
\begin{pgfscope}%
\pgfsys@transformshift{3.172883in}{1.478905in}%
\pgfsys@useobject{currentmarker}{}%
\end{pgfscope}%
\begin{pgfscope}%
\pgfsys@transformshift{2.822413in}{1.496505in}%
\pgfsys@useobject{currentmarker}{}%
\end{pgfscope}%
\begin{pgfscope}%
\pgfsys@transformshift{2.826812in}{1.496606in}%
\pgfsys@useobject{currentmarker}{}%
\end{pgfscope}%
\begin{pgfscope}%
\pgfsys@transformshift{2.831302in}{1.496711in}%
\pgfsys@useobject{currentmarker}{}%
\end{pgfscope}%
\begin{pgfscope}%
\pgfsys@transformshift{2.835887in}{1.496820in}%
\pgfsys@useobject{currentmarker}{}%
\end{pgfscope}%
\begin{pgfscope}%
\pgfsys@transformshift{2.840570in}{1.496932in}%
\pgfsys@useobject{currentmarker}{}%
\end{pgfscope}%
\begin{pgfscope}%
\pgfsys@transformshift{2.845356in}{1.497049in}%
\pgfsys@useobject{currentmarker}{}%
\end{pgfscope}%
\begin{pgfscope}%
\pgfsys@transformshift{2.850250in}{1.497170in}%
\pgfsys@useobject{currentmarker}{}%
\end{pgfscope}%
\begin{pgfscope}%
\pgfsys@transformshift{2.855256in}{1.497296in}%
\pgfsys@useobject{currentmarker}{}%
\end{pgfscope}%
\begin{pgfscope}%
\pgfsys@transformshift{2.860380in}{1.497427in}%
\pgfsys@useobject{currentmarker}{}%
\end{pgfscope}%
\begin{pgfscope}%
\pgfsys@transformshift{2.865627in}{1.497563in}%
\pgfsys@useobject{currentmarker}{}%
\end{pgfscope}%
\begin{pgfscope}%
\pgfsys@transformshift{2.871004in}{1.497705in}%
\pgfsys@useobject{currentmarker}{}%
\end{pgfscope}%
\begin{pgfscope}%
\pgfsys@transformshift{2.900082in}{1.498500in}%
\pgfsys@useobject{currentmarker}{}%
\end{pgfscope}%
\begin{pgfscope}%
\pgfsys@transformshift{2.933650in}{1.499459in}%
\pgfsys@useobject{currentmarker}{}%
\end{pgfscope}%
\begin{pgfscope}%
\pgfsys@transformshift{3.084589in}{1.499970in}%
\pgfsys@useobject{currentmarker}{}%
\end{pgfscope}%
\begin{pgfscope}%
\pgfsys@transformshift{2.973352in}{1.500549in}%
\pgfsys@useobject{currentmarker}{}%
\end{pgfscope}%
\begin{pgfscope}%
\pgfsys@transformshift{3.021943in}{1.501434in}%
\pgfsys@useobject{currentmarker}{}%
\end{pgfscope}%
\end{pgfscope}%
\begin{pgfscope}%
\pgfsetbuttcap%
\pgfsetmiterjoin%
\definecolor{currentfill}{rgb}{1.000000,1.000000,1.000000}%
\pgfsetfillcolor{currentfill}%
\pgfsetlinewidth{0.000000pt}%
\definecolor{currentstroke}{rgb}{0.000000,0.000000,0.000000}%
\pgfsetstrokecolor{currentstroke}%
\pgfsetstrokeopacity{0.000000}%
\pgfsetdash{}{0pt}%
\pgfpathmoveto{\pgfqpoint{3.637789in}{1.164970in}}%
\pgfpathlineto{\pgfqpoint{4.860892in}{1.164970in}}%
\pgfpathlineto{\pgfqpoint{4.860892in}{1.772918in}}%
\pgfpathlineto{\pgfqpoint{3.637789in}{1.772918in}}%
\pgfpathclose%
\pgfusepath{fill}%
\end{pgfscope}%
\begin{pgfscope}%
\pgfpathrectangle{\pgfqpoint{3.637789in}{1.164970in}}{\pgfqpoint{1.223103in}{0.607948in}}%
\pgfusepath{clip}%
\pgfsetbuttcap%
\pgfsetmiterjoin%
\definecolor{currentfill}{rgb}{0.000000,0.000000,1.000000}%
\pgfsetfillcolor{currentfill}%
\pgfsetfillopacity{0.100000}%
\pgfsetlinewidth{0.803000pt}%
\definecolor{currentstroke}{rgb}{0.000000,0.000000,1.000000}%
\pgfsetstrokecolor{currentstroke}%
\pgfsetstrokeopacity{0.100000}%
\pgfsetdash{}{0pt}%
\pgfpathmoveto{\pgfqpoint{3.637789in}{1.490552in}}%
\pgfpathlineto{\pgfqpoint{3.637789in}{1.491578in}}%
\pgfpathlineto{\pgfqpoint{4.860892in}{1.491578in}}%
\pgfpathlineto{\pgfqpoint{4.860892in}{1.490552in}}%
\pgfpathclose%
\pgfusepath{stroke,fill}%
\end{pgfscope}%
\begin{pgfscope}%
\pgfpathrectangle{\pgfqpoint{3.637789in}{1.164970in}}{\pgfqpoint{1.223103in}{0.607948in}}%
\pgfusepath{clip}%
\pgfsetbuttcap%
\pgfsetroundjoin%
\definecolor{currentfill}{rgb}{0.000000,0.501961,0.000000}%
\pgfsetfillcolor{currentfill}%
\pgfsetfillopacity{0.500000}%
\pgfsetlinewidth{0.803000pt}%
\definecolor{currentstroke}{rgb}{0.000000,0.501961,0.000000}%
\pgfsetstrokecolor{currentstroke}%
\pgfsetstrokeopacity{0.500000}%
\pgfsetdash{}{0pt}%
\pgfpathmoveto{\pgfqpoint{3.637789in}{1.492004in}}%
\pgfpathlineto{\pgfqpoint{3.637789in}{1.491051in}}%
\pgfpathlineto{\pgfqpoint{3.878498in}{1.492054in}}%
\pgfpathlineto{\pgfqpoint{3.990029in}{1.493051in}}%
\pgfpathlineto{\pgfqpoint{4.063425in}{1.494042in}}%
\pgfpathlineto{\pgfqpoint{4.118221in}{1.495027in}}%
\pgfpathlineto{\pgfqpoint{4.161963in}{1.496006in}}%
\pgfpathlineto{\pgfqpoint{4.198371in}{1.496980in}}%
\pgfpathlineto{\pgfqpoint{4.229555in}{1.497948in}}%
\pgfpathlineto{\pgfqpoint{4.256828in}{1.498911in}}%
\pgfpathlineto{\pgfqpoint{4.281062in}{1.499870in}}%
\pgfpathlineto{\pgfqpoint{4.302867in}{1.500823in}}%
\pgfpathlineto{\pgfqpoint{4.322686in}{1.501770in}}%
\pgfpathlineto{\pgfqpoint{4.340851in}{1.502680in}}%
\pgfpathlineto{\pgfqpoint{4.357617in}{1.503590in}}%
\pgfpathlineto{\pgfqpoint{4.373183in}{1.504501in}}%
\pgfpathlineto{\pgfqpoint{4.387711in}{1.505413in}}%
\pgfpathlineto{\pgfqpoint{4.401329in}{1.506326in}}%
\pgfpathlineto{\pgfqpoint{4.414146in}{1.507240in}}%
\pgfpathlineto{\pgfqpoint{4.426250in}{1.508153in}}%
\pgfpathlineto{\pgfqpoint{4.437717in}{1.509067in}}%
\pgfpathlineto{\pgfqpoint{4.448610in}{1.509980in}}%
\pgfpathlineto{\pgfqpoint{4.458984in}{1.510892in}}%
\pgfpathlineto{\pgfqpoint{4.468886in}{1.511803in}}%
\pgfpathlineto{\pgfqpoint{4.478357in}{1.512711in}}%
\pgfpathlineto{\pgfqpoint{4.487434in}{1.513616in}}%
\pgfpathlineto{\pgfqpoint{4.496147in}{1.514513in}}%
\pgfpathlineto{\pgfqpoint{4.504525in}{1.515396in}}%
\pgfpathlineto{\pgfqpoint{4.512593in}{1.516266in}}%
\pgfpathlineto{\pgfqpoint{4.520372in}{1.517126in}}%
\pgfpathlineto{\pgfqpoint{4.527883in}{1.517980in}}%
\pgfpathlineto{\pgfqpoint{4.535144in}{1.518827in}}%
\pgfpathlineto{\pgfqpoint{4.542170in}{1.519666in}}%
\pgfpathlineto{\pgfqpoint{4.548977in}{1.520498in}}%
\pgfpathlineto{\pgfqpoint{4.555577in}{1.521322in}}%
\pgfpathlineto{\pgfqpoint{4.561983in}{1.522139in}}%
\pgfpathlineto{\pgfqpoint{4.568206in}{1.522948in}}%
\pgfpathlineto{\pgfqpoint{4.574256in}{1.523748in}}%
\pgfpathlineto{\pgfqpoint{4.580143in}{1.524539in}}%
\pgfpathlineto{\pgfqpoint{4.585874in}{1.525322in}}%
\pgfpathlineto{\pgfqpoint{4.591459in}{1.526095in}}%
\pgfpathlineto{\pgfqpoint{4.596904in}{1.526859in}}%
\pgfpathlineto{\pgfqpoint{4.602216in}{1.527613in}}%
\pgfpathlineto{\pgfqpoint{4.607401in}{1.528357in}}%
\pgfpathlineto{\pgfqpoint{4.612466in}{1.529091in}}%
\pgfpathlineto{\pgfqpoint{4.617416in}{1.529814in}}%
\pgfpathlineto{\pgfqpoint{4.622256in}{1.530525in}}%
\pgfpathlineto{\pgfqpoint{4.626991in}{1.531226in}}%
\pgfpathlineto{\pgfqpoint{4.631625in}{1.531914in}}%
\pgfpathlineto{\pgfqpoint{4.636162in}{1.532589in}}%
\pgfpathlineto{\pgfqpoint{4.640607in}{1.533177in}}%
\pgfpathlineto{\pgfqpoint{4.640607in}{1.533262in}}%
\pgfpathlineto{\pgfqpoint{4.640607in}{1.533262in}}%
\pgfpathlineto{\pgfqpoint{4.636162in}{1.532644in}}%
\pgfpathlineto{\pgfqpoint{4.631625in}{1.532066in}}%
\pgfpathlineto{\pgfqpoint{4.626991in}{1.531459in}}%
\pgfpathlineto{\pgfqpoint{4.622256in}{1.530823in}}%
\pgfpathlineto{\pgfqpoint{4.617416in}{1.530161in}}%
\pgfpathlineto{\pgfqpoint{4.612466in}{1.529473in}}%
\pgfpathlineto{\pgfqpoint{4.607401in}{1.528763in}}%
\pgfpathlineto{\pgfqpoint{4.602216in}{1.528031in}}%
\pgfpathlineto{\pgfqpoint{4.596904in}{1.527279in}}%
\pgfpathlineto{\pgfqpoint{4.591459in}{1.526509in}}%
\pgfpathlineto{\pgfqpoint{4.585874in}{1.525723in}}%
\pgfpathlineto{\pgfqpoint{4.580143in}{1.524922in}}%
\pgfpathlineto{\pgfqpoint{4.574256in}{1.524106in}}%
\pgfpathlineto{\pgfqpoint{4.568206in}{1.523279in}}%
\pgfpathlineto{\pgfqpoint{4.561983in}{1.522439in}}%
\pgfpathlineto{\pgfqpoint{4.555577in}{1.521590in}}%
\pgfpathlineto{\pgfqpoint{4.548977in}{1.520731in}}%
\pgfpathlineto{\pgfqpoint{4.542170in}{1.519863in}}%
\pgfpathlineto{\pgfqpoint{4.535144in}{1.518988in}}%
\pgfpathlineto{\pgfqpoint{4.527883in}{1.518107in}}%
\pgfpathlineto{\pgfqpoint{4.520372in}{1.517220in}}%
\pgfpathlineto{\pgfqpoint{4.512593in}{1.516329in}}%
\pgfpathlineto{\pgfqpoint{4.504525in}{1.515436in}}%
\pgfpathlineto{\pgfqpoint{4.496147in}{1.514545in}}%
\pgfpathlineto{\pgfqpoint{4.487434in}{1.513661in}}%
\pgfpathlineto{\pgfqpoint{4.478357in}{1.512775in}}%
\pgfpathlineto{\pgfqpoint{4.468886in}{1.511885in}}%
\pgfpathlineto{\pgfqpoint{4.458984in}{1.510990in}}%
\pgfpathlineto{\pgfqpoint{4.448610in}{1.510090in}}%
\pgfpathlineto{\pgfqpoint{4.437717in}{1.509185in}}%
\pgfpathlineto{\pgfqpoint{4.426250in}{1.508275in}}%
\pgfpathlineto{\pgfqpoint{4.414146in}{1.507360in}}%
\pgfpathlineto{\pgfqpoint{4.401329in}{1.506441in}}%
\pgfpathlineto{\pgfqpoint{4.387711in}{1.505516in}}%
\pgfpathlineto{\pgfqpoint{4.373183in}{1.504587in}}%
\pgfpathlineto{\pgfqpoint{4.357617in}{1.503653in}}%
\pgfpathlineto{\pgfqpoint{4.340851in}{1.502715in}}%
\pgfpathlineto{\pgfqpoint{4.322686in}{1.501774in}}%
\pgfpathlineto{\pgfqpoint{4.302867in}{1.500867in}}%
\pgfpathlineto{\pgfqpoint{4.281062in}{1.499964in}}%
\pgfpathlineto{\pgfqpoint{4.256828in}{1.499064in}}%
\pgfpathlineto{\pgfqpoint{4.229555in}{1.498167in}}%
\pgfpathlineto{\pgfqpoint{4.198371in}{1.497273in}}%
\pgfpathlineto{\pgfqpoint{4.161963in}{1.496383in}}%
\pgfpathlineto{\pgfqpoint{4.118221in}{1.495497in}}%
\pgfpathlineto{\pgfqpoint{4.063425in}{1.494616in}}%
\pgfpathlineto{\pgfqpoint{3.990029in}{1.493740in}}%
\pgfpathlineto{\pgfqpoint{3.878498in}{1.492869in}}%
\pgfpathlineto{\pgfqpoint{3.637789in}{1.492004in}}%
\pgfpathclose%
\pgfusepath{stroke,fill}%
\end{pgfscope}%
\begin{pgfscope}%
\pgfpathrectangle{\pgfqpoint{3.637789in}{1.164970in}}{\pgfqpoint{1.223103in}{0.607948in}}%
\pgfusepath{clip}%
\pgfsetroundcap%
\pgfsetroundjoin%
\pgfsetlinewidth{0.501875pt}%
\definecolor{currentstroke}{rgb}{0.000000,0.000000,1.000000}%
\pgfsetstrokecolor{currentstroke}%
\pgfsetstrokeopacity{0.800000}%
\pgfsetdash{}{0pt}%
\pgfpathmoveto{\pgfqpoint{3.637789in}{1.491065in}}%
\pgfpathlineto{\pgfqpoint{4.860892in}{1.491065in}}%
\pgfusepath{stroke}%
\end{pgfscope}%
\begin{pgfscope}%
\pgfpathrectangle{\pgfqpoint{3.637789in}{1.164970in}}{\pgfqpoint{1.223103in}{0.607948in}}%
\pgfusepath{clip}%
\pgfsetbuttcap%
\pgfsetroundjoin%
\pgfsetlinewidth{1.003750pt}%
\definecolor{currentstroke}{rgb}{0.000000,0.000000,0.000000}%
\pgfsetstrokecolor{currentstroke}%
\pgfsetdash{{3.700000pt}{1.600000pt}}{0.000000pt}%
\pgfpathmoveto{\pgfqpoint{3.637789in}{1.491207in}}%
\pgfpathlineto{\pgfqpoint{4.860892in}{1.491207in}}%
\pgfusepath{stroke}%
\end{pgfscope}%
\begin{pgfscope}%
\pgfsetroundcap%
\pgfsetroundjoin%
\pgfsetlinewidth{0.501875pt}%
\definecolor{currentstroke}{rgb}{0.000000,0.000000,1.000000}%
\pgfsetstrokecolor{currentstroke}%
\pgfsetstrokeopacity{0.800000}%
\pgfsetdash{}{0pt}%
\pgfpathmoveto{\pgfqpoint{4.464868in}{1.609169in}}%
\pgfpathquadraticcurveto{\pgfqpoint{4.389004in}{1.557891in}}{\pgfqpoint{4.313141in}{1.506614in}}%
\pgfusepath{stroke}%
\end{pgfscope}%
\begin{pgfscope}%
\pgfsetfillopacity{0.800000}%
\pgfsetstrokeopacity{0.800000}%
\definecolor{textcolor}{rgb}{0.000000,0.000000,1.000000}%
\pgfsetstrokecolor{textcolor}%
\pgfsetfillcolor{textcolor}%
\pgftext[x=4.378430in,y=1.673450in,left,base]{\color{textcolor}\sffamily\fontsize{5.647059}{6.776471}\selectfont 3.53639(84)}%
\end{pgfscope}%
\begin{pgfscope}%
\pgfsetbuttcap%
\pgfsetroundjoin%
\definecolor{currentfill}{rgb}{0.150000,0.150000,0.150000}%
\pgfsetfillcolor{currentfill}%
\pgfsetlinewidth{1.003750pt}%
\definecolor{currentstroke}{rgb}{0.150000,0.150000,0.150000}%
\pgfsetstrokecolor{currentstroke}%
\pgfsetdash{}{0pt}%
\pgfsys@defobject{currentmarker}{\pgfqpoint{0.000000in}{-0.066667in}}{\pgfqpoint{0.000000in}{0.000000in}}{%
\pgfpathmoveto{\pgfqpoint{0.000000in}{0.000000in}}%
\pgfpathlineto{\pgfqpoint{0.000000in}{-0.066667in}}%
\pgfusepath{stroke,fill}%
}%
\begin{pgfscope}%
\pgfsys@transformshift{3.637789in}{1.164970in}%
\pgfsys@useobject{currentmarker}{}%
\end{pgfscope}%
\end{pgfscope}%
\begin{pgfscope}%
\pgfsetbuttcap%
\pgfsetroundjoin%
\definecolor{currentfill}{rgb}{0.150000,0.150000,0.150000}%
\pgfsetfillcolor{currentfill}%
\pgfsetlinewidth{1.003750pt}%
\definecolor{currentstroke}{rgb}{0.150000,0.150000,0.150000}%
\pgfsetstrokecolor{currentstroke}%
\pgfsetdash{}{0pt}%
\pgfsys@defobject{currentmarker}{\pgfqpoint{0.000000in}{-0.066667in}}{\pgfqpoint{0.000000in}{0.000000in}}{%
\pgfpathmoveto{\pgfqpoint{0.000000in}{0.000000in}}%
\pgfpathlineto{\pgfqpoint{0.000000in}{-0.066667in}}%
\pgfusepath{stroke,fill}%
}%
\begin{pgfscope}%
\pgfsys@transformshift{4.139198in}{1.164970in}%
\pgfsys@useobject{currentmarker}{}%
\end{pgfscope}%
\end{pgfscope}%
\begin{pgfscope}%
\pgfsetbuttcap%
\pgfsetroundjoin%
\definecolor{currentfill}{rgb}{0.150000,0.150000,0.150000}%
\pgfsetfillcolor{currentfill}%
\pgfsetlinewidth{1.003750pt}%
\definecolor{currentstroke}{rgb}{0.150000,0.150000,0.150000}%
\pgfsetstrokecolor{currentstroke}%
\pgfsetdash{}{0pt}%
\pgfsys@defobject{currentmarker}{\pgfqpoint{0.000000in}{-0.066667in}}{\pgfqpoint{0.000000in}{0.000000in}}{%
\pgfpathmoveto{\pgfqpoint{0.000000in}{0.000000in}}%
\pgfpathlineto{\pgfqpoint{0.000000in}{-0.066667in}}%
\pgfusepath{stroke,fill}%
}%
\begin{pgfscope}%
\pgfsys@transformshift{4.640607in}{1.164970in}%
\pgfsys@useobject{currentmarker}{}%
\end{pgfscope}%
\end{pgfscope}%
\begin{pgfscope}%
\pgfsetbuttcap%
\pgfsetroundjoin%
\definecolor{currentfill}{rgb}{0.150000,0.150000,0.150000}%
\pgfsetfillcolor{currentfill}%
\pgfsetlinewidth{0.803000pt}%
\definecolor{currentstroke}{rgb}{0.150000,0.150000,0.150000}%
\pgfsetstrokecolor{currentstroke}%
\pgfsetdash{}{0pt}%
\pgfsys@defobject{currentmarker}{\pgfqpoint{0.000000in}{-0.044444in}}{\pgfqpoint{0.000000in}{0.000000in}}{%
\pgfpathmoveto{\pgfqpoint{0.000000in}{0.000000in}}%
\pgfpathlineto{\pgfqpoint{0.000000in}{-0.044444in}}%
\pgfusepath{stroke,fill}%
}%
\begin{pgfscope}%
\pgfsys@transformshift{3.788728in}{1.164970in}%
\pgfsys@useobject{currentmarker}{}%
\end{pgfscope}%
\end{pgfscope}%
\begin{pgfscope}%
\pgfsetbuttcap%
\pgfsetroundjoin%
\definecolor{currentfill}{rgb}{0.150000,0.150000,0.150000}%
\pgfsetfillcolor{currentfill}%
\pgfsetlinewidth{0.803000pt}%
\definecolor{currentstroke}{rgb}{0.150000,0.150000,0.150000}%
\pgfsetstrokecolor{currentstroke}%
\pgfsetdash{}{0pt}%
\pgfsys@defobject{currentmarker}{\pgfqpoint{0.000000in}{-0.044444in}}{\pgfqpoint{0.000000in}{0.000000in}}{%
\pgfpathmoveto{\pgfqpoint{0.000000in}{0.000000in}}%
\pgfpathlineto{\pgfqpoint{0.000000in}{-0.044444in}}%
\pgfusepath{stroke,fill}%
}%
\begin{pgfscope}%
\pgfsys@transformshift{3.877021in}{1.164970in}%
\pgfsys@useobject{currentmarker}{}%
\end{pgfscope}%
\end{pgfscope}%
\begin{pgfscope}%
\pgfsetbuttcap%
\pgfsetroundjoin%
\definecolor{currentfill}{rgb}{0.150000,0.150000,0.150000}%
\pgfsetfillcolor{currentfill}%
\pgfsetlinewidth{0.803000pt}%
\definecolor{currentstroke}{rgb}{0.150000,0.150000,0.150000}%
\pgfsetstrokecolor{currentstroke}%
\pgfsetdash{}{0pt}%
\pgfsys@defobject{currentmarker}{\pgfqpoint{0.000000in}{-0.044444in}}{\pgfqpoint{0.000000in}{0.000000in}}{%
\pgfpathmoveto{\pgfqpoint{0.000000in}{0.000000in}}%
\pgfpathlineto{\pgfqpoint{0.000000in}{-0.044444in}}%
\pgfusepath{stroke,fill}%
}%
\begin{pgfscope}%
\pgfsys@transformshift{3.939667in}{1.164970in}%
\pgfsys@useobject{currentmarker}{}%
\end{pgfscope}%
\end{pgfscope}%
\begin{pgfscope}%
\pgfsetbuttcap%
\pgfsetroundjoin%
\definecolor{currentfill}{rgb}{0.150000,0.150000,0.150000}%
\pgfsetfillcolor{currentfill}%
\pgfsetlinewidth{0.803000pt}%
\definecolor{currentstroke}{rgb}{0.150000,0.150000,0.150000}%
\pgfsetstrokecolor{currentstroke}%
\pgfsetdash{}{0pt}%
\pgfsys@defobject{currentmarker}{\pgfqpoint{0.000000in}{-0.044444in}}{\pgfqpoint{0.000000in}{0.000000in}}{%
\pgfpathmoveto{\pgfqpoint{0.000000in}{0.000000in}}%
\pgfpathlineto{\pgfqpoint{0.000000in}{-0.044444in}}%
\pgfusepath{stroke,fill}%
}%
\begin{pgfscope}%
\pgfsys@transformshift{3.988258in}{1.164970in}%
\pgfsys@useobject{currentmarker}{}%
\end{pgfscope}%
\end{pgfscope}%
\begin{pgfscope}%
\pgfsetbuttcap%
\pgfsetroundjoin%
\definecolor{currentfill}{rgb}{0.150000,0.150000,0.150000}%
\pgfsetfillcolor{currentfill}%
\pgfsetlinewidth{0.803000pt}%
\definecolor{currentstroke}{rgb}{0.150000,0.150000,0.150000}%
\pgfsetstrokecolor{currentstroke}%
\pgfsetdash{}{0pt}%
\pgfsys@defobject{currentmarker}{\pgfqpoint{0.000000in}{-0.044444in}}{\pgfqpoint{0.000000in}{0.000000in}}{%
\pgfpathmoveto{\pgfqpoint{0.000000in}{0.000000in}}%
\pgfpathlineto{\pgfqpoint{0.000000in}{-0.044444in}}%
\pgfusepath{stroke,fill}%
}%
\begin{pgfscope}%
\pgfsys@transformshift{4.027961in}{1.164970in}%
\pgfsys@useobject{currentmarker}{}%
\end{pgfscope}%
\end{pgfscope}%
\begin{pgfscope}%
\pgfsetbuttcap%
\pgfsetroundjoin%
\definecolor{currentfill}{rgb}{0.150000,0.150000,0.150000}%
\pgfsetfillcolor{currentfill}%
\pgfsetlinewidth{0.803000pt}%
\definecolor{currentstroke}{rgb}{0.150000,0.150000,0.150000}%
\pgfsetstrokecolor{currentstroke}%
\pgfsetdash{}{0pt}%
\pgfsys@defobject{currentmarker}{\pgfqpoint{0.000000in}{-0.044444in}}{\pgfqpoint{0.000000in}{0.000000in}}{%
\pgfpathmoveto{\pgfqpoint{0.000000in}{0.000000in}}%
\pgfpathlineto{\pgfqpoint{0.000000in}{-0.044444in}}%
\pgfusepath{stroke,fill}%
}%
\begin{pgfscope}%
\pgfsys@transformshift{4.061528in}{1.164970in}%
\pgfsys@useobject{currentmarker}{}%
\end{pgfscope}%
\end{pgfscope}%
\begin{pgfscope}%
\pgfsetbuttcap%
\pgfsetroundjoin%
\definecolor{currentfill}{rgb}{0.150000,0.150000,0.150000}%
\pgfsetfillcolor{currentfill}%
\pgfsetlinewidth{0.803000pt}%
\definecolor{currentstroke}{rgb}{0.150000,0.150000,0.150000}%
\pgfsetstrokecolor{currentstroke}%
\pgfsetdash{}{0pt}%
\pgfsys@defobject{currentmarker}{\pgfqpoint{0.000000in}{-0.044444in}}{\pgfqpoint{0.000000in}{0.000000in}}{%
\pgfpathmoveto{\pgfqpoint{0.000000in}{0.000000in}}%
\pgfpathlineto{\pgfqpoint{0.000000in}{-0.044444in}}%
\pgfusepath{stroke,fill}%
}%
\begin{pgfscope}%
\pgfsys@transformshift{4.090606in}{1.164970in}%
\pgfsys@useobject{currentmarker}{}%
\end{pgfscope}%
\end{pgfscope}%
\begin{pgfscope}%
\pgfsetbuttcap%
\pgfsetroundjoin%
\definecolor{currentfill}{rgb}{0.150000,0.150000,0.150000}%
\pgfsetfillcolor{currentfill}%
\pgfsetlinewidth{0.803000pt}%
\definecolor{currentstroke}{rgb}{0.150000,0.150000,0.150000}%
\pgfsetstrokecolor{currentstroke}%
\pgfsetdash{}{0pt}%
\pgfsys@defobject{currentmarker}{\pgfqpoint{0.000000in}{-0.044444in}}{\pgfqpoint{0.000000in}{0.000000in}}{%
\pgfpathmoveto{\pgfqpoint{0.000000in}{0.000000in}}%
\pgfpathlineto{\pgfqpoint{0.000000in}{-0.044444in}}%
\pgfusepath{stroke,fill}%
}%
\begin{pgfscope}%
\pgfsys@transformshift{4.116254in}{1.164970in}%
\pgfsys@useobject{currentmarker}{}%
\end{pgfscope}%
\end{pgfscope}%
\begin{pgfscope}%
\pgfsetbuttcap%
\pgfsetroundjoin%
\definecolor{currentfill}{rgb}{0.150000,0.150000,0.150000}%
\pgfsetfillcolor{currentfill}%
\pgfsetlinewidth{0.803000pt}%
\definecolor{currentstroke}{rgb}{0.150000,0.150000,0.150000}%
\pgfsetstrokecolor{currentstroke}%
\pgfsetdash{}{0pt}%
\pgfsys@defobject{currentmarker}{\pgfqpoint{0.000000in}{-0.044444in}}{\pgfqpoint{0.000000in}{0.000000in}}{%
\pgfpathmoveto{\pgfqpoint{0.000000in}{0.000000in}}%
\pgfpathlineto{\pgfqpoint{0.000000in}{-0.044444in}}%
\pgfusepath{stroke,fill}%
}%
\begin{pgfscope}%
\pgfsys@transformshift{4.290137in}{1.164970in}%
\pgfsys@useobject{currentmarker}{}%
\end{pgfscope}%
\end{pgfscope}%
\begin{pgfscope}%
\pgfsetbuttcap%
\pgfsetroundjoin%
\definecolor{currentfill}{rgb}{0.150000,0.150000,0.150000}%
\pgfsetfillcolor{currentfill}%
\pgfsetlinewidth{0.803000pt}%
\definecolor{currentstroke}{rgb}{0.150000,0.150000,0.150000}%
\pgfsetstrokecolor{currentstroke}%
\pgfsetdash{}{0pt}%
\pgfsys@defobject{currentmarker}{\pgfqpoint{0.000000in}{-0.044444in}}{\pgfqpoint{0.000000in}{0.000000in}}{%
\pgfpathmoveto{\pgfqpoint{0.000000in}{0.000000in}}%
\pgfpathlineto{\pgfqpoint{0.000000in}{-0.044444in}}%
\pgfusepath{stroke,fill}%
}%
\begin{pgfscope}%
\pgfsys@transformshift{4.378430in}{1.164970in}%
\pgfsys@useobject{currentmarker}{}%
\end{pgfscope}%
\end{pgfscope}%
\begin{pgfscope}%
\pgfsetbuttcap%
\pgfsetroundjoin%
\definecolor{currentfill}{rgb}{0.150000,0.150000,0.150000}%
\pgfsetfillcolor{currentfill}%
\pgfsetlinewidth{0.803000pt}%
\definecolor{currentstroke}{rgb}{0.150000,0.150000,0.150000}%
\pgfsetstrokecolor{currentstroke}%
\pgfsetdash{}{0pt}%
\pgfsys@defobject{currentmarker}{\pgfqpoint{0.000000in}{-0.044444in}}{\pgfqpoint{0.000000in}{0.000000in}}{%
\pgfpathmoveto{\pgfqpoint{0.000000in}{0.000000in}}%
\pgfpathlineto{\pgfqpoint{0.000000in}{-0.044444in}}%
\pgfusepath{stroke,fill}%
}%
\begin{pgfscope}%
\pgfsys@transformshift{4.441076in}{1.164970in}%
\pgfsys@useobject{currentmarker}{}%
\end{pgfscope}%
\end{pgfscope}%
\begin{pgfscope}%
\pgfsetbuttcap%
\pgfsetroundjoin%
\definecolor{currentfill}{rgb}{0.150000,0.150000,0.150000}%
\pgfsetfillcolor{currentfill}%
\pgfsetlinewidth{0.803000pt}%
\definecolor{currentstroke}{rgb}{0.150000,0.150000,0.150000}%
\pgfsetstrokecolor{currentstroke}%
\pgfsetdash{}{0pt}%
\pgfsys@defobject{currentmarker}{\pgfqpoint{0.000000in}{-0.044444in}}{\pgfqpoint{0.000000in}{0.000000in}}{%
\pgfpathmoveto{\pgfqpoint{0.000000in}{0.000000in}}%
\pgfpathlineto{\pgfqpoint{0.000000in}{-0.044444in}}%
\pgfusepath{stroke,fill}%
}%
\begin{pgfscope}%
\pgfsys@transformshift{4.489667in}{1.164970in}%
\pgfsys@useobject{currentmarker}{}%
\end{pgfscope}%
\end{pgfscope}%
\begin{pgfscope}%
\pgfsetbuttcap%
\pgfsetroundjoin%
\definecolor{currentfill}{rgb}{0.150000,0.150000,0.150000}%
\pgfsetfillcolor{currentfill}%
\pgfsetlinewidth{0.803000pt}%
\definecolor{currentstroke}{rgb}{0.150000,0.150000,0.150000}%
\pgfsetstrokecolor{currentstroke}%
\pgfsetdash{}{0pt}%
\pgfsys@defobject{currentmarker}{\pgfqpoint{0.000000in}{-0.044444in}}{\pgfqpoint{0.000000in}{0.000000in}}{%
\pgfpathmoveto{\pgfqpoint{0.000000in}{0.000000in}}%
\pgfpathlineto{\pgfqpoint{0.000000in}{-0.044444in}}%
\pgfusepath{stroke,fill}%
}%
\begin{pgfscope}%
\pgfsys@transformshift{4.529370in}{1.164970in}%
\pgfsys@useobject{currentmarker}{}%
\end{pgfscope}%
\end{pgfscope}%
\begin{pgfscope}%
\pgfsetbuttcap%
\pgfsetroundjoin%
\definecolor{currentfill}{rgb}{0.150000,0.150000,0.150000}%
\pgfsetfillcolor{currentfill}%
\pgfsetlinewidth{0.803000pt}%
\definecolor{currentstroke}{rgb}{0.150000,0.150000,0.150000}%
\pgfsetstrokecolor{currentstroke}%
\pgfsetdash{}{0pt}%
\pgfsys@defobject{currentmarker}{\pgfqpoint{0.000000in}{-0.044444in}}{\pgfqpoint{0.000000in}{0.000000in}}{%
\pgfpathmoveto{\pgfqpoint{0.000000in}{0.000000in}}%
\pgfpathlineto{\pgfqpoint{0.000000in}{-0.044444in}}%
\pgfusepath{stroke,fill}%
}%
\begin{pgfscope}%
\pgfsys@transformshift{4.562937in}{1.164970in}%
\pgfsys@useobject{currentmarker}{}%
\end{pgfscope}%
\end{pgfscope}%
\begin{pgfscope}%
\pgfsetbuttcap%
\pgfsetroundjoin%
\definecolor{currentfill}{rgb}{0.150000,0.150000,0.150000}%
\pgfsetfillcolor{currentfill}%
\pgfsetlinewidth{0.803000pt}%
\definecolor{currentstroke}{rgb}{0.150000,0.150000,0.150000}%
\pgfsetstrokecolor{currentstroke}%
\pgfsetdash{}{0pt}%
\pgfsys@defobject{currentmarker}{\pgfqpoint{0.000000in}{-0.044444in}}{\pgfqpoint{0.000000in}{0.000000in}}{%
\pgfpathmoveto{\pgfqpoint{0.000000in}{0.000000in}}%
\pgfpathlineto{\pgfqpoint{0.000000in}{-0.044444in}}%
\pgfusepath{stroke,fill}%
}%
\begin{pgfscope}%
\pgfsys@transformshift{4.592015in}{1.164970in}%
\pgfsys@useobject{currentmarker}{}%
\end{pgfscope}%
\end{pgfscope}%
\begin{pgfscope}%
\pgfsetbuttcap%
\pgfsetroundjoin%
\definecolor{currentfill}{rgb}{0.150000,0.150000,0.150000}%
\pgfsetfillcolor{currentfill}%
\pgfsetlinewidth{0.803000pt}%
\definecolor{currentstroke}{rgb}{0.150000,0.150000,0.150000}%
\pgfsetstrokecolor{currentstroke}%
\pgfsetdash{}{0pt}%
\pgfsys@defobject{currentmarker}{\pgfqpoint{0.000000in}{-0.044444in}}{\pgfqpoint{0.000000in}{0.000000in}}{%
\pgfpathmoveto{\pgfqpoint{0.000000in}{0.000000in}}%
\pgfpathlineto{\pgfqpoint{0.000000in}{-0.044444in}}%
\pgfusepath{stroke,fill}%
}%
\begin{pgfscope}%
\pgfsys@transformshift{4.617663in}{1.164970in}%
\pgfsys@useobject{currentmarker}{}%
\end{pgfscope}%
\end{pgfscope}%
\begin{pgfscope}%
\pgfsetbuttcap%
\pgfsetroundjoin%
\definecolor{currentfill}{rgb}{0.150000,0.150000,0.150000}%
\pgfsetfillcolor{currentfill}%
\pgfsetlinewidth{0.803000pt}%
\definecolor{currentstroke}{rgb}{0.150000,0.150000,0.150000}%
\pgfsetstrokecolor{currentstroke}%
\pgfsetdash{}{0pt}%
\pgfsys@defobject{currentmarker}{\pgfqpoint{0.000000in}{-0.044444in}}{\pgfqpoint{0.000000in}{0.000000in}}{%
\pgfpathmoveto{\pgfqpoint{0.000000in}{0.000000in}}%
\pgfpathlineto{\pgfqpoint{0.000000in}{-0.044444in}}%
\pgfusepath{stroke,fill}%
}%
\begin{pgfscope}%
\pgfsys@transformshift{4.791546in}{1.164970in}%
\pgfsys@useobject{currentmarker}{}%
\end{pgfscope}%
\end{pgfscope}%
\begin{pgfscope}%
\pgfsetbuttcap%
\pgfsetroundjoin%
\definecolor{currentfill}{rgb}{0.150000,0.150000,0.150000}%
\pgfsetfillcolor{currentfill}%
\pgfsetlinewidth{1.003750pt}%
\definecolor{currentstroke}{rgb}{0.150000,0.150000,0.150000}%
\pgfsetstrokecolor{currentstroke}%
\pgfsetdash{}{0pt}%
\pgfsys@defobject{currentmarker}{\pgfqpoint{-0.066667in}{0.000000in}}{\pgfqpoint{0.000000in}{0.000000in}}{%
\pgfpathmoveto{\pgfqpoint{0.000000in}{0.000000in}}%
\pgfpathlineto{\pgfqpoint{-0.066667in}{0.000000in}}%
\pgfusepath{stroke,fill}%
}%
\begin{pgfscope}%
\pgfsys@transformshift{3.637789in}{1.164970in}%
\pgfsys@useobject{currentmarker}{}%
\end{pgfscope}%
\end{pgfscope}%
\begin{pgfscope}%
\pgfsetbuttcap%
\pgfsetroundjoin%
\definecolor{currentfill}{rgb}{0.150000,0.150000,0.150000}%
\pgfsetfillcolor{currentfill}%
\pgfsetlinewidth{1.003750pt}%
\definecolor{currentstroke}{rgb}{0.150000,0.150000,0.150000}%
\pgfsetstrokecolor{currentstroke}%
\pgfsetdash{}{0pt}%
\pgfsys@defobject{currentmarker}{\pgfqpoint{-0.066667in}{0.000000in}}{\pgfqpoint{0.000000in}{0.000000in}}{%
\pgfpathmoveto{\pgfqpoint{0.000000in}{0.000000in}}%
\pgfpathlineto{\pgfqpoint{-0.066667in}{0.000000in}}%
\pgfusepath{stroke,fill}%
}%
\begin{pgfscope}%
\pgfsys@transformshift{3.637789in}{1.491207in}%
\pgfsys@useobject{currentmarker}{}%
\end{pgfscope}%
\end{pgfscope}%
\begin{pgfscope}%
\pgfsetbuttcap%
\pgfsetroundjoin%
\definecolor{currentfill}{rgb}{0.150000,0.150000,0.150000}%
\pgfsetfillcolor{currentfill}%
\pgfsetlinewidth{1.003750pt}%
\definecolor{currentstroke}{rgb}{0.150000,0.150000,0.150000}%
\pgfsetstrokecolor{currentstroke}%
\pgfsetdash{}{0pt}%
\pgfsys@defobject{currentmarker}{\pgfqpoint{-0.066667in}{0.000000in}}{\pgfqpoint{0.000000in}{0.000000in}}{%
\pgfpathmoveto{\pgfqpoint{0.000000in}{0.000000in}}%
\pgfpathlineto{\pgfqpoint{-0.066667in}{0.000000in}}%
\pgfusepath{stroke,fill}%
}%
\begin{pgfscope}%
\pgfsys@transformshift{3.637789in}{1.772918in}%
\pgfsys@useobject{currentmarker}{}%
\end{pgfscope}%
\end{pgfscope}%
\begin{pgfscope}%
\pgfpathrectangle{\pgfqpoint{3.637789in}{1.164970in}}{\pgfqpoint{1.223103in}{0.607948in}}%
\pgfusepath{clip}%
\pgfsetroundcap%
\pgfsetroundjoin%
\pgfsetlinewidth{1.204500pt}%
\definecolor{currentstroke}{rgb}{0.000000,0.501961,0.000000}%
\pgfsetstrokecolor{currentstroke}%
\pgfsetdash{}{0pt}%
\pgfpathmoveto{\pgfqpoint{3.637789in}{1.491528in}}%
\pgfpathlineto{\pgfqpoint{3.878498in}{1.492462in}}%
\pgfpathlineto{\pgfqpoint{3.990029in}{1.493396in}}%
\pgfpathlineto{\pgfqpoint{4.063425in}{1.494329in}}%
\pgfpathlineto{\pgfqpoint{4.118221in}{1.495262in}}%
\pgfpathlineto{\pgfqpoint{4.161963in}{1.496195in}}%
\pgfpathlineto{\pgfqpoint{4.198371in}{1.497127in}}%
\pgfpathlineto{\pgfqpoint{4.229555in}{1.498058in}}%
\pgfpathlineto{\pgfqpoint{4.256828in}{1.498988in}}%
\pgfpathlineto{\pgfqpoint{4.281062in}{1.499917in}}%
\pgfpathlineto{\pgfqpoint{4.302867in}{1.500845in}}%
\pgfpathlineto{\pgfqpoint{4.322686in}{1.501772in}}%
\pgfpathlineto{\pgfqpoint{4.340851in}{1.502697in}}%
\pgfpathlineto{\pgfqpoint{4.357617in}{1.503621in}}%
\pgfpathlineto{\pgfqpoint{4.373183in}{1.504544in}}%
\pgfpathlineto{\pgfqpoint{4.387711in}{1.505465in}}%
\pgfpathlineto{\pgfqpoint{4.401329in}{1.506383in}}%
\pgfpathlineto{\pgfqpoint{4.414146in}{1.507300in}}%
\pgfpathlineto{\pgfqpoint{4.426250in}{1.508214in}}%
\pgfpathlineto{\pgfqpoint{4.437717in}{1.509126in}}%
\pgfpathlineto{\pgfqpoint{4.448610in}{1.510035in}}%
\pgfpathlineto{\pgfqpoint{4.458984in}{1.510941in}}%
\pgfpathlineto{\pgfqpoint{4.468886in}{1.511844in}}%
\pgfpathlineto{\pgfqpoint{4.478357in}{1.512743in}}%
\pgfpathlineto{\pgfqpoint{4.487434in}{1.513638in}}%
\pgfpathlineto{\pgfqpoint{4.496147in}{1.514529in}}%
\pgfpathlineto{\pgfqpoint{4.504525in}{1.515416in}}%
\pgfpathlineto{\pgfqpoint{4.512593in}{1.516297in}}%
\pgfpathlineto{\pgfqpoint{4.520372in}{1.517173in}}%
\pgfpathlineto{\pgfqpoint{4.527883in}{1.518044in}}%
\pgfpathlineto{\pgfqpoint{4.535144in}{1.518908in}}%
\pgfpathlineto{\pgfqpoint{4.542170in}{1.519765in}}%
\pgfpathlineto{\pgfqpoint{4.548977in}{1.520614in}}%
\pgfpathlineto{\pgfqpoint{4.555577in}{1.521456in}}%
\pgfpathlineto{\pgfqpoint{4.561983in}{1.522289in}}%
\pgfpathlineto{\pgfqpoint{4.568206in}{1.523113in}}%
\pgfpathlineto{\pgfqpoint{4.574256in}{1.523927in}}%
\pgfpathlineto{\pgfqpoint{4.580143in}{1.524731in}}%
\pgfpathlineto{\pgfqpoint{4.585874in}{1.525523in}}%
\pgfpathlineto{\pgfqpoint{4.591459in}{1.526302in}}%
\pgfpathlineto{\pgfqpoint{4.596904in}{1.527069in}}%
\pgfpathlineto{\pgfqpoint{4.602216in}{1.527822in}}%
\pgfpathlineto{\pgfqpoint{4.607401in}{1.528560in}}%
\pgfpathlineto{\pgfqpoint{4.612466in}{1.529282in}}%
\pgfpathlineto{\pgfqpoint{4.617416in}{1.529987in}}%
\pgfpathlineto{\pgfqpoint{4.622256in}{1.530674in}}%
\pgfpathlineto{\pgfqpoint{4.626991in}{1.531342in}}%
\pgfpathlineto{\pgfqpoint{4.631625in}{1.531990in}}%
\pgfpathlineto{\pgfqpoint{4.636162in}{1.532616in}}%
\pgfpathlineto{\pgfqpoint{4.640607in}{1.533220in}}%
\pgfusepath{stroke}%
\end{pgfscope}%
\begin{pgfscope}%
\pgfsetrectcap%
\pgfsetmiterjoin%
\pgfsetlinewidth{1.003750pt}%
\definecolor{currentstroke}{rgb}{0.150000,0.150000,0.150000}%
\pgfsetstrokecolor{currentstroke}%
\pgfsetdash{}{0pt}%
\pgfpathmoveto{\pgfqpoint{3.637789in}{1.164970in}}%
\pgfpathlineto{\pgfqpoint{3.637789in}{1.772918in}}%
\pgfusepath{stroke}%
\end{pgfscope}%
\begin{pgfscope}%
\pgfsetrectcap%
\pgfsetmiterjoin%
\pgfsetlinewidth{1.003750pt}%
\definecolor{currentstroke}{rgb}{0.150000,0.150000,0.150000}%
\pgfsetstrokecolor{currentstroke}%
\pgfsetdash{}{0pt}%
\pgfpathmoveto{\pgfqpoint{3.637789in}{1.164970in}}%
\pgfpathlineto{\pgfqpoint{4.860892in}{1.164970in}}%
\pgfusepath{stroke}%
\end{pgfscope}%
\begin{pgfscope}%
\pgfpathrectangle{\pgfqpoint{3.637789in}{1.164970in}}{\pgfqpoint{1.223103in}{0.607948in}}%
\pgfusepath{clip}%
\pgfsetbuttcap%
\pgfsetroundjoin%
\definecolor{currentfill}{rgb}{0.000000,0.000000,0.000000}%
\pgfsetfillcolor{currentfill}%
\pgfsetlinewidth{1.003750pt}%
\definecolor{currentstroke}{rgb}{0.000000,0.000000,0.000000}%
\pgfsetstrokecolor{currentstroke}%
\pgfsetdash{}{0pt}%
\pgfsys@defobject{currentmarker}{\pgfqpoint{-0.013889in}{-0.013889in}}{\pgfqpoint{0.013889in}{0.013889in}}{%
\pgfpathmoveto{\pgfqpoint{0.000000in}{-0.013889in}}%
\pgfpathcurveto{\pgfqpoint{0.003683in}{-0.013889in}}{\pgfqpoint{0.007216in}{-0.012425in}}{\pgfqpoint{0.009821in}{-0.009821in}}%
\pgfpathcurveto{\pgfqpoint{0.012425in}{-0.007216in}}{\pgfqpoint{0.013889in}{-0.003683in}}{\pgfqpoint{0.013889in}{0.000000in}}%
\pgfpathcurveto{\pgfqpoint{0.013889in}{0.003683in}}{\pgfqpoint{0.012425in}{0.007216in}}{\pgfqpoint{0.009821in}{0.009821in}}%
\pgfpathcurveto{\pgfqpoint{0.007216in}{0.012425in}}{\pgfqpoint{0.003683in}{0.013889in}}{\pgfqpoint{0.000000in}{0.013889in}}%
\pgfpathcurveto{\pgfqpoint{-0.003683in}{0.013889in}}{\pgfqpoint{-0.007216in}{0.012425in}}{\pgfqpoint{-0.009821in}{0.009821in}}%
\pgfpathcurveto{\pgfqpoint{-0.012425in}{0.007216in}}{\pgfqpoint{-0.013889in}{0.003683in}}{\pgfqpoint{-0.013889in}{0.000000in}}%
\pgfpathcurveto{\pgfqpoint{-0.013889in}{-0.003683in}}{\pgfqpoint{-0.012425in}{-0.007216in}}{\pgfqpoint{-0.009821in}{-0.009821in}}%
\pgfpathcurveto{\pgfqpoint{-0.007216in}{-0.012425in}}{\pgfqpoint{-0.003683in}{-0.013889in}}{\pgfqpoint{0.000000in}{-0.013889in}}%
\pgfpathclose%
\pgfusepath{stroke,fill}%
}%
\begin{pgfscope}%
\pgfsys@transformshift{4.290137in}{1.500308in}%
\pgfsys@useobject{currentmarker}{}%
\end{pgfscope}%
\begin{pgfscope}%
\pgfsys@transformshift{4.294536in}{1.500494in}%
\pgfsys@useobject{currentmarker}{}%
\end{pgfscope}%
\begin{pgfscope}%
\pgfsys@transformshift{4.299026in}{1.500687in}%
\pgfsys@useobject{currentmarker}{}%
\end{pgfscope}%
\begin{pgfscope}%
\pgfsys@transformshift{4.303611in}{1.500889in}%
\pgfsys@useobject{currentmarker}{}%
\end{pgfscope}%
\begin{pgfscope}%
\pgfsys@transformshift{4.308294in}{1.501099in}%
\pgfsys@useobject{currentmarker}{}%
\end{pgfscope}%
\begin{pgfscope}%
\pgfsys@transformshift{4.313080in}{1.501318in}%
\pgfsys@useobject{currentmarker}{}%
\end{pgfscope}%
\begin{pgfscope}%
\pgfsys@transformshift{4.317974in}{1.501548in}%
\pgfsys@useobject{currentmarker}{}%
\end{pgfscope}%
\begin{pgfscope}%
\pgfsys@transformshift{4.322980in}{1.501788in}%
\pgfsys@useobject{currentmarker}{}%
\end{pgfscope}%
\begin{pgfscope}%
\pgfsys@transformshift{4.328104in}{1.502040in}%
\pgfsys@useobject{currentmarker}{}%
\end{pgfscope}%
\begin{pgfscope}%
\pgfsys@transformshift{4.333351in}{1.502303in}%
\pgfsys@useobject{currentmarker}{}%
\end{pgfscope}%
\begin{pgfscope}%
\pgfsys@transformshift{4.338728in}{1.502580in}%
\pgfsys@useobject{currentmarker}{}%
\end{pgfscope}%
\begin{pgfscope}%
\pgfsys@transformshift{4.367806in}{1.504200in}%
\pgfsys@useobject{currentmarker}{}%
\end{pgfscope}%
\begin{pgfscope}%
\pgfsys@transformshift{4.401374in}{1.506357in}%
\pgfsys@useobject{currentmarker}{}%
\end{pgfscope}%
\begin{pgfscope}%
\pgfsys@transformshift{4.441076in}{1.509365in}%
\pgfsys@useobject{currentmarker}{}%
\end{pgfscope}%
\begin{pgfscope}%
\pgfsys@transformshift{4.489667in}{1.513842in}%
\pgfsys@useobject{currentmarker}{}%
\end{pgfscope}%
\begin{pgfscope}%
\pgfsys@transformshift{4.552313in}{1.521116in}%
\pgfsys@useobject{currentmarker}{}%
\end{pgfscope}%
\begin{pgfscope}%
\pgfsys@transformshift{4.640607in}{1.533194in}%
\pgfsys@useobject{currentmarker}{}%
\end{pgfscope}%
\end{pgfscope}%
\begin{pgfscope}%
\pgfsetbuttcap%
\pgfsetmiterjoin%
\definecolor{currentfill}{rgb}{1.000000,1.000000,1.000000}%
\pgfsetfillcolor{currentfill}%
\pgfsetlinewidth{0.000000pt}%
\definecolor{currentstroke}{rgb}{0.000000,0.000000,0.000000}%
\pgfsetstrokecolor{currentstroke}%
\pgfsetstrokeopacity{0.000000}%
\pgfsetdash{}{0pt}%
\pgfpathmoveto{\pgfqpoint{5.105513in}{1.164970in}}%
\pgfpathlineto{\pgfqpoint{6.328616in}{1.164970in}}%
\pgfpathlineto{\pgfqpoint{6.328616in}{1.772918in}}%
\pgfpathlineto{\pgfqpoint{5.105513in}{1.772918in}}%
\pgfpathclose%
\pgfusepath{fill}%
\end{pgfscope}%
\begin{pgfscope}%
\pgfpathrectangle{\pgfqpoint{5.105513in}{1.164970in}}{\pgfqpoint{1.223103in}{0.607948in}}%
\pgfusepath{clip}%
\pgfsetbuttcap%
\pgfsetmiterjoin%
\definecolor{currentfill}{rgb}{0.000000,0.000000,1.000000}%
\pgfsetfillcolor{currentfill}%
\pgfsetfillopacity{0.100000}%
\pgfsetlinewidth{0.803000pt}%
\definecolor{currentstroke}{rgb}{0.000000,0.000000,1.000000}%
\pgfsetstrokecolor{currentstroke}%
\pgfsetstrokeopacity{0.100000}%
\pgfsetdash{}{0pt}%
\pgfpathmoveto{\pgfqpoint{5.105513in}{1.490039in}}%
\pgfpathlineto{\pgfqpoint{5.105513in}{1.491877in}}%
\pgfpathlineto{\pgfqpoint{6.328616in}{1.491877in}}%
\pgfpathlineto{\pgfqpoint{6.328616in}{1.490039in}}%
\pgfpathclose%
\pgfusepath{stroke,fill}%
\end{pgfscope}%
\begin{pgfscope}%
\pgfpathrectangle{\pgfqpoint{5.105513in}{1.164970in}}{\pgfqpoint{1.223103in}{0.607948in}}%
\pgfusepath{clip}%
\pgfsetbuttcap%
\pgfsetroundjoin%
\definecolor{currentfill}{rgb}{0.000000,0.501961,0.000000}%
\pgfsetfillcolor{currentfill}%
\pgfsetfillopacity{0.500000}%
\pgfsetlinewidth{0.803000pt}%
\definecolor{currentstroke}{rgb}{0.000000,0.501961,0.000000}%
\pgfsetstrokecolor{currentstroke}%
\pgfsetstrokeopacity{0.500000}%
\pgfsetdash{}{0pt}%
\pgfpathmoveto{\pgfqpoint{5.105513in}{1.492354in}}%
\pgfpathlineto{\pgfqpoint{5.105513in}{1.490735in}}%
\pgfpathlineto{\pgfqpoint{5.346222in}{1.492097in}}%
\pgfpathlineto{\pgfqpoint{5.457753in}{1.493408in}}%
\pgfpathlineto{\pgfqpoint{5.531149in}{1.494675in}}%
\pgfpathlineto{\pgfqpoint{5.585945in}{1.495906in}}%
\pgfpathlineto{\pgfqpoint{5.629687in}{1.497108in}}%
\pgfpathlineto{\pgfqpoint{5.666095in}{1.498285in}}%
\pgfpathlineto{\pgfqpoint{5.697279in}{1.499442in}}%
\pgfpathlineto{\pgfqpoint{5.724552in}{1.500586in}}%
\pgfpathlineto{\pgfqpoint{5.748786in}{1.501718in}}%
\pgfpathlineto{\pgfqpoint{5.770591in}{1.502843in}}%
\pgfpathlineto{\pgfqpoint{5.790410in}{1.503959in}}%
\pgfpathlineto{\pgfqpoint{5.808575in}{1.505063in}}%
\pgfpathlineto{\pgfqpoint{5.825341in}{1.506169in}}%
\pgfpathlineto{\pgfqpoint{5.840907in}{1.507276in}}%
\pgfpathlineto{\pgfqpoint{5.855435in}{1.508386in}}%
\pgfpathlineto{\pgfqpoint{5.869053in}{1.509498in}}%
\pgfpathlineto{\pgfqpoint{5.881870in}{1.510613in}}%
\pgfpathlineto{\pgfqpoint{5.893974in}{1.511730in}}%
\pgfpathlineto{\pgfqpoint{5.905441in}{1.512849in}}%
\pgfpathlineto{\pgfqpoint{5.916334in}{1.513970in}}%
\pgfpathlineto{\pgfqpoint{5.926708in}{1.515092in}}%
\pgfpathlineto{\pgfqpoint{5.936610in}{1.516215in}}%
\pgfpathlineto{\pgfqpoint{5.946082in}{1.517338in}}%
\pgfpathlineto{\pgfqpoint{5.955158in}{1.518463in}}%
\pgfpathlineto{\pgfqpoint{5.963871in}{1.519589in}}%
\pgfpathlineto{\pgfqpoint{5.972249in}{1.520716in}}%
\pgfpathlineto{\pgfqpoint{5.980317in}{1.521844in}}%
\pgfpathlineto{\pgfqpoint{5.988096in}{1.522973in}}%
\pgfpathlineto{\pgfqpoint{5.995607in}{1.524102in}}%
\pgfpathlineto{\pgfqpoint{6.002868in}{1.525229in}}%
\pgfpathlineto{\pgfqpoint{6.009894in}{1.526350in}}%
\pgfpathlineto{\pgfqpoint{6.016701in}{1.527463in}}%
\pgfpathlineto{\pgfqpoint{6.023301in}{1.528569in}}%
\pgfpathlineto{\pgfqpoint{6.029707in}{1.529672in}}%
\pgfpathlineto{\pgfqpoint{6.035930in}{1.530772in}}%
\pgfpathlineto{\pgfqpoint{6.041980in}{1.531872in}}%
\pgfpathlineto{\pgfqpoint{6.047867in}{1.532972in}}%
\pgfpathlineto{\pgfqpoint{6.053598in}{1.534077in}}%
\pgfpathlineto{\pgfqpoint{6.059183in}{1.535188in}}%
\pgfpathlineto{\pgfqpoint{6.064628in}{1.536309in}}%
\pgfpathlineto{\pgfqpoint{6.069940in}{1.537443in}}%
\pgfpathlineto{\pgfqpoint{6.075126in}{1.538595in}}%
\pgfpathlineto{\pgfqpoint{6.080191in}{1.539770in}}%
\pgfpathlineto{\pgfqpoint{6.085140in}{1.540974in}}%
\pgfpathlineto{\pgfqpoint{6.089980in}{1.542213in}}%
\pgfpathlineto{\pgfqpoint{6.094715in}{1.543493in}}%
\pgfpathlineto{\pgfqpoint{6.099349in}{1.544823in}}%
\pgfpathlineto{\pgfqpoint{6.103886in}{1.546212in}}%
\pgfpathlineto{\pgfqpoint{6.108331in}{1.547643in}}%
\pgfpathlineto{\pgfqpoint{6.108331in}{1.547676in}}%
\pgfpathlineto{\pgfqpoint{6.108331in}{1.547676in}}%
\pgfpathlineto{\pgfqpoint{6.103886in}{1.546457in}}%
\pgfpathlineto{\pgfqpoint{6.099349in}{1.545259in}}%
\pgfpathlineto{\pgfqpoint{6.094715in}{1.544059in}}%
\pgfpathlineto{\pgfqpoint{6.089980in}{1.542859in}}%
\pgfpathlineto{\pgfqpoint{6.085140in}{1.541660in}}%
\pgfpathlineto{\pgfqpoint{6.080191in}{1.540462in}}%
\pgfpathlineto{\pgfqpoint{6.075126in}{1.539267in}}%
\pgfpathlineto{\pgfqpoint{6.069940in}{1.538076in}}%
\pgfpathlineto{\pgfqpoint{6.064628in}{1.536888in}}%
\pgfpathlineto{\pgfqpoint{6.059183in}{1.535705in}}%
\pgfpathlineto{\pgfqpoint{6.053598in}{1.534526in}}%
\pgfpathlineto{\pgfqpoint{6.047867in}{1.533352in}}%
\pgfpathlineto{\pgfqpoint{6.041980in}{1.532184in}}%
\pgfpathlineto{\pgfqpoint{6.035930in}{1.531020in}}%
\pgfpathlineto{\pgfqpoint{6.029707in}{1.529862in}}%
\pgfpathlineto{\pgfqpoint{6.023301in}{1.528710in}}%
\pgfpathlineto{\pgfqpoint{6.016701in}{1.527564in}}%
\pgfpathlineto{\pgfqpoint{6.009894in}{1.526425in}}%
\pgfpathlineto{\pgfqpoint{6.002868in}{1.525294in}}%
\pgfpathlineto{\pgfqpoint{5.995607in}{1.524170in}}%
\pgfpathlineto{\pgfqpoint{5.988096in}{1.523048in}}%
\pgfpathlineto{\pgfqpoint{5.980317in}{1.521925in}}%
\pgfpathlineto{\pgfqpoint{5.972249in}{1.520801in}}%
\pgfpathlineto{\pgfqpoint{5.963871in}{1.519675in}}%
\pgfpathlineto{\pgfqpoint{5.955158in}{1.518549in}}%
\pgfpathlineto{\pgfqpoint{5.946082in}{1.517422in}}%
\pgfpathlineto{\pgfqpoint{5.936610in}{1.516296in}}%
\pgfpathlineto{\pgfqpoint{5.926708in}{1.515170in}}%
\pgfpathlineto{\pgfqpoint{5.916334in}{1.514045in}}%
\pgfpathlineto{\pgfqpoint{5.905441in}{1.512922in}}%
\pgfpathlineto{\pgfqpoint{5.893974in}{1.511799in}}%
\pgfpathlineto{\pgfqpoint{5.881870in}{1.510678in}}%
\pgfpathlineto{\pgfqpoint{5.869053in}{1.509558in}}%
\pgfpathlineto{\pgfqpoint{5.855435in}{1.508439in}}%
\pgfpathlineto{\pgfqpoint{5.840907in}{1.507321in}}%
\pgfpathlineto{\pgfqpoint{5.825341in}{1.506203in}}%
\pgfpathlineto{\pgfqpoint{5.808575in}{1.505085in}}%
\pgfpathlineto{\pgfqpoint{5.790410in}{1.503968in}}%
\pgfpathlineto{\pgfqpoint{5.770591in}{1.502864in}}%
\pgfpathlineto{\pgfqpoint{5.748786in}{1.501768in}}%
\pgfpathlineto{\pgfqpoint{5.724552in}{1.500678in}}%
\pgfpathlineto{\pgfqpoint{5.697279in}{1.499594in}}%
\pgfpathlineto{\pgfqpoint{5.666095in}{1.498519in}}%
\pgfpathlineto{\pgfqpoint{5.629687in}{1.497453in}}%
\pgfpathlineto{\pgfqpoint{5.585945in}{1.496400in}}%
\pgfpathlineto{\pgfqpoint{5.531149in}{1.495361in}}%
\pgfpathlineto{\pgfqpoint{5.457753in}{1.494338in}}%
\pgfpathlineto{\pgfqpoint{5.346222in}{1.493335in}}%
\pgfpathlineto{\pgfqpoint{5.105513in}{1.492354in}}%
\pgfpathclose%
\pgfusepath{stroke,fill}%
\end{pgfscope}%
\begin{pgfscope}%
\pgfpathrectangle{\pgfqpoint{5.105513in}{1.164970in}}{\pgfqpoint{1.223103in}{0.607948in}}%
\pgfusepath{clip}%
\pgfsetroundcap%
\pgfsetroundjoin%
\pgfsetlinewidth{0.501875pt}%
\definecolor{currentstroke}{rgb}{0.000000,0.000000,1.000000}%
\pgfsetstrokecolor{currentstroke}%
\pgfsetstrokeopacity{0.800000}%
\pgfsetdash{}{0pt}%
\pgfpathmoveto{\pgfqpoint{5.105513in}{1.490958in}}%
\pgfpathlineto{\pgfqpoint{6.328616in}{1.490958in}}%
\pgfusepath{stroke}%
\end{pgfscope}%
\begin{pgfscope}%
\pgfpathrectangle{\pgfqpoint{5.105513in}{1.164970in}}{\pgfqpoint{1.223103in}{0.607948in}}%
\pgfusepath{clip}%
\pgfsetbuttcap%
\pgfsetroundjoin%
\pgfsetlinewidth{1.003750pt}%
\definecolor{currentstroke}{rgb}{0.000000,0.000000,0.000000}%
\pgfsetstrokecolor{currentstroke}%
\pgfsetdash{{3.700000pt}{1.600000pt}}{0.000000pt}%
\pgfpathmoveto{\pgfqpoint{5.105513in}{1.491207in}}%
\pgfpathlineto{\pgfqpoint{6.328616in}{1.491207in}}%
\pgfusepath{stroke}%
\end{pgfscope}%
\begin{pgfscope}%
\pgfsetroundcap%
\pgfsetroundjoin%
\pgfsetlinewidth{0.501875pt}%
\definecolor{currentstroke}{rgb}{0.000000,0.000000,1.000000}%
\pgfsetstrokecolor{currentstroke}%
\pgfsetstrokeopacity{0.800000}%
\pgfsetdash{}{0pt}%
\pgfpathmoveto{\pgfqpoint{5.919361in}{1.608295in}}%
\pgfpathquadraticcurveto{\pgfqpoint{5.849843in}{1.557787in}}{\pgfqpoint{5.780325in}{1.507279in}}%
\pgfusepath{stroke}%
\end{pgfscope}%
\begin{pgfscope}%
\pgfsetfillopacity{0.800000}%
\pgfsetstrokeopacity{0.800000}%
\definecolor{textcolor}{rgb}{0.000000,0.000000,1.000000}%
\pgfsetstrokecolor{textcolor}%
\pgfsetfillcolor{textcolor}%
\pgftext[x=5.846155in,y=1.673342in,left,base]{\color{textcolor}\sffamily\fontsize{5.647059}{6.776471}\selectfont 3.5362(15)}%
\end{pgfscope}%
\begin{pgfscope}%
\pgfsetbuttcap%
\pgfsetroundjoin%
\definecolor{currentfill}{rgb}{0.150000,0.150000,0.150000}%
\pgfsetfillcolor{currentfill}%
\pgfsetlinewidth{1.003750pt}%
\definecolor{currentstroke}{rgb}{0.150000,0.150000,0.150000}%
\pgfsetstrokecolor{currentstroke}%
\pgfsetdash{}{0pt}%
\pgfsys@defobject{currentmarker}{\pgfqpoint{0.000000in}{-0.066667in}}{\pgfqpoint{0.000000in}{0.000000in}}{%
\pgfpathmoveto{\pgfqpoint{0.000000in}{0.000000in}}%
\pgfpathlineto{\pgfqpoint{0.000000in}{-0.066667in}}%
\pgfusepath{stroke,fill}%
}%
\begin{pgfscope}%
\pgfsys@transformshift{5.105513in}{1.164970in}%
\pgfsys@useobject{currentmarker}{}%
\end{pgfscope}%
\end{pgfscope}%
\begin{pgfscope}%
\pgfsetbuttcap%
\pgfsetroundjoin%
\definecolor{currentfill}{rgb}{0.150000,0.150000,0.150000}%
\pgfsetfillcolor{currentfill}%
\pgfsetlinewidth{1.003750pt}%
\definecolor{currentstroke}{rgb}{0.150000,0.150000,0.150000}%
\pgfsetstrokecolor{currentstroke}%
\pgfsetdash{}{0pt}%
\pgfsys@defobject{currentmarker}{\pgfqpoint{0.000000in}{-0.066667in}}{\pgfqpoint{0.000000in}{0.000000in}}{%
\pgfpathmoveto{\pgfqpoint{0.000000in}{0.000000in}}%
\pgfpathlineto{\pgfqpoint{0.000000in}{-0.066667in}}%
\pgfusepath{stroke,fill}%
}%
\begin{pgfscope}%
\pgfsys@transformshift{5.606922in}{1.164970in}%
\pgfsys@useobject{currentmarker}{}%
\end{pgfscope}%
\end{pgfscope}%
\begin{pgfscope}%
\pgfsetbuttcap%
\pgfsetroundjoin%
\definecolor{currentfill}{rgb}{0.150000,0.150000,0.150000}%
\pgfsetfillcolor{currentfill}%
\pgfsetlinewidth{1.003750pt}%
\definecolor{currentstroke}{rgb}{0.150000,0.150000,0.150000}%
\pgfsetstrokecolor{currentstroke}%
\pgfsetdash{}{0pt}%
\pgfsys@defobject{currentmarker}{\pgfqpoint{0.000000in}{-0.066667in}}{\pgfqpoint{0.000000in}{0.000000in}}{%
\pgfpathmoveto{\pgfqpoint{0.000000in}{0.000000in}}%
\pgfpathlineto{\pgfqpoint{0.000000in}{-0.066667in}}%
\pgfusepath{stroke,fill}%
}%
\begin{pgfscope}%
\pgfsys@transformshift{6.108331in}{1.164970in}%
\pgfsys@useobject{currentmarker}{}%
\end{pgfscope}%
\end{pgfscope}%
\begin{pgfscope}%
\pgfsetbuttcap%
\pgfsetroundjoin%
\definecolor{currentfill}{rgb}{0.150000,0.150000,0.150000}%
\pgfsetfillcolor{currentfill}%
\pgfsetlinewidth{0.803000pt}%
\definecolor{currentstroke}{rgb}{0.150000,0.150000,0.150000}%
\pgfsetstrokecolor{currentstroke}%
\pgfsetdash{}{0pt}%
\pgfsys@defobject{currentmarker}{\pgfqpoint{0.000000in}{-0.044444in}}{\pgfqpoint{0.000000in}{0.000000in}}{%
\pgfpathmoveto{\pgfqpoint{0.000000in}{0.000000in}}%
\pgfpathlineto{\pgfqpoint{0.000000in}{-0.044444in}}%
\pgfusepath{stroke,fill}%
}%
\begin{pgfscope}%
\pgfsys@transformshift{5.256452in}{1.164970in}%
\pgfsys@useobject{currentmarker}{}%
\end{pgfscope}%
\end{pgfscope}%
\begin{pgfscope}%
\pgfsetbuttcap%
\pgfsetroundjoin%
\definecolor{currentfill}{rgb}{0.150000,0.150000,0.150000}%
\pgfsetfillcolor{currentfill}%
\pgfsetlinewidth{0.803000pt}%
\definecolor{currentstroke}{rgb}{0.150000,0.150000,0.150000}%
\pgfsetstrokecolor{currentstroke}%
\pgfsetdash{}{0pt}%
\pgfsys@defobject{currentmarker}{\pgfqpoint{0.000000in}{-0.044444in}}{\pgfqpoint{0.000000in}{0.000000in}}{%
\pgfpathmoveto{\pgfqpoint{0.000000in}{0.000000in}}%
\pgfpathlineto{\pgfqpoint{0.000000in}{-0.044444in}}%
\pgfusepath{stroke,fill}%
}%
\begin{pgfscope}%
\pgfsys@transformshift{5.344746in}{1.164970in}%
\pgfsys@useobject{currentmarker}{}%
\end{pgfscope}%
\end{pgfscope}%
\begin{pgfscope}%
\pgfsetbuttcap%
\pgfsetroundjoin%
\definecolor{currentfill}{rgb}{0.150000,0.150000,0.150000}%
\pgfsetfillcolor{currentfill}%
\pgfsetlinewidth{0.803000pt}%
\definecolor{currentstroke}{rgb}{0.150000,0.150000,0.150000}%
\pgfsetstrokecolor{currentstroke}%
\pgfsetdash{}{0pt}%
\pgfsys@defobject{currentmarker}{\pgfqpoint{0.000000in}{-0.044444in}}{\pgfqpoint{0.000000in}{0.000000in}}{%
\pgfpathmoveto{\pgfqpoint{0.000000in}{0.000000in}}%
\pgfpathlineto{\pgfqpoint{0.000000in}{-0.044444in}}%
\pgfusepath{stroke,fill}%
}%
\begin{pgfscope}%
\pgfsys@transformshift{5.407391in}{1.164970in}%
\pgfsys@useobject{currentmarker}{}%
\end{pgfscope}%
\end{pgfscope}%
\begin{pgfscope}%
\pgfsetbuttcap%
\pgfsetroundjoin%
\definecolor{currentfill}{rgb}{0.150000,0.150000,0.150000}%
\pgfsetfillcolor{currentfill}%
\pgfsetlinewidth{0.803000pt}%
\definecolor{currentstroke}{rgb}{0.150000,0.150000,0.150000}%
\pgfsetstrokecolor{currentstroke}%
\pgfsetdash{}{0pt}%
\pgfsys@defobject{currentmarker}{\pgfqpoint{0.000000in}{-0.044444in}}{\pgfqpoint{0.000000in}{0.000000in}}{%
\pgfpathmoveto{\pgfqpoint{0.000000in}{0.000000in}}%
\pgfpathlineto{\pgfqpoint{0.000000in}{-0.044444in}}%
\pgfusepath{stroke,fill}%
}%
\begin{pgfscope}%
\pgfsys@transformshift{5.455982in}{1.164970in}%
\pgfsys@useobject{currentmarker}{}%
\end{pgfscope}%
\end{pgfscope}%
\begin{pgfscope}%
\pgfsetbuttcap%
\pgfsetroundjoin%
\definecolor{currentfill}{rgb}{0.150000,0.150000,0.150000}%
\pgfsetfillcolor{currentfill}%
\pgfsetlinewidth{0.803000pt}%
\definecolor{currentstroke}{rgb}{0.150000,0.150000,0.150000}%
\pgfsetstrokecolor{currentstroke}%
\pgfsetdash{}{0pt}%
\pgfsys@defobject{currentmarker}{\pgfqpoint{0.000000in}{-0.044444in}}{\pgfqpoint{0.000000in}{0.000000in}}{%
\pgfpathmoveto{\pgfqpoint{0.000000in}{0.000000in}}%
\pgfpathlineto{\pgfqpoint{0.000000in}{-0.044444in}}%
\pgfusepath{stroke,fill}%
}%
\begin{pgfscope}%
\pgfsys@transformshift{5.495685in}{1.164970in}%
\pgfsys@useobject{currentmarker}{}%
\end{pgfscope}%
\end{pgfscope}%
\begin{pgfscope}%
\pgfsetbuttcap%
\pgfsetroundjoin%
\definecolor{currentfill}{rgb}{0.150000,0.150000,0.150000}%
\pgfsetfillcolor{currentfill}%
\pgfsetlinewidth{0.803000pt}%
\definecolor{currentstroke}{rgb}{0.150000,0.150000,0.150000}%
\pgfsetstrokecolor{currentstroke}%
\pgfsetdash{}{0pt}%
\pgfsys@defobject{currentmarker}{\pgfqpoint{0.000000in}{-0.044444in}}{\pgfqpoint{0.000000in}{0.000000in}}{%
\pgfpathmoveto{\pgfqpoint{0.000000in}{0.000000in}}%
\pgfpathlineto{\pgfqpoint{0.000000in}{-0.044444in}}%
\pgfusepath{stroke,fill}%
}%
\begin{pgfscope}%
\pgfsys@transformshift{5.529252in}{1.164970in}%
\pgfsys@useobject{currentmarker}{}%
\end{pgfscope}%
\end{pgfscope}%
\begin{pgfscope}%
\pgfsetbuttcap%
\pgfsetroundjoin%
\definecolor{currentfill}{rgb}{0.150000,0.150000,0.150000}%
\pgfsetfillcolor{currentfill}%
\pgfsetlinewidth{0.803000pt}%
\definecolor{currentstroke}{rgb}{0.150000,0.150000,0.150000}%
\pgfsetstrokecolor{currentstroke}%
\pgfsetdash{}{0pt}%
\pgfsys@defobject{currentmarker}{\pgfqpoint{0.000000in}{-0.044444in}}{\pgfqpoint{0.000000in}{0.000000in}}{%
\pgfpathmoveto{\pgfqpoint{0.000000in}{0.000000in}}%
\pgfpathlineto{\pgfqpoint{0.000000in}{-0.044444in}}%
\pgfusepath{stroke,fill}%
}%
\begin{pgfscope}%
\pgfsys@transformshift{5.558330in}{1.164970in}%
\pgfsys@useobject{currentmarker}{}%
\end{pgfscope}%
\end{pgfscope}%
\begin{pgfscope}%
\pgfsetbuttcap%
\pgfsetroundjoin%
\definecolor{currentfill}{rgb}{0.150000,0.150000,0.150000}%
\pgfsetfillcolor{currentfill}%
\pgfsetlinewidth{0.803000pt}%
\definecolor{currentstroke}{rgb}{0.150000,0.150000,0.150000}%
\pgfsetstrokecolor{currentstroke}%
\pgfsetdash{}{0pt}%
\pgfsys@defobject{currentmarker}{\pgfqpoint{0.000000in}{-0.044444in}}{\pgfqpoint{0.000000in}{0.000000in}}{%
\pgfpathmoveto{\pgfqpoint{0.000000in}{0.000000in}}%
\pgfpathlineto{\pgfqpoint{0.000000in}{-0.044444in}}%
\pgfusepath{stroke,fill}%
}%
\begin{pgfscope}%
\pgfsys@transformshift{5.583978in}{1.164970in}%
\pgfsys@useobject{currentmarker}{}%
\end{pgfscope}%
\end{pgfscope}%
\begin{pgfscope}%
\pgfsetbuttcap%
\pgfsetroundjoin%
\definecolor{currentfill}{rgb}{0.150000,0.150000,0.150000}%
\pgfsetfillcolor{currentfill}%
\pgfsetlinewidth{0.803000pt}%
\definecolor{currentstroke}{rgb}{0.150000,0.150000,0.150000}%
\pgfsetstrokecolor{currentstroke}%
\pgfsetdash{}{0pt}%
\pgfsys@defobject{currentmarker}{\pgfqpoint{0.000000in}{-0.044444in}}{\pgfqpoint{0.000000in}{0.000000in}}{%
\pgfpathmoveto{\pgfqpoint{0.000000in}{0.000000in}}%
\pgfpathlineto{\pgfqpoint{0.000000in}{-0.044444in}}%
\pgfusepath{stroke,fill}%
}%
\begin{pgfscope}%
\pgfsys@transformshift{5.757861in}{1.164970in}%
\pgfsys@useobject{currentmarker}{}%
\end{pgfscope}%
\end{pgfscope}%
\begin{pgfscope}%
\pgfsetbuttcap%
\pgfsetroundjoin%
\definecolor{currentfill}{rgb}{0.150000,0.150000,0.150000}%
\pgfsetfillcolor{currentfill}%
\pgfsetlinewidth{0.803000pt}%
\definecolor{currentstroke}{rgb}{0.150000,0.150000,0.150000}%
\pgfsetstrokecolor{currentstroke}%
\pgfsetdash{}{0pt}%
\pgfsys@defobject{currentmarker}{\pgfqpoint{0.000000in}{-0.044444in}}{\pgfqpoint{0.000000in}{0.000000in}}{%
\pgfpathmoveto{\pgfqpoint{0.000000in}{0.000000in}}%
\pgfpathlineto{\pgfqpoint{0.000000in}{-0.044444in}}%
\pgfusepath{stroke,fill}%
}%
\begin{pgfscope}%
\pgfsys@transformshift{5.846155in}{1.164970in}%
\pgfsys@useobject{currentmarker}{}%
\end{pgfscope}%
\end{pgfscope}%
\begin{pgfscope}%
\pgfsetbuttcap%
\pgfsetroundjoin%
\definecolor{currentfill}{rgb}{0.150000,0.150000,0.150000}%
\pgfsetfillcolor{currentfill}%
\pgfsetlinewidth{0.803000pt}%
\definecolor{currentstroke}{rgb}{0.150000,0.150000,0.150000}%
\pgfsetstrokecolor{currentstroke}%
\pgfsetdash{}{0pt}%
\pgfsys@defobject{currentmarker}{\pgfqpoint{0.000000in}{-0.044444in}}{\pgfqpoint{0.000000in}{0.000000in}}{%
\pgfpathmoveto{\pgfqpoint{0.000000in}{0.000000in}}%
\pgfpathlineto{\pgfqpoint{0.000000in}{-0.044444in}}%
\pgfusepath{stroke,fill}%
}%
\begin{pgfscope}%
\pgfsys@transformshift{5.908800in}{1.164970in}%
\pgfsys@useobject{currentmarker}{}%
\end{pgfscope}%
\end{pgfscope}%
\begin{pgfscope}%
\pgfsetbuttcap%
\pgfsetroundjoin%
\definecolor{currentfill}{rgb}{0.150000,0.150000,0.150000}%
\pgfsetfillcolor{currentfill}%
\pgfsetlinewidth{0.803000pt}%
\definecolor{currentstroke}{rgb}{0.150000,0.150000,0.150000}%
\pgfsetstrokecolor{currentstroke}%
\pgfsetdash{}{0pt}%
\pgfsys@defobject{currentmarker}{\pgfqpoint{0.000000in}{-0.044444in}}{\pgfqpoint{0.000000in}{0.000000in}}{%
\pgfpathmoveto{\pgfqpoint{0.000000in}{0.000000in}}%
\pgfpathlineto{\pgfqpoint{0.000000in}{-0.044444in}}%
\pgfusepath{stroke,fill}%
}%
\begin{pgfscope}%
\pgfsys@transformshift{5.957391in}{1.164970in}%
\pgfsys@useobject{currentmarker}{}%
\end{pgfscope}%
\end{pgfscope}%
\begin{pgfscope}%
\pgfsetbuttcap%
\pgfsetroundjoin%
\definecolor{currentfill}{rgb}{0.150000,0.150000,0.150000}%
\pgfsetfillcolor{currentfill}%
\pgfsetlinewidth{0.803000pt}%
\definecolor{currentstroke}{rgb}{0.150000,0.150000,0.150000}%
\pgfsetstrokecolor{currentstroke}%
\pgfsetdash{}{0pt}%
\pgfsys@defobject{currentmarker}{\pgfqpoint{0.000000in}{-0.044444in}}{\pgfqpoint{0.000000in}{0.000000in}}{%
\pgfpathmoveto{\pgfqpoint{0.000000in}{0.000000in}}%
\pgfpathlineto{\pgfqpoint{0.000000in}{-0.044444in}}%
\pgfusepath{stroke,fill}%
}%
\begin{pgfscope}%
\pgfsys@transformshift{5.997094in}{1.164970in}%
\pgfsys@useobject{currentmarker}{}%
\end{pgfscope}%
\end{pgfscope}%
\begin{pgfscope}%
\pgfsetbuttcap%
\pgfsetroundjoin%
\definecolor{currentfill}{rgb}{0.150000,0.150000,0.150000}%
\pgfsetfillcolor{currentfill}%
\pgfsetlinewidth{0.803000pt}%
\definecolor{currentstroke}{rgb}{0.150000,0.150000,0.150000}%
\pgfsetstrokecolor{currentstroke}%
\pgfsetdash{}{0pt}%
\pgfsys@defobject{currentmarker}{\pgfqpoint{0.000000in}{-0.044444in}}{\pgfqpoint{0.000000in}{0.000000in}}{%
\pgfpathmoveto{\pgfqpoint{0.000000in}{0.000000in}}%
\pgfpathlineto{\pgfqpoint{0.000000in}{-0.044444in}}%
\pgfusepath{stroke,fill}%
}%
\begin{pgfscope}%
\pgfsys@transformshift{6.030661in}{1.164970in}%
\pgfsys@useobject{currentmarker}{}%
\end{pgfscope}%
\end{pgfscope}%
\begin{pgfscope}%
\pgfsetbuttcap%
\pgfsetroundjoin%
\definecolor{currentfill}{rgb}{0.150000,0.150000,0.150000}%
\pgfsetfillcolor{currentfill}%
\pgfsetlinewidth{0.803000pt}%
\definecolor{currentstroke}{rgb}{0.150000,0.150000,0.150000}%
\pgfsetstrokecolor{currentstroke}%
\pgfsetdash{}{0pt}%
\pgfsys@defobject{currentmarker}{\pgfqpoint{0.000000in}{-0.044444in}}{\pgfqpoint{0.000000in}{0.000000in}}{%
\pgfpathmoveto{\pgfqpoint{0.000000in}{0.000000in}}%
\pgfpathlineto{\pgfqpoint{0.000000in}{-0.044444in}}%
\pgfusepath{stroke,fill}%
}%
\begin{pgfscope}%
\pgfsys@transformshift{6.059739in}{1.164970in}%
\pgfsys@useobject{currentmarker}{}%
\end{pgfscope}%
\end{pgfscope}%
\begin{pgfscope}%
\pgfsetbuttcap%
\pgfsetroundjoin%
\definecolor{currentfill}{rgb}{0.150000,0.150000,0.150000}%
\pgfsetfillcolor{currentfill}%
\pgfsetlinewidth{0.803000pt}%
\definecolor{currentstroke}{rgb}{0.150000,0.150000,0.150000}%
\pgfsetstrokecolor{currentstroke}%
\pgfsetdash{}{0pt}%
\pgfsys@defobject{currentmarker}{\pgfqpoint{0.000000in}{-0.044444in}}{\pgfqpoint{0.000000in}{0.000000in}}{%
\pgfpathmoveto{\pgfqpoint{0.000000in}{0.000000in}}%
\pgfpathlineto{\pgfqpoint{0.000000in}{-0.044444in}}%
\pgfusepath{stroke,fill}%
}%
\begin{pgfscope}%
\pgfsys@transformshift{6.085387in}{1.164970in}%
\pgfsys@useobject{currentmarker}{}%
\end{pgfscope}%
\end{pgfscope}%
\begin{pgfscope}%
\pgfsetbuttcap%
\pgfsetroundjoin%
\definecolor{currentfill}{rgb}{0.150000,0.150000,0.150000}%
\pgfsetfillcolor{currentfill}%
\pgfsetlinewidth{0.803000pt}%
\definecolor{currentstroke}{rgb}{0.150000,0.150000,0.150000}%
\pgfsetstrokecolor{currentstroke}%
\pgfsetdash{}{0pt}%
\pgfsys@defobject{currentmarker}{\pgfqpoint{0.000000in}{-0.044444in}}{\pgfqpoint{0.000000in}{0.000000in}}{%
\pgfpathmoveto{\pgfqpoint{0.000000in}{0.000000in}}%
\pgfpathlineto{\pgfqpoint{0.000000in}{-0.044444in}}%
\pgfusepath{stroke,fill}%
}%
\begin{pgfscope}%
\pgfsys@transformshift{6.259270in}{1.164970in}%
\pgfsys@useobject{currentmarker}{}%
\end{pgfscope}%
\end{pgfscope}%
\begin{pgfscope}%
\pgfsetbuttcap%
\pgfsetroundjoin%
\definecolor{currentfill}{rgb}{0.150000,0.150000,0.150000}%
\pgfsetfillcolor{currentfill}%
\pgfsetlinewidth{1.003750pt}%
\definecolor{currentstroke}{rgb}{0.150000,0.150000,0.150000}%
\pgfsetstrokecolor{currentstroke}%
\pgfsetdash{}{0pt}%
\pgfsys@defobject{currentmarker}{\pgfqpoint{-0.066667in}{0.000000in}}{\pgfqpoint{0.000000in}{0.000000in}}{%
\pgfpathmoveto{\pgfqpoint{0.000000in}{0.000000in}}%
\pgfpathlineto{\pgfqpoint{-0.066667in}{0.000000in}}%
\pgfusepath{stroke,fill}%
}%
\begin{pgfscope}%
\pgfsys@transformshift{5.105513in}{1.164970in}%
\pgfsys@useobject{currentmarker}{}%
\end{pgfscope}%
\end{pgfscope}%
\begin{pgfscope}%
\pgfsetbuttcap%
\pgfsetroundjoin%
\definecolor{currentfill}{rgb}{0.150000,0.150000,0.150000}%
\pgfsetfillcolor{currentfill}%
\pgfsetlinewidth{1.003750pt}%
\definecolor{currentstroke}{rgb}{0.150000,0.150000,0.150000}%
\pgfsetstrokecolor{currentstroke}%
\pgfsetdash{}{0pt}%
\pgfsys@defobject{currentmarker}{\pgfqpoint{-0.066667in}{0.000000in}}{\pgfqpoint{0.000000in}{0.000000in}}{%
\pgfpathmoveto{\pgfqpoint{0.000000in}{0.000000in}}%
\pgfpathlineto{\pgfqpoint{-0.066667in}{0.000000in}}%
\pgfusepath{stroke,fill}%
}%
\begin{pgfscope}%
\pgfsys@transformshift{5.105513in}{1.491207in}%
\pgfsys@useobject{currentmarker}{}%
\end{pgfscope}%
\end{pgfscope}%
\begin{pgfscope}%
\pgfsetbuttcap%
\pgfsetroundjoin%
\definecolor{currentfill}{rgb}{0.150000,0.150000,0.150000}%
\pgfsetfillcolor{currentfill}%
\pgfsetlinewidth{1.003750pt}%
\definecolor{currentstroke}{rgb}{0.150000,0.150000,0.150000}%
\pgfsetstrokecolor{currentstroke}%
\pgfsetdash{}{0pt}%
\pgfsys@defobject{currentmarker}{\pgfqpoint{-0.066667in}{0.000000in}}{\pgfqpoint{0.000000in}{0.000000in}}{%
\pgfpathmoveto{\pgfqpoint{0.000000in}{0.000000in}}%
\pgfpathlineto{\pgfqpoint{-0.066667in}{0.000000in}}%
\pgfusepath{stroke,fill}%
}%
\begin{pgfscope}%
\pgfsys@transformshift{5.105513in}{1.772918in}%
\pgfsys@useobject{currentmarker}{}%
\end{pgfscope}%
\end{pgfscope}%
\begin{pgfscope}%
\pgfpathrectangle{\pgfqpoint{5.105513in}{1.164970in}}{\pgfqpoint{1.223103in}{0.607948in}}%
\pgfusepath{clip}%
\pgfsetroundcap%
\pgfsetroundjoin%
\pgfsetlinewidth{1.204500pt}%
\definecolor{currentstroke}{rgb}{0.000000,0.501961,0.000000}%
\pgfsetstrokecolor{currentstroke}%
\pgfsetdash{}{0pt}%
\pgfpathmoveto{\pgfqpoint{5.105513in}{1.491544in}}%
\pgfpathlineto{\pgfqpoint{5.346222in}{1.492716in}}%
\pgfpathlineto{\pgfqpoint{5.457753in}{1.493873in}}%
\pgfpathlineto{\pgfqpoint{5.531149in}{1.495018in}}%
\pgfpathlineto{\pgfqpoint{5.585945in}{1.496153in}}%
\pgfpathlineto{\pgfqpoint{5.629687in}{1.497280in}}%
\pgfpathlineto{\pgfqpoint{5.666095in}{1.498402in}}%
\pgfpathlineto{\pgfqpoint{5.697279in}{1.499518in}}%
\pgfpathlineto{\pgfqpoint{5.724552in}{1.500632in}}%
\pgfpathlineto{\pgfqpoint{5.748786in}{1.501743in}}%
\pgfpathlineto{\pgfqpoint{5.770591in}{1.502854in}}%
\pgfpathlineto{\pgfqpoint{5.790410in}{1.503964in}}%
\pgfpathlineto{\pgfqpoint{5.808575in}{1.505074in}}%
\pgfpathlineto{\pgfqpoint{5.825341in}{1.506186in}}%
\pgfpathlineto{\pgfqpoint{5.840907in}{1.507298in}}%
\pgfpathlineto{\pgfqpoint{5.855435in}{1.508412in}}%
\pgfpathlineto{\pgfqpoint{5.869053in}{1.509528in}}%
\pgfpathlineto{\pgfqpoint{5.881870in}{1.510646in}}%
\pgfpathlineto{\pgfqpoint{5.893974in}{1.511765in}}%
\pgfpathlineto{\pgfqpoint{5.905441in}{1.512885in}}%
\pgfpathlineto{\pgfqpoint{5.916334in}{1.514008in}}%
\pgfpathlineto{\pgfqpoint{5.926708in}{1.515131in}}%
\pgfpathlineto{\pgfqpoint{5.936610in}{1.516255in}}%
\pgfpathlineto{\pgfqpoint{5.946082in}{1.517380in}}%
\pgfpathlineto{\pgfqpoint{5.955158in}{1.518506in}}%
\pgfpathlineto{\pgfqpoint{5.963871in}{1.519632in}}%
\pgfpathlineto{\pgfqpoint{5.972249in}{1.520758in}}%
\pgfpathlineto{\pgfqpoint{5.980317in}{1.521884in}}%
\pgfpathlineto{\pgfqpoint{5.988096in}{1.523010in}}%
\pgfpathlineto{\pgfqpoint{5.995607in}{1.524136in}}%
\pgfpathlineto{\pgfqpoint{6.002868in}{1.525262in}}%
\pgfpathlineto{\pgfqpoint{6.009894in}{1.526387in}}%
\pgfpathlineto{\pgfqpoint{6.016701in}{1.527513in}}%
\pgfpathlineto{\pgfqpoint{6.023301in}{1.528640in}}%
\pgfpathlineto{\pgfqpoint{6.029707in}{1.529767in}}%
\pgfpathlineto{\pgfqpoint{6.035930in}{1.530896in}}%
\pgfpathlineto{\pgfqpoint{6.041980in}{1.532028in}}%
\pgfpathlineto{\pgfqpoint{6.047867in}{1.533162in}}%
\pgfpathlineto{\pgfqpoint{6.053598in}{1.534301in}}%
\pgfpathlineto{\pgfqpoint{6.059183in}{1.535446in}}%
\pgfpathlineto{\pgfqpoint{6.064628in}{1.536598in}}%
\pgfpathlineto{\pgfqpoint{6.069940in}{1.537759in}}%
\pgfpathlineto{\pgfqpoint{6.075126in}{1.538931in}}%
\pgfpathlineto{\pgfqpoint{6.080191in}{1.540116in}}%
\pgfpathlineto{\pgfqpoint{6.085140in}{1.541317in}}%
\pgfpathlineto{\pgfqpoint{6.089980in}{1.542536in}}%
\pgfpathlineto{\pgfqpoint{6.094715in}{1.543776in}}%
\pgfpathlineto{\pgfqpoint{6.099349in}{1.545041in}}%
\pgfpathlineto{\pgfqpoint{6.103886in}{1.546334in}}%
\pgfpathlineto{\pgfqpoint{6.108331in}{1.547659in}}%
\pgfusepath{stroke}%
\end{pgfscope}%
\begin{pgfscope}%
\pgfsetrectcap%
\pgfsetmiterjoin%
\pgfsetlinewidth{1.003750pt}%
\definecolor{currentstroke}{rgb}{0.150000,0.150000,0.150000}%
\pgfsetstrokecolor{currentstroke}%
\pgfsetdash{}{0pt}%
\pgfpathmoveto{\pgfqpoint{5.105513in}{1.164970in}}%
\pgfpathlineto{\pgfqpoint{5.105513in}{1.772918in}}%
\pgfusepath{stroke}%
\end{pgfscope}%
\begin{pgfscope}%
\pgfsetrectcap%
\pgfsetmiterjoin%
\pgfsetlinewidth{1.003750pt}%
\definecolor{currentstroke}{rgb}{0.150000,0.150000,0.150000}%
\pgfsetstrokecolor{currentstroke}%
\pgfsetdash{}{0pt}%
\pgfpathmoveto{\pgfqpoint{5.105513in}{1.164970in}}%
\pgfpathlineto{\pgfqpoint{6.328616in}{1.164970in}}%
\pgfusepath{stroke}%
\end{pgfscope}%
\begin{pgfscope}%
\pgfpathrectangle{\pgfqpoint{5.105513in}{1.164970in}}{\pgfqpoint{1.223103in}{0.607948in}}%
\pgfusepath{clip}%
\pgfsetbuttcap%
\pgfsetroundjoin%
\definecolor{currentfill}{rgb}{0.000000,0.000000,0.000000}%
\pgfsetfillcolor{currentfill}%
\pgfsetlinewidth{1.003750pt}%
\definecolor{currentstroke}{rgb}{0.000000,0.000000,0.000000}%
\pgfsetstrokecolor{currentstroke}%
\pgfsetdash{}{0pt}%
\pgfsys@defobject{currentmarker}{\pgfqpoint{-0.013889in}{-0.013889in}}{\pgfqpoint{0.013889in}{0.013889in}}{%
\pgfpathmoveto{\pgfqpoint{0.000000in}{-0.013889in}}%
\pgfpathcurveto{\pgfqpoint{0.003683in}{-0.013889in}}{\pgfqpoint{0.007216in}{-0.012425in}}{\pgfqpoint{0.009821in}{-0.009821in}}%
\pgfpathcurveto{\pgfqpoint{0.012425in}{-0.007216in}}{\pgfqpoint{0.013889in}{-0.003683in}}{\pgfqpoint{0.013889in}{0.000000in}}%
\pgfpathcurveto{\pgfqpoint{0.013889in}{0.003683in}}{\pgfqpoint{0.012425in}{0.007216in}}{\pgfqpoint{0.009821in}{0.009821in}}%
\pgfpathcurveto{\pgfqpoint{0.007216in}{0.012425in}}{\pgfqpoint{0.003683in}{0.013889in}}{\pgfqpoint{0.000000in}{0.013889in}}%
\pgfpathcurveto{\pgfqpoint{-0.003683in}{0.013889in}}{\pgfqpoint{-0.007216in}{0.012425in}}{\pgfqpoint{-0.009821in}{0.009821in}}%
\pgfpathcurveto{\pgfqpoint{-0.012425in}{0.007216in}}{\pgfqpoint{-0.013889in}{0.003683in}}{\pgfqpoint{-0.013889in}{0.000000in}}%
\pgfpathcurveto{\pgfqpoint{-0.013889in}{-0.003683in}}{\pgfqpoint{-0.012425in}{-0.007216in}}{\pgfqpoint{-0.009821in}{-0.009821in}}%
\pgfpathcurveto{\pgfqpoint{-0.007216in}{-0.012425in}}{\pgfqpoint{-0.003683in}{-0.013889in}}{\pgfqpoint{0.000000in}{-0.013889in}}%
\pgfpathclose%
\pgfusepath{stroke,fill}%
}%
\begin{pgfscope}%
\pgfsys@transformshift{5.757861in}{1.502190in}%
\pgfsys@useobject{currentmarker}{}%
\end{pgfscope}%
\begin{pgfscope}%
\pgfsys@transformshift{5.762260in}{1.502408in}%
\pgfsys@useobject{currentmarker}{}%
\end{pgfscope}%
\begin{pgfscope}%
\pgfsys@transformshift{5.766750in}{1.502649in}%
\pgfsys@useobject{currentmarker}{}%
\end{pgfscope}%
\begin{pgfscope}%
\pgfsys@transformshift{5.771335in}{1.502886in}%
\pgfsys@useobject{currentmarker}{}%
\end{pgfscope}%
\begin{pgfscope}%
\pgfsys@transformshift{5.776018in}{1.503148in}%
\pgfsys@useobject{currentmarker}{}%
\end{pgfscope}%
\begin{pgfscope}%
\pgfsys@transformshift{5.780804in}{1.503406in}%
\pgfsys@useobject{currentmarker}{}%
\end{pgfscope}%
\begin{pgfscope}%
\pgfsys@transformshift{5.785698in}{1.503693in}%
\pgfsys@useobject{currentmarker}{}%
\end{pgfscope}%
\begin{pgfscope}%
\pgfsys@transformshift{5.790704in}{1.503975in}%
\pgfsys@useobject{currentmarker}{}%
\end{pgfscope}%
\begin{pgfscope}%
\pgfsys@transformshift{5.795828in}{1.504290in}%
\pgfsys@useobject{currentmarker}{}%
\end{pgfscope}%
\begin{pgfscope}%
\pgfsys@transformshift{5.801075in}{1.504599in}%
\pgfsys@useobject{currentmarker}{}%
\end{pgfscope}%
\begin{pgfscope}%
\pgfsys@transformshift{5.806452in}{1.504948in}%
\pgfsys@useobject{currentmarker}{}%
\end{pgfscope}%
\begin{pgfscope}%
\pgfsys@transformshift{5.835530in}{1.506903in}%
\pgfsys@useobject{currentmarker}{}%
\end{pgfscope}%
\begin{pgfscope}%
\pgfsys@transformshift{5.869098in}{1.509560in}%
\pgfsys@useobject{currentmarker}{}%
\end{pgfscope}%
\begin{pgfscope}%
\pgfsys@transformshift{5.908800in}{1.513214in}%
\pgfsys@useobject{currentmarker}{}%
\end{pgfscope}%
\begin{pgfscope}%
\pgfsys@transformshift{5.957391in}{1.518856in}%
\pgfsys@useobject{currentmarker}{}%
\end{pgfscope}%
\begin{pgfscope}%
\pgfsys@transformshift{6.020037in}{1.528033in}%
\pgfsys@useobject{currentmarker}{}%
\end{pgfscope}%
\begin{pgfscope}%
\pgfsys@transformshift{6.108331in}{1.547670in}%
\pgfsys@useobject{currentmarker}{}%
\end{pgfscope}%
\end{pgfscope}%
\begin{pgfscope}%
\pgfsetbuttcap%
\pgfsetmiterjoin%
\definecolor{currentfill}{rgb}{1.000000,1.000000,1.000000}%
\pgfsetfillcolor{currentfill}%
\pgfsetlinewidth{0.803000pt}%
\definecolor{currentstroke}{rgb}{1.000000,1.000000,1.000000}%
\pgfsetstrokecolor{currentstroke}%
\pgfsetdash{}{0pt}%
\pgfpathmoveto{\pgfqpoint{6.297392in}{1.703219in}}%
\pgfpathlineto{\pgfqpoint{6.297392in}{1.234670in}}%
\pgfpathlineto{\pgfqpoint{6.478411in}{1.234670in}}%
\pgfpathlineto{\pgfqpoint{6.478411in}{1.703219in}}%
\pgfpathclose%
\pgfusepath{stroke,fill}%
\end{pgfscope}%
\begin{pgfscope}%
\definecolor{textcolor}{rgb}{0.150000,0.150000,0.150000}%
\pgfsetstrokecolor{textcolor}%
\pgfsetfillcolor{textcolor}%
\pgftext[x=6.368294in,y=1.647533in,left,base,rotate=270.000000]{\color{textcolor}\sffamily\fontsize{5.647059}{6.776471}\selectfont nlevel = 4}%
\end{pgfscope}%
\begin{pgfscope}%
\pgfsetbuttcap%
\pgfsetmiterjoin%
\definecolor{currentfill}{rgb}{1.000000,1.000000,1.000000}%
\pgfsetfillcolor{currentfill}%
\pgfsetlinewidth{0.803000pt}%
\definecolor{currentstroke}{rgb}{1.000000,1.000000,1.000000}%
\pgfsetstrokecolor{currentstroke}%
\pgfsetdash{}{0pt}%
\pgfpathmoveto{\pgfqpoint{6.297392in}{1.703219in}}%
\pgfpathlineto{\pgfqpoint{6.297392in}{1.234670in}}%
\pgfpathlineto{\pgfqpoint{6.478411in}{1.234670in}}%
\pgfpathlineto{\pgfqpoint{6.478411in}{1.703219in}}%
\pgfpathclose%
\pgfusepath{stroke,fill}%
\end{pgfscope}%
\begin{pgfscope}%
\definecolor{textcolor}{rgb}{0.150000,0.150000,0.150000}%
\pgfsetstrokecolor{textcolor}%
\pgfsetfillcolor{textcolor}%
\pgftext[x=6.368294in,y=1.647533in,left,base,rotate=270.000000]{\color{textcolor}\sffamily\fontsize{5.647059}{6.776471}\selectfont nlevel = 4}%
\end{pgfscope}%
\begin{pgfscope}%
\pgfsetbuttcap%
\pgfsetmiterjoin%
\definecolor{currentfill}{rgb}{1.000000,1.000000,1.000000}%
\pgfsetfillcolor{currentfill}%
\pgfsetlinewidth{0.000000pt}%
\definecolor{currentstroke}{rgb}{0.000000,0.000000,0.000000}%
\pgfsetstrokecolor{currentstroke}%
\pgfsetstrokeopacity{0.000000}%
\pgfsetdash{}{0pt}%
\pgfpathmoveto{\pgfqpoint{0.702340in}{0.435433in}}%
\pgfpathlineto{\pgfqpoint{1.925444in}{0.435433in}}%
\pgfpathlineto{\pgfqpoint{1.925444in}{1.043381in}}%
\pgfpathlineto{\pgfqpoint{0.702340in}{1.043381in}}%
\pgfpathclose%
\pgfusepath{fill}%
\end{pgfscope}%
\begin{pgfscope}%
\pgfpathrectangle{\pgfqpoint{0.702340in}{0.435433in}}{\pgfqpoint{1.223103in}{0.607948in}}%
\pgfusepath{clip}%
\pgfsetbuttcap%
\pgfsetmiterjoin%
\definecolor{currentfill}{rgb}{0.000000,0.000000,1.000000}%
\pgfsetfillcolor{currentfill}%
\pgfsetfillopacity{0.100000}%
\pgfsetlinewidth{0.803000pt}%
\definecolor{currentstroke}{rgb}{0.000000,0.000000,1.000000}%
\pgfsetstrokecolor{currentstroke}%
\pgfsetstrokeopacity{0.100000}%
\pgfsetdash{}{0pt}%
\pgfpathmoveto{\pgfqpoint{0.702340in}{0.703858in}}%
\pgfpathlineto{\pgfqpoint{0.702340in}{0.704026in}}%
\pgfpathlineto{\pgfqpoint{1.925444in}{0.704026in}}%
\pgfpathlineto{\pgfqpoint{1.925444in}{0.703858in}}%
\pgfpathclose%
\pgfusepath{stroke,fill}%
\end{pgfscope}%
\begin{pgfscope}%
\pgfpathrectangle{\pgfqpoint{0.702340in}{0.435433in}}{\pgfqpoint{1.223103in}{0.607948in}}%
\pgfusepath{clip}%
\pgfsetbuttcap%
\pgfsetroundjoin%
\definecolor{currentfill}{rgb}{0.000000,0.501961,0.000000}%
\pgfsetfillcolor{currentfill}%
\pgfsetfillopacity{0.500000}%
\pgfsetlinewidth{0.803000pt}%
\definecolor{currentstroke}{rgb}{0.000000,0.501961,0.000000}%
\pgfsetstrokecolor{currentstroke}%
\pgfsetstrokeopacity{0.500000}%
\pgfsetdash{}{0pt}%
\pgfpathmoveto{\pgfqpoint{0.702340in}{0.703799in}}%
\pgfpathlineto{\pgfqpoint{0.702340in}{0.703644in}}%
\pgfpathlineto{\pgfqpoint{0.943050in}{0.703196in}}%
\pgfpathlineto{\pgfqpoint{1.054581in}{0.702729in}}%
\pgfpathlineto{\pgfqpoint{1.127977in}{0.702244in}}%
\pgfpathlineto{\pgfqpoint{1.182772in}{0.701739in}}%
\pgfpathlineto{\pgfqpoint{1.226515in}{0.701216in}}%
\pgfpathlineto{\pgfqpoint{1.262923in}{0.700674in}}%
\pgfpathlineto{\pgfqpoint{1.294107in}{0.700114in}}%
\pgfpathlineto{\pgfqpoint{1.321380in}{0.699535in}}%
\pgfpathlineto{\pgfqpoint{1.345614in}{0.698938in}}%
\pgfpathlineto{\pgfqpoint{1.367419in}{0.698322in}}%
\pgfpathlineto{\pgfqpoint{1.387238in}{0.697687in}}%
\pgfpathlineto{\pgfqpoint{1.405403in}{0.697030in}}%
\pgfpathlineto{\pgfqpoint{1.422168in}{0.696356in}}%
\pgfpathlineto{\pgfqpoint{1.437735in}{0.695665in}}%
\pgfpathlineto{\pgfqpoint{1.452262in}{0.694957in}}%
\pgfpathlineto{\pgfqpoint{1.465881in}{0.694233in}}%
\pgfpathlineto{\pgfqpoint{1.478698in}{0.693492in}}%
\pgfpathlineto{\pgfqpoint{1.490802in}{0.692734in}}%
\pgfpathlineto{\pgfqpoint{1.502269in}{0.691959in}}%
\pgfpathlineto{\pgfqpoint{1.513162in}{0.691168in}}%
\pgfpathlineto{\pgfqpoint{1.523536in}{0.690360in}}%
\pgfpathlineto{\pgfqpoint{1.533438in}{0.689535in}}%
\pgfpathlineto{\pgfqpoint{1.542909in}{0.688694in}}%
\pgfpathlineto{\pgfqpoint{1.551986in}{0.687836in}}%
\pgfpathlineto{\pgfqpoint{1.560699in}{0.686961in}}%
\pgfpathlineto{\pgfqpoint{1.569077in}{0.686066in}}%
\pgfpathlineto{\pgfqpoint{1.577145in}{0.685154in}}%
\pgfpathlineto{\pgfqpoint{1.584924in}{0.684226in}}%
\pgfpathlineto{\pgfqpoint{1.592435in}{0.683281in}}%
\pgfpathlineto{\pgfqpoint{1.599696in}{0.682320in}}%
\pgfpathlineto{\pgfqpoint{1.606722in}{0.681343in}}%
\pgfpathlineto{\pgfqpoint{1.613528in}{0.680350in}}%
\pgfpathlineto{\pgfqpoint{1.620129in}{0.679340in}}%
\pgfpathlineto{\pgfqpoint{1.626535in}{0.678315in}}%
\pgfpathlineto{\pgfqpoint{1.632758in}{0.677274in}}%
\pgfpathlineto{\pgfqpoint{1.638808in}{0.676217in}}%
\pgfpathlineto{\pgfqpoint{1.644694in}{0.675145in}}%
\pgfpathlineto{\pgfqpoint{1.650426in}{0.674057in}}%
\pgfpathlineto{\pgfqpoint{1.656011in}{0.672954in}}%
\pgfpathlineto{\pgfqpoint{1.661456in}{0.671835in}}%
\pgfpathlineto{\pgfqpoint{1.666768in}{0.670701in}}%
\pgfpathlineto{\pgfqpoint{1.671953in}{0.669553in}}%
\pgfpathlineto{\pgfqpoint{1.677018in}{0.668389in}}%
\pgfpathlineto{\pgfqpoint{1.681968in}{0.667210in}}%
\pgfpathlineto{\pgfqpoint{1.686808in}{0.666017in}}%
\pgfpathlineto{\pgfqpoint{1.691543in}{0.664809in}}%
\pgfpathlineto{\pgfqpoint{1.696176in}{0.663587in}}%
\pgfpathlineto{\pgfqpoint{1.700714in}{0.662350in}}%
\pgfpathlineto{\pgfqpoint{1.705158in}{0.661093in}}%
\pgfpathlineto{\pgfqpoint{1.705158in}{0.661099in}}%
\pgfpathlineto{\pgfqpoint{1.705158in}{0.661099in}}%
\pgfpathlineto{\pgfqpoint{1.700714in}{0.662355in}}%
\pgfpathlineto{\pgfqpoint{1.696176in}{0.663602in}}%
\pgfpathlineto{\pgfqpoint{1.691543in}{0.664833in}}%
\pgfpathlineto{\pgfqpoint{1.686808in}{0.666048in}}%
\pgfpathlineto{\pgfqpoint{1.681968in}{0.667247in}}%
\pgfpathlineto{\pgfqpoint{1.677018in}{0.668430in}}%
\pgfpathlineto{\pgfqpoint{1.671953in}{0.669597in}}%
\pgfpathlineto{\pgfqpoint{1.666768in}{0.670748in}}%
\pgfpathlineto{\pgfqpoint{1.661456in}{0.671883in}}%
\pgfpathlineto{\pgfqpoint{1.656011in}{0.673002in}}%
\pgfpathlineto{\pgfqpoint{1.650426in}{0.674105in}}%
\pgfpathlineto{\pgfqpoint{1.644694in}{0.675192in}}%
\pgfpathlineto{\pgfqpoint{1.638808in}{0.676262in}}%
\pgfpathlineto{\pgfqpoint{1.632758in}{0.677316in}}%
\pgfpathlineto{\pgfqpoint{1.626535in}{0.678354in}}%
\pgfpathlineto{\pgfqpoint{1.620129in}{0.679376in}}%
\pgfpathlineto{\pgfqpoint{1.613528in}{0.680382in}}%
\pgfpathlineto{\pgfqpoint{1.606722in}{0.681371in}}%
\pgfpathlineto{\pgfqpoint{1.599696in}{0.682344in}}%
\pgfpathlineto{\pgfqpoint{1.592435in}{0.683300in}}%
\pgfpathlineto{\pgfqpoint{1.584924in}{0.684240in}}%
\pgfpathlineto{\pgfqpoint{1.577145in}{0.685164in}}%
\pgfpathlineto{\pgfqpoint{1.569077in}{0.686071in}}%
\pgfpathlineto{\pgfqpoint{1.560699in}{0.686962in}}%
\pgfpathlineto{\pgfqpoint{1.551986in}{0.687839in}}%
\pgfpathlineto{\pgfqpoint{1.542909in}{0.688701in}}%
\pgfpathlineto{\pgfqpoint{1.533438in}{0.689545in}}%
\pgfpathlineto{\pgfqpoint{1.523536in}{0.690373in}}%
\pgfpathlineto{\pgfqpoint{1.513162in}{0.691183in}}%
\pgfpathlineto{\pgfqpoint{1.502269in}{0.691976in}}%
\pgfpathlineto{\pgfqpoint{1.490802in}{0.692752in}}%
\pgfpathlineto{\pgfqpoint{1.478698in}{0.693510in}}%
\pgfpathlineto{\pgfqpoint{1.465881in}{0.694251in}}%
\pgfpathlineto{\pgfqpoint{1.452262in}{0.694974in}}%
\pgfpathlineto{\pgfqpoint{1.437735in}{0.695680in}}%
\pgfpathlineto{\pgfqpoint{1.422168in}{0.696367in}}%
\pgfpathlineto{\pgfqpoint{1.405403in}{0.697037in}}%
\pgfpathlineto{\pgfqpoint{1.387238in}{0.697689in}}%
\pgfpathlineto{\pgfqpoint{1.367419in}{0.698328in}}%
\pgfpathlineto{\pgfqpoint{1.345614in}{0.698951in}}%
\pgfpathlineto{\pgfqpoint{1.321380in}{0.699558in}}%
\pgfpathlineto{\pgfqpoint{1.294107in}{0.700148in}}%
\pgfpathlineto{\pgfqpoint{1.262923in}{0.700720in}}%
\pgfpathlineto{\pgfqpoint{1.226515in}{0.701276in}}%
\pgfpathlineto{\pgfqpoint{1.182772in}{0.701815in}}%
\pgfpathlineto{\pgfqpoint{1.127977in}{0.702336in}}%
\pgfpathlineto{\pgfqpoint{1.054581in}{0.702841in}}%
\pgfpathlineto{\pgfqpoint{0.943050in}{0.703329in}}%
\pgfpathlineto{\pgfqpoint{0.702340in}{0.703799in}}%
\pgfpathclose%
\pgfusepath{stroke,fill}%
\end{pgfscope}%
\begin{pgfscope}%
\pgfpathrectangle{\pgfqpoint{0.702340in}{0.435433in}}{\pgfqpoint{1.223103in}{0.607948in}}%
\pgfusepath{clip}%
\pgfsetroundcap%
\pgfsetroundjoin%
\pgfsetlinewidth{0.501875pt}%
\definecolor{currentstroke}{rgb}{0.000000,0.000000,1.000000}%
\pgfsetstrokecolor{currentstroke}%
\pgfsetstrokeopacity{0.800000}%
\pgfsetdash{}{0pt}%
\pgfpathmoveto{\pgfqpoint{0.702340in}{0.703942in}}%
\pgfpathlineto{\pgfqpoint{1.925444in}{0.703942in}}%
\pgfusepath{stroke}%
\end{pgfscope}%
\begin{pgfscope}%
\pgfpathrectangle{\pgfqpoint{0.702340in}{0.435433in}}{\pgfqpoint{1.223103in}{0.607948in}}%
\pgfusepath{clip}%
\pgfsetbuttcap%
\pgfsetroundjoin%
\pgfsetlinewidth{1.003750pt}%
\definecolor{currentstroke}{rgb}{0.000000,0.000000,0.000000}%
\pgfsetstrokecolor{currentstroke}%
\pgfsetdash{{3.700000pt}{1.600000pt}}{0.000000pt}%
\pgfpathmoveto{\pgfqpoint{0.702340in}{0.703897in}}%
\pgfpathlineto{\pgfqpoint{1.925444in}{0.703897in}}%
\pgfusepath{stroke}%
\end{pgfscope}%
\begin{pgfscope}%
\pgfsetroundcap%
\pgfsetroundjoin%
\pgfsetlinewidth{0.501875pt}%
\definecolor{currentstroke}{rgb}{0.000000,0.000000,1.000000}%
\pgfsetstrokecolor{currentstroke}%
\pgfsetstrokeopacity{0.800000}%
\pgfsetdash{}{0pt}%
\pgfpathmoveto{\pgfqpoint{1.529420in}{0.822046in}}%
\pgfpathquadraticcurveto{\pgfqpoint{1.453556in}{0.770768in}}{\pgfqpoint{1.377692in}{0.719491in}}%
\pgfusepath{stroke}%
\end{pgfscope}%
\begin{pgfscope}%
\pgfsetfillopacity{0.800000}%
\pgfsetstrokeopacity{0.800000}%
\definecolor{textcolor}{rgb}{0.000000,0.000000,1.000000}%
\pgfsetstrokecolor{textcolor}%
\pgfsetfillcolor{textcolor}%
\pgftext[x=1.442982in,y=0.886326in,left,base]{\color{textcolor}\sffamily\fontsize{5.647059}{6.776471}\selectfont 1.44167(14)}%
\end{pgfscope}%
\begin{pgfscope}%
\pgfsetbuttcap%
\pgfsetroundjoin%
\definecolor{currentfill}{rgb}{0.150000,0.150000,0.150000}%
\pgfsetfillcolor{currentfill}%
\pgfsetlinewidth{1.003750pt}%
\definecolor{currentstroke}{rgb}{0.150000,0.150000,0.150000}%
\pgfsetstrokecolor{currentstroke}%
\pgfsetdash{}{0pt}%
\pgfsys@defobject{currentmarker}{\pgfqpoint{0.000000in}{-0.066667in}}{\pgfqpoint{0.000000in}{0.000000in}}{%
\pgfpathmoveto{\pgfqpoint{0.000000in}{0.000000in}}%
\pgfpathlineto{\pgfqpoint{0.000000in}{-0.066667in}}%
\pgfusepath{stroke,fill}%
}%
\begin{pgfscope}%
\pgfsys@transformshift{0.702340in}{0.435433in}%
\pgfsys@useobject{currentmarker}{}%
\end{pgfscope}%
\end{pgfscope}%
\begin{pgfscope}%
\definecolor{textcolor}{rgb}{0.150000,0.150000,0.150000}%
\pgfsetstrokecolor{textcolor}%
\pgfsetfillcolor{textcolor}%
\pgftext[x=0.702340in,y=0.320155in,,top]{\color{textcolor}\sffamily\fontsize{5.176471}{6.211765}\selectfont \(\displaystyle {10^{-3}}\)}%
\end{pgfscope}%
\begin{pgfscope}%
\pgfsetbuttcap%
\pgfsetroundjoin%
\definecolor{currentfill}{rgb}{0.150000,0.150000,0.150000}%
\pgfsetfillcolor{currentfill}%
\pgfsetlinewidth{1.003750pt}%
\definecolor{currentstroke}{rgb}{0.150000,0.150000,0.150000}%
\pgfsetstrokecolor{currentstroke}%
\pgfsetdash{}{0pt}%
\pgfsys@defobject{currentmarker}{\pgfqpoint{0.000000in}{-0.066667in}}{\pgfqpoint{0.000000in}{0.000000in}}{%
\pgfpathmoveto{\pgfqpoint{0.000000in}{0.000000in}}%
\pgfpathlineto{\pgfqpoint{0.000000in}{-0.066667in}}%
\pgfusepath{stroke,fill}%
}%
\begin{pgfscope}%
\pgfsys@transformshift{1.203749in}{0.435433in}%
\pgfsys@useobject{currentmarker}{}%
\end{pgfscope}%
\end{pgfscope}%
\begin{pgfscope}%
\definecolor{textcolor}{rgb}{0.150000,0.150000,0.150000}%
\pgfsetstrokecolor{textcolor}%
\pgfsetfillcolor{textcolor}%
\pgftext[x=1.203749in,y=0.320155in,,top]{\color{textcolor}\sffamily\fontsize{5.176471}{6.211765}\selectfont \(\displaystyle {10^{-2}}\)}%
\end{pgfscope}%
\begin{pgfscope}%
\pgfsetbuttcap%
\pgfsetroundjoin%
\definecolor{currentfill}{rgb}{0.150000,0.150000,0.150000}%
\pgfsetfillcolor{currentfill}%
\pgfsetlinewidth{1.003750pt}%
\definecolor{currentstroke}{rgb}{0.150000,0.150000,0.150000}%
\pgfsetstrokecolor{currentstroke}%
\pgfsetdash{}{0pt}%
\pgfsys@defobject{currentmarker}{\pgfqpoint{0.000000in}{-0.066667in}}{\pgfqpoint{0.000000in}{0.000000in}}{%
\pgfpathmoveto{\pgfqpoint{0.000000in}{0.000000in}}%
\pgfpathlineto{\pgfqpoint{0.000000in}{-0.066667in}}%
\pgfusepath{stroke,fill}%
}%
\begin{pgfscope}%
\pgfsys@transformshift{1.705158in}{0.435433in}%
\pgfsys@useobject{currentmarker}{}%
\end{pgfscope}%
\end{pgfscope}%
\begin{pgfscope}%
\definecolor{textcolor}{rgb}{0.150000,0.150000,0.150000}%
\pgfsetstrokecolor{textcolor}%
\pgfsetfillcolor{textcolor}%
\pgftext[x=1.705158in,y=0.320155in,,top]{\color{textcolor}\sffamily\fontsize{5.176471}{6.211765}\selectfont \(\displaystyle {10^{-1}}\)}%
\end{pgfscope}%
\begin{pgfscope}%
\pgfsetbuttcap%
\pgfsetroundjoin%
\definecolor{currentfill}{rgb}{0.150000,0.150000,0.150000}%
\pgfsetfillcolor{currentfill}%
\pgfsetlinewidth{0.803000pt}%
\definecolor{currentstroke}{rgb}{0.150000,0.150000,0.150000}%
\pgfsetstrokecolor{currentstroke}%
\pgfsetdash{}{0pt}%
\pgfsys@defobject{currentmarker}{\pgfqpoint{0.000000in}{-0.044444in}}{\pgfqpoint{0.000000in}{0.000000in}}{%
\pgfpathmoveto{\pgfqpoint{0.000000in}{0.000000in}}%
\pgfpathlineto{\pgfqpoint{0.000000in}{-0.044444in}}%
\pgfusepath{stroke,fill}%
}%
\begin{pgfscope}%
\pgfsys@transformshift{0.853280in}{0.435433in}%
\pgfsys@useobject{currentmarker}{}%
\end{pgfscope}%
\end{pgfscope}%
\begin{pgfscope}%
\pgfsetbuttcap%
\pgfsetroundjoin%
\definecolor{currentfill}{rgb}{0.150000,0.150000,0.150000}%
\pgfsetfillcolor{currentfill}%
\pgfsetlinewidth{0.803000pt}%
\definecolor{currentstroke}{rgb}{0.150000,0.150000,0.150000}%
\pgfsetstrokecolor{currentstroke}%
\pgfsetdash{}{0pt}%
\pgfsys@defobject{currentmarker}{\pgfqpoint{0.000000in}{-0.044444in}}{\pgfqpoint{0.000000in}{0.000000in}}{%
\pgfpathmoveto{\pgfqpoint{0.000000in}{0.000000in}}%
\pgfpathlineto{\pgfqpoint{0.000000in}{-0.044444in}}%
\pgfusepath{stroke,fill}%
}%
\begin{pgfscope}%
\pgfsys@transformshift{0.941573in}{0.435433in}%
\pgfsys@useobject{currentmarker}{}%
\end{pgfscope}%
\end{pgfscope}%
\begin{pgfscope}%
\pgfsetbuttcap%
\pgfsetroundjoin%
\definecolor{currentfill}{rgb}{0.150000,0.150000,0.150000}%
\pgfsetfillcolor{currentfill}%
\pgfsetlinewidth{0.803000pt}%
\definecolor{currentstroke}{rgb}{0.150000,0.150000,0.150000}%
\pgfsetstrokecolor{currentstroke}%
\pgfsetdash{}{0pt}%
\pgfsys@defobject{currentmarker}{\pgfqpoint{0.000000in}{-0.044444in}}{\pgfqpoint{0.000000in}{0.000000in}}{%
\pgfpathmoveto{\pgfqpoint{0.000000in}{0.000000in}}%
\pgfpathlineto{\pgfqpoint{0.000000in}{-0.044444in}}%
\pgfusepath{stroke,fill}%
}%
\begin{pgfscope}%
\pgfsys@transformshift{1.004219in}{0.435433in}%
\pgfsys@useobject{currentmarker}{}%
\end{pgfscope}%
\end{pgfscope}%
\begin{pgfscope}%
\pgfsetbuttcap%
\pgfsetroundjoin%
\definecolor{currentfill}{rgb}{0.150000,0.150000,0.150000}%
\pgfsetfillcolor{currentfill}%
\pgfsetlinewidth{0.803000pt}%
\definecolor{currentstroke}{rgb}{0.150000,0.150000,0.150000}%
\pgfsetstrokecolor{currentstroke}%
\pgfsetdash{}{0pt}%
\pgfsys@defobject{currentmarker}{\pgfqpoint{0.000000in}{-0.044444in}}{\pgfqpoint{0.000000in}{0.000000in}}{%
\pgfpathmoveto{\pgfqpoint{0.000000in}{0.000000in}}%
\pgfpathlineto{\pgfqpoint{0.000000in}{-0.044444in}}%
\pgfusepath{stroke,fill}%
}%
\begin{pgfscope}%
\pgfsys@transformshift{1.052810in}{0.435433in}%
\pgfsys@useobject{currentmarker}{}%
\end{pgfscope}%
\end{pgfscope}%
\begin{pgfscope}%
\pgfsetbuttcap%
\pgfsetroundjoin%
\definecolor{currentfill}{rgb}{0.150000,0.150000,0.150000}%
\pgfsetfillcolor{currentfill}%
\pgfsetlinewidth{0.803000pt}%
\definecolor{currentstroke}{rgb}{0.150000,0.150000,0.150000}%
\pgfsetstrokecolor{currentstroke}%
\pgfsetdash{}{0pt}%
\pgfsys@defobject{currentmarker}{\pgfqpoint{0.000000in}{-0.044444in}}{\pgfqpoint{0.000000in}{0.000000in}}{%
\pgfpathmoveto{\pgfqpoint{0.000000in}{0.000000in}}%
\pgfpathlineto{\pgfqpoint{0.000000in}{-0.044444in}}%
\pgfusepath{stroke,fill}%
}%
\begin{pgfscope}%
\pgfsys@transformshift{1.092512in}{0.435433in}%
\pgfsys@useobject{currentmarker}{}%
\end{pgfscope}%
\end{pgfscope}%
\begin{pgfscope}%
\pgfsetbuttcap%
\pgfsetroundjoin%
\definecolor{currentfill}{rgb}{0.150000,0.150000,0.150000}%
\pgfsetfillcolor{currentfill}%
\pgfsetlinewidth{0.803000pt}%
\definecolor{currentstroke}{rgb}{0.150000,0.150000,0.150000}%
\pgfsetstrokecolor{currentstroke}%
\pgfsetdash{}{0pt}%
\pgfsys@defobject{currentmarker}{\pgfqpoint{0.000000in}{-0.044444in}}{\pgfqpoint{0.000000in}{0.000000in}}{%
\pgfpathmoveto{\pgfqpoint{0.000000in}{0.000000in}}%
\pgfpathlineto{\pgfqpoint{0.000000in}{-0.044444in}}%
\pgfusepath{stroke,fill}%
}%
\begin{pgfscope}%
\pgfsys@transformshift{1.126080in}{0.435433in}%
\pgfsys@useobject{currentmarker}{}%
\end{pgfscope}%
\end{pgfscope}%
\begin{pgfscope}%
\pgfsetbuttcap%
\pgfsetroundjoin%
\definecolor{currentfill}{rgb}{0.150000,0.150000,0.150000}%
\pgfsetfillcolor{currentfill}%
\pgfsetlinewidth{0.803000pt}%
\definecolor{currentstroke}{rgb}{0.150000,0.150000,0.150000}%
\pgfsetstrokecolor{currentstroke}%
\pgfsetdash{}{0pt}%
\pgfsys@defobject{currentmarker}{\pgfqpoint{0.000000in}{-0.044444in}}{\pgfqpoint{0.000000in}{0.000000in}}{%
\pgfpathmoveto{\pgfqpoint{0.000000in}{0.000000in}}%
\pgfpathlineto{\pgfqpoint{0.000000in}{-0.044444in}}%
\pgfusepath{stroke,fill}%
}%
\begin{pgfscope}%
\pgfsys@transformshift{1.155158in}{0.435433in}%
\pgfsys@useobject{currentmarker}{}%
\end{pgfscope}%
\end{pgfscope}%
\begin{pgfscope}%
\pgfsetbuttcap%
\pgfsetroundjoin%
\definecolor{currentfill}{rgb}{0.150000,0.150000,0.150000}%
\pgfsetfillcolor{currentfill}%
\pgfsetlinewidth{0.803000pt}%
\definecolor{currentstroke}{rgb}{0.150000,0.150000,0.150000}%
\pgfsetstrokecolor{currentstroke}%
\pgfsetdash{}{0pt}%
\pgfsys@defobject{currentmarker}{\pgfqpoint{0.000000in}{-0.044444in}}{\pgfqpoint{0.000000in}{0.000000in}}{%
\pgfpathmoveto{\pgfqpoint{0.000000in}{0.000000in}}%
\pgfpathlineto{\pgfqpoint{0.000000in}{-0.044444in}}%
\pgfusepath{stroke,fill}%
}%
\begin{pgfscope}%
\pgfsys@transformshift{1.180806in}{0.435433in}%
\pgfsys@useobject{currentmarker}{}%
\end{pgfscope}%
\end{pgfscope}%
\begin{pgfscope}%
\pgfsetbuttcap%
\pgfsetroundjoin%
\definecolor{currentfill}{rgb}{0.150000,0.150000,0.150000}%
\pgfsetfillcolor{currentfill}%
\pgfsetlinewidth{0.803000pt}%
\definecolor{currentstroke}{rgb}{0.150000,0.150000,0.150000}%
\pgfsetstrokecolor{currentstroke}%
\pgfsetdash{}{0pt}%
\pgfsys@defobject{currentmarker}{\pgfqpoint{0.000000in}{-0.044444in}}{\pgfqpoint{0.000000in}{0.000000in}}{%
\pgfpathmoveto{\pgfqpoint{0.000000in}{0.000000in}}%
\pgfpathlineto{\pgfqpoint{0.000000in}{-0.044444in}}%
\pgfusepath{stroke,fill}%
}%
\begin{pgfscope}%
\pgfsys@transformshift{1.354689in}{0.435433in}%
\pgfsys@useobject{currentmarker}{}%
\end{pgfscope}%
\end{pgfscope}%
\begin{pgfscope}%
\pgfsetbuttcap%
\pgfsetroundjoin%
\definecolor{currentfill}{rgb}{0.150000,0.150000,0.150000}%
\pgfsetfillcolor{currentfill}%
\pgfsetlinewidth{0.803000pt}%
\definecolor{currentstroke}{rgb}{0.150000,0.150000,0.150000}%
\pgfsetstrokecolor{currentstroke}%
\pgfsetdash{}{0pt}%
\pgfsys@defobject{currentmarker}{\pgfqpoint{0.000000in}{-0.044444in}}{\pgfqpoint{0.000000in}{0.000000in}}{%
\pgfpathmoveto{\pgfqpoint{0.000000in}{0.000000in}}%
\pgfpathlineto{\pgfqpoint{0.000000in}{-0.044444in}}%
\pgfusepath{stroke,fill}%
}%
\begin{pgfscope}%
\pgfsys@transformshift{1.442982in}{0.435433in}%
\pgfsys@useobject{currentmarker}{}%
\end{pgfscope}%
\end{pgfscope}%
\begin{pgfscope}%
\pgfsetbuttcap%
\pgfsetroundjoin%
\definecolor{currentfill}{rgb}{0.150000,0.150000,0.150000}%
\pgfsetfillcolor{currentfill}%
\pgfsetlinewidth{0.803000pt}%
\definecolor{currentstroke}{rgb}{0.150000,0.150000,0.150000}%
\pgfsetstrokecolor{currentstroke}%
\pgfsetdash{}{0pt}%
\pgfsys@defobject{currentmarker}{\pgfqpoint{0.000000in}{-0.044444in}}{\pgfqpoint{0.000000in}{0.000000in}}{%
\pgfpathmoveto{\pgfqpoint{0.000000in}{0.000000in}}%
\pgfpathlineto{\pgfqpoint{0.000000in}{-0.044444in}}%
\pgfusepath{stroke,fill}%
}%
\begin{pgfscope}%
\pgfsys@transformshift{1.505628in}{0.435433in}%
\pgfsys@useobject{currentmarker}{}%
\end{pgfscope}%
\end{pgfscope}%
\begin{pgfscope}%
\pgfsetbuttcap%
\pgfsetroundjoin%
\definecolor{currentfill}{rgb}{0.150000,0.150000,0.150000}%
\pgfsetfillcolor{currentfill}%
\pgfsetlinewidth{0.803000pt}%
\definecolor{currentstroke}{rgb}{0.150000,0.150000,0.150000}%
\pgfsetstrokecolor{currentstroke}%
\pgfsetdash{}{0pt}%
\pgfsys@defobject{currentmarker}{\pgfqpoint{0.000000in}{-0.044444in}}{\pgfqpoint{0.000000in}{0.000000in}}{%
\pgfpathmoveto{\pgfqpoint{0.000000in}{0.000000in}}%
\pgfpathlineto{\pgfqpoint{0.000000in}{-0.044444in}}%
\pgfusepath{stroke,fill}%
}%
\begin{pgfscope}%
\pgfsys@transformshift{1.554219in}{0.435433in}%
\pgfsys@useobject{currentmarker}{}%
\end{pgfscope}%
\end{pgfscope}%
\begin{pgfscope}%
\pgfsetbuttcap%
\pgfsetroundjoin%
\definecolor{currentfill}{rgb}{0.150000,0.150000,0.150000}%
\pgfsetfillcolor{currentfill}%
\pgfsetlinewidth{0.803000pt}%
\definecolor{currentstroke}{rgb}{0.150000,0.150000,0.150000}%
\pgfsetstrokecolor{currentstroke}%
\pgfsetdash{}{0pt}%
\pgfsys@defobject{currentmarker}{\pgfqpoint{0.000000in}{-0.044444in}}{\pgfqpoint{0.000000in}{0.000000in}}{%
\pgfpathmoveto{\pgfqpoint{0.000000in}{0.000000in}}%
\pgfpathlineto{\pgfqpoint{0.000000in}{-0.044444in}}%
\pgfusepath{stroke,fill}%
}%
\begin{pgfscope}%
\pgfsys@transformshift{1.593921in}{0.435433in}%
\pgfsys@useobject{currentmarker}{}%
\end{pgfscope}%
\end{pgfscope}%
\begin{pgfscope}%
\pgfsetbuttcap%
\pgfsetroundjoin%
\definecolor{currentfill}{rgb}{0.150000,0.150000,0.150000}%
\pgfsetfillcolor{currentfill}%
\pgfsetlinewidth{0.803000pt}%
\definecolor{currentstroke}{rgb}{0.150000,0.150000,0.150000}%
\pgfsetstrokecolor{currentstroke}%
\pgfsetdash{}{0pt}%
\pgfsys@defobject{currentmarker}{\pgfqpoint{0.000000in}{-0.044444in}}{\pgfqpoint{0.000000in}{0.000000in}}{%
\pgfpathmoveto{\pgfqpoint{0.000000in}{0.000000in}}%
\pgfpathlineto{\pgfqpoint{0.000000in}{-0.044444in}}%
\pgfusepath{stroke,fill}%
}%
\begin{pgfscope}%
\pgfsys@transformshift{1.627489in}{0.435433in}%
\pgfsys@useobject{currentmarker}{}%
\end{pgfscope}%
\end{pgfscope}%
\begin{pgfscope}%
\pgfsetbuttcap%
\pgfsetroundjoin%
\definecolor{currentfill}{rgb}{0.150000,0.150000,0.150000}%
\pgfsetfillcolor{currentfill}%
\pgfsetlinewidth{0.803000pt}%
\definecolor{currentstroke}{rgb}{0.150000,0.150000,0.150000}%
\pgfsetstrokecolor{currentstroke}%
\pgfsetdash{}{0pt}%
\pgfsys@defobject{currentmarker}{\pgfqpoint{0.000000in}{-0.044444in}}{\pgfqpoint{0.000000in}{0.000000in}}{%
\pgfpathmoveto{\pgfqpoint{0.000000in}{0.000000in}}%
\pgfpathlineto{\pgfqpoint{0.000000in}{-0.044444in}}%
\pgfusepath{stroke,fill}%
}%
\begin{pgfscope}%
\pgfsys@transformshift{1.656567in}{0.435433in}%
\pgfsys@useobject{currentmarker}{}%
\end{pgfscope}%
\end{pgfscope}%
\begin{pgfscope}%
\pgfsetbuttcap%
\pgfsetroundjoin%
\definecolor{currentfill}{rgb}{0.150000,0.150000,0.150000}%
\pgfsetfillcolor{currentfill}%
\pgfsetlinewidth{0.803000pt}%
\definecolor{currentstroke}{rgb}{0.150000,0.150000,0.150000}%
\pgfsetstrokecolor{currentstroke}%
\pgfsetdash{}{0pt}%
\pgfsys@defobject{currentmarker}{\pgfqpoint{0.000000in}{-0.044444in}}{\pgfqpoint{0.000000in}{0.000000in}}{%
\pgfpathmoveto{\pgfqpoint{0.000000in}{0.000000in}}%
\pgfpathlineto{\pgfqpoint{0.000000in}{-0.044444in}}%
\pgfusepath{stroke,fill}%
}%
\begin{pgfscope}%
\pgfsys@transformshift{1.682215in}{0.435433in}%
\pgfsys@useobject{currentmarker}{}%
\end{pgfscope}%
\end{pgfscope}%
\begin{pgfscope}%
\pgfsetbuttcap%
\pgfsetroundjoin%
\definecolor{currentfill}{rgb}{0.150000,0.150000,0.150000}%
\pgfsetfillcolor{currentfill}%
\pgfsetlinewidth{0.803000pt}%
\definecolor{currentstroke}{rgb}{0.150000,0.150000,0.150000}%
\pgfsetstrokecolor{currentstroke}%
\pgfsetdash{}{0pt}%
\pgfsys@defobject{currentmarker}{\pgfqpoint{0.000000in}{-0.044444in}}{\pgfqpoint{0.000000in}{0.000000in}}{%
\pgfpathmoveto{\pgfqpoint{0.000000in}{0.000000in}}%
\pgfpathlineto{\pgfqpoint{0.000000in}{-0.044444in}}%
\pgfusepath{stroke,fill}%
}%
\begin{pgfscope}%
\pgfsys@transformshift{1.856098in}{0.435433in}%
\pgfsys@useobject{currentmarker}{}%
\end{pgfscope}%
\end{pgfscope}%
\begin{pgfscope}%
\definecolor{textcolor}{rgb}{0.150000,0.150000,0.150000}%
\pgfsetstrokecolor{textcolor}%
\pgfsetfillcolor{textcolor}%
\pgftext[x=1.313892in,y=0.183333in,,top]{\color{textcolor}\sffamily\fontsize{5.647059}{6.776471}\selectfont \(\displaystyle \epsilon [\mathrm{fm}]\)}%
\end{pgfscope}%
\begin{pgfscope}%
\pgfsetbuttcap%
\pgfsetroundjoin%
\definecolor{currentfill}{rgb}{0.150000,0.150000,0.150000}%
\pgfsetfillcolor{currentfill}%
\pgfsetlinewidth{1.003750pt}%
\definecolor{currentstroke}{rgb}{0.150000,0.150000,0.150000}%
\pgfsetstrokecolor{currentstroke}%
\pgfsetdash{}{0pt}%
\pgfsys@defobject{currentmarker}{\pgfqpoint{-0.066667in}{0.000000in}}{\pgfqpoint{0.000000in}{0.000000in}}{%
\pgfpathmoveto{\pgfqpoint{0.000000in}{0.000000in}}%
\pgfpathlineto{\pgfqpoint{-0.066667in}{0.000000in}}%
\pgfusepath{stroke,fill}%
}%
\begin{pgfscope}%
\pgfsys@transformshift{0.702340in}{0.435433in}%
\pgfsys@useobject{currentmarker}{}%
\end{pgfscope}%
\end{pgfscope}%
\begin{pgfscope}%
\definecolor{textcolor}{rgb}{0.150000,0.150000,0.150000}%
\pgfsetstrokecolor{textcolor}%
\pgfsetfillcolor{textcolor}%
\pgftext[x=0.413148in,y=0.410485in,left,base]{\color{textcolor}\sffamily\fontsize{5.176471}{6.211765}\selectfont 1.000}%
\end{pgfscope}%
\begin{pgfscope}%
\pgfsetbuttcap%
\pgfsetroundjoin%
\definecolor{currentfill}{rgb}{0.150000,0.150000,0.150000}%
\pgfsetfillcolor{currentfill}%
\pgfsetlinewidth{1.003750pt}%
\definecolor{currentstroke}{rgb}{0.150000,0.150000,0.150000}%
\pgfsetstrokecolor{currentstroke}%
\pgfsetdash{}{0pt}%
\pgfsys@defobject{currentmarker}{\pgfqpoint{-0.066667in}{0.000000in}}{\pgfqpoint{0.000000in}{0.000000in}}{%
\pgfpathmoveto{\pgfqpoint{0.000000in}{0.000000in}}%
\pgfpathlineto{\pgfqpoint{-0.066667in}{0.000000in}}%
\pgfusepath{stroke,fill}%
}%
\begin{pgfscope}%
\pgfsys@transformshift{0.702340in}{0.703897in}%
\pgfsys@useobject{currentmarker}{}%
\end{pgfscope}%
\end{pgfscope}%
\begin{pgfscope}%
\definecolor{textcolor}{rgb}{0.150000,0.150000,0.150000}%
\pgfsetstrokecolor{textcolor}%
\pgfsetfillcolor{textcolor}%
\pgftext[x=0.413148in,y=0.678949in,left,base]{\color{textcolor}\sffamily\fontsize{5.176471}{6.211765}\selectfont 1.442}%
\end{pgfscope}%
\begin{pgfscope}%
\pgfsetbuttcap%
\pgfsetroundjoin%
\definecolor{currentfill}{rgb}{0.150000,0.150000,0.150000}%
\pgfsetfillcolor{currentfill}%
\pgfsetlinewidth{1.003750pt}%
\definecolor{currentstroke}{rgb}{0.150000,0.150000,0.150000}%
\pgfsetstrokecolor{currentstroke}%
\pgfsetdash{}{0pt}%
\pgfsys@defobject{currentmarker}{\pgfqpoint{-0.066667in}{0.000000in}}{\pgfqpoint{0.000000in}{0.000000in}}{%
\pgfpathmoveto{\pgfqpoint{0.000000in}{0.000000in}}%
\pgfpathlineto{\pgfqpoint{-0.066667in}{0.000000in}}%
\pgfusepath{stroke,fill}%
}%
\begin{pgfscope}%
\pgfsys@transformshift{0.702340in}{1.043381in}%
\pgfsys@useobject{currentmarker}{}%
\end{pgfscope}%
\end{pgfscope}%
\begin{pgfscope}%
\definecolor{textcolor}{rgb}{0.150000,0.150000,0.150000}%
\pgfsetstrokecolor{textcolor}%
\pgfsetfillcolor{textcolor}%
\pgftext[x=0.413148in,y=1.018433in,left,base]{\color{textcolor}\sffamily\fontsize{5.176471}{6.211765}\selectfont 2.000}%
\end{pgfscope}%
\begin{pgfscope}%
\definecolor{textcolor}{rgb}{0.150000,0.150000,0.150000}%
\pgfsetstrokecolor{textcolor}%
\pgfsetfillcolor{textcolor}%
\pgftext[x=0.357592in,y=0.739407in,,bottom,rotate=90.000000]{\color{textcolor}\sffamily\fontsize{5.647059}{6.776471}\selectfont \(\displaystyle x = \frac{2 \mu E L^2}{4 \pi^2}\)}%
\end{pgfscope}%
\begin{pgfscope}%
\pgfpathrectangle{\pgfqpoint{0.702340in}{0.435433in}}{\pgfqpoint{1.223103in}{0.607948in}}%
\pgfusepath{clip}%
\pgfsetroundcap%
\pgfsetroundjoin%
\pgfsetlinewidth{1.204500pt}%
\definecolor{currentstroke}{rgb}{0.000000,0.501961,0.000000}%
\pgfsetstrokecolor{currentstroke}%
\pgfsetdash{}{0pt}%
\pgfpathmoveto{\pgfqpoint{0.702340in}{0.703721in}}%
\pgfpathlineto{\pgfqpoint{0.943050in}{0.703262in}}%
\pgfpathlineto{\pgfqpoint{1.054581in}{0.702785in}}%
\pgfpathlineto{\pgfqpoint{1.127977in}{0.702290in}}%
\pgfpathlineto{\pgfqpoint{1.182772in}{0.701777in}}%
\pgfpathlineto{\pgfqpoint{1.226515in}{0.701246in}}%
\pgfpathlineto{\pgfqpoint{1.262923in}{0.700697in}}%
\pgfpathlineto{\pgfqpoint{1.294107in}{0.700131in}}%
\pgfpathlineto{\pgfqpoint{1.321380in}{0.699546in}}%
\pgfpathlineto{\pgfqpoint{1.345614in}{0.698945in}}%
\pgfpathlineto{\pgfqpoint{1.367419in}{0.698325in}}%
\pgfpathlineto{\pgfqpoint{1.387238in}{0.697688in}}%
\pgfpathlineto{\pgfqpoint{1.405403in}{0.697034in}}%
\pgfpathlineto{\pgfqpoint{1.422168in}{0.696362in}}%
\pgfpathlineto{\pgfqpoint{1.437735in}{0.695672in}}%
\pgfpathlineto{\pgfqpoint{1.452262in}{0.694966in}}%
\pgfpathlineto{\pgfqpoint{1.465881in}{0.694242in}}%
\pgfpathlineto{\pgfqpoint{1.478698in}{0.693501in}}%
\pgfpathlineto{\pgfqpoint{1.490802in}{0.692743in}}%
\pgfpathlineto{\pgfqpoint{1.502269in}{0.691968in}}%
\pgfpathlineto{\pgfqpoint{1.513162in}{0.691175in}}%
\pgfpathlineto{\pgfqpoint{1.523536in}{0.690366in}}%
\pgfpathlineto{\pgfqpoint{1.533438in}{0.689540in}}%
\pgfpathlineto{\pgfqpoint{1.542909in}{0.688697in}}%
\pgfpathlineto{\pgfqpoint{1.551986in}{0.687838in}}%
\pgfpathlineto{\pgfqpoint{1.560699in}{0.686961in}}%
\pgfpathlineto{\pgfqpoint{1.569077in}{0.686069in}}%
\pgfpathlineto{\pgfqpoint{1.577145in}{0.685159in}}%
\pgfpathlineto{\pgfqpoint{1.584924in}{0.684233in}}%
\pgfpathlineto{\pgfqpoint{1.592435in}{0.683291in}}%
\pgfpathlineto{\pgfqpoint{1.599696in}{0.682332in}}%
\pgfpathlineto{\pgfqpoint{1.606722in}{0.681357in}}%
\pgfpathlineto{\pgfqpoint{1.613528in}{0.680366in}}%
\pgfpathlineto{\pgfqpoint{1.620129in}{0.679358in}}%
\pgfpathlineto{\pgfqpoint{1.626535in}{0.678335in}}%
\pgfpathlineto{\pgfqpoint{1.632758in}{0.677295in}}%
\pgfpathlineto{\pgfqpoint{1.638808in}{0.676240in}}%
\pgfpathlineto{\pgfqpoint{1.644694in}{0.675168in}}%
\pgfpathlineto{\pgfqpoint{1.650426in}{0.674081in}}%
\pgfpathlineto{\pgfqpoint{1.656011in}{0.672978in}}%
\pgfpathlineto{\pgfqpoint{1.661456in}{0.671859in}}%
\pgfpathlineto{\pgfqpoint{1.666768in}{0.670725in}}%
\pgfpathlineto{\pgfqpoint{1.671953in}{0.669575in}}%
\pgfpathlineto{\pgfqpoint{1.677018in}{0.668410in}}%
\pgfpathlineto{\pgfqpoint{1.681968in}{0.667229in}}%
\pgfpathlineto{\pgfqpoint{1.686808in}{0.666033in}}%
\pgfpathlineto{\pgfqpoint{1.691543in}{0.664821in}}%
\pgfpathlineto{\pgfqpoint{1.696176in}{0.663594in}}%
\pgfpathlineto{\pgfqpoint{1.700714in}{0.662353in}}%
\pgfpathlineto{\pgfqpoint{1.705158in}{0.661096in}}%
\pgfusepath{stroke}%
\end{pgfscope}%
\begin{pgfscope}%
\pgfsetrectcap%
\pgfsetmiterjoin%
\pgfsetlinewidth{1.003750pt}%
\definecolor{currentstroke}{rgb}{0.150000,0.150000,0.150000}%
\pgfsetstrokecolor{currentstroke}%
\pgfsetdash{}{0pt}%
\pgfpathmoveto{\pgfqpoint{0.702340in}{0.435433in}}%
\pgfpathlineto{\pgfqpoint{0.702340in}{1.043381in}}%
\pgfusepath{stroke}%
\end{pgfscope}%
\begin{pgfscope}%
\pgfsetrectcap%
\pgfsetmiterjoin%
\pgfsetlinewidth{1.003750pt}%
\definecolor{currentstroke}{rgb}{0.150000,0.150000,0.150000}%
\pgfsetstrokecolor{currentstroke}%
\pgfsetdash{}{0pt}%
\pgfpathmoveto{\pgfqpoint{0.702340in}{0.435433in}}%
\pgfpathlineto{\pgfqpoint{1.925444in}{0.435433in}}%
\pgfusepath{stroke}%
\end{pgfscope}%
\begin{pgfscope}%
\pgfpathrectangle{\pgfqpoint{0.702340in}{0.435433in}}{\pgfqpoint{1.223103in}{0.607948in}}%
\pgfusepath{clip}%
\pgfsetbuttcap%
\pgfsetroundjoin%
\definecolor{currentfill}{rgb}{0.000000,0.000000,0.000000}%
\pgfsetfillcolor{currentfill}%
\pgfsetlinewidth{1.003750pt}%
\definecolor{currentstroke}{rgb}{0.000000,0.000000,0.000000}%
\pgfsetstrokecolor{currentstroke}%
\pgfsetdash{}{0pt}%
\pgfsys@defobject{currentmarker}{\pgfqpoint{-0.013889in}{-0.013889in}}{\pgfqpoint{0.013889in}{0.013889in}}{%
\pgfpathmoveto{\pgfqpoint{0.000000in}{-0.013889in}}%
\pgfpathcurveto{\pgfqpoint{0.003683in}{-0.013889in}}{\pgfqpoint{0.007216in}{-0.012425in}}{\pgfqpoint{0.009821in}{-0.009821in}}%
\pgfpathcurveto{\pgfqpoint{0.012425in}{-0.007216in}}{\pgfqpoint{0.013889in}{-0.003683in}}{\pgfqpoint{0.013889in}{0.000000in}}%
\pgfpathcurveto{\pgfqpoint{0.013889in}{0.003683in}}{\pgfqpoint{0.012425in}{0.007216in}}{\pgfqpoint{0.009821in}{0.009821in}}%
\pgfpathcurveto{\pgfqpoint{0.007216in}{0.012425in}}{\pgfqpoint{0.003683in}{0.013889in}}{\pgfqpoint{0.000000in}{0.013889in}}%
\pgfpathcurveto{\pgfqpoint{-0.003683in}{0.013889in}}{\pgfqpoint{-0.007216in}{0.012425in}}{\pgfqpoint{-0.009821in}{0.009821in}}%
\pgfpathcurveto{\pgfqpoint{-0.012425in}{0.007216in}}{\pgfqpoint{-0.013889in}{0.003683in}}{\pgfqpoint{-0.013889in}{0.000000in}}%
\pgfpathcurveto{\pgfqpoint{-0.013889in}{-0.003683in}}{\pgfqpoint{-0.012425in}{-0.007216in}}{\pgfqpoint{-0.009821in}{-0.009821in}}%
\pgfpathcurveto{\pgfqpoint{-0.007216in}{-0.012425in}}{\pgfqpoint{-0.003683in}{-0.013889in}}{\pgfqpoint{0.000000in}{-0.013889in}}%
\pgfpathclose%
\pgfusepath{stroke,fill}%
}%
\begin{pgfscope}%
\pgfsys@transformshift{1.705158in}{0.661098in}%
\pgfsys@useobject{currentmarker}{}%
\end{pgfscope}%
\begin{pgfscope}%
\pgfsys@transformshift{1.616865in}{0.679853in}%
\pgfsys@useobject{currentmarker}{}%
\end{pgfscope}%
\begin{pgfscope}%
\pgfsys@transformshift{1.554219in}{0.687619in}%
\pgfsys@useobject{currentmarker}{}%
\end{pgfscope}%
\begin{pgfscope}%
\pgfsys@transformshift{1.505628in}{0.691734in}%
\pgfsys@useobject{currentmarker}{}%
\end{pgfscope}%
\begin{pgfscope}%
\pgfsys@transformshift{1.465926in}{0.694245in}%
\pgfsys@useobject{currentmarker}{}%
\end{pgfscope}%
\begin{pgfscope}%
\pgfsys@transformshift{1.432358in}{0.695922in}%
\pgfsys@useobject{currentmarker}{}%
\end{pgfscope}%
\begin{pgfscope}%
\pgfsys@transformshift{1.403280in}{0.697116in}%
\pgfsys@useobject{currentmarker}{}%
\end{pgfscope}%
\begin{pgfscope}%
\pgfsys@transformshift{1.397903in}{0.697314in}%
\pgfsys@useobject{currentmarker}{}%
\end{pgfscope}%
\begin{pgfscope}%
\pgfsys@transformshift{1.392656in}{0.697501in}%
\pgfsys@useobject{currentmarker}{}%
\end{pgfscope}%
\begin{pgfscope}%
\pgfsys@transformshift{1.387532in}{0.697678in}%
\pgfsys@useobject{currentmarker}{}%
\end{pgfscope}%
\begin{pgfscope}%
\pgfsys@transformshift{1.382525in}{0.697846in}%
\pgfsys@useobject{currentmarker}{}%
\end{pgfscope}%
\begin{pgfscope}%
\pgfsys@transformshift{1.377632in}{0.698006in}%
\pgfsys@useobject{currentmarker}{}%
\end{pgfscope}%
\begin{pgfscope}%
\pgfsys@transformshift{1.372846in}{0.698157in}%
\pgfsys@useobject{currentmarker}{}%
\end{pgfscope}%
\begin{pgfscope}%
\pgfsys@transformshift{1.368163in}{0.698301in}%
\pgfsys@useobject{currentmarker}{}%
\end{pgfscope}%
\begin{pgfscope}%
\pgfsys@transformshift{1.363578in}{0.698438in}%
\pgfsys@useobject{currentmarker}{}%
\end{pgfscope}%
\begin{pgfscope}%
\pgfsys@transformshift{1.359088in}{0.698569in}%
\pgfsys@useobject{currentmarker}{}%
\end{pgfscope}%
\begin{pgfscope}%
\pgfsys@transformshift{1.354689in}{0.698694in}%
\pgfsys@useobject{currentmarker}{}%
\end{pgfscope}%
\end{pgfscope}%
\begin{pgfscope}%
\pgfsetbuttcap%
\pgfsetmiterjoin%
\definecolor{currentfill}{rgb}{1.000000,1.000000,1.000000}%
\pgfsetfillcolor{currentfill}%
\pgfsetlinewidth{0.000000pt}%
\definecolor{currentstroke}{rgb}{0.000000,0.000000,0.000000}%
\pgfsetstrokecolor{currentstroke}%
\pgfsetstrokeopacity{0.000000}%
\pgfsetdash{}{0pt}%
\pgfpathmoveto{\pgfqpoint{2.170064in}{0.435433in}}%
\pgfpathlineto{\pgfqpoint{3.393168in}{0.435433in}}%
\pgfpathlineto{\pgfqpoint{3.393168in}{1.043381in}}%
\pgfpathlineto{\pgfqpoint{2.170064in}{1.043381in}}%
\pgfpathclose%
\pgfusepath{fill}%
\end{pgfscope}%
\begin{pgfscope}%
\pgfpathrectangle{\pgfqpoint{2.170064in}{0.435433in}}{\pgfqpoint{1.223103in}{0.607948in}}%
\pgfusepath{clip}%
\pgfsetbuttcap%
\pgfsetmiterjoin%
\definecolor{currentfill}{rgb}{0.000000,0.000000,1.000000}%
\pgfsetfillcolor{currentfill}%
\pgfsetfillopacity{0.100000}%
\pgfsetlinewidth{0.803000pt}%
\definecolor{currentstroke}{rgb}{0.000000,0.000000,1.000000}%
\pgfsetstrokecolor{currentstroke}%
\pgfsetstrokeopacity{0.100000}%
\pgfsetdash{}{0pt}%
\pgfpathmoveto{\pgfqpoint{2.170064in}{0.703647in}}%
\pgfpathlineto{\pgfqpoint{2.170064in}{0.703977in}}%
\pgfpathlineto{\pgfqpoint{3.393168in}{0.703977in}}%
\pgfpathlineto{\pgfqpoint{3.393168in}{0.703647in}}%
\pgfpathclose%
\pgfusepath{stroke,fill}%
\end{pgfscope}%
\begin{pgfscope}%
\pgfpathrectangle{\pgfqpoint{2.170064in}{0.435433in}}{\pgfqpoint{1.223103in}{0.607948in}}%
\pgfusepath{clip}%
\pgfsetbuttcap%
\pgfsetroundjoin%
\definecolor{currentfill}{rgb}{0.000000,0.501961,0.000000}%
\pgfsetfillcolor{currentfill}%
\pgfsetfillopacity{0.500000}%
\pgfsetlinewidth{0.803000pt}%
\definecolor{currentstroke}{rgb}{0.000000,0.501961,0.000000}%
\pgfsetstrokecolor{currentstroke}%
\pgfsetstrokeopacity{0.500000}%
\pgfsetdash{}{0pt}%
\pgfpathmoveto{\pgfqpoint{2.170064in}{0.704089in}}%
\pgfpathlineto{\pgfqpoint{2.170064in}{0.703782in}}%
\pgfpathlineto{\pgfqpoint{2.410774in}{0.704054in}}%
\pgfpathlineto{\pgfqpoint{2.522305in}{0.704322in}}%
\pgfpathlineto{\pgfqpoint{2.595701in}{0.704589in}}%
\pgfpathlineto{\pgfqpoint{2.650497in}{0.704852in}}%
\pgfpathlineto{\pgfqpoint{2.694239in}{0.705112in}}%
\pgfpathlineto{\pgfqpoint{2.730647in}{0.705370in}}%
\pgfpathlineto{\pgfqpoint{2.761831in}{0.705624in}}%
\pgfpathlineto{\pgfqpoint{2.789104in}{0.705876in}}%
\pgfpathlineto{\pgfqpoint{2.813338in}{0.706125in}}%
\pgfpathlineto{\pgfqpoint{2.835143in}{0.706370in}}%
\pgfpathlineto{\pgfqpoint{2.854962in}{0.706610in}}%
\pgfpathlineto{\pgfqpoint{2.873127in}{0.706838in}}%
\pgfpathlineto{\pgfqpoint{2.889892in}{0.707066in}}%
\pgfpathlineto{\pgfqpoint{2.905459in}{0.707293in}}%
\pgfpathlineto{\pgfqpoint{2.919986in}{0.707518in}}%
\pgfpathlineto{\pgfqpoint{2.933605in}{0.707742in}}%
\pgfpathlineto{\pgfqpoint{2.946422in}{0.707964in}}%
\pgfpathlineto{\pgfqpoint{2.958526in}{0.708184in}}%
\pgfpathlineto{\pgfqpoint{2.969993in}{0.708403in}}%
\pgfpathlineto{\pgfqpoint{2.980886in}{0.708619in}}%
\pgfpathlineto{\pgfqpoint{2.991260in}{0.708833in}}%
\pgfpathlineto{\pgfqpoint{3.001162in}{0.709044in}}%
\pgfpathlineto{\pgfqpoint{3.010633in}{0.709252in}}%
\pgfpathlineto{\pgfqpoint{3.019710in}{0.709457in}}%
\pgfpathlineto{\pgfqpoint{3.028423in}{0.709653in}}%
\pgfpathlineto{\pgfqpoint{3.036801in}{0.709843in}}%
\pgfpathlineto{\pgfqpoint{3.044869in}{0.710028in}}%
\pgfpathlineto{\pgfqpoint{3.052648in}{0.710210in}}%
\pgfpathlineto{\pgfqpoint{3.060159in}{0.710388in}}%
\pgfpathlineto{\pgfqpoint{3.067420in}{0.710562in}}%
\pgfpathlineto{\pgfqpoint{3.074446in}{0.710731in}}%
\pgfpathlineto{\pgfqpoint{3.081253in}{0.710896in}}%
\pgfpathlineto{\pgfqpoint{3.087853in}{0.711057in}}%
\pgfpathlineto{\pgfqpoint{3.094259in}{0.711214in}}%
\pgfpathlineto{\pgfqpoint{3.100482in}{0.711366in}}%
\pgfpathlineto{\pgfqpoint{3.106532in}{0.711514in}}%
\pgfpathlineto{\pgfqpoint{3.112419in}{0.711657in}}%
\pgfpathlineto{\pgfqpoint{3.118150in}{0.711795in}}%
\pgfpathlineto{\pgfqpoint{3.123735in}{0.711929in}}%
\pgfpathlineto{\pgfqpoint{3.129180in}{0.712059in}}%
\pgfpathlineto{\pgfqpoint{3.134492in}{0.712183in}}%
\pgfpathlineto{\pgfqpoint{3.139677in}{0.712303in}}%
\pgfpathlineto{\pgfqpoint{3.144742in}{0.712418in}}%
\pgfpathlineto{\pgfqpoint{3.149692in}{0.712528in}}%
\pgfpathlineto{\pgfqpoint{3.154532in}{0.712632in}}%
\pgfpathlineto{\pgfqpoint{3.159267in}{0.712732in}}%
\pgfpathlineto{\pgfqpoint{3.163901in}{0.712827in}}%
\pgfpathlineto{\pgfqpoint{3.168438in}{0.712916in}}%
\pgfpathlineto{\pgfqpoint{3.172883in}{0.712984in}}%
\pgfpathlineto{\pgfqpoint{3.172883in}{0.713001in}}%
\pgfpathlineto{\pgfqpoint{3.172883in}{0.713001in}}%
\pgfpathlineto{\pgfqpoint{3.168438in}{0.712925in}}%
\pgfpathlineto{\pgfqpoint{3.163901in}{0.712856in}}%
\pgfpathlineto{\pgfqpoint{3.159267in}{0.712780in}}%
\pgfpathlineto{\pgfqpoint{3.154532in}{0.712695in}}%
\pgfpathlineto{\pgfqpoint{3.149692in}{0.712603in}}%
\pgfpathlineto{\pgfqpoint{3.144742in}{0.712503in}}%
\pgfpathlineto{\pgfqpoint{3.139677in}{0.712395in}}%
\pgfpathlineto{\pgfqpoint{3.134492in}{0.712281in}}%
\pgfpathlineto{\pgfqpoint{3.129180in}{0.712159in}}%
\pgfpathlineto{\pgfqpoint{3.123735in}{0.712031in}}%
\pgfpathlineto{\pgfqpoint{3.118150in}{0.711897in}}%
\pgfpathlineto{\pgfqpoint{3.112419in}{0.711756in}}%
\pgfpathlineto{\pgfqpoint{3.106532in}{0.711609in}}%
\pgfpathlineto{\pgfqpoint{3.100482in}{0.711456in}}%
\pgfpathlineto{\pgfqpoint{3.094259in}{0.711298in}}%
\pgfpathlineto{\pgfqpoint{3.087853in}{0.711135in}}%
\pgfpathlineto{\pgfqpoint{3.081253in}{0.710966in}}%
\pgfpathlineto{\pgfqpoint{3.074446in}{0.710792in}}%
\pgfpathlineto{\pgfqpoint{3.067420in}{0.710614in}}%
\pgfpathlineto{\pgfqpoint{3.060159in}{0.710431in}}%
\pgfpathlineto{\pgfqpoint{3.052648in}{0.710244in}}%
\pgfpathlineto{\pgfqpoint{3.044869in}{0.710053in}}%
\pgfpathlineto{\pgfqpoint{3.036801in}{0.709858in}}%
\pgfpathlineto{\pgfqpoint{3.028423in}{0.709659in}}%
\pgfpathlineto{\pgfqpoint{3.019710in}{0.709459in}}%
\pgfpathlineto{\pgfqpoint{3.010633in}{0.709262in}}%
\pgfpathlineto{\pgfqpoint{3.001162in}{0.709061in}}%
\pgfpathlineto{\pgfqpoint{2.991260in}{0.708856in}}%
\pgfpathlineto{\pgfqpoint{2.980886in}{0.708647in}}%
\pgfpathlineto{\pgfqpoint{2.969993in}{0.708435in}}%
\pgfpathlineto{\pgfqpoint{2.958526in}{0.708219in}}%
\pgfpathlineto{\pgfqpoint{2.946422in}{0.708000in}}%
\pgfpathlineto{\pgfqpoint{2.933605in}{0.707777in}}%
\pgfpathlineto{\pgfqpoint{2.919986in}{0.707551in}}%
\pgfpathlineto{\pgfqpoint{2.905459in}{0.707321in}}%
\pgfpathlineto{\pgfqpoint{2.889892in}{0.707088in}}%
\pgfpathlineto{\pgfqpoint{2.873127in}{0.706852in}}%
\pgfpathlineto{\pgfqpoint{2.854962in}{0.706613in}}%
\pgfpathlineto{\pgfqpoint{2.835143in}{0.706381in}}%
\pgfpathlineto{\pgfqpoint{2.813338in}{0.706151in}}%
\pgfpathlineto{\pgfqpoint{2.789104in}{0.705921in}}%
\pgfpathlineto{\pgfqpoint{2.761831in}{0.705691in}}%
\pgfpathlineto{\pgfqpoint{2.730647in}{0.705460in}}%
\pgfpathlineto{\pgfqpoint{2.694239in}{0.705230in}}%
\pgfpathlineto{\pgfqpoint{2.650497in}{0.705001in}}%
\pgfpathlineto{\pgfqpoint{2.595701in}{0.704772in}}%
\pgfpathlineto{\pgfqpoint{2.522305in}{0.704543in}}%
\pgfpathlineto{\pgfqpoint{2.410774in}{0.704316in}}%
\pgfpathlineto{\pgfqpoint{2.170064in}{0.704089in}}%
\pgfpathclose%
\pgfusepath{stroke,fill}%
\end{pgfscope}%
\begin{pgfscope}%
\pgfpathrectangle{\pgfqpoint{2.170064in}{0.435433in}}{\pgfqpoint{1.223103in}{0.607948in}}%
\pgfusepath{clip}%
\pgfsetroundcap%
\pgfsetroundjoin%
\pgfsetlinewidth{0.501875pt}%
\definecolor{currentstroke}{rgb}{0.000000,0.000000,1.000000}%
\pgfsetstrokecolor{currentstroke}%
\pgfsetstrokeopacity{0.800000}%
\pgfsetdash{}{0pt}%
\pgfpathmoveto{\pgfqpoint{2.170064in}{0.703812in}}%
\pgfpathlineto{\pgfqpoint{3.393168in}{0.703812in}}%
\pgfusepath{stroke}%
\end{pgfscope}%
\begin{pgfscope}%
\pgfpathrectangle{\pgfqpoint{2.170064in}{0.435433in}}{\pgfqpoint{1.223103in}{0.607948in}}%
\pgfusepath{clip}%
\pgfsetbuttcap%
\pgfsetroundjoin%
\pgfsetlinewidth{1.003750pt}%
\definecolor{currentstroke}{rgb}{0.000000,0.000000,0.000000}%
\pgfsetstrokecolor{currentstroke}%
\pgfsetdash{{3.700000pt}{1.600000pt}}{0.000000pt}%
\pgfpathmoveto{\pgfqpoint{2.170064in}{0.703897in}}%
\pgfpathlineto{\pgfqpoint{3.393168in}{0.703897in}}%
\pgfusepath{stroke}%
\end{pgfscope}%
\begin{pgfscope}%
\pgfsetroundcap%
\pgfsetroundjoin%
\pgfsetlinewidth{0.501875pt}%
\definecolor{currentstroke}{rgb}{0.000000,0.000000,1.000000}%
\pgfsetstrokecolor{currentstroke}%
\pgfsetstrokeopacity{0.800000}%
\pgfsetdash{}{0pt}%
\pgfpathmoveto{\pgfqpoint{2.997144in}{0.821916in}}%
\pgfpathquadraticcurveto{\pgfqpoint{2.921280in}{0.770638in}}{\pgfqpoint{2.845417in}{0.719361in}}%
\pgfusepath{stroke}%
\end{pgfscope}%
\begin{pgfscope}%
\pgfsetfillopacity{0.800000}%
\pgfsetstrokeopacity{0.800000}%
\definecolor{textcolor}{rgb}{0.000000,0.000000,1.000000}%
\pgfsetstrokecolor{textcolor}%
\pgfsetfillcolor{textcolor}%
\pgftext[x=2.910706in,y=0.886196in,left,base]{\color{textcolor}\sffamily\fontsize{5.647059}{6.776471}\selectfont 1.44145(27)}%
\end{pgfscope}%
\begin{pgfscope}%
\pgfsetbuttcap%
\pgfsetroundjoin%
\definecolor{currentfill}{rgb}{0.150000,0.150000,0.150000}%
\pgfsetfillcolor{currentfill}%
\pgfsetlinewidth{1.003750pt}%
\definecolor{currentstroke}{rgb}{0.150000,0.150000,0.150000}%
\pgfsetstrokecolor{currentstroke}%
\pgfsetdash{}{0pt}%
\pgfsys@defobject{currentmarker}{\pgfqpoint{0.000000in}{-0.066667in}}{\pgfqpoint{0.000000in}{0.000000in}}{%
\pgfpathmoveto{\pgfqpoint{0.000000in}{0.000000in}}%
\pgfpathlineto{\pgfqpoint{0.000000in}{-0.066667in}}%
\pgfusepath{stroke,fill}%
}%
\begin{pgfscope}%
\pgfsys@transformshift{2.170064in}{0.435433in}%
\pgfsys@useobject{currentmarker}{}%
\end{pgfscope}%
\end{pgfscope}%
\begin{pgfscope}%
\definecolor{textcolor}{rgb}{0.150000,0.150000,0.150000}%
\pgfsetstrokecolor{textcolor}%
\pgfsetfillcolor{textcolor}%
\pgftext[x=2.170064in,y=0.320155in,,top]{\color{textcolor}\sffamily\fontsize{5.176471}{6.211765}\selectfont \(\displaystyle {10^{-3}}\)}%
\end{pgfscope}%
\begin{pgfscope}%
\pgfsetbuttcap%
\pgfsetroundjoin%
\definecolor{currentfill}{rgb}{0.150000,0.150000,0.150000}%
\pgfsetfillcolor{currentfill}%
\pgfsetlinewidth{1.003750pt}%
\definecolor{currentstroke}{rgb}{0.150000,0.150000,0.150000}%
\pgfsetstrokecolor{currentstroke}%
\pgfsetdash{}{0pt}%
\pgfsys@defobject{currentmarker}{\pgfqpoint{0.000000in}{-0.066667in}}{\pgfqpoint{0.000000in}{0.000000in}}{%
\pgfpathmoveto{\pgfqpoint{0.000000in}{0.000000in}}%
\pgfpathlineto{\pgfqpoint{0.000000in}{-0.066667in}}%
\pgfusepath{stroke,fill}%
}%
\begin{pgfscope}%
\pgfsys@transformshift{2.671473in}{0.435433in}%
\pgfsys@useobject{currentmarker}{}%
\end{pgfscope}%
\end{pgfscope}%
\begin{pgfscope}%
\definecolor{textcolor}{rgb}{0.150000,0.150000,0.150000}%
\pgfsetstrokecolor{textcolor}%
\pgfsetfillcolor{textcolor}%
\pgftext[x=2.671473in,y=0.320155in,,top]{\color{textcolor}\sffamily\fontsize{5.176471}{6.211765}\selectfont \(\displaystyle {10^{-2}}\)}%
\end{pgfscope}%
\begin{pgfscope}%
\pgfsetbuttcap%
\pgfsetroundjoin%
\definecolor{currentfill}{rgb}{0.150000,0.150000,0.150000}%
\pgfsetfillcolor{currentfill}%
\pgfsetlinewidth{1.003750pt}%
\definecolor{currentstroke}{rgb}{0.150000,0.150000,0.150000}%
\pgfsetstrokecolor{currentstroke}%
\pgfsetdash{}{0pt}%
\pgfsys@defobject{currentmarker}{\pgfqpoint{0.000000in}{-0.066667in}}{\pgfqpoint{0.000000in}{0.000000in}}{%
\pgfpathmoveto{\pgfqpoint{0.000000in}{0.000000in}}%
\pgfpathlineto{\pgfqpoint{0.000000in}{-0.066667in}}%
\pgfusepath{stroke,fill}%
}%
\begin{pgfscope}%
\pgfsys@transformshift{3.172883in}{0.435433in}%
\pgfsys@useobject{currentmarker}{}%
\end{pgfscope}%
\end{pgfscope}%
\begin{pgfscope}%
\definecolor{textcolor}{rgb}{0.150000,0.150000,0.150000}%
\pgfsetstrokecolor{textcolor}%
\pgfsetfillcolor{textcolor}%
\pgftext[x=3.172883in,y=0.320155in,,top]{\color{textcolor}\sffamily\fontsize{5.176471}{6.211765}\selectfont \(\displaystyle {10^{-1}}\)}%
\end{pgfscope}%
\begin{pgfscope}%
\pgfsetbuttcap%
\pgfsetroundjoin%
\definecolor{currentfill}{rgb}{0.150000,0.150000,0.150000}%
\pgfsetfillcolor{currentfill}%
\pgfsetlinewidth{0.803000pt}%
\definecolor{currentstroke}{rgb}{0.150000,0.150000,0.150000}%
\pgfsetstrokecolor{currentstroke}%
\pgfsetdash{}{0pt}%
\pgfsys@defobject{currentmarker}{\pgfqpoint{0.000000in}{-0.044444in}}{\pgfqpoint{0.000000in}{0.000000in}}{%
\pgfpathmoveto{\pgfqpoint{0.000000in}{0.000000in}}%
\pgfpathlineto{\pgfqpoint{0.000000in}{-0.044444in}}%
\pgfusepath{stroke,fill}%
}%
\begin{pgfscope}%
\pgfsys@transformshift{2.321004in}{0.435433in}%
\pgfsys@useobject{currentmarker}{}%
\end{pgfscope}%
\end{pgfscope}%
\begin{pgfscope}%
\pgfsetbuttcap%
\pgfsetroundjoin%
\definecolor{currentfill}{rgb}{0.150000,0.150000,0.150000}%
\pgfsetfillcolor{currentfill}%
\pgfsetlinewidth{0.803000pt}%
\definecolor{currentstroke}{rgb}{0.150000,0.150000,0.150000}%
\pgfsetstrokecolor{currentstroke}%
\pgfsetdash{}{0pt}%
\pgfsys@defobject{currentmarker}{\pgfqpoint{0.000000in}{-0.044444in}}{\pgfqpoint{0.000000in}{0.000000in}}{%
\pgfpathmoveto{\pgfqpoint{0.000000in}{0.000000in}}%
\pgfpathlineto{\pgfqpoint{0.000000in}{-0.044444in}}%
\pgfusepath{stroke,fill}%
}%
\begin{pgfscope}%
\pgfsys@transformshift{2.409297in}{0.435433in}%
\pgfsys@useobject{currentmarker}{}%
\end{pgfscope}%
\end{pgfscope}%
\begin{pgfscope}%
\pgfsetbuttcap%
\pgfsetroundjoin%
\definecolor{currentfill}{rgb}{0.150000,0.150000,0.150000}%
\pgfsetfillcolor{currentfill}%
\pgfsetlinewidth{0.803000pt}%
\definecolor{currentstroke}{rgb}{0.150000,0.150000,0.150000}%
\pgfsetstrokecolor{currentstroke}%
\pgfsetdash{}{0pt}%
\pgfsys@defobject{currentmarker}{\pgfqpoint{0.000000in}{-0.044444in}}{\pgfqpoint{0.000000in}{0.000000in}}{%
\pgfpathmoveto{\pgfqpoint{0.000000in}{0.000000in}}%
\pgfpathlineto{\pgfqpoint{0.000000in}{-0.044444in}}%
\pgfusepath{stroke,fill}%
}%
\begin{pgfscope}%
\pgfsys@transformshift{2.471943in}{0.435433in}%
\pgfsys@useobject{currentmarker}{}%
\end{pgfscope}%
\end{pgfscope}%
\begin{pgfscope}%
\pgfsetbuttcap%
\pgfsetroundjoin%
\definecolor{currentfill}{rgb}{0.150000,0.150000,0.150000}%
\pgfsetfillcolor{currentfill}%
\pgfsetlinewidth{0.803000pt}%
\definecolor{currentstroke}{rgb}{0.150000,0.150000,0.150000}%
\pgfsetstrokecolor{currentstroke}%
\pgfsetdash{}{0pt}%
\pgfsys@defobject{currentmarker}{\pgfqpoint{0.000000in}{-0.044444in}}{\pgfqpoint{0.000000in}{0.000000in}}{%
\pgfpathmoveto{\pgfqpoint{0.000000in}{0.000000in}}%
\pgfpathlineto{\pgfqpoint{0.000000in}{-0.044444in}}%
\pgfusepath{stroke,fill}%
}%
\begin{pgfscope}%
\pgfsys@transformshift{2.520534in}{0.435433in}%
\pgfsys@useobject{currentmarker}{}%
\end{pgfscope}%
\end{pgfscope}%
\begin{pgfscope}%
\pgfsetbuttcap%
\pgfsetroundjoin%
\definecolor{currentfill}{rgb}{0.150000,0.150000,0.150000}%
\pgfsetfillcolor{currentfill}%
\pgfsetlinewidth{0.803000pt}%
\definecolor{currentstroke}{rgb}{0.150000,0.150000,0.150000}%
\pgfsetstrokecolor{currentstroke}%
\pgfsetdash{}{0pt}%
\pgfsys@defobject{currentmarker}{\pgfqpoint{0.000000in}{-0.044444in}}{\pgfqpoint{0.000000in}{0.000000in}}{%
\pgfpathmoveto{\pgfqpoint{0.000000in}{0.000000in}}%
\pgfpathlineto{\pgfqpoint{0.000000in}{-0.044444in}}%
\pgfusepath{stroke,fill}%
}%
\begin{pgfscope}%
\pgfsys@transformshift{2.560237in}{0.435433in}%
\pgfsys@useobject{currentmarker}{}%
\end{pgfscope}%
\end{pgfscope}%
\begin{pgfscope}%
\pgfsetbuttcap%
\pgfsetroundjoin%
\definecolor{currentfill}{rgb}{0.150000,0.150000,0.150000}%
\pgfsetfillcolor{currentfill}%
\pgfsetlinewidth{0.803000pt}%
\definecolor{currentstroke}{rgb}{0.150000,0.150000,0.150000}%
\pgfsetstrokecolor{currentstroke}%
\pgfsetdash{}{0pt}%
\pgfsys@defobject{currentmarker}{\pgfqpoint{0.000000in}{-0.044444in}}{\pgfqpoint{0.000000in}{0.000000in}}{%
\pgfpathmoveto{\pgfqpoint{0.000000in}{0.000000in}}%
\pgfpathlineto{\pgfqpoint{0.000000in}{-0.044444in}}%
\pgfusepath{stroke,fill}%
}%
\begin{pgfscope}%
\pgfsys@transformshift{2.593804in}{0.435433in}%
\pgfsys@useobject{currentmarker}{}%
\end{pgfscope}%
\end{pgfscope}%
\begin{pgfscope}%
\pgfsetbuttcap%
\pgfsetroundjoin%
\definecolor{currentfill}{rgb}{0.150000,0.150000,0.150000}%
\pgfsetfillcolor{currentfill}%
\pgfsetlinewidth{0.803000pt}%
\definecolor{currentstroke}{rgb}{0.150000,0.150000,0.150000}%
\pgfsetstrokecolor{currentstroke}%
\pgfsetdash{}{0pt}%
\pgfsys@defobject{currentmarker}{\pgfqpoint{0.000000in}{-0.044444in}}{\pgfqpoint{0.000000in}{0.000000in}}{%
\pgfpathmoveto{\pgfqpoint{0.000000in}{0.000000in}}%
\pgfpathlineto{\pgfqpoint{0.000000in}{-0.044444in}}%
\pgfusepath{stroke,fill}%
}%
\begin{pgfscope}%
\pgfsys@transformshift{2.622882in}{0.435433in}%
\pgfsys@useobject{currentmarker}{}%
\end{pgfscope}%
\end{pgfscope}%
\begin{pgfscope}%
\pgfsetbuttcap%
\pgfsetroundjoin%
\definecolor{currentfill}{rgb}{0.150000,0.150000,0.150000}%
\pgfsetfillcolor{currentfill}%
\pgfsetlinewidth{0.803000pt}%
\definecolor{currentstroke}{rgb}{0.150000,0.150000,0.150000}%
\pgfsetstrokecolor{currentstroke}%
\pgfsetdash{}{0pt}%
\pgfsys@defobject{currentmarker}{\pgfqpoint{0.000000in}{-0.044444in}}{\pgfqpoint{0.000000in}{0.000000in}}{%
\pgfpathmoveto{\pgfqpoint{0.000000in}{0.000000in}}%
\pgfpathlineto{\pgfqpoint{0.000000in}{-0.044444in}}%
\pgfusepath{stroke,fill}%
}%
\begin{pgfscope}%
\pgfsys@transformshift{2.648530in}{0.435433in}%
\pgfsys@useobject{currentmarker}{}%
\end{pgfscope}%
\end{pgfscope}%
\begin{pgfscope}%
\pgfsetbuttcap%
\pgfsetroundjoin%
\definecolor{currentfill}{rgb}{0.150000,0.150000,0.150000}%
\pgfsetfillcolor{currentfill}%
\pgfsetlinewidth{0.803000pt}%
\definecolor{currentstroke}{rgb}{0.150000,0.150000,0.150000}%
\pgfsetstrokecolor{currentstroke}%
\pgfsetdash{}{0pt}%
\pgfsys@defobject{currentmarker}{\pgfqpoint{0.000000in}{-0.044444in}}{\pgfqpoint{0.000000in}{0.000000in}}{%
\pgfpathmoveto{\pgfqpoint{0.000000in}{0.000000in}}%
\pgfpathlineto{\pgfqpoint{0.000000in}{-0.044444in}}%
\pgfusepath{stroke,fill}%
}%
\begin{pgfscope}%
\pgfsys@transformshift{2.822413in}{0.435433in}%
\pgfsys@useobject{currentmarker}{}%
\end{pgfscope}%
\end{pgfscope}%
\begin{pgfscope}%
\pgfsetbuttcap%
\pgfsetroundjoin%
\definecolor{currentfill}{rgb}{0.150000,0.150000,0.150000}%
\pgfsetfillcolor{currentfill}%
\pgfsetlinewidth{0.803000pt}%
\definecolor{currentstroke}{rgb}{0.150000,0.150000,0.150000}%
\pgfsetstrokecolor{currentstroke}%
\pgfsetdash{}{0pt}%
\pgfsys@defobject{currentmarker}{\pgfqpoint{0.000000in}{-0.044444in}}{\pgfqpoint{0.000000in}{0.000000in}}{%
\pgfpathmoveto{\pgfqpoint{0.000000in}{0.000000in}}%
\pgfpathlineto{\pgfqpoint{0.000000in}{-0.044444in}}%
\pgfusepath{stroke,fill}%
}%
\begin{pgfscope}%
\pgfsys@transformshift{2.910706in}{0.435433in}%
\pgfsys@useobject{currentmarker}{}%
\end{pgfscope}%
\end{pgfscope}%
\begin{pgfscope}%
\pgfsetbuttcap%
\pgfsetroundjoin%
\definecolor{currentfill}{rgb}{0.150000,0.150000,0.150000}%
\pgfsetfillcolor{currentfill}%
\pgfsetlinewidth{0.803000pt}%
\definecolor{currentstroke}{rgb}{0.150000,0.150000,0.150000}%
\pgfsetstrokecolor{currentstroke}%
\pgfsetdash{}{0pt}%
\pgfsys@defobject{currentmarker}{\pgfqpoint{0.000000in}{-0.044444in}}{\pgfqpoint{0.000000in}{0.000000in}}{%
\pgfpathmoveto{\pgfqpoint{0.000000in}{0.000000in}}%
\pgfpathlineto{\pgfqpoint{0.000000in}{-0.044444in}}%
\pgfusepath{stroke,fill}%
}%
\begin{pgfscope}%
\pgfsys@transformshift{2.973352in}{0.435433in}%
\pgfsys@useobject{currentmarker}{}%
\end{pgfscope}%
\end{pgfscope}%
\begin{pgfscope}%
\pgfsetbuttcap%
\pgfsetroundjoin%
\definecolor{currentfill}{rgb}{0.150000,0.150000,0.150000}%
\pgfsetfillcolor{currentfill}%
\pgfsetlinewidth{0.803000pt}%
\definecolor{currentstroke}{rgb}{0.150000,0.150000,0.150000}%
\pgfsetstrokecolor{currentstroke}%
\pgfsetdash{}{0pt}%
\pgfsys@defobject{currentmarker}{\pgfqpoint{0.000000in}{-0.044444in}}{\pgfqpoint{0.000000in}{0.000000in}}{%
\pgfpathmoveto{\pgfqpoint{0.000000in}{0.000000in}}%
\pgfpathlineto{\pgfqpoint{0.000000in}{-0.044444in}}%
\pgfusepath{stroke,fill}%
}%
\begin{pgfscope}%
\pgfsys@transformshift{3.021943in}{0.435433in}%
\pgfsys@useobject{currentmarker}{}%
\end{pgfscope}%
\end{pgfscope}%
\begin{pgfscope}%
\pgfsetbuttcap%
\pgfsetroundjoin%
\definecolor{currentfill}{rgb}{0.150000,0.150000,0.150000}%
\pgfsetfillcolor{currentfill}%
\pgfsetlinewidth{0.803000pt}%
\definecolor{currentstroke}{rgb}{0.150000,0.150000,0.150000}%
\pgfsetstrokecolor{currentstroke}%
\pgfsetdash{}{0pt}%
\pgfsys@defobject{currentmarker}{\pgfqpoint{0.000000in}{-0.044444in}}{\pgfqpoint{0.000000in}{0.000000in}}{%
\pgfpathmoveto{\pgfqpoint{0.000000in}{0.000000in}}%
\pgfpathlineto{\pgfqpoint{0.000000in}{-0.044444in}}%
\pgfusepath{stroke,fill}%
}%
\begin{pgfscope}%
\pgfsys@transformshift{3.061646in}{0.435433in}%
\pgfsys@useobject{currentmarker}{}%
\end{pgfscope}%
\end{pgfscope}%
\begin{pgfscope}%
\pgfsetbuttcap%
\pgfsetroundjoin%
\definecolor{currentfill}{rgb}{0.150000,0.150000,0.150000}%
\pgfsetfillcolor{currentfill}%
\pgfsetlinewidth{0.803000pt}%
\definecolor{currentstroke}{rgb}{0.150000,0.150000,0.150000}%
\pgfsetstrokecolor{currentstroke}%
\pgfsetdash{}{0pt}%
\pgfsys@defobject{currentmarker}{\pgfqpoint{0.000000in}{-0.044444in}}{\pgfqpoint{0.000000in}{0.000000in}}{%
\pgfpathmoveto{\pgfqpoint{0.000000in}{0.000000in}}%
\pgfpathlineto{\pgfqpoint{0.000000in}{-0.044444in}}%
\pgfusepath{stroke,fill}%
}%
\begin{pgfscope}%
\pgfsys@transformshift{3.095213in}{0.435433in}%
\pgfsys@useobject{currentmarker}{}%
\end{pgfscope}%
\end{pgfscope}%
\begin{pgfscope}%
\pgfsetbuttcap%
\pgfsetroundjoin%
\definecolor{currentfill}{rgb}{0.150000,0.150000,0.150000}%
\pgfsetfillcolor{currentfill}%
\pgfsetlinewidth{0.803000pt}%
\definecolor{currentstroke}{rgb}{0.150000,0.150000,0.150000}%
\pgfsetstrokecolor{currentstroke}%
\pgfsetdash{}{0pt}%
\pgfsys@defobject{currentmarker}{\pgfqpoint{0.000000in}{-0.044444in}}{\pgfqpoint{0.000000in}{0.000000in}}{%
\pgfpathmoveto{\pgfqpoint{0.000000in}{0.000000in}}%
\pgfpathlineto{\pgfqpoint{0.000000in}{-0.044444in}}%
\pgfusepath{stroke,fill}%
}%
\begin{pgfscope}%
\pgfsys@transformshift{3.124291in}{0.435433in}%
\pgfsys@useobject{currentmarker}{}%
\end{pgfscope}%
\end{pgfscope}%
\begin{pgfscope}%
\pgfsetbuttcap%
\pgfsetroundjoin%
\definecolor{currentfill}{rgb}{0.150000,0.150000,0.150000}%
\pgfsetfillcolor{currentfill}%
\pgfsetlinewidth{0.803000pt}%
\definecolor{currentstroke}{rgb}{0.150000,0.150000,0.150000}%
\pgfsetstrokecolor{currentstroke}%
\pgfsetdash{}{0pt}%
\pgfsys@defobject{currentmarker}{\pgfqpoint{0.000000in}{-0.044444in}}{\pgfqpoint{0.000000in}{0.000000in}}{%
\pgfpathmoveto{\pgfqpoint{0.000000in}{0.000000in}}%
\pgfpathlineto{\pgfqpoint{0.000000in}{-0.044444in}}%
\pgfusepath{stroke,fill}%
}%
\begin{pgfscope}%
\pgfsys@transformshift{3.149939in}{0.435433in}%
\pgfsys@useobject{currentmarker}{}%
\end{pgfscope}%
\end{pgfscope}%
\begin{pgfscope}%
\pgfsetbuttcap%
\pgfsetroundjoin%
\definecolor{currentfill}{rgb}{0.150000,0.150000,0.150000}%
\pgfsetfillcolor{currentfill}%
\pgfsetlinewidth{0.803000pt}%
\definecolor{currentstroke}{rgb}{0.150000,0.150000,0.150000}%
\pgfsetstrokecolor{currentstroke}%
\pgfsetdash{}{0pt}%
\pgfsys@defobject{currentmarker}{\pgfqpoint{0.000000in}{-0.044444in}}{\pgfqpoint{0.000000in}{0.000000in}}{%
\pgfpathmoveto{\pgfqpoint{0.000000in}{0.000000in}}%
\pgfpathlineto{\pgfqpoint{0.000000in}{-0.044444in}}%
\pgfusepath{stroke,fill}%
}%
\begin{pgfscope}%
\pgfsys@transformshift{3.323822in}{0.435433in}%
\pgfsys@useobject{currentmarker}{}%
\end{pgfscope}%
\end{pgfscope}%
\begin{pgfscope}%
\definecolor{textcolor}{rgb}{0.150000,0.150000,0.150000}%
\pgfsetstrokecolor{textcolor}%
\pgfsetfillcolor{textcolor}%
\pgftext[x=2.781616in,y=0.183333in,,top]{\color{textcolor}\sffamily\fontsize{5.647059}{6.776471}\selectfont \(\displaystyle \epsilon [\mathrm{fm}]\)}%
\end{pgfscope}%
\begin{pgfscope}%
\pgfsetbuttcap%
\pgfsetroundjoin%
\definecolor{currentfill}{rgb}{0.150000,0.150000,0.150000}%
\pgfsetfillcolor{currentfill}%
\pgfsetlinewidth{1.003750pt}%
\definecolor{currentstroke}{rgb}{0.150000,0.150000,0.150000}%
\pgfsetstrokecolor{currentstroke}%
\pgfsetdash{}{0pt}%
\pgfsys@defobject{currentmarker}{\pgfqpoint{-0.066667in}{0.000000in}}{\pgfqpoint{0.000000in}{0.000000in}}{%
\pgfpathmoveto{\pgfqpoint{0.000000in}{0.000000in}}%
\pgfpathlineto{\pgfqpoint{-0.066667in}{0.000000in}}%
\pgfusepath{stroke,fill}%
}%
\begin{pgfscope}%
\pgfsys@transformshift{2.170064in}{0.435433in}%
\pgfsys@useobject{currentmarker}{}%
\end{pgfscope}%
\end{pgfscope}%
\begin{pgfscope}%
\pgfsetbuttcap%
\pgfsetroundjoin%
\definecolor{currentfill}{rgb}{0.150000,0.150000,0.150000}%
\pgfsetfillcolor{currentfill}%
\pgfsetlinewidth{1.003750pt}%
\definecolor{currentstroke}{rgb}{0.150000,0.150000,0.150000}%
\pgfsetstrokecolor{currentstroke}%
\pgfsetdash{}{0pt}%
\pgfsys@defobject{currentmarker}{\pgfqpoint{-0.066667in}{0.000000in}}{\pgfqpoint{0.000000in}{0.000000in}}{%
\pgfpathmoveto{\pgfqpoint{0.000000in}{0.000000in}}%
\pgfpathlineto{\pgfqpoint{-0.066667in}{0.000000in}}%
\pgfusepath{stroke,fill}%
}%
\begin{pgfscope}%
\pgfsys@transformshift{2.170064in}{0.703897in}%
\pgfsys@useobject{currentmarker}{}%
\end{pgfscope}%
\end{pgfscope}%
\begin{pgfscope}%
\pgfsetbuttcap%
\pgfsetroundjoin%
\definecolor{currentfill}{rgb}{0.150000,0.150000,0.150000}%
\pgfsetfillcolor{currentfill}%
\pgfsetlinewidth{1.003750pt}%
\definecolor{currentstroke}{rgb}{0.150000,0.150000,0.150000}%
\pgfsetstrokecolor{currentstroke}%
\pgfsetdash{}{0pt}%
\pgfsys@defobject{currentmarker}{\pgfqpoint{-0.066667in}{0.000000in}}{\pgfqpoint{0.000000in}{0.000000in}}{%
\pgfpathmoveto{\pgfqpoint{0.000000in}{0.000000in}}%
\pgfpathlineto{\pgfqpoint{-0.066667in}{0.000000in}}%
\pgfusepath{stroke,fill}%
}%
\begin{pgfscope}%
\pgfsys@transformshift{2.170064in}{1.043381in}%
\pgfsys@useobject{currentmarker}{}%
\end{pgfscope}%
\end{pgfscope}%
\begin{pgfscope}%
\pgfpathrectangle{\pgfqpoint{2.170064in}{0.435433in}}{\pgfqpoint{1.223103in}{0.607948in}}%
\pgfusepath{clip}%
\pgfsetroundcap%
\pgfsetroundjoin%
\pgfsetlinewidth{1.204500pt}%
\definecolor{currentstroke}{rgb}{0.000000,0.501961,0.000000}%
\pgfsetstrokecolor{currentstroke}%
\pgfsetdash{}{0pt}%
\pgfpathmoveto{\pgfqpoint{2.170064in}{0.703936in}}%
\pgfpathlineto{\pgfqpoint{2.410774in}{0.704185in}}%
\pgfpathlineto{\pgfqpoint{2.522305in}{0.704433in}}%
\pgfpathlineto{\pgfqpoint{2.595701in}{0.704680in}}%
\pgfpathlineto{\pgfqpoint{2.650497in}{0.704926in}}%
\pgfpathlineto{\pgfqpoint{2.694239in}{0.705171in}}%
\pgfpathlineto{\pgfqpoint{2.730647in}{0.705415in}}%
\pgfpathlineto{\pgfqpoint{2.761831in}{0.705657in}}%
\pgfpathlineto{\pgfqpoint{2.789104in}{0.705898in}}%
\pgfpathlineto{\pgfqpoint{2.813338in}{0.706138in}}%
\pgfpathlineto{\pgfqpoint{2.835143in}{0.706375in}}%
\pgfpathlineto{\pgfqpoint{2.854962in}{0.706611in}}%
\pgfpathlineto{\pgfqpoint{2.873127in}{0.706845in}}%
\pgfpathlineto{\pgfqpoint{2.889892in}{0.707077in}}%
\pgfpathlineto{\pgfqpoint{2.905459in}{0.707307in}}%
\pgfpathlineto{\pgfqpoint{2.919986in}{0.707534in}}%
\pgfpathlineto{\pgfqpoint{2.933605in}{0.707759in}}%
\pgfpathlineto{\pgfqpoint{2.946422in}{0.707982in}}%
\pgfpathlineto{\pgfqpoint{2.958526in}{0.708202in}}%
\pgfpathlineto{\pgfqpoint{2.969993in}{0.708419in}}%
\pgfpathlineto{\pgfqpoint{2.980886in}{0.708633in}}%
\pgfpathlineto{\pgfqpoint{2.991260in}{0.708844in}}%
\pgfpathlineto{\pgfqpoint{3.001162in}{0.709052in}}%
\pgfpathlineto{\pgfqpoint{3.010633in}{0.709257in}}%
\pgfpathlineto{\pgfqpoint{3.019710in}{0.709458in}}%
\pgfpathlineto{\pgfqpoint{3.028423in}{0.709656in}}%
\pgfpathlineto{\pgfqpoint{3.036801in}{0.709850in}}%
\pgfpathlineto{\pgfqpoint{3.044869in}{0.710041in}}%
\pgfpathlineto{\pgfqpoint{3.052648in}{0.710227in}}%
\pgfpathlineto{\pgfqpoint{3.060159in}{0.710410in}}%
\pgfpathlineto{\pgfqpoint{3.067420in}{0.710588in}}%
\pgfpathlineto{\pgfqpoint{3.074446in}{0.710762in}}%
\pgfpathlineto{\pgfqpoint{3.081253in}{0.710931in}}%
\pgfpathlineto{\pgfqpoint{3.087853in}{0.711096in}}%
\pgfpathlineto{\pgfqpoint{3.094259in}{0.711256in}}%
\pgfpathlineto{\pgfqpoint{3.100482in}{0.711411in}}%
\pgfpathlineto{\pgfqpoint{3.106532in}{0.711561in}}%
\pgfpathlineto{\pgfqpoint{3.112419in}{0.711706in}}%
\pgfpathlineto{\pgfqpoint{3.118150in}{0.711846in}}%
\pgfpathlineto{\pgfqpoint{3.123735in}{0.711980in}}%
\pgfpathlineto{\pgfqpoint{3.129180in}{0.712109in}}%
\pgfpathlineto{\pgfqpoint{3.134492in}{0.712232in}}%
\pgfpathlineto{\pgfqpoint{3.139677in}{0.712349in}}%
\pgfpathlineto{\pgfqpoint{3.144742in}{0.712460in}}%
\pgfpathlineto{\pgfqpoint{3.149692in}{0.712565in}}%
\pgfpathlineto{\pgfqpoint{3.154532in}{0.712664in}}%
\pgfpathlineto{\pgfqpoint{3.159267in}{0.712756in}}%
\pgfpathlineto{\pgfqpoint{3.163901in}{0.712842in}}%
\pgfpathlineto{\pgfqpoint{3.168438in}{0.712920in}}%
\pgfpathlineto{\pgfqpoint{3.172883in}{0.712992in}}%
\pgfusepath{stroke}%
\end{pgfscope}%
\begin{pgfscope}%
\pgfsetrectcap%
\pgfsetmiterjoin%
\pgfsetlinewidth{1.003750pt}%
\definecolor{currentstroke}{rgb}{0.150000,0.150000,0.150000}%
\pgfsetstrokecolor{currentstroke}%
\pgfsetdash{}{0pt}%
\pgfpathmoveto{\pgfqpoint{2.170064in}{0.435433in}}%
\pgfpathlineto{\pgfqpoint{2.170064in}{1.043381in}}%
\pgfusepath{stroke}%
\end{pgfscope}%
\begin{pgfscope}%
\pgfsetrectcap%
\pgfsetmiterjoin%
\pgfsetlinewidth{1.003750pt}%
\definecolor{currentstroke}{rgb}{0.150000,0.150000,0.150000}%
\pgfsetstrokecolor{currentstroke}%
\pgfsetdash{}{0pt}%
\pgfpathmoveto{\pgfqpoint{2.170064in}{0.435433in}}%
\pgfpathlineto{\pgfqpoint{3.393168in}{0.435433in}}%
\pgfusepath{stroke}%
\end{pgfscope}%
\begin{pgfscope}%
\pgfpathrectangle{\pgfqpoint{2.170064in}{0.435433in}}{\pgfqpoint{1.223103in}{0.607948in}}%
\pgfusepath{clip}%
\pgfsetbuttcap%
\pgfsetroundjoin%
\definecolor{currentfill}{rgb}{0.000000,0.000000,0.000000}%
\pgfsetfillcolor{currentfill}%
\pgfsetlinewidth{1.003750pt}%
\definecolor{currentstroke}{rgb}{0.000000,0.000000,0.000000}%
\pgfsetstrokecolor{currentstroke}%
\pgfsetdash{}{0pt}%
\pgfsys@defobject{currentmarker}{\pgfqpoint{-0.013889in}{-0.013889in}}{\pgfqpoint{0.013889in}{0.013889in}}{%
\pgfpathmoveto{\pgfqpoint{0.000000in}{-0.013889in}}%
\pgfpathcurveto{\pgfqpoint{0.003683in}{-0.013889in}}{\pgfqpoint{0.007216in}{-0.012425in}}{\pgfqpoint{0.009821in}{-0.009821in}}%
\pgfpathcurveto{\pgfqpoint{0.012425in}{-0.007216in}}{\pgfqpoint{0.013889in}{-0.003683in}}{\pgfqpoint{0.013889in}{0.000000in}}%
\pgfpathcurveto{\pgfqpoint{0.013889in}{0.003683in}}{\pgfqpoint{0.012425in}{0.007216in}}{\pgfqpoint{0.009821in}{0.009821in}}%
\pgfpathcurveto{\pgfqpoint{0.007216in}{0.012425in}}{\pgfqpoint{0.003683in}{0.013889in}}{\pgfqpoint{0.000000in}{0.013889in}}%
\pgfpathcurveto{\pgfqpoint{-0.003683in}{0.013889in}}{\pgfqpoint{-0.007216in}{0.012425in}}{\pgfqpoint{-0.009821in}{0.009821in}}%
\pgfpathcurveto{\pgfqpoint{-0.012425in}{0.007216in}}{\pgfqpoint{-0.013889in}{0.003683in}}{\pgfqpoint{-0.013889in}{0.000000in}}%
\pgfpathcurveto{\pgfqpoint{-0.013889in}{-0.003683in}}{\pgfqpoint{-0.012425in}{-0.007216in}}{\pgfqpoint{-0.009821in}{-0.009821in}}%
\pgfpathcurveto{\pgfqpoint{-0.007216in}{-0.012425in}}{\pgfqpoint{-0.003683in}{-0.013889in}}{\pgfqpoint{0.000000in}{-0.013889in}}%
\pgfpathclose%
\pgfusepath{stroke,fill}%
}%
\begin{pgfscope}%
\pgfsys@transformshift{2.822413in}{0.706239in}%
\pgfsys@useobject{currentmarker}{}%
\end{pgfscope}%
\begin{pgfscope}%
\pgfsys@transformshift{2.826812in}{0.706287in}%
\pgfsys@useobject{currentmarker}{}%
\end{pgfscope}%
\begin{pgfscope}%
\pgfsys@transformshift{2.831302in}{0.706336in}%
\pgfsys@useobject{currentmarker}{}%
\end{pgfscope}%
\begin{pgfscope}%
\pgfsys@transformshift{2.835887in}{0.706387in}%
\pgfsys@useobject{currentmarker}{}%
\end{pgfscope}%
\begin{pgfscope}%
\pgfsys@transformshift{2.840570in}{0.706440in}%
\pgfsys@useobject{currentmarker}{}%
\end{pgfscope}%
\begin{pgfscope}%
\pgfsys@transformshift{2.845356in}{0.706496in}%
\pgfsys@useobject{currentmarker}{}%
\end{pgfscope}%
\begin{pgfscope}%
\pgfsys@transformshift{2.850250in}{0.706554in}%
\pgfsys@useobject{currentmarker}{}%
\end{pgfscope}%
\begin{pgfscope}%
\pgfsys@transformshift{2.855256in}{0.706614in}%
\pgfsys@useobject{currentmarker}{}%
\end{pgfscope}%
\begin{pgfscope}%
\pgfsys@transformshift{2.860380in}{0.706678in}%
\pgfsys@useobject{currentmarker}{}%
\end{pgfscope}%
\begin{pgfscope}%
\pgfsys@transformshift{2.865627in}{0.706744in}%
\pgfsys@useobject{currentmarker}{}%
\end{pgfscope}%
\begin{pgfscope}%
\pgfsys@transformshift{2.871004in}{0.706814in}%
\pgfsys@useobject{currentmarker}{}%
\end{pgfscope}%
\begin{pgfscope}%
\pgfsys@transformshift{2.900082in}{0.707218in}%
\pgfsys@useobject{currentmarker}{}%
\end{pgfscope}%
\begin{pgfscope}%
\pgfsys@transformshift{2.933650in}{0.707750in}%
\pgfsys@useobject{currentmarker}{}%
\end{pgfscope}%
\begin{pgfscope}%
\pgfsys@transformshift{2.973352in}{0.708475in}%
\pgfsys@useobject{currentmarker}{}%
\end{pgfscope}%
\begin{pgfscope}%
\pgfsys@transformshift{3.021943in}{0.709508in}%
\pgfsys@useobject{currentmarker}{}%
\end{pgfscope}%
\begin{pgfscope}%
\pgfsys@transformshift{3.084589in}{0.711036in}%
\pgfsys@useobject{currentmarker}{}%
\end{pgfscope}%
\begin{pgfscope}%
\pgfsys@transformshift{3.172883in}{0.712988in}%
\pgfsys@useobject{currentmarker}{}%
\end{pgfscope}%
\end{pgfscope}%
\begin{pgfscope}%
\pgfsetbuttcap%
\pgfsetmiterjoin%
\definecolor{currentfill}{rgb}{1.000000,1.000000,1.000000}%
\pgfsetfillcolor{currentfill}%
\pgfsetlinewidth{0.000000pt}%
\definecolor{currentstroke}{rgb}{0.000000,0.000000,0.000000}%
\pgfsetstrokecolor{currentstroke}%
\pgfsetstrokeopacity{0.000000}%
\pgfsetdash{}{0pt}%
\pgfpathmoveto{\pgfqpoint{3.637789in}{0.435433in}}%
\pgfpathlineto{\pgfqpoint{4.860892in}{0.435433in}}%
\pgfpathlineto{\pgfqpoint{4.860892in}{1.043381in}}%
\pgfpathlineto{\pgfqpoint{3.637789in}{1.043381in}}%
\pgfpathclose%
\pgfusepath{fill}%
\end{pgfscope}%
\begin{pgfscope}%
\pgfpathrectangle{\pgfqpoint{3.637789in}{0.435433in}}{\pgfqpoint{1.223103in}{0.607948in}}%
\pgfusepath{clip}%
\pgfsetbuttcap%
\pgfsetmiterjoin%
\definecolor{currentfill}{rgb}{0.000000,0.000000,1.000000}%
\pgfsetfillcolor{currentfill}%
\pgfsetfillopacity{0.100000}%
\pgfsetlinewidth{0.803000pt}%
\definecolor{currentstroke}{rgb}{0.000000,0.000000,1.000000}%
\pgfsetstrokecolor{currentstroke}%
\pgfsetstrokeopacity{0.100000}%
\pgfsetdash{}{0pt}%
\pgfpathmoveto{\pgfqpoint{3.637789in}{0.703873in}}%
\pgfpathlineto{\pgfqpoint{3.637789in}{0.703907in}}%
\pgfpathlineto{\pgfqpoint{4.860892in}{0.703907in}}%
\pgfpathlineto{\pgfqpoint{4.860892in}{0.703873in}}%
\pgfpathclose%
\pgfusepath{stroke,fill}%
\end{pgfscope}%
\begin{pgfscope}%
\pgfpathrectangle{\pgfqpoint{3.637789in}{0.435433in}}{\pgfqpoint{1.223103in}{0.607948in}}%
\pgfusepath{clip}%
\pgfsetbuttcap%
\pgfsetroundjoin%
\definecolor{currentfill}{rgb}{0.000000,0.501961,0.000000}%
\pgfsetfillcolor{currentfill}%
\pgfsetfillopacity{0.500000}%
\pgfsetlinewidth{0.803000pt}%
\definecolor{currentstroke}{rgb}{0.000000,0.501961,0.000000}%
\pgfsetstrokecolor{currentstroke}%
\pgfsetstrokeopacity{0.500000}%
\pgfsetdash{}{0pt}%
\pgfpathmoveto{\pgfqpoint{3.637789in}{0.704103in}}%
\pgfpathlineto{\pgfqpoint{3.637789in}{0.704071in}}%
\pgfpathlineto{\pgfqpoint{3.878498in}{0.704473in}}%
\pgfpathlineto{\pgfqpoint{3.990029in}{0.704874in}}%
\pgfpathlineto{\pgfqpoint{4.063425in}{0.705276in}}%
\pgfpathlineto{\pgfqpoint{4.118221in}{0.705677in}}%
\pgfpathlineto{\pgfqpoint{4.161963in}{0.706079in}}%
\pgfpathlineto{\pgfqpoint{4.198371in}{0.706480in}}%
\pgfpathlineto{\pgfqpoint{4.229555in}{0.706882in}}%
\pgfpathlineto{\pgfqpoint{4.256828in}{0.707284in}}%
\pgfpathlineto{\pgfqpoint{4.281062in}{0.707686in}}%
\pgfpathlineto{\pgfqpoint{4.302867in}{0.708088in}}%
\pgfpathlineto{\pgfqpoint{4.322686in}{0.708489in}}%
\pgfpathlineto{\pgfqpoint{4.340851in}{0.708890in}}%
\pgfpathlineto{\pgfqpoint{4.357617in}{0.709291in}}%
\pgfpathlineto{\pgfqpoint{4.373183in}{0.709693in}}%
\pgfpathlineto{\pgfqpoint{4.387711in}{0.710094in}}%
\pgfpathlineto{\pgfqpoint{4.401329in}{0.710496in}}%
\pgfpathlineto{\pgfqpoint{4.414146in}{0.710898in}}%
\pgfpathlineto{\pgfqpoint{4.426250in}{0.711300in}}%
\pgfpathlineto{\pgfqpoint{4.437717in}{0.711703in}}%
\pgfpathlineto{\pgfqpoint{4.448610in}{0.712105in}}%
\pgfpathlineto{\pgfqpoint{4.458984in}{0.712508in}}%
\pgfpathlineto{\pgfqpoint{4.468886in}{0.712911in}}%
\pgfpathlineto{\pgfqpoint{4.478357in}{0.713314in}}%
\pgfpathlineto{\pgfqpoint{4.487434in}{0.713717in}}%
\pgfpathlineto{\pgfqpoint{4.496147in}{0.714120in}}%
\pgfpathlineto{\pgfqpoint{4.504525in}{0.714522in}}%
\pgfpathlineto{\pgfqpoint{4.512593in}{0.714925in}}%
\pgfpathlineto{\pgfqpoint{4.520372in}{0.715327in}}%
\pgfpathlineto{\pgfqpoint{4.527883in}{0.715729in}}%
\pgfpathlineto{\pgfqpoint{4.535144in}{0.716131in}}%
\pgfpathlineto{\pgfqpoint{4.542170in}{0.716533in}}%
\pgfpathlineto{\pgfqpoint{4.548977in}{0.716935in}}%
\pgfpathlineto{\pgfqpoint{4.555577in}{0.717337in}}%
\pgfpathlineto{\pgfqpoint{4.561983in}{0.717739in}}%
\pgfpathlineto{\pgfqpoint{4.568206in}{0.718141in}}%
\pgfpathlineto{\pgfqpoint{4.574256in}{0.718542in}}%
\pgfpathlineto{\pgfqpoint{4.580143in}{0.718944in}}%
\pgfpathlineto{\pgfqpoint{4.585874in}{0.719346in}}%
\pgfpathlineto{\pgfqpoint{4.591459in}{0.719747in}}%
\pgfpathlineto{\pgfqpoint{4.596904in}{0.720148in}}%
\pgfpathlineto{\pgfqpoint{4.602216in}{0.720550in}}%
\pgfpathlineto{\pgfqpoint{4.607401in}{0.720951in}}%
\pgfpathlineto{\pgfqpoint{4.612466in}{0.721352in}}%
\pgfpathlineto{\pgfqpoint{4.617416in}{0.721752in}}%
\pgfpathlineto{\pgfqpoint{4.622256in}{0.722153in}}%
\pgfpathlineto{\pgfqpoint{4.626991in}{0.722554in}}%
\pgfpathlineto{\pgfqpoint{4.631625in}{0.722954in}}%
\pgfpathlineto{\pgfqpoint{4.636162in}{0.723354in}}%
\pgfpathlineto{\pgfqpoint{4.640607in}{0.723752in}}%
\pgfpathlineto{\pgfqpoint{4.640607in}{0.723754in}}%
\pgfpathlineto{\pgfqpoint{4.640607in}{0.723754in}}%
\pgfpathlineto{\pgfqpoint{4.636162in}{0.723355in}}%
\pgfpathlineto{\pgfqpoint{4.631625in}{0.722957in}}%
\pgfpathlineto{\pgfqpoint{4.626991in}{0.722559in}}%
\pgfpathlineto{\pgfqpoint{4.622256in}{0.722160in}}%
\pgfpathlineto{\pgfqpoint{4.617416in}{0.721761in}}%
\pgfpathlineto{\pgfqpoint{4.612466in}{0.721361in}}%
\pgfpathlineto{\pgfqpoint{4.607401in}{0.720961in}}%
\pgfpathlineto{\pgfqpoint{4.602216in}{0.720561in}}%
\pgfpathlineto{\pgfqpoint{4.596904in}{0.720160in}}%
\pgfpathlineto{\pgfqpoint{4.591459in}{0.719758in}}%
\pgfpathlineto{\pgfqpoint{4.585874in}{0.719357in}}%
\pgfpathlineto{\pgfqpoint{4.580143in}{0.718955in}}%
\pgfpathlineto{\pgfqpoint{4.574256in}{0.718553in}}%
\pgfpathlineto{\pgfqpoint{4.568206in}{0.718150in}}%
\pgfpathlineto{\pgfqpoint{4.561983in}{0.717748in}}%
\pgfpathlineto{\pgfqpoint{4.555577in}{0.717345in}}%
\pgfpathlineto{\pgfqpoint{4.548977in}{0.716942in}}%
\pgfpathlineto{\pgfqpoint{4.542170in}{0.716539in}}%
\pgfpathlineto{\pgfqpoint{4.535144in}{0.716136in}}%
\pgfpathlineto{\pgfqpoint{4.527883in}{0.715733in}}%
\pgfpathlineto{\pgfqpoint{4.520372in}{0.715330in}}%
\pgfpathlineto{\pgfqpoint{4.512593in}{0.714926in}}%
\pgfpathlineto{\pgfqpoint{4.504525in}{0.714523in}}%
\pgfpathlineto{\pgfqpoint{4.496147in}{0.714120in}}%
\pgfpathlineto{\pgfqpoint{4.487434in}{0.713718in}}%
\pgfpathlineto{\pgfqpoint{4.478357in}{0.713316in}}%
\pgfpathlineto{\pgfqpoint{4.468886in}{0.712913in}}%
\pgfpathlineto{\pgfqpoint{4.458984in}{0.712511in}}%
\pgfpathlineto{\pgfqpoint{4.448610in}{0.712109in}}%
\pgfpathlineto{\pgfqpoint{4.437717in}{0.711707in}}%
\pgfpathlineto{\pgfqpoint{4.426250in}{0.711304in}}%
\pgfpathlineto{\pgfqpoint{4.414146in}{0.710902in}}%
\pgfpathlineto{\pgfqpoint{4.401329in}{0.710500in}}%
\pgfpathlineto{\pgfqpoint{4.387711in}{0.710098in}}%
\pgfpathlineto{\pgfqpoint{4.373183in}{0.709696in}}%
\pgfpathlineto{\pgfqpoint{4.357617in}{0.709294in}}%
\pgfpathlineto{\pgfqpoint{4.340851in}{0.708892in}}%
\pgfpathlineto{\pgfqpoint{4.322686in}{0.708490in}}%
\pgfpathlineto{\pgfqpoint{4.302867in}{0.708089in}}%
\pgfpathlineto{\pgfqpoint{4.281062in}{0.707689in}}%
\pgfpathlineto{\pgfqpoint{4.256828in}{0.707289in}}%
\pgfpathlineto{\pgfqpoint{4.229555in}{0.706889in}}%
\pgfpathlineto{\pgfqpoint{4.198371in}{0.706490in}}%
\pgfpathlineto{\pgfqpoint{4.161963in}{0.706091in}}%
\pgfpathlineto{\pgfqpoint{4.118221in}{0.705693in}}%
\pgfpathlineto{\pgfqpoint{4.063425in}{0.705295in}}%
\pgfpathlineto{\pgfqpoint{3.990029in}{0.704897in}}%
\pgfpathlineto{\pgfqpoint{3.878498in}{0.704500in}}%
\pgfpathlineto{\pgfqpoint{3.637789in}{0.704103in}}%
\pgfpathclose%
\pgfusepath{stroke,fill}%
\end{pgfscope}%
\begin{pgfscope}%
\pgfpathrectangle{\pgfqpoint{3.637789in}{0.435433in}}{\pgfqpoint{1.223103in}{0.607948in}}%
\pgfusepath{clip}%
\pgfsetroundcap%
\pgfsetroundjoin%
\pgfsetlinewidth{0.501875pt}%
\definecolor{currentstroke}{rgb}{0.000000,0.000000,1.000000}%
\pgfsetstrokecolor{currentstroke}%
\pgfsetstrokeopacity{0.800000}%
\pgfsetdash{}{0pt}%
\pgfpathmoveto{\pgfqpoint{3.637789in}{0.703890in}}%
\pgfpathlineto{\pgfqpoint{4.860892in}{0.703890in}}%
\pgfusepath{stroke}%
\end{pgfscope}%
\begin{pgfscope}%
\pgfpathrectangle{\pgfqpoint{3.637789in}{0.435433in}}{\pgfqpoint{1.223103in}{0.607948in}}%
\pgfusepath{clip}%
\pgfsetbuttcap%
\pgfsetroundjoin%
\pgfsetlinewidth{1.003750pt}%
\definecolor{currentstroke}{rgb}{0.000000,0.000000,0.000000}%
\pgfsetstrokecolor{currentstroke}%
\pgfsetdash{{3.700000pt}{1.600000pt}}{0.000000pt}%
\pgfpathmoveto{\pgfqpoint{3.637789in}{0.703897in}}%
\pgfpathlineto{\pgfqpoint{4.860892in}{0.703897in}}%
\pgfusepath{stroke}%
\end{pgfscope}%
\begin{pgfscope}%
\pgfsetroundcap%
\pgfsetroundjoin%
\pgfsetlinewidth{0.501875pt}%
\definecolor{currentstroke}{rgb}{0.000000,0.000000,1.000000}%
\pgfsetstrokecolor{currentstroke}%
\pgfsetstrokeopacity{0.800000}%
\pgfsetdash{}{0pt}%
\pgfpathmoveto{\pgfqpoint{4.478180in}{0.822711in}}%
\pgfpathquadraticcurveto{\pgfqpoint{4.395905in}{0.770723in}}{\pgfqpoint{4.313631in}{0.718735in}}%
\pgfusepath{stroke}%
\end{pgfscope}%
\begin{pgfscope}%
\pgfsetfillopacity{0.800000}%
\pgfsetstrokeopacity{0.800000}%
\definecolor{textcolor}{rgb}{0.000000,0.000000,1.000000}%
\pgfsetstrokecolor{textcolor}%
\pgfsetfillcolor{textcolor}%
\pgftext[x=4.378430in,y=0.886274in,left,base]{\color{textcolor}\sffamily\fontsize{5.647059}{6.776471}\selectfont 1.441579(28)}%
\end{pgfscope}%
\begin{pgfscope}%
\pgfsetbuttcap%
\pgfsetroundjoin%
\definecolor{currentfill}{rgb}{0.150000,0.150000,0.150000}%
\pgfsetfillcolor{currentfill}%
\pgfsetlinewidth{1.003750pt}%
\definecolor{currentstroke}{rgb}{0.150000,0.150000,0.150000}%
\pgfsetstrokecolor{currentstroke}%
\pgfsetdash{}{0pt}%
\pgfsys@defobject{currentmarker}{\pgfqpoint{0.000000in}{-0.066667in}}{\pgfqpoint{0.000000in}{0.000000in}}{%
\pgfpathmoveto{\pgfqpoint{0.000000in}{0.000000in}}%
\pgfpathlineto{\pgfqpoint{0.000000in}{-0.066667in}}%
\pgfusepath{stroke,fill}%
}%
\begin{pgfscope}%
\pgfsys@transformshift{3.637789in}{0.435433in}%
\pgfsys@useobject{currentmarker}{}%
\end{pgfscope}%
\end{pgfscope}%
\begin{pgfscope}%
\definecolor{textcolor}{rgb}{0.150000,0.150000,0.150000}%
\pgfsetstrokecolor{textcolor}%
\pgfsetfillcolor{textcolor}%
\pgftext[x=3.637789in,y=0.320155in,,top]{\color{textcolor}\sffamily\fontsize{5.176471}{6.211765}\selectfont \(\displaystyle {10^{-3}}\)}%
\end{pgfscope}%
\begin{pgfscope}%
\pgfsetbuttcap%
\pgfsetroundjoin%
\definecolor{currentfill}{rgb}{0.150000,0.150000,0.150000}%
\pgfsetfillcolor{currentfill}%
\pgfsetlinewidth{1.003750pt}%
\definecolor{currentstroke}{rgb}{0.150000,0.150000,0.150000}%
\pgfsetstrokecolor{currentstroke}%
\pgfsetdash{}{0pt}%
\pgfsys@defobject{currentmarker}{\pgfqpoint{0.000000in}{-0.066667in}}{\pgfqpoint{0.000000in}{0.000000in}}{%
\pgfpathmoveto{\pgfqpoint{0.000000in}{0.000000in}}%
\pgfpathlineto{\pgfqpoint{0.000000in}{-0.066667in}}%
\pgfusepath{stroke,fill}%
}%
\begin{pgfscope}%
\pgfsys@transformshift{4.139198in}{0.435433in}%
\pgfsys@useobject{currentmarker}{}%
\end{pgfscope}%
\end{pgfscope}%
\begin{pgfscope}%
\definecolor{textcolor}{rgb}{0.150000,0.150000,0.150000}%
\pgfsetstrokecolor{textcolor}%
\pgfsetfillcolor{textcolor}%
\pgftext[x=4.139198in,y=0.320155in,,top]{\color{textcolor}\sffamily\fontsize{5.176471}{6.211765}\selectfont \(\displaystyle {10^{-2}}\)}%
\end{pgfscope}%
\begin{pgfscope}%
\pgfsetbuttcap%
\pgfsetroundjoin%
\definecolor{currentfill}{rgb}{0.150000,0.150000,0.150000}%
\pgfsetfillcolor{currentfill}%
\pgfsetlinewidth{1.003750pt}%
\definecolor{currentstroke}{rgb}{0.150000,0.150000,0.150000}%
\pgfsetstrokecolor{currentstroke}%
\pgfsetdash{}{0pt}%
\pgfsys@defobject{currentmarker}{\pgfqpoint{0.000000in}{-0.066667in}}{\pgfqpoint{0.000000in}{0.000000in}}{%
\pgfpathmoveto{\pgfqpoint{0.000000in}{0.000000in}}%
\pgfpathlineto{\pgfqpoint{0.000000in}{-0.066667in}}%
\pgfusepath{stroke,fill}%
}%
\begin{pgfscope}%
\pgfsys@transformshift{4.640607in}{0.435433in}%
\pgfsys@useobject{currentmarker}{}%
\end{pgfscope}%
\end{pgfscope}%
\begin{pgfscope}%
\definecolor{textcolor}{rgb}{0.150000,0.150000,0.150000}%
\pgfsetstrokecolor{textcolor}%
\pgfsetfillcolor{textcolor}%
\pgftext[x=4.640607in,y=0.320155in,,top]{\color{textcolor}\sffamily\fontsize{5.176471}{6.211765}\selectfont \(\displaystyle {10^{-1}}\)}%
\end{pgfscope}%
\begin{pgfscope}%
\pgfsetbuttcap%
\pgfsetroundjoin%
\definecolor{currentfill}{rgb}{0.150000,0.150000,0.150000}%
\pgfsetfillcolor{currentfill}%
\pgfsetlinewidth{0.803000pt}%
\definecolor{currentstroke}{rgb}{0.150000,0.150000,0.150000}%
\pgfsetstrokecolor{currentstroke}%
\pgfsetdash{}{0pt}%
\pgfsys@defobject{currentmarker}{\pgfqpoint{0.000000in}{-0.044444in}}{\pgfqpoint{0.000000in}{0.000000in}}{%
\pgfpathmoveto{\pgfqpoint{0.000000in}{0.000000in}}%
\pgfpathlineto{\pgfqpoint{0.000000in}{-0.044444in}}%
\pgfusepath{stroke,fill}%
}%
\begin{pgfscope}%
\pgfsys@transformshift{3.788728in}{0.435433in}%
\pgfsys@useobject{currentmarker}{}%
\end{pgfscope}%
\end{pgfscope}%
\begin{pgfscope}%
\pgfsetbuttcap%
\pgfsetroundjoin%
\definecolor{currentfill}{rgb}{0.150000,0.150000,0.150000}%
\pgfsetfillcolor{currentfill}%
\pgfsetlinewidth{0.803000pt}%
\definecolor{currentstroke}{rgb}{0.150000,0.150000,0.150000}%
\pgfsetstrokecolor{currentstroke}%
\pgfsetdash{}{0pt}%
\pgfsys@defobject{currentmarker}{\pgfqpoint{0.000000in}{-0.044444in}}{\pgfqpoint{0.000000in}{0.000000in}}{%
\pgfpathmoveto{\pgfqpoint{0.000000in}{0.000000in}}%
\pgfpathlineto{\pgfqpoint{0.000000in}{-0.044444in}}%
\pgfusepath{stroke,fill}%
}%
\begin{pgfscope}%
\pgfsys@transformshift{3.877021in}{0.435433in}%
\pgfsys@useobject{currentmarker}{}%
\end{pgfscope}%
\end{pgfscope}%
\begin{pgfscope}%
\pgfsetbuttcap%
\pgfsetroundjoin%
\definecolor{currentfill}{rgb}{0.150000,0.150000,0.150000}%
\pgfsetfillcolor{currentfill}%
\pgfsetlinewidth{0.803000pt}%
\definecolor{currentstroke}{rgb}{0.150000,0.150000,0.150000}%
\pgfsetstrokecolor{currentstroke}%
\pgfsetdash{}{0pt}%
\pgfsys@defobject{currentmarker}{\pgfqpoint{0.000000in}{-0.044444in}}{\pgfqpoint{0.000000in}{0.000000in}}{%
\pgfpathmoveto{\pgfqpoint{0.000000in}{0.000000in}}%
\pgfpathlineto{\pgfqpoint{0.000000in}{-0.044444in}}%
\pgfusepath{stroke,fill}%
}%
\begin{pgfscope}%
\pgfsys@transformshift{3.939667in}{0.435433in}%
\pgfsys@useobject{currentmarker}{}%
\end{pgfscope}%
\end{pgfscope}%
\begin{pgfscope}%
\pgfsetbuttcap%
\pgfsetroundjoin%
\definecolor{currentfill}{rgb}{0.150000,0.150000,0.150000}%
\pgfsetfillcolor{currentfill}%
\pgfsetlinewidth{0.803000pt}%
\definecolor{currentstroke}{rgb}{0.150000,0.150000,0.150000}%
\pgfsetstrokecolor{currentstroke}%
\pgfsetdash{}{0pt}%
\pgfsys@defobject{currentmarker}{\pgfqpoint{0.000000in}{-0.044444in}}{\pgfqpoint{0.000000in}{0.000000in}}{%
\pgfpathmoveto{\pgfqpoint{0.000000in}{0.000000in}}%
\pgfpathlineto{\pgfqpoint{0.000000in}{-0.044444in}}%
\pgfusepath{stroke,fill}%
}%
\begin{pgfscope}%
\pgfsys@transformshift{3.988258in}{0.435433in}%
\pgfsys@useobject{currentmarker}{}%
\end{pgfscope}%
\end{pgfscope}%
\begin{pgfscope}%
\pgfsetbuttcap%
\pgfsetroundjoin%
\definecolor{currentfill}{rgb}{0.150000,0.150000,0.150000}%
\pgfsetfillcolor{currentfill}%
\pgfsetlinewidth{0.803000pt}%
\definecolor{currentstroke}{rgb}{0.150000,0.150000,0.150000}%
\pgfsetstrokecolor{currentstroke}%
\pgfsetdash{}{0pt}%
\pgfsys@defobject{currentmarker}{\pgfqpoint{0.000000in}{-0.044444in}}{\pgfqpoint{0.000000in}{0.000000in}}{%
\pgfpathmoveto{\pgfqpoint{0.000000in}{0.000000in}}%
\pgfpathlineto{\pgfqpoint{0.000000in}{-0.044444in}}%
\pgfusepath{stroke,fill}%
}%
\begin{pgfscope}%
\pgfsys@transformshift{4.027961in}{0.435433in}%
\pgfsys@useobject{currentmarker}{}%
\end{pgfscope}%
\end{pgfscope}%
\begin{pgfscope}%
\pgfsetbuttcap%
\pgfsetroundjoin%
\definecolor{currentfill}{rgb}{0.150000,0.150000,0.150000}%
\pgfsetfillcolor{currentfill}%
\pgfsetlinewidth{0.803000pt}%
\definecolor{currentstroke}{rgb}{0.150000,0.150000,0.150000}%
\pgfsetstrokecolor{currentstroke}%
\pgfsetdash{}{0pt}%
\pgfsys@defobject{currentmarker}{\pgfqpoint{0.000000in}{-0.044444in}}{\pgfqpoint{0.000000in}{0.000000in}}{%
\pgfpathmoveto{\pgfqpoint{0.000000in}{0.000000in}}%
\pgfpathlineto{\pgfqpoint{0.000000in}{-0.044444in}}%
\pgfusepath{stroke,fill}%
}%
\begin{pgfscope}%
\pgfsys@transformshift{4.061528in}{0.435433in}%
\pgfsys@useobject{currentmarker}{}%
\end{pgfscope}%
\end{pgfscope}%
\begin{pgfscope}%
\pgfsetbuttcap%
\pgfsetroundjoin%
\definecolor{currentfill}{rgb}{0.150000,0.150000,0.150000}%
\pgfsetfillcolor{currentfill}%
\pgfsetlinewidth{0.803000pt}%
\definecolor{currentstroke}{rgb}{0.150000,0.150000,0.150000}%
\pgfsetstrokecolor{currentstroke}%
\pgfsetdash{}{0pt}%
\pgfsys@defobject{currentmarker}{\pgfqpoint{0.000000in}{-0.044444in}}{\pgfqpoint{0.000000in}{0.000000in}}{%
\pgfpathmoveto{\pgfqpoint{0.000000in}{0.000000in}}%
\pgfpathlineto{\pgfqpoint{0.000000in}{-0.044444in}}%
\pgfusepath{stroke,fill}%
}%
\begin{pgfscope}%
\pgfsys@transformshift{4.090606in}{0.435433in}%
\pgfsys@useobject{currentmarker}{}%
\end{pgfscope}%
\end{pgfscope}%
\begin{pgfscope}%
\pgfsetbuttcap%
\pgfsetroundjoin%
\definecolor{currentfill}{rgb}{0.150000,0.150000,0.150000}%
\pgfsetfillcolor{currentfill}%
\pgfsetlinewidth{0.803000pt}%
\definecolor{currentstroke}{rgb}{0.150000,0.150000,0.150000}%
\pgfsetstrokecolor{currentstroke}%
\pgfsetdash{}{0pt}%
\pgfsys@defobject{currentmarker}{\pgfqpoint{0.000000in}{-0.044444in}}{\pgfqpoint{0.000000in}{0.000000in}}{%
\pgfpathmoveto{\pgfqpoint{0.000000in}{0.000000in}}%
\pgfpathlineto{\pgfqpoint{0.000000in}{-0.044444in}}%
\pgfusepath{stroke,fill}%
}%
\begin{pgfscope}%
\pgfsys@transformshift{4.116254in}{0.435433in}%
\pgfsys@useobject{currentmarker}{}%
\end{pgfscope}%
\end{pgfscope}%
\begin{pgfscope}%
\pgfsetbuttcap%
\pgfsetroundjoin%
\definecolor{currentfill}{rgb}{0.150000,0.150000,0.150000}%
\pgfsetfillcolor{currentfill}%
\pgfsetlinewidth{0.803000pt}%
\definecolor{currentstroke}{rgb}{0.150000,0.150000,0.150000}%
\pgfsetstrokecolor{currentstroke}%
\pgfsetdash{}{0pt}%
\pgfsys@defobject{currentmarker}{\pgfqpoint{0.000000in}{-0.044444in}}{\pgfqpoint{0.000000in}{0.000000in}}{%
\pgfpathmoveto{\pgfqpoint{0.000000in}{0.000000in}}%
\pgfpathlineto{\pgfqpoint{0.000000in}{-0.044444in}}%
\pgfusepath{stroke,fill}%
}%
\begin{pgfscope}%
\pgfsys@transformshift{4.290137in}{0.435433in}%
\pgfsys@useobject{currentmarker}{}%
\end{pgfscope}%
\end{pgfscope}%
\begin{pgfscope}%
\pgfsetbuttcap%
\pgfsetroundjoin%
\definecolor{currentfill}{rgb}{0.150000,0.150000,0.150000}%
\pgfsetfillcolor{currentfill}%
\pgfsetlinewidth{0.803000pt}%
\definecolor{currentstroke}{rgb}{0.150000,0.150000,0.150000}%
\pgfsetstrokecolor{currentstroke}%
\pgfsetdash{}{0pt}%
\pgfsys@defobject{currentmarker}{\pgfqpoint{0.000000in}{-0.044444in}}{\pgfqpoint{0.000000in}{0.000000in}}{%
\pgfpathmoveto{\pgfqpoint{0.000000in}{0.000000in}}%
\pgfpathlineto{\pgfqpoint{0.000000in}{-0.044444in}}%
\pgfusepath{stroke,fill}%
}%
\begin{pgfscope}%
\pgfsys@transformshift{4.378430in}{0.435433in}%
\pgfsys@useobject{currentmarker}{}%
\end{pgfscope}%
\end{pgfscope}%
\begin{pgfscope}%
\pgfsetbuttcap%
\pgfsetroundjoin%
\definecolor{currentfill}{rgb}{0.150000,0.150000,0.150000}%
\pgfsetfillcolor{currentfill}%
\pgfsetlinewidth{0.803000pt}%
\definecolor{currentstroke}{rgb}{0.150000,0.150000,0.150000}%
\pgfsetstrokecolor{currentstroke}%
\pgfsetdash{}{0pt}%
\pgfsys@defobject{currentmarker}{\pgfqpoint{0.000000in}{-0.044444in}}{\pgfqpoint{0.000000in}{0.000000in}}{%
\pgfpathmoveto{\pgfqpoint{0.000000in}{0.000000in}}%
\pgfpathlineto{\pgfqpoint{0.000000in}{-0.044444in}}%
\pgfusepath{stroke,fill}%
}%
\begin{pgfscope}%
\pgfsys@transformshift{4.441076in}{0.435433in}%
\pgfsys@useobject{currentmarker}{}%
\end{pgfscope}%
\end{pgfscope}%
\begin{pgfscope}%
\pgfsetbuttcap%
\pgfsetroundjoin%
\definecolor{currentfill}{rgb}{0.150000,0.150000,0.150000}%
\pgfsetfillcolor{currentfill}%
\pgfsetlinewidth{0.803000pt}%
\definecolor{currentstroke}{rgb}{0.150000,0.150000,0.150000}%
\pgfsetstrokecolor{currentstroke}%
\pgfsetdash{}{0pt}%
\pgfsys@defobject{currentmarker}{\pgfqpoint{0.000000in}{-0.044444in}}{\pgfqpoint{0.000000in}{0.000000in}}{%
\pgfpathmoveto{\pgfqpoint{0.000000in}{0.000000in}}%
\pgfpathlineto{\pgfqpoint{0.000000in}{-0.044444in}}%
\pgfusepath{stroke,fill}%
}%
\begin{pgfscope}%
\pgfsys@transformshift{4.489667in}{0.435433in}%
\pgfsys@useobject{currentmarker}{}%
\end{pgfscope}%
\end{pgfscope}%
\begin{pgfscope}%
\pgfsetbuttcap%
\pgfsetroundjoin%
\definecolor{currentfill}{rgb}{0.150000,0.150000,0.150000}%
\pgfsetfillcolor{currentfill}%
\pgfsetlinewidth{0.803000pt}%
\definecolor{currentstroke}{rgb}{0.150000,0.150000,0.150000}%
\pgfsetstrokecolor{currentstroke}%
\pgfsetdash{}{0pt}%
\pgfsys@defobject{currentmarker}{\pgfqpoint{0.000000in}{-0.044444in}}{\pgfqpoint{0.000000in}{0.000000in}}{%
\pgfpathmoveto{\pgfqpoint{0.000000in}{0.000000in}}%
\pgfpathlineto{\pgfqpoint{0.000000in}{-0.044444in}}%
\pgfusepath{stroke,fill}%
}%
\begin{pgfscope}%
\pgfsys@transformshift{4.529370in}{0.435433in}%
\pgfsys@useobject{currentmarker}{}%
\end{pgfscope}%
\end{pgfscope}%
\begin{pgfscope}%
\pgfsetbuttcap%
\pgfsetroundjoin%
\definecolor{currentfill}{rgb}{0.150000,0.150000,0.150000}%
\pgfsetfillcolor{currentfill}%
\pgfsetlinewidth{0.803000pt}%
\definecolor{currentstroke}{rgb}{0.150000,0.150000,0.150000}%
\pgfsetstrokecolor{currentstroke}%
\pgfsetdash{}{0pt}%
\pgfsys@defobject{currentmarker}{\pgfqpoint{0.000000in}{-0.044444in}}{\pgfqpoint{0.000000in}{0.000000in}}{%
\pgfpathmoveto{\pgfqpoint{0.000000in}{0.000000in}}%
\pgfpathlineto{\pgfqpoint{0.000000in}{-0.044444in}}%
\pgfusepath{stroke,fill}%
}%
\begin{pgfscope}%
\pgfsys@transformshift{4.562937in}{0.435433in}%
\pgfsys@useobject{currentmarker}{}%
\end{pgfscope}%
\end{pgfscope}%
\begin{pgfscope}%
\pgfsetbuttcap%
\pgfsetroundjoin%
\definecolor{currentfill}{rgb}{0.150000,0.150000,0.150000}%
\pgfsetfillcolor{currentfill}%
\pgfsetlinewidth{0.803000pt}%
\definecolor{currentstroke}{rgb}{0.150000,0.150000,0.150000}%
\pgfsetstrokecolor{currentstroke}%
\pgfsetdash{}{0pt}%
\pgfsys@defobject{currentmarker}{\pgfqpoint{0.000000in}{-0.044444in}}{\pgfqpoint{0.000000in}{0.000000in}}{%
\pgfpathmoveto{\pgfqpoint{0.000000in}{0.000000in}}%
\pgfpathlineto{\pgfqpoint{0.000000in}{-0.044444in}}%
\pgfusepath{stroke,fill}%
}%
\begin{pgfscope}%
\pgfsys@transformshift{4.592015in}{0.435433in}%
\pgfsys@useobject{currentmarker}{}%
\end{pgfscope}%
\end{pgfscope}%
\begin{pgfscope}%
\pgfsetbuttcap%
\pgfsetroundjoin%
\definecolor{currentfill}{rgb}{0.150000,0.150000,0.150000}%
\pgfsetfillcolor{currentfill}%
\pgfsetlinewidth{0.803000pt}%
\definecolor{currentstroke}{rgb}{0.150000,0.150000,0.150000}%
\pgfsetstrokecolor{currentstroke}%
\pgfsetdash{}{0pt}%
\pgfsys@defobject{currentmarker}{\pgfqpoint{0.000000in}{-0.044444in}}{\pgfqpoint{0.000000in}{0.000000in}}{%
\pgfpathmoveto{\pgfqpoint{0.000000in}{0.000000in}}%
\pgfpathlineto{\pgfqpoint{0.000000in}{-0.044444in}}%
\pgfusepath{stroke,fill}%
}%
\begin{pgfscope}%
\pgfsys@transformshift{4.617663in}{0.435433in}%
\pgfsys@useobject{currentmarker}{}%
\end{pgfscope}%
\end{pgfscope}%
\begin{pgfscope}%
\pgfsetbuttcap%
\pgfsetroundjoin%
\definecolor{currentfill}{rgb}{0.150000,0.150000,0.150000}%
\pgfsetfillcolor{currentfill}%
\pgfsetlinewidth{0.803000pt}%
\definecolor{currentstroke}{rgb}{0.150000,0.150000,0.150000}%
\pgfsetstrokecolor{currentstroke}%
\pgfsetdash{}{0pt}%
\pgfsys@defobject{currentmarker}{\pgfqpoint{0.000000in}{-0.044444in}}{\pgfqpoint{0.000000in}{0.000000in}}{%
\pgfpathmoveto{\pgfqpoint{0.000000in}{0.000000in}}%
\pgfpathlineto{\pgfqpoint{0.000000in}{-0.044444in}}%
\pgfusepath{stroke,fill}%
}%
\begin{pgfscope}%
\pgfsys@transformshift{4.791546in}{0.435433in}%
\pgfsys@useobject{currentmarker}{}%
\end{pgfscope}%
\end{pgfscope}%
\begin{pgfscope}%
\definecolor{textcolor}{rgb}{0.150000,0.150000,0.150000}%
\pgfsetstrokecolor{textcolor}%
\pgfsetfillcolor{textcolor}%
\pgftext[x=4.249340in,y=0.183333in,,top]{\color{textcolor}\sffamily\fontsize{5.647059}{6.776471}\selectfont \(\displaystyle \epsilon [\mathrm{fm}]\)}%
\end{pgfscope}%
\begin{pgfscope}%
\pgfsetbuttcap%
\pgfsetroundjoin%
\definecolor{currentfill}{rgb}{0.150000,0.150000,0.150000}%
\pgfsetfillcolor{currentfill}%
\pgfsetlinewidth{1.003750pt}%
\definecolor{currentstroke}{rgb}{0.150000,0.150000,0.150000}%
\pgfsetstrokecolor{currentstroke}%
\pgfsetdash{}{0pt}%
\pgfsys@defobject{currentmarker}{\pgfqpoint{-0.066667in}{0.000000in}}{\pgfqpoint{0.000000in}{0.000000in}}{%
\pgfpathmoveto{\pgfqpoint{0.000000in}{0.000000in}}%
\pgfpathlineto{\pgfqpoint{-0.066667in}{0.000000in}}%
\pgfusepath{stroke,fill}%
}%
\begin{pgfscope}%
\pgfsys@transformshift{3.637789in}{0.435433in}%
\pgfsys@useobject{currentmarker}{}%
\end{pgfscope}%
\end{pgfscope}%
\begin{pgfscope}%
\pgfsetbuttcap%
\pgfsetroundjoin%
\definecolor{currentfill}{rgb}{0.150000,0.150000,0.150000}%
\pgfsetfillcolor{currentfill}%
\pgfsetlinewidth{1.003750pt}%
\definecolor{currentstroke}{rgb}{0.150000,0.150000,0.150000}%
\pgfsetstrokecolor{currentstroke}%
\pgfsetdash{}{0pt}%
\pgfsys@defobject{currentmarker}{\pgfqpoint{-0.066667in}{0.000000in}}{\pgfqpoint{0.000000in}{0.000000in}}{%
\pgfpathmoveto{\pgfqpoint{0.000000in}{0.000000in}}%
\pgfpathlineto{\pgfqpoint{-0.066667in}{0.000000in}}%
\pgfusepath{stroke,fill}%
}%
\begin{pgfscope}%
\pgfsys@transformshift{3.637789in}{0.703897in}%
\pgfsys@useobject{currentmarker}{}%
\end{pgfscope}%
\end{pgfscope}%
\begin{pgfscope}%
\pgfsetbuttcap%
\pgfsetroundjoin%
\definecolor{currentfill}{rgb}{0.150000,0.150000,0.150000}%
\pgfsetfillcolor{currentfill}%
\pgfsetlinewidth{1.003750pt}%
\definecolor{currentstroke}{rgb}{0.150000,0.150000,0.150000}%
\pgfsetstrokecolor{currentstroke}%
\pgfsetdash{}{0pt}%
\pgfsys@defobject{currentmarker}{\pgfqpoint{-0.066667in}{0.000000in}}{\pgfqpoint{0.000000in}{0.000000in}}{%
\pgfpathmoveto{\pgfqpoint{0.000000in}{0.000000in}}%
\pgfpathlineto{\pgfqpoint{-0.066667in}{0.000000in}}%
\pgfusepath{stroke,fill}%
}%
\begin{pgfscope}%
\pgfsys@transformshift{3.637789in}{1.043381in}%
\pgfsys@useobject{currentmarker}{}%
\end{pgfscope}%
\end{pgfscope}%
\begin{pgfscope}%
\pgfpathrectangle{\pgfqpoint{3.637789in}{0.435433in}}{\pgfqpoint{1.223103in}{0.607948in}}%
\pgfusepath{clip}%
\pgfsetroundcap%
\pgfsetroundjoin%
\pgfsetlinewidth{1.204500pt}%
\definecolor{currentstroke}{rgb}{0.000000,0.501961,0.000000}%
\pgfsetstrokecolor{currentstroke}%
\pgfsetdash{}{0pt}%
\pgfpathmoveto{\pgfqpoint{3.637789in}{0.704087in}}%
\pgfpathlineto{\pgfqpoint{3.878498in}{0.704486in}}%
\pgfpathlineto{\pgfqpoint{3.990029in}{0.704886in}}%
\pgfpathlineto{\pgfqpoint{4.063425in}{0.705285in}}%
\pgfpathlineto{\pgfqpoint{4.118221in}{0.705685in}}%
\pgfpathlineto{\pgfqpoint{4.161963in}{0.706085in}}%
\pgfpathlineto{\pgfqpoint{4.198371in}{0.706485in}}%
\pgfpathlineto{\pgfqpoint{4.229555in}{0.706886in}}%
\pgfpathlineto{\pgfqpoint{4.256828in}{0.707286in}}%
\pgfpathlineto{\pgfqpoint{4.281062in}{0.707687in}}%
\pgfpathlineto{\pgfqpoint{4.302867in}{0.708088in}}%
\pgfpathlineto{\pgfqpoint{4.322686in}{0.708489in}}%
\pgfpathlineto{\pgfqpoint{4.340851in}{0.708891in}}%
\pgfpathlineto{\pgfqpoint{4.357617in}{0.709292in}}%
\pgfpathlineto{\pgfqpoint{4.373183in}{0.709694in}}%
\pgfpathlineto{\pgfqpoint{4.387711in}{0.710096in}}%
\pgfpathlineto{\pgfqpoint{4.401329in}{0.710498in}}%
\pgfpathlineto{\pgfqpoint{4.414146in}{0.710900in}}%
\pgfpathlineto{\pgfqpoint{4.426250in}{0.711302in}}%
\pgfpathlineto{\pgfqpoint{4.437717in}{0.711705in}}%
\pgfpathlineto{\pgfqpoint{4.448610in}{0.712107in}}%
\pgfpathlineto{\pgfqpoint{4.458984in}{0.712510in}}%
\pgfpathlineto{\pgfqpoint{4.468886in}{0.712912in}}%
\pgfpathlineto{\pgfqpoint{4.478357in}{0.713315in}}%
\pgfpathlineto{\pgfqpoint{4.487434in}{0.713717in}}%
\pgfpathlineto{\pgfqpoint{4.496147in}{0.714120in}}%
\pgfpathlineto{\pgfqpoint{4.504525in}{0.714523in}}%
\pgfpathlineto{\pgfqpoint{4.512593in}{0.714926in}}%
\pgfpathlineto{\pgfqpoint{4.520372in}{0.715328in}}%
\pgfpathlineto{\pgfqpoint{4.527883in}{0.715731in}}%
\pgfpathlineto{\pgfqpoint{4.535144in}{0.716134in}}%
\pgfpathlineto{\pgfqpoint{4.542170in}{0.716536in}}%
\pgfpathlineto{\pgfqpoint{4.548977in}{0.716939in}}%
\pgfpathlineto{\pgfqpoint{4.555577in}{0.717341in}}%
\pgfpathlineto{\pgfqpoint{4.561983in}{0.717743in}}%
\pgfpathlineto{\pgfqpoint{4.568206in}{0.718146in}}%
\pgfpathlineto{\pgfqpoint{4.574256in}{0.718548in}}%
\pgfpathlineto{\pgfqpoint{4.580143in}{0.718949in}}%
\pgfpathlineto{\pgfqpoint{4.585874in}{0.719351in}}%
\pgfpathlineto{\pgfqpoint{4.591459in}{0.719753in}}%
\pgfpathlineto{\pgfqpoint{4.596904in}{0.720154in}}%
\pgfpathlineto{\pgfqpoint{4.602216in}{0.720555in}}%
\pgfpathlineto{\pgfqpoint{4.607401in}{0.720956in}}%
\pgfpathlineto{\pgfqpoint{4.612466in}{0.721357in}}%
\pgfpathlineto{\pgfqpoint{4.617416in}{0.721757in}}%
\pgfpathlineto{\pgfqpoint{4.622256in}{0.722157in}}%
\pgfpathlineto{\pgfqpoint{4.626991in}{0.722556in}}%
\pgfpathlineto{\pgfqpoint{4.631625in}{0.722956in}}%
\pgfpathlineto{\pgfqpoint{4.636162in}{0.723354in}}%
\pgfpathlineto{\pgfqpoint{4.640607in}{0.723753in}}%
\pgfusepath{stroke}%
\end{pgfscope}%
\begin{pgfscope}%
\pgfsetrectcap%
\pgfsetmiterjoin%
\pgfsetlinewidth{1.003750pt}%
\definecolor{currentstroke}{rgb}{0.150000,0.150000,0.150000}%
\pgfsetstrokecolor{currentstroke}%
\pgfsetdash{}{0pt}%
\pgfpathmoveto{\pgfqpoint{3.637789in}{0.435433in}}%
\pgfpathlineto{\pgfqpoint{3.637789in}{1.043381in}}%
\pgfusepath{stroke}%
\end{pgfscope}%
\begin{pgfscope}%
\pgfsetrectcap%
\pgfsetmiterjoin%
\pgfsetlinewidth{1.003750pt}%
\definecolor{currentstroke}{rgb}{0.150000,0.150000,0.150000}%
\pgfsetstrokecolor{currentstroke}%
\pgfsetdash{}{0pt}%
\pgfpathmoveto{\pgfqpoint{3.637789in}{0.435433in}}%
\pgfpathlineto{\pgfqpoint{4.860892in}{0.435433in}}%
\pgfusepath{stroke}%
\end{pgfscope}%
\begin{pgfscope}%
\pgfpathrectangle{\pgfqpoint{3.637789in}{0.435433in}}{\pgfqpoint{1.223103in}{0.607948in}}%
\pgfusepath{clip}%
\pgfsetbuttcap%
\pgfsetroundjoin%
\definecolor{currentfill}{rgb}{0.000000,0.000000,0.000000}%
\pgfsetfillcolor{currentfill}%
\pgfsetlinewidth{1.003750pt}%
\definecolor{currentstroke}{rgb}{0.000000,0.000000,0.000000}%
\pgfsetstrokecolor{currentstroke}%
\pgfsetdash{}{0pt}%
\pgfsys@defobject{currentmarker}{\pgfqpoint{-0.013889in}{-0.013889in}}{\pgfqpoint{0.013889in}{0.013889in}}{%
\pgfpathmoveto{\pgfqpoint{0.000000in}{-0.013889in}}%
\pgfpathcurveto{\pgfqpoint{0.003683in}{-0.013889in}}{\pgfqpoint{0.007216in}{-0.012425in}}{\pgfqpoint{0.009821in}{-0.009821in}}%
\pgfpathcurveto{\pgfqpoint{0.012425in}{-0.007216in}}{\pgfqpoint{0.013889in}{-0.003683in}}{\pgfqpoint{0.013889in}{0.000000in}}%
\pgfpathcurveto{\pgfqpoint{0.013889in}{0.003683in}}{\pgfqpoint{0.012425in}{0.007216in}}{\pgfqpoint{0.009821in}{0.009821in}}%
\pgfpathcurveto{\pgfqpoint{0.007216in}{0.012425in}}{\pgfqpoint{0.003683in}{0.013889in}}{\pgfqpoint{0.000000in}{0.013889in}}%
\pgfpathcurveto{\pgfqpoint{-0.003683in}{0.013889in}}{\pgfqpoint{-0.007216in}{0.012425in}}{\pgfqpoint{-0.009821in}{0.009821in}}%
\pgfpathcurveto{\pgfqpoint{-0.012425in}{0.007216in}}{\pgfqpoint{-0.013889in}{0.003683in}}{\pgfqpoint{-0.013889in}{0.000000in}}%
\pgfpathcurveto{\pgfqpoint{-0.013889in}{-0.003683in}}{\pgfqpoint{-0.012425in}{-0.007216in}}{\pgfqpoint{-0.009821in}{-0.009821in}}%
\pgfpathcurveto{\pgfqpoint{-0.007216in}{-0.012425in}}{\pgfqpoint{-0.003683in}{-0.013889in}}{\pgfqpoint{0.000000in}{-0.013889in}}%
\pgfpathclose%
\pgfusepath{stroke,fill}%
}%
\begin{pgfscope}%
\pgfsys@transformshift{4.290137in}{0.707850in}%
\pgfsys@useobject{currentmarker}{}%
\end{pgfscope}%
\begin{pgfscope}%
\pgfsys@transformshift{4.294536in}{0.707931in}%
\pgfsys@useobject{currentmarker}{}%
\end{pgfscope}%
\begin{pgfscope}%
\pgfsys@transformshift{4.299026in}{0.708015in}%
\pgfsys@useobject{currentmarker}{}%
\end{pgfscope}%
\begin{pgfscope}%
\pgfsys@transformshift{4.303611in}{0.708103in}%
\pgfsys@useobject{currentmarker}{}%
\end{pgfscope}%
\begin{pgfscope}%
\pgfsys@transformshift{4.308294in}{0.708195in}%
\pgfsys@useobject{currentmarker}{}%
\end{pgfscope}%
\begin{pgfscope}%
\pgfsys@transformshift{4.313080in}{0.708291in}%
\pgfsys@useobject{currentmarker}{}%
\end{pgfscope}%
\begin{pgfscope}%
\pgfsys@transformshift{4.317974in}{0.708391in}%
\pgfsys@useobject{currentmarker}{}%
\end{pgfscope}%
\begin{pgfscope}%
\pgfsys@transformshift{4.322980in}{0.708496in}%
\pgfsys@useobject{currentmarker}{}%
\end{pgfscope}%
\begin{pgfscope}%
\pgfsys@transformshift{4.328104in}{0.708606in}%
\pgfsys@useobject{currentmarker}{}%
\end{pgfscope}%
\begin{pgfscope}%
\pgfsys@transformshift{4.333351in}{0.708721in}%
\pgfsys@useobject{currentmarker}{}%
\end{pgfscope}%
\begin{pgfscope}%
\pgfsys@transformshift{4.338728in}{0.708842in}%
\pgfsys@useobject{currentmarker}{}%
\end{pgfscope}%
\begin{pgfscope}%
\pgfsys@transformshift{4.367806in}{0.709551in}%
\pgfsys@useobject{currentmarker}{}%
\end{pgfscope}%
\begin{pgfscope}%
\pgfsys@transformshift{4.401374in}{0.710498in}%
\pgfsys@useobject{currentmarker}{}%
\end{pgfscope}%
\begin{pgfscope}%
\pgfsys@transformshift{4.441076in}{0.711825in}%
\pgfsys@useobject{currentmarker}{}%
\end{pgfscope}%
\begin{pgfscope}%
\pgfsys@transformshift{4.489667in}{0.713819in}%
\pgfsys@useobject{currentmarker}{}%
\end{pgfscope}%
\begin{pgfscope}%
\pgfsys@transformshift{4.552313in}{0.717143in}%
\pgfsys@useobject{currentmarker}{}%
\end{pgfscope}%
\begin{pgfscope}%
\pgfsys@transformshift{4.640607in}{0.723752in}%
\pgfsys@useobject{currentmarker}{}%
\end{pgfscope}%
\end{pgfscope}%
\begin{pgfscope}%
\pgfsetbuttcap%
\pgfsetmiterjoin%
\definecolor{currentfill}{rgb}{1.000000,1.000000,1.000000}%
\pgfsetfillcolor{currentfill}%
\pgfsetlinewidth{0.000000pt}%
\definecolor{currentstroke}{rgb}{0.000000,0.000000,0.000000}%
\pgfsetstrokecolor{currentstroke}%
\pgfsetstrokeopacity{0.000000}%
\pgfsetdash{}{0pt}%
\pgfpathmoveto{\pgfqpoint{5.105513in}{0.435433in}}%
\pgfpathlineto{\pgfqpoint{6.328616in}{0.435433in}}%
\pgfpathlineto{\pgfqpoint{6.328616in}{1.043381in}}%
\pgfpathlineto{\pgfqpoint{5.105513in}{1.043381in}}%
\pgfpathclose%
\pgfusepath{fill}%
\end{pgfscope}%
\begin{pgfscope}%
\pgfpathrectangle{\pgfqpoint{5.105513in}{0.435433in}}{\pgfqpoint{1.223103in}{0.607948in}}%
\pgfusepath{clip}%
\pgfsetbuttcap%
\pgfsetmiterjoin%
\definecolor{currentfill}{rgb}{0.000000,0.000000,1.000000}%
\pgfsetfillcolor{currentfill}%
\pgfsetfillopacity{0.100000}%
\pgfsetlinewidth{0.803000pt}%
\definecolor{currentstroke}{rgb}{0.000000,0.000000,1.000000}%
\pgfsetstrokecolor{currentstroke}%
\pgfsetstrokeopacity{0.100000}%
\pgfsetdash{}{0pt}%
\pgfpathmoveto{\pgfqpoint{5.105513in}{0.703446in}}%
\pgfpathlineto{\pgfqpoint{5.105513in}{0.704175in}}%
\pgfpathlineto{\pgfqpoint{6.328616in}{0.704175in}}%
\pgfpathlineto{\pgfqpoint{6.328616in}{0.703446in}}%
\pgfpathclose%
\pgfusepath{stroke,fill}%
\end{pgfscope}%
\begin{pgfscope}%
\pgfpathrectangle{\pgfqpoint{5.105513in}{0.435433in}}{\pgfqpoint{1.223103in}{0.607948in}}%
\pgfusepath{clip}%
\pgfsetbuttcap%
\pgfsetroundjoin%
\definecolor{currentfill}{rgb}{0.000000,0.501961,0.000000}%
\pgfsetfillcolor{currentfill}%
\pgfsetfillopacity{0.500000}%
\pgfsetlinewidth{0.803000pt}%
\definecolor{currentstroke}{rgb}{0.000000,0.501961,0.000000}%
\pgfsetstrokecolor{currentstroke}%
\pgfsetstrokeopacity{0.500000}%
\pgfsetdash{}{0pt}%
\pgfpathmoveto{\pgfqpoint{5.105513in}{0.704382in}}%
\pgfpathlineto{\pgfqpoint{5.105513in}{0.703740in}}%
\pgfpathlineto{\pgfqpoint{5.346222in}{0.704318in}}%
\pgfpathlineto{\pgfqpoint{5.457753in}{0.704876in}}%
\pgfpathlineto{\pgfqpoint{5.531149in}{0.705418in}}%
\pgfpathlineto{\pgfqpoint{5.585945in}{0.705946in}}%
\pgfpathlineto{\pgfqpoint{5.629687in}{0.706463in}}%
\pgfpathlineto{\pgfqpoint{5.666095in}{0.706971in}}%
\pgfpathlineto{\pgfqpoint{5.697279in}{0.707473in}}%
\pgfpathlineto{\pgfqpoint{5.724552in}{0.707969in}}%
\pgfpathlineto{\pgfqpoint{5.748786in}{0.708461in}}%
\pgfpathlineto{\pgfqpoint{5.770591in}{0.708950in}}%
\pgfpathlineto{\pgfqpoint{5.790410in}{0.709436in}}%
\pgfpathlineto{\pgfqpoint{5.808575in}{0.709918in}}%
\pgfpathlineto{\pgfqpoint{5.825341in}{0.710402in}}%
\pgfpathlineto{\pgfqpoint{5.840907in}{0.710887in}}%
\pgfpathlineto{\pgfqpoint{5.855435in}{0.711373in}}%
\pgfpathlineto{\pgfqpoint{5.869053in}{0.711860in}}%
\pgfpathlineto{\pgfqpoint{5.881870in}{0.712349in}}%
\pgfpathlineto{\pgfqpoint{5.893974in}{0.712839in}}%
\pgfpathlineto{\pgfqpoint{5.905441in}{0.713329in}}%
\pgfpathlineto{\pgfqpoint{5.916334in}{0.713821in}}%
\pgfpathlineto{\pgfqpoint{5.926708in}{0.714313in}}%
\pgfpathlineto{\pgfqpoint{5.936610in}{0.714805in}}%
\pgfpathlineto{\pgfqpoint{5.946082in}{0.715299in}}%
\pgfpathlineto{\pgfqpoint{5.955158in}{0.715792in}}%
\pgfpathlineto{\pgfqpoint{5.963871in}{0.716286in}}%
\pgfpathlineto{\pgfqpoint{5.972249in}{0.716781in}}%
\pgfpathlineto{\pgfqpoint{5.980317in}{0.717275in}}%
\pgfpathlineto{\pgfqpoint{5.988096in}{0.717770in}}%
\pgfpathlineto{\pgfqpoint{5.995607in}{0.718264in}}%
\pgfpathlineto{\pgfqpoint{6.002868in}{0.718756in}}%
\pgfpathlineto{\pgfqpoint{6.009894in}{0.719246in}}%
\pgfpathlineto{\pgfqpoint{6.016701in}{0.719734in}}%
\pgfpathlineto{\pgfqpoint{6.023301in}{0.720221in}}%
\pgfpathlineto{\pgfqpoint{6.029707in}{0.720707in}}%
\pgfpathlineto{\pgfqpoint{6.035930in}{0.721192in}}%
\pgfpathlineto{\pgfqpoint{6.041980in}{0.721677in}}%
\pgfpathlineto{\pgfqpoint{6.047867in}{0.722164in}}%
\pgfpathlineto{\pgfqpoint{6.053598in}{0.722652in}}%
\pgfpathlineto{\pgfqpoint{6.059183in}{0.723144in}}%
\pgfpathlineto{\pgfqpoint{6.064628in}{0.723640in}}%
\pgfpathlineto{\pgfqpoint{6.069940in}{0.724142in}}%
\pgfpathlineto{\pgfqpoint{6.075126in}{0.724651in}}%
\pgfpathlineto{\pgfqpoint{6.080191in}{0.725171in}}%
\pgfpathlineto{\pgfqpoint{6.085140in}{0.725702in}}%
\pgfpathlineto{\pgfqpoint{6.089980in}{0.726248in}}%
\pgfpathlineto{\pgfqpoint{6.094715in}{0.726812in}}%
\pgfpathlineto{\pgfqpoint{6.099349in}{0.727396in}}%
\pgfpathlineto{\pgfqpoint{6.103886in}{0.728005in}}%
\pgfpathlineto{\pgfqpoint{6.108331in}{0.728629in}}%
\pgfpathlineto{\pgfqpoint{6.108331in}{0.728644in}}%
\pgfpathlineto{\pgfqpoint{6.108331in}{0.728644in}}%
\pgfpathlineto{\pgfqpoint{6.103886in}{0.728104in}}%
\pgfpathlineto{\pgfqpoint{6.099349in}{0.727575in}}%
\pgfpathlineto{\pgfqpoint{6.094715in}{0.727046in}}%
\pgfpathlineto{\pgfqpoint{6.089980in}{0.726518in}}%
\pgfpathlineto{\pgfqpoint{6.085140in}{0.725990in}}%
\pgfpathlineto{\pgfqpoint{6.080191in}{0.725463in}}%
\pgfpathlineto{\pgfqpoint{6.075126in}{0.724937in}}%
\pgfpathlineto{\pgfqpoint{6.069940in}{0.724413in}}%
\pgfpathlineto{\pgfqpoint{6.064628in}{0.723891in}}%
\pgfpathlineto{\pgfqpoint{6.059183in}{0.723371in}}%
\pgfpathlineto{\pgfqpoint{6.053598in}{0.722853in}}%
\pgfpathlineto{\pgfqpoint{6.047867in}{0.722337in}}%
\pgfpathlineto{\pgfqpoint{6.041980in}{0.721823in}}%
\pgfpathlineto{\pgfqpoint{6.035930in}{0.721312in}}%
\pgfpathlineto{\pgfqpoint{6.029707in}{0.720802in}}%
\pgfpathlineto{\pgfqpoint{6.023301in}{0.720296in}}%
\pgfpathlineto{\pgfqpoint{6.016701in}{0.719791in}}%
\pgfpathlineto{\pgfqpoint{6.009894in}{0.719288in}}%
\pgfpathlineto{\pgfqpoint{6.002868in}{0.718789in}}%
\pgfpathlineto{\pgfqpoint{5.995607in}{0.718291in}}%
\pgfpathlineto{\pgfqpoint{5.988096in}{0.717795in}}%
\pgfpathlineto{\pgfqpoint{5.980317in}{0.717301in}}%
\pgfpathlineto{\pgfqpoint{5.972249in}{0.716807in}}%
\pgfpathlineto{\pgfqpoint{5.963871in}{0.716312in}}%
\pgfpathlineto{\pgfqpoint{5.955158in}{0.715818in}}%
\pgfpathlineto{\pgfqpoint{5.946082in}{0.715324in}}%
\pgfpathlineto{\pgfqpoint{5.936610in}{0.714830in}}%
\pgfpathlineto{\pgfqpoint{5.926708in}{0.714337in}}%
\pgfpathlineto{\pgfqpoint{5.916334in}{0.713844in}}%
\pgfpathlineto{\pgfqpoint{5.905441in}{0.713352in}}%
\pgfpathlineto{\pgfqpoint{5.893974in}{0.712861in}}%
\pgfpathlineto{\pgfqpoint{5.881870in}{0.712370in}}%
\pgfpathlineto{\pgfqpoint{5.869053in}{0.711880in}}%
\pgfpathlineto{\pgfqpoint{5.855435in}{0.711391in}}%
\pgfpathlineto{\pgfqpoint{5.840907in}{0.710902in}}%
\pgfpathlineto{\pgfqpoint{5.825341in}{0.710414in}}%
\pgfpathlineto{\pgfqpoint{5.808575in}{0.709927in}}%
\pgfpathlineto{\pgfqpoint{5.790410in}{0.709440in}}%
\pgfpathlineto{\pgfqpoint{5.770591in}{0.708956in}}%
\pgfpathlineto{\pgfqpoint{5.748786in}{0.708478in}}%
\pgfpathlineto{\pgfqpoint{5.724552in}{0.708002in}}%
\pgfpathlineto{\pgfqpoint{5.697279in}{0.707530in}}%
\pgfpathlineto{\pgfqpoint{5.666095in}{0.707062in}}%
\pgfpathlineto{\pgfqpoint{5.629687in}{0.706598in}}%
\pgfpathlineto{\pgfqpoint{5.585945in}{0.706140in}}%
\pgfpathlineto{\pgfqpoint{5.531149in}{0.705688in}}%
\pgfpathlineto{\pgfqpoint{5.457753in}{0.705244in}}%
\pgfpathlineto{\pgfqpoint{5.346222in}{0.704808in}}%
\pgfpathlineto{\pgfqpoint{5.105513in}{0.704382in}}%
\pgfpathclose%
\pgfusepath{stroke,fill}%
\end{pgfscope}%
\begin{pgfscope}%
\pgfpathrectangle{\pgfqpoint{5.105513in}{0.435433in}}{\pgfqpoint{1.223103in}{0.607948in}}%
\pgfusepath{clip}%
\pgfsetroundcap%
\pgfsetroundjoin%
\pgfsetlinewidth{0.501875pt}%
\definecolor{currentstroke}{rgb}{0.000000,0.000000,1.000000}%
\pgfsetstrokecolor{currentstroke}%
\pgfsetstrokeopacity{0.800000}%
\pgfsetdash{}{0pt}%
\pgfpathmoveto{\pgfqpoint{5.105513in}{0.703810in}}%
\pgfpathlineto{\pgfqpoint{6.328616in}{0.703810in}}%
\pgfusepath{stroke}%
\end{pgfscope}%
\begin{pgfscope}%
\pgfpathrectangle{\pgfqpoint{5.105513in}{0.435433in}}{\pgfqpoint{1.223103in}{0.607948in}}%
\pgfusepath{clip}%
\pgfsetbuttcap%
\pgfsetroundjoin%
\pgfsetlinewidth{1.003750pt}%
\definecolor{currentstroke}{rgb}{0.000000,0.000000,0.000000}%
\pgfsetstrokecolor{currentstroke}%
\pgfsetdash{{3.700000pt}{1.600000pt}}{0.000000pt}%
\pgfpathmoveto{\pgfqpoint{5.105513in}{0.703897in}}%
\pgfpathlineto{\pgfqpoint{6.328616in}{0.703897in}}%
\pgfusepath{stroke}%
\end{pgfscope}%
\begin{pgfscope}%
\pgfsetroundcap%
\pgfsetroundjoin%
\pgfsetlinewidth{0.501875pt}%
\definecolor{currentstroke}{rgb}{0.000000,0.000000,1.000000}%
\pgfsetstrokecolor{currentstroke}%
\pgfsetstrokeopacity{0.800000}%
\pgfsetdash{}{0pt}%
\pgfpathmoveto{\pgfqpoint{5.932592in}{0.821914in}}%
\pgfpathquadraticcurveto{\pgfqpoint{5.856728in}{0.770637in}}{\pgfqpoint{5.780865in}{0.719359in}}%
\pgfusepath{stroke}%
\end{pgfscope}%
\begin{pgfscope}%
\pgfsetfillopacity{0.800000}%
\pgfsetstrokeopacity{0.800000}%
\definecolor{textcolor}{rgb}{0.000000,0.000000,1.000000}%
\pgfsetstrokecolor{textcolor}%
\pgfsetfillcolor{textcolor}%
\pgftext[x=5.846155in,y=0.886195in,left,base]{\color{textcolor}\sffamily\fontsize{5.647059}{6.776471}\selectfont 1.44145(60)}%
\end{pgfscope}%
\begin{pgfscope}%
\pgfsetbuttcap%
\pgfsetroundjoin%
\definecolor{currentfill}{rgb}{0.150000,0.150000,0.150000}%
\pgfsetfillcolor{currentfill}%
\pgfsetlinewidth{1.003750pt}%
\definecolor{currentstroke}{rgb}{0.150000,0.150000,0.150000}%
\pgfsetstrokecolor{currentstroke}%
\pgfsetdash{}{0pt}%
\pgfsys@defobject{currentmarker}{\pgfqpoint{0.000000in}{-0.066667in}}{\pgfqpoint{0.000000in}{0.000000in}}{%
\pgfpathmoveto{\pgfqpoint{0.000000in}{0.000000in}}%
\pgfpathlineto{\pgfqpoint{0.000000in}{-0.066667in}}%
\pgfusepath{stroke,fill}%
}%
\begin{pgfscope}%
\pgfsys@transformshift{5.105513in}{0.435433in}%
\pgfsys@useobject{currentmarker}{}%
\end{pgfscope}%
\end{pgfscope}%
\begin{pgfscope}%
\definecolor{textcolor}{rgb}{0.150000,0.150000,0.150000}%
\pgfsetstrokecolor{textcolor}%
\pgfsetfillcolor{textcolor}%
\pgftext[x=5.105513in,y=0.320155in,,top]{\color{textcolor}\sffamily\fontsize{5.176471}{6.211765}\selectfont \(\displaystyle {10^{-3}}\)}%
\end{pgfscope}%
\begin{pgfscope}%
\pgfsetbuttcap%
\pgfsetroundjoin%
\definecolor{currentfill}{rgb}{0.150000,0.150000,0.150000}%
\pgfsetfillcolor{currentfill}%
\pgfsetlinewidth{1.003750pt}%
\definecolor{currentstroke}{rgb}{0.150000,0.150000,0.150000}%
\pgfsetstrokecolor{currentstroke}%
\pgfsetdash{}{0pt}%
\pgfsys@defobject{currentmarker}{\pgfqpoint{0.000000in}{-0.066667in}}{\pgfqpoint{0.000000in}{0.000000in}}{%
\pgfpathmoveto{\pgfqpoint{0.000000in}{0.000000in}}%
\pgfpathlineto{\pgfqpoint{0.000000in}{-0.066667in}}%
\pgfusepath{stroke,fill}%
}%
\begin{pgfscope}%
\pgfsys@transformshift{5.606922in}{0.435433in}%
\pgfsys@useobject{currentmarker}{}%
\end{pgfscope}%
\end{pgfscope}%
\begin{pgfscope}%
\definecolor{textcolor}{rgb}{0.150000,0.150000,0.150000}%
\pgfsetstrokecolor{textcolor}%
\pgfsetfillcolor{textcolor}%
\pgftext[x=5.606922in,y=0.320155in,,top]{\color{textcolor}\sffamily\fontsize{5.176471}{6.211765}\selectfont \(\displaystyle {10^{-2}}\)}%
\end{pgfscope}%
\begin{pgfscope}%
\pgfsetbuttcap%
\pgfsetroundjoin%
\definecolor{currentfill}{rgb}{0.150000,0.150000,0.150000}%
\pgfsetfillcolor{currentfill}%
\pgfsetlinewidth{1.003750pt}%
\definecolor{currentstroke}{rgb}{0.150000,0.150000,0.150000}%
\pgfsetstrokecolor{currentstroke}%
\pgfsetdash{}{0pt}%
\pgfsys@defobject{currentmarker}{\pgfqpoint{0.000000in}{-0.066667in}}{\pgfqpoint{0.000000in}{0.000000in}}{%
\pgfpathmoveto{\pgfqpoint{0.000000in}{0.000000in}}%
\pgfpathlineto{\pgfqpoint{0.000000in}{-0.066667in}}%
\pgfusepath{stroke,fill}%
}%
\begin{pgfscope}%
\pgfsys@transformshift{6.108331in}{0.435433in}%
\pgfsys@useobject{currentmarker}{}%
\end{pgfscope}%
\end{pgfscope}%
\begin{pgfscope}%
\definecolor{textcolor}{rgb}{0.150000,0.150000,0.150000}%
\pgfsetstrokecolor{textcolor}%
\pgfsetfillcolor{textcolor}%
\pgftext[x=6.108331in,y=0.320155in,,top]{\color{textcolor}\sffamily\fontsize{5.176471}{6.211765}\selectfont \(\displaystyle {10^{-1}}\)}%
\end{pgfscope}%
\begin{pgfscope}%
\pgfsetbuttcap%
\pgfsetroundjoin%
\definecolor{currentfill}{rgb}{0.150000,0.150000,0.150000}%
\pgfsetfillcolor{currentfill}%
\pgfsetlinewidth{0.803000pt}%
\definecolor{currentstroke}{rgb}{0.150000,0.150000,0.150000}%
\pgfsetstrokecolor{currentstroke}%
\pgfsetdash{}{0pt}%
\pgfsys@defobject{currentmarker}{\pgfqpoint{0.000000in}{-0.044444in}}{\pgfqpoint{0.000000in}{0.000000in}}{%
\pgfpathmoveto{\pgfqpoint{0.000000in}{0.000000in}}%
\pgfpathlineto{\pgfqpoint{0.000000in}{-0.044444in}}%
\pgfusepath{stroke,fill}%
}%
\begin{pgfscope}%
\pgfsys@transformshift{5.256452in}{0.435433in}%
\pgfsys@useobject{currentmarker}{}%
\end{pgfscope}%
\end{pgfscope}%
\begin{pgfscope}%
\pgfsetbuttcap%
\pgfsetroundjoin%
\definecolor{currentfill}{rgb}{0.150000,0.150000,0.150000}%
\pgfsetfillcolor{currentfill}%
\pgfsetlinewidth{0.803000pt}%
\definecolor{currentstroke}{rgb}{0.150000,0.150000,0.150000}%
\pgfsetstrokecolor{currentstroke}%
\pgfsetdash{}{0pt}%
\pgfsys@defobject{currentmarker}{\pgfqpoint{0.000000in}{-0.044444in}}{\pgfqpoint{0.000000in}{0.000000in}}{%
\pgfpathmoveto{\pgfqpoint{0.000000in}{0.000000in}}%
\pgfpathlineto{\pgfqpoint{0.000000in}{-0.044444in}}%
\pgfusepath{stroke,fill}%
}%
\begin{pgfscope}%
\pgfsys@transformshift{5.344746in}{0.435433in}%
\pgfsys@useobject{currentmarker}{}%
\end{pgfscope}%
\end{pgfscope}%
\begin{pgfscope}%
\pgfsetbuttcap%
\pgfsetroundjoin%
\definecolor{currentfill}{rgb}{0.150000,0.150000,0.150000}%
\pgfsetfillcolor{currentfill}%
\pgfsetlinewidth{0.803000pt}%
\definecolor{currentstroke}{rgb}{0.150000,0.150000,0.150000}%
\pgfsetstrokecolor{currentstroke}%
\pgfsetdash{}{0pt}%
\pgfsys@defobject{currentmarker}{\pgfqpoint{0.000000in}{-0.044444in}}{\pgfqpoint{0.000000in}{0.000000in}}{%
\pgfpathmoveto{\pgfqpoint{0.000000in}{0.000000in}}%
\pgfpathlineto{\pgfqpoint{0.000000in}{-0.044444in}}%
\pgfusepath{stroke,fill}%
}%
\begin{pgfscope}%
\pgfsys@transformshift{5.407391in}{0.435433in}%
\pgfsys@useobject{currentmarker}{}%
\end{pgfscope}%
\end{pgfscope}%
\begin{pgfscope}%
\pgfsetbuttcap%
\pgfsetroundjoin%
\definecolor{currentfill}{rgb}{0.150000,0.150000,0.150000}%
\pgfsetfillcolor{currentfill}%
\pgfsetlinewidth{0.803000pt}%
\definecolor{currentstroke}{rgb}{0.150000,0.150000,0.150000}%
\pgfsetstrokecolor{currentstroke}%
\pgfsetdash{}{0pt}%
\pgfsys@defobject{currentmarker}{\pgfqpoint{0.000000in}{-0.044444in}}{\pgfqpoint{0.000000in}{0.000000in}}{%
\pgfpathmoveto{\pgfqpoint{0.000000in}{0.000000in}}%
\pgfpathlineto{\pgfqpoint{0.000000in}{-0.044444in}}%
\pgfusepath{stroke,fill}%
}%
\begin{pgfscope}%
\pgfsys@transformshift{5.455982in}{0.435433in}%
\pgfsys@useobject{currentmarker}{}%
\end{pgfscope}%
\end{pgfscope}%
\begin{pgfscope}%
\pgfsetbuttcap%
\pgfsetroundjoin%
\definecolor{currentfill}{rgb}{0.150000,0.150000,0.150000}%
\pgfsetfillcolor{currentfill}%
\pgfsetlinewidth{0.803000pt}%
\definecolor{currentstroke}{rgb}{0.150000,0.150000,0.150000}%
\pgfsetstrokecolor{currentstroke}%
\pgfsetdash{}{0pt}%
\pgfsys@defobject{currentmarker}{\pgfqpoint{0.000000in}{-0.044444in}}{\pgfqpoint{0.000000in}{0.000000in}}{%
\pgfpathmoveto{\pgfqpoint{0.000000in}{0.000000in}}%
\pgfpathlineto{\pgfqpoint{0.000000in}{-0.044444in}}%
\pgfusepath{stroke,fill}%
}%
\begin{pgfscope}%
\pgfsys@transformshift{5.495685in}{0.435433in}%
\pgfsys@useobject{currentmarker}{}%
\end{pgfscope}%
\end{pgfscope}%
\begin{pgfscope}%
\pgfsetbuttcap%
\pgfsetroundjoin%
\definecolor{currentfill}{rgb}{0.150000,0.150000,0.150000}%
\pgfsetfillcolor{currentfill}%
\pgfsetlinewidth{0.803000pt}%
\definecolor{currentstroke}{rgb}{0.150000,0.150000,0.150000}%
\pgfsetstrokecolor{currentstroke}%
\pgfsetdash{}{0pt}%
\pgfsys@defobject{currentmarker}{\pgfqpoint{0.000000in}{-0.044444in}}{\pgfqpoint{0.000000in}{0.000000in}}{%
\pgfpathmoveto{\pgfqpoint{0.000000in}{0.000000in}}%
\pgfpathlineto{\pgfqpoint{0.000000in}{-0.044444in}}%
\pgfusepath{stroke,fill}%
}%
\begin{pgfscope}%
\pgfsys@transformshift{5.529252in}{0.435433in}%
\pgfsys@useobject{currentmarker}{}%
\end{pgfscope}%
\end{pgfscope}%
\begin{pgfscope}%
\pgfsetbuttcap%
\pgfsetroundjoin%
\definecolor{currentfill}{rgb}{0.150000,0.150000,0.150000}%
\pgfsetfillcolor{currentfill}%
\pgfsetlinewidth{0.803000pt}%
\definecolor{currentstroke}{rgb}{0.150000,0.150000,0.150000}%
\pgfsetstrokecolor{currentstroke}%
\pgfsetdash{}{0pt}%
\pgfsys@defobject{currentmarker}{\pgfqpoint{0.000000in}{-0.044444in}}{\pgfqpoint{0.000000in}{0.000000in}}{%
\pgfpathmoveto{\pgfqpoint{0.000000in}{0.000000in}}%
\pgfpathlineto{\pgfqpoint{0.000000in}{-0.044444in}}%
\pgfusepath{stroke,fill}%
}%
\begin{pgfscope}%
\pgfsys@transformshift{5.558330in}{0.435433in}%
\pgfsys@useobject{currentmarker}{}%
\end{pgfscope}%
\end{pgfscope}%
\begin{pgfscope}%
\pgfsetbuttcap%
\pgfsetroundjoin%
\definecolor{currentfill}{rgb}{0.150000,0.150000,0.150000}%
\pgfsetfillcolor{currentfill}%
\pgfsetlinewidth{0.803000pt}%
\definecolor{currentstroke}{rgb}{0.150000,0.150000,0.150000}%
\pgfsetstrokecolor{currentstroke}%
\pgfsetdash{}{0pt}%
\pgfsys@defobject{currentmarker}{\pgfqpoint{0.000000in}{-0.044444in}}{\pgfqpoint{0.000000in}{0.000000in}}{%
\pgfpathmoveto{\pgfqpoint{0.000000in}{0.000000in}}%
\pgfpathlineto{\pgfqpoint{0.000000in}{-0.044444in}}%
\pgfusepath{stroke,fill}%
}%
\begin{pgfscope}%
\pgfsys@transformshift{5.583978in}{0.435433in}%
\pgfsys@useobject{currentmarker}{}%
\end{pgfscope}%
\end{pgfscope}%
\begin{pgfscope}%
\pgfsetbuttcap%
\pgfsetroundjoin%
\definecolor{currentfill}{rgb}{0.150000,0.150000,0.150000}%
\pgfsetfillcolor{currentfill}%
\pgfsetlinewidth{0.803000pt}%
\definecolor{currentstroke}{rgb}{0.150000,0.150000,0.150000}%
\pgfsetstrokecolor{currentstroke}%
\pgfsetdash{}{0pt}%
\pgfsys@defobject{currentmarker}{\pgfqpoint{0.000000in}{-0.044444in}}{\pgfqpoint{0.000000in}{0.000000in}}{%
\pgfpathmoveto{\pgfqpoint{0.000000in}{0.000000in}}%
\pgfpathlineto{\pgfqpoint{0.000000in}{-0.044444in}}%
\pgfusepath{stroke,fill}%
}%
\begin{pgfscope}%
\pgfsys@transformshift{5.757861in}{0.435433in}%
\pgfsys@useobject{currentmarker}{}%
\end{pgfscope}%
\end{pgfscope}%
\begin{pgfscope}%
\pgfsetbuttcap%
\pgfsetroundjoin%
\definecolor{currentfill}{rgb}{0.150000,0.150000,0.150000}%
\pgfsetfillcolor{currentfill}%
\pgfsetlinewidth{0.803000pt}%
\definecolor{currentstroke}{rgb}{0.150000,0.150000,0.150000}%
\pgfsetstrokecolor{currentstroke}%
\pgfsetdash{}{0pt}%
\pgfsys@defobject{currentmarker}{\pgfqpoint{0.000000in}{-0.044444in}}{\pgfqpoint{0.000000in}{0.000000in}}{%
\pgfpathmoveto{\pgfqpoint{0.000000in}{0.000000in}}%
\pgfpathlineto{\pgfqpoint{0.000000in}{-0.044444in}}%
\pgfusepath{stroke,fill}%
}%
\begin{pgfscope}%
\pgfsys@transformshift{5.846155in}{0.435433in}%
\pgfsys@useobject{currentmarker}{}%
\end{pgfscope}%
\end{pgfscope}%
\begin{pgfscope}%
\pgfsetbuttcap%
\pgfsetroundjoin%
\definecolor{currentfill}{rgb}{0.150000,0.150000,0.150000}%
\pgfsetfillcolor{currentfill}%
\pgfsetlinewidth{0.803000pt}%
\definecolor{currentstroke}{rgb}{0.150000,0.150000,0.150000}%
\pgfsetstrokecolor{currentstroke}%
\pgfsetdash{}{0pt}%
\pgfsys@defobject{currentmarker}{\pgfqpoint{0.000000in}{-0.044444in}}{\pgfqpoint{0.000000in}{0.000000in}}{%
\pgfpathmoveto{\pgfqpoint{0.000000in}{0.000000in}}%
\pgfpathlineto{\pgfqpoint{0.000000in}{-0.044444in}}%
\pgfusepath{stroke,fill}%
}%
\begin{pgfscope}%
\pgfsys@transformshift{5.908800in}{0.435433in}%
\pgfsys@useobject{currentmarker}{}%
\end{pgfscope}%
\end{pgfscope}%
\begin{pgfscope}%
\pgfsetbuttcap%
\pgfsetroundjoin%
\definecolor{currentfill}{rgb}{0.150000,0.150000,0.150000}%
\pgfsetfillcolor{currentfill}%
\pgfsetlinewidth{0.803000pt}%
\definecolor{currentstroke}{rgb}{0.150000,0.150000,0.150000}%
\pgfsetstrokecolor{currentstroke}%
\pgfsetdash{}{0pt}%
\pgfsys@defobject{currentmarker}{\pgfqpoint{0.000000in}{-0.044444in}}{\pgfqpoint{0.000000in}{0.000000in}}{%
\pgfpathmoveto{\pgfqpoint{0.000000in}{0.000000in}}%
\pgfpathlineto{\pgfqpoint{0.000000in}{-0.044444in}}%
\pgfusepath{stroke,fill}%
}%
\begin{pgfscope}%
\pgfsys@transformshift{5.957391in}{0.435433in}%
\pgfsys@useobject{currentmarker}{}%
\end{pgfscope}%
\end{pgfscope}%
\begin{pgfscope}%
\pgfsetbuttcap%
\pgfsetroundjoin%
\definecolor{currentfill}{rgb}{0.150000,0.150000,0.150000}%
\pgfsetfillcolor{currentfill}%
\pgfsetlinewidth{0.803000pt}%
\definecolor{currentstroke}{rgb}{0.150000,0.150000,0.150000}%
\pgfsetstrokecolor{currentstroke}%
\pgfsetdash{}{0pt}%
\pgfsys@defobject{currentmarker}{\pgfqpoint{0.000000in}{-0.044444in}}{\pgfqpoint{0.000000in}{0.000000in}}{%
\pgfpathmoveto{\pgfqpoint{0.000000in}{0.000000in}}%
\pgfpathlineto{\pgfqpoint{0.000000in}{-0.044444in}}%
\pgfusepath{stroke,fill}%
}%
\begin{pgfscope}%
\pgfsys@transformshift{5.997094in}{0.435433in}%
\pgfsys@useobject{currentmarker}{}%
\end{pgfscope}%
\end{pgfscope}%
\begin{pgfscope}%
\pgfsetbuttcap%
\pgfsetroundjoin%
\definecolor{currentfill}{rgb}{0.150000,0.150000,0.150000}%
\pgfsetfillcolor{currentfill}%
\pgfsetlinewidth{0.803000pt}%
\definecolor{currentstroke}{rgb}{0.150000,0.150000,0.150000}%
\pgfsetstrokecolor{currentstroke}%
\pgfsetdash{}{0pt}%
\pgfsys@defobject{currentmarker}{\pgfqpoint{0.000000in}{-0.044444in}}{\pgfqpoint{0.000000in}{0.000000in}}{%
\pgfpathmoveto{\pgfqpoint{0.000000in}{0.000000in}}%
\pgfpathlineto{\pgfqpoint{0.000000in}{-0.044444in}}%
\pgfusepath{stroke,fill}%
}%
\begin{pgfscope}%
\pgfsys@transformshift{6.030661in}{0.435433in}%
\pgfsys@useobject{currentmarker}{}%
\end{pgfscope}%
\end{pgfscope}%
\begin{pgfscope}%
\pgfsetbuttcap%
\pgfsetroundjoin%
\definecolor{currentfill}{rgb}{0.150000,0.150000,0.150000}%
\pgfsetfillcolor{currentfill}%
\pgfsetlinewidth{0.803000pt}%
\definecolor{currentstroke}{rgb}{0.150000,0.150000,0.150000}%
\pgfsetstrokecolor{currentstroke}%
\pgfsetdash{}{0pt}%
\pgfsys@defobject{currentmarker}{\pgfqpoint{0.000000in}{-0.044444in}}{\pgfqpoint{0.000000in}{0.000000in}}{%
\pgfpathmoveto{\pgfqpoint{0.000000in}{0.000000in}}%
\pgfpathlineto{\pgfqpoint{0.000000in}{-0.044444in}}%
\pgfusepath{stroke,fill}%
}%
\begin{pgfscope}%
\pgfsys@transformshift{6.059739in}{0.435433in}%
\pgfsys@useobject{currentmarker}{}%
\end{pgfscope}%
\end{pgfscope}%
\begin{pgfscope}%
\pgfsetbuttcap%
\pgfsetroundjoin%
\definecolor{currentfill}{rgb}{0.150000,0.150000,0.150000}%
\pgfsetfillcolor{currentfill}%
\pgfsetlinewidth{0.803000pt}%
\definecolor{currentstroke}{rgb}{0.150000,0.150000,0.150000}%
\pgfsetstrokecolor{currentstroke}%
\pgfsetdash{}{0pt}%
\pgfsys@defobject{currentmarker}{\pgfqpoint{0.000000in}{-0.044444in}}{\pgfqpoint{0.000000in}{0.000000in}}{%
\pgfpathmoveto{\pgfqpoint{0.000000in}{0.000000in}}%
\pgfpathlineto{\pgfqpoint{0.000000in}{-0.044444in}}%
\pgfusepath{stroke,fill}%
}%
\begin{pgfscope}%
\pgfsys@transformshift{6.085387in}{0.435433in}%
\pgfsys@useobject{currentmarker}{}%
\end{pgfscope}%
\end{pgfscope}%
\begin{pgfscope}%
\pgfsetbuttcap%
\pgfsetroundjoin%
\definecolor{currentfill}{rgb}{0.150000,0.150000,0.150000}%
\pgfsetfillcolor{currentfill}%
\pgfsetlinewidth{0.803000pt}%
\definecolor{currentstroke}{rgb}{0.150000,0.150000,0.150000}%
\pgfsetstrokecolor{currentstroke}%
\pgfsetdash{}{0pt}%
\pgfsys@defobject{currentmarker}{\pgfqpoint{0.000000in}{-0.044444in}}{\pgfqpoint{0.000000in}{0.000000in}}{%
\pgfpathmoveto{\pgfqpoint{0.000000in}{0.000000in}}%
\pgfpathlineto{\pgfqpoint{0.000000in}{-0.044444in}}%
\pgfusepath{stroke,fill}%
}%
\begin{pgfscope}%
\pgfsys@transformshift{6.259270in}{0.435433in}%
\pgfsys@useobject{currentmarker}{}%
\end{pgfscope}%
\end{pgfscope}%
\begin{pgfscope}%
\definecolor{textcolor}{rgb}{0.150000,0.150000,0.150000}%
\pgfsetstrokecolor{textcolor}%
\pgfsetfillcolor{textcolor}%
\pgftext[x=5.717064in,y=0.183333in,,top]{\color{textcolor}\sffamily\fontsize{5.647059}{6.776471}\selectfont \(\displaystyle \epsilon [\mathrm{fm}]\)}%
\end{pgfscope}%
\begin{pgfscope}%
\pgfsetbuttcap%
\pgfsetroundjoin%
\definecolor{currentfill}{rgb}{0.150000,0.150000,0.150000}%
\pgfsetfillcolor{currentfill}%
\pgfsetlinewidth{1.003750pt}%
\definecolor{currentstroke}{rgb}{0.150000,0.150000,0.150000}%
\pgfsetstrokecolor{currentstroke}%
\pgfsetdash{}{0pt}%
\pgfsys@defobject{currentmarker}{\pgfqpoint{-0.066667in}{0.000000in}}{\pgfqpoint{0.000000in}{0.000000in}}{%
\pgfpathmoveto{\pgfqpoint{0.000000in}{0.000000in}}%
\pgfpathlineto{\pgfqpoint{-0.066667in}{0.000000in}}%
\pgfusepath{stroke,fill}%
}%
\begin{pgfscope}%
\pgfsys@transformshift{5.105513in}{0.435433in}%
\pgfsys@useobject{currentmarker}{}%
\end{pgfscope}%
\end{pgfscope}%
\begin{pgfscope}%
\pgfsetbuttcap%
\pgfsetroundjoin%
\definecolor{currentfill}{rgb}{0.150000,0.150000,0.150000}%
\pgfsetfillcolor{currentfill}%
\pgfsetlinewidth{1.003750pt}%
\definecolor{currentstroke}{rgb}{0.150000,0.150000,0.150000}%
\pgfsetstrokecolor{currentstroke}%
\pgfsetdash{}{0pt}%
\pgfsys@defobject{currentmarker}{\pgfqpoint{-0.066667in}{0.000000in}}{\pgfqpoint{0.000000in}{0.000000in}}{%
\pgfpathmoveto{\pgfqpoint{0.000000in}{0.000000in}}%
\pgfpathlineto{\pgfqpoint{-0.066667in}{0.000000in}}%
\pgfusepath{stroke,fill}%
}%
\begin{pgfscope}%
\pgfsys@transformshift{5.105513in}{0.703897in}%
\pgfsys@useobject{currentmarker}{}%
\end{pgfscope}%
\end{pgfscope}%
\begin{pgfscope}%
\pgfsetbuttcap%
\pgfsetroundjoin%
\definecolor{currentfill}{rgb}{0.150000,0.150000,0.150000}%
\pgfsetfillcolor{currentfill}%
\pgfsetlinewidth{1.003750pt}%
\definecolor{currentstroke}{rgb}{0.150000,0.150000,0.150000}%
\pgfsetstrokecolor{currentstroke}%
\pgfsetdash{}{0pt}%
\pgfsys@defobject{currentmarker}{\pgfqpoint{-0.066667in}{0.000000in}}{\pgfqpoint{0.000000in}{0.000000in}}{%
\pgfpathmoveto{\pgfqpoint{0.000000in}{0.000000in}}%
\pgfpathlineto{\pgfqpoint{-0.066667in}{0.000000in}}%
\pgfusepath{stroke,fill}%
}%
\begin{pgfscope}%
\pgfsys@transformshift{5.105513in}{1.043381in}%
\pgfsys@useobject{currentmarker}{}%
\end{pgfscope}%
\end{pgfscope}%
\begin{pgfscope}%
\pgfpathrectangle{\pgfqpoint{5.105513in}{0.435433in}}{\pgfqpoint{1.223103in}{0.607948in}}%
\pgfusepath{clip}%
\pgfsetroundcap%
\pgfsetroundjoin%
\pgfsetlinewidth{1.204500pt}%
\definecolor{currentstroke}{rgb}{0.000000,0.501961,0.000000}%
\pgfsetstrokecolor{currentstroke}%
\pgfsetdash{}{0pt}%
\pgfpathmoveto{\pgfqpoint{5.105513in}{0.704061in}}%
\pgfpathlineto{\pgfqpoint{5.346222in}{0.704563in}}%
\pgfpathlineto{\pgfqpoint{5.457753in}{0.705060in}}%
\pgfpathlineto{\pgfqpoint{5.531149in}{0.705553in}}%
\pgfpathlineto{\pgfqpoint{5.585945in}{0.706043in}}%
\pgfpathlineto{\pgfqpoint{5.629687in}{0.706531in}}%
\pgfpathlineto{\pgfqpoint{5.666095in}{0.707017in}}%
\pgfpathlineto{\pgfqpoint{5.697279in}{0.707501in}}%
\pgfpathlineto{\pgfqpoint{5.724552in}{0.707986in}}%
\pgfpathlineto{\pgfqpoint{5.748786in}{0.708469in}}%
\pgfpathlineto{\pgfqpoint{5.770591in}{0.708953in}}%
\pgfpathlineto{\pgfqpoint{5.790410in}{0.709438in}}%
\pgfpathlineto{\pgfqpoint{5.808575in}{0.709923in}}%
\pgfpathlineto{\pgfqpoint{5.825341in}{0.710408in}}%
\pgfpathlineto{\pgfqpoint{5.840907in}{0.710895in}}%
\pgfpathlineto{\pgfqpoint{5.855435in}{0.711382in}}%
\pgfpathlineto{\pgfqpoint{5.869053in}{0.711870in}}%
\pgfpathlineto{\pgfqpoint{5.881870in}{0.712359in}}%
\pgfpathlineto{\pgfqpoint{5.893974in}{0.712850in}}%
\pgfpathlineto{\pgfqpoint{5.905441in}{0.713341in}}%
\pgfpathlineto{\pgfqpoint{5.916334in}{0.713832in}}%
\pgfpathlineto{\pgfqpoint{5.926708in}{0.714325in}}%
\pgfpathlineto{\pgfqpoint{5.936610in}{0.714818in}}%
\pgfpathlineto{\pgfqpoint{5.946082in}{0.715311in}}%
\pgfpathlineto{\pgfqpoint{5.955158in}{0.715805in}}%
\pgfpathlineto{\pgfqpoint{5.963871in}{0.716299in}}%
\pgfpathlineto{\pgfqpoint{5.972249in}{0.716794in}}%
\pgfpathlineto{\pgfqpoint{5.980317in}{0.717288in}}%
\pgfpathlineto{\pgfqpoint{5.988096in}{0.717783in}}%
\pgfpathlineto{\pgfqpoint{5.995607in}{0.718277in}}%
\pgfpathlineto{\pgfqpoint{6.002868in}{0.718772in}}%
\pgfpathlineto{\pgfqpoint{6.009894in}{0.719267in}}%
\pgfpathlineto{\pgfqpoint{6.016701in}{0.719762in}}%
\pgfpathlineto{\pgfqpoint{6.023301in}{0.720258in}}%
\pgfpathlineto{\pgfqpoint{6.029707in}{0.720755in}}%
\pgfpathlineto{\pgfqpoint{6.035930in}{0.721252in}}%
\pgfpathlineto{\pgfqpoint{6.041980in}{0.721750in}}%
\pgfpathlineto{\pgfqpoint{6.047867in}{0.722250in}}%
\pgfpathlineto{\pgfqpoint{6.053598in}{0.722752in}}%
\pgfpathlineto{\pgfqpoint{6.059183in}{0.723257in}}%
\pgfpathlineto{\pgfqpoint{6.064628in}{0.723765in}}%
\pgfpathlineto{\pgfqpoint{6.069940in}{0.724277in}}%
\pgfpathlineto{\pgfqpoint{6.075126in}{0.724794in}}%
\pgfpathlineto{\pgfqpoint{6.080191in}{0.725317in}}%
\pgfpathlineto{\pgfqpoint{6.085140in}{0.725846in}}%
\pgfpathlineto{\pgfqpoint{6.089980in}{0.726383in}}%
\pgfpathlineto{\pgfqpoint{6.094715in}{0.726929in}}%
\pgfpathlineto{\pgfqpoint{6.099349in}{0.727486in}}%
\pgfpathlineto{\pgfqpoint{6.103886in}{0.728054in}}%
\pgfpathlineto{\pgfqpoint{6.108331in}{0.728636in}}%
\pgfusepath{stroke}%
\end{pgfscope}%
\begin{pgfscope}%
\pgfsetrectcap%
\pgfsetmiterjoin%
\pgfsetlinewidth{1.003750pt}%
\definecolor{currentstroke}{rgb}{0.150000,0.150000,0.150000}%
\pgfsetstrokecolor{currentstroke}%
\pgfsetdash{}{0pt}%
\pgfpathmoveto{\pgfqpoint{5.105513in}{0.435433in}}%
\pgfpathlineto{\pgfqpoint{5.105513in}{1.043381in}}%
\pgfusepath{stroke}%
\end{pgfscope}%
\begin{pgfscope}%
\pgfsetrectcap%
\pgfsetmiterjoin%
\pgfsetlinewidth{1.003750pt}%
\definecolor{currentstroke}{rgb}{0.150000,0.150000,0.150000}%
\pgfsetstrokecolor{currentstroke}%
\pgfsetdash{}{0pt}%
\pgfpathmoveto{\pgfqpoint{5.105513in}{0.435433in}}%
\pgfpathlineto{\pgfqpoint{6.328616in}{0.435433in}}%
\pgfusepath{stroke}%
\end{pgfscope}%
\begin{pgfscope}%
\pgfpathrectangle{\pgfqpoint{5.105513in}{0.435433in}}{\pgfqpoint{1.223103in}{0.607948in}}%
\pgfusepath{clip}%
\pgfsetbuttcap%
\pgfsetroundjoin%
\definecolor{currentfill}{rgb}{0.000000,0.000000,0.000000}%
\pgfsetfillcolor{currentfill}%
\pgfsetlinewidth{1.003750pt}%
\definecolor{currentstroke}{rgb}{0.000000,0.000000,0.000000}%
\pgfsetstrokecolor{currentstroke}%
\pgfsetdash{}{0pt}%
\pgfsys@defobject{currentmarker}{\pgfqpoint{-0.013889in}{-0.013889in}}{\pgfqpoint{0.013889in}{0.013889in}}{%
\pgfpathmoveto{\pgfqpoint{0.000000in}{-0.013889in}}%
\pgfpathcurveto{\pgfqpoint{0.003683in}{-0.013889in}}{\pgfqpoint{0.007216in}{-0.012425in}}{\pgfqpoint{0.009821in}{-0.009821in}}%
\pgfpathcurveto{\pgfqpoint{0.012425in}{-0.007216in}}{\pgfqpoint{0.013889in}{-0.003683in}}{\pgfqpoint{0.013889in}{0.000000in}}%
\pgfpathcurveto{\pgfqpoint{0.013889in}{0.003683in}}{\pgfqpoint{0.012425in}{0.007216in}}{\pgfqpoint{0.009821in}{0.009821in}}%
\pgfpathcurveto{\pgfqpoint{0.007216in}{0.012425in}}{\pgfqpoint{0.003683in}{0.013889in}}{\pgfqpoint{0.000000in}{0.013889in}}%
\pgfpathcurveto{\pgfqpoint{-0.003683in}{0.013889in}}{\pgfqpoint{-0.007216in}{0.012425in}}{\pgfqpoint{-0.009821in}{0.009821in}}%
\pgfpathcurveto{\pgfqpoint{-0.012425in}{0.007216in}}{\pgfqpoint{-0.013889in}{0.003683in}}{\pgfqpoint{-0.013889in}{0.000000in}}%
\pgfpathcurveto{\pgfqpoint{-0.013889in}{-0.003683in}}{\pgfqpoint{-0.012425in}{-0.007216in}}{\pgfqpoint{-0.009821in}{-0.009821in}}%
\pgfpathcurveto{\pgfqpoint{-0.007216in}{-0.012425in}}{\pgfqpoint{-0.003683in}{-0.013889in}}{\pgfqpoint{0.000000in}{-0.013889in}}%
\pgfpathclose%
\pgfusepath{stroke,fill}%
}%
\begin{pgfscope}%
\pgfsys@transformshift{5.757861in}{0.708665in}%
\pgfsys@useobject{currentmarker}{}%
\end{pgfscope}%
\begin{pgfscope}%
\pgfsys@transformshift{5.762260in}{0.708760in}%
\pgfsys@useobject{currentmarker}{}%
\end{pgfscope}%
\begin{pgfscope}%
\pgfsys@transformshift{5.766750in}{0.708865in}%
\pgfsys@useobject{currentmarker}{}%
\end{pgfscope}%
\begin{pgfscope}%
\pgfsys@transformshift{5.771335in}{0.708968in}%
\pgfsys@useobject{currentmarker}{}%
\end{pgfscope}%
\begin{pgfscope}%
\pgfsys@transformshift{5.776018in}{0.709082in}%
\pgfsys@useobject{currentmarker}{}%
\end{pgfscope}%
\begin{pgfscope}%
\pgfsys@transformshift{5.780804in}{0.709195in}%
\pgfsys@useobject{currentmarker}{}%
\end{pgfscope}%
\begin{pgfscope}%
\pgfsys@transformshift{5.785698in}{0.709320in}%
\pgfsys@useobject{currentmarker}{}%
\end{pgfscope}%
\begin{pgfscope}%
\pgfsys@transformshift{5.790704in}{0.709442in}%
\pgfsys@useobject{currentmarker}{}%
\end{pgfscope}%
\begin{pgfscope}%
\pgfsys@transformshift{5.795828in}{0.709580in}%
\pgfsys@useobject{currentmarker}{}%
\end{pgfscope}%
\begin{pgfscope}%
\pgfsys@transformshift{5.801075in}{0.709715in}%
\pgfsys@useobject{currentmarker}{}%
\end{pgfscope}%
\begin{pgfscope}%
\pgfsys@transformshift{5.806452in}{0.709866in}%
\pgfsys@useobject{currentmarker}{}%
\end{pgfscope}%
\begin{pgfscope}%
\pgfsys@transformshift{5.835530in}{0.710719in}%
\pgfsys@useobject{currentmarker}{}%
\end{pgfscope}%
\begin{pgfscope}%
\pgfsys@transformshift{5.869098in}{0.711879in}%
\pgfsys@useobject{currentmarker}{}%
\end{pgfscope}%
\begin{pgfscope}%
\pgfsys@transformshift{5.908800in}{0.713477in}%
\pgfsys@useobject{currentmarker}{}%
\end{pgfscope}%
\begin{pgfscope}%
\pgfsys@transformshift{5.957391in}{0.715949in}%
\pgfsys@useobject{currentmarker}{}%
\end{pgfscope}%
\begin{pgfscope}%
\pgfsys@transformshift{6.020037in}{0.719983in}%
\pgfsys@useobject{currentmarker}{}%
\end{pgfscope}%
\begin{pgfscope}%
\pgfsys@transformshift{6.108331in}{0.728642in}%
\pgfsys@useobject{currentmarker}{}%
\end{pgfscope}%
\end{pgfscope}%
\begin{pgfscope}%
\pgfsetbuttcap%
\pgfsetmiterjoin%
\definecolor{currentfill}{rgb}{1.000000,1.000000,1.000000}%
\pgfsetfillcolor{currentfill}%
\pgfsetlinewidth{0.803000pt}%
\definecolor{currentstroke}{rgb}{1.000000,1.000000,1.000000}%
\pgfsetstrokecolor{currentstroke}%
\pgfsetdash{}{0pt}%
\pgfpathmoveto{\pgfqpoint{6.297392in}{0.973681in}}%
\pgfpathlineto{\pgfqpoint{6.297392in}{0.505132in}}%
\pgfpathlineto{\pgfqpoint{6.478411in}{0.505132in}}%
\pgfpathlineto{\pgfqpoint{6.478411in}{0.973681in}}%
\pgfpathclose%
\pgfusepath{stroke,fill}%
\end{pgfscope}%
\begin{pgfscope}%
\definecolor{textcolor}{rgb}{0.150000,0.150000,0.150000}%
\pgfsetstrokecolor{textcolor}%
\pgfsetfillcolor{textcolor}%
\pgftext[x=6.368294in,y=0.917995in,left,base,rotate=270.000000]{\color{textcolor}\sffamily\fontsize{5.647059}{6.776471}\selectfont nlevel = 2}%
\end{pgfscope}%
\begin{pgfscope}%
\pgfsetbuttcap%
\pgfsetmiterjoin%
\definecolor{currentfill}{rgb}{1.000000,1.000000,1.000000}%
\pgfsetfillcolor{currentfill}%
\pgfsetlinewidth{0.803000pt}%
\definecolor{currentstroke}{rgb}{1.000000,1.000000,1.000000}%
\pgfsetstrokecolor{currentstroke}%
\pgfsetdash{}{0pt}%
\pgfpathmoveto{\pgfqpoint{6.297392in}{0.973681in}}%
\pgfpathlineto{\pgfqpoint{6.297392in}{0.505132in}}%
\pgfpathlineto{\pgfqpoint{6.478411in}{0.505132in}}%
\pgfpathlineto{\pgfqpoint{6.478411in}{0.973681in}}%
\pgfpathclose%
\pgfusepath{stroke,fill}%
\end{pgfscope}%
\begin{pgfscope}%
\definecolor{textcolor}{rgb}{0.150000,0.150000,0.150000}%
\pgfsetstrokecolor{textcolor}%
\pgfsetfillcolor{textcolor}%
\pgftext[x=6.368294in,y=0.917995in,left,base,rotate=270.000000]{\color{textcolor}\sffamily\fontsize{5.647059}{6.776471}\selectfont nlevel = 2}%
\end{pgfscope}%
\begin{pgfscope}%
\pgfsetfillopacity{0.200000}%
\pgfsetstrokeopacity{0.200000}%
\definecolor{textcolor}{rgb}{0.501961,0.501961,0.501961}%
\pgfsetstrokecolor{textcolor}%
\pgfsetfillcolor{textcolor}%
\pgftext[x=1.943589in,y=2.519368in,left,base,rotate=30.000000]{\color{textcolor}\sffamily\fontsize{60.000000}{72.000000}\selectfont Preliminary}%
\end{pgfscope}%
\end{pgfpicture}%
\makeatother%
\endgroup%



\bibliography{master}

\end{document}
