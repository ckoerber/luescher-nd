\section{Introduction}\label{sec:intro}

Many physically interesting systems comprise strongly-interacting fermions.
In three spatial dimensions the scattering of fermions with a short-range interaction can be completely characterized by a scattering length, and when that length diverges the details of the potential are washed out and no dimensionful scales remain.
Such \emph{unitary fermions} exhibit interactions as strong as can be without forming bound states, and provide an interesting guide for understanding other strong interactions because of their universal behavior.
For example, the nuclear interaction in the deuteron channel has an extremely long scattering length, and trapped ultracold atoms can be tuned to unitarity by applying external magnetic fields and leveraging Feshbach resonances.

By tuning a quantum-mechanical two-body contact interaction, one should be able to completely control the scattering length and, absent other interactions, have that scattering length completely describe the scattering.
With such an interaction in hand, a variety of interesting many-body problems are unlocked.
Since all other dimensionful quantities are gone, all observables must be determined by naive dimensional analysis in the the density, times some non-perturbative numerical factor, such as the Bertsch parameter\cite{}\todo{CITE} in the case of the energy density.

In fact, a contact interaction can be shown to always produce momentum-independent scattering amplitudes (in three dimensions, for example, a flat $p \cot \delta$), and it ought to be possible to produce any amplitude, unless otherwise restricted by the Wigner bound\cite{Wigner:1955zz,Phillips:1996ae,Hammer:2010fw}.

Such scale-free results must result from peculiar potentials.
In three dimensions, for example, a delta function potential requires regulation, and to get scale-free dynamics its dimensionful strength must be sent to zero with the removal of the regulator in just such a way as to keep the phase shift at $\pi/2$.
In one dimension the strength is also dimensionful and a delta function potential needs no regulation, but nevertheless is regulated when space is discretized; the
In two dimensions the strength of the delta function potential is dimensionless, which entails a more complicated story we discuss in \Secref{2D}.

Numerical computations are often performed in discretized boxes with periodic boundary conditions.
\Luscher's finite-volume formalism\cite{Hamber198399,luscher:1986I,luscher:1986II,wiese1989,Luscher1991,Luscher1991237} is the method by which we can extract infinite-volume real-time scattering data from the finite-volume Euclidean spectrum of a theory, taking advantage of the interplay between the physical scattering and the finite-volume boundary conditions in determining the spectrum.

The usual understanding of \Luscher's formalism is that one should find the continuum zero-temperature finite-volume energy levels, holding the physical volume fixed, and put that cold, continuum spectrum through \Luscher's formula to extract continuum scattering data.

In practice, few results of lattice QCD calculations are zero-temperature- or, more seriously, continuum-extrapolated, but are nevertheless put through \Luscher's formula to get an estimate of the continuum scattering data, assuming thermal and discretization effects to be much smaller than the statistical uncertainties\todo{cite cite cite}.
In particular, no continuum-limit study of any baryonic channel exists, even at unphysically heavy pion masses.

While alternatives, including the potential method (see, for example \Refs{Ishii:2006ec,Nemura:2008sp,Aoki:2009ji,Murano:2011nz,Aoki:2012bb,Kurth:2013tua,Sugiura:2017vwo,Yamazaki:2019vid,Aoki:2017yru,Yamazaki:2018qut,Iritani:2017rlk,Iritani:2018zbt,Gongyo:2018gou,Akahoshi:2019klc,Namekawa:2019xiy}) and the imposition of spherical walls\cite{Borasoy:2007vy,Borasoy:2007vi,Lee:2008fa,Epelbaum:2008vj,Epelbaum:2010xt,Elhatisari:2015iga,Elhatisari:2016owd,Elhatisari:2016hby,Klein:2018lqz,Li:2019ldq,Bovermann:2019jbt}, can be used to translate finite-volume physics to infinite-volume observables we focus here on the \Luscher finite-volume formalism.
Moreover, to our knowledge, no numerical work leveraging these methods is in the continuum, either.

Here, we construct example Hamiltonians explicitly and diagonalize them exactly.
This allows us to circumvent all of the issues of statistical uncertainty that accompanies Monte Carlo data, and lets us completely isolate the features of the formalism itself, removing, for example, any finite-temperature effects that should in principle be extrapolated away in any finite-temperature method like Lattice QCD.

We find that it is in practice actually quite difficult to reliably extrapolate the spectrum to the continuum limit in a way that reproduces the exact known result, but that taking the continuum limit of the lattice-artifact-contaminated phase shifts seems to produce a more reliable result.

Our main innovation, however, is to explain how to incorporate lattice artefacts into \Luscher's formula, for systems described by a contact interaction, accounting both for the Brillouin zone of the lattice and the lattice-induced dispersion relation.
This lattice improvement can be quite useful.
In pursuit of a lattice formulation of unitary fermions, the authors of \Ref{Endres:2011er} followed the tuning procedure of \Ref{Lee:2007ae}, parametrizing the contact interaction as a sum of a tower of Galilean-invariant operators, tuning their coefficients so as to drive the lowest interacting energy levels to the zeros of the \Luscher finite-volume zeta function.
However, in \Ref{Endres:2012cw} they found that even with a highly-improved construction the states ultimately deviated from a $\pi/2$ phase shift (see, for example, Figure 3).
We introduce a new continuum-limit prescription for achieving unitarity in lattice simulations by tuning just the simplest contact operator, but taking the discretization effects into account by incorporating the lattice dispersion relation into the finite-volume zeta function, both in the tuning step and in the analysis step.
By re-tuning the interaction at each lattice spacing we can very easily and smoothly take the continuum limit after applying the lattice-aware finite-volume formula.
We demonstrate that this allows us to maintain a constant phase shift deep into the spectrum, covering as many \Aoneg states as exist in the lattice of interest.

This paper is organized as follow.  In \Secref{scattering} we give a brief summary of two particle scattering in $D$ dimensions.
In \Secref{hamiltonian} we give specifics about the latticized contact-interaction Hamiltonians we study numerically.
In \Secref{luescher} we provide a traditional continuum derivation of \Luscher's formula and in \Secref{dispersion} explain how to adapt it to include finite spacing effects by truncating the usual sum to just the momentum modes in the lattice and incorporating the dispersion relation into the appropriate propagators, yielding a lattice-improved generalized \Luscher zeta function.

Then, we leverage our dispersion zeta function, studying concrete examples.
In \Secref{3D} we study the three-dimensional case.
First we compare a continuum-extrapolated energy spectrum fed through the continuum zeta function and the continuum extrapolation of the finite-spacing spectra fed through the continuum zeta.
In \Secref{3D dispersion} we tune and analyze the same problem using our lattice-aware dispersion zeta function, and show that the resulting scattering $p\cot\delta$ remains constant deep into the spectrum; when we tune to unitarity the results stay at zero as accurately as the initial tuning is made.
We then study the one dimensional case in \Secref{1D}, where the absence of a counterterm makes things particularly simple.
In \Secref{2D} we repeat the story for the more intricate two-dimensional case, where dimensional transmutation and logarithmic singularities require special attention and care.
We find that our lattice-aware \Luscher function handles this case with no difficulty.  Further, in all dimensions considered here we provide correction terms that come about when using energies calculated in a discrete space but fed through continuum \Luscher formula, which when applied to three dimensions corrects for the deviation found in \Ref{Endres:2012cw}.  Our corrections are valid only for the case of a contact interaction.

Finally, in \Secref{conclusion} we summarize \todo{but I can't say more until we write the conclusion!}
