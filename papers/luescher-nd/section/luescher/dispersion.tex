\subsection{The Dispersion Method}\label{sec:dispersion}

The tuning of the contact interaction itself distinguishes this setup from, for example, lattice QCD.
In a usual lattice calculation, the continuum limit is sought by finding a line of constant physics where some parts of the single-hadron spectrum are held fixed as the continuum is approached.
Then, the hadron-hadron interactions are already determined by QCD itself, and the interaction one measures depends on the lattice spacing and approaches the correct interaction in the continuum.
In contrast, in our setup the masses are set by hand and their interactions may be tuned independently.
The procedure we advocate here is to tune the contact interaction so that the scattering length is held fixed as we take the lattice spacing to zero.

This tuning is accomplished by in a similar manner as the tuning procedure mentioned at the end of \Secref{continuum}, where the interaction at each spacing is adjusted so that the lowest energy state, when put through $S^{\spherical}$ matches the desired amplitude.
However, at each spacing we construct a lattice-aware function $S^{\dispersion}$ and tune accordingly.

To construct such a lattice-aware function we return to the derivation of \Luscher's finite-volume formalism and, recognizing that we're interested in incorporating lattice artifacts from the start, replace the continuum dispersion relation with the lattice dispersion relation in the propagators and require that the integrals are cut off in the same way---with a momentum cutoff that corresponds to that imposed by the lattice.
Starting from $I_0$ expressed in relative coordinates \eqref{I0 in relative coordinates} we replace $\vec{q}^2/2\mu$ with the lattice dispersion relation,
\begin{align}
    \label{eq:dispersion I0}
    I_0^{\dispersion}(E)
    &=\frac{1}{(2\pi)^D}
    \int_{-\Lambda}^{+\Lambda}
        \mathrm{d}^D \vec{q}
        \left[
            \PV \left(
                \frac{1}{
                    E - \frac{1}{2\mu} K_{qq} }
                \right)
            -i \pi \delta\left(E - \frac{1}{2\mu}K_{qq}\right)
        \right]
\end{align}
where $K_{qq}$ is a momentum-space matrix element of the Laplacian (which, of course, is diagonal in momentum space).
We adopt a $\dispersion$ superscript to indicate the quantity is defined accounting for the lattice data---the dispersion relation given by $K$ and the range of momenta in the Brillouin zone (on a square lattice, each momentum component cut off independently).

Defining the finite-spacing T-matrix to take the same form as \eqref{cot delta} with a lattice-spacing sensitivity and following the same procedure as before, we need to solve
\begin{equation}
    \frac{\mu}{2\F_D^\epsilon}(\cot \delta_0^\epsilon(E) - i) = I_0^{\dispersion \epsilon}(E) - I_{0,\FV}^{\dispersion \epsilon}(E),
\end{equation}
where we have suppressed each quantity's dependence on the form of the finite-difference Laplacian.
Demanding the seemingly-miraculous cancellation of the imaginary parts, one finds, identifying the cutoff $\Lambda$ with $\pi/\epsilon$,
\begin{align}
    \frac{1}{\F_D^\epsilon(2\mu E)} &= 4\pi \int_{-\pi/\epsilon}^{+\pi/\epsilon} \frac{\mathrm{d}^D q}{(2\pi)^D}\
    \delta\left(2\mu E - K_{qq}\right)
\end{align}
and we see that $\F_D^\epsilon(2\mu E)$ depends on the energy, and the discretization through $K$, so that it lattice-spacing aware.
In contrast, it does not depend on the size of the finite volume, which is encouraging because it is appears in the finite-spacing infinite-volume T-matrix.
It is easy to see that when $\epsilon\goesto0$ the dispersion relation goes to the exact $p^2$ relation and the limits of the integral go to infinity so that we may execute the integral spherically and recover the continuum $\F_D$ in \eqref{spherical FD}.
Unfortunately, achieving a closed-form expression for $\F_D^\epsilon$ is quite challenging but it is numerically tractable.

With $\F_D^\epsilon$ so determined, we solve the quantization condition for the finite-spacing phase shift,
\begin{align}
    \cot \delta_0^\epsilon(2\mu E)
    =
    \frac{\F_D^\epsilon(2\mu E)}{\pi^2 L^{D-2}}
    \left(\int \mathrm{d}^D n\; \PV - \sum_n \right)_{-N/2}^{+N/2}\  \frac{1}{\frac{2\mu E L^2}{4\pi^2} - \frac{L^2}{4\pi^2} K_{nn}}
\end{align}
where we rescaled $q\goesto 2\pi n/L$, the limits are understood for each spatial direction independently, and which can be compared to \eqref{spherical quantization}.
The $n$-dependent piece of the denominator goes to $n^2$ in the continuum limit, but even at finite spacing actually only depends on $N$ rather than $L$, which is clear from considering the particular Laplacians \eqref{laplacian} we study numerically and the elimination of the dimensionful scale effected by the rescaling from $q$ to $n$.
Note that the momentum-dependent term in the denominator goes to $n^2$ when $N$ is large.

We can construct, therefore, a \Luscher-like formalism,
\todo{It doesn't make sense to write the integral in a way that knows about N---it knows about epsilon only because it is at infinite volume!  Review the dispersion counterterm appendix and make sure the integral is as claimed.  This counterterm isn't a large-N thing, but it's exact.}
\begin{align}
    \label{eq:dispersion quantization}
    \cot \delta_0^{\epsilon}( 2\mu E ) &= \frac{\F_D^\epsilon(2\mu E)}{\pi^2 L^{D-2}} S^{\dispersion N}_{D}\left(\frac{2\mu E L^2}{(2\pi)^2}\right)
    \\
    \label{eq:dispersion S}
    S^{\dispersion N}_{D}( x )
    &=
    \left(\sum_n - \int \mathrm{d}^D n\; \PV \right)_{-N/2}^{+N/2}\  \frac{1}{ \frac{L^2}{4\pi^2}K_{nn} - x}.
\end{align}
where the dispersion zeta function $S^{\dispersion N}_{D}$ knows about the particular finite-differencing Laplacian or the dispersion relation as well as the discretization of the box into $N$ sites.
In contrast, in the usual finite-volume procedure, no UV details of the box infect the zeta function.
On the left-hand side of the quantization condition \eqref{dispersion quantization} we get the infinite-volume $A_1^+$ phase shift at scattering energy $E$ while on the right-hand size we need knowledge of the box size $L$, its discretization $N$ (which determines the lattice spacing $\epsilon=L/N$), as well as the finite-volume spectrum.
There is, strictly speaking, no divergence in the generalized zeta function, because we are always interested in a real calculation performed with finite $N$.
However, as the number of sites grows, we can approximate the integral by its terms leading in $N$, recovering the usual UV-agnostic formula where the integral cancels an actual divergence in the sum.
\todo{do we do a 1/N expansion anywhere?}

In the usual case, the on-shell condition is leveraged to trade $2\mu E$ for the scattering momentum $p$.  However, with a finite lattice spacing the on-shell condition is not so simple to invert.
In fact, there are multiple momenta that all correspond to the same energy, because the lattice dispersion relation begins decreasing once the momentum leaves the lattice's first Brillouin zone, and the energy repeats indefinitely so that there are infinitely many momenta that correspond to that energy.
Leaving the dependence on energy alone and not the momentum allows us to account naturally for Umklapp scattering processes and the violation of crystal momentum conservation in the infinite lattice volume.

The quantization condition \eqref{dispersion quantization} can be thought of as \Luscher's zero-center-of-mass-momentum finite-volume formula non-perturbatively improved for discretization effects.
To arrive at formulas for nonzero center of mass momentum is substantially more complicated, because only at zero center of mass momentum does the change from single-particle coordinates in \eqref{particle hamiltonian} to center-of-mass coordinates in \eqref{hamiltonian} commute with performing the spatial discretization, yielding the same dispersion relation in the effective one-body problem as in the two-body problem.
The ordering matters, as in a realistic many-body calculation (and in physical crystals!), each individual particle sees the lattice discretization.\footnote{
We note in passing that when $\nstep=\infty$ and the dispersion relation is exactly $p^2$ all the way to the edge of the Brillouin zone, the change to Jacobi coordinates seems to again commute with the discretization.}
To construct a lattice-improved finite-volume formula for two particles with finite center-of-mass momentum, one must backtrack even further, earlier than \eqref{dispersion I0}, to an equation more like \eqref{I0 in two particle language} before the energy integral is performed, replacing the single-particle dispersion relations there and changing the domain of integration to match the Brillouin zone.
We leave such a construction to future work.

In the following sections we leverage the dispersion quantization condition to tune a lattice finite-volume contact interaction, and show that we get a completely flat $p \cot \delta$ which remains zero as well as we can tune.


\todo{HERE IS THE PLACE TO TELL THE NOT-A-CONTACT INFECTED-ZETA STORY.}
