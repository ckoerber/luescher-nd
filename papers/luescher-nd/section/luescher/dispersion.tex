\subsection{The Dispersion Method}\label{sec:dispersion}

To correctly tune an interaction, we must insist that long-distance observables such as scattering data are correctly reproduced; the potential itself is not observable.
To take a sensible continuum limit, we must hold these observables fixed while sending the lattice spacing to zero.
Of course, observables will be infected by lattice artifacts, and so we must adjust the parameters of the potential as we take the limit.
Put another way, our renormalization scheme is to hold scattering data like the scattering length fixed as a function of lattice spacing.

This tuning distinguishes our setup from, for example, lattice QCD.
In a usual lattice calculation, the continuum limit is sought by finding a line of constant physics where some parts of the single-hadron spectrum are held fixed as the continuum is approached.
Then, at any finite spacing, the hadron-hadron interactions are already determined by the finite-spacing QCD itself, and the interaction one measures depends on the lattice spacing and approaches the correct interaction in the continuum.
In contrast, in our setup the masses are set by hand and their interactions may be tuned independently.
A continuum limit of QCD could, in principle, be taken along a line of constant deuteron-channel scattering length, but practical issues abound, even if simpler scattering channels like $I=2$ $\pi\pi$ scattering are picked instead.

Knowing that we must readjust the strength of our contact interaction as a function of lattice spacing raises the question of exactly what to tune to.
As mentioned at the end of \Secref{continuum}, on approach is to require the lowest energy state, when put through $S^{\spherical}$, match the desired scattering amplitude, an approach taken in, for example, \Refs{Endres:2011er,Lee:2007ae,Endres:2012cw}.
Tuned this way, one finds induced momentum dependence in the phase shift (see the $N_\mathcal{O}=1$ behavior of the left panel of Figure 2 of \Ref{Endres:2011er}, for example).

Here, we construct at each spacing a lattice-aware function $S^{\dispersion}$ and tune the lowest energy at that spacing to the desired amplitude.
With that tuning accomplished, other finite-volume same-spacing energy levels may be extracted and analyzed using the spacing-appropriate $S^{\dispersion}$.
We will show that tuning and analysis with $S^{\dispersion}$ yields momentum-independent scattering for the simple lattice contact interaction described in \Secref{hamiltonian}.

To construct such a lattice-aware function we return to the derivation of \Luscher's finite-volume formalism and, recognizing that we're interested in incorporating lattice artifacts from the start, replace the continuum dispersion relation with the lattice dispersion relation in the propagators and require that the integrals are cut off in the same way---with a momentum cutoff that corresponds to that imposed by the lattice.
The derivation provided in the previous section matched two $T$-matrices together---the finite- and infinite-volume matrices both in the continuum.
However, the spirit of our renormalization procedure is to ensure that the finite-spacing finite-volume $T$-matrix matches the continuum infinite-volume $T$-matrix as best it can.

To match the $T$-matrices together we need to solve
\begin{equation}
    \label{eq:finite spacing matching}
    \frac{\mu}{2 \F_D(\sqrt{2\mu E})}\left(\cot\delta^\epsilon_D(\sqrt{2\mu E})-i\right)
    =
    I_D^{\cartesian}(\sqrt{2\mu E}) - I_{D,FV}^{\dispersion,\epsilon}(\sqrt{2\mu E})
\end{equation}
where $\F_D$ is the usual continuum kinematic factor \eqref{spherical FD}, $I_D^{\cartesian}$ is the continuum-limit of the usual term with $\vec{q}^2/2\mu$ replaced by the lattice dispersion relation,
\begin{align}
    I_D^{\cartesian}(\sqrt{2\mu E}) &= \lim_{\epsilon\goesto0}I_D^{\dispersion}(\sqrt{2\mu E})
    &
    \label{eq:dispersion I0}
    I_D^{\dispersion}(\sqrt{2 \mu E})
    &=\frac{1}{(2\pi)^D}
    \int_{-\pi/\epsilon}^{+\pi/\epsilon}
        \mathrm{d}^D \vec{q}
        \left[
            \PV \left(
                \frac{1}{
                    E - \frac{1}{2\mu} K_{qq} }
                \right)
            -i \pi \delta\left(E - \frac{1}{2\mu}K_{qq}\right)
        \right]
\end{align}
where $K_{qq}$ is a momentum-space matrix element of the Laplacian (which, of course, is diagonal in momentum space), and the integral's cutoff $\Lambda$ in \eqref{I0 in relative coordinates} is taken to be $\pi/\epsilon$, matching the lattice's Brillouin zone.
We adopt a $\dispersion$ superscript to indicate the quantity is defined accounting for the lattice data---the dispersion relation given by $K$ and the range of momenta in the Brillouin zone (on a square lattice, each momentum component cut off independently).
That $\F_D$ appears in \eqref{finite spacing matching} reflect the fact that the infinite volume $I_D^{\cartesian}$ is in the continuum limit, so that the imaginary part of \eqref{dispersion I0} matches the continuum result from \eqref{I0 in relative coordinates}.

One may calculate a spacing-aware $\F_D^\epsilon$ by simply considering the imaginary part of the infinite-volume integral \eqref{dispersion I0}.
It is easy to see that when $\epsilon\goesto0$ the dispersion relation goes to the exact $p^2$ relation and the limits of the integral go to infinity so that we may execute the integral spherically and recover the continuum $\F_D$ in \eqref{spherical FD}.
Unfortunately, achieving a closed-form expression for $\F_D^\epsilon$ is quite challenging though it is numerically tractable.
Anyway, since we aim to match a lattice $T$ matrix to a continuum $T$ matrix, the $\F$ that appears is from the continuum.

We can solve the quantization condition for the finite-spacing phase shift,
\begin{align}
    \cot \delta_D^\epsilon(\sqrt{2 \mu E})
    =
    \frac{\F_D(\sqrt{2 \mu E})}{\pi^2 L^{D-2}}
    \left(\sum_{n\in\BZ} - \lim_{N\goesto\infty}\int_{-N/2}^{+N/2} \mathrm{d}^D n\; \PV \right)\  \frac{1}{\frac{L^2}{4\pi^2} K_{nn}-\frac{2\mu E L^2}{4\pi^2}}
\end{align}
where we rescaled $q\goesto 2\pi n/L$, the limits are understood for each spatial direction independently, and which can be compared to the usual quantization condition \eqref{spherical quantization}, and the Brillouin zone (\BZ) runs over all the finite-volume lattice modes.
The $n$-dependent piece of the denominator goes to $n^2$ in the continuum limit (fixed $L$, $N\goesto\infty$), but even at finite spacing actually only depends on $N$ rather than $L$, which is clear from considering the particular Laplacians we study numerically \eqref{laplacian} and the elimination of the dimensionful scale effected by the rescaling from $q$ to $n$.
Note that the momentum-dependent term in the denominator goes to $n^2$ when $N$ is large.

We can construct, therefore, a \Luscher-like formalism,
\todo{It doesn't make sense to write the integral in a way that knows about N---it knows about epsilon only because it is at infinite volume!  Review the dispersion counterterm appendix and make sure the integral is as claimed.  This counterterm isn't a large-N thing, but it's exact.} \todo{actually, this is fine---it shouldn't know about L either!  The dependences exactly cancel.}
\begin{align}
    \label{eq:dispersion quantization}
    \cot \delta_D^{\epsilon}( \sqrt{2 \mu E} ) &= \frac{\F_D(\sqrt{2 \mu E})}{\pi^2 L^{D-2}} S^{\dispersion N}_{D}\left(\frac{2\mu E L^2}{(2\pi)^2}\right)
    \\
    \label{eq:dispersion S}
    S^{\dispersion N}_{D}( x )
    &=
    \left(\sum_{n\in\BZ} - \lim_{N\goesto\infty}\int \mathrm{d}^D n\; \PV \right)_{-N/2}^{+N/2}\  \frac{1}{ \frac{L^2}{4\pi^2}K_{nn} - x} = \sum_{n\in\BZ}\frac{1}{ \frac{L^2}{4\pi^2}K_{nn} - x} - \counterterm_D^{\cartesian}N^{D-2}
\end{align}
where the dispersion zeta function $S^{\dispersion N}_{D}$ knows about the particular finite-differencing Laplacian or the dispersion relation as well as the discretization of the box into $N$ sites.
In contrast, in the usual finite-volume procedure, no UV details of the box infect the zeta function.
On the left-hand side of the quantization condition \eqref{dispersion quantization} we get the infinite-volume $A_1^+$ phase shift at scattering energy $E$ while on the right-hand size we need knowledge of the box size $L$, its discretization $N$ (which determines the lattice spacing $\epsilon=L/N$), as well as the finite-volume spectrum.
There is, strictly speaking, no divergence in the generalized zeta function, because we are always interested in a real calculation performed with finite $N$.
However, as the number of sites grows, we can approximate the integral by its terms leading in $N$, recovering the usual UV-agnostic formula where the integral cancels an actual divergence in the sum.

In the usual case, the on-shell condition is leveraged to trade $2\mu E$ for the scattering momentum $p$.  However, with a finite lattice spacing the on-shell condition is not so simple to invert.
In fact, there are multiple momenta that all correspond to the same energy, because the lattice dispersion relation begins decreasing once the momentum leaves the lattice's first Brillouin zone, and the energy repeats indefinitely so that there are infinitely many momenta that correspond to that energy.

Leaving the dependence on energy alone and not the momentum allows us to account naturally for Umklapp scattering processes and the violation of crystal momentum conservation.
This would be important for capturing physics of physical crystals, were we to match instead to an infinite volume lattice.
Matching to a real physical crystal requires formulating a spacing-dependent kinematic factor $\F_D^\epsilon$ from \eqref{dispersion I0} and keeping $N$ finite in the integral in the dispersion zeta function \eqref{dispersion S}, which introduces a whole tower of terms, each down by $N^2$, that vanish because we are matching to the continuum.

The quantization condition \eqref{dispersion quantization} can be thought of as \Luscher's zero-center-of-mass-momentum finite-volume formula non-perturbatively improved for discretization effects.
To arrive at formulas for nonzero center of mass momentum is substantially more complicated, because only at zero center of mass momentum does the change from single-particle coordinates in \eqref{particle hamiltonian} to center-of-mass coordinates in \eqref{hamiltonian} commute with performing the spatial discretization, yielding the same dispersion relation in the effective one-body problem as in the two-body problem.
The ordering matters, as in a realistic many-body calculation (and in physical crystals!), each individual particle sees the lattice discretization.\footnote{
We note in passing that when $\nstep=\infty$ and the dispersion relation is exactly $p^2$ all the way to the edge of the Brillouin zone, the change to Jacobi coordinates seems to again commute with the discretization.}
To construct a lattice-improved finite-volume formula for two particles with finite center-of-mass momentum, one must backtrack even further, earlier than \eqref{dispersion I0}, to an equation more like \eqref{I0 in two particle language} before the energy integral is performed, replacing the single-particle dispersion relations there and changing the domain of integration to match the Brillouin zone.
We leave such a construction to future work.

In the following sections we leverage the dispersion quantization condition to tune a lattice finite-volume contact interaction, and show that, in three dimensions, we get a completely flat $p \cot \delta$ when we feed finite-volume finite-spacing energy levels through the finite-spacing quantization condition \eqref{dispersion quantization}.
Analogous results will be shown for one and two spatial dimensions as well, and we will also extract corrections that arise from putting finite-spacing contact-interaction energy levels through the continuum quantization condition \eqref{general luscher}.

While our derivation focuses on matching $T$-matrices together, one could equally well formulate this quantization condition from requiring the wavefunctions on the finite volume's boundary matches those from a continuum volumes' boundary, so that although our derivation was restricted to the contact interaction, the quantization condition \eqref{dispersion quantization} should be universal, as long as at every finite spacing the potential is retuned to reproduce scattering data as well as possible.
