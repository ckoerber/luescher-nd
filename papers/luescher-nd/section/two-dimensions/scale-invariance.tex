\subsection{The scale invariant limit in 2 dimensions}
In Ref.~\cite{Nishida:2006eu} Nishida and Son analyzed the 2-D unitary limit (or near it) by investigating the system in $2+\epsilon$ dimensions, using $\epsilon$ as a perturbative parameter (not to be confused with the lattice spacing).  They found the trivial solution in the scale-invariant limit; namely, the two-body system becomes non-interacting as the scattering length $\tilde a_{20} \goesto\infty$.  Our results also show this behavior, since the ``flat line" goes upward as $\tilde a_{20}$ becomes larger (for fixed $L$).  And since the 2-D zeta function approaches the non-interacting energies when one goes well above the $y=0$ axis (and also as one goes well below the $y=0$ axis), the intersection of the flat line with the zeta function coincides with the two-body non-interacting energies, indicating the trivial solution. \todo{Need to reference the figure on $S_2$ here--T.L.}

In exactly 2-D, however, the logarithmic dependence of the scattering length requires an accompanying scale so as to make the argument of the logarithm dimensionless.  This is a consequence of the fact that in 2 dimensions, the interaction parameters of the Schr\"odinger equation are dimensionless.  \emph{Dimensionful} observables occur via dimensional transmutation \cite{} which requires some fiducial external scale, which in any finite-volume calculation is naturally given by the size of the volume.  Thus the scale-invariant limit becomes sensitive to the order in which one takes either $\tilde a_{20}\goesto\infty$ and $L\goesto\infty$ limits.  Taking either limit first, and indpendently of each other, will give the trivial scale-invariant limit found in \cite{Nishida:2006eu}. %An obvious question then arises about the unitary limit and how one approaches a scale-invariant limit in 2 dimensions.  %, and whether or not such a limit exist.  We argue that such a limit does exist.
However, there is a particular limit in which the scale-invariant system gives non-trivial solutions.
To see this, note that when $\tilde a_{20}=\frac{L}{2\pi}$ the LHS of eq.~\eqref{2d luscher} vanishes and we have an analogous quantization condition as in 3-D (when $a_{30}\to\infty$) and 1-D (when $a_{10}\to 0$).  In 3-D and 1-D, such a situation is akin to the unitary limit, once the infinite volume limit has been taken.  This limit can be done independently of taking $a_{30}\to\infty$ (3-D) or $a_{10}\to0$ (1-D).  However, in 2-D, because of the logarithmic dependence, demanding that the ``flat line" remains at $y=0$ axis requires that the infinite volume limit $L\to\infty$ cannot be taken independently of $\tilde a_{20}$.  In particular, for the \emph{non-trivial} scale-invariant limit, one must take both $\tilde a_{20}\goesto \infty$ and $L\to\infty$ limits simultaneously \emph{such that} $L/\tilde a_{20}=2\pi$.  Such a procedure maintains the ``flatness" of the L\"uscher equation at a the $y=0$ horizontal line at each step of the $\tilde a_{20}\to\infty$ and $L\to\infty$ extrapolation, and results in a non-trivial, scale-invariant, two-body system in 2-D.  \todo{You guys can probably say this better than me, I'm sure.  T.L.}
