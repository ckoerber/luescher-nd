\subsection{The scale invariant limit in 2 dimensions}
The logarithmic dependence of the scattering length and its accompanying scale is a consequence of the fact that in 2 dimensions, the interaction parameters of the Schr\"odinger equation are dimensionless.  \emph{Dimensionful} observables occur via dimensional transmutation \cite{} which requires some fiducial external scale, which in any finite-volume calculation is naturally given by the size of the volume.  An obvious question then arises about the unitary limit and how one approaches a scale-invariant limit in 2 dimensions, and whether or not such a limit exist.  We argue that such a limit does exist.

To see this, note that when $\tilde a_{20}=\frac{L}{2\pi}$ the LHS of eq.~\eqref{2d luscher} vanishes and we have an analogous quantization condition as in 3-D (when $a_{30}\to\infty$) and 1-D (when $a_{10}\to 0$).  In 3-D and 1-D, such a situation is akin to the unitary limit, once the infinite volume limit has been taken.  This limit can be done independently of taking $a_{30}\to\infty$ (3-D) or $a_{10}\to0$ (1-D).  However, in 2-D, because of the logarithmic dependence, the infinite volume limit $L\to\infty$ cannot be taken independently of $\tilde a_{20}$.  In particular, for the scale-invariant limit, one must take both $\tilde a_{20}\goesto \infty$ and $L\to\infty$ limits simultaneously \emph{such that} $L/\tilde a_{20}=2\pi$.  Such a procedure maintains the ``flatness" of the L\"uscher equation on the horizontal line at each step of the $\tilde a_{20}\to\infty$ and $L\to\infty$ extrapolation, and results in a non-trivial, scale-invariant, two-body system in 2-D.  \todo{You guys can probably say this better than me, I'm sure.  T.L.}
