%!TEX root =  ../../master.tex
\subsection{Dispersion L\"uscher in 2 dimensions}

The discussion above is valid only in the continuum.
For a discretized lattice, an additional length scale is introduced that must be accounted for.
As is the case in both 3-D and 1-D, there exists a dispersion L\"uscher equation that is valid for the contact interaction and accounts for the discretization.
In \Appref{counterterm/dispersion} we derive this dispersion L\"uscher formula for 2D and only state the result here.

Identifying the lattice spacing $\epsilon=N/L$, we have
\begin{align}
    \frac{2}{\pi} \log \left(\frac{2\pi \tilde a_{2}}{L}\right)
    &=
    \frac{1}{\pi^2}
    \left(
        \sum_{n_x,n_y=-\frac{N}{2}}^{\frac{N}{2}-1}\frac{1}{\tilde{K}^N_{nn}-x^\dispersion}
        -2\pi \log \left(\mathcal{L}^\dispersion_2\frac{N}{2}\right)
    \right)
    \nonumber\\
    &\equiv\frac{1}{\pi^2}S^{\dispersion}_2\left(x^\dispersion\right)\ ,\label{eq:2d dispersion luscher}
\end{align}
where $x$ is defined as usual,
\begin{equation}
    \mathcal{L}^\dispersion_{2}
    =
    \exp \left(\log (2)-G \frac{2}{\pi}\right)
    =
    1.116306393581637659468497 \ldots
\end{equation}
and $G$ is Catalan's constant. %\todo{Hmmm. . . this counterterm has an $N$ dependence, right?}

Given a contact interaction with coefficient appropriately tuned to the continuum scattering length $\tilde a_{2}$ and accounts for the discretization $\epsilon=N/L$ (see \Appref{dispersion-counterterm}),
\begin{equation}
C(N/L)=-\frac{ \pi}{\mu \log \left(\tilde a_{2} \mathcal{L}^\dispersion_2 \frac{N}{L}\pi\right)}\ ,
\end{equation}
the finite-spacing finite-volume energy eigenvalues of Schr\"odinger's equation on a square lattice will exactly satisfy the quantization condition \eqref{2d dispersion luscher}.

%\todo{\@TOM: was this tuned using $S^{\spherical}$ or $S^{\dispersion}$? Am I getting your procedure right?}
To demonstrate the success of this formula we tuned lattices with $N=10$, 20, and 40 to $\tilde a_{2}/L = .1$, which allows for a bound state, using $S^{\dispersion}$.
In \Figref{luescher2d} the black points were analyzed through $S^\dispersion_2$, and lie on a flat line, indicating that our dispersion \Luscher formula has correctly accounted for discretization effects.
On the other hand, if we use the same energies but analyze them with the usual continuum \Luscher function $S^{\spherical}_2$, shown as colored points, we see induced momentum-dependence and the flat line behavior is lost.

\begin{figure}
    \center
    \input{figure/2d.pgf}
    \caption{
        (Top panel)
        Results in 2-D using finite-spacing eigenvalues $x=2\mu EL^2/4\pi^2$ of the Schr\"odinger equation with $N=4$, 10, 20, and 40, (triangles, squares, diamonds, and hexagons, respectively) tuned so that $\tilde a_{2}/L=.1$ (closed symbols) and $\tilde a_{2}/L=1/(2\pi)$ (open symbols).
        The colored points are obtained using $S^\spherical_2(x)$ for analysis; red, green, blue, and purple corresponding to $N=4$, 10, 20, and 40 respectively.
        The thin colored lines are the derived induced momentum-dependent terms for each $N$ as given in \Tabref{induced terms in 2 d}.
        The black points are obtained using the $N$-appropriate $S^{\dispersion}_2(x)$ and exhibit the correct flat-line behavior.
        The dashed gray line is $S_2^\spherical$, as in the bottom panel.
        (Bottom panel)
        Two two-dimensional zeta functions, the spherical function $S_2^{\spherical}$ (light gray) given in \eqref{2d luscher} and $S_2^{\dispersion}$ (red) given in \eqref{2d dispersion luscher} with $N=4$ and $\nstep=\infty$.
        The difference between the dispersion and spherical curves is responsible for moving the red triangles to the black triangles in the top panel.
        }
    \label{fig:luescher2d}
\end{figure}

As was done in the three- and one-dimensional cases, we can derive the functional form of the induced momentum-dependent terms.
The derivation is identical to those cases; for concision we show only the end result.
Expanded around small $x^\dispersion$ one finds
\begin{equation}
    \label{eq:2D corrections}
    S^\bigcirc_{2}\left(x^\dispersion\right)
    =
    2\pi\log\left(2\pi \frac{\tilde a_{2}}{L}\right)
    + \alpha_1(N)
    + \alpha_2(N) x^\dispersion
    + \alpha_3(N) (x^\dispersion)^2
    + \ldots
\end{equation}
The coefficients $\alpha_i(N)$ have an implicit dependence on $N$ since the sums are restricted outside of the Brillouin zone.  Further, in 2-D the sums involved in $\alpha_i$ converge sufficiently fast and thus require no acceleration techniques.
We provide the numerical values of $\alpha_i(N)$ in \autoref{tab:induced terms in 2 d} for the discretizations shown in \autoref{fig:luescher2d}.
These functions were also used to calculate the thin colored lines in \autoref{fig:luescher2d}, where we see the small-$x$ expansion lose accuracy quickly for $N=4$ (consider the bound state, for example) but hold deeper into the spectrum for larger $N$.

\begin{table}
    \caption{The coefficients $\alpha_i$ of the induced momentum-dependent terms in \eqref{2D corrections} due to a contact interaction using $S^\bigcirc_2(x^\dispersion)$ as a function of discretization $N$, assuming $\nstep=\infty$.
    }
    \label{tab:induced terms in 2 d}
    \begin{tabular}{c|ccc}
    $N$ &   $\alpha_1$      &   $\alpha_2$      & $\alpha_3$                \\
    \hline
    4   &   $0.20642$   &   $0.726184$ & $0.0937105$           \\
    10  &   $0.0340198$   &   $0.105117$ & $0.00186065$           \\
    20  &   $0.000862459$   &   $0.0258495$ & $0.000110789$           \\
    40  &   $0.00851213$   &   $0.00643292$ & $6.83616\times10^{-6}$    \\
\end{tabular}
\end{table}
