\section{The dispersion method in three and two dimensions}\label{sec:3D dispersion}
In this section we explicitly derive the the dispersion formalism in both three and two dimensions.
We show that it is possible to tune the contact strength parameter in a finite volume for a given discretization scheme such that one directly obtains continuum infinite volume results when using the dispersion formalism---without any further extrapolation.

\subsection{Three dimensions}

According to \eqref{T matrix}, \eqref{I0} and \eqref{spherical FD}, we find that the phase shifts are related to the contact interaction by
\begin{equation}\label{eq:blah blah}
	p \cot \delta_3(p)
	= \lim\limits_{\Lambda \to \infty}\frac{2 \pi}{\mu}\frac{1}{T(p, \Lambda)} + i p
	= \lim\limits_{\Lambda \to \infty}
		\frac{2 \pi}{\mu} \left[
			\frac{1}{C(\Lambda)} - I_3(p, \Lambda)
		\right]
	\, ,
\end{equation}
with
\begin{equation}
	I_3(p, \Lambda)
	=
	-\frac{\mu}{2 \pi}
	\left[
	\frac{2 \Lambda}{\pi} + \frac{2  p}{\pi} \log \left( \frac{\Lambda - p}{\Lambda + p}\right)
	\right]
\end{equation}
The contact interaction cannot depend on any dynamic momenta; it is only possible to absorb momentum independent regulator terms when renormalizing the contact interaction.
It is still possible to renormalize the interaction such that the phase shifts, in the limit of $\Lambda \to \infty$, are independent of the cutoff by choosing
\begin{equation}\label{eq:three-d-counterterm}
	\frac{2 \pi}{\mu} \frac{1}{C_R(\Lambda)} + \frac{2 \Lambda}{\pi}
	\equiv
	- \frac{1}{a_3}
	=
	p \cot \delta_3(p)
	\, .
\end{equation}

In particular, because the limit of $\Lambda \to \infty$ is well defined for this choice of the contact interaction parameter $C_R(\Lambda)$, one is able to evaluate both sides for a given momentum, e.g., for $p=0$
\begin{equation}
	- \frac{1}{a_3}
	=
	\lim\limits_{p \to 0}\lim\limits_{\Lambda \to \infty}
		\left[
			\frac{2 \pi}{\mu}\frac{1}{T(p, \Lambda)} \bigg|_{C=C_R} + i p
	\right]
	=
	\lim\limits_{\Lambda \to \infty}
	\frac{2 \pi}{\mu}
		\left[
		\frac{1}{C_R(\Lambda)} - I_3(0, \Lambda)
		\right]
	\, .
\end{equation}

We now want to find an equivalent expression to the finite-volume zeta functions in presence of a discretization scheme.
In particular, the discretization scheme depends on the implementation of the kinetic operator $K^{(n_s)}$ and thus depends on the $n_s$ parameter.
The lattice spacing can be identified with the hard momentum cutoff $\Lambda = \pi / \epsilon$.
That is, the expectation value of the dispersion scales as $\hat K^{(n_s)}(\epsilon) \ket{p} = p^2 [1 + \mathcal O(\epsilon p)^{2 n_s}]\ket{p}$.

If one replaces the continuum momentum dispersion $q^2$ in $I_3$ with the kinetic operator for a given lattice spacing and discretization, one defines a sequence in $n_s$ which converges against $I_3$ in the limit of $n_s \to \infty$
\begin{equation}
	\lim\limits_{n_s \to \infty} I^{(n_s)}_3(p, \Lambda) = I_3(p, \Lambda)
	\, , \qquad
	I^{(n_s)}_3\left(p, \Lambda=\frac{\pi}{\epsilon} \right)
	=
	    \int\limits_{-\pi/\epsilon}^{+\pi/\epsilon}
        \mathrm{d}^3 \vec{q}
        \left[
            \PV \left(
                \frac{1}{
                    E - \frac{1}{2\mu} K_{qq}^{(n_s)} }
                \right)
            -i \pi \delta\left(E - \frac{1}{2\mu}K_{qq}^{(n_s)}\right)
        \right]
        \, .
\end{equation}
We furthermore define a sequence for the contact strength parameter depending on the cutoff and the employed discretization scheme which equivalently converges against the continuum result.
This sequence is determined by matching against the dispersion integral for each value of the cutoff and for each discretization scheme
\begin{equation}\label{eq:dispersion-renormalization}
	\lim\limits_{n_s \to \infty} C^{(n_s)}_R(\Lambda) = C_R(\Lambda) \, ,
	\qquad
	- \frac{1}{a_3}
	\equiv
	\frac{2 \pi}{\mu}
		\left[
		\frac{1}{C_R^{(n_s)}(\Lambda)} - I_3^{(n_s)}(0, \Lambda)
		\right]
	\, .
\end{equation}
It is possible to make this choice since both terms do not depend on any external momentum $p$.
This is specific for the contact interaction.
One can view this choice as the renormalization equation for contact interaction in presence of lattice discretization which, by definition, trivially satisfies
\begin{equation}
	- \frac{1}{a_3}
	=
	\lim\limits_{\Lambda \to \infty} \lim\limits_{n_s \to \infty}
	\frac{2 \pi}{\mu}
		\left[
		\frac{1}{C_R^{(n_s)}(\Lambda)} - I_3^{(n_s)}(0, \Lambda)
		\right]
	\, .
\end{equation}
In fact, it satisfies this equation even without the limits.

Next we address how this renormalization choice relates to the dispersion zeta function.
For any lattice implementation of a contact interaction with strength $c^\dispersion$ in finite volume, the Schr\"odinger equation can be rewritten as
\begin{equation}\label{eq:schroe}
	\hat G(E) \hat V \ket{\psi} = E\ket{\psi}
	\quad \Rightarrow \quad
	0 = 1 - c^\dispersion I_{3, \FV}^{(n_s)}\left(\sqrt{2 \mu E^\dispersion}, \Lambda = \frac{\pi}{\epsilon}\right) \, ,
\end{equation}
where $E^\dispersion$ are the finite volume energy levels, which depend on the employed discretization scheme and on the contact interaction of strength $c^\dispersion$.
The finite volume sum $I_{3, \FV}^{(n_s)}(p, \pi/\epsilon)$ is obtained by replacing the integral $d^3 \vec q$ in $I_3^{(n_s)}(p, \Lambda)$ with a sum  over vectors $\vec q = 2 \pi \vec n / L$.
Because the above equation is true for any value of $c^\dispersion$ and it's corresponding spectrum, it is especially true for $c^\dispersion = C_R^{(n_s)}(\Lambda)$.
This means that
\begin{equation}
	- \frac{1}{a_3}
	=
	\frac{2 \pi}{\mu}
		\left[
		I_{3, \FV}^{(n_s)}\left(\sqrt{2 \mu E_i}, \frac{\pi}{\epsilon}\right)
		- I_3^{(n_s)}\left(0, \frac{\pi}{\epsilon}\right)
		\right]
	\, ,
\end{equation}
which defines the dispersion zeta function
\begin{align}\label{eq:dispersion-zeta-form}
	- \frac{1}{a_3}
	=
	\frac{1}{\pi L}
	S^{\dispersion}_3(x^\dispersion)
	&=\frac{1}{\pi L}\left(\sum\limits_{n \in \BZ}\frac{1}{K_{nn}^{(n_s)} - x^\dispersion} - \mathcal{L}_3^{\dispersion} \frac{N}{2}\right)
	\, ,
	\\ \label{eq:dispersion-zeta-contact}
	\mathcal{L}_3^\dispersion
	&=
	\frac{2 \pi^2 L}{\mu}
	I_3^{(n_s)}\left(0, \Lambda = \frac{\pi}{\epsilon}\right)
	\overset{n_s\to\infty}{\longrightarrow} 15.348
	\, .
\end{align}

See also section \ref{sec:dispersion-counterterm} for the computation of this coefficient.
Equation \eqref{dispersion-zeta-form} explains why results directly match the continuum infinite volume phase shifts when computed with this modified zeta function.
Note that this result does not hold for general finite-range interactions, i.e., if it not possible to make an equivalent choice as in \eqref{dispersion-renormalization}.
We stress that this derivation uses the analytic  expression for the $T$-matrix and simplifies drastically because the phase shifts for a renormalized contact interaction are momentum independent.
This momentum independence had the consequence that the counter term in \eqref{dispersion-zeta-contact} is momentum independent as well.

\subsection{Two dimensions}

In two dimensions the analog of~\eqref{blah blah} is
\begin{equation}
\cot \delta_2(p)=\lim_{\Lambda\to\infty}\frac{2}{\mu}\left(\frac{1}{C(\Lambda)}- I_2(p, \Lambda)\right)\ , %\frac{1}{2\pi}\mathcal{P}\int_0^\Lambda  \mathrm { d } q \ q \left( \frac { 1 } { E - \frac{\vec{q}^2}{m} } \right)\right)\ ,
\end{equation}
with
\begin{equation}
I_2(p, \Lambda)=-\frac{\mu}{\pi } \log \left(\frac{p}{\sqrt{\Lambda ^2-p^2}}\right)\ .
\end{equation}
Our renormalized coefficient is defined by using the phase shift condition for a contact interaction in 2-D~\eqref{2d contact phase shift} in the $\Lambda\to\infty$ limit,
\begin{equation}\label{eq:log stuff}
	\frac{2}{\mu}\frac{1}{C_R(\Lambda)} + \frac{2}{\pi } \log \left(\frac{p}{\Lambda}\right)
	=
	\frac { 2 } { \pi } \log \left( p \tilde a _ { 2 } \right)\ ,
\end{equation}
which ensures $C_R(\Lambda)$ is momentum independent.
With the kinetic operator for a given lattice spacing and discretization, we again define a sequence in $n_s$ which converges against $I_2$ in the limit of $n_s \to \infty$,
\begin{equation}
	\lim\limits_{n_s \to \infty} I^{(n_s)}_2(p, \Lambda) = I_2(p, \Lambda)
	\, , \qquad
	I^{(n_s)}_2\left(p, \Lambda=\frac{\pi}{\epsilon} \right)
	=
	    \int\limits_{-\pi/\epsilon}^{+\pi/\epsilon}
        \mathrm{d}^2 \vec{q}
        \left[
            \PV \left(
                \frac{1}{
                    E - \frac{1}{2\mu} K_{qq}^{(n_s)} }
                \right)
            -i \pi \delta\left(E - \frac{1}{2\mu}K_{qq}^{(n_s)}\right)
        \right]
        \, .
\end{equation}
%\todo{Tom, check validity of the equation above!!!}
As was done prior to~\eqref{dispersion-renormalization}, we also define a sequence for the discrete coefficient $C^{(n_s)}_R(\Lambda)$ that is determined by matching against the dispersion integral for each value of the cutoff and for each discretization scheme.  However, in this case, due to the presence of logarithms in~\eqref{log stuff}, we first subtract the expression $\frac{2}{\pi}\log(pL/2\pi)$ prior to setting $p=0$,
\begin{equation}
%	\lim\limits_{n_s \to \infty} C^{(n_s)}_R(\Lambda) = C_R(\Lambda) \, ,
%	\quad
	\lim\limits_{p\to 0}
	\left[
		\frac { 2 } { \pi } \log \left( p \tilde a _ { 2 } \right)-\frac{2}{\pi } \log \left(\frac{pL}{2\pi}\right)
	\right]
	=
	\frac { 2 } { \pi } \log \left(2\pi \frac{\tilde a _ { 2 }}{L} \right)
	\equiv
	\frac{2 \pi}{\mu}
		\left[
		\frac{1}{C_R^{(n_s)}(\Lambda)} - \left.\left(I_3^{(n_s)}(p, \Lambda)+\frac{\mu}{\pi^2} \log \left(\frac{pL}{2\pi}\right)\right)\right|_{p=0}
		\right]
	\, .
\end{equation}
%\todo{Tom!! The $2\pi$ in the argument of the log seems arbitrary!!! Can we explain this better???}
Our sequence $\lim_{n_s\to\infty}C^{(n_s)}_R(\Lambda) = C_R(\Lambda)$ is well defined but implicitly depends on an external length scale $L$ due to the presence of the logarithm.  To arrive at the dispersion equation in 2-D one repeats the steps from~\eqref{schroe} leading up to~\eqref{dispersion-zeta-form}, but now~\eqref{dispersion-zeta-form} becomes
\begin{equation}
    \frac{2}{\pi} \log \left(\frac{2\pi \tilde a_{2}}{L}\right)=\frac{1}{\pi^2}S^{\dispersion}_2\left(x^\dispersion\right)
    =
    \frac{1}{\pi^2}
    \left(
        \sum_{n\in\operatorname{B.Z.}}\frac{1}{\tilde{K}^N_{nn}-x^\dispersion}
        -2\pi \log \left(\mathcal{L}^\dispersion_2\frac{N}{2}\right)
    \right)\ .
\end{equation}
Here
\begin{equation}
    \mathcal{L}^\dispersion_{2}
    =
    \exp \left(\log (2)-G \frac{2}{\pi}\right)
    =
    1.116306393581637659468497 \ldots
\end{equation}
and $G$ is Catalan's constant.  We derive this counterterm in \ref{sec:dispersion-counterterm}.  The renormalized coefficient in this case is
\begin{equation}
C^{(n_s)}_R(\Lambda)=-\frac{ \pi}{\mu \log \left(\tilde a_{2} \counterterm^\dispersion_2\Lambda\right)}\ .
\end{equation}
\todo{Where is the $n_s$ dependence in this equation?  It must be in $\counterterm_2^\dispersion$, right??}
