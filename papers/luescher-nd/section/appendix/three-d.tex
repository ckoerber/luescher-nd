\section{The dispersion method in three dimensions}\label{sec:3D dispersion}
In this section we derive the validity of the dispersion formalism in three dimensions.
We show that it is possible to tune the contact strength parameter in a finite volume for a given discretization scheme such that one directly obtains continuum infinite volume results when using the dispersion formalism---without any further extrapolation.

According to \eqref{T matrix}, \eqref{I0} and \eqref{spherical FD}, we find that the phase shifts are related to the contact interaction by
\begin{equation}
	p \cot \delta_3(p)
	= \lim\limits_{\Lambda \to \infty}\frac{2 \pi}{\mu}\frac{1}{T(p, \Lambda)} + i p
	= \lim\limits_{\Lambda \to \infty}
		\frac{2 \pi}{\mu} \left[
			\frac{1}{C(\Lambda)} - I_3(p, \Lambda)
		\right]
	\, ,
\end{equation}
with
\begin{equation}
	I_3(p, \Lambda)
	=
	-\frac{\mu}{2 \pi}
	\left[
	\frac{2 \Lambda}{\pi} + \frac{2  p}{\pi} \log \left( \frac{\Lambda - p}{\Lambda + p}\right)
	\right]
\end{equation}
The contact interaction cannot depend on any dynamic momenta; it is only possible to absorb momentum independent regulator terms when renormalizing the contact interaction.
It is still possible to renormalize the interaction such that the phase shifts, in the limit of $\Lambda \to \infty$, are independent of the cutoff by choosing
\begin{equation}\label{eq:three-d-counterterm}
	\frac{2 \pi}{\mu} \frac{1}{C_R(\Lambda)} + \frac{2 \Lambda}{\pi}
	\equiv
	- \frac{1}{a_3}
	=
	p \cot \delta_3(p)
	\, .
\end{equation}

In particular, because the limit of $\Lambda \to \infty$ is well defined for this choice of the contact interaction parameter $C_R(\Lambda)$, one is able to evaluate both sides for a given momentum, e.g., for $p=0$
\begin{equation}
	- \frac{1}{a_3}
	=
	\lim\limits_{p \to 0}\lim\limits_{\Lambda \to \infty}
		\left[
			\frac{2 \pi}{\mu}\frac{1}{T(p, \Lambda)} \bigg|_{C=C_R} + i p
	\right]
	=
	\lim\limits_{\Lambda \to \infty}
	\frac{2 \pi}{\mu}
		\left[
		\frac{1}{C_R(\Lambda)} - I_3(0, \Lambda)
		\right]
	\, .
\end{equation}

We now want to find an equivalent expression to the finite-volume zeta functions in presence of a discretization scheme.
In particular, the discretization scheme depends on the implementation of the kinetic operator $K^{(n_s)}$ and thus depends on the $n_s$ parameter.
The lattice spacing can be identified with the hard momentum cutoff $\Lambda = \pi / \epsilon$.
That is, the expectation value of the dispersion scales as $\hat K^{(n_s)}(\epsilon) \ket{p} = p^2 [1 + \mathcal O(\epsilon p)^{2 n_s}]\ket{p}$.

If one replaces the continuum momentum dispersion $q^2$ in $I_3$ with the kinetic operator for a given lattice spacing and discretization, one defines a sequence in $n_s$ which converges against $I_3$ in the limit of $n_s \to \infty$
\begin{equation}
	\lim\limits_{n_s \to \infty} I^{(n_s)}_3(p, \Lambda) = I_3(p, \Lambda)
	\, , \qquad
	I^{(n_s)}_3\left(p, \Lambda=\frac{\pi}{\epsilon} \right)
	=
	    \int\limits_{-\pi/\epsilon}^{+\pi/\epsilon}
        \mathrm{d}^3 \vec{q}
        \left[
            \PV \left(
                \frac{1}{
                    E - \frac{1}{2\mu} K_{qq}^{(n_s)} }
                \right)
            -i \pi \delta\left(E - \frac{1}{2\mu}K_{qq}^{(n_s)}\right)
        \right]
        \, .
\end{equation}
We furthermore define a sequence for the contact strength parameter depending on the cutoff and the employed discretization scheme which equivalently converges against the continuum result.
This sequence is determined by matching against the dispersion integral for each value of the cutoff and for each discretization scheme
\begin{equation}\label{eq:dispersion-renormalization}
	\lim\limits_{n_s \to \infty} C^{(n_s)}_R(\Lambda) = C_R(\Lambda) \, ,
	\qquad
	- \frac{1}{a_3}
	\equiv
	\frac{2 \pi}{\mu}
		\left[
		\frac{1}{C_R^{(n_s)}(\Lambda)} - I_3^{(n_s)}(0, \Lambda)
		\right]
	\, .
\end{equation}
It is possible to make this choice since both terms do not depend on any external momentum $p$.
This is specific for the contact interaction.
One can view this choice as the renormalization equation for contact interaction in presence of lattice discretization which, by definition, trivially satisfies
\begin{equation}
	- \frac{1}{a_3}
	=
	\lim\limits_{\Lambda \to \infty} \lim\limits_{n_s \to \infty}
	\frac{2 \pi}{\mu}
		\left[
		\frac{1}{C_R^{(n_s)}(\Lambda)} - I_3^{(n_s)}(0, \Lambda)
		\right]
	\, .
\end{equation}
In fact, it satisfies this equation even without the limits.

Next we address how this renormalization choice relates to the dispersion zeta function.
For any lattice implementation of a contact interaction with strength $c^\dispersion$ in finite volume, the Schrödinger equation can be rewritten as
\begin{equation}
	\hat G(E) \hat V \ket{\psi} = E\ket{\psi}
	\quad \Rightarrow \quad
	0 = 1 - c^\dispersion I_{3, \FV}^{(n_s)}\left(\sqrt{2 \mu E^\dispersion}, \Lambda = \frac{\pi}{\epsilon}\right) \, ,
\end{equation}
where $E^\dispersion$ are the finite volume energy levels, which depend on the employed discretization scheme and on the contact interaction of strength $c^\dispersion$.
The finite volume sum $I_{3, \FV}^{(n_s)}(p, \pi/\epsilon)$ is obtained by replacing the integral $d^3 \vec q$ in $I_3^{(n_s)}(p, \Lambda)$ with a sum  over vectors $\vec q = 2 \pi \vec n / L$.
Because the above equation is true for any value of $c^\dispersion$ and it's corresponding spectrum, it is especially true for $c^\dispersion = C_R^{(n_s)}(\Lambda)$.
This means that
\begin{equation}
	- \frac{1}{a_3}
	=
	\frac{2 \pi}{\mu}
		\left[
		I_{3, \FV}^{(n_s)}\left(\sqrt{2 \mu E_i}, \frac{\pi}{\epsilon}\right)
		- I_3^{(n_s)}\left(0, \frac{\pi}{\epsilon}\right)
		\right]
	\, ,
\end{equation}
which defines the dispersion zeta function
\begin{align}\label{eq:dispersion-zeta-contact}
	S^{\dispersion}(x^\dispersion)
	=
	\sum\limits_{n \in \BZ}
	\frac{1}{K_{nn}^{(n_s)} - x^\dispersion} - &\mathcal{L}_3^{\dispersion} \frac{N}{2}
	\, ,
	\\
	&\mathcal{L}_3^\dispersion
	=
	\frac{2 \pi^2 L}{\mu}
	I_3^{(n_s)}\left(p=0, \Lambda = \frac{\pi}{\epsilon}\right)
	\, .
\end{align}
To evaluate the infinite-volume integral in \eqref{dispersion-zeta-contact},
\todo{I think there is a $4 \pi^2$ missing in the denominator as $\cos \to 4 \pi^2 n^2$}
\begin{equation}
    4\pi^2 \int_{-N/2}^{+N/2} \mathrm{d}^Dn\; \PV \frac{1}{N^2 \sum_{ds} \gamma^{(\nstep)}_s \cos \frac{2\pi n_d s}{N} - x}
    =
    4\pi^2 \left(\frac{N}{2}\right)^{D-2} \int_{-1}^{+1} \mathrm{d}^D\nu\; \PV \frac{1}{4\sum_{ds} \gamma^{(\nstep)}_s \cos \pi \nu s - \xtilde}
\end{equation}
where as in the Cartesian case we rescaled and $\xtilde = x/(N/2)^2$.
Note that when we replace the sum over dimensions and steps with the exact $p^2$ result $(\pi \nu)^2$ and let $\xtilde$ vanish, we can match the Cartesian integral \eqref{cartesian integral}.

We can use the same trick to isolate the leading behavior in $N/2$, introducing the dispersion relation in the exponent rather than $n^2$.
One finds
\begin{equation}
    \label{eq:dispersion counterterm}
    4 \pi^2 \left(\frac{N}{2}\right)^{D-2}\int_{0}^{\infty} 2\mu\; \mathrm{d}\mu\; e^{\xtilde\mu^2}\left(\int_{-1}^{+1} \mathrm{d}\nu\; e^{-4\mu^2 \sum_s \gamma_s^{(\nstep)} \cos \pi \nu s}\right)^D
\end{equation}
which can be numerically evaluated quickly for $\xtilde\leq0$ where the derivation holds, assuming one has the dispersion relation coefficients in hand.
The counterterm for the leading divergence $\counterterm_D^{\dispersion (\nstep)}$ is the $\xtilde=0$ value.
Note that when $\nstep=\infty$ the continuum Cartesian result is obtained.
In \Figref{nstep counterterm} we show this counterterm and how it differs from the Cartesian counterterm in \eqref{cartesian counterterm}.

\begin{table}[htb]
    \begin{tabular}[t]{S[table-format=1.0]S[table-format=-1.15,table-auto-round=false]}
\toprule
  {$n_{s}$} &  {$\mathcal L^{\dispersion}_3$} \\ \midrule
        1 &                   19.95484069754250 \\
        2 &                   17.29373815124490 \\
        3 &                   16.52937382320310 \\
        4 &                   16.18180866760400 \\
        5 &                   15.98674421926657 \\
        6 &                   15.86306583941247 \\
        7 &                   15.77814195390823 \\
        8 &                   15.71645955032004 \\
        9 &                   15.66975024312220 \\
       10 &                   15.63322294937334 \\
       11 &                   15.60391847090050 \\
       12 &                   15.57991454331338 \\
       13 &                   15.55991041141987 \\
       14 &                   15.54299563540736 \\
       15 &                   15.52851460484342 \\
       16 &                   15.51598363500783 \\
       17 &                   15.50503834281211 \\
\bottomrule
\end{tabular}
\hspace{2em}
\begin{tabular}[t]{S[table-format=1.0]S[table-format=-1.15,table-auto-round=false]}
\toprule
  {$n_{s}$} &  {$\mathcal L^{\dispersion}_3$} \\ \midrule
       18 &                   15.49539917228799 \\
       19 &                   15.48684818183953 \\
       20 &                   15.47921303345776 \\
       21 &                   15.47235571159063 \\
       22 &                   15.46616442211375 \\
       23 &                   15.46054767500336 \\
       24 &                   15.45542989514885 \\
       25 &                   15.45074812098955 \\
       26 &                   15.44644948965527 \\
       27 &                   15.44248929886918 \\
       28 &                   15.43882949733264 \\
       29 &                   15.43543749725523 \\
       30 &                   15.43228523176677 \\
       31 &                   15.42934840038550 \\
       32 &                   15.42660586027747 \\
       33 &                   15.42403913154090 \\
       34 &                   15.42163199240652 \\
\bottomrule
\end{tabular}
\hspace{2em}
\begin{tabular}[t]{S[table-format=1.0]S[table-format=-1.15,table-auto-round=false]}
\toprule
  {$n_{s}$} &  {$\mathcal L^{\dispersion}_3$} \\ \midrule
       35 &                   15.41937014589034 \\
       36 &                   15.41724094363697 \\
       37 &                   15.41523315584890 \\
       38 &                   15.41333677859132 \\
       39 &                   15.41154287159014 \\
       40 &                   15.40984342105034 \\
       41 &                   15.40823122311402 \\
       42 &                   15.40669978443130 \\
       43 &                   15.40524323698864 \\
       44 &                   15.40385626486986 \\
       45 &                   15.40253404104781 \\
       46 &                   15.40127217264322 \\
       47 &                   15.40006665335855 \\
       48 &                   15.39891382201564 \\
       49 &                   15.39781032630417 \\
       50 &                   15.39675309099405 \\
 $\infty$ &                   15.34824844606382 \\
\bottomrule
\end{tabular}

    \caption{
    	\label{tab:diserpersion-zeta-3d-counterterm-counterterm}
		Counter term coefficients for the three-dimensional dispersion zeta function defined in \eqref{eq:dispersion quantization}.
    }
\end{table}

\begin{figure}[htb]
    \input{figure/counterterm-nstep.pgf}
    \caption{
    	In the top panel we show the dispersion counterterm $\counterterm^{\dispersion}_{3}$ in \eqref{dispersion counterterm} as a function of $\nstep$, and the $\nstep=\infty$ result ($\counterterm^{\cartesian}_3$), as a dashed line.
	In the bottom panel we show a better view into how the counterterm converges to the Cartesian one.
    }
    \label{fig:nstep counterterm}
\end{figure}




See also table \tabref{diserpersion-zeta-3d-counterterm-counterterm} for more precise counter term coefficients for more $\nstep$ values.
This explains why we expect results to directly match the continuum infinite volume phase shifts when computed with this modified zeta function.
Note that This result does not hold for general finite-range interactions, i.e., if it not possible to make an equivalent choice as in \eqref{dispersion-renormalization}.
We stress that this derivation uses the analytic  expression for the $T$-matrix and simplifies drastically because the phase shifts for a renormalized contact interaction are momentum independent.
This momentum independence had the consequence that the counter term in \eqref{dispersion-zeta-contact} is momentum independent as well.




