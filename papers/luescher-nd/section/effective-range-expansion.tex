\section{Two-particle scattering}

\subsection{The effective range expansion}
Scattering information of a two-particle system are captured by the scattering-matrix which is related to the $T$-matrix by
\begin{equation}
    S(\vec p', \vec p, E)
    =
    \delta^{(D)}(\vec p' - \vec p)
    +
    2 \pi i  \delta^{(1)}(E_{\vec p'} - E_{\vec p})T(\vec p', \vec p, E)\, ,
\end{equation}
where $\vec p', \vec p$ are the outgoing and incoming relative scattering momenta of the two-body system and $E_{\vec p'}, E_{\vec p}$ the respective energies.
The off-shell $T$-matrix can be obtained for a two-particle interaction $\hat V$ by solving the Lippmann-Schwinger equation
\begin{align}
	T(\vec p', \vec p, E)
	&=
	V(\vec p', \vec p) + \lim\limits_{\epsilon \to 0}\int \frac{d \vec k^3}{(2\pi)^3} V(\vec p', \vec k) G(\vec k, E + i \epsilon) T(\vec k, \vec p, E) \, ,
	&
	G(\vec k, E+ i \epsilon) = \frac{1}{E + i \epsilon - \frac{k^2}{2\mu}}
	\, .
\end{align}
After projecting the on-shell $T$-matrix onto partial waves,
\begin{equation}
    T_{l' m' l m}(p', p, E)
    =
    \frac{1}{4\pi}
    \int \mathrm{d} \hat p' \int \mathrm{d} \hat p \,
    Y_{l'm'}^*(\hat p') Y_{lm}(\hat p)
    T(\vec p', \vec p, E)
    \, ,
\end{equation}
the $T$-matrix can be described by scattering phase shifts
\begin{equation}\label{eq:on-shell-T}
	T_l(\gamma) \equiv \frac{1}{2l+1} \sum_{-l \leq m \leq l} T_{l m l m}(\gamma, \gamma , E_\gamma)
    = \frac{2}{\mu}
    \mathcal F^{D}(\gamma) \frac{1}{\cot (\delta_l(\gamma)) - i} \, ,
\end{equation}
where $E_\gamma = \gamma^2 / (2 \mu)$ and
\begin{equation}\label{eq:spherical FD}
    \mathcal F^{D}(\gamma)
    =
    \begin{cases}
        \gamma/\pi     & (D=1)\\
        1       & (D=2)\\
        \pi/\gamma   & (D=3)
    \end{cases}
\end{equation}
is a dimension-dependent kinematic function of the on-shell momentum.

The expansion of the inverse of eq.~(\ref{eq:on-shell-T}) in scattering momenta $\gamma$ is called the effective range expansion (ERE) and, e.g., the S-wave ($l=0$) component in three-dimensions takes the form
\begin{equation}
    \gamma \cot (\delta_0(\gamma)) - i \gamma
    =
    - \frac{1}{a_0}  - i \gamma + r_e \gamma^2 + \mathcal{O}(\gamma^4)\,,
\end{equation}
where $a_0$ is the so called scattering length and $r_e$ the effective range of the two-particle interactions.

\subsection{Derivative expansion of interactions}

\todo{Here comes some general stuff regarding the ERE and a contact tower...}

\subsection{Tuning to unitarity}


\todo{Here comes some stuff regarding the ERE for a contact and a hard cutoff}

Unitarity is obtained if the scattering is scale invariant, e.g., the ERE is independent of dimensional quantites.
As an example, for an interaction with zero effective range $r_e = 0$ (and higher terms in the order of scattering momenta equal to zero as well), the scattering length should diverge or equivalently $1/a_0 \to 0$.

We concentrate our analysis on a (momentum independent) contact interaction
\begin{align}
    \braket{\vec p' }{ V(\Lambda) \;\middle\vert\; \vec p }
    & =
    C(\Lambda)
    \, ,
    &
    \braket{\vec r' }{ V(\Lambda) \;\middle\vert\;  \vec r }
    & =
    C(\Lambda) \delta^{(D)}(\vec r - \vec r') \delta^{(D)}(\vec r)
    \, ,
\end{align}


This simplfies the analysis because
\begin{itemize}
    \item only S-wave information are not vanishing for a contact interaction
    \item the corresponding ERE has no effective range (and higher terms).
\end{itemize}
Thus, to obtain unitarity, one only has to tune the interaction to the S-wave scattering length.

For this derivation we explicitly utilize a hard cutoff regularization mimicing the UV-regularization of the coordinate space lattice.

In this case, the solution to the LS is
\begin{align}\label{eq:quantization-contact-physical}
	T(p', p, E) &= \frac{c(\Lambda)}{1 - c(\Lambda) I_0(E, \Lambda)} \, , &
	I_0(E) = \lim\limits_{\epsilon \to 0} \int\limits_{|\vec k| < \Lambda} \frac{d \vec k^3}{(2\pi)^3} G(\vec k, E + i \epsilon)
	\, .
\end{align}
