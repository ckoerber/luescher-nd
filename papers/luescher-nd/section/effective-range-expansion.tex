\section{Two-particle scattering}

\subsection{The effective range expansion}
Scattering information of a two-particle system are captured by the scattering-matrix which is related to the $T$-matrix.
The off-shell $T$-matrix can be obtained for a two-particle interaction $\hat V$ by solving the Lippmann-Schwinger equation
\begin{align}
	T_D(\vec p', \vec p, E)
	&=
	V(\vec p', \vec p) + \lim\limits_{\epsilon \to 0}\int \frac{d \vec k^D}{(2\pi)^D} V(\vec p', \vec k) G(\vec k, E + i \epsilon) T(\vec k, \vec p, E) \, ,
	&
	G(\vec k, E+ i \epsilon) = \frac{1}{E + i \epsilon - \frac{k^2}{2\mu}}
	\, .
\end{align}
After projecting the on-shell $T$-matrix onto partial waves,
the $T$-matrix is related to scattering phase shifts by
\begin{align}\label{eq:on-shell-T}
	\frac{1}{T_{lD}(p)}
    \equiv
    \frac{1}{T_{lD}(p, p, E_p)}
    = \frac{\mu}{2}
    \frac{1}{\mathcal F_{l D}(p)} \left[\cot (\delta_{l D}(p)) - i\right] \, ,
\end{align}
where $E_p = p^2 / (2 \mu)$ and $\mathcal F_{l D}(p)$ is a dimension-dependent kinematic function of the on-shell momentum.

Unitarity is obtained if the scattering is scale invariant, e.g., the phase shifts are independent of dimensional quantites
\begin{align}
    \cot (\delta_{l D}(p)) &= 0
    \,, &
    \frac{\partial}{\partial p}\cot (\delta_{l D}(p)) &= 0
    \, &
    \, \forall p\,.
\end{align}

Low energy theories usually consider the expansion of eq.~(\ref{eq:on-shell-T}) in scattering momenta $p$, called the effective range expansion (ERE), which takes the form \cite{Hammer:2010fw}
\begin{align}
    \cot \left(\delta_{l D}(p)\right)
    &=
    \delta_D \frac{2}{\pi}  \ln \left(p R_{l D}\right)
    -
    \frac{1}{a_{l D}} p^{2 - 2 l - D} +\frac{1}{2} r_{l D} p^{4 - 2 l - D} + O\left(p^{6 - 2 l - D}\right)
    \, , &
    \delta_D &= \begin{cases}
        0 & D \;\text{odd} \\ 1 & D \;\text{even}
    \end{cases}
    \, .
\end{align}
The parameter $R_{l D}$ is an arbitrary length scale.
The following terms containing $a_{l D}$ (for $l=0$ and $d=3$ called the scattering length), $r_{l D}$, (or $l=0$ and $d=3$ called effective range) and terms coming with higher powers of the scattering momenta correspond to properties of the two-particle interaction.

\todo{Address cutoff}
For a contact interaction, all contributions to the ERE for partial waves different than S-wave ($l=0$) vanish
\begin{align}\label{eq:quantization-contact-physical}
	V(\vec p', \vec p) &= c(\Lambda)
	\, \Rightarrow &
	T_D(p) = T_{0D}(p) &= \frac{c(\Lambda)}{1 - c(\Lambda) I_D(E_p, \Lambda)}
	\, , &
	I_D(E_p, \Lambda) = \lim\limits_{\epsilon \to 0} \int\limits_{|\vec k| < \Lambda} \frac{d \vec k^D}{(2\pi)^D} G(\vec k, E_p + i \epsilon)
	\, .
\end{align}
In particular, for a contact interaction, one finds
\begin{align}
    \cot \left(\delta_{l D}(p)\right)
    =
    \delta_{l,0}
    \cot \left(\delta_{0 D}(p)\right)
    & =
    i +
    \frac{2}{\mu}
    \frac{1}{\mathcal{F}_{0D}}
    \left[
        \frac{1}{c(\Lambda)}
        -
        \lim\limits_{\epsilon\to0}{}_2F_1\left(1, \frac{D}{2}, \frac{D+3}{2}, \frac{\Lambda^2}{(p + i \epsilon)^2} \right)
    \right]
    \\
    &=
    \begin{cases}
        ... & (D=1)\\
        ... & (D=2)\\
        ... & (D=3)
    \end{cases}
\end{align}
\todo{needs to be finalized}
