%!TEX root =  ../master.tex
\section{One Dimension}\label{sec:1D}

Here we consider L\"uscher's formula in one dimension with a contact interaction.
Since the sum in the quantization condition \eqref{general luscher}or the one-loop finite-volume sum \eqref{I0 FV} with $D=1$ is convergent, the counterterm $\counterterm_{D=1}^\spherical=0$ and things are particularly simple,
\begin{align}
    C(\Lambda)
        &=
            -\frac{1}{\mu a_{10}}
    &
    \frac{a_{10}}{L}
        &=
            \frac{1}{2 \pi^{2}}
            \sum_{n=-\infty}^{\infty} \frac{1}{n^{2}-x}
        \equiv
            \frac{1}{2 \pi^{2}}
            S^\bigcirc_{1}\left(x\right)
    &
    \Bigg(x
        &=
            \frac{2\mu E L^2}{4\pi^2}\Bigg)
\end{align}
and $E$ must be a finite-volume energy on a torus of circumference $L$.
In one dimension the sum in the zeta function is well behaved and has a compact form,
\begin{equation}\label{eq:1d luscher}
S^\bigcirc_{1}(x) \equiv \sum_{n=-\infty}^{\infty} \frac{1}{n^{2}-x}=-\pi \frac{\cot (\pi \sqrt{x})}{\sqrt{x}}\ ,
\end{equation}
which gives a closed form expression for L\"uscher's formula,
\begin{equation}\label{eq:1d luscher}
\frac{a_{10}}{L} =-\frac{1}{pL}\cot\left(\frac{pL}{2}\right),
\end{equation}
consistent with those found in \cite{}.

Since there is no counterterm in one dimension, the dispersion form of L\"uscher's formula is straightforward to obtain.  If one identifies the lattice spacing as the cutoff, then the sum in the zeta function is restricted to the Brillouin zone and one has
\begin{equation}
    \frac{a_{10}}{L}
    =
    \frac{1}{2 \pi^{2}} \sum_{n=-\frac{N}{2}}^{\frac{N}{2}-1} \frac{1}{\frac{L^2}{4\pi^2}K_{nn}-x}
    =
    \frac{1}{2 \pi^{2}} \sum_{n=-\frac{N}{2}}^{\frac{N}{2}-1} \frac{1}{\frac{N^2}{4\pi^2}\left(\sum_{s}\gamma_s^{(\nstep)}\cos\frac{2\pi n s}{N}\right)-x}
    \equiv
    \frac{1}{2 \pi^{2}} S^{\dispersion}_{1}\left(x\right),\label{eq:1d dispersion luscher}
\end{equation}
where we explicitly show that the dispersion function $S^{\dispersion}$ depends on $\nstep$ and $N$ but not on $L$ or $\epsilon$ explicitly.


As stressed in the previous section, only continuum-extrapolated energies should be used in the quantization condition \eqref{1d luscher}, or induced momentum dependence terms will result.
For example, in \Figref{luescher1d} we show the induced momentum dependence terms when non-continuum eigenvalues are inserted into $S^{\spherical}_1(x)$ (colored points) for lattice sizes of $N=4$, 10, 12, and 14 and $\nstep=\infty$.
Against expectations for a contact interaction, the extracted amplitude is not flat.
However, we also show the scattering data determined through $S^{\dispersion}_1(x)$ (black points), and these all lie on a flat line, as expected.

\begin{figure}
\center
    \center
    \input{figure/1d.pgf}
%\includegraphics[width=.65\textwidth]{figure/1d.pdf}
    \caption{
        (Top panel)
        Results in 1-D using finite-spacing eigenvalues $x=2\mu EL^2/4\pi^2$ of the Schr\"odinger equation with $N=4$, 10, 12, and 14, (triangles, squares, diamonds, and hexagons, respectively) tuned so that $\tilde a_{20}/L=.1$ (closed symbols) and $\tilde a_{20}/L=0$ (open symbols).
        The colored points are obtained using $S^\spherical_1(x)$ for analysis; red, green, blue, and purple corresponding to $N=4$, 10, 12, and 14 respectively.
        The thin colored lines are the derived induced momentum-dependent terms for each $N$ as given in \Tabref{induced terms in 1 d}.
        The black points are obtained using the $N$-appropriate $S^{\dispersion}_1(x)$ and exhibit the correct flat-line behavior.
        The dashed gray line is $S_1^\spherical$, as in the bottom panel.
        (Bottom panel)
        Two one-dimensional zeta functions, the spherical function $S_1^{\spherical}$ (light gray) given in \eqref{1d luscher} and $S_1^{\dispersion}$ (red) given in \eqref{1d dispersion luscher} with $N=4$ and $\nstep=\infty$.
        The difference between the dispersion and spherical curves is responsible for moving the red triangles to the black triangles in the top panel.
        }
        \label{fig:luescher1d}
\end{figure}

We can derive the functional form of these induced momentum-dependent terms by extracting the contribution to the dispersion L\"uscher from the continuum L\"uscher.
\begin{align*}
 \frac{1}{2 \pi^{2}} S^\bigcirc_{1}\left(x\right)=\frac{1}{2 \pi^{2}} \sum_{n=-\infty}^{\infty} \frac{1}{n^{2}-x} &=\frac{1}{2 \pi^{2}} \sum_{n\in \operatorname{B.Z.}} \frac{1}{n^{2}-x} +\frac{1}{2 \pi^{2}} \sum_{n\notin \operatorname{B.Z.}} \frac{1}{n^{2}-x} \\
 &= \frac{1}{2 \pi^{2}} S^{\dispersion}_{1}\left(x\right)+\frac{1}{2 \pi^{2}} \sum_{n\notin \operatorname{B.Z.}} \frac{1}{n^{2}-x}\\
&=\frac{a_{10}}{L}+\frac{1}{2 \pi^{2}} \sum_{n\notin \operatorname{B.Z.}} \frac{1}{n^{2}-x} \ .
\end{align*}
where the energy in $x$ is a finite-spacing finite-volume $\nstep=\infty$ energy.
In the second term the sum is restricted to values of $n$ outside the Brillouin zone.
Assuming $n^2\gg x$, we can expand in powers or $x$,
\begin{equation}
    \frac{1}{2 \pi^{2}} \sum_{n=-\infty}^{\infty} \frac{1}{n^{2}-x}
        =
    \frac{a_{10}}{L}+\frac{1}{2 \pi^{2}} \sum_{n\notin \operatorname{B.Z.}} \frac{1}{n^{2}-x}
        =
    \frac{a_{10}}{L}+\frac{1}{2 \pi^{2}} \sum_{n\notin \operatorname{B.Z.}} \left(\frac{1}{n^2}+\frac{x}{n^4}+\frac{x^2}{n^6}+\mathcal{O}(x^3)\right)
\end{equation}
We make the replacement $\sum_{n\notin\operatorname{B.Z.}}\mapsto\left(\sum_{n=-\infty}^\infty-\sum_{n\in\operatorname{B.Z.}}\right)$, which allows each summation in the expansion to be determined term by term to arbitrary precision.

\begin{table}
    \caption{
    The induced momentum-dependent terms to order $x^2$ due to a contact interaction using $S^\bigcirc_1(x)$ as a function of discretization $N$.
    Here $x$ is determined by a finite-spacing finite-volume $\nstep=\infty$ eigenenergy.}
    \label{tab:induced terms in 1 d}
    \begin{tabular}{c|c}
    $N$
        &
            $\frac{1}{2 \pi^{2}} \sum_{n\notin \operatorname{B.Z.}} \left(\frac{1}{n^2}+\frac{x}{n^4}+\frac{x^2}{n^6}\right)$       \\
        \hline
      4 &    $0.052680335 + 0.00517480049 x + 9.6564782\times10^{-4} x^2$\\
    10  &   $0.020398280 + 0.00028079187 x + 7.1106444\times10^{-6} x^2$    \\
    12  &   $0.016964618 + 0.00016064496 x + 2.7825342\times10^{-6} x^2$    \\
    14  &   $0.014523490 + 0.00010045521 x + 1.2660940\times10^{-6} x^2$    \\
    \end{tabular}
\end{table}

\Tabref{induced terms in 1 d} shows these terms for $N=4$, 10, 12, and 14.
The thin colored lines in \Figref{luescher1d} correspond to the functions given in this table.  In the limit $N\goesto\infty$ all states are included in the Brillouin zone so all terms vanish and one recovers the flat, momentum-independent behavior.
Of course, analyzing the finite-spacing dispersion zeta function $S^{\dispersion}_1$ with $\nstep=\infty$ produces the flat behavior all the way through.
This demonstrates that it was not the contact operator that caused the momentum dependence, but that the dependence was induced by leveraging the continuum finite-volume formalism itself.

Finally, we draw the reader's attention to the structure of $S^\dispersion$ at any finite $N$ in the bottom panel of \Figref{luescher1d}, where the $N=4$ dispersion zeta is shown.
Note that any flat function of $x$ can only ever intercept the zeta function three times---which makes sense, as there are only three states in the \Aoneg sector of a one-dimensional $N=4$ lattice, $\ket{0}$, $(\ket{-1}+\ket{+1})/\sqrt{2}$ and $\ket{2}$.
If one tunes a contact interaction so that the scattering amplitude vanishes, one will see one state with $0<x<1$, one with $1<x<4$, and one state with $\abs{x}$ very large and a sign depending on whether one is slightly above or below zero numerically.
The finiteness of the \Aoneg sector puts constraints on the interactions that can be faithfully put onto such a small lattice: one cannot create any interaction at all whatsoever where the scattering amplitude intersects $S^{\dispersion}$ four times, because that would entail too many finite-volume states.
Of course, this is a generic feature in any number of dimensions, and the constraint ultimately vanishes in the continuum limit $N\goesto\infty$, though as $N$ increases the number of accessible $n^2$ shells grows in a dimension-dependent way.
