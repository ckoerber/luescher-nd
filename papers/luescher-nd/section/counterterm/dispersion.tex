\section{The Dispersion Counterterm}
\label{sec:dispersion-counterterm}

To evaluate the infinite-volume integral in \eqref{dispersion-counter-integral}, we rescale the $n$ integration to extract $N$ out of the integral and rescale and $x \to \xtilde = 4 \pi^2  x/(N/2)^2$
\begin{equation}\label{eq:rescaled-counterterm-integral}
    \int_{-N/2}^{+N/2} \mathrm{d}^Dn\; \PV \frac{1}{\tilde K_{nn}^N - x}
    =
    4\pi^2 \left(\frac{N}{2}\right)^{D-2} \int_{-1}^{+1} \mathrm{d}^D\nu\; \PV \frac{1}{4\sum_{ds} \gamma^{(\nstep)}_s \cos \pi \nu s - \xtilde}
    \, .
\end{equation}
Note that when we replace the sum over dimensions and steps with the exact $p^2$ result $(\pi \nu)^2$ and let $\xtilde$ vanish, we can match the Cartesian integral \eqref{cartesian integral}.

We can use the same trick to isolate the leading behavior in $N/2$, introducing the dispersion relation in the exponent rather than $n^2$.
One finds
\begin{equation}
    \label{eq:dispersion counterterm}
    \eqref{rescaled-counterterm-integral}
    =
    4 \pi^2 \left(\frac{N}{2}\right)^{D-2}\int_{0}^{\infty} 2\mu\; \mathrm{d}\mu\; e^{\xtilde\mu^2}\left(\int_{-1}^{+1} \mathrm{d}\nu\; e^{-4\mu^2 \sum_s \gamma_s^{(\nstep)} \cos \pi \nu s}\right)^D
\end{equation}
which can be numerically evaluated quickly for $\xtilde\leq0$ where the derivation holds, assuming one has the dispersion relation coefficients in hand.
The counterterm for the leading divergence $\counterterm_D^{\dispersion (\nstep)}$ is the $\xtilde=0$ value.
Note that when $\nstep=\infty$ the continuum Cartesian result is obtained.
In \Figref{nstep counterterm} we show this counterterm and how it differs from the Cartesian counterterm in \eqref{cartesian counterterm}.



\begin{figure}[htb]
    \input{figure/counterterm-nstep.pgf}
    \caption{
    	In the top panel we show the dispersion counterterm $\counterterm^{\dispersion}_{3}$ in \eqref{dispersion counterterm} as a function of $\nstep$, and the $\nstep=\infty$ result ($\counterterm^{\cartesian}_3$), as a dashed line.
	In the bottom panel we show a better view into how the counterterm converges to the Cartesian one.
    }
    \label{fig:nstep counterterm}
\end{figure}
