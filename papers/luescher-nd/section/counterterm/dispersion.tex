\section{The Dispersion Counterterm}
\label{sec:dispersion-counterterm}

To evaluate the infinite-volume integral in \eqref{dispersion-counter-integral}, we rescale the $n$ integration to extract $N$ out of the integral and rescale  $x \to \xtilde = 4 \pi^2  x/(N/2)^2$
\begin{align}\label{eq:rescaled-counterterm-integral}
    \int_{-N/2}^{+N/2} \mathrm{d}^Dn\; \PV \frac{1}{\tilde K_{nn}^N - x}
    =
    4\pi^2 \left(\frac{N}{2}\right)^{D-2} \int_{-1}^{+1} \mathrm{d}^D\nu\; \PV \frac{1}{4\sum_{ds} \gamma^{(\nstep)}_s \cos \pi \nu s - \xtilde}
    \, ,
\end{align}
which is well defined for any $d$ if $\xtilde\neq0$ and for $\xtilde=0$ if $d>2$.
For $\xtilde\leq0$ The sum over dimensions can be isolated by introducing another integral
\begin{equation}
    \label{eq:dispersion counterterm}
    \eqref{rescaled-counterterm-integral}
    =
    4 \pi^2 \left(\frac{N}{2}\right)^{D-2}\int_{0}^{\infty} 2y\; \mathrm{d}y\; e^{\xtilde y^2}\left(\int_{-1}^{+1} \mathrm{d}\nu\; e^{-4y^2 \sum_s \gamma_s^{(\nstep)} \cos \pi \nu s}\right)^D
\end{equation}
which can be numerically evaluated.
The counterterm for the leading divergence $\counterterm_D^{\dispersion}$ is the $\xtilde=0$ value.
For three dimensions, we show this counterterm and how it differs from the $\nstep \to \infty$ counterterm in \Figref{nstep counterterm} and provide precise values in table \tabref{diserpersion-zeta-3d-counterterm-counterterm}.
\begin{table}[htb]
    \begin{tabular}[t]{S[table-format=1.0]S[table-format=-1.15,table-auto-round=false]}
\toprule
  {$n_{s}$} &  {$\mathcal L^{\dispersion}_3$} \\ \midrule
        1 &                   19.95484069754250 \\
        2 &                   17.29373815124490 \\
        3 &                   16.52937382320310 \\
        4 &                   16.18180866760400 \\
        5 &                   15.98674421926657 \\
        6 &                   15.86306583941247 \\
        7 &                   15.77814195390823 \\
        8 &                   15.71645955032004 \\
        9 &                   15.66975024312220 \\
       10 &                   15.63322294937334 \\
       11 &                   15.60391847090050 \\
       12 &                   15.57991454331338 \\
       13 &                   15.55991041141987 \\
       14 &                   15.54299563540736 \\
       15 &                   15.52851460484342 \\
       16 &                   15.51598363500783 \\
       17 &                   15.50503834281211 \\
\bottomrule
\end{tabular}
\hspace{2em}
\begin{tabular}[t]{S[table-format=1.0]S[table-format=-1.15,table-auto-round=false]}
\toprule
  {$n_{s}$} &  {$\mathcal L^{\dispersion}_3$} \\ \midrule
       18 &                   15.49539917228799 \\
       19 &                   15.48684818183953 \\
       20 &                   15.47921303345776 \\
       21 &                   15.47235571159063 \\
       22 &                   15.46616442211375 \\
       23 &                   15.46054767500336 \\
       24 &                   15.45542989514885 \\
       25 &                   15.45074812098955 \\
       26 &                   15.44644948965527 \\
       27 &                   15.44248929886918 \\
       28 &                   15.43882949733264 \\
       29 &                   15.43543749725523 \\
       30 &                   15.43228523176677 \\
       31 &                   15.42934840038550 \\
       32 &                   15.42660586027747 \\
       33 &                   15.42403913154090 \\
       34 &                   15.42163199240652 \\
\bottomrule
\end{tabular}
\hspace{2em}
\begin{tabular}[t]{S[table-format=1.0]S[table-format=-1.15,table-auto-round=false]}
\toprule
  {$n_{s}$} &  {$\mathcal L^{\dispersion}_3$} \\ \midrule
       35 &                   15.41937014589034 \\
       36 &                   15.41724094363697 \\
       37 &                   15.41523315584890 \\
       38 &                   15.41333677859132 \\
       39 &                   15.41154287159014 \\
       40 &                   15.40984342105034 \\
       41 &                   15.40823122311402 \\
       42 &                   15.40669978443130 \\
       43 &                   15.40524323698864 \\
       44 &                   15.40385626486986 \\
       45 &                   15.40253404104781 \\
       46 &                   15.40127217264322 \\
       47 &                   15.40006665335855 \\
       48 &                   15.39891382201564 \\
       49 &                   15.39781032630417 \\
       50 &                   15.39675309099405 \\
 $\infty$ &                   15.34824844606382 \\
\bottomrule
\end{tabular}

    \caption{
    	\label{tab:diserpersion-zeta-3d-counterterm-counterterm}
		Counter term coefficients for the three-dimensional dispersion zeta function defined in \eqref{dispersion-zeta-contact}.
    }
\end{table}

\begin{figure}[htb]
    \scalebox{0.9}{\input{figure/counterterm-nstep.pgf}}
    \caption{
    	In the top panel we show the dispersion counterterm $\counterterm^{\dispersion}_{3}$ in \eqref{dispersion-zeta-contact} as a function of $\nstep$, and the $\nstep=\infty$ result, as a dashed line.
	In the bottom panel we show a better view into how the counterterm converges to the Cartesian one.
    }
    \label{fig:nstep counterterm}
\end{figure}

If we assume $n_{s}=\infty$ we can obtain analytic solutions when $\tilde x=0$.  For $D=3$ we find
\begin{multline}
\counterterm_3^{\dispersion}=-8 G-4 i \left\{2 \text{Li}_2\left(1-\sqrt[4]{-1}\right)-2 \text{Li}_2\left(1+(-1)^{3/4}\right)+\text{Li}_2\left(3 i-2 i
   \sqrt{2}\right)-2 \text{Li}_2\left(\frac{1}{2} \left((-1-i)+\sqrt{2}\right)\right)\right.\\
   \left.+2 \text{Li}_2\left(\frac{1}{2}
   \left((-1+i)+\sqrt{2}\right)\right)-2 \text{Li}_2\left(\frac{2}{(1-i)+\sqrt{2}}\right)+2
   \text{Li}_2\left(\frac{2}{(1+i)+\sqrt{2}}\right)+2 \text{Li}_2\left(\frac{2 i}{(1+3 i)+(1+2 i)
   \sqrt{2}}\right)\right.\\
   \left.-\text{Li}_2\left(i \left(-3+2 \sqrt{2}\right)\right)-2 \text{Li}_2\left(\frac{2}{(3+i)+(2+i)
   \sqrt{2}}\right)\right\}+\pi  \log \left(7880+5572 \sqrt{2}\right)\ ,
   \end{multline}
where $G$ is Catalan's constant and $\text{Li}_2$ is a polylogarithm of order 2.  For $D=2$ the dominant $N$ part of~\eqref{rescaled-counterterm-integral}, after subtracting off the logarithmic singularity in $\sqrt{x}$, is logarithmic,
\begin{equation}
 \int_{-N/2}^{+N/2} \mathrm{d}^2n\; \PV \frac{1}{\bm n^2 - x}-2\pi\log\left(\sqrt{x}\right)= 2\pi\log\left(\frac{N}{2}\right)-4\left(G-\frac{\pi }{2}\log(2)\right)+\mathcal{O}(N^{-1})=2\pi\log\left(\counterterm_2^{\dispersion}\frac{N}{2}\right)+\mathcal{O}(N^{-1})\ ,
\end{equation}
with
\begin{equation}
\counterterm_2^\dispersion=\exp\left(\log(2)-G\frac{2}{\pi}\right)\ .
\end{equation}
