\subsection{Tuning to phase shifts on the lattice}\label{sec:tuning}

Observables in a FV lattice theory must converge against their physical counterpart in the IV and continuum limit.
Because the hamiltonian is implmented in a discrete finite volume, we must match the interactions to a discrete finite observable.
The implementation is well behaved if, after the matching of one observable (input), the IV and continuum limit for other observales (prediction) converges against the physical counterpart.
The implementation of the hamiltonian includes only one parameter, the strength of the contact interaction, which we match against scattering information.
The goal is to reproduce the entire phase shifts as a function of scattering momenta by matching the interactions to just one scattering momentum point.
The FV scattering spectrum can be exactly computed by using L\"{u}scher's formalism \textit{in reverse}: we compute the intersections of the phase shifts with the zeta function to extract the finit volume spectrum (see also \Figref{tuning})
\begin{equation}
    S^\spherical_D(x_i)
    =
    \frac{1}{\pi^2 L^{D-2}}
    \begin{cases}
        -\frac{1}{a_{03} \pi}   & D=3\\
        \frac{1}{\pi}\log(x_i) + \frac{2}{\pi} \log \left(\frac{ 2 \pi R_{02}}{L}\right) & D=2\\
        2 a_{01}   & D=1
    \end{cases}
	\, .
\end{equation}

We compute the FV discrete spectrum $\{E_i(\epsilon)\}$, adjust the contact interaction such that the ground state energy $E_0(\epsilon)$ matches the first zero of the zeta function $x_0 = 2 \mu E_0(\epsilon) L^2 / (2 \pi)^2$  and repeat the procedure for different lattice spacings.
