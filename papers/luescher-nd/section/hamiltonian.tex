\section{Discretized Hamiltonian}\label{sec:hamiltonian}

We consider a cubic finite volume (FV) of linear size $L$ with periodic boundary conditions and lattice spacing $\epsilon$ so that $N=L/\epsilon$ is an even integer that counts the number of sites in one spatial direction.

The contact interaction Hamiltonian \eqref{hamiltonian} is implemented on the lattice as an entirely local operator, vanishing everywhere except at the origin where it is of strength $C$---e.g., interaction is not smeared.
The Hamiltonian is given by
\begin{equation}
    \left\langle \vec{r}' \middle| H \middle| \vec{r} \right\rangle
    \rightarrow
    H_{\vec{r}',\vec{r}}^\dispersion
    =
    \frac{1}{2\mu} K_{\vec{r}',\vec{r}}^\dispersion + \frac{1}{\epsilon^D} C^\dispersion \delta_{\vec{r}',\vec{r}} \delta_{\vec{r},\vec{0}}
\end{equation}
where $K$ is a discretized Laplacian, implementing the momentum squared.
The $\dispersion$ symbol indicates that quantities depend on the lattice spacing $\epsilon$ and the explicit implementation of discretization effects like derivatives.

To ensure we control the discretization effects in generality, we study a variety of kinetic operators $K_{xy}^\dispersion$.
An often-used set of finite-difference kinetic operators are constructed from the one-dimensional finite-difference Laplacian that reaches $\nstep$ nearest neighbors,
\begin{equation}
    \Laplacian^\dispersion_{r'r} = \frac{1}{\epsilon^2}\sum_{s=-\nstep}^{\nstep} \gamma_{\abs{s}}^{(\nstep)} \delta_{r',r+\epsilon s}^{(L)}
\end{equation}
where the $(L)$ index of the Kronecker delta indicates that the spatial indices are understood modulo the periodic boundary conditions of the lattice.
In $D$ dimensions we simply take on-axis finite differences, so that the Laplacian is a $(1+2\nstep D)$-point stencil
\begin{equation}\label{eq:kinetic}
    K_{\vec{r}',\vec{r}}^\dispersion(\epsilon)
    =
    - \sum_{d=1}^D \Laplacian_{r_d'r_d^{}}^\dispersion
    \, .
\end{equation}

In the Fourier transformed space, momentum space, the one-dimensional Laplacian may be written
\begin{equation}
    \label{eq:laplacian}
    -\Laplacian_{r'r^{}}^\dispersion
    \overset{\text{F.T.}}{\longleftrightarrow}
    \Laplacian^\dispersion_{p'p}
    =
    \frac{1}{\epsilon^2}
    \delta_{p'p}
    \sum_{s=0}^{\nstep} \gamma_s^{(\nstep)} \cos(s p \epsilon)
    \, , \qquad p = \frac{2 \pi}{L} n \,
\end{equation}
where $n$ is an integer.
In $D$ dimensions we just sum the same expression over the different components of momentum.
Note that this is a specialization, in the sense that it contains no off-axis differencing (in position space) or products of different components (in momentum space).
However, since the numerical formalism we will describe is valid for every $\nstep$, we believe it holds for every possible kinetic operator.

\begin{figure}
    \section{The Dispersion \Luscher Finite-Volume Counterterm}\label{sec:counterterm/dispersion}

To evaluate the infinite-volume integral in \eqref{dispersion S},
\begin{equation}
    \left(\frac{N}{2}\right)^{D-2}
    4\pi^2 \int_{-N/2}^{+N/2} \mathrm{d}^Dn\; \PV \frac{1}{N^2 \sum_{ds} \gamma^{(\nstep)}_s \cos \frac{2\pi n_d s}{N} - x}
    =
    4\pi^2 \left(\frac{N}{2}\right) \int_{-1}^{+1} \mathrm{d}^D\nu\; \PV \frac{1}{4\sum_{ds} \gamma^{(\nstep)}_s \cos \pi \nu s - \tilde{x}}
\end{equation}
where as in the Cartesian case we rescaled and $\tilde{x} = x/(N/2)^2$.
Note that when we replace the sum over dimensions and steps with the exact $p^2$ result $(\pi \nu)^2$ and let $\tilde{x}$ vanish, we can match \eqref{cartesian S counterterm}.

We can use the same trick to isolate the leading behavior in $N/2$, introducing the dispersion relation in the exponent rather than $n^2$.
One finds a coefficient multiplying $(N/2)^{D-2}$,
\begin{equation}
    \label{eq:dispersion counterterm}
    \mathcal{L}^{\dispersion (\nstep)}_D = 4 \pi^2 \int_{0}^{\infty} 2\mu\; \mathrm{d}\mu\; \left(\int_{-1}^{+1} \mathrm{d}\nu\; e^{-4\mu^2 \sum_s \gamma_s^{(\nstep)} \cos \pi \nu s}\right)^D
\end{equation}
which can be numerically evaluated quickly, assuming one has the dispersion relation coefficients in hand.
In \Figref{nstep counterterm} we show this counterterm and how it differs from the $\nstep=\infty$ counterterm.

\begin{figure}
    \includegraphics{figure/counterterm-nstep.pdf}
    \caption{In the top panel we show the dispersion counterterm $\mathcal{L}^{\dispersion (\nstep)}_{3}$in \eqref{dispersion counterterm} as a function of $\nstep$, and the $\nstep=\infty$ result, which is the same as the Cartesian result $\mathcal{L}^{\cartesian}_3$, as a dashed line.  In the bottom panel we show a better view into how the counterterm converges to the Cartesian one.
    }
    \label{fig:nstep counterterm}
\end{figure}

    \caption{We show the continuum dispersion relation of energy as a function of momentum for different one-dimensional $\nstep$ derivatives.  For a finite number of lattice points $N$, the allowed momenta are evenly-spaced in steps of $2\pi/N$.
    As additional steps are incorporated into the finite difference, the dispersion relation more and more faithfully reproduces the desired $p^2$~behavior of $\nstep=\infty$.
    }
    \label{fig:dispersion relation}
\end{figure}

The coefficients $\gamma_{s}^{(\nstep)}$ are determined by requiring the dispersion relation be as quadratic as possible,
\begin{equation}
    \label{eq:gamma determination}
    \Laplacian^\dispersion_{p'p}
    \overset{!}{=}
    \delta_{p'p} \;
    p^2 \left[
        1 + \order{(\epsilon p)^{2\nstep}}
    \right].
\end{equation}
Additionally, we study a nonlocal operator with $\nstep=\infty$ which, in momentum space, can be implemented by multiplying by $p^2$ directly,
\begin{equation}
    \lim\limits_{n_s \to \infty}
    \Laplacian^\dispersion_{p'p}
    =
    \delta_{p'p} p^2,
\end{equation}
including at the edge of the Brillouin zone, the Laplacian implementation of the ungauged SLAC derivative.
Including the edge of the Brillouin zone does not introduce a discontinuity at the boundary, nor does including the corners pose any problem.
In addition to the $\nstep=\infty$ operator, we also call this kinetic operator the \emph{exact-$p^2$} operator.
The resulting dispersion relations are presented in \Figref{dispersion relation} for a variety of $\nstep$s and in \Appref{coefficients} we collect the required $\gamma$ coefficients.
In \Refs{Endres:2011er,Endres:2012cw} the exact dispersion relation is cut off by a LEGO sphere in momentum space (see equation (6) and the discussion after (9) in those references, respectively).
The formalism we develop here should work for these cut off operators, though the analytic results are harder to extract and we do not discuss such operators further.

The Hamiltonian in momentum space reads
\begin{equation}
    \label{eq:p space hamiltonian}
    \left\langle \vec{p}' \middle| H \middle| \vec{p} \right\rangle
    \rightarrow
    H_{\vec{p}',\vec{p}}^\dispersion
    =
    \frac{4 \pi^2}{2\mu L^2} \tilde K_{\vec n \vec n}^{N}
    +\frac{1}{L^D}C^\dispersion
\end{equation}
where $\vec{p} = 2\pi \vec{n}/L$ for a $D$-plet of integers $\vec{n} \in (-N/2, +N/2]^D$, and the coefficients $\gamma_{s}^{(\nstep)}$ are determined as described above.
Furthermore, we replaced the lattice-spacing-dependent kinetic Hamiltonian with the $N$-dependent
\begin{equation}\label{eq:normalized-kinetic-hamitlonian}
	\tilde K_{\vec n \vec n}^{N}
	= \frac{L^2}{4\pi^2} K_{\vec p \vec p}^{\dispersion} \bigg|_{p=\frac{2\pi \vec n}{L}}
	= \frac{N^2}{4\pi^2}
    \sum_{i=1}^{D}\sum_{s=0}^{\nstep} \gamma_s^{(\nstep)} \cos\left(\frac{2 \pi s n_i}{N}\right)
\end{equation}
which goes to $n^2$ in the continuum limit $N\goesto\infty$.

Although the non-interacting energy levels are no longer proportional to $n^2$ at generic \nstep, $n^2$ is still a useful classification for states, as long as it is understood simply as the magnitude of the lattice momentum, rather than as a proxy for energy.

\subsection{Reduction to  \texorpdfstring{\Aoneg}{A-one-g}}

Because we are interested in contact interactions, infinite-volume arguments suggest that only the s-wave will feel the interaction; such arguments translate to the lattice relatively cleanly.
Since the s-wave is most like \Aoneg we will focus on the spectrum in that irreducible representation of the cubic symmetry group $O_h$ in three dimensions, of the symmetry group of the square $D_{4h}$ in two dimensions, or $Z_2$ in one dimension, where an \Aoneg restriction amounts to focusing on parity-even states.

With a projection operator to the \Aoneg sector $P_{\Aoneg}$ we can raise the energy of all the other states an arbitrary amount $\alpha$ by supplementing the Hamiltonian
\begin{equation}
    H(\alpha) = H + \alpha (\one - P_{\Aoneg}) \, ,
\end{equation}
Because $P_{\Aoneg}$ commutes with $H$, $H$ and $H(\alpha)$ have the same spectrum within the $\Aoneg$ irrep.
If $\alpha$ is much larger than the expected energies of the Hamiltonian, the \Aoneg states remain low-lying and all other states are shifted to much higher energies.
Then, exact diagonalization for low-lying eigenvalues of $H(\alpha)$ provides an easier extraction of \Aoneg eigenenergies.

Because of the simplicity of \Aoneg we can also easily construct the Hamiltonian directly in that sector.
In momentum space we can label plane wave states by a vector on integers $\vec{n}$.
In the \Aoneg basis we can use one plane wave label and understand that we intend a normalized unweighted average of every plane wave state.
That is,
\begin{equation}
    \ket{\Aoneg\; \vec{n}} = \frac{1}{\sqrt{\normalization}} \sum_{g \in G} \ket{ g\vec{n}}
\end{equation}
where $g$ is an element of the group $G$, the sum is over all inequivalent states, and $\normalization$ the normalization.
When $\vec{n}$ is large we should be careful not to double-count states that live right on the edge of the Brillouin zone.
The states may be labeled by symmetry-inequivalent vectors with components all as large as $N/2$.
As a simple example, in three dimensions the $N/2(1,1,1)$ plane wave state in one corner of the Brillouin zone is invariant under all the $O_h$ operations modulo periodicity in momentum space, so $\normalization=1$ for that state.

Formulated in this basis, the kinetic energy operator remains diagonal and proportional to $n^2$ when $N\goesto\infty$.
Reading off the momentum-state potential matrix element from \eqref{p space hamiltonian}, the contact interaction is given by
\begin{equation}
    \braMket{\Aoneg\; \vec{n}'}{V}{\Aoneg\; \vec{n}}
    =
    \sum_{g'g \in G}
        \frac{1}{\sqrt{\normalization'}\sqrt{\normalization}} \braMket{g'\vec{n}'}{V}{g\vec{n}}
    =
    \frac{C^\dispersion}{L^D} \sqrt{\normalization'\normalization},
\end{equation}
so that every \Aoneg state talks to every other.
So, the Hamiltonian is in this sector is
\begin{equation}
    H_{\vec{n}'\vec{n}}^\dispersion = \frac{4 \pi^2}{2\mu L^2} \tilde K_{\vec n \vec n}^{N} + \frac{C^\dispersion}{L^D} \sqrt{\normalization'\normalization}
\end{equation}
and we divide by $4\pi^2/\mu L^2$ to make everything dimensionless.

We have implemented both this \Aoneg-only Hamiltonian and the general Hamiltonian with an energy penalty for non-\Aoneg states and verified that the spectra match where expected to as much precision as desired.

For a given $N$ some $n^2$ shells have multiple solutions.
When $N=5$ there are two $n^2=9$ shells corresponding to $n=(2,2,1)$ and $n=(3,0,0)$, which lives on the edge of the Brillouin zone.
For the contact interaction and $\nstep=\infty$, one linear combination of these \Aoneg states overlaps the $s$-wave and has a nontrivial finite-volume energy, and the other overlaps a higher partial wave and has $x=9$ to machine precision.
For finite $\nstep$, the non-interacting $x$ are not perfect integers, but the same moral holds.
In contrast, when $N=4$ there is no $n=(3,0,0)$ state, and the $(2,2,1)$ state is itself an eigenstate.
When $N$ is very large sometimes there are multiple eigenstates that have no support for the delta function---$n^2=41, 50, 54\ldots$ have two non-interacting states, while $n^2=81, 89, 101\ldots$ have three non-interacting states, and $n^2=146$ is the first shell with four non-interacting states, for example.
We do not discuss these non-interacting states further.
