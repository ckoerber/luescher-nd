\section{Discretized Hamiltonian}\label{sec:hamiltonian}

We consider a cubic finite volume (FV) of linear size $L$ with periodic boundary conditions and lattice spacing $\epsilon$ so that $N=L/\epsilon$ is an even integer that counts the number of sites in one spatial direction.

The contact interaction is implemented on the lattice as an entirely local operator, vanishing everywhere except at the origin where it is of strength $C$.
The contact interaction is not smeared.
The Hamiltonian is given by
\begin{equation}
    \left\langle \vec{r}' \middle| H \middle| \vec{r} \right\rangle
    \rightarrow
    H_{\vec{r}',\vec{r}}
    =
    \frac{1}{2\mu} K_{\vec{r}',\vec{r}} + \frac{1}{\epsilon^D} C(\epsilon) \delta_{\vec{r}',\vec{r}} \delta_{\vec{r},\vec{0}}
\end{equation}
where $K$ is a discretized Laplacian, implementing the momentum squared.

To ensure we control the discretization effects in generality, we study a variety of kinetic operators $K_{xy}$.
An often-used set of finite-difference kinetic operators are constructed from the one-dimensional finite-difference Laplacian
\begin{equation}
    \Laplacian^{(\nstep)}_{r'r} = \frac{1}{\epsilon^2}\sum_{s=-\nstep}^{\nstep} \gamma_{\abs{s}}^{(\nstep)} \delta_{r',r+\epsilon s}
\end{equation}
where the spatial indices are understood modulo the periodic boundary conditions of the lattice.
In $D$ dimensions we simply take on-axis finite differences, so that the Laplacian is a $(1+2\nstep D)$-point stencil
\begin{equation}
    K_{\vec{r}',\vec{r}}
    =
    - \sum_{d=1}^D \Laplacian_{r_d'r_d^{}}^{(\nstep)}
    \, .
\end{equation}

In the Fourier transformed space, momentum space, the one-dimensional Laplacian may be written
\begin{equation}
    \label{eq:laplacian}
    -\Laplacian_{r'r^{}}^{(\nstep)}
    \overset{\text{F.T.}}{\longleftrightarrow}
    \Laplacian^{(\nstep)}_{p'p}
    =
    \frac{1}{\epsilon^2}
    \delta_{p'p}
    \sum_{s=0}^{\nstep} \gamma_s^{(\nstep)} \cos(s p \epsilon)
\end{equation}
while in $D$ dimensions we just sum the same expression over the different components of momentum.
Note that this is a specialization, in the sense that it contains no off-axis differencing (in position space) or products of different components (in momentum space).

The coefficients $\gamma_{s}^{(\nstep)}$ are determined by requiring the dispersion relation be as quadratic as possible,
\begin{equation}
    \label{eq:gamma determination}
    \Laplacian^{(\nstep)}_{p'p}
    \overset{!}{=}
    \delta_{p'p} \;
    p^2 \left[
        1 + \order{(\epsilon p)^{2\nstep}}
    \right].
\end{equation}
The resulting dispersion relations are presented in \Figref{dispersion relation} for a variety of $\nstep$s and in \Appref{coefficients} we collect the required $\gamma$ coefficients.

\begin{figure}
    \subsubsection{Dispersion L\"{u}scher removes induced artifacts}

In this section, we again attempt to tune our contact interaction to unitarity by matching the first zero of the \Luscher zeta function.
However, the difference is that at each lattice spacing we tune to that spacing's respective $S^{\dispersion N}_D$, leveraging the dispersion relation for that derivative.
Then, when we extract finite-volume and finite-spacing energy levels, we put them through the dispersion equation \todo{eqref} using the same $S$ function.\footnote{We actually make the replacement $\F_D^\epsilon\goesto\F_D$ to avoid numerically evaluating $F_D^\epsilon$.  With a controlled continuum limit the error from this change obviously vanishes.}
The numerical results of said procedure are shown in \Figref{unimproved dispersion}.
Note that the results for $p\cot\delta$ are now flat across the spectrum, matching the known result for a contact interaction.
Moreover, comparing the scale to that in, for example, \Figref{unimproved spherical}, there the deviations were of order~1, while here the results remain within $10^{-8}$ of zero, with the value entirely reflecting how well the contact interaction was tuned.

\begin{figure}[htb]
    \scalebox{0.9}{\input{figure/ere-contact-fitted_a-inv_+0.0_zeta_dispersion_projector_a1g_n-eigs_200.pgf}}
    \caption{The same as \Figref{unimproved spherical}, but tuned and subsequently analyzed using the appropriate latticized \Luscher function, matching the cutoff on the sum to the lattice scale and accounting for the dispersion relation.}
    \label{fig:unimproved dispersion}
\end{figure}

\begin{figure}[hbt]
    \scalebox{0.9}{\input{figure/ere-contact-fitted_a-inv_-5.0_zeta_dispersion_projector_a1g_n-eigs_200.pgf}}
    \caption{The same as \Figref{unimproved spherical}, but tuned and subsequently analyzed using the appropriate latticized \Luscher function, matching the cutoff on the sum to the lattice scale and accounting for the dispersion relation for finite scattering lenght.}
    \label{fig:unimproved dispersion finite a}
\end{figure}

In \Figref{dispersion running of strength} we show how the strength of the contact interaction runs with the lattice scale.  According to \todo{something we know} it should be \todo{some formula that depends on cutoff}.

\begin{figure}
    \input{figure/contact-scaling-contact-fitted_a-inv_+0.0_zeta_dispersion_projector_a1g_n-eigs_200.pgf}
    \caption{
        Scaling of the contact interaction strength $C(\epsilon)$ fitted using the dispersion method at unitarity.
        Data points are fitted values, solid lines are analytical scaling predictions following $C(\epsilon) = \frac{2}{\mu} \frac{1}{\mathcal L^{\dispersion n_\mathrm{step}}} \epsilon $ and the dashed line corresponds to the spherical predictions.
        Bar diagrams below present the absolute error between prediction and extracted value.
    }
    \label{fig:dispersion running of strength}
\end{figure}

\clearpage

    \caption{We show the continuum dispersion relation of energy as a function of momentum for different one-dimensional $\nstep$ derivatives.  For a finite number of lattice points $N$, the allowed momenta are evenly-spaced in steps of $2\pi/N$.
    As additional steps are incorporated into the finite difference, the dispersion relation more and more faithfully reproduces the desired $p^2$~behavior of $\nstep=\infty$.
    }
    \label{fig:dispersion relation}
\end{figure}

Additionally, we study a nonlocal operator with $\nstep=\infty$ which, in momentum space, can be implemented by multiplying by $p^2$ directly,
\begin{equation}
    \Laplacian^{(\infty)}_{p'p}
    =
    \delta_{p'p} p^2,
\end{equation}
including at the edge of the Brillouin zone, the Laplacian implementation of the ungauged SLAC derivative.
Including the edge of the Brillouin zone does not introduce a discontinuity at the boundary.
We also call this kinetic operator the \emph{exact-$p^2$} operator.
The resulting dispersion relations are presented in \Figref{dispersion relation} for a variety of $\nstep$s and
in \Appref{coefficients} we collect the required $\gamma$ coefficients.

The Hamiltonian in momentum space reads
\begin{equation}
    \label{eq:p space hamiltonian}
    \left\langle \vec{p}' \middle| H \middle| \vec{p} \right\rangle
    \rightarrow
    H_{\vec{p}',\vec{p}}
    =
    \frac{1}{2\mu} K_{\vec{p}'p}
    +\frac{1}{L^D}C(\epsilon)
\end{equation}
where $\vec{p} = 2\pi \vec{n}/L$ for a $D$-plet of integers $\vec{n} \in (-N/2, +N/2]^D$, and the coefficients $\gamma_{s}^{(\nstep)}$ are determined as described above.


\subsection{Reduction to \Aoneg}

Because we are interested in contact interactions, infinite-volume arguments suggest that only the s-wave will feel the interaction.
Since the s-wave is most like \Aoneg we will focus on the spectrum in that irreducible representation of the cubic symmetry group $O_h$ in three dimensions, of the symmetry group of the square $D_{4h}$ in two dimensions, or $Z_2$ in one dimension, where an \Aoneg restriction amounts to focusing on parity-even states.

With a projection operator $P_{\Aoneg}$ we can raise the energy of all the other states an arbitrary amount $\alpha$ by supplementing the Hamiltonian
\begin{equation}
    H(\alpha) = H + \alpha (\one - P_{\Aoneg}) \, ,
\end{equation}
Because $P_{\Aoneg}$ commutes with $H$, $H$ and $H(\alpha)$ have the same spectrum within the $\Aoneg$ irrep.
If $\alpha$ is much larger than the expected energies of the Hamiltonian, the \Aoneg states remain low-lying and all other states are shifted to much higher energies.
Then, exact diagonalization for low-lying eigenvalues of $H(\alpha)$ provides an easier extraction of \Aoneg eigenenergies.

Because of the simplicity of \Aoneg we can also easily construct the Hamiltonian directly in that sector.
In momentum space we can label plane wave states by a vector on integers $\vec{n}$.
In the \Aoneg basis we can use one plane wave label and understand that we intend a normalized unweighted average of every plane wave state.
That is,
\begin{equation}
    \ket{\Aoneg\; \vec{n}} = \frac{1}{\sqrt{\normalization}} \sum_{g \in G} \ket{ g\vec{n}}
\end{equation}
where $g$ is an element of the group $G$, the sum is over all inequivalent states, and $\normalization$ the normalization.
When $\vec{n}$ is large we should be careful not to double-count states that live right on the edge of the Brillouin zone.
The states may be labeled by symmetry-inequivalent vectors with components all as large as $N/2$.
As a simple example, in three dimensions the $N/2(1,1,1)$ plane wave state in one corner of the Brillouin zone is invariant under all the $O_h$ operations modulo periodicity in momentum space.

Formulated in this basis, the kinetic energy operator remains diagonal and proportional to $n^2$.
Reading off the momentum-state potential matrix element from \eqref{p space hamiltonian}, the contact interaction is given by
\begin{equation}
    \braMket{\Aoneg\; \vec{n}'}{V}{\Aoneg\; \vec{n}}
    =
    \sum_{g'g \in G}
        \frac{1}{\sqrt{\normalization'}\sqrt{\normalization}} \braMket{g'\vec{n}'}{V}{g\vec{n}}
    =
    \frac{C(\epsilon)}{L^D} \sqrt{\normalization'\normalization},
\end{equation}
so that every \Aoneg state can talk to every other.
So, the Hamiltonian is
\begin{equation}
    H_{\vec{n}'\vec{n}} = \frac{1}{2\mu} K_{\vec{n'}\vec{n}} + \frac{C}{L^D} \sqrt{\normalization'\normalization}
\end{equation}
where the kinetic piece is understood with appropriate factors of $2\pi/L$.

We have implemented both this \Aoneg-only Hamiltonian and the general Hamiltonian with an energy penalty for non-\Aoneg states and verified that the spectra match where expected to as much precision as desired.
