\section{Discretized Hamiltonian}\label{sec:hamiltonian}

The simplest interacting two-body problem, two particles interacting via a contact interaction, is the focus of our study
\begin{equation}
    \label{eq:particle hamiltonian}
    H = \frac{p_1^2}{2 m_1} + \frac{p_2^2}{2 m_2} + C \delta(x_1-x_2)
    \, .
\end{equation}
The subscripts indicate the particle of the position and momentum operators.
This hamiltonian becomes, moving to center-of-mass and relative coordinates,
\begin{equation}
    \label{eq:hamiltonian}
    H = \frac{P^2}{2 M} + \frac{p^2}{2 \mu} + C \delta(x)
\end{equation}
where capital letters represent center-of-mass variables, lower case implies relative coordinates, and $\mu$ is the reduced mass.
The problem is reducded to an effective one-body problem once we specialize to the rest frame, setting $P=0$.

We consider a finite cubic volume (FV) of linear size $L$ with periodic boundary conditions and lattice spacing $\epsilon$ so that $N=L/\epsilon$ is an even integer that counts the number of sites in one spatial direction.

The contact interaction is implemented on the lattice as an entirely local operator, vanishing everywhere except at the origin where it is of strength $C$.

In contrast, we study a variety of kinetic operators which we distinguish by the $\nstep$ label, which indicates how many nearest neighbors in each direction go into the finite-difference Laplacian.
For example, $\nstep=1$ denotes the symmetric nearest-neighbor finite-difference Laplacian.
We consider further stencils which extend on-axis steps so that the finite difference Laplacian is a $(1+2\nstep D)$-point stencil in $D$ dimensions,
\begin{equation}
    \left\langle \vec{r}' \middle| H \middle| \vec{r} \right\rangle
    \rightarrow
    H_{\vec{r}',\vec{r}}^{(L,\epsilon,\nstep)}
    =
    - \frac{1}{2 \mu \epsilon^2}
        \sum_{d=1}^{D} \sum_{s=-\nstep}^{+\nstep}
            \gamma^{(\nstep)}_{|s|} \delta_{\vec{r}',\vec{r}+\epsilon s \vec{e}_d}
    + \frac{1}{\epsilon^3}C(\epsilon) \delta_{\vec{r}',\vec{r}}\delta_{\vec{r},\vec{0}}
\end{equation}
where the spatial indices are understood modulo the periodic boundary conditions of the lattice.
In momentum space, this Hamiltonian may be written as
\begin{align}
    \label{eq:p space hamiltonian}
    \left\langle \vec{p}' \middle| H \middle| \vec{p} \right\rangle
    \rightarrow
    H_{\vec{p}',\vec{p}}^{(L,\epsilon,\nstep)}
    &=
    \delta_{\vec{p}',\vec{p}} \frac{1}{2\mu} \sum_{d=1}^{D} \omega^{(\nstep)}(p_d,\epsilon)
    +\frac{1}{L^3}C(\epsilon)
    \\
    \label{eq:gamma definition}
    \omega^{(\nstep)}(p_d,\epsilon)
    &= \frac{1}{\epsilon^2} \sum_{s=0}^{\nstep} \gamma_{s}^{(\nstep)} \cos(s p_d \epsilon)
\end{align}
where $\vec{p} = 2\pi \vec{n}/L$ for a $D$-plet of integers $\vec{n} \in (-N/2, +N/2]^D$, and the coefficients $\gamma_{s}^{(\nstep)}$ are determined by requiring the dispersion relation be as quadratic as possible,
\begin{equation}
    \label{eq:gamma determination}
    \omega^{(\nstep)}(p_d,\epsilon) \overset{!}{=} p_d^2 \left[ 1 + \order{(\epsilon p_d)^{2\nstep}}\right].
\end{equation}
The resulting dispersion relations are presented in \Figref{dispersion relation} for a variety of $\nstep$s and
in \Appref{coefficients} we collect the required $\gamma$ coefficients.
In addition, we use a nonlocal operator, denoted by $\nstep=\infty$ which, in momentum space can be implemented to multiplying by $p^2$ directly,
\begin{equation}
    \omega^{\infty}(p_d,\epsilon) = p_d^2,
\end{equation}
including at the edge of the Brillouin zone, the Laplacian implementation of the ungauged SLAC derivative.
Including the edge does not introduce a discontinuity at the boundary (though it does introduce a cusp).

Once constructed, a projection operator, is added to this Hamiltonian
\begin{equation}
    H(\alpha) = H + \alpha (\one - P_{\Aoneg}) \, ,
\end{equation}
where $P_{\Aoneg}$ is a projector to the \Aoneg irrep (needed for extracting S-wave information in the infinite volume).
Because $P_{\Aoneg}$ commutes with $H$, $H$ and $H(\alpha)$ have the same spectrum within the $\Aoneg$ irrep.
If $\alpha$ is much larger than the expected energies of the Hamiltonian, the \Aoneg states remain low-lying and all other states are shifted to much higher energies.
Then, exactly diagonalizing $H(\alpha)$ instead of $H$ provides an easier extraction of \Aoneg eigenenergies.

Throughout we focus on a three-dimensional system, though in \Appref{two-d} we study a two-dimensional system, where logarithmic divergences warrant special attention.

\begin{figure}
    \section{The Dispersion \Luscher Finite-Volume Counterterm}\label{sec:counterterm/dispersion}

To evaluate the infinite-volume integral in \eqref{dispersion S},
\begin{equation}
    \left(\frac{N}{2}\right)^{D-2}
    4\pi^2 \int_{-N/2}^{+N/2} \mathrm{d}^Dn\; \PV \frac{1}{N^2 \sum_{ds} \gamma^{(\nstep)}_s \cos \frac{2\pi n_d s}{N} - x}
    =
    4\pi^2 \left(\frac{N}{2}\right) \int_{-1}^{+1} \mathrm{d}^D\nu\; \PV \frac{1}{4\sum_{ds} \gamma^{(\nstep)}_s \cos \pi \nu s - \tilde{x}}
\end{equation}
where as in the Cartesian case we rescaled and $\tilde{x} = x/(N/2)^2$.
Note that when we replace the sum over dimensions and steps with the exact $p^2$ result $(\pi \nu)^2$ and let $\tilde{x}$ vanish, we can match \eqref{cartesian S counterterm}.

We can use the same trick to isolate the leading behavior in $N/2$, introducing the dispersion relation in the exponent rather than $n^2$.
One finds a coefficient multiplying $(N/2)^{D-2}$,
\begin{equation}
    \label{eq:dispersion counterterm}
    \mathcal{L}^{\dispersion (\nstep)}_D = 4 \pi^2 \int_{0}^{\infty} 2\mu\; \mathrm{d}\mu\; \left(\int_{-1}^{+1} \mathrm{d}\nu\; e^{-4\mu^2 \sum_s \gamma_s^{(\nstep)} \cos \pi \nu s}\right)^D
\end{equation}
which can be numerically evaluated quickly, assuming one has the dispersion relation coefficients in hand.
In \Figref{nstep counterterm} we show this counterterm and how it differs from the $\nstep=\infty$ counterterm.

\begin{figure}
    \includegraphics{figure/counterterm-nstep.pdf}
    \caption{In the top panel we show the dispersion counterterm $\mathcal{L}^{\dispersion (\nstep)}_{3}$in \eqref{dispersion counterterm} as a function of $\nstep$, and the $\nstep=\infty$ result, which is the same as the Cartesian result $\mathcal{L}^{\cartesian}_3$, as a dashed line.  In the bottom panel we show a better view into how the counterterm converges to the Cartesian one.
    }
    \label{fig:nstep counterterm}
\end{figure}

    \caption{We show the continuum dispersion relation of energy as a function of momentum for different one-dimensional $\nstep$ derivatives.  For a finite number of lattice points $N$, the allowed momenta are evenly-spaced in steps of $2\pi/N$.
    As additional steps are incorporated into the finite difference, the dispersion relation more and more faithfully reproduces the desired $p^2$~behavior of $\nstep=\infty$.
    }
    \label{fig:dispersion relation}
\end{figure}
