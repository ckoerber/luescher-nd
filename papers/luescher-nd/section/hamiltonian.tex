\section{Discretized Hamiltonian}\label{sec:hamiltonian}

A two-body system interacting via a contact interaction is the focus of our study.
By definition, only S-wave information are not vanishing for a contact interaction and the corresponding ERE has no effective range (and higher terms).
Thus, to obtain unitarity, one only has to tune the interaction to the S-wave scattering length.
The hamiltonian, with contact strength $C$, describing the system is given by
\begin{equation}
    \label{eq:particle hamiltonian}
    \hat H = \frac{\hat p_1^2}{2 m_1} + \frac{\hat p_2^2}{2 m_2} + C \delta(\hat x_1 - \hat x_2)
    \, .
\end{equation}
The subscripts indicate the particle of the position and momentum operators.
This hamiltonian becomes, moving to center-of-mass and relative coordinates,
\begin{equation}
    \label{eq:hamiltonian}
    \hat H = \frac{\hat P^2}{2 M} + \frac{\hat p^2}{2 \mu} + C \delta(\hat x)
\end{equation}
where capital letters represent center-of-mass variables, lower case implies relative coordinates.
The problem is reducded to an effective one-body problem once we specialize to the rest frame, setting $P=0$.

We consider a finite cubic volume (FV) of linear size $L$ with periodic boundary conditions and lattice spacing $\epsilon$ so that $N=L/\epsilon$ is an even integer that counts the number of sites in one spatial direction.

The contact interaction is implemented on the lattice as an entirely local operator, vanishing everywhere except at the origin where it is of strength $C$ (e.g., the contact interaction is not smeared).

In contrast, to analyze the effects of discretizations, we study a variety of kinetic operators which we distinguish by the $\nstep$ label, which indicates how many nearest neighbors in each direction go into the finite-difference Laplacian.
For example, $\nstep=1$ denotes the symmetric nearest-neighbor finite-difference Laplacian.
We consider further stencils which extend on-axis steps so that the finite difference Laplacian is a $(1+2\nstep D)$-point stencil in $D$ dimensions,
\begin{equation}
    \left\langle \vec{r}' \middle| H \middle| \vec{r} \right\rangle
    \rightarrow
    H_{\vec{r}',\vec{r}}^{(L,\epsilon,\nstep)}
    =
    - \frac{1}{2 \mu \epsilon^2}
        \sum_{d=1}^{D} \sum_{s=-\nstep}^{+\nstep}
            \gamma^{(\nstep)}_{|s|} \delta_{\vec{r}',\vec{r}+\epsilon s \vec{e}_d}
    + \frac{1}{\epsilon^D}C(\epsilon) \delta_{\vec{r}',\vec{r}}\delta_{\vec{r},\vec{0}}
\end{equation}
where the spatial indices are understood modulo the periodic boundary conditions of the lattice.
In momentum space, this Hamiltonian may be written as
\begin{align}
    \label{eq:p space hamiltonian}
    \left\langle \vec{p}' \middle| H \middle| \vec{p} \right\rangle
    \rightarrow
    H_{\vec{p}',\vec{p}}^{(L,\epsilon,\nstep)}
    &=
    \delta_{\vec{p}',\vec{p}} \frac{1}{2\mu} \sum_{d=1}^{D} \omega^{(\nstep)}(p_d,\epsilon)
    +\frac{1}{L^D}C(\epsilon)
    \\
    \label{eq:gamma definition}
    \omega^{(\nstep)}(p_d,\epsilon)
    &= \frac{1}{\epsilon^2} \sum_{s=0}^{\nstep} \gamma_{s}^{(\nstep)} \cos(s p_d \epsilon)
\end{align}
where $\vec{p} = 2\pi \vec{n}/L$ for a $D$-plet of integers $\vec{n} \in (-N/2, +N/2]^D$, and the coefficients $\gamma_{s}^{(\nstep)}$ are determined by requiring the dispersion relation be as quadratic as possible,
\begin{equation}
    \label{eq:gamma determination}
    \omega^{(\nstep)}(p_d,\epsilon) \overset{!}{=} p_d^2 \left[ 1 + \order{(\epsilon p_d)^{2\nstep}}\right].
\end{equation}
The resulting dispersion relations are presented in \Figref{dispersion relation} for a variety of $\nstep$s and
in \Appref{coefficients} we collect the required $\gamma$ coefficients.
In addition, we use a nonlocal operator, denoted by $\nstep=\infty$ which, in momentum space can be implemented to multiplying by $p^2$ directly,
\begin{equation}
    \omega^{\infty}(p_d,\epsilon) = p_d^2,
\end{equation}
including at the edge of the Brillouin zone, the Laplacian implementation of the ungauged SLAC derivative.
Including the edge does not introduce a discontinuity at the boundary (though it does introduce a cusp).

Once constructed, a projection operator, is added to this Hamiltonian
\begin{equation}
    H(\alpha) = H + \alpha (\one - P_{\Aoneg}) \, ,
\end{equation}
where $P_{\Aoneg}$ is a projector to the \Aoneg irrep (needed for extracting S-wave information in the infinite volume).
Because $P_{\Aoneg}$ commutes with $H$, $H$ and $H(\alpha)$ have the same spectrum within the $\Aoneg$ irrep.
If $\alpha$ is much larger than the expected energies of the Hamiltonian, the \Aoneg states remain low-lying and all other states are shifted to much higher energies.
Then, exactly diagonalizing $H(\alpha)$ instead of $H$ provides an easier extraction of \Aoneg eigenenergies.

Throughout we focus on a three-dimensional system, though in \Appref{two-d} we study a two-dimensional system, where logarithmic divergences warrant special attention.

\begin{figure}
    \subsubsection{Dispersion L\"{u}scher removes induced artifacts}

In this section, we again attempt to tune our contact interaction to unitarity by matching the first zero of the \Luscher zeta function.
However, the difference is that at each lattice spacing we tune to that spacing's respective $S^{\dispersion N}_D$, leveraging the dispersion relation for that derivative.
Then, when we extract finite-volume and finite-spacing energy levels, we put them through the dispersion equation \todo{eqref} using the same $S$ function.\footnote{We actually make the replacement $\F_D^\epsilon\goesto\F_D$ to avoid numerically evaluating $F_D^\epsilon$.  With a controlled continuum limit the error from this change obviously vanishes.}
The numerical results of said procedure are shown in \Figref{unimproved dispersion}.
Note that the results for $p\cot\delta$ are now flat across the spectrum, matching the known result for a contact interaction.
Moreover, comparing the scale to that in, for example, \Figref{unimproved spherical}, there the deviations were of order~1, while here the results remain within $10^{-8}$ of zero, with the value entirely reflecting how well the contact interaction was tuned.

\begin{figure}[htb]
    \scalebox{0.9}{\input{figure/ere-contact-fitted_a-inv_+0.0_zeta_dispersion_projector_a1g_n-eigs_200.pgf}}
    \caption{The same as \Figref{unimproved spherical}, but tuned and subsequently analyzed using the appropriate latticized \Luscher function, matching the cutoff on the sum to the lattice scale and accounting for the dispersion relation.}
    \label{fig:unimproved dispersion}
\end{figure}

\begin{figure}[hbt]
    \scalebox{0.9}{\input{figure/ere-contact-fitted_a-inv_-5.0_zeta_dispersion_projector_a1g_n-eigs_200.pgf}}
    \caption{The same as \Figref{unimproved spherical}, but tuned and subsequently analyzed using the appropriate latticized \Luscher function, matching the cutoff on the sum to the lattice scale and accounting for the dispersion relation for finite scattering lenght.}
    \label{fig:unimproved dispersion finite a}
\end{figure}

In \Figref{dispersion running of strength} we show how the strength of the contact interaction runs with the lattice scale.  According to \todo{something we know} it should be \todo{some formula that depends on cutoff}.

\begin{figure}
    \input{figure/contact-scaling-contact-fitted_a-inv_+0.0_zeta_dispersion_projector_a1g_n-eigs_200.pgf}
    \caption{
        Scaling of the contact interaction strength $C(\epsilon)$ fitted using the dispersion method at unitarity.
        Data points are fitted values, solid lines are analytical scaling predictions following $C(\epsilon) = \frac{2}{\mu} \frac{1}{\mathcal L^{\dispersion n_\mathrm{step}}} \epsilon $ and the dashed line corresponds to the spherical predictions.
        Bar diagrams below present the absolute error between prediction and extracted value.
    }
    \label{fig:dispersion running of strength}
\end{figure}

\clearpage

    \caption{We show the continuum dispersion relation of energy as a function of momentum for different one-dimensional $\nstep$ derivatives.  For a finite number of lattice points $N$, the allowed momenta are evenly-spaced in steps of $2\pi/N$.
    As additional steps are incorporated into the finite difference, the dispersion relation more and more faithfully reproduces the desired $p^2$~behavior of $\nstep=\infty$.
    }
    \label{fig:dispersion relation}
\end{figure}
