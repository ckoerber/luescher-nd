%!TEX root =  ../master.tex


\subsubsection{Results of the dispersion method in three dimensions}

In this section, we again attempt to tune our contact interaction to unitarity by matching the first zero of the \Luscher zeta function.
However, the difference is that at each lattice spacing we tune to that spacing's respective $S^{\dispersion}_D$, leveraging the dispersion relation for that derivative.
Then, when we extract finite-volume and finite-spacing energy levels, we put them through the dispersion equation \eqref{dispersion-zeta-contact} using the same $S^{\dispersion}$ function.
The numerical results of said procedure are shown in \Figref{unimproved dispersion}.
Note that the results for $p\cot\delta$ are now flat across the spectrum, matching the known result for a contact interaction.
Moreover, comparing the scale to that in, for example, \Figref{unimproved spherical}, there the deviations were of order~1, while here the results remain within $10^{-8}$ of zero, with the value entirely reflecting how well the contact interaction was tuned.

\begin{figure}[htb]
    \scalebox{0.9}{\input{figure/ere-contact-fitted_a-inv_+0.0_zeta_dispersion_projector_a1g_n-eigs_200.pgf}}
    \caption{The same as \Figref{unimproved spherical}, but tuned and subsequently analyzed using the appropriate latticized \Luscher function, matching the cutoff on the sum to the lattice scale and accounting for the dispersion relation.
    We emphasize the results are on a log scale, and the tuning was to $-1/a_0 = 0$.
    }
    \label{fig:unimproved dispersion}
\end{figure}

In \Figref{dispersion running of strength} we show how the strength of the contact interaction runs with the lattice scale according to \eqref{dispersion-renormalization}.
Note that the lines are not fits to the data; though the difference is down at $10^{-12}$ or better.
Again, this difference depends on the accuracy of the tuning.

\begin{figure}
    \scalebox{0.8}{\input{figure/contact-scaling-contact-fitted_a-inv_+0.0_zeta_dispersion_projector_a1g_n-eigs_200.pgf}}
    \caption{
        Scaling of the contact interaction strength $C_R^{(n_s)}(\epsilon)$ fitted using the dispersion method at unitarity.
        Data points are values of the contact interaction fitted to the first intersection of the phase shifts with the dispersion zeta function.
        The solid lines are analytical scaling predictions following \eqref{dispersion-renormalization} and the dashed line corresponds to the spherical counter term $ \mathcal{L}^{\spherical}_3 = 2 \pi$.
        Bar diagrams below present the absolute error between prediction and extracted value.
    }
    \label{fig:dispersion running of strength}
\end{figure}

We note that even in the limit of $n_s \to \infty$ the dispersion counter term $\mathcal{L}^{(n_s)}_3$ will not match the spherical counter term $\mathcal{L}^{\spherical}_3$.
The issue is that at any finite $N$ the spherical integral and cartesian integrals differ---if the radius of the sphere is $N/2$, the corners of the lattice's Brillouin zone are absent; the cartesian integral matches the Brillouin zone correctly, critical for any finite-$N$ result.

\begin{figure}
    \scalebox{0.8}{\input{figure/3dtuned.pgf}}
    \caption{
        Here we show a contact interaction in three dimensions with the ground state tuned to the first zero of the spherical zeta function $S^{\spherical}_3$ on cubic lattices with $N=10,$ 20, 40, 80 (squares, diamonds, hexagons, and circles, respectively), with the resulting spectrum analyzed with $S^{\spherical}_3$ (colored points) and the $N$-appropriate $S^{\dispersion}_3$ (black points).
        The gray dashed line is $S^\spherical_3$ and the thin vertical lines are at the non-interacting $x$s where it diverges.
        The colored lines are the second-order analytic prediction for the difference between the dispersion and spherical analysis as a function of $x$.
        For clarity of the continuum limit we show, in the bottom panel, a limited range in $x$ and $pL\cot\delta_{30}$, where it is clear that each $N$ hits the zero of $S^\spherical_3$ but that the flat behavior at any finite $N$ is away from an infinite scattering length when analyzed with $S^\dispersion_3$.
    }
    \label{fig:3d-corrections}
\end{figure}

In \Figref{3d-corrections} we show the result of tuning a finite-spacing contact interaction to the first zero of the continuum zeta function $S^\spherical_3$.
At each spacing the spectrum is fed through the continuum zeta for analysis, resulting in an apparent spacing-dependent momentum dependence that matches the small-$x$ expansion \todo{equation}.
The same spectrum is also fed through the spacing-appropriate dispersion zeta $S^\dispersion$, resulting in the flat black lines.
Shown in detail in the bottom panel, it's clear that the continuum limit taken this way results in any finite spacing having a nonzero scattering length that vanishes with the continuum limit.
In contrast, tuning to the dispersion function directly, as in \Figref{unimproved dispersion}, is flat and nearly zero at each individual lattice spacing.
We expect that this difference explains the induced momentum dependence of, for example, \Refs{Endres:2011er,Endres:2012cw}.
