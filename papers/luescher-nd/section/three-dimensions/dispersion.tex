%!TEX root =  ../master.tex
\subsubsection{The dispersion method in three dimensions}\label{sec:3D dispersion}
In this section we want to highlight that specifically for a contact interaction, it is possible to tune the contact strength parameter in a finite volume for a given discretization scheme in such a way that one directly obtains continuum infinite volume results -- without any further extrapolation -- when using the dispersion formalism.
We first present an analytic derivation, which may help to generalize this formalism to other interactions, and display results in the section afterwards.

According to \eqref{T matrix}, \eqref{I0} and \eqref{spherical FD}, we find that the phase shifts are related to the contact interaction by
\begin{equation}
	p \cot \delta_3(p)
	= \lim\limits_{\Lambda \to \infty}\frac{2 \pi}{\mu}\frac{1}{T(p, \Lambda)} + i p
	= \lim\limits_{\Lambda \to \infty}
		\frac{2 \pi}{\mu} \left[
			\frac{1}{C(\Lambda)} - I_3(p, \Lambda)
		\right]
	\, ,
\end{equation}
with the solution for the integral
\begin{equation}
	I_3(p, \Lambda)
	=
	-\frac{\mu}{2 \pi}
	\left[
	\frac{2 \Lambda}{\pi} + \frac{2  p}{\pi} \log \left( \frac{\Lambda - p}{\Lambda + p}\right)
	\right]
\end{equation}
Because the contact interaction does not have any momentum dependence, it is only possible to absorb momentum independent regulator terms when renormalizing the contact interaction.
Thus, it is still possible to completely renormalize the interaction such that the phase shifts, in the limit of $\Lambda \to \infty$, are independent of the cutoff by choosing
\begin{equation}
	\frac{2 \pi}{\mu} \frac{1}{C_R(\Lambda)} + \frac{2 \Lambda}{\pi} \equiv p \cot \delta_3(p) = - \frac{1}{a_3} \, .
\end{equation}


In particular, because the limit of $\Lambda \to \infty$ is well defined for this choice of the contact interaction strength $C_R(\Lambda)$, one is able to evaluate both sides for a given momentum, e.g., for $p=0$
\begin{equation}
	- \frac{1}{a_3}
	=
	\lim\limits_{p \to 0}\lim\limits_{\Lambda \to \infty}
		\left[
			\frac{2 \pi}{\mu}\frac{1}{T(p, \Lambda)} \bigg|_{C=C_R} + i p
	\right]
	=
	\lim\limits_{\Lambda \to \infty}
	\frac{2 \pi}{\mu}
		\left[
		\frac{1}{C_R(\Lambda)} - I_3(0, \Lambda)
		\right]
	\, .
\end{equation}

We now want to find an equivalent expression to the finite-volume zeta functions in presence of a discretization scheme.
In particular, the discretization scheme depends on the implementation of the kinetic operator $K^{(n_s)}$ and thus depends on the $n_s$ parameter.
The lattice spacing can be identified with the hard momentum cutoff $\Lambda = \pi / \epsilon$.
That is, the expectation value of the dispersion scales as $\hat K^{(n_s)}(\epsilon) \ket{p} = p^2 [1 + \mathcal O(\epsilon p)^{2 n_s}]\ket{p}$.

If one replaces the continuum momentum dispersion $q^2$ in $I_3$ with the kinetic operator for a given lattice spacing and discretization, one defines a sequence in $n_s$ which converges against $I_3$ in the limit of $n_s \to \infty$
\begin{equation}
	\lim\limits_{n_s \to \infty} I^{(n_s)}_3(p, \Lambda) = I_3(p, \Lambda)
	\, , \qquad
	I^{(n_s)}_3\left(p, \Lambda=\frac{\pi}{\epsilon} \right)
	=
	    \int\limits_{-\pi/\epsilon}^{+\pi/\epsilon}
        \mathrm{d}^3 \vec{q}
        \left[
            \PV \left(
                \frac{1}{
                    E - \frac{1}{2\mu} K_{qq}^{(n_s)} }
                \right)
            -i \pi \delta\left(E - \frac{1}{2\mu}K_{qq}^{(n_s)}\right)
        \right]
        \, .
\end{equation}
We furthermore define a sequence for the contact strength parameter depending on the cutoff and the employed discretization scheme which equivalently converges against the continuum result.
This sequence is determined by matching the against the dispersion integral for each value of the cutoff and for each discretization scheme
\begin{equation}\label{eq:dispersion-renormalization}
	\lim\limits_{n_s \to \infty} C^{(n_s)}_R(\Lambda) = C_R(\Lambda) \, ,
	\qquad
	- \frac{1}{a_3}
	\equiv
	\frac{2 \pi}{\mu}
		\left[
		\frac{1}{C_R^{(n_s)}(\Lambda)} - I_3^{(n_s)}(0, \Lambda)
		\right]
	\, .
\end{equation}
It is possible to make this choice since both terms do not depend on any external momentum $p$.
This is specific for the contact interaction.
One can view this choice as the renormalization equation for contact interaction in presence of lattice discretization which, by definition, trivially satisfy
\begin{equation}
	- \frac{1}{a_3}
	=
	\lim\limits_{\Lambda \to \infty} \lim\limits_{n_s \to \infty}
	\frac{2 \pi}{\mu}
		\left[
		\frac{1}{C_R^{(n_s)}(\Lambda)} - I_3^{(n_s)}(0, \Lambda)
		\right]
	\, .
\end{equation}
In fact, it satisfies this equation even without the limits.

Next we address how this renormalization choice relates to the dispersion zeta function.
For any lattice implementation of a contact interaction with strength $\tilde c$ in finite volume, the Schrödinger equation can be rewritten as
\begin{equation}
	0 = 1 - \tilde c I_{3, \FV}^{(n_s)}\left(\sqrt{2 \mu E_i}, \Lambda = \frac{\pi}{\epsilon}\right) \, ,
\end{equation}
where $E_i \equiv E_i^{(n_s)}(L, \epsilon, \tilde c)$ are the finite volume energy levels, which depend on the employed discretization scheme and on the contact interaction of strength $\tilde c$.
The finite volume sum $I_{3, \FV}^{(n_s)}(p, \pi/\epsilon)$ is obtained by replacing the integral $d^3 \vec q$ in $I_3^{(n_s)}(p, \Lambda)$ with a sum  over vectors $\vec q = 2 \pi \vec n / L$.
Because the above equation is true for any value of $\tilde c$ and it's corresponding spectrum, it is especially true for $\tilde c = C_R^{(n_s)}(\Lambda)$.
This means that
\begin{equation}
	- \frac{1}{a_3}
	=
	\frac{2 \pi}{\mu}
		\left[
		I_{3, \FV}^{(n_s)}\left(\sqrt{2 \mu E_i}, \frac{\pi}{\epsilon}\right)
		- I_3^{(n_s)}\left(0, \frac{\pi}{\epsilon}\right)
		\right]
	\, ,
\end{equation}
which defines the dispersion zeta function
\begin{align}\label{eq:dispersion-zeta-contact}
	S^{\dispersion}(x_i)
	&=
	S^{(n_s)}(x_i)
	=
	\sum\limits_n^N
	\frac{1}{K_{nn}^{(n_s)} - x_i} - \mathcal{L}_3^{\dispersion (n_s)} \frac{N}{2}
	\, ,
	\\
	\mathcal{L}_3^{(n_s)}
	&=
	\frac{2 \pi^2 L}{\mu}
	I_3^{(n_s)}\left(p=0, \Lambda = \frac{\pi}{\epsilon}\right)
	\overset{n_s \to \infty}{\longrightarrow}
	15.348248444887464047104634
	\, .
\end{align}
\todo{I am to tired to find out the factor now...}\todo{I put in the factor, but it is valid for $n_s\mapsto\infty$.  I don't know how to determine this factor as a function of $n_s$.--T.L.}
It furthermore explains why we expect results computed with this modified zeta function to directly match the continuum, infinite volume phase shifts.
This result does not hold for general finite-range interactions, i.e., if it not possible to make an equivalent choice as in \eqref{dispersion-renormalization}.
We stress that this derivation uses the analytic  expression for the $T$-matrix and simplifies drastically because the phase shifts for a renormalized contact interaction are momentum independent.
This momentum independence had the consequence that the counter term in \eqref{dispersion-zeta-contact} is momentum independent as well.


\subsubsection{Results of the dispersion method in three dimensions}

In this section, we again attempt to tune our contact interaction to unitarity by matching the first zero of the \Luscher zeta function.
However, the difference is that at each lattice spacing we tune to that spacing's respective $S^{\dispersion}_D$, leveraging the dispersion relation for that derivative.
Then, when we extract finite-volume and finite-spacing energy levels, we put them through the dispersion equation \eqref{dispersion-zeta-contact} using the same $S^{\dispersion}$ function.
The numerical results of said procedure are shown in \Figref{unimproved dispersion}.
Note that the results for $p\cot\delta$ are now flat across the spectrum, matching the known result for a contact interaction.
Moreover, comparing the scale to that in, for example, \Figref{unimproved spherical}, there the deviations were of order~1, while here the results remain within $10^{-8}$ of zero, with the value entirely reflecting how well the contact interaction was tuned.

\begin{figure}[htb]
    \scalebox{0.9}{\input{figure/ere-contact-fitted_a-inv_+0.0_zeta_dispersion_projector_a1g_n-eigs_200.pgf}}
    \caption{The same as \Figref{unimproved spherical}, but tuned and subsequently analyzed using the appropriate latticized \Luscher function, matching the cutoff on the sum to the lattice scale and accounting for the dispersion relation.
    Note that results are presented at a log scale.
    }
    \label{fig:unimproved dispersion}
\end{figure}

In \Figref{dispersion running of strength} we show how the strength of the contact interaction runs with the lattice scale according to \eqref{dispersion-renormalization}.
Note that the lines are not fits to the data; though the difference is down at $10^{-12}$ or better.
Again, this difference depends on the accuracy of the tuning.

\begin{figure}
    \scalebox{0.8}{\input{figure/contact-scaling-contact-fitted_a-inv_+0.0_zeta_dispersion_projector_a1g_n-eigs_200.pgf}}
    \caption{
        Scaling of the contact interaction strength $C_R^{(n_s)}(\epsilon)$ fitted using the dispersion method at unitarity.
        Data points are values of the contact interaction fitted to the first intersection of the phase shifts with the dispersion zeta function.
        The solid lines are analytical scaling predictions following \eqref{dispersion-renormalization} and the dashed line corresponds to the spherical counter term $ \mathcal{L}^{\spherical}_3 = 2 \pi$.
        Bar diagrams below present the absolute error between prediction and extracted value.
    }
    \label{fig:dispersion running of strength}
\end{figure}

We note that even in the limit of $n_s \to \infty$ the dispersion counter term $\mathcal{L}^{(n_s)}_3$ will not match the spherical counter term $\mathcal{L}^{\spherical}_3$.
The issue is that at any finite $N$ the spherical integral and cartesian integrals differ---if the radius of the sphere is $N/2$, the corners of the lattice's Brillouin zone are absent; the cartesian integral matches the Brillouin zone correctly, critical for any finite-$N$ result.

\todo{Do we discuss \emph{corrections} to spherical L\"uscher in the case of a contact interaction?  I.e., do we show how to fix the linear behavior found in figures 7 and 8 and Amy's paper?--T.L.}
