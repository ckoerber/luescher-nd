%!TEX root =  ../../master.tex

\subsection{Results of the dispersion method in three dimensions}

In this section, we again attempt to tune our contact interaction to unitarity by matching the first zero of the zeta function.
However, the difference is that at each lattice spacing we tune to that spacing's respective $S^{\dispersion}_3$, leveraging the dispersion relation for that derivative.
Then, when we extract finite-volume and finite-spacing energy levels, we put them through the dispersion equation \eqref{dispersion-zeta-contact} using the same $S^{\dispersion}$ function.
The numerical results of said procedure are shown in \Figref{unimproved dispersion}.
Note that the results for $p\cot\delta$ are now flat across the spectrum, matching the known result for a contact interaction.
Moreover, comparing the scale to that in, for example, \Figref{unimproved spherical}, there the deviations were of order~1, while here the results remain within $10^{-8}$ of zero, with the value entirely reflecting how well the contact interaction was tuned.
Put another way, we have verified that the dispersion zeta function provides exact finite-spacing energy levels for our contact interaction \eqref{p space hamiltonian}, just as one would hope for a contact interaction in the continuum.

\begin{figure}[!htb]
	\centering
    \scalebox{0.8}{\input{figure/ere-contact-fitted_a-inv_+0.0_zeta_dispersion_projector_a1g_n-eigs_200.pgf}}
    \caption{The same as \Figref{unimproved spherical}, but tuned and subsequently analyzed using the appropriate latticized \Luscher function, matching the cutoff on the sum to the lattice scale and accounting for the dispersion relation.
    We emphasize the results are on a log scale, and the tuning was to $-1/a_3 = 0$.
    }
    \label{fig:unimproved dispersion}
\end{figure}

In \Figref{dispersion running of strength} we show how the strength of the contact interaction runs with the lattice scale according to \eqref{dispersion-renormalization}.
Note that the lines are not fits to the data; though the difference is down at $10^{-12}$ or better.
Again, this difference depends on the accuracy of the tuning.

\begin{figure}[!htb]
	\centering
    \scalebox{0.7}{\input{figure/contact-scaling-contact-fitted_a-inv_+0.0_zeta_dispersion_projector_a1g_n-eigs_200.pgf}}
    \caption{
        Scaling of the contact interaction strength $C_R^{(n_s)}(\epsilon)$ fitted using the dispersion method at unitarity.
        Data points are values of the contact interaction fitted to the first intersection of the phase shifts with the dispersion zeta function.
        The solid lines are analytical scaling predictions following \eqref{dispersion-renormalization} and the dashed line corresponds to the spherical counter term $ \mathcal{L}^{\spherical}_3 = 2 \pi$.
        Bar diagrams below present the absolute error between prediction and extracted value.
    }
    \label{fig:dispersion running of strength}
\end{figure}

We note that
\begin{enumerate}
	\item when matching the contact interaction parameter using spherical \Luscher data and finite spacing eigenvalues, the data points did not exactly match the analytic spherical contact scaling.
	The error at the smallest lattice spacing had a relative error on the percent scale and it got worse for larger lattice spacings.
	\item even in the limit of $n_s \to \infty$ the dispersion counter term $\mathcal{L}^\dispersion_3$ will not match the spherical counter term $\mathcal{L}^{\spherical}_3$.
At any finite $N$ the spherical integral and cartesian integrals differ---if the radius of the sphere is $N/2$, the corners of the lattice's Brillouin zone are absent; the cartesian integral matches the Brillouin zone correctly, critical for any finite-$N$ result.
\end{enumerate}



\subsection{Momentum-induced terms of  \texorpdfstring{$S^\bigcirc_3(x^\dispersion)$}{S3-spherical} due to discretization\label{sec:3d induced momenta}}
The zeta function in L\"uscher's formula, $S^\bigcirc_3(x)$, is derived in the continuum.
As such, it requires continuum energies $x$ for its argument.
If one instead feeds discretized energies $x^\dispersion$ through $S^\bigcirc_3(x)$ then momentum-dependent terms are subsequently induced.

\begin{figure}[htb]
    \scalebox{0.8}{\input{figure/3dtuned.pgf}}
    \caption{
        Here we show a contact interaction in three dimensions with the ground state tuned to the first zero of the spherical zeta function $S^{\spherical}_3$ on cubic lattices with $N=10,$ 20, 40, 80 (squares, diamonds, hexagons, and circles, respectively), with the resulting spectrum analyzed with $S^{\spherical}_3$ (colored points) and the $N$-appropriate $S^{\dispersion}_3$ (black points).
        The gray dashed line is $S^\spherical_3$ and the thin vertical lines are at the non-interacting $x$s where it diverges.
        The colored lines are the second-order analytic prediction for the difference between the dispersion and spherical analysis as a function of $x$.
        For clarity of the continuum limit we show, in the bottom panel, a limited range in $x$ and $pL\cot\delta_{30}$, where it is clear that each $N$ hits the zero of $S^\spherical_3$ but that the flat behavior at any finite $N$ is away from an infinite scattering length when analyzed with $S^\dispersion_3$.
    }
    \label{fig:3d-corrections}
\end{figure}

This is particularly evident for the contact interaction as was observed, for example, in \Ref{Endres:2012cw}.
To understand the source of these terms, consider
\begin{multline}
\frac{1}{\pi L}S^\bigcirc_3(x^\dispersion)=\frac{1}{\pi L}\left(S^{\dispersion}_3(x^\dispersion)+\left(S^\bigcirc_3(x^\dispersion)-S^{\dispersion}_3(x^\dispersion)\right)\right)=\frac{-1}{a_3}+\frac{1}{\pi L}\left(S^\bigcirc_3(x^\dispersion)-S^{\dispersion}_3(x^\dispersion)\right)\\
=\frac{-1}{a_3}+\lim_{\eta \to\infty}\frac{1}{\pi L}\left(\sum_{\bm n\notin \mathrm{B.Z.}}^{|\bm n|<\eta / 2} \frac{1}{\bm n^{2}-x^\dispersion}-\mathcal{L}^\bigcirc_3\frac{\eta}{2}+\mathcal{L}^\dispersion_3\frac{N}{2}\right)\ .
\end{multline}
In the first line we added and subtracted $S^\dispersion_3$ and used the dispersion results~\eqref{dispersion-zeta-form} and~\eqref{dispersion-zeta-contact} to introduce the scattering length in the case of a contact interaction.
For convenience we assume $n_s=\infty$\footnote{
The logic of the following derivation remains them same also for $n_s < \infty$, but, in this case, the expressions $\bm n^2$ must be replaces with the proper dispersion $\tilde K^{(n_s)}_{\bm n \bm n}$ \eqref{normalized-kinetic-hamitlonian}, which makes it difficult to obtained closed expressions.}.
In the second line, since $\bm n$ is now restricted to be \emph{outside} the Brillouin zone, we can assume that $\bm n^2\gg x^\dispersion$ and expand in small $x^\dispersion$ under the summation,
\begin{align}
	\label{eqn:observable}
	S^\bigcirc_3(x^\dispersion)
	&=
	\frac{-\pi L}{a_3}
	+\mathcal{L}^\dispersion_3\frac{N}{2}
	+\lim_{\eta \to\infty}\left(\sum_{\bm n\notin \mathrm{B.Z.}}^{|\bm n|<\eta / 2} \frac{1}{\bm n^{2}}
	-\mathcal{L}^\bigcirc_3\frac{\eta}{2}\right)
	+x^\dispersion\lim_{\eta \to\infty}\sum_{\bm n\notin \mathrm{B.Z.}}^{|\bm n|<\eta / 2} \frac{1}{\bm n^{4}}
	+(x^\dispersion)^2\lim_{\eta \to\infty}\sum_{\bm n\notin \mathrm{B.Z.}}^{|\bm n|<\eta / 2} \frac{1}{\bm n^{6}}
	+ \ldots
	\\
	\label{eq:small x}
	&\equiv \frac{-\pi L}{a_3}+\alpha_1(N)+\alpha_2(N)x^\dispersion+\alpha_3(N)(x^\dispersion)^2
	+\ldots
\end{align}
The last line above shows explicitly the induced momentum-dependence in $x^\dispersion$ and defines the coefficients $\alpha_i(N)$ in terms of particular lattice summations similar to those of the three-dimensional zeta function.  The dependence of these coefficients on $N$ comes from the \emph{exclusion} of momentum modes within the Brillouin zone in the summation.   The fact that these coefficients do \emph{not} depend on $L$ is a unique feature of the contact interaction.  The numerical values of the coefficients $\alpha_i(N)$ can be determined using standard acceleration techniques (see, for example, Appendix B of~\Ref{Luu:2011ep}).  We provide values for select cases of $N$ in \autoref{tab:slopes}.
\begin{table}
\caption{Coefficients $\alpha_i(N)$ as a function of $N$ in 3-D.\label{tab:slopes}}
\center
\begin{tabular}{c|ccc}
$N$ & $\alpha_1$ & $\alpha_2$ & $\alpha_3$ \\
\hline
10 & 0.34622847019345 &2.1088361299026 &0.02096728133239\\
20 & 0.17384029798483 &1.0470052482673 &0.00253774588732\\
40 & 0.08701147975728 &0.5225652776531 &0.00031456311910\\
50 & 0.06961796407968 &0.4179620004936 &0.00016089237674\\
80 & 0.04351717442702 &0.2611651268184 &0.00003923695720\\
100 &0.03481483765136 &0.2089208128674 &0.00002008418957
\end{tabular}
\end{table}

In \Figref{3d-corrections} we show the result of tuning a finite-spacing contact interaction to the first zero of the continuum zeta function $S^\spherical_3$.
At each spacing the spectrum is fed through the continuum zeta for analysis, resulting in an apparent spacing-dependent momentum dependence that matches the small-$x$ expansion~\eqref{small x} discussed in the next section.
The same spectrum is also fed through the spacing-appropriate dispersion zeta $S^\dispersion$, resulting in the flat black lines.
Shown in detail in the bottom panel, it's clear that the continuum limit taken this way results in any finite spacing having a nonzero scattering length that vanishes with the continuum limit.
In contrast, tuning to the dispersion function directly, as in \Figref{unimproved dispersion}, is flat and nearly zero at each individual lattice spacing.
We expect that this difference explains the induced momentum dependence of, for example, \Refs{Endres:2011er,Endres:2012cw}.
