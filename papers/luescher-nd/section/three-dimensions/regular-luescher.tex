\subsubsection{Cutoff artifacts induce effective range}

After tuning the contact interactio to the first zero of the spherical zeta function, we compute the spectrum of the hamiltonian.
Next, we extrapolate the obtained energy eigenvalues to the continuum $\epsilon \to 0$ using a polynomial fit
\begin{equation}
    E^{(n_s)}_i(\epsilon) = E_i^{(n_s)} + \sum\limits_{n=1}^{n_\mathrm{max}} e_{i,n}^{(n_s)} \epsilon^n \, .
\end{equation}
We employ the following strategies and model selection criteria to estimate the continuum energy $E_i^{(n_s)}$ (which should be $n_s$ independent):
\begin{enumerate}
    \item Individual fits for each $n_s$ and compare $E_i^{(n_s)}$.
    \item A constraint fit for all $n_s$ simultaniously with $E_i^{(n_s)} = E_i$ fixed but different $e_{i,n}^{(n_s)}$.
\end{enumerate}
We vary the fits over a range of $n_\mathrm{max}=1, \cdots 8$ and employ bayesian information criteria (comparison of bayes factors) to select the best fit meta parameter $n_\mathrm{max}$ using \cite{peter_lepage_2016_60221}.
Because the contact interaction is expected to scale linear with $\frac{1}{\epsilon}$ (see \eqref{FV pole}), one cannot generally expect the fit coefficients $e_{i,n}^{(n_s)}$ to be zero for even $n$ or $n < n_s$ as the dispersion suggests.
Nevertheless, we would expect the small $n$ coefficient for larger $n_s$ to be relatively smaller then small $n$ coefficients for larger $n_s$.
We present the best fit to our spectrum in \figref{spectrum-fit} and a summary of other results in table \tabref{spectrum-fit-gable}.
Also, we provide access to the raw data and fitting scripts onlie at \cite{repo}.


\begin{figure}[th]
    \scalebox{0.9}{\input{figure/continuum-ere_a-inv_+0.0_zeta_spherical_projector_a1g_n-eigs_200.pgf}}
    \caption{Same as \Figref{unimproved spherical} but spectrum was extrapolated to continuum before fed to zeta function.}
    \label{fig:unimproved spherical continuum extrapolation}
\end{figure}

\begin{figure}[th]
    \scalebox{0.9}{\input{figure/continuum-ere_a-inv_+0.0_zeta_spherical_projector_a1g_n-eigs_200.pgf}}
    \caption{Same as \Figref{unimproved spherical} but spectrum was extrapolated to continuum before fed to zeta function.}
    \label{fig:unimproved spherical continuum extrapolation}
\end{figure}

\Figref{unimproved spherical}

\begin{figure}[th]
    \scalebox{0.9}{\input{figure/ere-contact-fitted_a-inv_+0.0_zeta_spherical_projector_a1g_n-eigs_200.pgf}}
    \caption{We show the result of tuning the contact interaction so that the ground state matches the first zero of $S^{\spherical}_3$.  In the top row we show results for $L=1.0$~fm, in bottom we show $L=2.0$~fm, while in different columns we show different finite differences as described in \eqref{p space hamiltonian}, \eqref{gamma definition}, and \eqref{gamma determination}. \todo{WHAT DO THESE UNITS MEAN?  I would advocate just making things non-dimensional by using factors of $m$ or something?}}
    \label{fig:unimproved spherical}
\end{figure}



\todo{Comment on the continuum-extrapolated results, which I can't do yet since they're not there :D}

In \Figref{finite a spherical} we show the result of tuning the contact interaction so that the ground state, when put through $S^{\spherical}_3$ yields a scattering length of \todo{something}.
Again, since the interaction is a contact interaction only, we expect a completely flat behavior.
In constrast, we see \todo{shapes that make us sad / curious}.

\begin{figure}[th]
    \scalebox{0.9}{\input{figure/ere-contact-fitted_a-inv_-5.0_zeta_spherical_projector_a1g_n-eigs_200.pgf}}
    \caption{We results like \Figref{unimproved spherical}, but where the contact interaction was tuned so that the ground state matches yields $p\cot\delta = $\todo{some finite scattering length; part of \issue{19}} rather than to 0.  Similar deviation from completely flat behavior is apparent at any finite lattice spacing.}
    \label{fig:finite a spherical}
\end{figure}

\clearpage
